\documentclass[10pt]{article}
    \usepackage[utf8]{inputenc}
    \usepackage[swedish]{babel}
    
    \def\year{2019}
    \def\doctitle{Samarbetsavtal mellan E-sektionen inom TLTH och Knightec AB  {\year}}
    \def\date{2019-02-22} %YYYY-MM-DD
    \def\docauthor{Edvard Carlsson}								    
    \usepackage{../../e-sek}
    
 
    
    \begin{document}
        \section*{\doctitle}
        Detta avtal avser ett samarbete mellan E-sektionen inom TLTH hädanefter refererat till som "E-sektionen", och Knightec AB, hädanefter refererat till som "Knightec" under verksamhetsåret 2019. Detta avtal står till grund för fortsatt samarbete mellan E-sektionen och Knightec.
\\
        \section{Syfte}
        Syftet med samarbetet är att öka Knightecs relation med studenter som studerar Elektroteknik
(E) och Medicin och Teknik (BME). Att studenter får upp ögonen för Knightecs verksamhet samt öka attraktionskraften för företaget. Dessa syften skall uppnås med senare i avtalet
nämna åtaganden.


        \section{Rättigheter och åtaganden}
        \subsection{Knightecs åtaganden}
        Överenskommelsen mellan parterna är att Knightec ska ansvarar för

        \begin{attsatser}
            \att innehållet under evenemangen är relevanta för studenterna på E-sektionen. Detta förhoppningsvis i mån för att väcka ett intresse för framtida karriärmöjligheter.

            \att informera näringslivsutskottet på E-sektionen inför evenemang för att underlätta för marknadsföringen.

               \end{attsatser}

        \subsection{E-sektionens åtaganden}
        Överenskommelsen mellan parterna är att E-sektionen ska ansvara för
        \begin{attsatser}
            \att tillsammans med Knightec genomföra ett (1) till två (2) stycken evenemang ihop. Typen av dessa evenemang ska ömsesidigt beslutas mellan parterna.
            \att informera sektionens studenter om Knightecs kommande evenemang genom sektionens sociala kanaler. 
            \att hålla Knightec uppdaterade om intresset från studenterna till evenmanget samt antal anmälda studenter. Skulle Knightec anse att intresset är för lågt har de möjligheten att ändra evenemangets upplägg eller typ, alternativt flytta fram evenemanget.
            \att behandla personuppgifter för på evenemang anmälda studenter, vilka kan lämnas över till Knightec vid godkännande från berörd student. 
        \end{attsatser}
        
        \section{Villkor}
        Knightec ersätter E-sektionen efter genomfört evenemang i enlighet med E-sektionen prissättning, se punkt 5. Denna ersättning ska erläggas till E-sektionen mot faktura senast \textbf{30 dagar} efter fakturadatum.
        \newline

        Detta avtal har tryckts i två exemplar där båda parter fått vardera ett.
        
        \section{Avtalstid}
        Avtalet gäller under 2019, 19-02-22 t.o.m 19-12-31.

	 \section{Prissättning}
	 Prissättningen av evenmang till företag utanför TLTH följer de direktiv som finns efter E-sektionens prislista (bifogat). Utifall att det inte finns något evenemang som motsvarar det som företaget vill anordna och därmed finns ingen motsvarande prissättning förs en dialog mellan E-sektionens Näringslivsordförande samt företagskontakt där prissättningen gemensamt framställs.


		
        \begin{signatures}{3}
            Lund, \date
            \signature{Edvard Carlsson}{Ordförande, E-sektionen}
            \signature{Henrik Ramström}{Förvaltningschef, E-sektionen}
            \signature{Rebecca Dovega}{Talent Acquisition Specialist, Knightec AB
}
        \end{signatures}
    \end{document}
    