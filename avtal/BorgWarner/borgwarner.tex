\documentclass[10pt]{article}
    \usepackage[utf8]{inputenc}
    \usepackage[swedish]{babel}
    
    \def\year{2019}
    \def\doctitle{Samarbetsavtal mellan E-sektionen inom TLTH och BorgWarner Sweden AB {\year}}
    \def\date{2019-04-16} %YYYY-MM-DD
    \def\docauthor{Edvard Carlsson}
    
    \usepackage{../../e-sek}
    
    \begin{document}
        \section*{\doctitle}
        Detta avtal avser ett samarbete mellan E-sektionen inom TLTH (845001-2284), hädanefter refererat till som “E-sektionen“”, och BorgWarner Sweden AB (556040-2736), hädanefter refererat till som  “BorgWarner”, gällande samarbete som huvudsponsor under hösten 2019.\\
        \section{Syfte}
       
       
        Syftet med samarbetet är att öka BorgWarners relation med studenter som studerar Elektroteknik (E) och Medicin och Teknik (BME). Att studenter får upp ögonen för BorgWarners verksamhet samt öka attraktionskraften för företaget. Syftena skall uppnås med senare i avtalet nämnda åtaganden.
       
       
       
        \section{Åtaganden}
        \subsection{BorgWarners åtaganden}
        Överenskommelsen mellan parterna är att BorgWarner ska stå för
        \begin{attsatser}
            \att hålla i ett (1) lunchföredrag under mottagningen där deltagarna bjuds på lunch.
            \att stå för lunchkostnaden för detta föredrag. 
            \att genomföra ett (1) kvällsevenemang efter mottagningveckorna, dvs. vecka 40 eller senare.
            \att förse näringslivsutskottet på E-sektionen med marknadsföringsmaterial.
        \end{attsatser}

        \subsection{E-sektionens åtaganden}
        Överenskommelsen mellan parterna är att E-sektionen ska står för
        \begin{attsatser}
            \att marknadsföra alla evenemang till studenterna, sköta anmälningen till dessa samt beställa mat till tillhörande evenamang, med en budget på \SI{50}{kr} per deltagande student.
            \att marknadsföra Borgwarner på TV-skärmar under mottagningen, på ett uppslag i introduktionsmanualen “Nolleguiden” samt i ett Facebook-inlägg.
            \att måla Borgwarners logga på väggen i lunchsalen i “Edekvata” och ha kvar den ett år.
            \att få Borgwarners logga att synas kontinuerligt under mottagningen.
            \att marknadsföra Borgwarner inför och under “Gasquen”.
            
        \end{attsatser}
        
        \section{Villkor}
        BorgWarner ersätter E-sektionen med \SI{30000}{kr}. Denna ersättning ska erläggas till E-sektionen mot faktura senast \textbf{30 dagar} efter fakturadatum.
        \newline

        Detta avtal har tryckts i två exemplar där båda parter fått vardera ett.
        
        \section{Avtalstid}
        Avtalet gäller under hösten 2019, 19-08-26 t.o.m 19-12-31.

        \begin{signatures}{3}
            Lund, \date
            \signature{Edvard Carlsson}{Ordförande, E-sektionen}
            \signature{Henrik Ramström}{Förvaltningschef, E-sektionen}
            \signature{Petter Johansson}{Improvement Catalyst, BorgWarner Sweden AB}
        \end{signatures}
    \end{document}
    