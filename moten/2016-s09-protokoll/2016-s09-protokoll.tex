\documentclass[10pt]{article}
\usepackage[utf8]{inputenc}
\usepackage[swedish]{babel}

\def\mo{Fredrik Peterson}
\def\ms{Erik Månsson}
\def\ji{Malin Lindström}

\def\doctype{Protokoll} %ex. Kallelse, Handlingar, Protkoll
\def\mname{styrelsemöte} %ex. styrelsemöte, vårterminsmöte
\def\mnum{S09/16} %ex S02/16, E1/15, VT/13
\def\date{2016-04-14} %YYYY-MM-DD
\def\docauthor{\ms}

\usepackage{../e-mote}
\usepackage{../../e-sek}

\begin{document}
\showsignfoot

\heading{{\doctype} för {\mname} {\mnum}}

\section*{Närvarande}
\subsection*{Ordinarie}
\begin{narvarolista}
\nv{Ordförande}{Fredrik Peterson}{E14}{}
\nv{Kontaktor}{Erik Månsson}{E14}{}
\nv{Förvaltningschef}{Anders Nilsson}{E13}{}
\nv{Cafémästare}{Stephanie Mirsky}{E13}{}
\nv{Øverphøs}{Molly Rusk}{BME14}{}
\nv{ENU-ordförande}{Johannes Koch}{E13}{}
\nv{Sexmästare}{Martin Gemborn Nilsson}{E14}{\nafr{7}}
\nv{Krögare}{Malin Lindström}{BME14}{}
\nv{Entertainer}{Dalia Khairallah}{E15}{}
\end{narvarolista}

\subsection*{Ständigt adjungerande}
\begin{narvarolista}
\nv{Kårrepresentant}{Daniel Damberg}{}{}
\nv{Kårrepresentant}{John Alvén}{}{}
\end{narvarolista}

\subsection*{Adjungerande}
\begin{narvarolista}
\nv{Likabehandlingsombud}{Niklas Gustafson}{E15}{}
\end{narvarolista}

\section*{Protokoll}
\begin{paragrafer}
\p{1}{OFSÖ}{\bes}
Ordförande {\mo} förklarade mötet öppnat 12:15.

\p{2}{Val av mötesordförande}{\bes}
\valavmo

\p{3}{Val av mötessekreterare}{\bes}
\valavms

\p{4}{Tid och sätt}{\bes}
\tosg

\p{5}{Adjungeringar}{\bes}
Niklas Gustafson adjungerades

\p{6}{Val av justeringsperson}{\bes}
\valavj

\p{7}{Föredragningslistan}{\bes}
Dalia \ypa att lägga till \S13.5 ``Inköp av tält''.

Malin \ypa att lägga till \S13.75 ``Inköp av streckkodsläsare''.

Föredragningslistan godkändes med samtliga yrkanden.

\p{8}{Föregående mötesprotokoll}{\bes}
\latillprot{S08/16}

\p{9}{Fyllnadsval/Entledigande av funktionärer}{\bes}
\begin{fyllnadsval} %"Inga fyllnadsval." fylls i automatiskt
\end{fyllnadsval}

\p{10}{Rapporter}{}
\begin{paragrafer}
\subp{A}{Check in}{\info}
FvU har inte gjort så mycket denna veckan. Anders har haft mycket att göra i skolan.

Johannes blev lite ledsen över att ha tappat sina vantar när han cyklade hit. ``Lunch med en ingenjör'' har gått bra.

Martin hade FpT igår, det gick bra.

Malin har fått en förfrågan om vi vill hålla ett ``sexgille'' från P6. Ska hålla ett Alumnigille.

Dalia ska på DÖMD snart och planerar att hålla E-lounge. Sporta med E går bra.

Niklas mår bra, han har sina handskar.

Molly mår bra. Har fixat inför temasläppet. Temasläppsfilmen är klar. Phøset har spexat på FpT. Molly informerade också om att de fått ihop en massa pluggphaddrar tillsammans med SRE. Ska i helgen prata med KTH:arna om hur deras nollning fungerar.

Stephanie mår bra. Hade lunchmöte tillsammans med alla dioder. Caféfesten ska hållas den 17:e maj.

Erik mår bra, han har nu släppt handlingarna och vet inte riktigt vad han ska göra istället för att jobba på handlingarna.

Fredrik mår också bra, har haft skypemys med Erik de senaste kvällarna och fixat med handlingarna.

\subp{B}{Kåren informerar}{\info}
John och Daniel sammanfattade vad FM beslutade på deras förra möte, och vad styrelsen gjorde på deras förra möte.

Corneliusbalen är på lördag.

\subp{C}{Ekonomi}{\info}
Anders informerade om ekonomin. Denna veckan har banken stått ganska stilla.

\end{paragrafer}

\p{11}{Remissvar Internationalisering}{\dis}
Fredrik presenterade sitt svar på remissen, vilken mötet var positivt till.

\p{12}{Remissvar Policy för likabehandling}{\dis}
Fredrik presenterade remissen.

Mötet diskuterade likabehandlingsutbildningen. Mötet funderade på hur stor utbildningen ska vara. Daniel sa att det förmodligen rör sig om några timmar, och uppmanade styrelsen om att skriva som svar att de vill ha ett förtydligande på detta.

Niklas, vårt likabehandlingsombud, sa att han är positiv till en likabehandlingsutbildning på några timmar.

Mötet diskuterade konsekvenser av att bryta mot policyn. Fredrik sa att egentligen har Sektionen redan rätt att sätta av funktionärer och ställa in event, men att policyn förtydligar detta. Anders är positiv till att ett förtydligande av konsekvenser ligger i styrdokument högre upp än sektionens.

Daniel tycker att styrelsen borde förmedla tillbaka detta också, ifall alla håller med. Han sa också att Kåren gärna undviker att detaljstyra sektionernas verksamhet.

Fredrik summerade vad styrelsen sagt och kommer skriva ett svar som han presenterar vid ett senare möte.

\p{13}{Tilldelning av meritvärde}{\dis}
Fredrik berättade om att LTH ska uppdatera tilldelning av meritvärde för t.ex. sektionsarbete, och gör det på Kårens rekommendation. Fredrik har läst igenom förslaget och var positiv till det.

Molly tyckte att det var oklart hur mycket poäng som faktiskt ges ut, vilket Fredrik höll med om. Fredrik sa att dokumentet är mer av en förklaring till hur poänggivandet ska gå till och hur tanken gick bakom detta.

Daniel informerade om att förra året togs mycket av meritpoängsutdelningen bort, vilket sektionerna och Kåren inte var särskilt positiva till. Därför kommer nu en del av poängen införas igen.

Johannes funderade på att om man kan få poäng retroaktivt, t.ex. ifall man gjort ett uppdrag under ett år där poäng inte gavs för detta. Daniel sa att han tror att detta kan vara möjligt, men det är osäkert.

\p{13.5}{Inköp av tält}{}
Dalia vill köpa in ett partytält till Sektionen, istället för att hyra av Kåren varje gång.

Anders sa att vi kan lägga det på dispositionsfonden. Han sa att vi bör ha minst \SI{100000}{kr} kvar i fonden vid årets slut, och att fonden kommer ligga på \SI{128000}{kr} efter Vårterminsmötet, om förslaget går igenom.

Anders förslog att lägga det på vanliga budgeten, vilket Dalia inte var positiv till.

Erik tycker att vi bör köpa in ett rejält tält som håller länge, och ta det på dispositionsfonden, eftersom hela Sektionen ska använda det.

Fredrik vill köpa in \emph{minst} ett tält och lägga det på dispositionsfonden.

Johannes tänkte att vi kan köpa in tält som vi också kan ha på Utedischot.

NöjU kommer ta fram förslag på vilket tält som ska köpas in.

\p{13.75}{Inköp av streckkodsläsare}{}
Malin vill köpa en streckkodsläsare för att effektivisera KM:s arbete. Det rör sig om ungefär \SI{500}{kr} till \SI{1000}{kr}. Anders sa att det rör sig om så lite pengar att man kan lägga det på budgeten.

Anders tycker att Malin ska ta fram ett förslag tillsammans med Macapärerna, vilket hon var positiv till och kommer göra.

\p{14}{Nästa styrelsemöte}{\bes}
\Mba nästa styrelsemöte ska äga rum 2016-04-21 12:10 i E:1426.

\p{15}{Beslutsuppföljning}{\bes}
\Ibfu

\p{16}{Övrigt}{\dis}
Fredrik frågade om någon kan gå på värdegrundsworkshopen på måndag. Kåren ser gärna att någon från varje sektion är där. Malin sa att hon kan gå, vilket mötet gav applåder till!

Erik sa att han kommer skriva ut handlingarna och lägga i HK så alla kan skriva på.

\p{17}{OFSA}{\bes}
{\mo} förklarade mötet avslutat 12:59.

\end{paragrafer}

%\newpage
\hidesignfoot
\begin{signatures}{3}
\signature{\mo}{Mötesordförande}
\signature{\ms}{Mötessekreterare}
\signature{\ji}{Justerare}
\end{signatures}
\end{document}
