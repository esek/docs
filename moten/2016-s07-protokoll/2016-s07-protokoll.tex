\documentclass[10pt]{article}
\usepackage[utf8]{inputenc}
\usepackage[swedish]{babel}

\def\mo{Fredrik Peterson}
\def\ms{Erik Månsson}
\def\ji{Martin Gemborn Nilsson}
%\def\jii{}

\def\doctype{Protokoll} %ex. Kallelse, Handlingar, Protkoll
\def\mname{styrelsemöte} %ex. styrelsemöte, vårterminsmöte
\def\mnum{S07/16} %ex S02/16, E1/15, VT/13
\def\date{2016-03-31} %YYYY-MM-DD
\def\docauthor{\ms}

\usepackage{../e-mote}
\usepackage{../../e-sek}

\begin{document}
\showsignfoot

\heading{{\doctype} för {\mname} {\mnum}}

%\naun{}{} %närvarane under
%\nati{}{} %närvarande till och med
%\nafr{}{} %närvarande från och med
\section*{Närvarande}
\subsection*{Ordinarie}
\begin{narvarolista}
\nv{Ordförande}{Fredrik Peterson}{E14}{}
\nv{Kontaktor}{Erik Månsson}{E14}{}
\nv{Förvaltningschef}{Anders Nilsson}{E13}{}
\nv{Cafémästare}{Stephanie Mirsky}{E13}{}
\nv{Øverphøs}{Molly Rusk}{BME14}{}
\nv{SRE-ordförande}{Johan Persson}{E13}{}
%\nv{ENU-ordförande}{Johannes Koch}{E13}{}
\nv{Sexmästare}{Martin Gemborn Nilsson}{E14}{}
\nv{Krögare}{Malin Lindström}{BME14}{}
\nv{Entertainer}{Dalia Khairallah}{E15}{}
\end{narvarolista}

\subsection*{Ständigt adjungerande}
\begin{narvarolista}
%\nv{Valberedningens ordförande}{Elin Magnusson}{}{}
%\nv{Skattmästare}{Sophia Grimmeiss Grahm}{}{}
\nv{Kårrepresentant}{Daniel Damberg}{}{}
\nv{Kårrepresentant}{John Alvén}{}{}
%\nv{Talman}{Johan Westerlund}{E11}{}
%\nv{Elektras Ordförande}{Elisabeth Pongratz}{}{}
%\nv{Inspektor}{Monica Almqvist}{}{}
\end{narvarolista}

\begin{comment}
\subsection*{Adjungerande}
\begin{narvarolista}
%\nv{Post}{Namn}{Klass}{}
\end{narvarolista}
\end{comment}

\section*{Protokoll}
\begin{paragrafer}
\p{1}{OFSÖ}{\bes}
Ordförande {\mo} förklarade mötet öppnat 12:14.

\p{2}{Val av mötesordförande}{\bes}
\valavmo

\p{3}{Val av mötessekreterare}{\bes}
\valavms

\p{4}{Tid och sätt}{\bes}
\tosg

\p{5}{Adjungeringar}{\bes}
\ingaadj

\p{6}{Val av justeringsperson}{\bes}
Martin nominerades.\\
Martin godtog inte nomineringen.\\
Martin godtog nomineringen.

\valavj

\p{7}{Föredragningslistan}{\bes}
Föredragningslistan godkändes.
%Fredrik \ypa att lägga till \S18b ``Teknikfokus utnyttjande av LED-café''.
%Föredragningslistan godkändes med yrkandet.
%Föredragningslistan godkändes med samtliga yrkanden.

\p{8}{Fyllnadsval/Entledigande av funktionärer}{\bes}
\begin{fyllnadsval} %"Inga fyllnadsval." fylls i automatiskt
%\fval{Namn}{Post}
\entl{Rebecka Lindqvist}{Fotograf}
\end{fyllnadsval}

\p{9}{Föregående mötesprotokoll}{\bes}
\latillprot{S04/15, S05/15, S25/15 och S06/16}
%\ingaprot

\p{10}{Rapporter}{}
\begin{paragrafer}
\subp{A}{Kåren informerar}{\info}
TLTH har inlett kontakt med polisen om sluta tillstånd
senaste styrelsemötet motion till fm om ny rulle. hyran kommer förmodlignei nte påverkas. vilken typ av bil är inte bestämt än.

kåren har reviderat styrdokumenten. om fm klubbar igenom...

\subp{B}{Check in}{\info}
Anders har kommit ikapp på Ekonomin.

Mår bra. Malin har inget gille i veckan eftersom det är sittning i helgen. Nästa gille bara för phaddrar

Steph mår bra. Har gjort prshöj i led för att uppnå budget

Johan mår bra, jobbar med CEQ. Har bestämt med NollU att ha pluggphaddrar

Dalia mår bra, håller på lite med tandem och ska snart på dömd. Fotbollsturneringen blir förmodligen inte av.

Molly har valt uppdragsphaddrar och internationella phaddrar. Har börjat planera temasläpp och temasläppsfilem. Phadderutbildning. Möte med sexet om alla sittningar. Har skipat alla uppdrag och evenemang inom kåren. KOmmer ha ne infokväll under en pluggkväll. Har hämtat mantlar

Martin har hunnit me en hel del. Har haft ET-sittning. Martin mår bra, har sovit en massa under lovet. Har sittning imorgon tillsammans med många ndra sektioner. Har haft möte med Sexet och Phöset. Kommer förmodligen ha en sittning den 4:e maj för sektionen.

John informaterade om att den 5:e maj är det slaget om lund (isek)

Erik mår bra, har dock varit lite sjuk i veckan. Har jobbat en del med det kommande sektionsmötet.

Fredrik mår också bra. Har varit på workshops med ÖPK och möte med OK.

\subp{C}{Ekonomi}{\info}

Anders informerade om ekonomiska läget. Den mår bra, inget speciellt har hänt.

\end{paragrafer}

\p{11}{Instagram i reglementet}{\dis}
Mötet diskuterade fram att Erik ska skria en proposition för att införa facebook och instagram i reglementet samt införa en riktlinje från styrelsen.

\p{12}{Godkännande av fasadmotivskyld}{\bes}
Fredrik informerade om projektet som han jobbat en del med.

Säger att vi kan sättau pp den i samband med fönsterbytet för att undvika onödiga kostnader.

Motivet ska godkännas av hysstyrelsen coh styrelsen.

Förslaget är en vit stor rund skylt med Hacke på.

Fredrik försökte fibrilt med att visa upp hacke för styrelsen.

Fredrik har pratat med huset men de har inte svarat än.

Molly vill ha något som liknar det i märkesbacken.

Fredrik föreslog att ha sigillet men med hacke i mitten.

Mötet godkände att ha hacke som motiv på skylten.

\p{13}{LEDs ekonomi}{\info}
Anders informerade om att resultatet i LED från förra året diffade runt 70000kr. Han har försökt kolla över vad som hänt och kommit fram till 3 problem.

- Quixter försvann i höstas, vilket gav en stor reducering av intäkterna.
- Det har pantats dåligt från läsklagret i CM, vilket belastar CM. Anders tycker att det inte är att CM ska behöva ta smällen för det. Håller på att kolla på alternativ. Förr kom mer av panten tillbaka men nu för tiden förloras mycket pant, inte minst pga att verksamheten flyttat ut från bara edekvata. Fuderar på en ändring itll sommaren.
- Generell prishöjning av inköpspriser under året. Svårt att upptäcka direkt, med tanke på volymerna som beställs in.

Dalia föreslog att minska storleken på sallad och mackor. Anders sa att det är lätt att nya jobbare i led gör lite fel. Det är också ofta så att man upptäcker först problemen när man gör boksllut.

CM har nu höjt priserna och försöker har hårdare koll på vad som köps in. Kvartalsrapporter och kontinuerlig budgetuppföljning.

Anders sa att förra årets CM förmodligen inte kunde gjort så mycket åt det.

Erik föreslog att panten ingpr i kostnader och att poant ses som donation.

Anders sa att det är ett förslag som man funderar på att genomföra till posten. Inkommande pant till sektionen centralt.

Dalia förslog att man inte ska kunna köpa läsk i CM. Anders sa att problemet är inte CM i sig, utan hur vi budgeterar för det.

Anders sa att läskdifferansen i laget låg på ungefär 1500kr, vilket får ses som ganska lite.

Fredrik ser det som att där finns två delar att detta problem. Det bokföringstekniska vilket inte är så mycket att diskutera, det får ekonomiansvarig fixa till. Och att CM inte går tillräckligt med vinst ändå.. %TODO

\p{14}{Styrelsekepsar}{\bes}
Styrelsen ska rösta om vilka kepsar som styrelsen ska köpa, och vilka som vill ha kepsar.

Den dyra är en märkeskeps, den billiga är en noname

Båda har samma tryck.

Mötet är överrens om att den billigare är lämplig för

\Mba köpa in den billiga kepsen.

\Mba låta Johannes sköta om beställingen.

\p{15}{Hyresmall för Edekvata/Teknik}{\bes}
Anders pratade om Anders och Eriks förslag för hur dokumenten till uthyrning ska se ut.

Anders tycker inte att vi ska konkurrera med kåren och har därför lite högre priser för andra sektioner. Funktionärer får hyra för i princip självkostnadspris.

Anders frågade om vi ska ha en kategori för E-sektionsmedlemmar (och funktionär/övr medl i kåren).

Anders vill ha input på priserna ``privat''

Lägga kort i reglementet... %TODO

Skriva mall för villkor och priser som riktlinjer. %TODO

Martin frågade om huruvida utskott får förtur. Anders sa att sektionens egna verksamhet alltid kommer före uthyrning.

Erik förslog att E-sektionsmedlemmar får priser runt halvvägs mellan funk och övr.sekt

\Mba godkänna anders förslag.

\p{16}{Utskottssafari}{\dis}
Molly berättade om ett förslag till Utskottssafari under nollingen. Bra sätt för nya medlemmar att se att det finns mer verksamhet än bara KM och E6. Molly sa att Styrelsen till stor del får ta hand om event, som preliminiärt ska ligga på söndagen vecka 1. Vill ägga det tidigit för att engagemanget är stort under de första veckorna.

Styrelsen är intresserad av att vara där och hålla i eventet.

Anders sa att förra året hade phöset ett drivedokuemtn med kalender för nollningen med info om vilka som ska närvara osv. Molly sa att phöset kommer fixa detta.

Molly tyckte det är kul att alla är taggade.

\p{17}{Jobbare på JätteFunktionärsFesten}{\dis}
Kåren söker jobbare till JFF den 21:e maj.

Styrelsen sa de kunde hjälpa till.

\p{18}{Sektionskalender}{\dis}
Punkten ströks.

\p{19}{Nästa styrelsemöte}{\bes}
\Mba nästa styrelsemöte ska äga rum 2016-04-07 12:10 i E:1426.

\p{20}{Beslutsuppföljning}{\bes}
\Ibfu

\p{21}{Övrigt}{\dis}
Inget övrigt togs upp.

\p{22}{OFSA}{\bes}
{\mo} förklarade mötet avslutat 13:07.

\end{paragrafer}

%\newpage
\hidesignfoot
\begin{signatures}{3}
\signature{\mo}{Mötesordförande}
\signature{\ms}{Mötessekreterare}
\signature{\ji}{Justerare}
\end{signatures}
\end{document}
