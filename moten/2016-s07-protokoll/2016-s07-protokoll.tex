\documentclass[10pt]{article}
\usepackage[utf8]{inputenc}
\usepackage[swedish]{babel}

\def\mo{Fredrik Peterson}
\def\ms{Erik Månsson}
\def\ji{Martin Gemborn Nilsson}

\def\doctype{Protokoll} %ex. Kallelse, Handlingar, Protkoll
\def\mname{styrelsemöte} %ex. styrelsemöte, vårterminsmöte
\def\mnum{S07/16} %ex S02/16, E1/15, VT/13
\def\date{2016-03-31} %YYYY-MM-DD
\def\docauthor{\ms}

\usepackage{../e-mote}
\usepackage{../../e-sek}

\begin{document}
\showsignfoot

\heading{{\doctype} för {\mname} {\mnum}}

\section*{Närvarande}
\subsection*{Ordinarie}
\begin{narvarolista}
\nv{Ordförande}{Fredrik Peterson}{E14}{}
\nv{Kontaktor}{Erik Månsson}{E14}{}
\nv{Förvaltningschef}{Anders Nilsson}{E13}{}
\nv{Cafémästare}{Stephanie Mirsky}{E13}{}
\nv{Øverphøs}{Molly Rusk}{BME14}{}
\nv{SRE-ordförande}{Johan Persson}{E13}{}
\nv{Sexmästare}{Martin Gemborn Nilsson}{E14}{}
\nv{Krögare}{Malin Lindström}{BME14}{}
\nv{Entertainer}{Dalia Khairallah}{E15}{}
\end{narvarolista}

\subsection*{Ständigt adjungerande}
\begin{narvarolista}
\nv{Kårrepresentant}{Daniel Damberg}{}{}
\nv{Kårrepresentant}{John Alvén}{}{}
\end{narvarolista}

\section*{Protokoll}
\begin{paragrafer}
\p{1}{OFSÖ}{\bes}
Ordförande {\mo} förklarade mötet öppnat 12:14.

\p{2}{Val av mötesordförande}{\bes}
\valavmo

\p{3}{Val av mötessekreterare}{\bes}
\valavms

\p{4}{Tid och sätt}{\bes}
\tosg

\p{5}{Adjungeringar}{\bes}
\ingaadj

\p{6}{Val av justeringsperson}{\bes}
Martin nominerades.\\
Martin godtog inte nomineringen.\\
Martin ångrade sig och godtog nomineringen.

\valavj

\p{7}{Föredragningslistan}{\bes}
Föredragningslistan godkändes.

\p{8}{Fyllnadsval/Entledigande av funktionärer}{\bes}
\begin{fyllnadsval} %"Inga fyllnadsval." fylls i automatiskt
\entl{Rebecka Lindqvist}{Fotograf}
\end{fyllnadsval}

\p{9}{Föregående mötesprotokoll}{\bes}
\latillprot{S04/15, S05/15, S25/15 och S06/16}

\p{10}{Rapporter}{}
\begin{paragrafer}
\subp{A}{Kåren informerar}{\info}
TLTH har inlett kontakt med Polisen om Kårens slutna tillståndet.

Kårstyrelsen har skickat en motion till FM om ny rulle. Kostnaden för att hyra rulle kommer förmodligen inte förändras.

Kårstyrelsen har reviderat kårens styrdokument och lämnat över till FM.

\subp{B}{Check in}{\info}
Anders har kommit ikapp på Ekonomin tillsammans med FVU.

Malin mår bra. KM har inget gille i veckan eftersom det är sittning. Nästa gille är bara för Phaddrar.

Stephanie mår bra. Har gjort prishöjningar i LED för att uppnå budget.

Johan mår bra. Har jobbat på med CEQ och har tillsammans med NollU infört pluggphaddrar för nästa nollning.

Dalia mår bra, håller på lite med Tandem och ska snart på DÖMD. Fotbollsturneringen blir förmodligen inte av.

Molly har valt uppdragsphaddrar och internationella phaddrar. NollU har börjat planera temasläpp, temasläppsfilm och phadderutbildning. De har haft möte med E6 om alla sittningar under Nollning. Alla uppdrag och evenemang är spikade hos Kåren nu. NollU kommer hålla en informationskväll under en pluggkväll framöver.

Martin har hunnit med en hel del, t.ex. haft ET-sittning. Han mår bra och har sovit en massa under lovet. Ska hålla sittning imorgon tillsammans med många andra sektioner. Har också haft möte med E6 och NollU. E6 kommer förmodligen hålla en sektionssittning den 4:e maj.

John informerade om att den 5:E maj är det Slaget om Lund, vilket I-sektionen håller i.

Erik mår bra, har dock varit lite sjuk i veckan. Har jobbat en del med det kommande sektionsmötet.

Fredrik mår också bra. Han har varit på workshops med ÖPK och möte med OK.

\subp{C}{Ekonomi}{\info}

Anders informerade om ekonomiska läget. Ekonomin rullar på, inget speciellt har hänt.

\end{paragrafer}

\p{11}{Instagram i reglementet}{\dis}
Mötet diskuterade fram att Erik ska skria en proposition för att införa Facebook och Instagram i reglementet samt införa en riktlinje från Styrelsen angående detta.

\p{12}{Godkännande av fasadmotivsskylt}{\bes}
Fredrik informerade om projektet,vilket han hjälpt till en del med. Han sa att vi kan sätta upp skylten gratis i samband med fönsterbytet som sker under våren.

Motivet ska godkännas av husstyrelsen och E-styrelsen. Förslaget är en stor vit rund skylt med Hacke på. Fredrik försökte fibrilt visa upp Hacke för styrelsen, vilket han lyckades hyfsat bra med till slut. Han har skickat till husstyrelsen men de har inte svarat ännu.

Molly vill ha ett motiv som liknar det i märkesbacken.

Fredrik föreslog att ha sigillet men med Hacke i mitten.

\textbf{Mötet godkände att ha Hacke som motiv på skylten.}

\p{13}{LEDs ekonomi}{\info}
Anders informerade om att resultatet för CM förra året hade en differens på ungefär 70000kr. Han har försökt reda ut vad som hänt och kommit fram till 3 problem:

\begin{dashlist}
\item Quixter försvann i höstas, vilket gav en stor reducering av intäkterna.
\item Burkarna från läsklagret i CM har inte pantats lika mycket som förr, vilket belastar CM. Anders tycker inte att CM ska ta smällen för det och håller på att kolla på andra alternativ för framtiden. Förr kom mer av panten tillbaka men nu förloras mycket pant, inte minst p.g.a. att mer av E-sektionens verksamhet sker utanför Edekvata.
\item Det hade skett en allmän gradvis höjning av inköpspriser under året, vilket är svårt att upptäcka direkt.
\end{dashlist}

Dalia föreslog att minska portionsstorleken på sallad och mackor. Anders sa att det är lätt att nya jobbare i LED gör lite fel, och att det ofta är så att man upptäcker att man har ett problem först när man gör ett bokslut.

CM har nu höjt priserna och försöker ha hårdare koll på vad som köps in. Kvartalsrapporter och kontinuerlig budgetuppföljning ska göras.

Anders ville poängtera att förra årets CM förmodligen inte kunde gjort så mycket för att förhindra det som hänt.

Erik föreslog att panten ingår i kostnaden och att all pant som kommer in ses som extra vinst till Sektionen.

Anders höll med och sa att det är ett förslag som man funderar på att genomföra till hösten. Inkommande pant ska alltså tillfalla Sektionen centralt.

Dalia förslog att man inte ska kunna köpa läsk och kakor direkt från CM-lagret. Anders sa att problemet är inte CM-lagret i sig, utan hur vi budgeterar för det. Anders upplyste också om att läskdifferensen i laget låg på ungefär 1500kr, vilket får ses som relativt lite.

Fredrik sa att den bokföringstekniska biten av problemet inte är så mycket att diskutera, Förvaltningschefen kommer lösa detta problem. Han sa också att trots den bokföringstekniska ``missen'' gick CM inte tillräckligt med vinst ändå, vilket man bör diskutera hur man löser.

\p{14}{Styrelsekepsar}{\bes}
Styrelsen ska rösta om vilka kepsar som ska köpas in till Styrelsen. Det finns två alternativ, en billigare variant och en dyrare märkeskeps.

\textbf{\Mba köpa in den billigare varianten.}

\textbf{\Mba låta Johannes sköta om beställningen.}

\p{15}{Hyresmall för Edekvata/Teknik}{\bes}
Anders pratade om förslaget för hur dokumenten för uthyrning ska se ut.

Anders tycker inte att Sektionen ska konkurrera med Kåren och har därför satt lite högre priser för de utomstående E-sektionen. Funktionärer får hyra för självkostnadspris.

Anders frågade om vi ska ha en priskategori för våra medlemmar (mellan funktionärer och övriga sektioner).

Anders vill ha input på priserna.

Anders sa att det är tänkt att lägga det mesta av hyresvillkor och priser som en riktlinje från Styrelsen, och bara skriva hur dessa ska utformas i reglementet.

Martin frågade om huruvida våra utskott får förtur. Anders sa att Sektionens egna verksamhet alltid kommer före uthyrning till andra.

Erik förslog att våra medlemmar får hyra för runt halvvägs mellan priset för funktionärer och priset för övriga sektioner.

\textbf{\Mba godkänna förslaget.}

\p{16}{Utskottssafari}{\dis}
Molly berättade om ett förslag till Utskottssafari under nollningen. Det är ett bra sätt för nya medlemmar att se att det finns mer verksamhet i Sektionen än bara E6 och KM. Molly sa att det till stor del kommer bli Styrelsen som får ta hand om eventet, som preliminärt ska ligga på söndagen läsvecka 1. NollU vill lägga det tidigt under Nollningen för att engagemanget är som störst då.

Mötet är positivt till förslaget och Styrelsen är intresserad av att hålla i eventet.

Anders att förra året hade NollU ett dokument med en kalender för nollningen med information om t.ex. vilka funktionärer/styrelsemedlemmar som behöver vara närvara. Molly svarade med att NollU kommer fixa detta.

Molly tyckte det var kul att alla var taggade!

\p{17}{Jobbare på JätteFunktionärsFesten}{\dis}
Kåren söker jobbare till JFF den 21:e maj. Styrelsen var överens om att de kunde hjälpa till.

\p{18}{Sektionskalender}{\dis}
Punkten senarelades.

\p{19}{Nästa styrelsemöte}{\bes}
\Mba nästa styrelsemöte ska äga rum 2016-04-07 12:10 i E:1426.

\p{20}{Beslutsuppföljning}{\bes}
\Ibfu

\p{21}{Övrigt}{\dis}
Inget övrigt togs upp.

\p{22}{OFSA}{\bes}
{\mo} förklarade mötet avslutat 13:07.

\end{paragrafer}

\newpage
\hidesignfoot
\begin{signatures}{3}
\signature{\mo}{Mötesordförande}
\signature{\ms}{Mötessekreterare}
\signature{\ji}{Justerare}
\end{signatures}
\end{document}
