\documentclass[10pt]{article}
\usepackage[utf8]{inputenc}
\usepackage[swedish]{babel}

\def\mo{Theo Nyman}
\def\ms{Hannes Björk}
\def\ji{Rasmus Sobel}
%\def\jii{}

\def\doctype{Protokoll} %ex. Kallelse, Handlingar, Protkoll
\def\mname{Styrelsemöte} %ex. styrelsemöte, Vårterminsmöte
\def\mnum{S04/20} %ex S02/16, E1/15, VT/13
\def\date{2020-02-11} %YYYY-MM-DD
\def\docauthor{\ms}

\usepackage{../e-mote}
\usepackage{../../../e-sek}

\begin{document}
\showsignfoot

\heading{{\doctype} för {\mname} {\mnum}}

%\naun{}{} %närvarane under
%\nati{} %närvarande till och med
%\nafr{} %närvarande från och med
\section*{Närvarande}
\subsection*{Styrelsen}
\begin{narvarolista}
\nv{Ordförande}{Theo Nyman}{BME18}{}
\nv{Kontaktor}{Hannes Björk}{E17}{}
\nv{Förvaltningschef}{Rasmus Sobel}{BME16}{}
\nv{Cafémästare}{Frida Pilcher}{E18}{} 
\nv{Sexmästare}{Anna Hollsten}{BME19}{}
\nv{Krögare}{Love Sjelvgren}{E18}{}
\nv{Entertainer}{Amir Ghanaatifard}{E19}{}
\nv{SRE-ordförande}{Hanna Bengtsson}{BME18}{}
\nv{ENU-ordförande}{Fredrik Berg}{E17}{\nati{19}}
\nv{Øverphøs}{Sophia Carlsson}{BME17}{}
\end{narvarolista}


\subsection*{Ständigt adjungerande}
\begin{narvarolista}
%\nv{}{}{}{}
%\nv{Skattmästare}{Daniel Bakic}{E15}{\nafr{10}}
%\nv{Vice Krögare}{Klara Indebetou}{BME17}{}
%\nv{Vice Krögare}{Hjalmar Tingberg}{BME16}{}
%\nv{Vice Förvaltningschef}{Rasmus Sobel}{BME16}{}
\nv{Kårrepresentant}{Hanna Jonson}{}{}
\nv{Kårrepresentant}{Ivar Vänglund}{}{}
%\nv{Valberedningens ordförande}{Axel Voss}{E15}{\nafr{10b}}
%\nv{Fullmäktigeledamot}{Magnus Lundh}{E15}{\nafr{12}}
%\nv{Chefredaktör}{Emil Eriksson}{E18}{}
%\nv{Inspektor}{Monica Almqvist}{}{}
%\nv{Cophøs}{Elina Yrlid}{E18}{}
%\nv{Vice Kontaktor}{Matilda Horn}{BME18}{}
\nv{Talman}{Henrik Ramström}{E16}{}
\nv{Valberedningens Ordförande}{Davida Åström}{BME17}{}

\end{narvarolista}

%\begin{comment}
\subsection*{Adjungerande}
\begin{narvarolista}
%\nv{post}{namn}{klass}{nati/nafr/tom}
%\nv{Likabehandlingsombud}{Jonna Fahrman}{BME17}{}
%\nv{Likabehandlingsombud}{Hanna Bengtsson}{BME18}{}
%\nv{Projekfunktionär}{Emma Hjörneby}{BME17}{}
%\nv{Macapär}{Filip Larsson}{E17}{}
%\nv{Kodhackare}{Vincent Palmer}{E18}{}
%\nv{Sexmästare Electus}{Anna Hollsten}{BME19}{}
%\nv{Entertainer Electus}{Amir Ghanaatifard}{E19}{}
%\nv{Øverphøs Electus}{Sophia Carlsson}{BME17}{}
%\nv{Cafemästare Electus}{Frida Pilcher}{E18}{}
%\nv{ENU-ordförande Electus}{Fredrik Berg}{E17}{}
%\nv{Krögare Electus}{Love Sjelvgren}{E18}{}
%\nv{SRE-ordförande Electus}{Hanna Bengtsson}{BME18}{}
%\nv{Husstyrelserepresentant}{Joakim Magnusson Fredluend}{BME19}{}
%\nv{Sigillbevarare}{Matilda Horn}{BME18}{\nati{14}}
\nv{Cophøs}{Marcus Lindell}{BME18}{}
\nv{Cophøs}{Morgan Bryer}{E18}{}
\nv{Cophøs}{Vincent Palmer}{E18}{}
\nv{Teknikfokusansvarig}{Alexander Wik}{BME17}{}
%\nv{Teknokrat}{Jonathan Benitez}{E17}{\nafr{9}}
%\nv{Cophøs}{Elina Yrlid}{E18}{}

\end{narvarolista}
%\end{comment}

\section*{Protokoll}
\begin{paragrafer}
\p{1}{OFMÖ}{\bes}
Ordförande {\mo} förklarade mötet öppnat kl 12.10

\p{2}{Val av mötesordförande}{\bes}
{\valavmo}

\p{3}{Val av mötessekreterare}{\bes}
{\valavms}

\p{4}{Val av justeringsperson}{\bes}
{\valavj}

\p{5}{Godkännande av tid och sätt}{\bes}
{\tosg}

\p{6}{Adjungeringar}{\bes}
%Adam Belfrage adjungerades.{}
%Hanna Bengtsson adjungerades. \\
%Jonna Fahrman adjungerades.
%Vincent Palmer adjungerades.\\
%Filip Larsson adjungerades. 

%Anna Hollsten adjungerades. 

%Amir Ghanaatifard adjungerades.

%Sophia Carlsson adjungerades.

%Fredrik Berg adjungerades.

%Frida Pilcher adjungerades.

%Rasmus Solel adjungerades.

%Love Sjelvgren adjungerades.

%Hanna Bengtsson adjungerades. 

%Joakim Magnusson Fredluend adjungerades.

%Jonathan Benitez adjungerades.

Alexander Wik adjungerades.

Vincent Palmer adjungerades.

Morgan Bryer adjungerades.

Marcus Lindell adjungerades.

%\textit{Inga adjungeringar.}


\p{7}{Godkännande av dagordningen}{\bes}

%Davida \ypa lägga till punkten ``Lophtet'' till dagordningen.\\
%Edvard \ypa lägga till punkten ``Ordensband'' til dagordningen.
%Fredrik \ypa att lägga till \S18b ``Teknikfokus utnyttjande av LED-café''.
%Jonathan \ypa ändra punkten §12 från att vara en beslutspunkt till diskussion. \\
%Föredragningslistan godkändes med yrkandet.
%Henrik \ypa lägga till punkten ``Faktura till F'' som §13.
%Jakob Pettersson \ypa tägga till punkten ''Øverphøs informerar'' som \S16.


%Theo \ypa lägga till punkten 'Äskning av pengar för inköp av hemsida på annan server' som §16

%Theo \ypa lägga till punkten 'Inköp av en Raspberry Pi' som §17

Henrik \ypa lägga till sena handlingen 'Äskning av pengar för inköp av extern hårddisk' som §16.

Alexander \ypa lägga till diskussionspunkten 'Teknikfokus och LED-café' som §17.

Föredragningslistan godkändes med samtliga yrkande.
%Föredragningslistan godkändes med yrkandet.
%Föredragningslistan godkändes.

\p{8}{Föregående mötesprotokoll}{\bes}
\latillprotgodkand{S03/20}
%\textit{\ingaprot}

\p{9}{Fyllnadsval och entledigande av funktionärer}{\bes}
\begin{fyllnadsval} %"Inga fyllnadsval." fylls i automatiskt
%\fval{Moa Rönnlund}{Halvledare}
%\entl{Fanny Månefjord}{Husstyrelserepresentant från och med 30 juni}
%\fval{Klara Wahldén}{Inköps och Lagerchef från och med 1 januari 2020}

%\fval{Amanda Darell}{Näringslivskontakt}


\fval{Daniel Bakic}{Banan}

\end{fyllnadsval}

\p{10}{Rapporter}{}
\begin{paragrafer}
\subp{A}{Hur mår alla?}{\info}

Punkten protokollfördes ej.

\subp{B}{Utskottsrapporter}{\info}

Theo var på OK-skiphte i lördags. 
Pratat med D-sektionen om skiphtet, mer information inväntas. 
Börjat planera Vårterminsmötet med Talman. 
Var på fullmäktigemöte för att få vår stadga stadfäst, vilket den blev. 
Nu kan vi göra redaktionella ändringar med godkännande från fullmäktiga.
Diskuterat Cedervall med D-sek. Styrelsen var och lämnade tårta till PH och tryckeriet. 
Nästan klar med bokföring av KPL. 
Organiserat styrelsedriven.

Frida och Cafémästeriet bakade kärleksmums i söndags inför alla hjärtans dag på fredag. 
Den här veckan kan man köpa delicato-bollar att ge till någon man tycker om, personen kommer att få ett anonymt mail på fredag. 
Tydligare regler för vad som gäller angående arbetsglädje och att köpa saker från LED Café har skrivits ner. 
I lördags hade de brunch och kickoff med cafékollegiet.

Rasmus meddelar att det inte har hänt så mycket nytt i Förvaltningsutskottet. 
Kontinuerligt arbete med bokföring har fortsatt. 
Skattmästaren har inte kommit igång med bokföringen i Fortnox, men de ska arbeta med det på onsdag tillsammans med föregående skattmästare. 

Hannes och Informationsutskottet hade sin kick-off i helgen och utskottet verkade uppskatta det.
Kontaktor och Vice har haft sitt uppstartsmöte med redaktionen och de jobbar på bra.
Picassos har haft det fullt upp.
Ett schema för Fotografer och Teknokrater har upprättats.
Arbetet med kodhackarprojekt fortgår.
Macapärerna har påbörjat migrering av Git-projekt till GitLab.
Utskottet har planerat upplägg av våra Raspberry Pi:s.
I nuläget så ska utskottet inte vidare uppdatera eller göra annan destruktiv aktivitet på servrarna och istället reparera det som är trasigt så att daglig verksamhet kan fortgå, istället ska serverbytet senare i vår ske i samband med ett helt byte av mjukvara av hemsida. Detta utvecklas parallellt. InfU planerar att bibehålla GitLab som permanent lösning.

Love och Källarmästeriet har haft gille förra fredagen och alla som var där verkade uppskatta det, det var uppskattat att InfU kunde kombinera det med sin kick-off. 
De nyinköpta kortlekarna verkade verkligen uppskattas. 
Utskottet köpte även in ett nytt hänglås till Edekvata köket då detta saknades. 
Allt rullar på och gillen framöver planeras redan.

Sophia och Nolleutskottet har i veckan diskuterat pluggphaddrar tillsammans med SRE, gjort grovplanering för kommande månader, mailat företag och planerat phadderinfo till på fredag. 
Phørstärkarna har fått reda på att de är valda, fått dörraccess och de har planerat in lite häng med allihop. 
ØPK har haft en helg tillsammans, och resten av phøben har haft sittning tillsammans vilket verkade uppskattat.

Fredrik meddelar att i Näringlivsutskottet rullar arbetet på. 
Näringslivskontakterna har nu skickat sitt första mail till olika företag. 
På mötet på onsdag ska de diskutera hur det gick och se om det är något som kan förbättras eller ändras. 
Teknikfokus kör på som tåget och har flera lunchföreläsningar samt pubar. 
Förra veckan var Axis i Edekvata och invigde mikrovågsugnarna, både sektionen och Axis var nöjda. 
Fredriks konversation med S.B. Agency har fortsatt och de vill nu marknadsföra sig på vår hemsida mer permanent under 1 år, utskottet håller på att bestämma de sista detaljerna och utformar inläggen. 

Amir och Nöjesutskottet informerar om att KillErgame avslutades förra veckan och vinnaren avslöjades på gillet. 
Øverbanan hade bananuttagning och valde åtta nya bananer. 
I fredags hade de första mötet med hela utskottet. 
Stridsropen ska börja med sin planering nästa vecka. 
Som vanligt fortsätter planeringen inför UtEDischot långsamt men framåt.

Anna berättar att Sexmästeriet hade sin första sittning i lördags och det gick förträffligt. 
Tyvärr kom det färre gäster än planerat men då det var deras första sittning var det till utskottets fördel. 
Allting flöt på som det skulle och sittningen verkade lyckad.
Förutom alla förberedelser inför sittningen så har vardera mästare i utskottet haft möte med sin respektive i D-sektionens sexmästeri och utskottet är taggat för ännu en sittning nästa vecka. 

Hanna Bengtsson och Studierådet har censurerat CEQ-enkäter och bokat en del CEQ-möten. 
Årskursrepresentanerna från E1 och BME1 har lärt sig CEQ-systemet. 
Hanna själv har också varit på SRX-möte och delar av utskotten ska på utbildning för utbildningsbevakare ikväll. 
HTF har letat faddrar, vilket har lyckats. 

Davida informerar att Valberedningen har kommit igång och har varit på utbildning i kompetensbaserad valberedning från kåren. 
Det nystartade valberedningskollegiet ska starta upp.
De har även påbörjat sin planering inför Vårterminsmötet.

\subp{C}{Ekonomisk rapport}{\info}
 
Rasmus informerar att det ser bra ut, men att kvitton inte kommit in i Fortnox än så en överblick är svår att få.
 
\subp{D}{Kåren informerar}{\info}

Fullmäktigval kommer ta plats den 5:e april. Då kan man söka förtroendeposter på kåren. Under datumen 6:e februari till 15:e mars kan man nominera. 

Denna veckan håller kåren i Green PubenPuben. Där kommer serveras god, grön mat.

2021 fyller lth 60 år, och kåren har en projektgrupp för firandet av detta. Är du intresserad av att delta kan du kontakta utbildningsansvarig för externa frågor.

Utbildningsutskottet kommer sitta på expeditionen för att svara på utbildningsrelaterade frågor på fredag kl.13:00 till kl.15:00.

Blodbussen kommer till LTH campus 18:e februari. Drop in finns, men anmälan rekommenderas.

På fulmäktigemötet i torsdags kom det en motion att donera till AFborgens renovering.
Då TLTH ska donera en summa så ska kåren få någon slags symbol i AFborgen som tack.

TLTH Förvaltning AB kommer övergå till en konsultfirma med mål att anställa studenter.

\subp{E}{Omvärldsrapport}{\info}

Hannes har haft kontakt med Aalto inför jubileumsbalen.

Inga övriga anmodningar har mottagits.

\end{paragrafer}

\p{11}{NollU - Transparans mot Styrelsen}{\dis}

Theo introducerar diskussionspunkten med att påpeka att detta togs upp förra året också och frågar mötet till vilken mån NollU ska dela med sig med information om nollningensteman och dylikt.

Sophia noterar att hon endast håller saker hermligt för att det ofta uppskattas och inte för att hon har något underliggande syfte bakom.
Hon beskriver hur uppdelningen av delade Drives kommer se ut, och att det ger viss inblick för styrelsen.
Sedan påpekar hon att man kan alltid fråga Phøset om det uppkommer ett specifikt spörsmål.

\p{12}{Äskning - Stolar Arkivet MH04 s.1}{\bes}

Sophia presenterar motionen.

Henrik påminner att all inventarie bör belasta FVU.

\Mdf

Sophia lägger ner motionen.

\p{13}{Vårterminsmötet MH04 s.2-3}{\bes}

Henrik presenterar diskussionspunkten och pekar på viktiga datum i handlingen.

Alexander säger att det kan bli kort med tid då Teknikfokusansvarig ska välja in funktionärer.

Mötet diskuterade datum och planering.

\Mba Vårterminsmötet tar plats 6 Maj.

\p{14}{Träffa Cedervall och ny husprefekt}{\dis}

Nästa styrelsemöte, 18:e mars, kommer sektionens inspektor på besök.

Veckan efter, 25:e mars, kommer Mats Cedervall och Johan, vår nya husprefekt, på besök.

\p{15}{Styrelsedriven}{\info}

Theo har omorganiserat i styrelsedriven.

I varje mapp finns en submapp för föregående år, för att hålla driven organiserad.

Endast ett fåtal filer ska ligga direkt i styrelsedriven.

\p{16}{Äskning av pengar för inköp av extern hårddisk}{\bes}

Henrik presenterar motionen.

\Mdf

\Mba bifalla motionen.

\p{17}{Teknikfokus och LED-café}{\dis}

Alexander informerar om att Teknikfokus kommer vara nästa vecka.
LED-café kommer användas till att servera frukost för deltagare i.
Tidigare har det blivit stora oklarheter kring vad som skall köpas och vad det är som bjuds på.
Alexander hade tidigare undersökt med Frida om det är möjligt att hålla caféet stängt under frukosttimmar men hon motsätter sig detta.

Frida föreslår att Teknikfokus använder silverkylarna då detta område inte används av LED.
Hon förbereder sig för problem som kan uppstå i samband med detta.

\p{18}{Nästa styrelsemöte}{\bes}

\Mba nästa styrelsemöte ska äga rum 2020-02-18 12.10 i okänd lokal.

\p{19}{Beslutsuppföljning}{\bes}

%Jonathan \ypa stryka ''Inköp av Raspberry Pi'' från beslutsuppföljningen.

%\Mbaby 

\textit{Inga beslut följdes upp.}

\p{20}{Övrigt}{\dis}

Theo påpekar att alla i styrelsen inte ska ha access till exempelvis dörrhantering på hemsidan, och att detta bör begränsas till en rimlig nivå.
Theo, Rasmus och Hannes ska sätta sig med det i veckan och fundera ut en rimlig lösning och sedan skicka ut till utskotten.

Frida påpekar att det i lördags hämtades 2 flak läsk, och sedan sedan skickades faktura.
Hon stryker under att man först ska fråga och sedan kan man hämta.

Love märker av att han kan inte hålla sig inom budget för arbetskläder med behovet han har.
Mötet påminner att vi kan gå över budget, även om det inte är rekommenderat.

Love undersöker möjligheten om en lunchsittning är möjlig.
Mötet är lagom intresserat.

Sophia noterar att F-sektionens fös jobbar i sitt café under våren och tar sina luncher på hösten.
Frida säger att det är helt möjligt även för E-sektionens phøs.

\p{21}{Sammanfattning av mötet}{\info}

Kåren gav en mängd bra information.

Flera punkter diskuterades.

Datum för Vårterminsmötet bestämdes.

Daniel Bakic valdes som Banan.

Mindre frågor diskuterades under 'Övrigt'.

\p{22}{OFMA}{\bes}
{\mo} förklarade mötet avslutat kl. 13.05
\end{paragrafer}

%\newpage
\hidesignfoot
\begin{signatures}{3}
\signature{\mo}{Mötesordförande}
\signature{\ms}{Mötessekreterare}
\signature{\ji}{Justerare}
\end{signatures}
\end{document}
