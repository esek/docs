\documentclass[10pt]{article}
\usepackage[utf8]{inputenc}
\usepackage[swedish]{babel}

\def\mo{Theo Nyman}
\def\ms{Hannes Björk}
\def\ji{Fredrik Berg}
%\def\jii{}

\def\doctype{Protokoll} %ex. Kallelse, Handlingar, Protkoll
\def\mname{Styrelsemöte} %ex. styrelsemöte, Vårterminsmöte
\def\mnum{S03/20} %ex S02/16, E1/15, VT/13
\def\date{2020-02-04} %YYYY-MM-DD
\def\docauthor{\ms}

\usepackage{../e-mote}
\usepackage{../../../e-sek}

\begin{document}
\showsignfoot

\heading{{\doctype} för {\mname} {\mnum}}

%\naun{}{} %närvarane under
%\nati{} %närvarande till och med
%\nafr{} %närvarande från och med
\section*{Närvarande}
\subsection*{Styrelsen}
\begin{narvarolista}
\nv{Ordförande}{Theo Nyman}{BME18}{}
\nv{Kontaktor}{Hannes Björk}{E17}{}
\nv{Förvaltningschef}{Rasmus Sobel}{BME16}{}
\nv{Cafémästare}{Frida Pilcher}{E18}{} 
\nv{Sexmästare}{Anna Hollsten}{BME19}{}
\nv{Krögare}{Love Sjelvgren}{E18}{}
\nv{Entertainer}{Amir Ghanaatifard}{E19}{}
\nv{SRE-ordförande}{Hanna Bengtsson}{BME18}{}
\nv{ENU-ordförande}{Fredrik Berg}{E17}{}
\nv{Øverphøs}{Sophia Carlsson}{BME17}{\nafr{9}}
\end{narvarolista}


\subsection*{Ständigt adjungerande}
\begin{narvarolista}
%\nv{}{}{}{}
%\nv{Skattmästare}{Daniel Bakic}{E15}{\nafr{10}}
%\nv{Vice Krögare}{Klara Indebetou}{BME17}{}
%\nv{Vice Krögare}{Hjalmar Tingberg}{BME16}{}
%\nv{Vice Förvaltningschef}{Rasmus Sobel}{BME16}{}
\nv{Kårrepresentant}{Hanna Jonson}{}{}
\nv{Kårrepresentant}{Ivar Vänglund}{}{}
%\nv{Valberedningens ordförande}{Axel Voss}{E15}{\nafr{10b}}
%\nv{Fullmäktigeledamot}{Magnus Lundh}{E15}{\nafr{12}}
%\nv{Chefredaktör}{Emil Eriksson}{E18}{}
%\nv{Inspektor}{Monica Almqvist}{}{}
%\nv{Cophøs}{Elina Yrlid}{E18}{}
%\nv{Vice Kontaktor}{Matilda Horn}{BME18}{}
\nv{Talman}{Henrik Ramström}{E16}{}

\end{narvarolista}

%\begin{comment}
\subsection*{Adjungerande}
\begin{narvarolista}
%\nv{post}{namn}{klass}{nati/nafr/tom}
%\nv{Likabehandlingsombud}{Jonna Fahrman}{BME17}{}
%\nv{Likabehandlingsombud}{Hanna Bengtsson}{BME18}{}
%\nv{Projekfunktionär}{Emma Hjörneby}{BME17}{}
%\nv{Macapär}{Filip Larsson}{E17}{}
%\nv{Kodhackare}{Vincent Palmer}{E18}{}
%\nv{Sexmästare Electus}{Anna Hollsten}{BME19}{}
%\nv{Entertainer Electus}{Amir Ghanaatifard}{E19}{}
%\nv{Øverphøs Electus}{Sophia Carlsson}{BME17}{}
%\nv{Cafemästare Electus}{Frida Pilcher}{E18}{}
%\nv{ENU-ordförande Electus}{Fredrik Berg}{E17}{}
%\nv{Krögare Electus}{Love Sjelvgren}{E18}{}
%\nv{SRE-ordförande Electus}{Hanna Bengtsson}{BME18}{}
%\nv{Husstyrelserepresentant}{Joakim Magnusson Fredluend}{BME19}{}
%\nv{Sigillbevarare}{Matilda Horn}{BME18}{\nati{14}}
\nv{Cophøs}{Marcus Lindell}{BME18}{}
\nv{Teknokrat}{Jonathan Benitez}{E17}{\nafr{9}}

\end{narvarolista}
%\end{comment}

\section*{Protokoll}
\begin{paragrafer}
\p{1}{OFMÖ}{\bes}
Ordförande {\mo} förklarade mötet öppnat kl 12.11

\p{2}{Val av mötesordförande}{\bes}
{\valavmo}

\p{3}{Val av mötessekreterare}{\bes}
{\valavms}

\p{4}{Val av justeringsperson}{\bes}
{\valavj}

\p{5}{Godkännande av tid och sätt}{\bes}
{\tosg}

\p{6}{Adjungeringar}{\bes}
%Adam Belfrage adjungerades.{}
%Hanna Bengtsson adjungerades. \\
%Jonna Fahrman adjungerades.
%Vincent Palmer adjungerades.\\
%Filip Larsson adjungerades. 

%Anna Hollsten adjungerades. 

%Amir Ghanaatifard adjungerades.

%Sophia Carlsson adjungerades.

%Fredrik Berg adjungerades.

%Frida Pilcher adjungerades.

%Rasmus Solel adjungerades.

%Love Sjelvgren adjungerades.

%Hanna Bengtsson adjungerades. 

%Joakim Magnusson Fredluend adjungerades.

Jonathan Benitez adjungerades.
%Henrik Ramström adjungerades.


Marcus Lindell adjungerades.
%\textit{Inga adjungeringar.}


\p{7}{Godkännande av dagordningen}{\bes}

%Davida \ypa lägga till punkten ``Lophtet'' till dagordningen.\\
%Edvard \ypa lägga till punkten ``Ordensband'' til dagordningen.
%Fredrik \ypa att lägga till \S18b ``Teknikfokus utnyttjande av LED-café''.
%Jonathan \ypa ändra punkten §12 från att vara en beslutspunkt till diskussion. \\
%Föredragningslistan godkändes med yrkandet.
%Henrik \ypa lägga till punkten ``Faktura till F'' som §13.
%Jakob Pettersson \ypa tägga till punkten ''Øverphøs informerar'' som \S16.


%Theo \ypa lägga till punkten 'Äskning av pengar för inköp av hemsida på annan server' som §16

%Theo \ypa lägga till punkten 'Inköp av en Raspberry Pi' som §17

%Föredragningslistan godkändes med samtliga yrkande.
%Föredragningslistan godkändes med yrkandet.
Föredragningslistan godkändes.

\p{8}{Föregående mötesprotokoll}{\bes}
\latillprotgodkand{S02/20}
%\textit{\ingaprot}

\p{9}{Fyllnadsval och entledigande av funktionärer}{\bes}
\begin{fyllnadsval} %"Inga fyllnadsval." fylls i automatiskt
%\fval{Moa Rönnlund}{Halvledare}
%\entl{Fanny Månefjord}{Husstyrelserepresentant från och med 30 juni}
%\fval{Klara Wahldén}{Inköps och Lagerchef från och med 1 januari 2020}

%\fval{Amanda Darell}{Näringslivskontakt}

\fval{Tina Tabandeh}{Phørstärkare}
\fval{Tove Börjeson}{Phørstärkare}
\fval{Philip Johansson}{Phørstärkare}
\fval{Jimmy Szentes}{Phørstärkare}
\fval{Jakob Wisth}{Phørstärkare}
\fval{Jacob Rinderud}{Phørstärkare}
\fval{Emil Eriksson}{Diod}
\fval{Henrik Ramström}{Diod} 
\fval{Casper Schwerin}{Diod}
\fval{Jakob Botvidsson}{Skyddsombud med likabehandlingsansvar}
\fval{Lukas Elmlund}{Banan}
\fval{Valter Möller}{Banan}
\fval{Oskar Magnusson}{Banan}

\end{fyllnadsval}

\p{10}{Rapporter}{}
\begin{paragrafer}
\subp{A}{Hur mår alla?}{\info}

Punkten protokollfördes ej.

\subp{B}{Utskottsrapporter}{\info}


Theo har varit på kollegiemöte. Han har mottagit en del mail från Chalmers och KTH om olika event de har. Han har börjat jobba på ett framtidsperspektiv.

Cafémästare och Vice har varit på årets första kollegiemöte, de diskuterade bland annat caféfesten. Det pratas om hyra av lokal och alkoholtillstånd.
Vice jobbar på med delicatoboll-event inför alla hjärtans dag. 
Inköparna är grymma, de beställer och fixar IC själva. De ska köpa in material så att eftermiddagsdioder kan baka.
CM hade ett lunchmöte igår, förbereder dem inför Fettisdagen och Alla Hjärtans Dag. 
LU servicedesk har kontaktats i väntan på svar och lösning på problemet med Eduroam i och utanför LED. Även Per-Henrik har kontaktats och Per Foreby från datordriftsgruppen på LTH.
Cafémästare och vice har haft ett möte med Dash UF som vill sälja tuggummin i caféet.

Förvaltningsutskottet har haft sitt första möte med utskottet. Framförallt diskuterades tröjor men även planer på kick-off. Förvaltningschefen har varit på vice ordförande kollegiemöte. Kontinuerligt arbete med bokföring och fakturabetalningar fortsätter.

I InfU har Kontaktor och Vice har planerat kickoff och utskottströjor.
Teknokrater har hittat ett analogt mixerbord som funkar. Sektionen kan använda den tillsammans med vårt digitala mixerbord till sittningar såsom Gasque. De har även börjat fixa våra gamla subwoofers.
Picassos har fått beställningar som de har löst galant. De har även upprättat ett formulär för bestälningar.
Arbetet går vidare med servern och Macapärerna har hittat en workaround för projektplats hacke och databasen. De har även löst de mer akuta problemen men AHS:systemet.
Vidare funderar vi på att för tillfället överge projektplats hacke och istället använda Github.
Till kvällen för styrelsemötet ska vi ha möte med ordförande för ETF och få hjälp med att uppdatera servern.
Ett schema för kommande aktivitet för Teknokrater och Fotografer har upprättats.

Källarmästeriet har legat lågt en vecka då det var svårt att hålla gille under KPL. Alla krögare har snart genomgått A-cert. De också har planerat inför fredag samt fixat restaurangrapporten ännu en gång

Nollningsutskottet har i veckan bland annat pratat om phaddrar, valt datum för phadderinfo, temasläpp och kick off, samt valt teman, som ska bestämmas definitivt på ØPK nästa måndag. Tiderna för phadderintervjuerna är satta, en strategi om hur processen ska effektiviseras ska testas där NollU kommer att delas upp och därmed ha flera intervjuer samtidigt. 
Kontakt med SVL har påbörjats, och ett möte med honom ska bestämmas, samt möte med Yvon som ska göra våra mantlar. 
Delar av phøset har gått på A-cert, och har andra delen ikväll. Igår bestämdes vår dag för gasque.

Näringlivsutskottet har haft sitt första möte med utskottet. Gick där mest igenom grundläggande info och startade upp mailandet. Har även fortsatt kontakten med Axis och de kommer nu på torsdag för att dela ut pizza till studenterna i Edekvata. De kollar även på sitt första event för året, en CV-fotografering på måndagen innan Teknikfokus den 17:e Februari.

I Nöjesutskottet så kollade Amir och William, UtEDischoansvarig, på den första delen av A-cert och ikväll ska de dit igen. Amir och Saga var på första kollegiemötet med AktU och träffade aktivitetsansvariga från de andra sektionerna i torsdags. Det pratades om hur de vill fungera som utskott och hur de kan få hjälp av Teknologkåren. På kvällen hade de spelkväll igen och det kom fler än veckan innan. Denna vecka har de killErgame och 60st personer skrev upp sig vilket var jättekul. Det eventet kommer fortsätta hela veckan fram tills 19:00 på fredag. I söndags fixade idrottsförmännen årets första Sporta med E och det kom runt 25st personer. 

Sexmästeriet har precis som flera andra varit på A-cert förra tisdagen. Annars har större delen av veckan gått åt till planering för kick-off-sittningen på lördag. Köket handlade igår och hovets beställning har kommit. De hade även kvällsmöte med D6 igår inför Banquetten, vilket gick väldigt bra. Både drink- och matmenyn sattes preliminärt och alla mästare har planerat in ytterligare ett möte med dess respektive i D6.

Studierådet har haft det första mötet med hela utskottet. De brainstormade om alla ledamöter och vilka evenmang de hade kunnat hålla i. Det diskuterades mycket kring konceptet med pluggkvällar och om eller hur de hade kunnat fortsätta efter nollningen. Hanna har också haft möte med studievägledaren och diskuterat om hur vi kan samarbeta. Her Tech Future har stått i foajen och för att få folk att söka phaddrar. 

\subp{C}{Ekonomisk rapport}{\info}
 
Ekonomin går bra.

\subp{D}{Kåren informerar}{\info}

Kåren har flera verksamhetsplanspunkter som vi kan delta i.
Exempelvis skulle kåren hålla i en pub med miljöfokus och uppmanar sektionsmedlemmar att delta, denna erbjuder 'klimat och najs mat'. Event finns på Facebook.

Styrelsemedlemmarna uppmanas att delta i kårens styrelseutbildning.

\subp{E}{Omvärldsrapport}{\info}

Theo informerar om ett antal mail Chalmers och KTH har skickat ut om utomlunda events. Ett gemensamt inlägg på Facebook med dessa events planeras. Vidare vill de även besöka oss på vårt sektionsmöte.

\end{paragrafer}

\p{11}{Extraphaddrar under nollning}{\dis}

Theo öppnar diskussionen om hurvida styrelsen är lämpliga som extraphaddrar.

Sophia informerar om att extraphaddrar kommer ha samma upplägg som tidigare år. Det vill säga att de kommer inte agera MVP och kommer inte ha lika mycket ansvar som ordinarie phaddrar. Hon ber även styrelsen beakta hur nollorna upplever situationen framför allt.

Theo tillägger att det trots extraphadderposten är styrelsearbetet som går först och att det finns risk för överbelastning då man försöker sig på båda poster.

Amir uttrycker att entertainers ändå brukar vara på evenemang så de kan lika gärna vara extraphadder. Henrik förklarar att det är både till fördel och nackdel då man tar en plats från någon som annars hade kunnat delta 

Sophia påvisar att syftet är att visa upp hur sektionsarbetet kan se ut på närmare håll. Slutmålet på detta är att det bidrar till engagemang. Vidare ska vi punktera att de inte har huvudansvar men att det inte går att leka bort helt och hållet. Hon förklarar att endast vissa poster kommer kunna bli extraphaddrar.

Anna hade velat att styret representerar sig själva som styrelsen under nollningen.
Amir tycker att det kan bli en vi-och-dem-mentalitet med det synsättet.
Frida anser att även om styrelsen rör sig mycket internt så blir det inte exkluderande.

Theo summerar; det är okej att söka phadder som styrelse, men vissa krav sätts på styrelsen och en post ska inte gå ut över en annan.

\p{12}{Funktionärsskiphtet}{\dis}

Theo informerar om att D-sektionen har inte svarat.

Planering fortsätter internt.

\p{13}{Nästa styrelsemöte}{\bes}

\Mba nästa styrelsemöte ska äga rum 2020-02-11 12.10 i E:1123.

\p{14}{Beslutsuppföljning}{\bes}

Jonathan upplyser om att motionen ''Inköp av Raspberry Pi'' gick under budget.

Jonathan \ypa stryka ''Inköp av Raspberry Pi'' från beslutsuppföljningen.

\Mbaby 

Hannes \ypa stryka ''Inköp av squarespace'' från beslutsuppföljningen.

\Mbaby

\p{15}{Övrigt}{\dis}

Amir undersöker om utförandet av killErgame kan anses stökigt.

Vidare hade NöjU möte med D-sektionen i onsdags. Alla organisatörer vill expandera operationen, och han föreslår behovet av en till UtEDischoansvarig. 
Theo uttrycker att det behövs en proposition till Vårterminsmötet.


Frida frågar om någon vet om skräpet från återvinningskärl faktiskt återvinns.
Henrik är säker på att kärlen ute vid vaktmästeriet drivs ordentligt, han är dock osäker på de i foajen.
Theo ska skicka frågan med husstyrelserepresentanten.

Utdelning av medaljer diskuteras.

Styrelsen diskuterade möten mellan NollU och övriga utskotten.

\p{16}{OFMA}{\bes}
{\mo} förklarade mötet avslutat kl. 12.58
\end{paragrafer}

%\newpage
\hidesignfoot
\begin{signatures}{3}
\signature{\mo}{Mötesordförande}
\signature{\ms}{Mötessekreterare}
\signature{\ji}{Justerare}
\end{signatures}
\end{document}
