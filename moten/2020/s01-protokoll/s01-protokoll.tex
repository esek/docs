\documentclass[10pt]{article}
\usepackage[utf8]{inputenc}
\usepackage[swedish]{babel}

\def\mo{Theo Nyman}
\def\ms{Hannes Björk}
\def\ji{Rasmus Sobel}
%\def\jii{}

\def\doctype{Protokoll} %ex. Kallelse, Handlingar, Protkoll
\def\mname{Styrelsemöte} %ex. styrelsemöte, Vårterminsmöte
\def\mnum{S01/20} %ex S02/16, E1/15, VT/13
\def\date{2020-01-21} %YYYY-MM-DD
\def\docauthor{\ms}

\usepackage{../e-mote}
\usepackage{../../../e-sek}

\begin{document}
\showsignfoot

\heading{{\doctype} för {\mname} {\mnum}}

%\naun{}{} %närvarane under
%\nati{} %närvarande till och med
%\nafr{} %närvarande från och med
\section*{Närvarande}
\subsection*{Styrelsen}
\begin{narvarolista}
\nv{Ordförande}{Theo Nyman}{BME18}{}
\nv{Kontaktor}{Hannes Björk}{E17}{}
\nv{Förvaltningschef}{Rasmus Sobel}{BME16}{}
\nv{Cafémästare}{Frida Pilcher}{E18}{} 
\nv{Sexmästare}{Anna Hollsten}{BME19}{\naun{12}{20}}
\nv{Krögare}{Love Sjelvgren}{E18}{}
\nv{Entertainer}{Amir Ghanaatifard}{E19}{}
\nv{SRE-ordförande}{Hanna Bengtsson}{BME18}{}
\nv{ENU-ordförande}{Fredrik Berg}{E17}{}
\end{narvarolista}


\subsection*{Ständigt adjungerande}
\begin{narvarolista}
%\nv{}{}{}{}
%\nv{Skattmästare}{Daniel Bakic}{E15}{\nafr{10}}
%\nv{Vice Krögare}{Klara Indebetou}{BME17}{}
%\nv{Vice Krögare}{Hjalmar Tingberg}{BME16}{}
\nv{Kårrepresentant}{Ivar Vänglund}{}{\nati{16}}
%\nv{Vice Förvaltningschef}{Rasmus Sobel}{BME16}{}
\nv{Kårrepresentant}{Hanna Jonson}{}{\nati{20}}
%\nv{Valberedningens ordförande}{Axel Voss}{E15}{\nafr{10b}}
%\nv{Fullmäktigeledamot}{Magnus Lundh}{E15}{\nafr{12}}
%\nv{Chefredaktör}{Emil Eriksson}{E18}{}
%\nv{Inspektor}{Monica Almqvist}{}{}
\nv{Cophøs}{Elina Yrlid}{E18}{}
\nv{Vice Kontaktor}{Matilda Horn}{BME18}{}
\nv{Talman}{Henrik Ramström}{E16}{}

\end{narvarolista}

%\begin{comment}
\subsection*{Adjungerande}
\begin{narvarolista}
%\nv{post}{namn}{klass}{nati/nafr/tom}
%\nv{Likabehandlingsombud}{Jonna Fahrman}{BME17}{}
%\nv{Likabehandlingsombud}{Hanna Bengtsson}{BME18}{}
%\nv{Projekfunktionär}{Emma Hjörneby}{BME17}{}
%\nv{Macapär}{Filip Larsson}{E17}{}
%\nv{Kodhackare}{Vincent Palmer}{E18}{}
%\nv{Sexmästare Electus}{Anna Hollsten}{BME19}{}
%\nv{Entertainer Electus}{Amir Ghanaatifard}{E19}{}
%\nv{Øverphøs Electus}{Sophia Carlsson}{BME17}{}
%\nv{Cafemästare Electus}{Frida Pilcher}{E18}{}
%\nv{ENU-ordförande Electus}{Fredrik Berg}{E17}{}
%\nv{Krögare Electus}{Love Sjelvgren}{E18}{}
%\nv{SRE-ordförande Electus}{Hanna Bengtsson}{BME18}{}
%\nv{Husstyrelserepresentant}{Joakim Magnusson Fredluend}{BME19}{}
%\nv{Sigillbevarare}{Matilda Horn}{BME18}{\nati{14}}
\nv{Teknokrat}{Jonathan Benitez}{E17}{}


\end{narvarolista}
%\end{comment}

\section*{Protokoll}
\begin{paragrafer}
\p{1}{OFMÖ}{\bes}
Ordförande {\mo} förklarade mötet öppnat kl 12.11

\p{2}{Val av mötesordförande}{\bes}
{\valavmo}

\p{3}{Val av mötessekreterare}{\bes}
{\valavms}

\p{4}{Val av justeringsperson}{\bes}
{\valavj}

\p{5}{Godkännande av tid och sätt}{\bes}
{\tosg}

\p{6}{Adjungeringar}{\bes}
%Adam Belfrage adjungerades.{}
%Hanna Bengtsson adjungerades. \\
%Jonna Fahrman adjungerades.
%Vincent Palmer adjungerades.\\
%Filip Larsson adjungerades. 

%Anna Hollsten adjungerades. 

%Amir Ghanaatifard adjungerades.

%Sophia Carlsson adjungerades.

%Fredrik Berg adjungerades.

%Frida Pilcher adjungerades.

%Rasmus Solel adjungerades.

%Love Sjelvgren adjungerades.

%Hanna Bengtsson adjungerades. 

%Joakim Magnusson Fredluend adjungerades.

Jonathan Benitez adjungerades.
Henrik Ramström adjungerades.
%\textit{Inga adjungeringar.}


\p{7}{Godkännande av dagordningen}{\bes}

%Davida \ypa lägga till punkten ``Lophtet'' till dagordningen.\\
%Edvard \ypa lägga till punkten ``Ordensband'' til dagordningen.
%Fredrik \ypa att lägga till \S18b ``Teknikfokus utnyttjande av LED-café''.
%Jonathan \ypa ändra punkten §12 från att vara en beslutspunkt till diskussion. \\
%Föredragningslistan godkändes med yrkandet.
%Henrik \ypa lägga till punkten ``Faktura till F'' som §13.
%Jakob Pettersson \ypa tägga till punkten ''Øverphøs informerar'' som \S16.


Theo \ypa lägga till punkten 'Äskning av pengar för inköp av hemsida på annan server' som §16

Theo \ypa lägga till punkten 'Inköp av en Raspberry Pi' som §17

Föredragningslistan godkändes med samtliga yrkande.
%Föredragningslistan godkändes med yrkandet.
%Föredragningslistan godkändes.

\p{8}{Föregående mötesprotokoll}{\bes}
%\latillprotgodkand{S25/19}
\textit{\ingaprot}

\p{9}{Fyllnadsval och entledigande av funktionärer}{\bes}
\begin{fyllnadsval} %"Inga fyllnadsval." fylls i automatiskt
%\fval{Moa Rönnlund}{Halvledare}
%\entl{Fanny Månefjord}{Husstyrelserepresentant från och med 30 juni}
%\fval{Klara Wahldén}{Inköps och Lagerchef från och med 1 januari 2020}

\fval{Jonathan Benitez}{Halvledare}
\fval{Sofie Johanneson}{Halvledare}
\fval{Evelina Morgan}{Halvledare}
\fval{Klara Eriksson}{Halvledare}
\fval{Alicia Lindmark}{Halvledare}
\fval{Johannes Larsson}{Diod}
\fval{Rasmus Sobel}{Diod}
\fval{Max Mauritsson}{Diod}
\fval{Arvid Hansson}{Diod}
\fval{Axel Sweger}{Diod}
\fval{Guillermo Palomino Lozano}{Diod}
\fval{Hannes Björk}{Diod}
\fval{Mattias Lundström}{Diod}
\fval{Jakob Pettersson}{Diod}
\fval{Jakob Wisth}{Diod}
\fval{Linn Gromert}{Diod}
\fval{Melina Alnasser}{Diod}
\fval{Petter Melander}{Diod}
\fval{Adam Lüni}{Diod}
\fval{Erik Wickström}{Diod}
\fval{León Salueña Martinez}{Diod}
\fval{Louise Rehme}{Diod}
\fval{Malva Persmark}{Diod}
\fval{Hilda Eliasson}{Diod}
\fval{Malin Heyden}{Diod}
\fval{Johanna Bengtsson}{Diod}
\fval{Jennie Karlsson}{Diod}
\fval{Richard Byström}{Diod}
\fval{Oskar Magnusson}{Diod}
\fval{Fabian Sondh}{Diod}
\fval{Tor Hammarbäck}{Diod}
\fval{Kai Zbirka}{Diod}
\fval{Enrico Corato}{Diod}
\fval{Stephanie Bol}{Näringslivskontakt}
\fval{Melina Alnasser}{Näringslivskontakt}
\fval{Linn Gromert}{Näringslivskontakt}
\fval{Jonathan Benitez}{Näringslivskontakt}
\fval{Richard Byström}{Näringslivskontakt}
\fval{Elin Johansson}{Näringslivskontakt}
\fval{Matilda Horn}{Näringslivskontakt}
\fval{Amanda Gustafsson}{Näringslivskontakt}
\fval{Markus Rahne}{Näringslivskontakt}
\fval{Amanda Zarkout}{Näringslivskontakt}
\fval{Evelina Morgan}{Näringslivskontakt}
\fval{Oliver Lindblom}{Näringslivskontakt}
\fval{Oskar Branzell}{Näringslivskontakt}
\fval{Ludvig Lifting}{Näringslivskontakt}
\fval{Isa Clementsson}{Näringslivskontakt}
\fval{Nelly Ostréus}{Näringslivskontakt}
\fval{Niklas Gustafson}{Alumniansvarig E}
\fval{Sanna Nordberg}{Alumniansvarig BME}
\entl{Olivia Matsson}{Skyddsombud med likabehandlingsansvar}

\end{fyllnadsval}

\p{10}{Rapporter}{}
\begin{paragrafer}
\subp{A}{Hur mår alla?}{\info}

Punkten protokollfördes ej.

\subp{B}{Utskottsrapporter}{\info}

Theo har haft många utbildningar.

Hanna har valt funktionärer, inte mycket har gjorts än i SRE.

Elina meddelar att PhørstärkarEintervjuerna är snart och att de har fått in fler än 6 ansökningare.

Rasmus skött inbetalningar och övrig mindre verksamhet.

Fredrik har valt funktionärer till ENU. De har redan fått några förfrågningar från företag.

Love har Gille på fredag. Han har hittat någon som kan hjälpa honom med alkoholtillstånd.

Frida meddelar att caféet har öppnat. Det börjar rulla på, och Inköps- och lagercheferna börjar ha koll.

Amir har hållt i E-spark. Har vidare även pratat med F-sektionen om Sporta med E och kollat på Utedischot.

Hannes har jobbat på en strategi för hemsidan och har påbörjat möten med eget utskott.

\subp{C}{Ekonomisk rapport}{\info}
 
Rasmus är förvirrad. Vidare arbete pågår. 

Ekonomin ser bra ut i nuläget.

\subp{D}{Kåren informerar}{\info}

Kåren har 3 nya heltidare. 

Det finns två vakanser i kårstyrelsen som behandlas på fullmäktigemötet. 

Kåren höll i welcome party i lördags. 

I framtiden är mecenatkort framför allt digitala, men om man är riktigt 'old school' kan man beställa in ett fysiskt. 

\subp{E}{Omvärldsrapporter}{\info}

Hannes informerar om att två styrelsemedlemmar har anmodats till Aalto Universitet i Helsingfors.

Många julkort har kommit den senaste månaden.

\end{paragrafer}

\p{11}{Vice Ordförande}{\bes}

Theo förklarar vilka ansvar att vara Vice Ordförande innefattar.

Theo nominerade Rasmus Sobel till Vice Ordförande.

\Mba välja Rasmus Sobel till Vice Ordförande.

\p{12}{Städvecka}{\info}

Theo informerar om ett excelark i styrelsedriven över städansvar, och vidare om städområden. Han påminner att ta hjälp av utskottet.

\p{13}{Avig och braig}{\info}

Theo informerar att vid evenemang behöver det finnas en ansvarig. Denna ska anmälas till Per-Henrik som nås vid ph@ehuset.lth.se. Om vi har evenemang på annat ställe än Edekvata ska man ansöka om tillstånd hos Mats Cedervall.

\p{14}{Budget till Kickoff}{\dis}

Rasmus och Theo föreslog en budget för att använda till utskottskickoffer. Detta låter oss ha ett lika stort funktionärstack som förra året.

\p{15}{Funkionärsskhiphte}{\dis}

Theo informerar om att lokal är bokad inför Funktionärsskhiphtet, 6:e Mars.

Vidare berättar han att tillståndsmyndigheten till högre grad kommer gå ut och kontrollera tillstånd på sittningar, och att det är viktigt att vi förhåller oss till detta.

Ett förslag till upplägg av Funkionärsskhiphte diskuterades.

\p{16}{Äskning av pengar för inköp av hemsida på annan server}{\bes}

Hannes presenterade motionen.

\Mdf

Theo \ypa sätta budget på 600kr.

Hannes \js för yrkandet.

\Mba bifalla motionen med alla tilläggsyrkande.

\p{17}{Inköp av en ny Raspberry Pi}{\bes}

Jonathan presenterar motionen.

Han förklarar ytterligare brister hemsidan har i nuläget.

\Mdf

\Mba bifalla motionen.


\p{18}{Nästa styrelsemöte}{\bes}

\Mba nästa styrelsemöte ska äga rum 2020-01-28 12.10 i E:1123.

\p{19}{Beslutsuppföljning}{\bes}

Theo \ypa stryka ''Inköp av symaskin'' från Beslutsuppföljning.

\Mbaby 

Theo \ypa stryka ''Funktionärstacket'' från Beslutsuppföljning.

\Mbaby 

\p{20}{Övrigt}{\dis}

Henrik informerar om den ekonomiska uppföljningen.

Theo undersöker möjligheten att använda Swish. Rasmus tycker att det hjälper under nollningen. Henrik berättar om bokföringsproblem, och avråder användning.

Theo påminner om att alltid ha liggare under alkoholevenemang.

Love planerar att beställa nya arbetskläder för KM, och undersöker om andra utskott har behov.

Utskottströjor diskuterades.

Hanna diskuterade det bästa sättet att söka efter nya likabehandlingsombud.

Hannes informerar om användning av G-Suite, anslagstavlor och övrig teknisk information.

\p{21}{OFMA}{\bes}
{\mo} förklarade mötet avslutat kl. 13.14
\end{paragrafer}

%\newpage
\hidesignfoot
\begin{signatures}{3}
\signature{\mo}{Mötesordförande}
\signature{\ms}{Mötessekreterare}
\signature{\ji}{Justerare}
\end{signatures}
\end{document}
