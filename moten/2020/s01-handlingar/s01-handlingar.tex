\documentclass[10pt]{article}
    \usepackage[utf8]{inputenc}
    \usepackage[swedish]{babel}
    
    \def\doctype{Handlingar} %ex. Kallelse, Handlingar, Protkoll
    \def\mname{Styrelsemöte} %ex. styrelsemöte, Vårterminsmöte
    \def\mnum{S1/20} %ex S02/16, E1/15, VT/13
    \def\date{2020-01-20} %YYYY-MM-DD
    \def\docauthor{Hannes Björk}
    
    \usepackage{../e-mote}
    \usepackage{../../../e-sek}
    
    \begin{document}
    
    \heading{{\doctype} till {\mname} {\mnum}}
    
    \section*{Äskning av pengar för inköp av hemsida på annan server}
    
 	Vår hemsida eee.esek.se har länge varit utdaterad, men ingen har än tagit hand om den. Det har nu kulminerat i att hemsidan inte längre fungerar på eduroam, och har vid skrivande punkt inte heller har gjort det i ungefär två veckor. Jag anser därför att vi måste handla snabbt. 
 	
 	Squarespace är en hemsida som tillåter en att med deras enkla(re) verktyg bygga och upprätthålla en hemsida. Jag har för några veckor sedan testat detta och att använda båda hemsidorna är möjligt samtidigt. Där den säkra och snygga hemsidan som Squarespace förser oss med agerar som framsida, men att tillgången till gamla sidan är kvar.
 	

Jag yrkar 


   \begin{attsatser}
        \att köpa Squarespace personal i 3 månader á 16USD per månad. Vilket uppkommer till ca. 457SEK. 
        \att budget sätts till \SI{470}{kr},
        \att kostnaden belastar dispositionsfonden, samt
        \att detta läggs på beslutsuppföljningen till S2/20 med undertecknad som ansvarig. 
    \end{attsatser}
    
    
    \begin{signatures}{1}
    \textit{\ist}
    \signature{Hannes Björk}{Kontaktor}
    \end{signatures}
    
\newpage
\section*{Inköp av en ny Raspberry Pi}
    
 	Som det ser ut just nu så står vår hemsida på sina sista ben. I och med att vår server är så utdaterad så kommer vår hemsida lägga ner i mars om vi inte uppdaterar mjukvaran. Vilket just nu är omöjligt om vi inte köper in en ny server, och tyvärr kommer vi inte hinna göra detta fram till mars. Därför vill vi köpa in en ny Raspberry Pi som en temporär lösning som ska vara en proxy till vår server, därför vill vi att det ska vara en Raspberry Pi av modell 4 då denna har mer processorkraft. Detta kommer även förhoppningsvis göra så att vi kan komma åt vår hemsida trots att servern är utdaterad. Vi kommer även kunna använda Raspberry Pi i framtiden till andra projekt, i och med att vi just nu har en brist på Pi:s. Till denna Pi kommer vi även behöva utrustning för att kunna få allting att fungera, som en usb-c till nätverkskort, 32gb sd kort och en power supply.  
 	

Jag yrkar 


   \begin{attsatser}
        \att att köpa in en Raspberry Pi av modell 4, en OKdo Raspberry Pi Power Supply (\href{https://www.webhallen.com/se/product/303341-Raspberry-Pi-4-Model-B-enkortsdator-4GB?ref=Prisjakt}{\textit{länk}}), Deltaco Gigabit-nätverkskort USB-C (\href{https://www.webhallen.com/se/product/287806-Deltaco-Gigabit-natverkskort-USB-C?fbclid=IwAR3cwUUneeHwetV2Wf9Af0kFY3SdnuOUb7dO--YaVZ8iGAKfFJYGQpZ74Ak}{\textit{länk}}), och SanDisk Ultra 32 GB / microSDHC / Class 10 / UHS-I / Adapter (\href{https://www.webhallen.com/se/product/276923-SanDisk-Ultra-32-GB-microSDHC-Class-10-UHS-I-Adapter?fbclid=IwAR1CDnflZ2lyeXEtmu_JbxddqwOeaVz4IcyH0Fv1HP7R38Z0Q6yOYFrgsZ0}{\textit{länk}})
        \att Att budgeten sätts till \SI{1200}{kr},
        \att kostnaden belastar dispositionsfonden, samt
        \att detta läggs på beslutsuppföljningen till S2/20 med undertecknad som ansvarig. 
    \end{attsatser}
    
    
    \begin{signatures}{1}
    \signature{Jonathan Benitez}{Teknokrat}
    \end{signatures}

   
    \end{document}
    
