\documentclass[10pt]{article}
    \usepackage[utf8]{inputenc}
    \usepackage[swedish]{babel}
    
    \def\doctype{Handlingar} %ex. Kallelse, Handlingar, Protkoll
    \def\mname{Styrelsemöte} %ex. styrelsemöte, Vårterminsmöte
    \def\mnum{S04/20} %ex S02/16, E1/15, VT/13
    \def\date{2020-02-11} %YYYY-MM-DD
    \def\docauthor{Sophia Carlsson}
    
    \usepackage{../e-mote}
    \usepackage{../../../e-sek}
    
    \begin{document}
    
    \heading{{\doctype} till {\mname} {\mnum}}
    
    \section*{Äskning av pengar för inköp av stolar till Arkivet}
    
    NollU19 tog förra året sönder en av stolarna i arkivet, vilket medfört att det endast finns 5 stycken nu. På NollUs möten är vi oftast 8 personer, och eftersom soffan är liten funkar det inte riktigt att sitta 3 personer där. Förutom detta tar de nuvarande stolarna väldigt mycket plats i ett redan litet rum 	
 	

Jag yrkar 


   \begin{attsatser}
        \att köpa in 6 st klappstolar av modellen GUNDE från IKEA á 69kr (\href{https://www.ikea.com/se/sv/p/gunde-klappstol-svart-00217797/}{\textit{Länk}}), 
        \att budget sätts till \si{540}{kr} (414kr för stolar + (6 mil * 20kr körersättning),
        \att kostnaden belastar dispositionsfonden, samt
        \att detta läggs på beslutsuppföljningen till S05/20 med undertecknad som ansvarig. 
    \end{attsatser}
    
    
    \begin{signatures}{1}
    \textit{\ist}
    \signature{Sophia Carlsson}{Øverphøs}
    \end{signatures}

    \newpage
    \section*{Diskussionspunkt - VTM20}
    Under årets gång sker ett antal sektionsmöten som enligt stadga och reglemente måste genomföras. Ett utavdessa är VTM som enligt stadgan, §4:9, ska hållas. På detta ska följande punker tas upp enligt stadgan:
    \begin{alphlist}
    \item 	val av funktionärer enligt Reglemente,
        \begin{numplist}
            \item Inspektor
            \item Teknikfokusansvarig
        \end{numplist}
    \item 	verksamhetsberättelse för föregående verksamhetsår,
    \item 	bokslut för föregående verksamhetsår,
    \item 	revisorernas berättelse för samma tid,
    \item 	resultatdisposition,
    \item 	frågan om ansvarsfrihet för:
        \begin{numplist}
            \item  	funktionärer,
            \item 	utskotten,
            \item 	styrelsen,
            \item 	revisorerna, samt
            \item 	valberedningen.
        \end{numplist}
    \item 	motioner inlämnade i rätt tid enl. §4:14, samt
    \item 	de punkter som föreskrivs i Reglementet. Vilka är:
    \end{alphlist}
    \begin{alphlist}
        \item beslutsuppföljning,
        \item utskottsrapporter (kort skriftlig redogörelse av utskottets, styrelsens ochvalberedningens verksamhet),
        \item uppföljning av verksamhetsplan (både nuvarande och avgående styrelse), samt
        \item ekonomisk rapport \newline
    \end{alphlist}

    Med detta sagt så är detta även vårt högst beslutande organ och därmed så är det viktigt att vi börjar tidigt med detta. Jag skulle därmed redan nu vilja sätta ut ett datum som passar alla och som alla kan boka upp och börja jobba efter.
    \newline \newline Därför satte sig jag och Theo efter förra styrelsemötet och tittade på när detta skulle passa som bäst och vi kom fram till att 7e Maj verkar optimalt i nuläget. Med detta datumet så skulle det betyda att andra datum av intresse skulle bli:
    \newline Kallelsen som ska ut 11 läsdagar innan skulle få slutdatumet, 3e April
    \newline Föredragningslistan som ska ut 5 läsdagar innan skulle få slutdatumet, 28e April
    \newline Sista motionen (exkluderat sena handlingar) ska skickas in 8 läsdagar innan vilket skulle bli, 8e April
    \newline \newline Anledningen av valda datumet är flera:
    \begin{numplist}
        \item Förra året var VTM 2019-04-09, vilket vi ansåg var tidigt och vi ville därmed flytta det till efter omtenta P.
        \item Detta medförde problemet att kallelsen hamnar långt innan, vilket jag personligen däremot ser mer som en fördel då detta kommer leda till att
        \begin{alphlist}
            \item Sektionen kommer har längre tid på sig att skicka in väl utförda motioner
            \item Ni i styrelsen kommer ha längre tid på er att jobba fram svar, utskottsrapporter samtverksamhetsberättelser
        \end{alphlist}
        \item Första veckan efter omtenta P skulle vi däremot säga är mindre optimal än andra då detta skulle motverka punkt 2.b och det finns även en risk att folk är bortresta däromkring påsken
        \item Vi tror att fler personer skulle närvara på en Tisdag/Torsdag än en Onsdag, detta pga. utomsektionlig fritidsaktivitet. Vi valde däremot Torsdagen över Tisdagen med samma motivering som i punkt 3
    \end{numplist}
    Det jag helst vill få fram med detta är att ni kommer igång med planering och kollar med era funktionärer vad de vill att vi ska förändra med sektionen för att få den att gå framåt samt se till att de blir informerade om mötet. Även så vore det bra om stora projekt som behövs utföras kommer fram samt att ni alla känner er väl införstådda med det som förväntas av er nu inför nästa sektionsmötet redan nu.
    
    \begin{signatures}{1}
    \textit{\isekt}
    \signature{Henrik Ramström}{Talman}
    \end{signatures}

    \end{document}
    
