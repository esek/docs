\documentclass[10pt]{article}
\usepackage[utf8]{inputenc}
\usepackage[swedish]{babel}

\def\mo{Theo Nyman}
\def\ms{Hannes Björk}
\def\ji{Sophia Carlsson}
%\def\jii{}

\def\doctype{Protokoll} %ex. Kallelse, Handlingar, Protkoll
\def\mname{Styrelsemöte} %ex. styrelsemöte, Vårterminsmöte
\def\mnum{S02/20} %ex S02/16, E1/15, VT/13
\def\date{2020-01-28} %YYYY-MM-DD
\def\docauthor{\ms}

\usepackage{../e-mote}
\usepackage{../../../e-sek}

\begin{document}
\showsignfoot

\heading{{\doctype} för {\mname} {\mnum}}

%\naun{}{} %närvarane under
%\nati{} %närvarande till och med
%\nafr{} %närvarande från och med
\section*{Närvarande}
\subsection*{Styrelsen}
\begin{narvarolista}
\nv{Ordförande}{Theo Nyman}{BME18}{}
\nv{Kontaktor}{Hannes Björk}{E17}{}
\nv{Förvaltningschef}{Rasmus Sobel}{BME16}{}
\nv{Cafémästare}{Frida Pilcher}{E18}{} 
\nv{Sexmästare}{Anna Hollsten}{BME19}{}
\nv{Krögare}{Love Sjelvgren}{E18}{}
\nv{Entertainer}{Amir Ghanaatifard}{E19}{}
\nv{SRE-ordförande}{Hanna Bengtsson}{BME18}{}
\nv{ENU-ordförande}{Fredrik Berg}{E17}{}
\nv{Øverphøs}{Sophia Carlsson}{BME17}{}
\end{narvarolista}


\subsection*{Ständigt adjungerande}
\begin{narvarolista}
%\nv{}{}{}{}
%\nv{Skattmästare}{Daniel Bakic}{E15}{\nafr{10}}
%\nv{Vice Krögare}{Klara Indebetou}{BME17}{}
%\nv{Vice Krögare}{Hjalmar Tingberg}{BME16}{}
%\nv{Kårrepresentant}{Ivar Vänglund}{}{\nati{16}}
%\nv{Vice Förvaltningschef}{Rasmus Sobel}{BME16}{}
\nv{Kårrepresentant}{Hanna Jonson}{}{}
%\nv{Valberedningens ordförande}{Axel Voss}{E15}{\nafr{10b}}
%\nv{Fullmäktigeledamot}{Magnus Lundh}{E15}{\nafr{12}}
%\nv{Chefredaktör}{Emil Eriksson}{E18}{}
%\nv{Inspektor}{Monica Almqvist}{}{}
%\nv{Cophøs}{Elina Yrlid}{E18}{}
%\nv{Vice Kontaktor}{Matilda Horn}{BME18}{}
%\nv{Talman}{Henrik Ramström}{E16}{}

\end{narvarolista}

\begin{comment}
\subsection*{Adjungerande}
\begin{narvarolista}
%\nv{post}{namn}{klass}{nati/nafr/tom}
%\nv{Likabehandlingsombud}{Jonna Fahrman}{BME17}{}
%\nv{Likabehandlingsombud}{Hanna Bengtsson}{BME18}{}
%\nv{Projekfunktionär}{Emma Hjörneby}{BME17}{}
%\nv{Macapär}{Filip Larsson}{E17}{}
%\nv{Kodhackare}{Vincent Palmer}{E18}{}
%\nv{Sexmästare Electus}{Anna Hollsten}{BME19}{}
%\nv{Entertainer Electus}{Amir Ghanaatifard}{E19}{}
%\nv{Øverphøs Electus}{Sophia Carlsson}{BME17}{}
%\nv{Cafemästare Electus}{Frida Pilcher}{E18}{}
%\nv{ENU-ordförande Electus}{Fredrik Berg}{E17}{}
%\nv{Krögare Electus}{Love Sjelvgren}{E18}{}
%\nv{SRE-ordförande Electus}{Hanna Bengtsson}{BME18}{}
%\nv{Husstyrelserepresentant}{Joakim Magnusson Fredluend}{BME19}{}
%\nv{Sigillbevarare}{Matilda Horn}{BME18}{\nati{14}}
%\nv{Teknokrat}{Jonathan Benitez}{E17}{}

\end{narvarolista}
\end{comment}

\section*{Protokoll}
\begin{paragrafer}
\p{1}{OFMÖ}{\bes}
Ordförande {\mo} förklarade mötet öppnat kl 12.11

\p{2}{Val av mötesordförande}{\bes}
{\valavmo}

\p{3}{Val av mötessekreterare}{\bes}
{\valavms}

\p{4}{Val av justeringsperson}{\bes}
{\valavj}

\p{5}{Godkännande av tid och sätt}{\bes}
{\tosg}

\p{6}{Adjungeringar}{\bes}
%Adam Belfrage adjungerades.{}
%Hanna Bengtsson adjungerades. \\
%Jonna Fahrman adjungerades.
%Vincent Palmer adjungerades.\\
%Filip Larsson adjungerades. 

%Anna Hollsten adjungerades. 

%Amir Ghanaatifard adjungerades.

%Sophia Carlsson adjungerades.

%Fredrik Berg adjungerades.

%Frida Pilcher adjungerades.

%Rasmus Solel adjungerades.

%Love Sjelvgren adjungerades.

%Hanna Bengtsson adjungerades. 

%Joakim Magnusson Fredluend adjungerades.

%Jonathan Benitez adjungerades.
%Henrik Ramström adjungerades.
\textit{Inga adjungeringar.}


\p{7}{Godkännande av dagordningen}{\bes}

%Davida \ypa lägga till punkten ``Lophtet'' till dagordningen.\\
%Edvard \ypa lägga till punkten ``Ordensband'' til dagordningen.
%Fredrik \ypa att lägga till \S18b ``Teknikfokus utnyttjande av LED-café''.
%Jonathan \ypa ändra punkten §12 från att vara en beslutspunkt till diskussion. \\
%Föredragningslistan godkändes med yrkandet.
%Henrik \ypa lägga till punkten ``Faktura till F'' som §13.
%Jakob Pettersson \ypa tägga till punkten ''Øverphøs informerar'' som \S16.


%Theo \ypa lägga till punkten 'Äskning av pengar för inköp av hemsida på annan server' som §16

%Theo \ypa lägga till punkten 'Inköp av en Raspberry Pi' som §17

%Föredragningslistan godkändes med samtliga yrkande.
%Föredragningslistan godkändes med yrkandet.
Föredragningslistan godkändes.

\p{8}{Föregående mötesprotokoll}{\bes}
\latillprotgodkand{S1/20}
%\textit{\ingaprot}

\p{9}{Fyllnadsval och entledigande av funktionärer}{\bes}
\begin{fyllnadsval} %"Inga fyllnadsval." fylls i automatiskt
%\fval{Moa Rönnlund}{Halvledare}
%\entl{Fanny Månefjord}{Husstyrelserepresentant från och med 30 juni}
%\fval{Klara Wahldén}{Inköps och Lagerchef från och med 1 januari 2020}

\fval{Emil Eriksson}{Kodhackare}
\fval{Mattias Lundström}{Kodhackare}
\fval{William Sjödin}{Kodhackare}
\fval{Jacob Forsell}{Kodhackare}
\fval{Alexander Simko}{Kodhackare}
\fval{Anton Palmén}{Kodhackare}
\fval{Carl Rutholm}{Kodhackare}
\fval{Markus Kvist}{Kodhackare}
\fval{Max Wither}{Kodhackare}
\fval{Måns Rosberg}{Kodhackare}
\fval{Axel Stenström }{Kodhackare}
\fval{Ludvig Lifting}{Kodhackare}
\fval{Johannes Larsson}{Kodhackare}
\fval{Jonathan Benitez}{Kodhackare}
\fval{Andrea Cicovic}{Kodhackare}
\fval{Henrik Ramström}{Kodhackare}
\fval{Vincent Palmer}{Kodhackare}
\fval{Tom Andersson}{Kodhackare}
\fval{Richard Byström}{Kodhackare}
\fval{Elias Gustafsson}{Kodhackare}
\fval{Fabian Sondh}{Kodhackare}
\fval{David Karlsson}{Kodhackare}
\fval{Adam Appelbring}{Kodhackare}
\fval{Agnes Wallén}{Kodhackare}
\fval{Max Johansson}{Kodhackare}
\fval{Jesper Laurell}{Kodhackare}
\fval{Marcus Ascard}{Kodhackare}
\fval{Tor Hammarbäck}{Kodhackare}
\fval{Isa Clementsson}{Kodhackare}
\fval{Jens Elfström}{Kodhackare}
\fval{Lina Tinnerberg}{Kodhackare}
\fval{Amanda Darell}{Näringslivskontakt}

\end{fyllnadsval}

\p{10}{Rapporter}{}
\begin{paragrafer}
\subp{A}{Hur mår alla?}{\info}

Punkten protokollfördes ej.

\subp{B}{Utskottsrapporter}{\info}

Ordförande Theo har fortsatt enligt rutin.

CM har klarat av första veckan och det har fungerat bra. De hade ett litet skiphte i fredags med CM19 och CM20. Eduroam fungerar dåligt i LED och det har gett problem med iZettle men Per Foreby i datordriften har kontaktats på anvisning av PH. CM hade ett möte igår där det bland annat planerades inför alla hjärtans dag och fettisdagen. De tänker köra en Delicato-kampanj veckan innan alla hjärtans dag där man kommer att kunna ge bort delicatobollar anonymt med en kärlekshälsning. Data öppnade sitt café den här veckan och de har nu också börjat sälja små mackor, de beställer nu bröd tillsammans med LED. Ett UF-företag vill börja sälja koffein- och vitamintuggumi i LED. Rödlök picklas numera regelbundet för att ha i salladerna. Picasso har kontaktas angående nya affischer till LED. 

FVU har påbörjat bokföringsarbetet. Nycklar har börjats lämnas ut till funktionärer som ska ha det. Möte har planerats in denna vecka med vice. Förvaltningschefen har varit på möte med pengakollegiet. Den nya soffan i Vega har kommit och är ihopbyggd.

INFU har haft möte mellan kontaktor, vice kontaktor och samtliga delar av utskottet förutom redaktionen.
En handlingsplan för hantering av hemsidan har gjorts. Handlingsplanen revs efter hemsidan fixades och en ny handlingsplan är på väg.
Hemsidans nuvarande status är att EFT hjälpte oss uppdatera servern. Därmed är säkerhetsprotokollen bra, men vissa delar av hemsidan är trasiga.
De har arbetat för att göra instagramkontot mer estetiskt enhetligt.
Licenser har fördelats och måste ses över innan ny faktura kommer i mars.

KM har haft sitt första gille för året och det gick bra. Alla som jobbade och var på gillet verkade ha det kul. De mindre problem som stöttes på löstes rätt snabbt. Restaurangrapporten är förhoppningsvis klar snart men har stött på problem tack vare vissa sorters alkohol.

NollU har haft sitt första möte för läsperioden, lunchmöte med mestadels information och små beslut. På torsdag är det första kvällsmötet då de kommer att diskutera vilka olika phaddrar de vill ha, när info för det ska komma ut och när ansökan ska öppna, samt definiera alla ansvarsområden. 
Den 27/1 var det även första riktiga ØPK-mötet. Det gick bra. F-sektionen vill även i år kandidera om att få gasque på fredagen eller lördagen, det beslutet kommer antagligen att tas på måndag nästa vecka redan.
NollU kommer att vilja ha möten under luncherna med alla utskott inför nollningen.

ENU ordförande har fortsatt att maila med S.B agency gällande deras inlägg på hemsidan samt Axis om invigningen av de nya microvågsugnarna och det kommer nu att ske torsdagen den 6/2 på lunchen. De kommer då hit och delar ut 100 pan-pizzor.
De kommer att ha vårens första möte med hela utskottet 29/1 på lunchen och med det börja kontakta olika företag.
Teknikfokus rullar på och kör flera olika event nu under veckorna som leder fram till mässan.

NöjU hade sin första spelkväll den 23/1 och allt gick jättebra. Fritidsledarna skötte allt som om de varit fritidsledare hela livet.
Victoriastadion är nu bokat varje söndag 15-16 som vanligt för Sporta med E. Amir hade möte med aktivitetsansvarig på F igår för att prata om Sporta med E. Våra idrottsförmän kommer att träffas nästa vecka för att skissa en planering inför denna terminen. 
Amir och UtEDischoansvariga William ska ha möte imorgon med D för att börja med planeringen inför Utedischot.
De kommer även köra KillErgame nästa vecka som börjar på måndagen den 3/2 och avslutas på fredagen därpå. KillErgame kommer introduceras på en spelkväll denna vecka den 30/1.

Sexmesteriet har haft sitt första planeringsmöte inför läsperiod 3, där Kickoff-sittningen för Teknikfokus planerades. Alla verkar ligga i fas då både menyn och drinklistan är satt och även alla dekorationer är beställda. Anna hade också möte med sexmästaren i D6 förra veckan och de har planerat in ett stort planeringsmöte med båda sexmesterien inför banketten den 18/2 för Teknikfokus. I söndags gick de även ut med ett formulär till alla sexiga där de fick fylla i vilka sittningar de ville jobba under LP3, så även schemat kan sättas inför kommande sittningar. 

SRE meddelar att det ska vara en del CEQ-möten i veckan som hänger kvar sen LP1. Årskursrepresentanterna för BME1 och E1 samt Hanna och vice är anmälda till kårens utbildning för utbildningsbevakare och de har bokat in en dag så att de kan lära dem CEQ-systemet. Alla poster verkar börja få rätt bra koll. Första mötet med utskottet på torsdag vilket ska bli kul. Hanna har också varit på SRX-möte samt censurerat pappers-CEQer.


\subp{C}{Ekonomisk rapport}{\info}
 
Ekonomin är bra.

\subp{D}{Kåren informerar}{\info}

Det kommer ske styrelseutbildning och alla berörda har fått mail om det.

Heltidare är på teambilding ochkommer tillbaka onsdag den 29/1 eftermiddag.


\end{paragrafer}

\p{11}{Alkoholkultur på sektionen och inom styrelsen}{\dis}

Theo lyfter hur vi representerar sektionen på evenemang
Vi som styrelse ska hålla oss på en rimlig nivå av berusning.
Speciellt då vi går på evenemang som styrelse, och än mer då vi blir anmodade som styrelse.

Sohpia håller med och tillägger att styrelsen är invalda och har ansvar.

Theo vidhåller dock att fallet är aningen annorlunda vid extern representation.

Sophia menar speciellt på nollningen, då bland annat att man ofta bär frackar som visar väldigt tydligt att man tillhör styrelsen.

Hanna från kåren påminner att vi respresenterar styrelsen i alla sammanhang även som privatperson, då vi rör oss tillsammans.

\p{12}{PH och Cedervall}{\info}

Per-Henrik Rasmussen, ofta förkortat PH, sitter i tryckeriet och är ansvarig för lokaler.

Styrelsen ska som tidigare år presentera oss för PH och ge en mindre gåva.

Typ av gåva diskuterades.

Mats Cedervall är nuvarande husprefekt i E-huset, vilket innebär att han har slutgiltiga ordet om vad som händer i huset.
Mats går tyvärr pension i Maj.
Theo pratade med D-sektionens orförande om en gemensam avskedspresent.

Avskedspresenten diskuterades.

\p{13}{Funktionärsskiphtet}{\dis}

Theo diskuterade upplägg för funktionärsskiphtet, samt dess fördelar och nackdelar.

Sophia anser att en lång sittning utan alkoholservering kommer folk tycka är tråkigt. Istället föreslåg hon lekar utomhus innan och en kort sittning.

Theo påminner om att vi har blivit fler i sektionen och att det redan är 200 personer inbjudn i evenemanget.

Amir tycker att lekar innan sittningen låter som en bra idé.

Berg trycker på att vi måste vara tydliga med deltagare om vilka regler som gäller.

Hanna från kåren föreslår att annan sektion håller i sittningen.

Theo ska fråga D-sektionen om de har möjlighet att hålla i en sittning, annars är lekar innan och en kort sittning den bästa idén.

\p{14}{Nästa styrelsemöte}{\bes}

\Mba nästa styrelsemöte ska äga rum 2020-02-04 12.10 i E:1123.

\p{15}{Beslutsuppföljning}{\bes}

Hannes \ypa skjuta upp ''Inköp av Raspberry Pi'' till nästa styrelsemöte S03/20.

\Mbaby 

Hannes \ypa skjuta upp ''Inköp av squarespace'' till nästa styrelsemöte S03/20.

\Mbaby 

\p{16}{Övrigt}{\dis}

Amir informerar läget med Victoriastadion. Victoriastadion kommer bokas periodsvis och därmed användas till mindre grad. Detta kommer leda till en större variation för Sporta med E och mindre onödiga utgifter.

Anna diskuterar hur vi ska hantera inköp av utskottströjor.
Utskottsordförande kom överens om att ha en gemensam deadline 5/2.

Hannes ber att alla utskottsordförande med evenemang som kräver ljus- eller ljudutrustning ska skriva ner det i angivet dokument.

\p{17}{Sammanfattning av mötet}{\info}

Valt in funktionärer

Present till Per-Hentik Rasmussen och Mats Cedervall har diskuterats.

Upplägg för funktionärsskiphtet har diskuterats, D-sektionen ska tillfrågas om sittningen.

Alkoholkulturen inom styrelsen har diskuterats.

Övriga mindre punkter togs upp.

\p{18}{OFMA}{\bes}
{\mo} förklarade mötet avslutat kl. 13.14
\end{paragrafer}

%\newpage
\hidesignfoot
\begin{signatures}{3}
\signature{\mo}{Mötesordförande}
\signature{\ms}{Mötessekreterare}
\signature{\ji}{Justerare}
\end{signatures}
\end{document}
