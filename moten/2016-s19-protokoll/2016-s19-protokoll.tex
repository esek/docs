\documentclass[10pt]{article}
\usepackage[utf8]{inputenc}
\usepackage[swedish]{babel}

\def\mo{Fredrik Peterson}
\def\ms{Erik Månsson}
\def\ji{Johannes Koch}

\def\doctype{Protokoll} %ex. Kallelse, Handlingar, Protkoll
\def\mname{styrelsemöte} %ex. styrelsemöte, Vårterminsmöte
\def\mnum{S19/16} %ex S02/16, E1/15, VT/13
\def\date{2016-09-22} %YYYY-MM-DD
\def\docauthor{\ms}

\usepackage{../e-mote}
\usepackage{../../e-sek}

\begin{document}
\showsignfoot

\heading{{\doctype} för {\mname} {\mnum}}

%\naun{}{} %närvarane under
%\nati{} %närvarande till och med
%\nafr{} %närvarande från och med
\section*{Närvarande}
\subsection*{Styrelsen}
\begin{narvarolista}
\nv{Ordförande}{Fredrik Peterson}{E14}{}
\nv{Kontaktor}{Erik Månsson}{E14}{}
\nv{Förvaltningschef}{Anders Nilsson}{E13}{}
\nv{Cafémästare}{Stephanie Mirsky}{E13}{\nafr{10A}}
\nv{SRE-ordförande}{Johan Persson}{E13}{}
\nv{ENU-ordförande}{Johannes Koch}{E13}{}
\nv{Sexmästare}{Martin Gemborn Nilsson}{E14}{}
\nv{Krögare}{Malin Lindström}{BME14}{}
\nv{Entertainer}{Dalia Khairallah}{E15}{}
\end{narvarolista}

\subsection*{Ständigt adjungerande}
\begin{narvarolista}
\nv{Kårordförande}{Linus Hammarlund}{}{}
\nv{Kårrepresentant}{Jacob Karlsson}{}{}
\end{narvarolista}

\section*{Protokoll}
\begin{paragrafer}
\p{1}{OFSÖ}{\bes}
Ordförande {\mo} förklarade mötet öppnat 12:12.

\p{2}{Val av mötesordförande}{\bes}
\valavmo

\p{3}{Val av mötessekreterare}{\bes}
\valavms

\p{4}{Tid och sätt}{\bes}
\tosg

\p{5}{Adjungeringar}{\bes}
\ingaadj

\p{6}{Val av justeringsperson}{\bes}
\valavj

\p{7}{Föredragningslistan}{\bes}
Föredragningslistan godkändes.

\p{8}{Föregående mötesprotokoll}{\bes}
\latillprot{S18/16}

\p{9}{Fyllnadsval/Entledigande av funktionärer}{\bes}
\begin{fyllnadsval} %"Inga fyllnadsval." fylls i automatiskt
\fval{Fredrik Peterson}{Valnämndsrepresentant}
\end{fyllnadsval}

\p{10}{Rapporter}{}
\begin{paragrafer}
\subp{A}{Check in}{\info}
Erik mår bra, han är taggad på den kommande helgen.

Steph mår bra. LED går bra, det är fullt med dioder!

Johan sa att han säger exakt samma som Erik.

Johannes går det bra för. Han och Johan har fixat de nya mikrovågsugnarna som fantastiskt nog är röststyrda.

NöjU går det bra för, de körde Phadderolympiaden igår vilket det var många som gick på.

Malin sa att det går bra för KM, de har fått sitt nya stekbord.

Martin har mycket att göra inför qasquen på lördag.

Anders mår bra!

Fredrk mår bra, han är för ovanlighetens skull inte sjuk!

\subp{B}{Kåren informerar}{\info}
Linus sa att Kåren har smygöppnat nomineringen till heltidsposter.

Kåren har haft FM och bl.a. beslutat att köpa in mattor till ARKAD som ska hålla i flera år istället för att använda engångsmattor. De har också reviderat budgeten för Kåren samt gjort lite förändringar i hur fria föreningar inom Kåren fungerar.

Kåren har stadfäst E-sektionens nya stadgar.

\subp{C}{Ekonomi}{\info}
Anders berättade om ekonomin, den mår bra och han börjar komma ikapp på bokföringen. Halvårsbokslutet är på gång.

\end{paragrafer}

\p{11}{Inköp av dammsugare}{\bes}
Fredrik har gjort sitt jobb i veckan och kollat upp dehär med dammsugare. Han presenterade några olika förslag.

Mötet kom fram till att köpa in en Kärcher T 7/1 Classic.

\textbf{\Mba köpa in en dammsugare till en kostnad av maximalt 1500kr vilken belastar dispositionsfonden, samt att lägga beslutet på beslutsuppföljningen till styrelsemöte 21 med Fredrik Peterson som ansvarig.}

\p{12}{Nästa styrelsemöte}{\bes}
\Mba nästa styrelsemöte ska äga rum 2016-09-29 12:10 i E:1426.

\p{13}{Beslutsuppföljning}{\bes}
Uppföljningen av \emph{Inköp av stekbord} senarelades två veckor.

Inga andra beslut följdes upp.

\p{14}{Övrigt}{\dis}
Dalia frågade om paintballen kan betalas genom Sektionen, vilket Anders sa var lätt att ordna.

Fredrk påminde om att rösta på vilken städdag vi ska välja.

\p{15}{OFSA}{\bes}
{\mo} förklarade mötet avslutat 12:30.

\end{paragrafer}

%\newpage
\hidesignfoot
\begin{signatures}{3}
\signature{\mo}{Mötesordförande}
\signature{\ms}{Mötessekreterare}
\signature{\ji}{Justerare}
\end{signatures}
\end{document}
