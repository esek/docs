\documentclass[10pt]{article}
\usepackage[utf8]{inputenc}
\usepackage[swedish]{babel}

\def\mo{Erik Månsson}
\def\ms{Johan Karlberg}
\def\ji{Edvard Carlsson}
%\def\jii{}

\def\doctype{Protokoll} %ex. Kallelse, Handlingar, Protkoll
\def\mname{styrelsemöte} %ex. styrelsemöte, Vårterminsmöte
\def\mnum{S18/17} %ex S02/16, E1/15, VT/13
\def\date{2017-09-07} %YYYY-MM-DD
\def\docauthor{\ms}

\usepackage{../e-mote}
\usepackage{../../../e-sek}

\begin{document}
\showsignfoot

\heading{{\doctype} för {\mname} {\mnum}}

%\naun{}{} %närvarane under
%\nati{} %närvarande till och med
%\nafr{} %närvarande från och med
\section*{Närvarande}
\subsection*{Styrelsen}
\begin{narvarolista}
\nv{Ordförande}{Erik Månsson}{E14}{}
\nv{Kontaktor}{Johan Karlberg}{E14}{}
\nv{Förvaltningschef}{Sophia Grimmeiss Grahm}{BME14}{}
\nv{Cafémästare}{Daniel Bakic}{E15}{}
%\nv{Øverphøs}{Niklas Gustafson}{E15}{}
\nv{SRE-ordförande}{Edvard Carlsson}{E16}{}
\nv{ENU-ordförande}{Josefine Sandström}{E14}{}
%\nv{Sexmästare}{Linnea Sjödahl}{BME15}{}
\nv{Krögare}{Markus Rahne}{BME14}{\nafr{5}}
%\nv{Entertainer}{Albin Nyström Eklund}{BME16}{}
\end{narvarolista}

\subsection*{Ständigt adjungerande}
\begin{narvarolista}
%\nv{Kårordförande}{Linus Hammarlund}{}{}
%\nv{Kårrepresentant}{Anders Nilsson}{}{}
%\nv{Kårrepresentant}{Caroline Svensson}{}{}
\nv{Kårrepresentant}{Agnes Sörliden}{}{}
%\nv{Valberedningens ordförande}{Elin Magnusson}{}{}
%\nv{Skattmästare}{Olle Oswald}{}{}
%\nv{Kårrepresentant}{Daniel Damberg}{}{}
%\nv{Kårrepresentant}{John Alvén}{}{}
%\nv{Talman}{Fredrik Peterson}{E14}{}
%\nv{Elektras Ordförande}{Elisabeth Pongratz}{}{}
%\nv{Inspektor}{Monica Almqvist}{}{}
\end{narvarolista}

\begin{comment}
\subsection*{Adjungerande}
\begin{narvarolista}
%\nv{Post}{Namn}{Klass}{}
\end{narvarolista}
\end{comment}

\section*{Protokoll}
\begin{paragrafer}
\p{1}{OFMÖ}{\bes}
Ordförande {\mo} förklarade mötet öppnat 12:10.

\p{2}{Val av mötesordförande}{\bes}
{\valavmo}

\p{3}{Val av mötessekreterare}{\bes}
{\valavms}

\p{4}{Val av justeringsperson}{\bes}
{\valavj}

\p{5}{Godkännande av tid och sätt}{\bes}
{\tosg}

\p{6}{Adjungeringar}{\bes}
{\ingaadj}

%Förnamn Efternamn adjungerades

\p{7}{Godkännande av dagordningen}{\bes}
%Dagordningen godkändes.
Daniel \ypa lägga till \S14 ``Sexa''.

Johan \ypa lägga till \S15 ``Utomlundare''.

Markus \ypa lägga till \S14 ``Skaffa riktiga knivar''.
%Föredragningslistan godkändes med yrkandet.

Föredragningslistan godkändes med samtliga yrkanden.

\p{8}{Föregående mötesprotokoll}{\bes}
\latillprot{S17/17}
%\ingaprot
\p{9}{Fyllnadsval och entledigande av funktionärer}{\bes}
\begin{fyllnadsval} %"Inga fyllnadsval." fylls i automatiskt
\entl{Niklas Karlsson}{Hustomte}
\entl{Malin Lindström}{Hustomte}
\end{fyllnadsval}
\p{10}{Rapporter}{}
\begin{paragrafer}
\subp{A}{Hur mår alla?}{\info}
Punkten protokollfördes ej.
\subp{B}{Utskottsrapporter}{\info}
Erik berättade att problemen rörande FlyING inte är utrett ännu, det är möte med kårstyrelsen på måndag. V-sektionen har samma ståndpunkt som oss. A-sektionen diskuterar fortfarande frågan.

Tillsynsmyndigheten var i edekvata, de hade synpunkter på brandvägarna.

Erik ska kämpa för att vi ska få tillbaka vår möteslokal E:1426.

Sophia berättade att FVU går bra, märkespickniken gick bättre än förväntat.

SRE har haft sitt första möte, de flesta kollegiena är i uppstart så de har inte dragit igång ännu. Deras andra pluggkväll gick bra, lite mer jobb då de lagade lite mer krävande mat. CEQ-möten bokas upp.

KMs gillen har gått bra. Draggningspuben blev lite rörig, det kom mer folk än vad Markus hade förväntat. Tillstånd kom även på besök under gillet men de var nöjda.

InfU har inte jobbat så mycket, deltagit i de nollningsaktiviteter som utskotten är med på.

ENU håller på att fixa allt till lunchföreläsningarna. De har haft möte med en från Academic Work där de snackade om kommande event. Första utskottsmötet hålls nästa vecka, där kommer de snacka igenom mer om framtida event. Teknikfokus har öppnat intresseanmälan för projektgruppen.

Det är mycket att göra i LED.
\subp{C}{Ekonomisk rapport}{\info}
Sophia rapporterade att ekonomin mår bra.
\subp{D}{Kåren informerar}{\info}
Agnes berättade att kåren har lyckats välja sin sista heltidare.
\end{paragrafer}

\p{11}{Datum för sektionsmöten}{\bes}
Erik föreslog att HT17 skall äga rum 2017-11-14 (tis. l.v. 3 l.p. 2) så att motionsstoppet hamnar efter tentorna, och VM17 exakt en vecka senare.

\Mba HT17 är 2017-11-14 och VM17 är 2017-11-21.

\p{12}{Eventuellt samarbete mellan E-shop och LED-café}{\dis}
Diskussion om ifall man hade kunnat fixa någon form av samarbete mellan E-shop och LED-café så att man t.ex. kan sälja tygmärken i LED-café men att Ekiperingsexperterna drar i det på något vis.

Mötet ansåg att detta var en bra idé. Sophia ska snacka med Ekiperingsexperterna.

\p{13}{Hälsa på Ulla}{\dis}
Punkten protokollfördes ej.

\p{14}{Sexa}{\dis}
Mötet diskuterade hur man sköter en eftersexa.

\p{15}{Utomlundare}{\dis}
Styrelsen ska uppdatera hur många sängplatser de har.

\p{16}{Skaffa riktiga knivar}{\dis}
Markus och Anton och Anders Nilsson har diskuterat möjligheten att köpa in ett bra set av knivar. Knivarna skulle vara inlåsta.

Ett annat förslag som kom upp är att köpa in lite billigare knivar men att ta hand om de bättre, fixa ställ och skicka iväg knivarna varje år.

Markus ska titta vidare på detta till ett framtida möte.
\p{14}{Nästa styrelsemöte}{\bes}
{\Mba}nästa styrelsemöte ska äga rum 2017-09-14 12:10 i E:1124.

\p{15}{Beslutsuppföljning}{\bes}
{\Ibfu}

\p{16}{Övrigt}{\dis}
Daniel berättade att hälsoinspektionen hade någon anmärkning på LED annars gick det bra.

\p{17}{OFMA}{\bes}
{\mo} förklarade mötet avslutat 12:57.

\end{paragrafer}

%\newpage
\hidesignfoot
\begin{signatures}{3}
\signature{\mo}{Mötesordförande}
\signature{\ms}{Mötessekreterare}
\signature{\ji}{Justerare}
\end{signatures}
\end{document}
