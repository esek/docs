\documentclass[10pt]{article}
\usepackage[utf8]{inputenc}
\usepackage[swedish]{babel}

\def\mo{Erik Månsson}
\def\ms{Johan Karlberg}
\def\ji{Niklas Gustafson}
%\def\jii{}

\def\doctype{Protokoll} %ex. Kallelse, Handlingar, Protkoll
\def\mname{styrelsemöte} %ex. styrelsemöte, Vårterminsmöte
\def\mnum{S24/17} %ex S02/16, E1/15, VT/13
\def\date{2017-10-18} %YYYY-MM-DD
\def\docauthor{\ms}

\usepackage{../e-mote}
\usepackage{../../../e-sek}

\begin{document}
\showsignfoot

\heading{{\doctype} för {\mname} {\mnum}}

%\naun{}{} %närvarane under
%\nati{} %närvarande till och med
%\nafr{} %närvarande från och med
\section*{Närvarande}
\subsection*{Styrelsen}
\begin{narvarolista}
\nv{Ordförande}{Erik Månsson}{E14}{}
\nv{Kontaktor}{Johan Karlberg}{E14}{}
\nv{Förvaltningschef}{Sophia Grimmeiss Grahm}{BME14}{}
%\nv{Cafémästare}{Daniel Bakic}{E15}{}
\nv{Øverphøs}{Niklas Gustafson}{E15}{}
\nv{SRE-ordförande}{Edvard Carlsson}{E16}{}
\nv{ENU-ordförande}{Josefine Sandström}{E14}{}
%\nv{Sexmästare}{Linnea Sjödahl}{BME15}{}
\nv{Krögare}{Markus Rahne}{BME14}{\nafr{9}}
%\nv{Entertainer}{Albin Nyström Eklund}{BME16}{}
\end{narvarolista}

\subsection*{Ständigt adjungerande}
\begin{narvarolista}
%\nv{Kårordförande}{Linus Hammarlund}{}{}
%\nv{Kårrepresentant}{Anders Nilsson}{}{}
\nv{Kårrepresentant}{Caroline Svensson}{}{}
\nv{Kårrepresentant}{Agnes Sörliden}{}{}
%\nv{Valberedningens ordförande}{Elin Magnusson}{}{}
%\nv{Skattmästare}{Olle Oswald}{}{}
%\nv{Kårrepresentant}{Daniel Damberg}{}{}
%\nv{Kårrepresentant}{John Alvén}{}{}
\nv{Talman}{Fredrik Peterson}{E14}{}
%\nv{Elektras Ordförande}{Elisabeth Pongratz}{}{}
%\nv{Inspektor}{Monica Almqvist}{}{}
\end{narvarolista}

\begin{comment}
\subsection*{Adjungerande}
\begin{narvarolista}
%\nv{Post}{Namn}{Klass}{}
\end{narvarolista}
\end{comment}

\section*{Protokoll}
\begin{paragrafer}
\p{1}{OFMÖ}{\bes}
Ordförande {\mo} förklarade mötet öppnat 12:14.

\p{2}{Val av mötesordförande}{\bes}
{\valavmo}

\p{3}{Val av mötessekreterare}{\bes}
{\valavms}

\p{4}{Val av justeringsperson}{\bes}
{\valavj}

\p{5}{Godkännande av tid och sätt}{\bes}
{\tosg}

\p{6}{Adjungeringar}{\bes}
{\ingaadj}
%Förnamn Efternamn adjungerades

\p{7}{Godkännande av dagordningen}{\bes}
Dagordningen godkändes.
%Fredrik \ypa att lägga till \S18b ``Teknikfokus utnyttjande av LED-café''.
%Föredragningslistan godkändes med yrkandet.
%Föredragningslistan godkändes med samtliga yrkanden.

\p{8}{Föregående mötesprotokoll}{\bes}
\latillprot{S23/17}
%\ingaprot

\p{9}{Fyllnadsval och entledigande av funktionärer}{\bes}
\begin{fyllnadsval} %"Inga fyllnadsval." fylls i automatiskt
\fval{Anes Mehmedagic}{Diod}
\fval{Malin Heyden}{Diod}
\fval{Tove Börjesson}{Stridsrop}
%\entl{Namn}{Post}
\end{fyllnadsval}

\p{10}{Rapporter}{}
\begin{paragrafer}
\subp{A}{Hur mår alla?}{\info}
Punkten protokollfördes ej.
\subp{B}{Utskottsrapporter}{\info}
Gillena går starkt framåt, förvånansvärt bra. Liten paus nu med inläsningsvecka sedan är det dags för ET-gille. Tapptornet är på plats igen, dock utan något inkopplat.

ENU är ganska lugnt nu lunch med ingenjör rullar på och Josefine har skickat en faktura.

Arbetet fortsätter som vanligt i InfU. Axel jobbar på en skärm till edekvata som ska visa vilka klägg som är färdiga.

Studierådet blir av med sin årskurs BME-3 ansvarig efter denna läsperiod, förhoppningsvis är ny ansvarig redan hittad.

Sophia och Olle har bokfört idag på morgonen.
\subp{C}{Ekonomisk rapport}{\info}
Sophia meddelade att ekonomin mår bra.
\subp{D}{Kåren informerar}{\info}
Kåren har öppnat funktionärsvalet, bland annat går det att söka 4 heltidarposter.

Pedellen kommer nu att svara lite långsammare då han är sjukskriven, han trillade ner från ett träd.

Kåren hälsar lycka till med tentorna.
\end{paragrafer}

\p{11}{Datum för funktack}{\dis}
Den 2:a december föreslås som datum.
\p{12}{HT-möte}{\dis}
Niklas upplyste mötet om problem som kan uppstå när vi väljer Øverphøs på Valmötet. Just nu kan inte de som vill söka Co-phøsare söka andra poster om de vill.

Mötet diskuterade problemet.

Mötet diskuterade övervakningspolicyn.

Mötet anser att sektionen inte ska återuppta övervakningen. Med bland annat argumenten att våra gäster på gillen är inte gäster som kräver övervakningskameror och de som inte vet att våra kamerahöljen är tomma blir bortskrämda av det.

Beslutsuppföljningen i övrigt ser bra ut. Det som skall dubbelkollas är inköp av funktionärskläder.

Alla i styrelsen skall se över dokument inför mötet.

Mötet diskuterade matlagningen till HT och VM.
\p{13}{Nästa styrelsemöte}{\bes}
{\Mba} nästa styrelsemöte ska äga rum 2017-11-02 12:10 i E:1124.

\p{14}{Beslutsuppföljning}{\bes}
{\Ibfu}

\p{15}{Övrigt}{\dis}
Mötet diskuterade kvällsmöten.
\p{16}{OFMA}{\bes}
{\mo} förklarade mötet avslutat 13:03.

\end{paragrafer}

%\newpage
\hidesignfoot
\begin{signatures}{3}
\signature{\mo}{Mötesordförande}
\signature{\ms}{Mötessekreterare}
\signature{\ji}{Justerare}
\end{signatures}
\end{document}
