\documentclass[10pt]{article}
\usepackage[utf8]{inputenc}
\usepackage[swedish]{babel}

\def\mo{Erik Månsson}
\def\ms{Johan Karlberg}
\def\ji{Niklas Gustafson}
%\def\jii{}

\def\doctype{Protokoll} %ex. Kallelse, Handlingar, Protkoll
\def\mname{styrelsemöte} %ex. styrelsemöte, Vårterminsmöte
\def\mnum{S08/17} %ex S02/16, E1/15, VT/13
\def\date{2017-03-23} %YYYY-MM-DD
\def\docauthor{\ms}

\usepackage{../e-mote}
\usepackage{../../../e-sek}

\begin{document}
\showsignfoot

\heading{{\doctype} för {\mname} {\mnum}}

%\naun{}{} %närvarane under
%\nati{} %närvarande till och med
%\nafr{} %närvarande från och med
\section*{Närvarande}
\subsection*{Styrelsen}
\begin{narvarolista}
\nv{Ordförande}{Erik Månsson}{E14}{}
\nv{Kontaktor}{Johan Karlberg}{E14}{}
\nv{Förvaltningschef}{Sophia Grimmeiss Grahm}{BME14}{}
\nv{Cafémästare}{Daniel Bakic}{E15}{}
\nv{Øverphøs}{Niklas Gustafson}{E15}{}
\nv{SRE-ordförande}{Pontus Landgren}{E14}{}
\nv{ENU-ordförande}{Josefine Sandström}{E14}{}
\nv{Sexmästare}{Linnea Sjödahl}{BME15}{}
\nv{Krögare}{Markus Rahne}{BME14}{\nafr{10}}
\nv{Entertainer}{Albin Nyström Eklund}{BME16}{}
\end{narvarolista}

\subsection*{Ständigt adjungerande}
\begin{narvarolista}
%\nv{Kårordförande}{Linus Hammarlund}{}{}
\nv{Kårrepresentant}{Jacob Karlsson}{}{}
%\nv{Aktivietssamordnare}{Lovisa Majtorp}{}{}
\nv{Kårrepresentant}{My Reimer}{}{}
\nv{Valberedningens ordförande}{Christian Benson}{\nafr{9}}{}
%\nv{Skattmästare}{Olle Oswald}{}{}
%\nv{Kårrepresentant}{Daniel Damberg}{}{}
%\nv{Kårrepresentant}{John Alvén}{}{}
%\nv{Talman}{Fredrik Peterson}{E14}{}
%\nv{Elektras Ordförande}{Elisabeth Pongratz}{}{}
%\nv{Inspektor}{Monica Almqvist}{}{}
\end{narvarolista}

\begin{comment}
\subsection*{Adjungerande}
\begin{narvarolista}
%\nv{Post}{Namn}{Klass}{}
\end{narvarolista}
\end{comment}

\section*{Protokoll}
\begin{paragrafer}
\p{1}{OFMÖ}{\bes}
Ordförande {\mo} förklarade mötet öppnat 12:10.

\p{2}{Val av mötesordförande}{\bes}
{\valavmo}

\p{3}{Val av mötessekreterare}{\bes}
{\valavms}

\p{4}{Val av justeringsperson}{\bes}
{\valavj}

\p{5}{Godkännande av tid och sätt}{\bes}
{\tosg}

\p{6}{Adjungeringar}{\bes}
{\ingaadj}

%Förnamn Efternamn adjungerades

\p{7}{Godkännande av dagordningen}{\bes}
%Dagordningen godkändes.
Josefine \ypa lägga till \S14.1 ``Spons från Buttericks''.\\
Albin \ypa lägga till \S14.2 ``Kostnad för biljard ''.\\
Erik \ypa lägga till \S14.3 ``Mentorprogram''.\\
%Föredragningslistan godkändes med yrkandet.
Föredragningslistan godkändes med samtliga yrkanden.
\p{8}{Föregående mötesprotokoll}{\bes}
\latillprot{S07/17}
%\ingaprot

\p{9}{Fyllnadsval och entledigande av funktionärer}{\bes}
\begin{fyllnadsval} %"Inga fyllnadsval." fylls i automatiskt
\fval{Matilda Dahlström}{Diod}
\fval{Lina Samnegård}{Diod}
\fval{Fanny Månefjord}{Diod}
\end{fyllnadsval}

\p{10}{Rapporter}{}
\begin{paragrafer}
\subp{A}{Hur mår alla?}{\info}
Punkten protokollfördes ej.
\subp{B}{Utskottsrapporter}{\info}
FVU har inte gjort särskilt mycket i och med tentaveckan.

Inom InfU händer det mycket bland annat så är nollehemsidan är snart uppe, lösningar till mediemojt tittas på och alumniansvariga medverkar på en sittning.

KM rullar på som vanligt, fick en välförtjänt paus i och med inläsningsveckan och drog igång igen med ett ET-gille, där mat caterades till Kavajnatten. Tyvärr lågt deltagande då folk var så oerhört utspridda vilket även syntes i resultatet även om mycket mat såldes. Fortsätter starkt kommande fredag med KMs hittills mest ambitiösa projekt, ``Hard Rock gille'', med pubquiz och liveband. Kontinuerlig planering mellan NöjU och KM fortsätter inför ``Agent 00E''.

Niklas meddelade att alla phaddrar är valda, NollU har mycket att göra men det går bra. Mantlarna är ivägskickade till brodyr, men i helgen är det myshelg.

ENU anordnade lunchföreläsning i måndags. De har även marknadsfört en kvällsföredrag med Svep som inte fått så mycket respons, mejlat Svep för att kolla om de fortfarande vill hålla i det. Fått mejladresser av Törmänen som vi ska mejla.

Albin meddelade att ``Agent 00E''s evenemang har kommit upp och börjar bli färdigt. Påskäggsjakt är ändrad till 6:e april och genomförs av fritidsledare.

Daniel är glad för att Ulla är tillbaka. Cm har fått in ganska många dioder till den nya läsperioden. Daniel bjöd på kaffe under tentaperioden folk uppskattade det.

E6 har haft möte om FpT-sittningen nästa vecka (onsdagen 29 mars) och det känns som allt är under kontroll. De har också alumnisittning nästa lördag (1 april) tillsammans med W6 och K6.
Vi hade Kavajnatten i lördags (18 mars) och det var väldigt kul och det verkade som att alla Sexiga också var nöjda.

SRE jobbar vidare med CEQ-möten och utvärderingar av kurser. Under tentaveckan låg arbetet i princip nere och nu ska vi börja jobba på att fokusera för att få en intrenationella mer i sektionen.

Erik har jobbat med mentorprogrammet.
\subp{C}{Ekonomisk rapport}{\info}
Sophia meddelade att det inte har hänt så mycket.
\subp{D}{Kåren informerar}{\info}
My meddelade om vad som händer på campus; nästa vecka på onsdag är det ``Startup Fair'', det är som ett mindre Arkad men för startups. Temasläpp för kårens nollning är den 5 april på lunchen. Det är val i FM den 2:a april. Arkad söker koordinatorer, ansökan öppnar den 26 mars.
Jakob meddelade att kårstyrelsen har haft mötet, de har kollat på budgeten.
\end{paragrafer}
\p{11}{Städvecka}{\info}
Erik påminnde om att det är vi som städar våra lokaler.
\p{12}{VT/17}{\dis}
Erik \ypa ha mötet 2 maj.\\
\Mbaby\\
Mötet äger rum den 2 maj med start 17:15 i E:B.\\
Handlingarna kommer ut senast den 24 april.\\
Erik \ypa lägga motionsstoppet den 11 april.\\
\Mbaby\\
Erik gick igenom lite vad vi ska förbereda inför mötet.
\p{13}{Riktlinje: Återbetalningsskyldighet}{\dis}
Till nästa möte ska ett förslag presenteras och förhoppningsvis klubbas.
\p{14.1}{Buttericks}{\dis}
Josefine berättade om hur Buttericks är intresserade av något slags samarbeter. 15\% rabatt mot att deras logga finns med på hemsidan.\\
Denna rabatt skulle gälla för alla sektionsmedlemar.\\
Mötet ser detta som något intressant.
\p{14.2}{Biljard}{\dis}
Albin har känslan av att det är för mycket att ta 150kr betalt för access till biljard. Han föreslår att sänka kostnaden till 20kr och köra med pant mot saker på t.ex. gillen.\\
Mötet diskuterade ämnet.\\
Mötet anser att iallafall pant under gillen är en bra idé.
\p{14.3}{Mentorskap}{\info}
Eriks idé är att inför att mentorsprogram på nollningen, som inte är parallellt med phøset. Erik har lagt upp ett dokument i driven. Det går i kort ut på att folk i ÅK 3-5 ska vara mentor för de nya som kommer hit. En mentor ska kunna svara på frågor om hur studieteknik och allmänna frågor om hur det går till att plugga på LTH. Ett mentorsskap är 3 träffar, där vi står för fika.\\
Åsikter ska lämnas till Erik innan söndag.
\p{15}{Nästa styrelsemöte}{\bes}
{\Mba}nästa styrelsemöte ska äga rum 2017-03-30 12:10 i E:1426.

\p{15}{Beslutsuppföljning}{\bes}
{\Ibfu}

\p{16}{Övrigt}{\dis}
Önskemål om företag till Arkad skickas till Erik.

Christian ville att VB och styrelsen ska skicka in motioner om valprocessen på HT.\\ Christian ska ta fram ett förslag och lämna det till styrelsen. Christian vill i allmänt få åsikter om hur valen går till.
\p{17}{OFMA}{\bes}
{\mo} förklarade mötet avslutat 13:02.

\end{paragrafer}

%\newpage
\hidesignfoot
\begin{signatures}{3}
\signature{\mo}{Mötesordförande}
\signature{\ms}{Mötessekreterare}
\signature{\ji}{Justerare}
\end{signatures}
\end{document}
