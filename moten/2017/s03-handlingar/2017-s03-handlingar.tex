\documentclass[10pt]{article}
\usepackage[utf8]{inputenc}
\usepackage[swedish]{babel}

\def\doctype{Handlingar} %ex. Kallelse, Handlingar, Protkoll
\def\mname{styrelsemöte} %ex. styrelsemöte, Vårterminsmöte
\def\mnum{S03/17} %ex S02/16, E1/15, VT/13
\def\date{2017-02-02} %YYYY-MM-DD
\def\docauthor{Erik Månsson}

\usepackage{../e-mote}
\usepackage{../../../e-sek}

\begin{document}

\heading{{\doctype} till {\mname} {\mnum}}

\section*{Information om punkterna}

\begin{paragrafer}

\p{10}{Rapporter}{}
\begin{paragrafer}
\subp{A}{Hur mår alla?}{\info}
Punkten protokollförs ej.

\subp{B}{Utskottsrapporter}{\info}
På denna punkten ger alla utskottschefer en kort rapport om hur det går för sina utskott och vad de gör. Utskottschefer, glöm inte bort att skriva rapporten i driven (gör ett nytt dokument under Utskottsrapporter/S02)!

\end{paragrafer}

\p{11}{Examensbankett}{\bes}
E12 och BME12 är intresserade av att hålla i en examensbankett. Förra året låg banketten helt utanför sektionen, och resulterade tyvärr i att den ansvariga för banketten blev personligen skyldig att betala för en oförutsedd kostnad. Styrelsen 2016 valde att låta sektionen täcka kostnaden genom en äskning, och tyckte samtidigt att banketten bör ligga på sektionen i framtiden för att inte få detta problem igen.

Mitt förslag är att tillsätta en ansvarig för banketten och lägga beslutet på beslutsuppföljningen för att enkelt kunna följa upp resultatet i efterhand.

\rnamnpost{Erik Månsson}{Ordförande}

\p{12}{Jubileum}{\bes}
Förra mötet beslutade styrelsen att Anders Nilsson ska vara ansvarig för årets Jubileum. Jag kom på i efterhand att det hade varit bra att det står på beslutsuppföljningen, så jag tänkte yrka på att lägga till det där.

\rnamnpost{Erik Månsson}{Ordförande}

\p{13}{Serveringsansvariga}{\dis}
\tupo{Förvaltningschef Sophia Grimmeiss Grahm}

\p{14}{Delning av testamente}{\dis}
\tupo{Förvaltningschef Sophia Grimmeiss Grahm}

\p{15}{Attestering}{\dis}
I dagsläget är det möjligt för styrelsen att attestera (godkänna) sina egna inköp vilket bör åtgärdas genom att ändra utseendet på utläggsräkningarna och kvittoförstärkningarna.

\rnamnpost{Sophia Grimmeiss Grahm}{Förvaltningschef}

I stadgan under \S9:2 Skyldigheter står det:
\begin{addmargin}[8mm]{0mm}
Det åligger sektionens Utskott
\begin{attlist}
\item följa gällande Stadgar och Reglemente, samt
\item verkställa av Styrelsen eller Sektionsmöte fattade beslut.
\end{attlist}
Det åligger utskottsordförande
\begin{attlist}
\item leda och fördela arbetet inom Utskottet,
\item budgetera, redovisa och följa upp Utskottets verksamhet, samt
\item senast två (2) läsveckor efter verksamhetsårets slut till Styrelsen
    inlämna verksamhetsberättelse.
\end{attlist}
\end{addmargin}

Vidare står det i reglementet för varje utskottschef att denne är övergripande ansvarig för sitt utskotts verksamhet.

Det är alltså utskottschefen för ett utskott som ansvarar för alla inköp och att budgeten hålls. Detta gäller oavsett vem i utskottet som utfört själva köpet. I dagsläget attesterar (godkänner) utskottscheferna själva alla inköp som görs, inklusive inköp gjorda av dem själva. När det kommer till styrelsen och sektionen är det på samma sätt Ordföranden som attesterar inköp och ansvarar för budgeten.

Detta har hittils inte orsakat några problem, men man kan ändå anse att det är olämpligt att en person både kan utföra inköpet och attestera det. Vidare kan man också anse det olämpligt att bara en person attesterar inköpet - ofta anser man det lämpligt att två personer i förening attesterar saker. T.ex. tecknar Ordföranden och Förvaltningschefen sektionens firma endast i förening.

Ta fallet med utskottschefen, d.v.s., inte Ordförande. En lösning kan vara att låta någon av firmatecknarna attestera inköpet när chefen själv utfört det, vilket resulterar i chefen inte attesterar sitt eget inköp, men egentligen utförs fortfarande ingen dubbel attestering. Här kan det tyckas vara orimligt att chefen är helt ansvarig för alla inköp och kan attestera alla inköp förutom sina egna, med tankte på att inköp från utskottets funktionärer ofta sker på uppdrag av chefen. För att komma runt det skulle man kunna kräva att alla inköp attesteras av två personer - chefen och en firmatecknare. Det skulle resultera i mer jobb för firmatecknarna, men attesteringsreglerna blir enklare.

Ta nu fallet med Ordföranden, som attesterar alla inköp till styrelsen och sektionen. Skulle man tillämpa den första lösningen presenterad i förra stycket innebär det mer eller mindre att Förvaltningschefen attesterar samtliga inköp Ordföranden gör till styrelsen och sektionen. Tillämpar man det senare förslaget attesterar Ordföranden och Förvaltningschefen alla inköp till styrelsen och sektionen i förening, oavsett vem som utför dem.

Två saker till:
\begin{tightdashlist}
    \item Det är möjligt att attestera muntligen, borde den möjligheten tas bort?
    \item Efter vi sett över attesteringen, borde reglerna på något sätt införas i styrdokumenten?
\end{tightdashlist}

Jag personligen har inga större bekymmer med att samtliga styrelsemedlemmar kan attestera sina egna inköp - det är trots allt alltid dem själva som är ansvariga för budgeten. Styrelsen och sektionen har alltid sista ordet när det kommer till godkännandet av ett inköp, och kan kräva att den som godkänt inköpet blir ersättningsskyldig.

I det fall att styrelsen och/eller sektionen anser att detta är ett problem tycker jag att man borde lösa det ordentligt, d.v.s. genom att alltid kräva attestering av två personer, alltså att en (annan) firmatecknare alltid attesterar också. Vidare tycker jag inte att det är rimligt att skilja på fallet när en utskottschef eller när en funktionär utför ett inköp. Chefen har det övergripande ansvaret, och om personen i fråga har friheten att attestera sina funktionärers inköp bör den också ha friheten att attestera sina egna inköp på samma sätt.

\newpage

Sammanfattningsvis, tänk över:
\begin{tightdashlist}
    \item Borde möjligheten att attestera muntligt tas bort?
    \item Är det ett problem att en utskottschef kan attestera sina egna inköp?
    \item I så fall, hur löser man det på bästa sätt?
    \item Borde attesteringsregler finnas med i styrdokumenten?
\end{tightdashlist}

\rnamnpost{Erik Månsson}{Ordförande}

\end{paragrafer}

\begin{signatures}{1}
\ist
\signature{\docauthor}{Ordförande}
\end{signatures}

\end{document}
