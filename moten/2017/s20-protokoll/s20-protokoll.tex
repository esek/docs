\documentclass[10pt]{article}
\usepackage[utf8]{inputenc}
\usepackage[swedish]{babel}

\def\mo{Erik Månsson}
\def\ms{Johan Karlberg}
\def\ji{Niklas Gustafson}
%\def\jii{}

\def\doctype{Protokoll} %ex. Kallelse, Handlingar, Protkoll
\def\mname{styrelsemöte} %ex. styrelsemöte, Vårterminsmöte
\def\mnum{S20/17} %ex S02/16, E1/15, VT/13
\def\date{2017-09-21} %YYYY-MM-DD
\def\docauthor{\ms}

\usepackage{../e-mote}
\usepackage{../../../e-sek}

\begin{document}
\showsignfoot

\heading{{\doctype} för {\mname} {\mnum}}

%\naun{}{} %närvarane under
%\nati{} %närvarande till och med
%\nafr{} %närvarande från och med
\section*{Närvarande}
\subsection*{Styrelsen}
\begin{narvarolista}
\nv{Ordförande}{Erik Månsson}{E14}{}
\nv{Kontaktor}{Johan Karlberg}{E14}{}
\nv{Förvaltningschef}{Sophia Grimmeiss Grahm}{BME14}{}
\nv{Cafémästare}{Daniel Bakic}{E15}{\nafr{11}}
\nv{Øverphøs}{Niklas Gustafson}{E15}{}
\nv{SRE-ordförande}{Edvard Carlsson}{E16}{}
\nv{ENU-ordförande}{Josefine Sandström}{E14}{}
\nv{Sexmästare}{Linnea Sjödahl}{BME15}{}
\nv{Krögare}{Markus Rahne}{BME14}{}
\nv{Entertainer}{Albin Nyström Eklund}{BME16}{}
\end{narvarolista}

\subsection*{Ständigt adjungerande}
\begin{narvarolista}
%\nv{Kårordförande}{Linus Hammarlund}{}{}
%\nv{Kårrepresentant}{Anders Nilsson}{}{}
\nv{Kårrepresentant}{Caroline Svensson}{}{}
\nv{Kårrepresentant}{Agnes Sörliden}{}{}
%\nv{Valberedningens ordförande}{Elin Magnusson}{}{}
%\nv{Skattmästare}{Olle Oswald}{}{}
%\nv{Kårrepresentant}{Daniel Damberg}{}{}
%\nv{Kårrepresentant}{John Alvén}{}{}
%\nv{Talman}{Fredrik Peterson}{E14}{}
%\nv{Elektras Ordförande}{Elisabeth Pongratz}{}{}
%\nv{Inspektor}{Monica Almqvist}{}{}
\end{narvarolista}

\begin{comment}
\subsection*{Adjungerande}
\begin{narvarolista}
%\nv{Post}{Namn}{Klass}{}
\end{narvarolista}
\end{comment}

\section*{Protokoll}
\begin{paragrafer}
\p{1}{OFMÖ}{\bes}
Ordförande {\mo} förklarade mötet öppnat 12:14.

\p{2}{Val av mötesordförande}{\bes}
{\valavmo}

\p{3}{Val av mötessekreterare}{\bes}
{\valavms}

\p{4}{Val av justeringsperson}{\bes}
{\valavj}

\p{5}{Godkännande av tid och sätt}{\bes}
{\tosg}

\p{6}{Adjungeringar}{\bes}
{\ingaadj}

%Förnamn Efternamn adjungerades

\p{7}{Godkännande av dagordningen}{\bes}
%Dagordningen godkändes.
Erik \ypa lägga till \S12 ``Kostnad och vinstdelning från Färgfesten''.

Erik \ypa lägga till \S13 ``Betalning av flaggor''.
%Föredragningslistan godkändes med yrkandet.

Föredragningslistan godkändes med samtliga yrkanden.
\p{8}{Föregående mötesprotokoll}{\bes}
\latillprot{S19/17}
%\ingaprot

\p{9}{Fyllnadsval och entledigande av funktionärer}{\bes}
\begin{fyllnadsval} %"Inga fyllnadsval." fylls i automatiskt
%\fval{Namn}{Post}
%\entl{Namn}{Post}
\end{fyllnadsval}

\p{10}{Rapporter}{}
\begin{paragrafer}
\subp{A}{Hur mår alla?}{\info}
Punkten protokollfördes ej.
\subp{B}{Utskottsrapporter}{\info}
NollU är inne på den sista veckan, det känns skönt. De har lite saker att fixa inför Gasquen annars är det ganska lugnt.

InfU börjar att komma igång. Fotograferna har fotat under nollningen och DDG hade precis uppstartsmöte för terminen.

FVU har bokfört. Anders har varit på husstyrelsemöte där bl.a. brandövning, SVL:s gamla lokaler och E:1426 togs upp.

ENU har mejlat lite men det har varit rätt lugnt annars. Planeringen inför nästa Lunch med Ingenjör är i full gång och det är dags att mejla företag till det. De har försökt att höra sig runt för att få fler BME företag, bl.a. mejlat programledningen och Sophia har kollat i sin klassgrupp på FB.

NöjUs planer för hösten är mer kvällshäng, spelkvällar, bastuhäng detta då Albins erfarenhet är att många fokuserar på plugget under hösten. Olympiaden gick bra, deltagarna var nöjda.

SRE har de sista CEQ-mötena idag. Utskottet ska träffas imorgon och sammanfatta allt. Nästa vecka är det utbytesmingel i samarbete med bland annat D och SVL.

KM hade tillsammans med SRE pluggille, det var uppskattat. Maten var sen men de fick mycket mat som kompensation. Nollningen över för KM:s del och gör comeback 6/10.

E6 har mycket att göra inför Gasquen. Trots att det känns som mycket känner de sig väl förberedda. Nollesittningen i lördags var uppskattad.
\subp{C}{Ekonomisk rapport}{\info}
Sophia meddelade att det går bra. Det är hög volatilitet men det finns pengar på banken.
\subp{D}{Kåren informerar}{\info}
Kåren behöver en representant i AFs-stipendiefond. Kontakta Patrik kårordförande om det finns intresse. Arkads värdansökan har öppnades i måndags.

Kåren söker personer till den internationella vårterminsstarten. Meddelade Agnes.
\end{paragrafer}

\p{11}{Expo}{\dis} %TODO
Erik föreslår vecka 40 och 41 som alternativ att ha Expot. Tidsmässigt är det ungefär klockan 10.00 till 15.00. 9/10 Är ett datum som verkar bra för alla på mötet.

Mötet diskuterade ämnet.
\p{12}{Kostnad och vinstdelning från Färgfesten}{}
Färgfesten kommer att kosta mycket pengar så Frida Ordförande på ING vill att alla ska dela på kostnaderna och vinsten, detta för att sprida risken.

Mötet diskuterade ämnet.

Mötet anser att vi kan stå tillsammans med de andra sektionerna som stöd.
\p{13}{Betalning av flaggor}{}
Erik \ypa betalningen av fakturan för flaggorna på \SI{2428}{kr} belastar dispositionsfonden.

\Mbaby
\p{14}{Nästa styrelsemöte}{\bes}
{\Mba} nästa styrelsemöte ska äga rum 2017-09-28 12:10 i E:1424.

\p{15}{Beslutsuppföljning}{\bes}
{\Ibfu}

\p{16}{Övrigt}{\dis}
Mötet diskuterade vad vi ska göra med utomlundarna.

Daniel tyckte att de som ska upp till Chalmers ska kolla på biljetter till tåget.
\p{17}{OFMA}{\bes}
{\mo} förklarade mötet avslutat 12:56.

\end{paragrafer}

%\newpage
\hidesignfoot
\begin{signatures}{3}
\signature{\mo}{Mötesordförande}
\signature{\ms}{Mötessekreterare}
\signature{\ji}{Justerare}
\end{signatures}
\end{document}
