\documentclass[10pt]{article}
\usepackage[utf8]{inputenc}
\usepackage[swedish]{babel}

\def\mo{Erik Månsson}
\def\ms{Johan Karlberg}
\def\ji{Linnea Sjödahl}
%\def\jii{}

\def\doctype{Protokoll} %ex. Kallelse, Handlingar, Protkoll
\def\mname{styrelsemöte} %ex. styrelsemöte, Vårterminsmöte
\def\mnum{S09/17} %ex S02/16, E1/15, VT/13
\def\date{2017-03-30} %YYYY-MM-DD
\def\docauthor{\ms}

\usepackage{../e-mote}
\usepackage{../../../e-sek}

\begin{document}
\showsignfoot

\heading{{\doctype} för {\mname} {\mnum}}

%\naun{}{} %närvarane under
%\nati{} %närvarande till och med
%\nafr{} %närvarande från och med
\section*{Närvarande}
\subsection*{Styrelsen}
\begin{narvarolista}
\nv{Ordförande}{Erik Månsson}{E14}{}
\nv{Kontaktor}{Johan Karlberg}{E14}{}
%\nv{Förvaltningschef}{Sophia Grimmeiss Grahm}{BME14}{}
\nv{Cafémästare}{Daniel Bakic}{E15}{\nafr{11}}
\nv{Øverphøs}{Niklas Gustafson}{E15}{}
\nv{SRE-ordförande}{Pontus Landgren}{E14}{}
\nv{ENU-ordförande}{Josefine Sandström}{E14}{}
\nv{Sexmästare}{Linnea Sjödahl}{BME15}{}
%\nv{Krögare}{Markus Rahne}{BME14}{}
\nv{Entertainer}{Albin Nyström Eklund}{BME16}{}
\end{narvarolista}

\subsection*{Ständigt adjungerande}
\begin{narvarolista}
%\nv{Kårordförande}{Linus Hammarlund}{}{}
\nv{Kårrepresentant}{Jacob Karlsson}{}{}
\nv{Aktivietssamordnare}{Lovisa Majtorp}{}{}
%\nv{Valberedningens ordförande}{Elin Magnusson}{}{}
%\nv{Skattmästare}{Olle Oswald}{}{}
%\nv{Kårrepresentant}{Daniel Damberg}{}{}
%\nv{Kårrepresentant}{John Alvén}{}{}
%\nv{Talman}{Fredrik Peterson}{E14}{}
%\nv{Elektras Ordförande}{Elisabeth Pongratz}{}{}
%\nv{Inspektor}{Monica Almqvist}{}{}
\end{narvarolista}

\begin{comment}
\subsection*{Adjungerande}
\begin{narvarolista}
%\nv{Post}{Namn}{Klass}{}
\end{narvarolista}
\end{comment}

\section*{Protokoll}
\begin{paragrafer}
\p{1}{OFMÖ}{\bes}
Ordförande {\mo} förklarade mötet öppnat 12:14.

\p{2}{Val av mötesordförande}{\bes}
{\valavmo}

\p{3}{Val av mötessekreterare}{\bes}
{\valavms}

\p{4}{Val av justeringsperson}{\bes}
{\valavj}

\p{5}{Godkännande av tid och sätt}{\bes}
{\tosg}

\p{6}{Adjungeringar}{\bes}
{\ingaadj}

%Förnamn Efternamn adjungerades

\p{7}{Godkännande av dagordningen}{\bes}
Dagordningen godkändes.
%Fredrik \ypa att lägga till \S18b ``Teknikfokus utnyttjande av LED-café''.
%Föredragningslistan godkändes med yrkandet.
%Föredragningslistan godkändes med samtliga yrkanden.

\p{8}{Föregående mötesprotokoll}{\bes}
\latillprot{S08/17}
%\ingaprot

\p{9}{Fyllnadsval och entledigande av funktionärer}{\bes}
\begin{fyllnadsval} %"Inga fyllnadsval." fylls i automatiskt
\fval{David Cordesius}{Kodhackare}
\fval{Johan Sievert Lindeskog}{Diod}
\fval{Viktor Drakfelt}{Diod}
\fval{Adem Saran}{Diod}
\fval{Mansoor Ashrati}{Diod}
\fval{Sanna Nordberg}{Diod}
\fval{Johan Wendt}{Diod}
\fval{Amanda Nilsson}{Diod}
\fval{Jonas Andersson}{Diod}
\entl{Niklas Karlsson}{Redaktör}
\end{fyllnadsval}
\p{10}{Rapporter}{}
\begin{paragrafer}
\subp{A}{Hur mår alla?}{\info}
Punkten protokollfördes ej.
\subp{B}{Utskottsrapporter}{\info}
LED flyter på som vanligt, dock lite dåligt med kunder i veckan. På lördag ska Daniel hålla en station i caféet på phadderutbildningen och samtidigt försöka ragga till sig fler Dioder.

Förvaltningsutskottet har bokfört en massa och jobbat med annat kontinuerligt arbete. Sophia har varit på ett superbra VOK-möte och är jättetaggad.

NollU har hållit på med phadderutbildningar och planerat temasläppet. Det går framåt.

ENU mejlar företag. De hade kvällsföredrag med Svep igår vilket var lyckat, ENU ska fakturera dem nästa vecka. Det går rykten om en pub med Ericsson i framtiden.

NöjU fixar med det sista inför Agent 00E. Anmälningar ligger uppe och det ser bra ut. Biljettsläppet till tandem är ute och planerar hållas 6:e april samma dag som fritidsledare håller i påskäggsjakt. Spelkväll är förslaget till 25:e april, Albin inväntar svar från spelemän.

E6 höll i en sittning för Flickor på Teknis igår och det gick bra. De ska ha möte med mästarna och prata igenom vad som gick bra och vad som kan förbättras.
De har en alumnisittning på lördag också så mycket den här veckan.

Arbetet i SRE flyter på, det är snart dags att ta hand om CEQ:er igen. Pontus har informerat programledning E angående mentorsprogrammet och ska även prata med PLED BME. Världsmästarna har börjat använda internationella gruppen på facebook och det verkar gå bra.

InfU rullar på, nollningshemsidan är nästan uppe och den nyinköpta tekniken fungerar bra.

Erik fixat med mentorprogrammet, programledningen har tagit emot den och de är positiva. Erik ska revidera förslaget en gång från programledningen och sen skicka ut till sektionen. Han har haft inspektorsmiddag, det var mysigt och han har fixat present till Data.
\subp{C}{Ekonomisk rapport}{\info}
Sophia var inte närvarande på mötet, men det ska inte vara några konstigheter med ekonomin.
\subp{D}{Kåren informerar}{\info}
Lovisa meddelade att den 26 april är det pubrunda, det går att söka poster till aktivitetsutskottet på kåren, även till F1-röj och lite annat. Det går att köpa eftersläppsbiljetter till Corneliusbalen. Temasläpp för kårens nollningstema är den 5 april.
\end{paragrafer}
\p{11}{VT/17}{\dis}
Mötet gick genom propositioner som är planerade.
\p{12}{Återbetalning}{\bes}
Punkten skjuts upp till nästa möte.
\p{13}{Missnöje på Jodel}{\dis}
Det som har skrivits är att sektionerna känns utfrysande. Vi har uppmanat till att folk ska vända sig till vårt likabehandlingsombud. Erik har pratat med ordförande på F, då de har skrivits mest om E och F.

Albin gav ett konkret exempel på vad han hörde när han började och påpekade att han tror att phadderskapet är en stor grej att bli bortvald från.

Josefine sa att på valmöten är valen öppna för alla, känner man någon som ska ställa upp på en post så röstar man förmodligen på sin kompis.

Erik tycker att phadder är en väldigt givande post till relativt lite arbete, så det är inte konstigt att det är eftertraktat att vara phadder

Linnea sa att KM och Sexet har kollat så att alla seriöst sökande får minst en post om de har sökt till både KM och Sexet.

Erik berättade lite om gamla problem som inte finns kvar längre.

Ämnet diskuterades ytterligare.
\p{14}{Nästa styrelsemöte}{\bes}
{\Mba} nästa styrelsemöte ska äga rum 2017-04-06 12:10 i E:1426.
\p{15}{Beslutsuppföljning}{\bes}
{\Ibfu}
\p{16}{Övrigt}{\dis}
Skriv texter till NollEguiden.

Linnea saknar gafflar, hon ska titta på att köpa in nya.

Världsmästarna vill att utbytesstudenter ska få möjligheten att gå med i utskott, t.ex. CM, KM och E6. Mandatperiod bör vara förkortad eftersom att de flesta är här i ett halvår.

Josefine vill dansa och spela badminton på Sporta med E.

Albin berättade om hur han har tänkt att göra med priserna för access till biljard. På gillen kommer saker att lånas ut mot pant.
\p{17}{OFMA}{\bes}
{\mo} förklarade mötet avslutat 13:05.

\end{paragrafer}

\newpage
\hidesignfoot
\begin{signatures}{3}
\signature{\mo}{Mötesordförande}
\signature{\ms}{Mötessekreterare}
\signature{\ji}{Justerare}
\end{signatures}
\end{document}
