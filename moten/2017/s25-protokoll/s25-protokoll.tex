\documentclass[10pt]{article}
\usepackage[utf8]{inputenc}
\usepackage[swedish]{babel}

\def\mo{Erik Månsson}
\def\ms{Johan Karlberg}
\def\ji{}
%\def\jii{}

\def\doctype{Protokoll} %ex. Kallelse, Handlingar, Protkoll
\def\mname{styrelsemöte} %ex. styrelsemöte, Vårterminsmöte
\def\mnum{S25/17} %ex S02/16, E1/15, VT/13
\def\date{2017-11-02} %YYYY-MM-DD
\def\docauthor{\ms}

\usepackage{../e-mote}
\usepackage{../../../e-sek}

\begin{document}
\showsignfoot

\heading{{\doctype} för {\mname} {\mnum}}

%\naun{}{} %närvarane under
%\nati{} %närvarande till och med
%\nafr{} %närvarande från och med
\section*{Närvarande}
\subsection*{Styrelsen}
\begin{narvarolista}
\nv{Ordförande}{Erik Månsson}{E14}{}
\nv{Kontaktor}{Johan Karlberg}{E14}{}
\nv{Förvaltningschef}{Sophia Grimmeiss Grahm}{BME14}{}
\nv{Cafémästare}{Daniel Bakic}{E15}{}
%\nv{Øverphøs}{Niklas Gustafson}{E15}{}
%\nv{SRE-ordförande}{Edvard Carlsson}{E16}{}
%\nv{ENU-ordförande}{Josefine Sandström}{E14}{}
\nv{Sexmästare}{Linnea Sjödahl}{BME15}{}
\nv{Krögare}{Markus Rahne}{BME14}{}
%\nv{Entertainer}{Albin Nyström Eklund}{BME16}{}
\end{narvarolista}

\subsection*{Ständigt adjungerande}
\begin{narvarolista}
%\nv{Kårordförande}{Linus Hammarlund}{}{}
%\nv{Kårrepresentant}{Anders Nilsson}{}{}
%\nv{Kårrepresentant}{Caroline Svensson}{}{}
\nv{Kårrepresentant}{Agnes Sörliden}{}{}
%\nv{Valberedningens ordförande}{Elin Magnusson}{}{}
%\nv{Skattmästare}{Olle Oswald}{}{}
%\nv{Kårrepresentant}{Daniel Damberg}{}{}
%\nv{Kårrepresentant}{John Alvén}{}{}
%\nv{Talman}{Fredrik Peterson}{E14}{}
%\nv{Elektras Ordförande}{Elisabeth Pongratz}{}{}
%\nv{Inspektor}{Monica Almqvist}{}{}
\end{narvarolista}

\begin{comment}
\subsection*{Adjungerande}
\begin{narvarolista}
%\nv{Post}{Namn}{Klass}{}
\end{narvarolista}
\end{comment}

\section*{Protokoll}
\begin{paragrafer}
\p{1}{OFMÖ}{\bes}
Ordförande {\mo} förklarade mötet öppnat 12:19.

\p{2}{Val av mötesordförande}{\bes}
{\valavmo}

\p{3}{Val av mötessekreterare}{\bes}
{\valavms}

\p{4}{Val av justeringsperson}{\bes}
{\valavj}

\p{5}{Godkännande av tid och sätt}{\bes}
{\tosg}

\p{6}{Adjungeringar}{\bes}
{\ingaadj}

%Förnamn Efternamn adjungerades

\p{7}{Godkännande av dagordningen}{\bes}
Dagordningen godkändes.
%Fredrik \ypa att lägga till \S18b ``Teknikfokus utnyttjande av LED-café''.
%Föredragningslistan godkändes med yrkandet.
%Föredragningslistan godkändes med samtliga yrkanden.

\p{8}{Föregående mötesprotokoll}{\bes}
\latillprot{S24/17}
%\ingaprot

\p{9}{Fyllnadsval och entledigande av funktionärer}{\bes}
\begin{fyllnadsval} %"Inga fyllnadsval." fylls i automatiskt
\fval{Filip Larsson}{Karnevalsmalaj}
%\entl{Namn}{Post}
\end{fyllnadsval}

\p{10}{Rapporter}{}
\begin{paragrafer}
\subp{A}{Hur mår alla?}{\info}
Punkten protokollfördes ej.
\subp{B}{Utskottsrapporter}{\info}
CM har äntligen fått tillbaka vår kära Ulla vilket är asnajs! Hon börjar sakta men säkert komma in i arbetet igen och vi gör allt vi kan för att hon ska känna sig trygg och välkommen.

Vi är nästan klara med propositionen om Dagsansvariga och nu kommer vi även ha mer tid till att ändra på saker i LED, typ som att sätta etiketter i alla hyllor vart saker ska vara, skriva ut tydliga instruktioner på hur sallader och mackor ska göras o.s.v.

Vi har en färdig design till stämpelkort för kaffe som ska skickas iväg till tryckeriet och sedan börja säljas.

TELAVOX har tagit kontakt med oss och vill köpa kanelbullar av oss under ARKAD, ska nog gå att lösa. ARKAD vill ha hjälp av oss då dem vill servera kaffe i sin Lounge som ska vara i E-huset. Jag hjälper gärna till.

FVU försöker bli klara med nollningens bokföring men det är fortfarande mycket arbete kvar.

KM har gille imorgon och kommer senare i år ha ölprovning.

E6 jobbade ET-slasque och det var kul. De skall ha möte nästa vecka och spika något ytterligare event än IKEA sittningen.

Erik; bifall på vår motion med tillägg att ING skall betala halva kostnaden. Vi jämkade oss.
\subp{C}{Ekonomisk rapport}{\info}
\SI{700000}{kr} på kontot.
\subp{D}{Kåren informerar}{\info}
Agnes; Enkät om funktionärsutbildning är utskickad. Det går att söka internationell phadder nu inför vårens nollning. Pedellen är sjukskriven tills efter Arkad.
\end{paragrafer}

\p{11}{HT-möte}{\dis}
Mötet gick igenom motioner som är inskickade.

Markus skjuter på ugnförslaget tills vidare.

Mötet diskuterade matlagningen inför mötena.
\p{12}{Nästa styrelsemöte}{\bes}
{\Mba} nästa styrelsemöte ska äga rum 2017-11-09 12:10 i E:1124.

\p{13}{Beslutsuppföljning}{\bes}
{\Ibfu}

\p{14}{Övrigt}{\dis}
Erik; städschema!
\p{15}{OFMA}{\bes}
{\mo} förklarade mötet avslutat 12:44.

\end{paragrafer}

%\newpage
\hidesignfoot
\begin{signatures}{3}
\signature{\mo}{Mötesordförande}
\signature{\ms}{Mötessekreterare}
\signature{\ji}{Justerare}
\end{signatures}
\end{document}
