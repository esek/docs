\documentclass[../_main/handlingar.tex]{subfiles}

\begin{document}
\proposition{Uppdatering av postbeskrivning i Nöjesutskottet}
Nöjesutskottet har under de senaste åren upplevt en problematik med ojämn arbetsbelastning. Genom att omformulera postbeskrivningar för utskottet kommer det bli en tydligare och därmed jämnare arbetsfördelningen. Postbeskrivningarna ändras för att matcha den verksamhet som utskottet de senaste åren bedrivit och för att ge en jämnare arbetsbelastning inom utskottet. Posten “Speleman” kommer således att tas bort och ansvaret överföras på fritidsledare och i viss mån Entertainer.

Med anledning av detta yrkar styrelsen på
\begin{attsatser}
    \att att ändra reglementet \S10:2:H till\par

      Entertainer (u)
        \begin{tightdashlist}
            \item har det övergripande ansvaret för Sektionens kultur-, nöjes- och fritidsaktiviteter
            \item ansvarar för Sektionens instrument,
            \item ansvarar för planering och genomförandet av UtEDischot tillsammans med D-sektionen,
            \item ansvarar för utkvittering av access till biljard- och pingisskåpet.
        \end{tightdashlist}

      Vice Entertainer (2)
          \begin{tightdashlist}
              \item bistår Entertainern i dennes arbete med nöjes- och fritidsaktiviteter samt övertar Entertainerns uppgifter om denne inte kan närvara vid tillställningen,
              \item \hl{ansvarar för genomförandet av tandemstaffeten.}
          \end{tightdashlist}

      Fritidsledare (4)
          \begin{tightdashlist}
              \item \hl{bistår Entertainern i dennes arbete med nöjes- och fritidsaktiviteter,}
              \item \hl{ansvarar för att arrangera mindre evenemang regelbundet under årets gång däribland spelkvällar,}
              \item \hl{ansvarar för att biljardbordet, pingisbordet och Sektionens sällskapsspel är i god kvalité.}
          \end{tightdashlist}

        Idrottsförman (2)
          \begin{tightdashlist}
              \item \hl{ansvarar för sektionens idrottsarrangemang.}
          \end{tightdashlist}

        Karnevalsmalaj (1)
          \begin{tightdashlist}
              \item organiserar Sektionens deltagande i och omkring Lundakarnevalen (läsår/vart fjärde år).
          \end{tightdashlist}

        Stridsrop (2)
          \begin{tightdashlist}
              \item ansvarar för sångarstridens planering, produktion och genomförande,
              \item organiserar Sektionens luciatåg,
              \item \hl{ansvarar för att en tackfest för Sektionens deltagare i Sångarstriden genomförs.}
          \end{tightdashlist}

        Umph-meister (2)
        \begin{tightdashlist}
            \item ansvarar för passande musik på de tillställningar där musik är passande.
        \end{tightdashlist}

        Øverbanan (1)
        \begin{tightdashlist}
            \item ledare och kontaktperson för E-sektionens husband - “E-lektro banana band”,
            \item ansvarar för husbandets medverkan i Sångarstriden,
            \item ansvarar för att rekrytera bandmedlemmar - “Bananer” - till husbandet.
        \end{tightdashlist}

\end{attsatser}
\begin{signatures}{1}
    \ist
    \signature{Albin Nyström Eklund}{Entertainer}
\end{signatures}

\end{document}
