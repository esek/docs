\documentclass[../_main/handlingar.tex]{subfiles}

\begin{document}
\utskottsrapport{Styrelsen}

Sedan Vårterminsmötet har Styrelsens arbete rullat på bra. Mycket av styrelsens fokus har legat på dagligverksamheten. Under våren har våra utskott arrangerat många uppskattade aktiviteter för våra medlemmar. Direkt efter vårterminsmötet arrangerades också ett mycket lyckat 55-årsjubileum!

Dessvärre fick nollningen en ganska tuff start för styrelsen och CM då Ulla tyvärr inte kunnat vara med oss på grund av personliga skäl. Mycket tid har gått till att stötta och hjälpa CM att driva caféet helt ideellt. Eftersom Ulla ska gå i pension vid årsskiftet var styrelsen inställd på att caféet skulle drivas ideellt i framtiden, men det var inte något som vi var beredda på redan till höstterminen. Lyckligtvis har det trots omständigheterna gått relativt bra och Ulla kommer förhoppningsvis vara med oss större delen av läsperiod 2!

Resten av nollningen har gått bra med ett fint samarbete mellan utskotten. Själva styrelsen har hjälpt till lite här och där när det behövdes, inte minst på första dagen på St. Hans vilket var väldigt trevligt! Några enstaka incidenter har inträffat men det är ingenting som vi inte kunnat reda ut. Nollningen rundades sedan av med en fantastisk NollEqasque där vi fick besök från några i styrelsen vid MiT på KTH.

Efter nollningen har styrelsen tagit en liten paus för att sedan raskt gå vidare till att börja jobba med med sektionsmötena. Innan inläsningsveckan drog igång höll styrelsen i det årliga expot som drog väldigt mycket folk. Många nomineringar och kandidaturer har kommit in vilket styrelsen tycker är otroligt kul! Styrelsens största frågor inför HT kommer vara hur caféet ska drivas och hur vi ska bygga upp vår budget för att ``kompensera'' för vår förväntade ökade vinst i CM. Vi är väldigt optimistiska och tror att nästa år kommer bli ett toppenår för sektionen!

Under nollningen skedde en liten incident där FlyING-biljetter såldes för fel pris vilket drabbade E, V, och A-sektionen. Kåren la det ekonomiska ansvaret på oss vilket styrelserna i de drabbade sektionerna ansåg vara fel. Efter diskussioner i styrelsen och med NollU har jag tillsammans med de andra ordföranden gjort en överklagan till Kårens fullmäktige.

Samarbetet med resten av Ordförandekollegiet har gått bra med mycket skoj under nollningen och många bra diskussioner!

\begin{signatures}{1}
    \mvh
    \signature{\ordf}{Ordförande}
\end{signatures}

\end{document}
