\documentclass[../_main/handlingar.tex]{subfiles}

\begin{document}
\utskottsrapport{Cafémästeriet}
Under hösten har utskottet haft otroligt mycket att göra gällande caféets verksamhet och alla har jobbat väldigt hårt för att allt ska flyta på så bra som möjligt.

Innan nollningens början städade vi i vårt förråd och monterade upp de nya hyllorna som köpts in under sommaren. Det blev riktigt bra då det nu ser mycket fräschare ut och det finns mycket mer plats att ställa saker på. Inför måndagen i läsvecka 0 gjorde vi, tillsammans med hjälp från andra, lunch till alla nollor, phaddrar, styrelsen, phöset och programledningen i form av pastasallader. När alla nollor kom och hämtade sin lunch i caféet fick dem en kort presentation av Cafémästeriet och vår verksamhet. Sedan har phaddrarna och nollorna/ettorna under lp1 och lp2 fått testa på att jobba i LED. Detta gav goda resultat och en del av dem har valt att engagera sig inom Cafémästeriet.

Då Ulla varit sjukskriven hela lp1 har vi i det centrala teamet (Cafémästare, Vice och Inköps-och lagerchefer) fått lägga ner väldigt mycket tid och energi för att driva caféet. Vi har ibland fått stänga tidigt och även haft helt stängt ett par dagar då det inte funnits nog med hjälp för att kunna hålla caféet öppet. Vi har under denna avsaknad av personal märkt hur pass mycket som behöver ändras inför nästa år, och har fått många idéer på hur vi kan driva caféet ideellt. Vi har bland annat skrivit ihop en proposition gällande införandet av en ny post i Cafémästeriet ``Halvledare'' som är tänkt ska agera som en dagsanvarig.

Miljö- och hälsovårdsmyndigheten har varit på besök två gånger under lp1. Första gången hade hon lite dålig timing för oss då det var extra kaos i caféet just den dagen. Hon hade därför lite för många klagomål för att godkänna oss direkt. Vi har utifrån hennes klagomål ändrat en del gällande hantering av kylvaror samt strukturerat om vårt lager och i diskrummet. När hon kom förbi andra gången var hon väldigt nöjd med våra förbättringar och godkände oss.

Vi har precis som i våras haft samarbete med andra utskott och sektioner genom att sälja kaffe till dem när det behövts. Under nollningen höll Cafémästaren en av stationerna på St. Hans och CM hade även ihop med Ordförande station på stadskringvandringen. Det var väldigt roligt att hålla i station och det är ett bra sätt att promota utskottet. Under kanelbullens dag bakade vi extra många kanelbullar och lyckades sälja en hel del, vilket var kul. Vi har blivit kontaktade av ARKAD som behöver hjälp med kaffe till deras lounge i E-huset och ska hjälpa dem med det. Vi ska även sälja och baka kanelbullar åt ett av företagen, Telavox. Vi ska även införa möjligheten att köpa stämpelkort för kaffe/te i LED.

Det har gått bra ekonomiskt och vi kommer förmodligen nå över vår budget då vi inte kommer betala ut lika mycket lön som vi budgeterat för.

\begin{signatures}{1}
    \mvh
    \signature{Daniel Bakic}{Cafémästare}
\end{signatures}

\end{document}
