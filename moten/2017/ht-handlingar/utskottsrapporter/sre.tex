\documentclass[../_main/handlingar.tex]{subfiles}

\begin{document}
\utskottsrapport{Studierådet}

Sedan VT har studierådet fokuserat mycket på synlighet, för att visa hur utskottet arbetar, vår tillgänglighet och på de resultat vi nått. Under nollningen syntes vi tidigt med både workshop och grillning första veckan. Vi har sedan dess sett ett stort och tidigt intresse för studierådet, speciellt från de nyintagna.

Under nollningen arrangerades fyra stycken studiekvällar, där konceptet var lite förändrat från tidigare år; vi valde att servera lite mer riktig mat i hopp om att det skulle få deltagarna att orka stanna lite längre och verkligen få ut en kväll av evenemanget. Den sista studiekvällen anordnades tillsammans med källarmästeriet, även detta med tidigare nämnd anledning. Detta anser SRE var ett lyckat och kul koncept, vi tror även att kvällens deltagare uppskattade det. Utöver detta avlastade det oss i studierådet från matlagningen och kunde istället fokusera på att administrera kvällen.

Vi har även arrangerat ett utbytesmingel tillsammans med D-sektionen. Detta för att öka intresset för utbytesstudier, då detta historiskt sett varit lågt, speciellt för E-programmet. Våra världsmästare har även arbetat med att försöka integrera och engagera inresande studenter i sektionen.

Utöver detta har studierådet censurerat de CEQ-rapporter som studenterna på E- och BME-programmen skrivit och sedan framfört klassens synpunkter på respektive kursutvärdeingsmöte tillsammans med programledning och kursanvarig. Ordförande har deltagit i studierådsordförandekollegiet SRX, där teknologkåren samt andra sektioner delar med sig av sitt arbete och diskuterar hur vi bäst bedriver vår verksamhet.

\begin{signatures}{1}
    \mvh
    \signature{Edvard Carlsson}{SRE-Ordförande}
\end{signatures}

\end{document}
