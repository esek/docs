\documentclass[../_main/handlingar.tex]{subfiles}

\begin{document}
\utskottsrapport{Källarmästeriet}
Sedan den förra utskottsrapporten publicering har mycket skett, och inte bara nollning. Internt så valdes en ny Vice Krögare, Saga Juniwik, in för att ersätta det planerade bortfallet av den tidigare Vice Krögare, Oscar Uggla, som åkte till Kina på utbyte. I form av inköp av material har KM köpt in nya delar för att få liv i kökets tapptorn ännu en gång, vilket har visats sig vara mycket uppskattat, och KM har hitintills kunnat erbjuda fatöl på två olika tillfällen. Det har även gjorts försök att laga en klämgrill, vänligt donerad av Krögare Emerita Malin Lindström, där en effektiv värmeisolering är det enda som saknas.

När det kommer till själva verksamheten avslutades våren på högsta möjliga nivå med Jubileumsgillet, en pub öppet för nya som gamla, teknologer som ofrälse. När nollningen sedan drog igång efter en välförtjänt sommarledighet höll KM i sju olika evenemang, däribland maten för UtEDischot och Regattan. Välkomstgillet hade planerats väl mellan KM och NollU och genomfördes med bravur från alla involverade. Tyvärr fick nollningens avslutande temagille flyttas på grund av Universitets 350-årsjubileum, vilket påverkade besöksmängden något jämfört med tidigare år, men KM hade kul ändå.

Efter nollningen har KMs verksamhet fått sig en rivstart med mycket välbesökta pubar, vilket är fantastiskt roligt. I skrivande stund jobbar KM inför en ölprovning och börjar planera inför det stora Julgillet!

Ekonomiskt sett ser det mycket bra ut för KM, med ett mycket bra resultat från våren, mycket tack vare ett mycket lyckat första gille, samt Jubileumsgillet. Nollningen gick sämre än väntat på grund av att KM var tvungna att betala för utökat tillstånd där det tidigare år inte behövts, men då hösten fortsatt som den gjort kommer det inte bli några problem för KM att slå sin budget med råge.

\begin{signatures}{1}
    \mvh
    \signature{Markus Rahne}{Krögare}
\end{signatures}

\end{document}
