\documentclass[../_main/handlingar.tex]{subfiles}

\begin{document}
\section{Ekonomisk rapport}
E-sektionens ekonomi mår bra och vi har en fortsatt mycket god förmåga att betala alla våra kortfristiga skulder. Några långfristiga skulder såsom banklån eller liknande har vi inte i nuläget. Våra tillgångar uppgår i dagsläget till \SI{734257.50}{kr} enligt bokfört underlag till och med 4:e november, exklusive lager. En stor del av dessa tillgångar är öronmärkta och ligger i Sektionens fiktiva fonder. De pengar som ligger i fonderna är avsedda för olika ändamål såsom inköp av ny utrustning, reparationer eller som ersättning vid skador med mera. Det bör tilläggas att det skulle vara mycket oklokt att spendera samtliga medel i fonderna eftersom det skulle lämna Sektionen ekonomiskt sårbar om vi skulle drabbas av en större oförutsedd utgift.

På nästa sida ser ni en balansrapport för Sektionen där ni kan se aktuella tillgångar och skulder. Dagsaktuella värden på våra bankkonton respektive handkassan samt övriga tillgångar redovisas på mötet.

Halvårsbokslutet som bifogats i handlingarna innefattar de transaktioner som har skett mellan 2017-01-01 och 2017-06-30. Det bör noteras att inga periodiseringar har gjorts vilket medför att intäkten för Teknikfokus därmed inte är med i resultatet för första halvåret. Det bör också tilläggas att sektionens verksamhet skiljer sig markant mellan våren och hösten.

Sektionens resultat för första halvåret är \SI{41836.32}{kr}.
\begin{signatures}{1}
    \mvh
    \signature{Sophia Grimmeiss Grahm}{Förvaltningschef}
\end{signatures}

\end{document}
