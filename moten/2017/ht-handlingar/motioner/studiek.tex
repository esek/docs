
\documentclass[../_main/handlingar.tex]{subfiles}

\begin{document}
\motion{Veckoliga Studiekvällar med tilltugg}
Bakgrund: Under nollningen HT17 hölls studiekvällar med kvällsmat en gång i veckan. Detta initiativ uppskattades av många nollor, som fann att det bidrog till en god studiemiljö. Möjligheten att kunna få hjälp av andra studenter, såsom elektrogud, samt tillgången till mat, innebar att studenterna
kunde studera effektivt och få mer gjort på en kväll. Då studenterna inte behövde återvända hem för att laga mat, kunde plugga i en god studiemiljö och få hjälp med alla svåra problem de stod inför.

Med detta i åtanke föreslås följande:

Under VT18 hålls studiekvällar som ett pilotprojekt i Edekvata en gång i veckan. Under dessa kvällar skall det finnas tillgång till mat. Maten bör antingen vara gratis eller ha ett förmånligt pris för
studenter. Detta kan exempelvis vara en gryta med ris, alltså någonting enkelt.

För att uppnå detta yrkar motionärerna på följande

\begin{attsatser}
  \att avsätta \SI{10000}{kr} till projektet,
  \att kostnaderna belastar utrustningsfonden,
  \att motionärerna väljs in som projektfunktionärer, samt
  \att beslutet läggs på beslutsuppföljningen till HT/18 med William Marnfeldt som ansvarig.
\end{attsatser}

\begin{signatures}{2}
    Dygder visar vägen
    \signature{William Marnfeldt}{E17}
    \signature{Filip Larsson}{E17}
\end{signatures}

\end{document}
