
\documentclass[../_main/handlingar.tex]{subfiles}

\begin{document}
\motion{Ändra namnet på likabehandlingsombudet}
Ett skyddsombud ska verka för en bra studiemiljö, både fysiskt och psykiskt och
skyddsombudets rättigheter är reglerade i arbetsmiljölagen. I E-sektionens reglemente har vi två olika funktionärsposter som motsvarar skyddsombudet, det är Skyddsombudet och Likabehandlingsombudet. Ett skyddsombud har rättigheter och skyldigheter enligt arbetsmiljölagen, vilket sektionens likabehandlingsombud inte har. Exempel på detta är att skyddsombudet har rätt till ”den utbildning och ledighet som krävs för att utföra uppdraget”,
”rätt till den information som behövs för uppdraget” (arbetsmiljölagen 6 kap. 18 §) och har rätt att begära åtgärder ifall arbetsmiljön inte är tillfredsställande (arbetsmiljölagen 6 kap. 6 a §).
Om likabehandlingsombudet ändrar namn till Skyddsombud med likabehandlingsansvar
kommer även denna post ha samma rättigheter och skyldigheter som ett skyddsombud samt att likabehandlingsombudet får rätt till en relevant utbildning för posten. Med ovanstående i åtanke yrkar motionärerna


\begin{attsatser}
  \att i reglementet under 10:2:M Funktionärerna i studierådet, SRE ändra på den
  nuvarande till det gulmarkerade:

  Likabehandlingsombud (2)
  \begin{dashlist}
      \item ansvarar för att bevaka Sektionens verksamhet från ett likabehandlingsperspektiv
      \item ska verka för en jämlik studiemiljö
      \item ska uppmärksamma styrelsen på situationer och miljöer som skulle kunna upplevas som kränkande av studenter vid Sektionen
      \item ska hålla Sektionen informerad om TLTH:s policy för likabehandling
      \item ska fungera som kontaktperson och hjälp för medlemmar som anser sig särbehandlade av personer kopplade till högskolan eller programmet
  \end{dashlist}
  Skyddsombud (1)
      \begin{dashlist}
          \item ska aktivt verka för allas trevnad i den egna arbetsmiljön
          \item ska kunna ta emot klagomål och svara på frågor om den egna arbetsmiljön
          \item ska påverka studenternas arbetsförhållanden i syfte att bidra till en god studiemiljö
          \item ska delta i hustes skyddsrond, bedöma hur förändringar påverkar studenternas arbetsmiljö samt att
hålla sig underrättad om arbetsmiljölagstiftningen
    \item ska finnas representerad i husets HMS-kommitté (Hälsa Miljö och Säkerhetskommitté) och där hålla
sig uppdaterad om läget i huset och lyfta frågor som berör studenternas studiemiljö
    \item ansvarar för att se till att sektionens styrelse är informerad om arbetet i HMS-kommittén
    \item ska hålla sektionen informerad om Teknologkårens arbetsmiljöarbete
      \end{dashlist}

      till

  \hl{Skyddsombud med ansvar för fysisk miljö (1)}
  \begin{tightdashlist}
      \item ska aktivt verka för allas trevnad i den egna arbetsmiljön
      \item ska kunna ta emot klagomål och svara på frågor om den egna arbetsmiljön
      \item ska påverka studenternas arbetsförhållanden i syfte att bidra till en god studiemiljö
      \item ska delta i \hl{husets} skyddsrond, bedöma hur förändringar påverkar studenternas arbetsmiljö samt att hålla sig underrättad om arbetsmiljölagstiftningen
      \item ska finnas representerad i husets HMS-kommitté (Hälsa Miljö och Säkerhetskommitté) och där hålla sig uppdaterad om läget i huset och lyfta frågor som berör studenternas studiemiljö
      \item ansvarar för att se till att sektionens styrelse är informerad om arbetet i HMS-kommittén
      \item ska hålla sektionen informerad om Teknologkårens arbetsmiljöarbete
  \end{tightdashlist}
  \hl{Skyddsombud med likabehandlingsansvar (2)}
  \begin{tightdashlist}
      \item ansvarar för att bevaka Sektionens verksamhet från ett likabehandlingsperspektiv
      \item ska verka för en jämlik studiemiljö
      \item ska uppmärksamma styrelsen på situationer och miljöer som skulle kunna upplevas som kränkande av studenter vid Sektionen
      \item ska hålla Sektionen informerad om TLTH:s policy för likabehandling
      \item ska fungera som kontaktperson och hjälp för medlemmar som anser sig särbehandlade av personer kopplade till högskolan eller programmet
  \end{tightdashlist}
\end{attsatser}

\begin{signatures}{2}
    Med vänliga hälsningar
    \signature{Fanny Månefjord}{Likabehandlingsombud}
    \signature{Lina Samnegård}{Likabehandlingsombud}
\end{signatures}

\end{document}
