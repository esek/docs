\documentclass[../_main/handlingar.tex]{subfiles}

\begin{document}
\verksplanuppf{HT 2017}

\subsubsection*{Styrelsen}
Styrelsen har sedan vårterminsmötet jobbat på mot verksamhetsplanen med mycket fokus på dagligverksamheten och ekonomi, i synnerhet med Cafémästeriet. Hur caféet ska drivas ideellt nästa år har varit en stor fråga vilken för med sig många andra frågor såsom budgetfrågor och utformningen av funktionärsposter. Vi har varit mer eller mindre tvungna att gå över kostnader och intäkter för varje utskott och har gjort en hel del förändringar i budgetförslaget för nästa år.

\subsubsection*{Informationsutskottet}
Informationsutskottet har lyckats att jobba mot de delmål som finns.

Utskottets struktur, införandet av nya poster samt förflyttningen av befintliga bedöms ha fungerat bra.

Vice Kontaktorsposten har god potential för utveckling, införandet av en ny post med vag postbeskrivning behöver bearbetas. Jag anser att posten fyller ett syfte och det har vid ett flertal tillfällen varit skönt att bolla idéer med någon. Främst har Vice Kontaktor varit med för att representera utskottet utåt men det finns potential för att göra mer gemensamt inom utskottet.

Alumniverksamheten har blivit bättre, införandet av de nya posterna bedöms vara bra. Dock har det saknats Alumniansvarig från BME efter halva året vilket var tråkigt.

Jag upplever att informationsspridningen fungerar bra och hoppas att all önskad information når ut till medlemmarna. Min bedömning är att det som fungerar bäst är facebook och affischer.
\subsubsection*{Källarmästeriet}
KM har under året fortsatt hålla upp en stadig marknadsföring i form av E-husets TV-skärmar och evenemang via E-sektionens Facebook-sida. Den förkrossande majoriteten är fortfarande gäster från E-sektionen och D-sektionen samt löst folk som nåtts av eventet på facebook samt \textit{word of mouth}. Försök har gjorts att få hjälp att sprida evenemang inom Sex-kollegiet men har sällan burit frukt. Under vårterminens slut designade Picasso en allmän affisch för KMs verksamhet som användes under de mer allmänna evenemangen som UtEDischot och Regattan, med planer att även någon gång hänga upp dem i de andra husen på Campus, men det är inget som det funnits tid till.

Under året som gått så har det framförallt gjorts sporadiska utbyten med NöjU, däribland Agent-00E samt spelnings av E-lectro banana band under en Hårdrockspub under våren. Inför nollningen planerades framförallt första gillet med NollU för att minimera det förväntade kaoset, vilket fungerade mycket bra. Utöver det har det tyvärr varit rätt tunnsått med samarbete över utskottsgränserna. Samarbete mellan övriga sexmästerier har undersökts men har varit knepigt med gemensamma datum. I stunden som detta skrivs undersöks fortfarande möjligheten med att hålla kårens gemensamma Pubrunda.

Priserna på Klägg höjdes vid årets början för att ha möjlighet att höja kvailtén på utbudet och framförallt erbjuda svenskt kött, men även att ge Krögartrion möjlighet att experimentera mer med menyn. I och med att priset på maten varit fast vid de tidigare 30 kr i åratal och att sektionen med kunder visat stor uppskattning för den nya menyn ser KM inga problem att priset höjdes. Pris på alkohol hålls inom den form alkohollagen är beskaffad och kan inte hållas lägre.

Svinn inom alkohollagret gällande öl och cider ligger inom tidigare års nivåer. I övrigt har KM jobbat vidare med sin Cøl att lämna tillbaka ny alkohol som köpts i bulk men inte sålt, för att undvika att en back med alkohol bara står osåld och går ut.

\subsubsection*{Nolleutskottet}
NollU har behållt en mångfald av aktiviteter.

Förslag från sektionens medlemmar togs in under våren via ett google formulär.

Att förmedla en positiv attityd till nollorna angående studier har gjorts genom phaddrarna.

Arbetsbördan har likt föregående år varit hög men vi har kommit fram till att det går att dela ut den mer jämt.

Något vi jobbat extra mycket på i år är att få alla andra utskott att synas mer under nollningen, vilket vi anser har gått bra. De har synts till och fått presentera sig på många olika aktiviteter.

Den internationella nollningen går att jobba mer på och där har vi inte gjort några större förändringar.

\subsubsection*{Cafémästeriet}
Vi har under hela året jobbat för att minimera mängden svinn i både LED och i CM-förrådet och lyckats ganska bra. Vi har satt upp lappar i förrådet så våra kära medlemmar vet vilken läsk de bör ta av i förstahand. För att minska svinn i caféet har vi varit noga med att se hur mycket av de olika varorna vi har inne samt försökt uppskatta hur mycket som går åt så vi inte köper in för mycket i onödan.

Vi har jobbat med att utöka caféets sortiment och har lyckats med det. Vi har bland annat köpt in olika sorters drycker och olika smaker och varianter av produkterna vi redan har i sortimentet. Vi har även börjat inhandla falafel till de vegetariska salladerna och mackorna, vilket verkar vara väldigt uppskattat.

Cafémästaren har gjort veckorapporter varje vecka och inköpscheferna har jobbat för att få IC-rapporterna klara i tid vilket har gjort att vi har haft ganska bra koll ekonomin under året.

Utvärdering av posterna som finns i utskottet är extra aktuellt inför nästa verksamhetsår då caféet kommer börja drivas helt ideellt. Vi har bland annat skrivit en proposition angående införandet av en ny post i caféet. Dessutom har vi jobbat på att skriva bättre testamenten till våra efterträdare som tydliggör vad dem olika posterna innebär.

\subsubsection*{Förvaltningsutskottet}
Edekvata har fräschats upp under sommaren då väggarna målades om. Då byttes även en del av tavlorna ut som hängde på väggarna och fick upp nya och spännande saker.

Utskottets nya struktur fungerade bra under våren då det fanns en Vice Förvaltningschef. Det var en välbehövd avlastning för Förvaltningschefen, speciellt när denne var ny och ovan vid sitt uppdrag. Efter årskiftet var det dock bara en Hustomte kvar vilket gjorde det svårt att prioritera lokalerna och arbeta löpande med underhållet. Det finns ideer om vad som kan göras med Edekvata men för att det ska hända bör utskottet vara komplett.

\subsubsection*{Studierådet}
Studierådet har satsat mycket på synlighet, för att försöka ge studenter en inblick i ett annars svåråtkomligt utskott. Detta gjordes både genom utskottets egna evenemang samt genom deltagandet i andra utskotts. Pluggkvällarna förändrades under nollningen lite från tidigare år och arrangerades flitigt under denna period men har därefter avtagit. För tillfället saknar SRE representanter från årskurs 4 i båda programmen. Det har även tidigare funnits problem med representanter från de senare årskurserna, detta hoppas vi kunna lösa långsiktigt genom en större grund i de lägre årskurserna likt den vi sett 2017. Svarsfrekvensen för CEQ-enkäterna kunde sett bättre ut, speciellt för E-programmet. Studierådet försöker informera studenter om CEQ-utvärderingarna och har diskuterat metoder för att förbättra statistiken, men behöver fortsätta arbeta på detta. SRE har representanter i programledningarna för Elektroteknik samt Medicin och teknik, och för regelbunden dialog med dessa såväl som studievägledningen.

\subsubsection*{Sexmästeriet}
E6-17 har nu hållit de flesta av evenemangen vi kommer att hålla under vårt år och där har vi självklart strävat efter att eventen ska vara prisvärda och hålla hög kvalité och vi tycker också att vi har lyckats med det.

Det är svårt att fördela evenemangen jämnt över året eftersom det förväntas och är naturligt att många event ligger under nollningen. Funktionärerna i utskottet skulle inte orka med att hålla samma tempo under resten av året som under nollningen, så därför är evenemangen inte helt jämnt fördelade. Därmed inte sagt att ingenting händer när det inte är nollning och E6-17 kommer att hålla event under det som är kvar av hösten.

Vi har arbetat och arbetar kontinuerligt för att ordningen i förrådet (Pump) ska vara bra. De nya hyllorna hjälper till och det finns mer plats sedan de sattes upp. Vi ska självklart lämna över Pump till nästa års sexmästeri i samma eller bättre skick som vi fick ta över det.

Under de event där vi använt sektionens alkoholförråd har vi ansträngt oss för att svinnet i lagret ska vara så lite som möjligt.

\subsubsection*{Nöjesutskottet}
Under årets som varit har nöjesutskottet mål varit att skapa tydligare arbetsuppgifter inom utskottet. Efter att ha genomfört en omformulering på fritidsledares roll tror jag att uppgifter och evenemang kommer att genomföras smidigare till efterföljande utskott.

Höstens stora evenemang utedischot fungerade bra. Hela utskottet var med och hjälpte till under evenemanget.

Husbandet sköter sin verksamhet självständigt och har genomfört det bra.

Utskottet har använt sig av Esek-events för att nå ut med information om aktiviteter och i vissa fall evenemang då det är större event. Det verkar ha fungerat bra och når ut på ett effektivt sätt till sektionens medlemmar.

\subsubsection*{Näringslivsutskottet}
ENU har kontaktat de flesta av företagen som kontaktades i våras. Utskottet har även frågat bl.a. BME-studenter och programledningen för BME om relevanta företag för att försöka få fler evenemang med inriktning medicinteknik. Ett problem är dock att få fram relevanta mejladresser till dessa företag. Prissättningen har inte setts över mer än i våras då den upplevs som bra. Prislistan får dock ibland justeras och anpassas till vissa företag för att sektionen ska kunna hålla en blandning av evenemang, t.ex. med företag med lägre omsättning. ENU-ordförande har fortfarande inte tagit tag i den nya hemsidan men förhoppningsvis ska den ligga uppe i början av nästa år.

\newpage
\begin{signatures}{10}
    \mvh
    \signature{Erik Månsson}{Ordförande}
    \signature{Johan Karlberg}{Kontaktor}
    \signature{Sophia Grimmeiss Grahm}{Förvaltningschef}
    \signature{Daniel Bakic}{Cafémästare}
    \signature{Niklas Gustafson}{Øverphøs}
    \signature{Edvard Carlsson}{SRE-ordförande}
    \signature{Josefine Sandström}{ENU-ordförande}
    \signature{Linnea Sjödahl}{Sexmästare}
    \signature{Markus Rahne}{Krögare}
    \signature{Albin Nyström Eklund}{Entertainer}
\end{signatures}

\end{document}
