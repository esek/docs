\documentclass[10pt]{article}
\usepackage[utf8]{inputenc}
\usepackage[swedish]{babel}
\def\mo{Fredrik Peterson}
\def\ms{Johan Karlberg}
\def\ji{David Uhler Brand}
\def\jii{Malin Lindström}

\def\doctype{Protokoll} %ex. Kallelse, Handlingar, Protkoll
\def\mname{Vårterminsmöte} %ex. styrelsemöte, Vårterminsmöte
\def\mnum{VT/17} %ex S02/16, E1/15, VT/13
\def\date{2017-05-02} %YYYY-MM-DD
\def\docauthor{\ms}

\usepackage{../e-mote}
\usepackage{../../../e-sek}

\begin{document}
\showsignfoot

\heading{{\doctype} för {\mname} {\mnum}}

%\naun{}{} %närvarane under
%\nati{}{} %närvarande till och med
%\nafr{}{} %närvarande från och med
\section*{Närvarande}
\subsection*{Styrelsen}
\begin{narvarolista}
  \nv{Ordförande}{Erik Månsson}{E14}{}
  \nv{Kontaktor}{Johan Karlberg}{E14}{}
  %\nv{Förvaltningschef}{Sophia Grimmeiss Grahm}{BME14}{}
  \nv{Cafémästare}{Daniel Bakic}{E15}{}
  \nv{Øverphøs}{Niklas Gustafson}{E15}{\naun{10}{14}, \S22H - \S24}
  \nv{SRE-ordförande}{Pontus Landgren}{E14}{\naun{1}{17}, \S20 - \S24}
  \nv{ENU-ordförande}{Josefine Sandström}{E14}{}
  \nv{Sexmästare}{Linnea Sjödahl}{BME15}{\nafr{20}{}}
  \nv{Krögare}{Markus Rahne}{BME14}{\nafr{21}{}}
  \nv{Entertainer}{Albin Nyström Eklund}{BME16}{}
\end{narvarolista}

\subsection*{Medlemmar}
\begin{narvarolista}
\nv{}{Anders Nilsson}{E13}{}
\nv{}{Andreas Bennström}{BME16}{}
\nv{}{Anton Jigsved}{BME16}{}
\nv{}{Axel Sondh}{E15}{}
\nv{}{Björn Johansson}{E15}{}
\nv{}{Christian Benson}{E14}{}
\nv{}{David Uhler Brand}{E14}{}
\nv{}{Daniel Johansson}{E14}{\nati{20}{E}}
\nv{}{Elin Johansson}{BME16}{}
\nv{}{Elisabeth Klint}{BME14}{}
\nv{}{Elisabeth Pongratz}{E14}{\nati{22}{A}}
\nv{}{Ester Randahl}{BME14}{}
\nv{}{Fanny Månefjord}{BME16}{}
\nv{}{Filip Kronström}{E15}{\nati{22}{A}}
\nv{}{Henrik Ramström}{E16}{}
\nv{}{Henrik Stålbom}{E16}{}
\nv{}{Hjalmar Tingberg}{BME16}{}
\nv{}{Isabella Hansen}{E16}{\nati{22}{A}}
\nv{}{Jennie Karlsson}{E16}{\nati{22}{A}}
\nv{}{Lina Samnegård}{BME16}{}
\nv{}{Linnea Wenäll}{BME16}{}
\nv{}{Louise Wedberg}{BME15}{}
\nv{}{Lukas Mattsson}{BME16}{}
\nv{}{Madeleine Arkenius}{BME15}{}
\nv{}{Magnus Lundh}{E15}{}
\nv{}{Malin Heyden}{E16}{\nati{22}{A}}
\nv{}{Malin Lindström}{BME14}{}
\nv{}{Matilda Dahlström}{BME16}{}
\nv{}{Oscar Uggla}{E15}{\nati{20}{E}}
\nv{}{Philip Johansson}{E16}{\nati{22}{A}}
\nv{}{Rasmus Sobel}{BME16}{}
\nv{}{Saga Juniwik}{E16}{}
\nv{}{Sanna Nordberg}{BME16}{}
\nv{}{Sofia Rokkones}{BME15}{}
\end{narvarolista}

\subsection*{Ständigt adjungerande}
\begin{narvarolista}
\nv{Talman}{Fredrik Peterson}{E14}{}
\nv{Revisor}{Jesper Ek}{E14}{\nati{20}{E}}
%\nv{Post}{Namn}{Klass}{}
\end{narvarolista}

\begin{comment}
\subsection*{Adjungerande}
\begin{narvarolista}
%\nv{Post}{Namn}{Klass}{}
\end{narvarolista}
\end{comment}

\newpage
\section*{Protokoll}
\begin{paragrafer}
\p{1}{TaFMÖ}{}
Talman {\mo} förklarade mötet öppnat 17:22.

\p{2}{Val av mötesordförande}{}
Talman {\mo} valdes.

\p{3}{Val av mötessekreterare}{}
Kontaktor {\ms} valdes.

\p{4}{Godkännande av tid och sätt}{}
Tid och sätt godkändes.

\p{5}{Val av två justeringspersoner}{}
\valavj

\p{6}{Adjungeringar}{}
\ingaadj

\p{7}{Godkännande av dagordningen}{}
%Föredragningslistan godkändes.
Fredrik Peterson \ypa att lägga till ``Utökning av antal sökande till posten co-phøsare'' under \S10.

Föredragningslistan godkändes med yrkandet.
%Föredragningslistan godkändes med samtliga yrkanden.

\p{8}{Föregående sektionsmötesprotokoll}{}
\latillprot{VM/16}

\p{9}{Meddelanden}{}
Ingen hade något att meddela.

\p{10}{Beslutsuppföljning}{}
Emil Harvig var inte närvarande under mötet. Fredrik \ypa skjuta upp \emph{Fanbärare behöver längre påle} till HT/17.

\Mbaby

Martin Gemborn Nilsson \ypa skjuta upp beslutsuppföljningen för \emph{Införandet av arbetskläder för utlåning till funktionärer} till HT/17 men med styrelsen som ansvarig.

\Mbaby

Erik Månsson \ypa stryka \emph{Inköp av ny huvudswitch} från beslutsuppföljningen. Total kostnad uppgick till \SI{1690}{kr} och budget var \SI{1700}{kr}.

\Mbaby

Erik Månsson \ypa stryka \emph{Inköp av nya datorer} från beslutsuppföljningen. Total kostnad uppgick till \SI{10990}{kr} och budget var \SI{11000}{kr}.

\Mbaby

Erik Månsson \ypa skjuta upp beslutsuppföljningen av \emph{Uppdatering av övervakningspolicyn} till HT/17.

\Mbaby

Fredirk Peterson \ypa stryka \emph{Renovering och ombyggnad av HK och BD} från beslutsuppföljningen. Total kostnad uppgick till \SI{28348}{kr} och budget var \SI{30000}{kr}.

\Mbaby

Anders Nilsson \ypa stryka \emph{Uppfräschning av gamla arkivet} från beslutsuppföljningen. Kostnaden var under budgeten som var satt till \SI{10000}{kr}.

\Mbaby

Christian Benson \ypa skjuta upp \emph{Utökning av antal sökande till posten co-phøsare} till HT/17.

\Mbaby

\p{11}{Utskottsrapporter}{}
Möjligheten att ställa frågor till styrelsen och valberedningen gavs.

\p{12}{Uppföljning av verksamhetsplan}{}
Möjligheten att ställa frågor till styrelsen och valberedningen gavs.

\p{13}{Ekonomisk rapport}{}
Sektionens ordförande Erik Månsson förmedlade sektionens förvaltningschef, Sophia Grimmeiss Grahms, rapport för sektionens ekonomi.

%\textbf{\Mba lägga den ekonomiska rapporten till handlingarna.}

\p{14}{Val}{}
\begin{paragrafer}
    \subp{A}{Val av funktionärer}{}

%Mötet vakantsatte allt i klump.

\textbf{\Mba vakantsätta posten Teknikfokusansvarig.}\par

\textbf{\Mba vakantsätta posten Redaktör.}\par

\textbf{\Mba vakantsätta posten Teknokrat.}\par

\textbf{\Mba vakantsätta posten Fritidsledare.}\par

\textbf{\Mba vakantsätta posten Karnevalsmalaj.}\par

\textbf{\Mba vakantsätta posten Stridsrop.}\par

\textbf{\Mba vakantsätta posten Umph-meister.}\par

\textbf{\Mba vakantsätta posten Revisorsuppleant.}\par
\end{paragrafer}
\end{paragrafer}
	\begin{paragrafer} \item[] %quick and dirty
	\begin{paragrafer}
    \subp{B}{Val av hedersmedlemmar}{}
    Erik Månsson \ypa välja Ulla Andersson till hedersmedlem.

    \textbf{Mötet valde Ulla Andersson till hedersmedlem.}
\end{paragrafer}
\p{15}{Verksamhetsberättelser för 2016}{}
%Styrelsen 2016 och Valberedningen 2016 gav sina verksamhetsberättelser för 2016.
\textbf{\Mba lägga verksamhetsberättelsen för 2016 till handlingarna. Beslutet togs på \S20}

\p{16}{Bokslut för 2016}{}
Anders Nilsson prestenterade bokslutet för 2016.
%\textbf{\Mba lägga bokslutet 2016 till handlingarna.}

\p{17}{Revisionsberättelse för 2016}{}
Jesper Ek presenterade revisionsberättelsen för 2016.
%\textbf{\Mba lägga revisionsberättelsen 2016 till handlingarna.}
Jesper Ek tog tillbaka sitt yrkande om
\begin{attsatser}
    \att resultat- och balansräkning fastställes,
    \att styrelsens förslag till resultatsdisposition tages, samt
    \att styrelsen för 2016 beviljas ansvarsfrihet.
\end{attsatser}

\p{18}{Styrelsens förslag till resultatdisposition}{}
Anders Nilsson presenterade förslaget till resultatdisposition.

Jesper Ek yrkade på
\begin{attsatser}
    \att resultat- och balansräkning fastställes, samt
    \att styrelsens förslag till resultatsdisposition tages.
\end{attsatser}

\Mbaby

\textbf{\Mba godkänna resultatdispositionen}

\p{19}{Uttag ur sektionens fonder sedan förra terminsmötet}{}
Erik Månsson berättade om uttagen ur sektionens fonder sedan förra terminsmötet.

\p{20}{Frågan om ansvarsfrihet för 2016}{}
    \begin{paragrafer}
        \subp{A}{Funktionärer}{}
        \textbf{\Mba finna funktionärerna 2016 ansvarsfria.}
        \subp{B}{Utskott}{}
        \textbf{\Mba finna utskotten 2016 ansvarsfria.}
        \subp{C}{Styrelse}{}
        \textbf{\Mba finna styrelsen 2016 ansvarsfria.}
        \subp{D}{Revisorer}{}
        \textbf{\Mba finna revisorerna 2016 ansvarsfria.}
        \subp{E}{Valberedning}{}
        \textbf{\Mba finna valberedningen 2016 ansvarsfria.}

        Erik Månsson \ypa ta 45 minuters matpaus.

        Fredrik Peterson \ypa ta 30 minuters matpaus.

        Rasmus Sobel \ypa ta 35 minuters matpaus.

        Josefine Sandström \ypa ta 40 minuters matpaus.

        Fredrik Peterson jämkade sig med Rasmus Sobel.

        \ji  \ypa ta 37,5 minuters matpaus.

        Erik Månsson begär sluten votering.

        Erik Månsson \ypa sträck i debatten.

        \Mbaby.

        Mötet bestlutade att ajournera mötet i 35 minuter.

        Mötet återupptogs 18:56.
    \end{paragrafer}

    \p{21}{Behandling av motioner}{}
        \begin{paragrafer}
          \subp{A}{Make E-lektro banana band great again}{}
          Lukas Mattsson presenterade motionen.

          Erik presenterade styrelsens svar.

          Anders Nilsson \ypa sektionen ska köpa in en uppsättning banandräkter till bandet som uppträdesutstyrsel.

          Lukas Mattsson jämkade sig med styrelsens förslag.

          Mötet beslutade att tillägga Anders förslag till motionen.

          \textbf{\Mba bifalla motionen, med styrelsens ändringsyrkande, i sin helhet.}

          Anders Nilsson \ypa sätta budget till 3500kr vilket ska räcka till 7 dräkter.

          Filip Kronström \ypa det ska vara klädseln enligt
          \href{http://www.ebay.com/itm/Willy-Adult-Unisex-Banana-Suits-Yellow-Costume-Light-Fruit-Party-Fancy-Dress-/152488835359?hash=item23810af91f:g:rA0AAOSwB-1Y2mCV}{\textit{(länk)}}

          Filip Kronström drog tillbaka sitt yrkande.

          Erik Månsson yrkade på att kostnaderna ska belasta utrustningsfonden.

          Anders Nilsson jämkade sig med Erik Månsson.

          Anders Nilsson yrkade på att beslutsuppföljning HT/17 med relevant ansvarig.

          Erik Månsson yrkade på att Albin Nyström Eklund (NöjU) ska vara ansvarig.

          Anders Nilsson jämkade sig med Erik Månsson.

          \Mba bifalla yrkandena om att sätta budget till \SI{3500}{kr} med Albin Nyström Eklund som ansvarig samt att det ska belasta utrustningsfonden.
          \subp{B}{BLED-café}{}
          Rasmus Sobel lyfter motionen.

          Erik Månsson presenterade styrelsens svar.

          Mötet beslutade att avslå motionen i sin helhet.
          \subp{C}{Förbättrad förvaring för sektionens lager}{}
          Sanna Nordberg lyfter motionen.

          Erik Månsson presenterade styrelsens svar.

          \Mba bifalla motionen i sin helhet.
          \subp{D}{Ansvarig för uppdatering av examenstavlor}{}
          Anders Nilsson lyfter motionen.

          Erik Månsson presenterade styrelsens svar.

          \Mba bifalla motionen i sin helhet.
          \subp{E}{Representationsklädsel åt Inspektorn}{}
          Anders Nilsson lyfter motionen.

          Erik Månsson presenterade styrelsens svar.

          \Mba bifalla motionen i sin helhet.
      \end{paragrafer}


      \p{22}{Behandling av propositioner}{}
          \begin{paragrafer}
            \subp{A}{Öppna kravprofiler för valberedning}{}
            Erik Månsson presenterade propositionen.

            \Mba bifalla propositionen i sin helhet.

            Pontus Landgren \ypa på 5 minuter paus efter propositionen.

            Mötet ajounerades i 5 minuter.

            Mötet återupptogs 19:57
            \subp{B}{Flytta policybeslut ``Närvaro vid Sektionsmöte'' till reglementet och uppdatera valmetoden}{}
            Erik Månsson presenterade propositionen.

            \Mba bifalla propositionen i sin helhet.
            \subp{C}{Äskning av pengar till mentorsprogram}{}
            Erik Månsson presenterade propositionen med bakgrund.

            \Mba bifalla propositionen i sin helhet.
            \subp{D}{Uppgradering av ljudsystem i Edekvata}{}
            Pontus Landgren presenterade propositionen.

            \Mba bifalla propositionen i sin helhet.
            \subp{E}{Inköp av PA-toppar}{}
            Pontus Landgren presenterade propositionen.

            \Mba bifalla propositionen i sin helhet.
            \subp{F}{Införande av projektfunktionärer}{}
            Erik Månsson presenterade propositionen.

            Erik Månsson yrkar på att ändra den sista punkten i att-satsen till:
            ``Har en mandatperiod som bestäms vid valtillfället, och är maximalt ett år lång.''

            Styrelsen jämkade sig med Erik Månssons yrkande.

            \Mba bifalla propositionen i sin helhet med ändringsyrkandet.
            \subp{G}{Borttagning av giltig terminsräkning i reglementet}{}
            Erik Månsson presenterade propositionen sittandes.

            \Mba bifalla propositionen i sin helhet.
            \subp{H}{Ändring av hur Sektionen väljer Phøs}{}
            Mötet ajounerades i 5 minuter.

            Niklas Gustafson presenterade propositionen.

            \Mba bifalla propositionen i sin helhet.
        \end{paragrafer}

\p{23}{Övrigt}{}
Anders Nilsson informerade om jubileumsveckan som just nu är igång.
\begin{itemize}
  \item På torsdag är det målning av cylindern, ätas korv och spelas musik.
  \item På fredagen är det gille, det är asnice.
  \item På lördag är det bal med eftersläpp.
\end{itemize}

Erik Månsson uppmanar folk till att söka mentor.

Daniel Bakic uppmanar alla phaddrar att de är välkomna att jobba i LED.

Anton Jigsved meddelade att de kommer att vara bugg kl 3 på sporta med E.

Erik Månsson hälsar från D-sektionen som också har möte.
\p{24}{TaFMA}{}
Talman {\mo} förklarade mötet avslutat 20:59.

\end{paragrafer}

%\newpage
\hidesignfoot
\begin{signatures}{4}
\signature{\mo}{Mötesordförande}
\signature{\ms}{Mötessekreterare}
\signature{\ji}{Justerare}
\signature{\jii}{Justerare}
\end{signatures}
\end{document}
