\documentclass[10pt]{article}
\usepackage[utf8]{inputenc}
\usepackage[swedish]{babel}

\def\mo{Erik Månsson}
\def\ms{Johan Karlberg}
\def\ji{Markus Rahne}
%\def\jii{}

\def\doctype{Protokoll} %ex. Kallelse, Handlingar, Protkoll
\def\mname{styrelsemöte} %ex. styrelsemöte, Vårterminsmöte
\def\mnum{S21/17} %ex S02/16, E1/15, VT/13
\def\date{2017-09-28} %YYYY-MM-DD
\def\docauthor{\ms}

\usepackage{../e-mote}
\usepackage{../../../e-sek}

\begin{document}
\showsignfoot

\heading{{\doctype} för {\mname} {\mnum}}

%\naun{}{} %närvarane under
%\nati{} %närvarande till och med
%\nafr{} %närvarande från och med
\section*{Närvarande}
\subsection*{Styrelsen}
\begin{narvarolista}
\nv{Ordförande}{Erik Månsson}{E14}{}
\nv{Kontaktor}{Johan Karlberg}{E14}{}
\nv{Förvaltningschef}{Sophia Grimmeiss Grahm}{BME14}{}
\nv{Cafémästare}{Daniel Bakic}{E15}{\nafr{11}}
\nv{Øverphøs}{Niklas Gustafson}{E15}{\nafr{9}}
\nv{SRE-ordförande}{Edvard Carlsson}{E16}{}
\nv{ENU-ordförande}{Josefine Sandström}{E14}{}
\nv{Sexmästare}{Linnea Sjödahl}{BME15}{}
\nv{Krögare}{Markus Rahne}{BME14}{}
%\nv{Entertainer}{Albin Nyström Eklund}{BME16}{}
\end{narvarolista}

\subsection*{Ständigt adjungerande}
\begin{narvarolista}
%\nv{Kårordförande}{Linus Hammarlund}{}{}
%\nv{Kårrepresentant}{Caroline Svensson}{}{}
%\nv{Kårrepresentant}{Agnes Sörliden}{}{}
%\nv{Valberedningens ordförande}{Elin Magnusson}{}{}
\nv{Skattmästare}{Olle Oswald}{}{}
%\nv{Kårrepresentant}{Daniel Damberg}{}{}
%\nv{Kårrepresentant}{John Alvén}{}{}
%\nv{Talman}{Fredrik Peterson}{E14}{}
%\nv{Elektras Ordförande}{Elisabeth Pongratz}{}{}
%\nv{Inspektor}{Monica Almqvist}{}{}
\end{narvarolista}

\begin{comment}
\subsection*{Adjungerande}
\begin{narvarolista}
%\nv{Post}{Namn}{Klass}{}
\end{narvarolista}
\end{comment}

\section*{Protokoll}
\begin{paragrafer}
\p{1}{OFMÖ}{\bes}
Ordförande {\mo} förklarade mötet öppnat 12:16.

\p{2}{Val av mötesordförande}{\bes}
{\valavmo}

\p{3}{Val av mötessekreterare}{\bes}
{\valavms}

\p{4}{Val av justeringsperson}{\bes}
{\valavj}

\p{5}{Godkännande av tid och sätt}{\bes}
{\tosg}

\p{6}{Adjungeringar}{\bes}
{\ingaadj}

%Förnamn Efternamn adjungerades

\p{7}{Godkännande av dagordningen}{\bes}
%Dagordningen godkändes.
Edvard \ypa lägga till \S11 ``Expo''.

Erik \ypa lägga till \S12 ``Drivandet av LED''

%Föredragningslistan godkändes med yrkandet.
Föredragningslistan godkändes med samtliga yrkanden.
\p{8}{Föregående mötesprotokoll}{\bes}
\latillprot{S20/17}
\p{9}{Fyllnadsval och entledigande av funktionärer}{\bes}
\begin{fyllnadsval} %"Inga fyllnadsval." fylls i automatiskt
\entl{Johannes Koch}{Arkivarie}
\fval{Fredrik Hammar}{Arkivarie}
\fval{Elisabeth Pongratz}{Teknikfokusansvarig}
\end{fyllnadsval}

\p{10}{Rapporter}{}
\begin{paragrafer}
\subp{A}{Hur mår alla?}{\info}
Punkten protokollfördes ej.
\subp{B}{Utskottsrapporter}{\info}
NollU är inne på nollningens enligt Kåren sista vecka, nu är det mest efterarbete.

Ekiperingsexperterna sålde ordensband på Gasquen, det gick bra.

Edvard har varit ute och pratat med E och BME i årskurs 1. Det verkar vara bra engagemang bland de som började nu. Idag är det utbytesmingel.

KM fortsätter sin semester och planerera lite teman som de ska köra under hösten. Julgillet är spikat till 9 december.

E6 har sovit denna veckan. Gasquen gick jättebra. De ska ha ett möte för att diskutera hur arbetet under nollningen har gått.

I InfU fortsätter utskottet att jobba självständigt. Fotograferna fotar.

\subp{C}{Ekonomisk rapport}{\info}
Sophia meddelade att ekonomin mår bra.
\subp{D}{Kåren informerar}{\info}
Genom Sophia meddelar Kåren att Arkad söker värdar, de behöver en ansvarig för Vårterminsnollningen. De har en engelsktalande i Fullmäktige, alla uppmanas att skriva på engelska om man skickar något. Val till Fullmäktige öppnar på söndag.
\end{paragrafer}

\p{11}{Expo}{\dis}
Edvard har fått förhinder den 9/10. Tillsammans med andra förhinder så föreslår mötet att expot skall äga rum 12/10 istället.

Expot börjar klockan 10.00, ansvariga får förbereda en liten stund innan så att de är redo att ta emot folk klockan 10. Expot varar till 13 och när folk börjar att gå därifrån så börjar vi städa.
\p{12}{Drivandet av LED}{\dis}
LED har haft lite paus. Någon som har koll på verksamheten måste vara där, vilket är problematiskt ibland. De planerar så att överlämningen till nästa år blir bättre.

De tänker försöka att inför efftermiddagsdioder. Detta stider mot reglementet men det anses vara okej eftersom LED inte kan drivas så som det var tänkt när reglementet skrevs.

Erik tyckte att vi ska göra allt för att de som jobbar i LED inte ska känna att det är en för tung belastning.

Det är lunchen som drar in mest pengar.

Mötet diskuterade huruvida LED möjligtvis bara ska ta kortbetalning. Det tar iallafall 30 minuter per dag att räkna pengarna, men Daniel tycker inte att det är en för stor belastning.

Mötet diskuterade ämnet vidare.
\p{13}{Nästa styrelsemöte}{\bes}
{\Mba} nästa styrelsemöte ska äga rum 2017-10-05 12:10 i E:1124.

\p{14}{Beslutsuppföljning}{\bes}
{\Ibfu}

\p{15}{Övrigt}{\dis}
Mötet diskuterad resan till Chalmers.

Nu när Julgillet är spikat så tänkte Markus gå ut med inbjudningar till gamla krögare. Mötet diskuterade huruvida de behöver söka extra tillstånd om de vill utöka sitt stadigvarande tillstånd med lista eller ej.

\p{16}{OFMA}{\bes}
{\mo} förklarade mötet avslutat 12:59.

\end{paragrafer}

%\newpage
\hidesignfoot
\begin{signatures}{3}
\signature{\mo}{Mötesordförande}
\signature{\ms}{Mötessekreterare}
\signature{\ji}{Justerare}
\end{signatures}
\end{document}
