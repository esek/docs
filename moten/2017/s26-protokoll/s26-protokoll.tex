\documentclass[10pt]{article}
\usepackage[utf8]{inputenc}
\usepackage[swedish]{babel}

\def\mo{Erik Månsson}
\def\ms{Johan Karlberg}
\def\ji{Sophia Grimmeiss Grahm}
%\def\jii{}

\def\doctype{Protokoll} %ex. Kallelse, Handlingar, Protkoll
\def\mname{styrelsemöte} %ex. styrelsemöte, Vårterminsmöte
\def\mnum{S26/17} %ex S02/16, E1/15, VT/13
\def\date{2017-11-09} %YYYY-MM-DD
\def\docauthor{\ms}

\usepackage{../e-mote}
\usepackage{../../../e-sek}

\begin{document}
\showsignfoot

\heading{{\doctype} för {\mname} {\mnum}}

%\naun{}{} %närvarane under
%\nati{} %närvarande till och med
%\nafr{} %närvarande från och med
\section*{Närvarande}
\subsection*{Styrelsen}
\begin{narvarolista}
\nv{Ordförande}{Erik Månsson}{E14}{}
\nv{Kontaktor}{Johan Karlberg}{E14}{}
\nv{Förvaltningschef}{Sophia Grimmeiss Grahm}{BME14}{}
\nv{Cafémästare}{Daniel Bakic}{E15}{}
\nv{Øverphøs}{Niklas Gustafson}{E15}{\nafr{10A}}
\nv{SRE-ordförande}{Edvard Carlsson}{E16}{}
\nv{ENU-ordförande}{Josefine Sandström}{E14}{}
\nv{Sexmästare}{Linnea Sjödahl}{BME15}{}
%\nv{Krögare}{Markus Rahne}{BME14}{}
%\nv{Entertainer}{Albin Nyström Eklund}{BME16}{}
\end{narvarolista}

\subsection*{Ständigt adjungerande}
\begin{narvarolista}
%\nv{Kårordförande}{Linus Hammarlund}{}{}
%\nv{Kårrepresentant}{Anders Nilsson}{}{}
%\nv{Kårrepresentant}{Caroline Svensson}{}{}
\nv{Kårrepresentant}{Agnes Sörliden}{}{}
%\nv{Valberedningens ordförande}{Elin Magnusson}{}{}
%\nv{Skattmästare}{Olle Oswald}{}{}
%\nv{Kårrepresentant}{Daniel Damberg}{}{}
%\nv{Kårrepresentant}{John Alvén}{}{}
%\nv{Talman}{Fredrik Peterson}{E14}{}
%\nv{Elektras Ordförande}{Elisabeth Pongratz}{}{}
%\nv{Inspektor}{Monica Almqvist}{}{}
\end{narvarolista}

\begin{comment}
\subsection*{Adjungerande}
\begin{narvarolista}
%\nv{Post}{Namn}{Klass}{}
\end{narvarolista}
\end{comment}

\section*{Protokoll}
\begin{paragrafer}
\p{1}{OFMÖ}{\bes}
Ordförande {\mo} förklarade mötet öppnat 12:17.

\p{2}{Val av mötesordförande}{\bes}
{\valavmo}

\p{3}{Val av mötessekreterare}{\bes}
{\valavms}

\p{4}{Val av justeringsperson}{\bes}
{\valavj}

\p{5}{Godkännande av tid och sätt}{\bes}
{\tosg}

\p{6}{Adjungeringar}{\bes}
{\ingaadj}

%Förnamn Efternamn adjungerades

\p{7}{Godkännande av dagordningen}{\bes}
%Dagordningen godkändes.
Daniel \ypa lägga till \S13 ``Sponsring till LED''.

Föredragningslistan godkändes med yrkandet.
%Föredragningslistan godkändes med samtliga yrkanden.

\p{8}{Föregående mötesprotokoll}{\bes}
\latillprot{S24/17}

\latillprot{S25/17}
%\ingaprot

\p{9}{Fyllnadsval och entledigande av funktionärer}{\bes}
\begin{fyllnadsval} %"Inga fyllnadsval." fylls i automatiskt
\fval{Eltayeb Bayomi}{Diod}
\fval{Johan Sievert Lindeskog}{Diod}
\fval{Viktor Drakfeldt}{Diod}
\fval{Anton Jigsved}{Diod}
\fval{Fabian Sondh}{Diod}
\fval{Isa Clementsson}{Diod}
\fval{Albin Pålsson}{Diod}
\fval{Adla Jebara}{Diod}
\fval{Jennifer Ramkull}{Diod}
\fval{August Millqvist}{Diod}
\fval{Björn Johnsson}{Diod}
\fval{Adam Rosandell}{Diod}
\fval{Malin Heyden}{Diod}
%\entl{Namn}{Post}
\end{fyllnadsval}

\p{10}{Rapporter}{}
\begin{paragrafer}
\subp{A}{Hur mår alla?}{\info}
Punkten protokollfördes ej.
\subp{B}{Utskottsrapporter}{\info}
I veckan har CM börjat sälja kaffekort, vilket är najs! Ulla har kommit in i rutinerna igen och verkar trivas ganska bra vilket är jättebra. Daniel har börjat kolla runt lite på någon form av mixer att köpa in till att strimla grönsaker. Annars flyter verksamheten på bra.

Sophia och Filip (Hustomte) har funderat på hur vi kan visa upp FlyING-pokalen på bästa sätt. E-sektionen har den ju varje år så det känns tråkigt att den bara får stå i arkivet och inte är tillgänglig för allmän beundran.

Sophia har varit på VOK-möte.

Ölprovning står för dörren, recept har spikats och mat ska inhandlas. Pubrundan har också spikats till torsdag 30/11.

Lunch med ingenjör rullar fortfarande på, anmälan är öppen. Josefine har marknadsfört åt några företag, bland annat till ALTENs case kväll. Josefine har även skrivit ihop ett kontrakt med Adsensus som ville byta en utbildning mot marknadsföring på E-sektionen. Den nya dealen innebär att Adsensus håller i en lunchföreläsning (och betalar för de kostnader som uppstår med den) mot att de får publicera två inlägg på Facebook.

E6 har inte gjort så mycket denna veckan eftersom de inte har så många event under denna perioden. De ska ha möte imorgon (fredag) och spika de event de vill hålla i höst.

Det är snart dags för SRE att censurera CEQ rapporterna, de på papper har redan kommit. Edvard håller läsperiodens första möte imorgon, då ska de diskutera vad de ska göra resten av året och välkomna två nya årskursansvariga.

NollU skall ta tag i utvärderingen av nollningen.
\subp{C}{Ekonomisk rapport}{\info}
Sophia meddelade att ekonomin mår bra. De har inte hunnit att sitta jättemycket med ekonomin i veckan. Olle skall nog sitta lite nu.
\newpage
\subp{D}{Kåren informerar}{\info}
Agnes meddelade att det är en full styrelse på kåren nu. De skall ha teambuilding och omstrukturera lite.

FM-val är öppet. 21 från E har röstat. Sektionerna och alla som röstar har chans att vinna fina priser.

Det är nu sista veckan att söka internationell-fadder.
\end{paragrafer}

\p{11}{Inköp av ny förstärkare till HK}{\dis}
Erik tycker att förstärkaren i HK börjar att bli tråkig. Bara en kanal fungerar.

Mötet diskuterade detta.

Erik \ypa köpa in ny förstärkare med budget \SI{5000}{kr}, kostnaden skall belasta dispositionsfonden och detta läggs på beslutsuppföljning och redovisas S28.

\Mbaby
\p{12}{Inköp av nya iPads}{\dis}
iPadsen strular iblad. Den i LED var dålig under förra veckan.

Erik \ypa köpa in ny iPad med budget \SI{4500}{kr}, kostnaden skall belasta dispositionsfonden och detta läggs på beslutsuppföljning och redovisas S28.

\Mbaby
\p{13}{Sponsring till LED}{\dis}
Daniel har två dioder som ska exjobba på Cybercom som är ett konsultföretag. Dioderna är deras studentambassadörer och har snackat med dem om sponsring till LED. Så gör vi mycket reklam för dem kan vi få massa najs grejer. Något man kan fundera på om vi vill göra och vad man isåfall skulle vilja sponsras med.

Pappersmuggar är något som är diskuterades. Det är bra marknadsföring.

Ibland får LED kampanjmuggar.

Alternativt köper de reklamplatser.

Mötet anser att det beror på hur mycket (pengar) de vill sponsra.
\p{14}{Nästa styrelsemöte}{\bes}
{\Mba} nästa styrelsemöte ska äga rum 2017-11-23 12:10 i E:1124.

\p{15}{Beslutsuppföljning}{\bes}
{\Ibfu}

\p{16}{Övrigt}{\dis}
Bjud in funktionärer till funkistack.
\p{17}{OFMA}{\bes}
{\mo} förklarade mötet avslutat 12:52.

\end{paragrafer}

\newpage
\hidesignfoot
\begin{signatures}{3}
\signature{\mo}{Mötesordförande}
\signature{\ms}{Mötessekreterare}
\signature{\ji}{Justerare}
\end{signatures}
\end{document}
