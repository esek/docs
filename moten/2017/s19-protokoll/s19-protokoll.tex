\documentclass[10pt]{article}
\usepackage[utf8]{inputenc}
\usepackage[swedish]{babel}

\def\mo{Erik Månsson}
\def\ms{Johan Karlberg}
\def\ji{Josefine Sandström}
%\def\jii{}

\def\doctype{Protokoll} %ex. Kallelse, Handlingar, Protkoll
\def\mname{styrelsemöte} %ex. styrelsemöte, Vårterminsmöte
\def\mnum{S19/17} %ex S02/16, E1/15, VT/13
\def\date{2017-09-14} %YYYY-MM-DD
\def\docauthor{\ms}

\usepackage{../e-mote}
\usepackage{../../../e-sek}

\begin{document}
\showsignfoot

\heading{{\doctype} för {\mname} {\mnum}}

%\naun{}{} %närvarane under
%\nati{} %närvarande till och med
%\nafr{} %närvarande från och med
\section*{Närvarande}
\subsection*{Styrelsen}
\begin{narvarolista}
\nv{Ordförande}{Erik Månsson}{E14}{}
\nv{Kontaktor}{Johan Karlberg}{E14}{}
\nv{Förvaltningschef}{Sophia Grimmeiss Grahm}{BME14}{}
%\nv{Cafémästare}{Daniel Bakic}{E15}{}
\nv{Øverphøs}{Niklas Gustafson}{E15}{\nafr{6}}
\nv{SRE-ordförande}{Edvard Carlsson}{E16}{}
\nv{ENU-ordförande}{Josefine Sandström}{E14}{}
%\nv{Sexmästare}{Linnea Sjödahl}{BME15}{}
\nv{Krögare}{Markus Rahne}{BME14}{\nafr{6}}
\nv{Entertainer}{Albin Nyström Eklund}{BME16}{}
\end{narvarolista}

\subsection*{Ständigt adjungerande}
\begin{narvarolista}
%\nv{Kårordförande}{Linus Hammarlund}{}{}
%\nv{Kårrepresentant}{Anders Nilsson}{}{}
%\nv{Kårrepresentant}{Caroline Svensson}{}{}
\nv{Kårrepresentant}{Agnes Sörliden}{}{}
%\nv{Valberedningens ordförande}{Elin Magnusson}{}{}
%\nv{Skattmästare}{Olle Oswald}{}{}
%\nv{Kårrepresentant}{Daniel Damberg}{}{}
%\nv{Kårrepresentant}{John Alvén}{}{}
%\nv{Talman}{Fredrik Peterson}{E14}{}
%\nv{Elektras Ordförande}{Elisabeth Pongratz}{}{}
%\nv{Inspektor}{Monica Almqvist}{}{}
\end{narvarolista}

\subsection*{Adjungerande}
\begin{narvarolista}
\nv{Inköps- och lagerchef}{Johanna Wikström}{E16}{}
\nv{Inköps- och lagerchef}{Jessica Kågeman}{BME16}{}
\nv{Inköps- och lagerchef}{Elin Johansson}{BME16}{}
\nv{Inköps- och lagerchef}{Filip Johansson}{E16}{}

\end{narvarolista}

\section*{Protokoll}
\begin{paragrafer}
\p{1}{OFMÖ}{\bes}
Ordförande {\mo} förklarade mötet öppnat 12:15.

\p{2}{Val av mötesordförande}{\bes}
{\valavmo}

\p{3}{Val av mötessekreterare}{\bes}
{\valavms}

\p{4}{Val av justeringsperson}{\bes}
{\valavj}

\p{5}{Godkännande av tid och sätt}{\bes}
{\tosg}

\p{6}{Adjungeringar}{\bes}
Johanna Wikström adjungerades.

Jessica Kågeman adjungerades.

Elin Johansson adjungerades.

Filip Johansson adjungerades.
\p{7}{Godkännande av dagordningen}{\bes}
Dagordningen godkändes.
%Föredragningslistan godkändes med yrkandet.
%Föredragningslistan godkändes med samtliga yrkanden.

\p{8}{Föregående mötesprotokoll}{\bes}
\latillprot{S18/17}
%\ingaprot

\p{9}{Fyllnadsval och entledigande av funktionärer}{\bes}
\begin{fyllnadsval} %"Inga fyllnadsval." fylls i automatiskt
\fval{Lina Samnegård}{Diod}
\fval{Gabriela Medina}{Diod}
\fval{Isa Clementsson}{Diod}
\fval{Johan Wendt}{Diod}
\fval{Adem Saran}{Diod}
\fval{Mansoor Ashrati}{Diod}
\fval{Anton Jigsved}{Diod}
\fval{Max Mauritsson}{Diod}
\fval{Fanny Månefjord}{Diod}
\fval{Paulina Sager}{Diod}
\fval{Viktor Drakfelt}{Diod}
\fval{Johan Sievert Lindeskog}{Diod}
\fval{Amanda Nilsson}{Diod}
\end{fyllnadsval}

\p{10}{Rapporter}{}
\begin{paragrafer}
\subp{A}{Hur mår alla?}{\info}
Punkten protokollfördes ej.
\subp{B}{Utskottsrapporter}{\info}
Erik berättade att styrelsen på kåren inte ville ta något beslut angående biljetterna till FlyING på mötet. På mötet ville de höra vad sektionerna hade att säga. V, E, A kommer att överklaga till Kårens fullmäktige.

Det rullar på för NollU, men i dag saknades det en iZettle dosa.

FVU har bokfört och E-shop har sålt lite mer. Märkespickniquen förra veckan gick superbra och E hade de tveklöst populäraste märkena!

NöjU går bra. De höll en förfest och mellanfest som var lyckad tillsammans med V-sektionen. Det regnade inte. Igår skulle de arrangera Agent 00E men det regnade så det var inte lyckat. De körde istället bastuhäng och det var väldigt väldigt lyckat.

E6 har haft mycket att göra den här veckan - Vettiquette förra onsdagen, WrEcK i lördags (tillsammans med W6 och K6) och VE och fasa - sittning och eftersläpp - tillsammans med V-sex i tisdags. Det har varit mycket men har flutit på bra och det mesta har funkat. Samarbetet med andra sexmästerier har gått bra även om kommunikation mellan så många människor alltid är svårt. De har lärt sig saker efter varje sittning och känslan är bra inför Nollegasquen!

De ser också mycket fram emot att få gå på Nollesittningen (där nollor och vice sexmästare jobbar) och få se blivande stjärnor i E6 jobba!

I InfU har det inte hänt så mycket, DDG har terminens första möte nästa vecka och fotograferna fotar.

CEQ-möten har börjat för SRE, de började denna veckan och håller på nästa vecka också. Rätt mycket rutin på dessa möten, så det är inga konstigheter. LTH ska utvärdera endimensionell analys. De hade många anmälda för att hjälpa till med utvärderingen men folk dök inte upp.

ENU hade lunchföreläsning igår vilket gick bra så nu ska Josefine bara skicka iväg fakturan. De har möte ikväll där de bland annat ska diskutera framtida evenemang. Josefine har svarat på lite mejl, ARM vill stå i foajén bl.a.

Det förra gillet gick bra för KM. Regattan gick bra, de sålde för över \SI{10000}{kr}.

CM har haft fullt upp i caféet med öppning, stängning, eftermiddagar och med att lära upp nya Dioder och styra upp. Det börjar dock bli mindre att göra i och med att Dioderna börjar känna sig mer och mer varma i kläderna. De funderar på att stänga ner caféet från onsdag nästa vecka tills onsdagen därpå i och med att det kommer bli mycket med Nollegasquen och för att de behöver en chans att vila upp sig för att ta nya tag när de kommer tillbaka. Inköps-och lagerchefer kommer vara med på mötet idag och dela med sig av hur dem känner och då kanske man kan hitta en rimligare lösning än att vi ska få jobba så pass mycket.

Hälsoinspektören var på besök förra veckan, hon hade några anmärkningar gällande varors temperatur. Tyvärr kom hon lite olägligt då de små kylarna utanför LED precis gått sönder. CM har bokat in en ny träff med henne torsdag 21/9, tills dess ska de ha fixat planscher med tydligare regler, de ska ha omstrukturerat i lagret, se till att all information finns på både svenska och engleska m.m.

I övrigt har det gått jättebra försäljningsmässigt, vilket är väldigt kul! De har redan några nollor som skrivit upp sig som Dioder för LP1 och även några internationella studenter. ARKAD kommer och köper kaffe av dem idag (14/9).
\subp{C}{Ekonomisk rapport}{\info}
Sophia meddelade att ekonomin mår bra. Det har inte hänt något speciellt.
\subp{D}{Kåren informerar}{\info}
Värdansökan till ARKAD öppnar snart. I år kommer ARKAD även att vara i E-huset vilket innebär att det är fler platser för värdar och företag.

Det finns fortfarande vakanta poster i styrelsen.
\end{paragrafer}

\p{11}{Slutkörda inköps- och lagerchefer}{\dis}
Inköps- och lagerchefer känner att de jobbar för mycket nu när Ulla inte är i LED.

Filip ser ett alternativ som är att hitta någon vikarie eller deltidsanställd.
Två andra eventuella lösningar är omorganisation eller högre kompensation för de som jobbar. Men inköps- och lagerchefer tror inte att högre kompensation kommer att hjälpa så mycket.

Största problemet just nu är efter klockan 12.00.

Inköps- och lagerchefer lägger mest av sin tid på att ta emot varor och öppna cafeét. Men nu när Ulla är borta så blir det alltid inköps- och lagerchefer som får ställa upp när det saknas folk.

Eriks åsikt är att sektionen driver cafeét efter den förmåga vi har, vi stänger hellre cafeét än att de som hjälper till i LED jobbar för mycket och inte hinner med studierna.

På V-sektionen får ``skiftansvarig'' \SI{1000}{kr} enligt Albin.

F-sektionen har dagsansvariga.

M-sektionen har så att man kan ladda sitt kort med pengar och blippa för en kaffe.

Mötet diskuterade problemen som finns.

CM har fria händer att testa olika lösningar, att hitta en deltidsanstäld går inte hur snabbt som helst. De dagar då det inte går att ha cafeét på eftermiddagen så stänger vi istället.
\p{12}{Nästa styrelsemöte}{\bes}
{\Mba}nästa styrelsemöte ska äga rum 2017-09-21 12:10 i E:1124.

\p{14}{Beslutsuppföljning}{\bes}
{\Ibfu}

\p{15}{Övrigt}{\dis}
Mötet diskuterar var den försvunna ipaden kan vara.

Mötet diskuterade sittningen för utomlundarna.
\p{16}{OFMA}{\bes}
{\mo} förklarade mötet avslutat 12:55.

\end{paragrafer}

%\newpage
\hidesignfoot
\begin{signatures}{3}
\signature{\mo}{Mötesordförande}
\signature{\ms}{Mötessekreterare}
\signature{\ji}{Justerare}
\end{signatures}
\end{document}
