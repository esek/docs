\documentclass[../_main/handlingar.tex]{subfiles}

\begin{document}

\section{En kort guide till Sektionsmötena}
Sektionsmötena är E-sektionens högsta beslutande organ, det vill säga att det är här är alla de största och viktigaste besluten under året klubbas igenom. Det är också på Sektionsmötena alla förtroendevalda och funktionärer väljs. I mötets början ska mötesförfarandet gås igenom, men är där fortfarande några oklarheter kan du ställa en \emph{ordningsfråga} på mötet.

När du lämnar eller ankommer till mötet under mötets gång ska du skriva upp dig på en av listorna som finns vid utgångarna. Detta eftersom mötessekreteraren och justerarna ska ha ha full koll på vilka som är/var närvarande.

För att alla ska kunna delta i mötet och vara med på noterna följer här en ordlista på de vanligaste och viktigaste begreppen.

\begin{description}[style=multiline, leftmargin=45mm]
    \item[Acklamation]
    Det vanligaste sättet att ta beslut, där röstningen sker muntligt. Talmannen frågar om mötet vill \emph{bifalla} det liggande förslaget, och de som vill det svarar ``ja''. Därefter frågar densamma om någon är emot bifall, och de som är emot svarar ``ja''. Man svarar alltså aldrig ``nej''. Det är därefter upp till Talmannen att avgöra vilket alternativ som överväger. Om reslutatet verkar osäkert kan man begära \emph{votering} innan klubban fallit och beslutet fastställs.
    \item[Adjungera]
    Att tillfälligt låta någon utanför Sektionen stå som medlem. Personen kommer få yttra sig och yrka, men inte rösta.
    \item[Ajournera]
    Att avbryta mötet för att senare återuppta det, till exempel för en matpaus eller bensträckare.
    \item[Ansvarsfrihet]
    Att sektionsmötet godkänner föregående års funktionärer, utskott, styrelse, revisorer samt valberedning och avstår rätten att i efterhand kräva skadestånd av dessa. Dock är det möjligt att dra tillbaka ansvarsfrihet, om så behövs.
    \item[Avslag]
    Att inte godkänna ett förslag.
    \item[Bifall]
    Att godkänna ett förslag.
    \item[Bokslut]
    Sammanställning av sektionens bokföring.
    \item[Bokslutsdisposition]
    Fördelning av föregående års vinst till sektionens fonder.
    \item[Bordläggning]
    Att skjuta upp behandlingen av en fråga till ett senare möte.
    \item[Justering av protokoll]
    För att mötesprotokollet ska vara giltigt måste justerarna (som väljs på mötet) kontrollera och godkänna det. Som justeringsperson bör man helst inte vara nominerad till en post behandlad av valberedningen.
    \item[Justering av röstlängd]
    Före varje \emph{votering} måste röstlängden justeras. Det innebär att alla på mötet prickas av på en lista med samtliga sektionsmedlemmar.
    \item[Jämka sig] När någon som lagt fram ett förslag, exempelvis en \emph{motionär}, ställer sig bakom ett \emph{yrkande} från annan person angående förslaget i fråga.
    \item[Jäv]
    Då någon är personligt berörd av ett beslut och det därför föreligger risk för partiskhet.
    \item[Motion]
    Ett förslag från en eller flera sektionsmedlemmar. I regel brukar styrelsen ge ett svar med deras åsikt om förslaget.
    \item[Motionär] Någon som fört fram en motion, eller någon som t.ex. joggar.
    \item[Ordningsfråga]
    En fråga som ej berör sakfrågan utan endast berör mötets procedur. Ordningsfrågor bryter talarlistan och bör endast användas då där är oklarheter i möteförfarandet eller du vill \emph{ajournera} mötet eller dra \emph{streck i debatten}.
    \item[Proposition]
    Ett förslag från styrelsen.
    \item[Replik]
    Om någon tilltalar dig personligen kan mötesordföranden besluta om att du får bemöta detta. Detta svar kallas replik och bör hållas kort eftersom det bryter talarlistan.
    \item[Reservation]
    Innebär att du inte vill vara juridiskt ansvarig för ett beslut. Reservationer görs skriftligen till mötessekreteraren. En reservation är ett mycket starkt avståndstagande och bör inte användas lättvindigt.
    \item[Sakupplysning]
    Används om någon säger något direkt felaktigt, eller då du har information som är viktig för debatten. Sakupplysningar bör användas sparsamt och får inte missbrukas.
    \item[Streck i debatten]
    Begärs om du tycker att en fråga diskuterats länge nog och inga nya argument framkommer. Mötet går då över i beslut om streck i debatten, vilket inte helt ovanligt kan leda till beslut om streck i debatten i diskussionen om streck i debatten. Detta kan ju såklart verka roligt, men i själva verket brukar det bara leda till slöseri med tid och bör undvikas.
    \item[Votering]
    Används i regel i personval och andra känsliga frågor, eller begärs då en \emph{acklamation} verkat för jämn för att avgöra utfallet. Vanlig votering är mer eller mindre en vanlig handuppräckning. Sluten votering sker antingen via papperslappar eller vårt egna fantastiska röstningssystem - \texttt{E-vote}.
    \item[Yrkande]
    Ett formellt förslag till beslut. Alla yrkande måste lämnas skriftligen eller på mail (\href{mailto:yrka@esek.se}{\texttt{yrka@esek.se}}) till mötesordföranden.
\end{description}

\begin{signatures}{1}
    \mvh
    \signature{Erik Månsson}{Kontaktor 2016}
\end{signatures}

\end{document}
