\documentclass[10pt]{article}
\usepackage[utf8]{inputenc}
\usepackage[swedish]{babel}

\def\mo{Erik Månsson}
\def\ms{Johan Karlberg}
\def\ji{Markus Rahne}
%\def\jii{}

\def\doctype{Protokoll} %ex. Kallelse, Handlingar, Protkoll
\def\mname{styrelsemöte} %ex. styrelsemöte, Vårterminsmöte
\def\mnum{S05/17} %ex S02/16, E1/15, VT/13
\def\date{2017-02-16} %YYYY-MM-DD
\def\docauthor{\ms}

\usepackage{../e-mote}
\usepackage{../../../e-sek}

\begin{document}
\showsignfoot

\heading{{\doctype} för {\mname} {\mnum}}

%\naun{}{} %närvarane under
%\nati{} %närvarande till och med
%\nafr{} %närvarande från och med
\section*{Närvarande}
\subsection*{Styrelsen}
\begin{narvarolista}
\nv{Ordförande}{Erik Månsson}{E14}{}
\nv{Kontaktor}{Johan Karlberg}{E14}{}
\nv{Förvaltningschef}{Sophia Grimmeiss Grahm}{BME14}{}
\nv{Cafémästare}{Daniel Bakic}{E15}{}
\nv{Øverphøs}{Niklas Gustafson}{E15}{}
\nv{SRE-ordförande}{Pontus Landgren}{E14}{}
%\nv{ENU-ordförande}{Josefine Sandström}{E14}{}
\nv{Sexmästare}{Linnea Sjödahl}{BME15}{\naun{10 B}{11.5}}
\nv{Krögare}{Markus Rahne}{BME14}{}
\nv{Entertainer}{Albin Nyström Eklund}{BME16}{}
\end{narvarolista}

\subsection*{Ständigt adjungerande}
\begin{narvarolista}
%\nv{Kårordförande}{Linus Hammarlund}{}{}
\nv{Kårrepresentant}{Jacob Karlsson}{}{}
%\nv{Aktivietssamordnare}{Lovisa Majtorp}{}{}
%\nv{Valberedningens ordförande}{Elin Magnusson}{}{}
%\nv{Skattmästare}{Olle Oswald}{}{}
%\nv{Kårrepresentant}{Daniel Damberg}{}{}
%\nv{Kårrepresentant}{John Alvén}{}{}
%\nv{Talman}{Fredrik Peterson}{E14}{}
%\nv{Elektras Ordförande}{Elisabeth Pongratz}{}{}
%\nv{Inspektor}{Monica Almqvist}{}{}
\end{narvarolista}

\begin{comment}
\subsection*{Adjungerande}
\begin{narvarolista}
%\nv{Post}{Namn}{Klass}{}
\end{narvarolista}
\end{comment}

\section*{Protokoll}
\begin{paragrafer}
\p{1}{OFMÖ}{\bes}
Ordförande {\mo} förklarade mötet öppnat 12:13.

\p{2}{Val av mötesordförande}{\bes}
{\valavmo}

\p{3}{Val av mötessekreterare}{\bes}
{\valavms}

\p{4}{Val av justeringsperson}{\bes}
{\valavj}

\p{5}{Godkännande av tid och sätt}{\bes}
{\tosg}

\p{6}{Adjungeringar}{\bes}
{\ingaadj}

%Förnamn Efternamn adjungerades

\p{7}{Godkännande av dagordningen}{\bes}
%Dagordningen godkändes.
Erik \ypa att lägga till \S11.5 ``Jubileum''.
%Föredragningslistan godkändes med yrkandet.
%Föredragningslistan godkändes med samtliga yrkanden.

\p{8}{Föregående mötesprotokoll}{\bes}
\latillprot{S04/17}
%\ingaprot

\p{9}{Fyllnadsval och entledigande av funktionärer}{\bes}
\begin{fyllnadsval} %"Inga fyllnadsval." fylls i automatiskt
%\fval{Namn}{Post}
%\entl{Namn}{Post}
\end{fyllnadsval}

\p{10}{Rapporter}{}
\begin{paragrafer}
\subp{A}{Hur mår alla?}{\info}
Punkten protokollfördes ej.
\subp{B}{Utskottsrapporter}{\info}
FvU har hyrt ut Edekvata några gånger, det har fungerat bra. De har även bokfört en del.\\
KM har börjat dra igång med marknadsföringen inför första gillet och kommer hålla gille efter phadderutbildning och ``Agent00E'' den 1 april.\\
NollU har arbetat med phadderproccesen samt börjat spåna kläder. Phøset ska iväg på Fhöb-helg, vilket ska bli väldigt trevligt. Sen börjar phadderintervjuerna samt mer planering av klädsel. De ska även börja skriva ett spex till FPT.\\
ENU meddelar att ``Lunch med en ingenjör''-gänget har stått i foajén och idag utanför edekvata och tar emot anmälningar till eventet. Luncherna drar igång nästa vecka och sista anmälningsdag är imorgon. På måndag ska Josefine och phøs-Axel till SVEP och prata om robotic challenge. Törmänen-gruppen har fått svar från Ericsson, de ville ha möte. \\
NöjU har arrangerat bowlingturnering vilken blev lyckad. Spelemännen har kommit igång och planerar första spelkvällen. Agent00E är spikat till 1 april i samarbete med KM och Nollu. \\
E6 planerar för fullt inför Skiphtet och ska laga mat ikväll och imorgon. Alla Sexiga verkar taggade på att jobba för första gången vilket är väldigt kul. Linnea och hennes vice ska gå B-cert i helgen. Linnea håller också kontakten med andra sexmästerier för att planera in events under våren.\\
Pontus meddelar att på onsdag, den 22 februari, är det pluggkväll i edekvata. Prisutdelningen av CEQ-tävling pågår, i genomsnitt ökade svarsfrekvensen för årskurs 1-3 med 10\% jämfört med denna läsperioden förra året. De har även planerat kick-off för utskottet.\\
CM går bra, det kommer att finnas semlor på fettisdagen.\\
InfU rullar på, DDG har haft möte med phøset angående nollningshemsidan. Alumnikollegiet och sektionens alumniansvariga jobbar så att det ryker.

\subp{C}{Ekonomisk rapport}{\info}
Det har inte hänt något speciellt meddelade Sophia.
\subp{D}{Kåren informerar}{\info}
Kåren laddar för sektionsstyrelseutbildningen i helgen, i övrigt har det inte hänt så mycket.
\end{paragrafer}

\p{11}{Attestering}{\dis}
Till nästa möte ska alla läsa riktlinjerna som Sophia nu har lagt upp i driven.
\p{11.5}{Jubileum}{\dis}
Erik, Sophia och Anders har pratat budget och lite om hur Anders har tänkt. Mat från bryggan, IKDC, istället för MOP detta för att få lite bättre mat.\\
Erik lyfter frågan hurvida intresse för fyrverkerier. Någon citerar ett känt citat från wikki. Albin tycker att fyrverkerier inte behövs för att sittningen ska bli bra, däremot kan det förstöra sittningen tidsmässigt då tillståndet för fyrverkerier är strikt.\\
Mötet anser att fyrverkerier inte behövs.\\
Mötet diskuterar när sittningen ska börja och hurvida det ska serveras fördrink.
Albin tycker att fördrink är en bra idé för att minska förfestandet.\\
Sophia föreslår alkoholfri fördrink, för att slippa att söka tillstånd.\\
Daniel påpekar att det hade varit nice att ta bilder i foajén.\\
Mötet anser att sittningen ska börja kl 19.00 och fördrink i god tid innan.\\
Mötet tycker att det vore kul med en medalj till alla som deltar.\\
På fredagen innan kommer det att vara pub med allmänt tillstånd, alltså får våra vänner från andra skolor gå vara med.\\
\p{12}{Nästa styrelsemöte}{\bes}
{\Mba}nästa styrelsemöte ska äga rum 2017-02-23 12:10 i E:1426.

\p{13}{Beslutsuppföljning}{\bes}
Erik \ypa stryka ``Åka till Kvidinge'' från beslutsuppföljningen.\\
\Mbaby
\p{14}{Övrigt}{\dis}
Daniel föreslår att vi ska ge något till Ulla på hennes tioårsdag.\\
Pontus påpekar att vi ska ha en presentation till styrelseutbildningen.\\
\p{15}{OFMA}{\bes}
{\mo} förklarade mötet avslutat 12:48.

\end{paragrafer}

\newpage
\hidesignfoot
\begin{signatures}{3}
\signature{\mo}{Mötesordförande}
\signature{\ms}{Mötessekreterare}
\signature{\ji}{Justerare}
\end{signatures}
\end{document}
