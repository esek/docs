\documentclass[10pt]{article}
\usepackage[utf8]{inputenc}
\usepackage[swedish]{babel}

\def\mo{Erik Månsson}
\def\ms{Johan Karlberg}
\def\ji{Sophia Grimmeiss Grahm}
%\def\jii{}

\def\doctype{Protokoll} %ex. Kallelse, Handlingar, Protkoll
\def\mname{styrelsemöte} %ex. styrelsemöte, Vårterminsmöte
\def\mnum{S23/17} %ex S02/16, E1/15, VT/13
\def\date{2017-10-10} %YYYY-MM-DD
\def\docauthor{\ms}

\usepackage{../e-mote}
\usepackage{../../../e-sek}

\begin{document}
\showsignfoot

\heading{{\doctype} för {\mname} {\mnum}}

%\naun{}{} %närvarane under
%\nati{} %närvarande till och med
%\nafr{} %närvarande från och med
\section*{Närvarande}
\subsection*{Styrelsen}
\begin{narvarolista}
\nv{Ordförande}{Erik Månsson}{E14}{}
\nv{Kontaktor}{Johan Karlberg}{E14}{}
\nv{Förvaltningschef}{Sophia Grimmeiss Grahm}{BME14}{}
\nv{Cafémästare}{Daniel Bakic}{E15}{}
\nv{Øverphøs}{Niklas Gustafson}{E15}{}
\nv{SRE-ordförande}{Edvard Carlsson}{E16}{}
\nv{ENU-ordförande}{Josefine Sandström}{E14}{}
\nv{Sexmästare}{Linnea Sjödahl}{BME15}{}
\nv{Krögare}{Markus Rahne}{BME14}{}
\nv{Entertainer}{Albin Nyström Eklund}{BME16}{}
\end{narvarolista}

\subsection*{Ständigt adjungerande}
\begin{narvarolista}
%\nv{Kårordförande}{Linus Hammarlund}{}{}
%\nv{Kårrepresentant}{Anders Nilsson}{}{}
\nv{Kårrepresentant}{Caroline Svensson}{}{}
\nv{Kårrepresentant}{Agnes Sörliden}{\nati{12}}{}
%\nv{Valberedningens ordförande}{Elin Magnusson}{}{}
%\nv{Skattmästare}{Olle Oswald}{}{}
%\nv{Kårrepresentant}{Daniel Damberg}{}{}
%\nv{Kårrepresentant}{John Alvén}{}{}
%\nv{Talman}{Fredrik Peterson}{E14}{}
%\nv{Elektras Ordförande}{Elisabeth Pongratz}{}{}
%\nv{Inspektor}{Monica Almqvist}{}{}
\end{narvarolista}

\begin{comment}
\subsection*{Adjungerande}
\begin{narvarolista}
%\nv{Post}{Namn}{Klass}{}
\end{narvarolista}
\end{comment}

\section*{Protokoll}
\begin{paragrafer}
\p{1}{OFMÖ}{\bes}
Ordförande {\mo} förklarade mötet öppnat 12:13.

\p{2}{Val av mötesordförande}{\bes}
{\valavmo}

\p{3}{Val av mötessekreterare}{\bes}
{\valavms}

\p{4}{Val av justeringsperson}{\bes}
{\valavj}

\p{5}{Godkännande av tid och sätt}{\bes}
{\tosg}

\p{6}{Adjungeringar}{\bes}
{\ingaadj}

%Förnamn Efternamn adjungerades

\p{7}{Godkännande av dagordningen}{\bes}
Dagordningen godkändes.
%Fredrik \ypa att lägga till \S18b ``Teknikfokus utnyttjande av LED-café''.
%Föredragningslistan godkändes med yrkandet.
%Föredragningslistan godkändes med samtliga yrkanden.

\p{8}{Föregående mötesprotokoll}{\bes}
\latillprot{S22/17}
%\ingaprot

\p{9}{Fyllnadsval och entledigande av funktionärer}{\bes}
\begin{fyllnadsval} %"Inga fyllnadsval." fylls i automatiskt
%\fval{Namn}{Post}
%\entl{Namn}{Post}
\end{fyllnadsval}

\p{10}{Rapporter}{}
\begin{paragrafer}
\subp{A}{Hur mår alla?}{\info}
Punkten protokollfördes ej.
\subp{B}{Utskottsrapporter}{\info}
Der alte Kellermesterei hat einen Oktoberfest och det gick bra. Över förväntan faktiskt, vilket är kul för att vara en icke-nollningspub.

Igår stod ARM i foajén, de var nöjda, och idag står Academic Work och har CV granskning. Planeringen inför lunch med ingenjör är i full gång. Josefine får lite mejl från företag som vill marknadsföra sig via affischer och på facebook.

Adsensus vill skriva ett kontrakt med E-sektionen där de erbjuder en utbildning för \SI{13500}{kr} för våra medlemmar i utbyte mot reklam på sektionen. Utbildningen kan handla om olika saker, till exempel säljteknik eller presentationsteknik.

NollU försöker att komma ikapp i skolan. Efter tentorna ska de jobba med utvärderingen av nollningen.

NöjU visade landskamp igår. De ska peppa folk till sångarstriden imorgon på Expo och sen har de tänkt lägga upp och hålla i uppstartsmöte på måndag.

CM jobbar fortfarande med hur de ska strukturera om så att caféet fungerar utan anställd. Tills helgen ska de kunna ha en färdigställd proposition om en ny post i form av en Dagsansvarig som sitter per läsperiod och då har ansvar för en dag i veckan. Till Expot har Daniels kära vice kommit med förslaget att ha olivätartävling, Daniel känner att de får spinna vidare på det. Annars går det ganska bra, de har fått in ganska många Dioder till LP2 och de har börjat nominera folk som vi tycker hade passat i caféet. De sålde kaffe till ARM idag (tisdag) och kommer sälja kaffe till Academic Work imorgon, vilket är najs. De har även skickat ut intressekoll till tackfest på nation och planerar preliminärt att ha det i LV2 i LP2.

Just nu händer inte så mycket inom SRE, i måndags var det möte med institutionsstyrelsen för EIT, det diskuterades budget, löner och nya kurser. De har även fått in fler intresseanmälningar till årskursansvariga, 2 stycken från BME1. Edvard får se hur de gör då de egentligen bara har en ledig plats, kanske blir som för E1 och utnyttja posten SRE-ledamot.

I FVU är Skattmästaren är på Mallorca så det blir inte jättemycket bokfört denna vecka. I övrigt är det lugnt i utskottet. Arkivarierna ska fixa lite gamla grejer till Expot.

E6 har planerat vad de ska göra på Expot denna veckan och de håller på att planera vilka event de ska hålla under hösten. Det är inte spikat ännu men de har en del idéer.

InfU har planerat lite inför Expo, annars så fortsätter utskottet sin vanliga verksamhet utan hinder.

Erik meddelade att flyingmotionen behandlas i oktober.
\subp{C}{Ekonomisk rapport}{\info}
Sophia meddelade att de har lyckats hitta differensen på \SI{5000}{kr}. Någon hade tagit fel på kontant och kort. De är klara med bokföring för första halvåret.
\subp{D}{Kåren informerar}{\info}
Val till balgruppen till Corneliusbalen har öppnat nu. Val till heltidare öppnas nästa vecka.
\end{paragrafer}

\p{11}{Expo}{\dis}
Josefine undrade vad alla hade för plan inför Expot. Det blir ganska spontant, men planen är att försöka blanda utskotten så att alla syns. Erik köper godis.
\p{12}{HT-möte}{\dis}
Mötet gick igenom budgetförslaget för 2018. Som det ser ut nu bör det spenderas mer pengar då sektionen inte får gå med för mycket vinst. Tankarna är att höja tacket till funktionärer, speciellt för kick-in och kick-off.

Albin tycker att NöjUs events inte borde vara 0 budgeterade.
\p{13}{Nästa styrelsemöte}{\bes}
{\Mba} nästa styrelsemöte ska äga rum 2017-10-18 12:10 i E:1124.

\p{14}{Beslutsuppföljning}{\bes}
{\Ibfu}

\p{15}{Övrigt}{\dis}
Albin föreslår att vi ska köpa en kamera och drönare.

Markus föreslår att köpa en iZettle-scanner och iPads.
\p{16}{OFMA}{\bes}
{\mo} förklarade mötet avslutat 13:07.

\end{paragrafer}

%\newpage
\hidesignfoot
\begin{signatures}{3}
\signature{\mo}{Mötesordförande}
\signature{\ms}{Mötessekreterare}
\signature{\ji}{Justerare}
\end{signatures}
\end{document}
