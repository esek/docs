\documentclass[../_main/handlingar.tex]{subfiles}

\begin{document}
\verksplanuppf{VT 2017 (nuvarande styrelsen)}

\subsubsection*{Styrelsen}
Överlag har styrelsen (tyvärr) inte jobbat så mycket direkt mot verksamhetsplanens delmål än så länge under 2017. Givetvis strävar styrelsen efter huvudmålet ``att göra medlemmarnas studietid så bra som möjligt'', och har till detta möte gjort det mest genom att fokusera på att göra valprocessen så rättvis som möjligt. Anledningen till det är att styrelsen inte vill att någon ska känna sig utstött, och att alla ska ha en lika bra möjlighet till att engagera sig i Sektionen.

Teknikfokus har fått ett ökat intresse från E-sektionen, vilket bådar gott för kommande år. Arbetet med Sektionens funktionärsposter har fortsatt och styrelsen har utvärderat en del av posterna. Kostnader och intäkter ses kontinuerligt över och budgetar diskuteras.

Sist men inte minst har vi sett till att det pantas mera genom att sätta ett pantkärl i LED-café!

\subsubsection*{Informationsutskottet}

\subsubsection*{Källarmästeriet}
Under terminen som gått har KM mest fokuserat på själva evenemangen och dess aktiviteter än på en ny innovativ marknadsföringsplan. Affischer har satts upp och TV-apparaterna har gått varma med reklam och vi har försökt att nå aktivt ut via sociala medier. Sex-kollegiet har också varit till hjälp med att nå ut till fler men det har tyvärr inte visats i någon publikstorm till gillena. Tanken i fortsättning är att försöka synas mer i de andra husen runt LTH med mer allmän information samt förhoppningsvis mer samarbete sektionerna emellan, något som under terminen som gått varit i det bristande laget. Istället har det flutit på bra med att anordna evenemang tillsammans med resterande förmågor inom sektionen, där KM fixat mat till E6s kick-off, ordnat pub åt Teknikfokus, utnyttjat E-sektionens fantastiska husband och anordnat Agent 00E tillsammans med NöjU.

Det traditionella priserna på gillena har varit de samma i flera år och till slut har råvarupriserna och skatter nått den gräns att en prishöjning blev ett måste. Maten höjdes med 5 kr då köttet ska vara svenskt samtidigt som det har gett KM mycket större friheter att komponera roliga och varierade former av hamburgare. Priset på starksprit höjdes även det för att kompensera för det höjda inköpspriset. Prishöjningarna i fråga är milda och välargumenterade och både mat och dryck är fortsatt mycket snälla mot studentplånboken.

Inga större diffar har upptäckts i alkohollagret så svinn hålls på en låg nivå.

\subsubsection*{Nolleutskottet}

\subsubsection*{Cafémästeriet}

\subsubsection*{Förvaltningsutskottet}

\subsubsection*{Studierådet}
Studierådet jobbar med att försöka synliggöra utskottets arbete. Planerat arbete är att låta DDG utveckla möjligheten att låta SRE lägga upp anteckningar från möten på hemsidan. Det finns även ett delmål för 2017 som handlar om pluggkvällar och dess struktur samt syfte. Arbetet på denna punkten pågår och diskussioner förs. Tillsammans med NollU diskuteras hur vi vill arrangera pluggkvällarna under nollningen. Utskottet har i nuläget representanter från båda programen år 1-4, årskurs 5 är tyvärr svår att få representerad (förmodligen pga närheten till examen). Under början av året infördes CEQ-tävling som ett försök att försöka öka svarsfrekvensen på CEQ-enkäterna. Utfallet var inte vad som önskades och SRE behöver fortsätta jobba med detta målet.  

\subsubsection*{Sexmästeriet}
Sexmästeriet har under våren hållit i ett antal sittningar och event där vi alltid strävat efter att eventen ska hålla hög kvalité. En annan tanke med att göra mycket på våren har varit att vi ska bli så bra och samkörda som möjligt så att de event vi håller i framtiden går ännu smidigare.

Det är svårt att fördela evenemangen helt jämnt under året eftersom många event naturligt ligger under nollningen och alla funktionärer inom utskottet måste få tid att lära sig allt de behöver kunna för att smidigt kunna hålla sittning. Det är också många andra event som sker under våren och därför är det svårt att hitta datum som passar.

Trots detta har E6 haft mycket att göra under våren. En del av dessa event har inte varit öppna för alla Sektionens medlemmar (exempelvis sittningar för Flickor på Teknis och Teknikfokus) men leder till att Sexmästeriet blir mer samkört och bättre på att hålla sittningar, vilket gagnar alla på Sektionen. De eventen är också sådana som E6 gärna vill och förväntas hålla i.

Tidigt på vårterminen tog hovmästarna på sig att inventera och städa i Pump, vilket de gjorde väldigt bra. De har också tittat på möjligheten att köpa in nya hyllor för att effektivisera förvaringen och lämnat in en motion om det. E6-16 lämnade över Pump i bra skick, vilket vi i E6-17 verkligen uppskattar!

När dessa handlingar skrivs har vi av olika anledningar ännu inte hållit i ett event där alkoholen kommer från vårt förråd, så vi har faktiskt inte haft någon kontakt med vinlagret, men ska självklart jobba mot detta mål när det är aktuellt.

\subsubsection*{Nöjesutskottet}
Utskottet har arbetat för att ge en jämnare arbetsbelastning mellan utskottets medlemmar. Vice respektive fritidsledare har fått uppgiften att själva arrangera och genomföra evenemang. Idrottsförmännen har tagit stort ansvar och hjälpt till på majoriteten av eventen och delvis fungerat som extra fritidsledare. 
     
Postbeskrivningar är inte uppdaterade men kommer att genomföras i god tid innan höstterminsmötet. 
     
Utskottet har aktivt arbetat för att sprida evenemang till en större skara. Dels genom att använda posters till alla evenemang och genom att lägga ut information om evenemang i såväl klassgrupper som sektionsgrupper. Det är oklart ifall detta gjort någon skillnad. 


\subsubsection*{Näringslivsutskottet}
ENU har anordnat en blandning av evenemang, både vinstdrivande och mindre vinstdrivande.  Prislistan har setts över men kan till viss del göras mer detaljerad och omfattande. I princip alla mejladresser från föregående år har kontaktats. Förra året fick ENU även en lista på hårdvaruföretag som har kontaktats i år. Eventet ”lunch med en ingenjör” har också med sitt billigare pris bidragit till nya kontakter som förhoppningsvis leder till långvariga samarbeten. Tyvärr är det fortfarande svårt att få ihop evenemang med BME-företag. DDG har gjort en ny företagssida som förhoppningsvis ska läggas upp innan sommaren. 

\newpage
\begin{signatures}{10}
    \mvh
    \signature{Erik Månsson}{Ordförande 2017}
    \signature{Johan Karlberg}{Kontaktor 2017}
    \signature{Sophia Grimmeiss Grahm}{Förvaltningschef 2017}
    \signature{Daniel Bakic}{Cafémästare 2017}
    \signature{Niklas Gustafson}{Øverphøs 2017}
    \signature{Pontus Landgren}{SRE-ordförande 2017}
    \signature{Josefine Sandström}{ENU-ordförande 2017}
    \signature{Linnea Sjödahl}{Sexmästare 2017}
    \signature{Markus Rahne}{Krögare 2017}
    \signature{Albin Nyström Eklund}{Entertainer 2017}
\end{signatures}

\end{document}
