\documentclass[../_main/handlingar.tex]{subfiles}

\begin{document}
\verksplanuppf{VT 2017 (nuvarande styrelsen)}

\subsubsection*{Styrelsen}
Överlag har styrelsen (tyvärr) inte jobbat så mycket direkt mot verksamhetsplanens delmål än så länge under 2017. Givetvis strävar styrelsen efter huvudmålet ``att göra medlemmarnas studietid så bra som möjligt'', och har till detta möte gjort det mest genom att fokusera på att göra valprocessen så rättvis som möjligt. Anledningen till det är att styrelsen inte vill att någon ska känna sig utstött, och att alla ska ha en lika bra möjlighet till att engagera sig i Sektionen.

Teknikfokus har fått ett ökat intresse från E-sektionen, vilket bådar gott för kommande år. Arbetet med Sektionens funktionärsposter har fortsatt och styrelsen har utvärderat en del av posterna. Kostnader och intäkter ses kontinuerligt över och budgetar diskuteras.

Sist men inte minst har vi sett till att det pantas mera genom att sätta ett pantkärl i LED-café!

\subsubsection*{Informationsutskottet}

\subsubsection*{Källarmästeriet}

\subsubsection*{Nolleutskottet}

\subsubsection*{Cafémästeriet}

\subsubsection*{Förvaltningsutskottet}

\subsubsection*{Studierådet}

\subsubsection*{Sexmästeriet}

\subsubsection*{Nöjesutskottet}

\subsubsection*{Näringslivsutskottet}
\newpage
\begin{signatures}{10}
    \mvh
    \signature{Erik Månsson}{Ordförande 2017}
    \signature{Johan Karlberg}{Kontaktor 2017}
    \signature{Sophia Grimmeiss Grahm}{Förvaltningschef 2017}
    \signature{Daniel Bakic}{Cafémästare 2017}
    \signature{Niklas Gustafson}{Øverphøs 2017}
    \signature{Pontus Landgren}{SRE-ordförande 2017}
    \signature{Josefine Sandström}{ENU-ordförande 2017}
    \signature{Linnea Sjödahl}{Sexmästare 2017}
    \signature{Markus Rahne}{Krögare 2017}
    \signature{Albin Nyström Eklund}{Entertainer 2017}
\end{signatures}

\end{document}
