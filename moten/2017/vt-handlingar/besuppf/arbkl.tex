\documentclass[../_main/handlingar.tex]{subfiles}

\begin{document}
\beslutsuppfoljning{Införandet av arbetskläder för utlåning till funktionärer}

På hösterminsmötet 2016 beslutades att Sektionen att avsätta 6000 kr jämnt fördelat på CM, E6, ENU och KM för att köpa arbetskläder som funktionärer, som inte vill lägga personliga pengar på sitt engagemang, ska kunna låna. Inte alla utskott har ännu utnyttjat detta. Huruvida de inköpta kläderna har använts och huruvida utskottens medlemmar har blivit informerade om denna möjlighet återstår att säga, antagligen till nästa års funktionärer blir valda samt att alla kläder är inköpta.  Själva beställandet har utförts av respektive utskottschef och det följande har köpts in.

\begin{dashlist}
    \item E6: 8st kavajer i blandad storlek har beställts/köpts in. Alla har inte kommit än och därför går det i nuläget inte att säga exakt vad frakten, och därmed inte häller den totala summan, i slutändan landar på.
    \item KM: 6st t-shirts av blandad storlek för den totala summan av \SI{1200}{kr}.
    \item ENU: 9st tröjor i blandad storlek för den totala summan av \SI{2003}{kr}.
    \item CM: Har än så länge inte sett över arbetskläder då det tidigare inte funnits och vilket inte har varit ett problem på samma sätt som för andra utskott.
\end{dashlist}

Med hänvisning till ovanstående så yrkar jag på

\begin{attsatser}
    \att skjuta upp beslutsuppföljningen för ``Införandet av arbetskläder för utlåning till funktionärer'' till HT/17 men med styrelsen som ansvarig.
\end{attsatser}

\begin{signatures}{1}
    \mvh
    \signature{Martin Gemborn Nilsson}{Sexmästare 2016}
\end{signatures}

\end{document}
