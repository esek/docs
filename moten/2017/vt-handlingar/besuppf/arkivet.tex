\documentclass[../_main/handlingar.tex]{subfiles}

\begin{document}
\beslutsuppfoljning{Uppfräschning av gamla arkivet}

På vårterminsmötet 2016 beslutades att Arkivet skulle renoveras. Renoveringen skulle framförallt behandla inköp av ny inredning som skulle passa till phösets verksamhet under nollningen och till mötesrum under resterande terminen. Arbetet skulle vara färdigt innan nollningen till en kostnad av maximalt \SI{10000}{kr}.

Innan terminens slut diskuterade jag och phöset vilka inventarier som skulle passa till rummet samt planer på ny belysning. Arbetet började med att jag och arkivarierna tömde ut resterande saker från arkivet och strax därefter sattes en whiteboard upp. De mesta av de nya inventarierna köptes in i samband med renoveringen av HK/BD och Diplomat och blev färdigställandet försenat p.g.a. husets fönsterbyte. Men precis innan nollningen byggde phöset ihop möblerna och någon vecka in fick vi fixat belysningen.

I efterhand kan konstateras att resultat varit mycket lyckat.

Då projektet kan anses avslutat, har uppfyllt sitt syfte samt hållit sig inom budgeten yrkar undertecknad på

\begin{attsatser}
    \att stryka \emph{Uppfräschning av gamla arkivet} från beslutsuppföljningen.
\end{attsatser}

\begin{signatures}{1}
    \mvh
    \signature{Anders Nilsson}{Förvaltningschef 2016}
\end{signatures}

\end{document}
