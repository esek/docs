\documentclass[./vt16.tex]{subfiles}

\begin{document}
\textbf{\Large{Tillägg till revisionsberättelse för E-sektionen 2016}}

Styrelseprotokollen och sektionsmötesprotokollen är överlag lätta att följa och det är lätt att förstå vilka beslut som har fattats. Styrelsen har tagit till sig av de rekommendationer som gavs efter föregående bokslut vad gäller sektionsmötesprotokollen vilket underlättede genomläsning av protokollen.

Efter granskning av ekonomin har inga oegentligheter uppdagats. Styrelsen har varit noggranna med att följa budgetriktlinjerna vilket underlättat granskningen betydligt. Styrelsen har även blivit mer engagerad i sektionens ekonomi vilket resulterat i att utskottsbudgetar har följts bättre.

Teknikfokus har inte följt budget. Detta är ett återkommande problem då budgeten detaljstyrs för 
evenemanget, utan att sedan följa denna. Vi rekommenderar att man inför nästa år sätter sig ned
med D-sektionen och tar fram en lösning för att detta inte ska fortsätta vara fallet i framtiden.

\textbf{SEK01} Funktionärsvård gick 19000 mindre minus än budgeterat.Tycker sektionen
fortfarande att man vill lägga 75000 kronor på funktionärerna så bör nästkommande styrelse fundera hur man skulle kunna spendera dessa pengar så att de går funktionärerna till gagn. Det bör även noteras att arbetsglädje 6000 mer minus än budgeterat. En övervägning om budgeten för dessa bör slås samman eller ändras. 

\textbf{SEK02} Revisorerna använder ej pengarna de är budgeterade. Revisorerna misstänker att budgeten är en kvarlevnad som var tänkt åt utskriftskostnader med mera. Sektionen bör möjligtvis överväga att avskaffa resultatenheten.


\end{document}
