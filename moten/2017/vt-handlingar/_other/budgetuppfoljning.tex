\documentclass[../_main/handlingar.tex]{subfiles}

\begin{document}
\proposition{Budgetförslag för 2017}

Styrelsen yrkar på

\begin{attsatser}
    \att antaga den bifogade budgeten för 2017,
    \att antaga de bifogade budgetriktlinjerna för 2017, samt
    \att antaga de bifogade långsikta ekonomiska målen för 2017.
\end{attsatser}

För att Sektionen ska kunna jobba efter de ekonomiska målen för de kommande verksamhetsåren krävs det en omformulering av Sektionens fondbeskrivningar samt en justering av saldot mellan fonderna.

Med anledning av ovanstående yrkar styrelsen också på

\begin{attsatser}
    \att ändra paragraf \S14:1:A Dispositionsfonden i reglementet till\par
    \begin{itshape}
         Styrelsen äger rätt att disponera de avsatta medlen.

         Dispositionsfondens medel är avsatta för att ge styrelsen möjligheten till mindre investeringar i nya inventarier eller verksamhetsutvecklande investeringar som ligger utanför verksamhetsårets rambudget.
    \end{itshape}

    \att ändra paragraf \S14:1:C Olycksfonden i reglementet till\par
    \begin{itshape}
        Styrelsen äger rätt att disponera de avsatta medlen.

        Olycksfondens medel är avsatta för att ersätta eller reparera funktionsodugliga inventarier, eller andra akuta kostnader, som krävs för att kunna bedriva Sektionens ordinarie verksamhet. Medlen ska också användas för kostnader i samband med olycksfall med personskador eller personlig utrustning.
    \end{itshape}

    \att ovanstående reglementesändringar börjar gälla nästa verksamhetsår, samt

    \att flytta om medlen i Sektionens fonder vid detta verksamhetsårets slut så att verksamhetsår 2017 har följande ingående balans innan resultatet för 2016 disponerats:\par
    \begin{tabular}{l r}
        Dispositionsfonden & \SI{50 000}{kr}\\
        Olycksfonden & \SI{100 000}{kr}\\
        Jubileumsfonden & \SI{30 000}{kr}\\
        Egetkapital & \SI{400 000}{kr}\\
        Modulo10fonden & \SI{7671}{kr}\\
    \end{tabular}\par
    där resterande medel flyttas till Utrustningsfonden, vilket skulle resultera i ungefär \SI{307000}{kr} i skrivande stund.
\end{attsatser}

\begin{signatures}{3}
    \ist
    \signature{\ordf}{Ordförande}
    \signature{Anders Nilsson}{Förvaltningschef}
    \signature{Johan Karlberg}{Kontaktor}
\end{signatures}

\newpage
\subsection*{Budget 2017}
\begin{tabularx}{10cm}{X r r}
    \textbf{\large Sektionen} & \textbf{2016} & \textbf{2017} \\
    \hline
    \textbf{SEK01} \\
    Sektionsavgifter & \SI{35000}{kr} & \SI{39000}{kr} \\
    Sektionsaktiviteter & \SI{-10000}{kr} & \SI{-10000}{kr} \\
    Representation & \SI{-5000}{kr} & \SI{-5000}{kr} \\
    Funkionärsvård & \SI{-75000}{kr} & \SI{-70000}{kr} \\
    Arbetglädje & \SI{-10000}{kr} & \SI{-12000}{kr} \\
    Arbetskläder Funktionärer & \SI{0}{kr} & \SI{-1000}{kr} \\
    Medaljer & \SI{0}{kr} & \SI{-500}{kr} \\
    Pant & \SI{0}{kr} & \SI{15000}{kr} \\
    Expo & \SI{-2000}{kr} & \SI{-2000}{kr} \\
    \textbf{SEK02}, Revisorer & \SI{-500}{kr} & \SI{-500}{kr} \\
    \textbf{SEK03}, Valbered & \SI{-500}{kr} & \SI{-500}{kr} \\
    \textbf{SEK05}, Skiphte & \SI{-10000}{kr} & \SI{-10000}{kr} \\
    \hline
    \textbf{Summa} & \textbf{\SI{-78000}{kr}} & \textbf{\SI{-57500}{kr}} \\
\end{tabularx}

\begin{tabularx}{10cm}{X r r}
    \textbf{\large Sexmästeriet} & \textbf{2016} & \textbf{2017} \\
    \hline
    \textbf{SEX01} \\
    E6 Allmänt & \SI{20000}{kr} & \SI{15000}{kr} \\
    \hline
    \textbf{Summa} & \textbf{\SI{20000}{kr}} & \textbf{\SI{15000}{kr}} \\
\end{tabularx}

\begin{tabularx}{10cm}{X r r}
    \textbf{\large Näringslivsutskottet} & \textbf{2016} & \textbf{2017} \\
    \hline
    \textbf{ARMU01} \\
    Sponsring & \SI{85000}{kr} & \SI{75000}{kr} \\
    Teknikfokus & \SI{100000}{kr} & \SI{110000}{kr} \\
    \hline
    \textbf{Summa} & \textbf{\SI{185000}{kr}} & \textbf{\SI{185000}{kr}} \\
\end{tabularx}

\begin{tabularx}{10cm}{X r r}
    \textbf{\large Informationsutskottet} & \textbf{2016} & \textbf{2017} \\
    \hline
    \textbf{INFU01} \\
    Alumniverksamhet & \SI{-1000}{kr} & \SI{-1000}{kr} \\
    Nolleguide & \SI{-5000}{kr} & \SI{-5000}{kr} \\
    E-shop & \SI{0}{kr} & Se FVU \\
    Datordrift & \SI{-3500}{kr} & \SI{-2000}{kr} \\
    Ljud och ljus & Se FVU & \SI{0}{kr} \\
    HeHE & \SI{-500}{kr} & \SI{0}{kr} \\
    \hline
    \textbf{Summa} & \textbf{\SI{-10000}{kr}} & \textbf{\SI{-8000}{kr}} \\
\end{tabularx}

\begin{tabularx}{10cm}{X r r}
    \textbf{\large Källarmästeriet} & \textbf{2016} & \textbf{2017} \\
    \hline
    \textbf{KM01} \\
    Gille & \SI{30000}{kr} & \SI{33000}{kr} \\
    Fast tillstånd & \SI{-14000}{kr} & \SI{-14000}{kr} \\
    \hline
    \textbf{Summa} & \textbf{\SI{16000}{kr}} & \textbf{\SI{19000}{kr}} \\
\end{tabularx}

\begin{tabularx}{10cm}{X r r}
    \textbf{\large Cafémästeriet} & \textbf{2016} & \textbf{2017} \\
    \hline
    \textbf{CM01} \\
    LED & \SI{80000}{kr} & \SI{40000}{kr} \\
    Tillsyn & \SI{-2000}{kr} & \SI{-2000}{kr} \\
    LED-tack & \SI{0}{kr} & \SI{0}{kr} \\
    \textbf{CM02} \\
    Mojter & \SI{0}{kr} & \SI{0}{kr} \\
    \hline
    \textbf{Summa} & \textbf{\SI{78000}{kr}} & \textbf{\SI{38000}{kr}} \\
\end{tabularx}

\begin{tabularx}{10cm}{X r r}
    \textbf{\large Styrelsen} & \textbf{2016} & \textbf{2017} \\
    \hline
    \textbf{STY01} \\
    Klädsel & \SI{-6500}{kr} & \SI{-6500}{kr} \\
    Styrelserepresentation & \SI{-5000}{kr} & \SI{-5000}{kr} \\
    Styrelsen internt & \SI{-11000}{kr} & \SI{-11000}{kr} \\
    \hline
    \textbf{Summa} & \textbf{\SI{-22500}{kr}} & \textbf{\SI{-22500}{kr}} \\
\end{tabularx}

\begin{tabularx}{10cm}{X r r}
    \textbf{\large Nöjesutskottet} & \textbf{2016} & \textbf{2017} \\
    \hline
    \textbf{NOJU01} \\
    Allmänt & \SI{-500}{kr} & \SI{-1500}{kr} \\
    \textbf{NOJU02} \\
    Biljard och spel & \SI{0}{kr} & \SI{0}{kr} \\
    Tandem & \SI{-1000}{kr} & \SI{-500}{kr} \\
    Sångarstriden & \SI{-2000}{kr} & \SI{-2000}{kr} \\
    Programverksamhet & \SI{-1000}{kr} & - \\
    Utedischo & \SI{20000}{kr} & \SI{25000}{kr} \\
    \textbf{NOJU03} \\
    Sporta med E & \SI{-10000}{kr} & \SI{-7000}{kr} \\
    \hline
    \textbf{Summa} & \textbf{\SI{5500}{kr}} & \textbf{\SI{14000}{kr}} \\
\end{tabularx}

\begin{tabularx}{10cm}{X r r}
    \textbf{\large Studierådet} & \textbf{2016} & \textbf{2017} \\
    \hline
    \textbf{SRE01} \\
    CEQ-intäkter & \SI{10000}{kr} & \SI{10000}{kr} \\
    SRE disponibelt & \SI{-12000}{kr} & \SI{-10000}{kr} \\
    \hline
    \textbf{Summa} & \textbf{\SI{-2000}{kr}} & \textbf{\SI{0}{kr}} \\
\end{tabularx}

\begin{tabularx}{10cm}{X r r}
    \textbf{\large Nolleutskottet} & \textbf{2016} & \textbf{2017} \\
    \hline
    \textbf{PHOS01} \\
    Nollning Allmänt & \SI{-40000}{kr} & \SI{-20000}{kr} \\
    Phöset internt & \SI{0}{kr} & \SI{-15000}{kr} \\
    Phaddertack & \SI{0}{kr} & \SI{-5000}{kr} \\
    \hline
    \textbf{Summa} & \textbf{\SI{-40000}{kr}} & \textbf{\SI{-40000}{kr}} \\
\end{tabularx}

\begin{tabularx}{9cm}{X r r}
    \textbf{\large Förvaltningsutskottet} & \textbf{2016} & \textbf{2017} \\
    \hline
    \textbf{FVU01} \\
    Expedition & \SI{-13000}{kr} & \SI{-13000}{kr} \\
    Kontantfri lösning & \SI{-10000}{kr} & \SI{-15000}{kr} \\
    Fin. int. och kost. & \SI{-5000}{kr} & \SI{-5000}{kr} \\
    E-shop & \SI{0}{kr} & \SI{5000}{kr} \\
    Arkiv & \SI{-500}{kr} & \SI{0}{kr} \\
    \textbf{FVU02} \\
    Ljud och ljus & \SI{-1000}{kr} & Se InfU \\
    Edekvata & \SI{-20000}{kr} & \SI{-20000}{kr} \\
    Trivsel & \SI{-1000}{kr} & \SI{-1000}{kr} \\
    \hline
    \textbf{Summa} & \textbf{\SI{-50500}{kr}} & \textbf{\SI{-49000}{kr}} \\
\end{tabularx}

\begin{tabularx}{9cm}{X r r}
    \textbf{\large Total summa} & \textbf{2016} & \textbf{2017} \\
    \hline
     & \textbf{\SI{101500}{kr}} & \textbf{\SI{94000}{kr}} \\
\end{tabularx}


\newpage
\subsection*{Budgetriktlinjer 2017}

\subsection*{Sektionen}
\titlerule[0.5pt]
\begin{description}[style=multiline, leftmargin=60mm]

\item[SEK01, Sektionsavgifter]
Detta är Sektionens andel av medlemmarnas betalda kåravgift. Hela summan är fördelad på utgifterna som kommer alla medlemmar till godo. 20 000 kronor är placerade på Edekvata (innefattar alla Sektionens uppehållslokaler), 10 000 kr på sektionsaktiviteter, 7000 kr på Sporta med E, 1000 kr på alumniverksamhet samt 1000 kr för hustomtar/trivsel.

\item[SEK01, Sektionsaktiviteter]
Avser kostnader för sektionsmöten.

\item[SEK01, Representation]
Detta konto inkluderar representation av styrelsens ordföranden eller dennes ställföreträdare, på såväl våra som vänsektioners fester samt på större arrangemang inom Teknologkåren, dock till en maximal kostnad av 1/5 av kontosumman. I kontot inkluderas även kostnader för välförtjänta gåvor till personer som på ett betydande sätt gagnat Sektionen. Kostnader för inbjudna gäster till Sektionens arrangemang ska belasta arrangemanget i fråga, med undantag för arrangemang som enbart syftar till att representera Sektionen vilka ska belasta detta konto.

\item[SEK01, Funktionärsvård]
All funktionärsvård skall belasta denna budgetpost. Detta inkluderar kostnaden för subventioneringen av gillemat och kaffe för funktionärer samt kosnaden för utskottsaktiviteter med syfte att stärka utskottsgemenskapen. Funktionärsvård i övriga fall, d.v.s. endast gällande styrelsen, skall täckas av styrelsen internt. En del av denna budgetpost bör användas för att bekosta ett arrangemang i syfte att avtacka funktionärerna för året.

\item[SEK01, Medaljer]
Avser kostnader för utdelade medaljer. Det vill säga lagerfört inköpspris för utdelad medalj.

\item[SEK01, Arbetsglädje]
Avser kostnader för arbetsglädje från Cafémästeriets förråd samt mindre inköp av mötesfika. I de fall då arbetsglädjen utgörs av mat från försäljning skall detta belasta utskottet ifråga.

\item[SEK01, Arbetskläder för funktionärer]
Används till underhåll samt mindre nyinköp av de arbetskläder som Sektionen lånar ut till sina funktionärer.

\item[SEK01, Pantintäkter]
Avser intäkter för pant.

\item[SEK01, Expo]
Avser kostnader för expot.

\item[SEK02,  Revisorer]
För revisorerna fritt disponibel summa.

\item[SEK03, Valberedning]
Avser kostnader för marknadsföring av Sektionens val.

\item[SEK05, Skiphte]
Avser kostnader för Sektionens funktionärsskiphte.

\end{description}

\subsection*{Styrelsen}
\titlerule[0.5pt]
\begin{description}[style=multiline, leftmargin=60mm]

\item[STY01, Klädsel]
Pengar tilldelade styrelsen för inköp av profilkläder samt frackbrodyr.

\item[STY01, Styrelserepresentation]
Avser kostnader för subventionering av biljetter till sektionstillställningar för styrelsen. Tanken är att på så vis uppmuntra styrelsemedlemmarna att gå på fler sektionstillställningar. Subventionen fördelas av styrelsen själva, men är med fördel jämnt fördelad på alla medlemmar. Subventionen kan innefatta till exempel inträde och mat, men inte alkohol.

\item[STY01, Styrelsen internt]
Avser styrelsens interna kostnader såsom ``Kurs På Landet'' (KPL), vars kostnader dock inte rekommenderas att överstiga 50\% av kontosumman, samt mat vid möten. Här finns även utrymme för styrelsens egna initiativ i form av arrangemang som inte bör finansieras på annat sätt.

\end{description}

\subsection*{Sexmästeriet}
\titlerule[0.5pt]
\begin{description}[style=multiline, leftmargin=60mm]

\item[SEX01, E6 allmänt]
Avser kostnader och intäkter rörande E6:s verksamhet. Sexmästeriet ska sikta på att enbart gå i vinst på alkohol och i övrigt hålla nere priserna så att så många som möjligt har råd att delta i deras arrangemang.

\end{description}

\subsection*{Förvaltningsutskottet}
\titlerule[0.5pt]
\begin{description}[style=multiline, leftmargin=60mm]

\item[FVU01, Expedition]
Avser kostnader för Hamnkontorets behov av kontorsmaterial samt kostnader för arkivering.

\item[FVU01, Kontantfri lösning]
Avser kostnaden för transaktionsavgifter för Sektionens kontantfria betalningslösningar.

\item[FVU01, Finansiella intäkter och kostnader]
Avser kostnader för förvaltandet av vårt kapital, till exempel bankkostnader, deponering, samt intäkter i form av avkastning på bankmedel och eventuella aktieaffärer.

\item[FVU01, E-shop]
Avser E-shopens intäkter och kostnader.

\item[FVU01, Arkiv]
Nollbudgeteras då kostnaderna för arkivering flyttats till FVU01, Expedition.

\item[FVU02, Ljud och Ljus]
Avser kostnader för löpande underhåll av, samt hyresintäkter från, befintlig ljud- och ljusanläggning.

\item[FVU02, Edekvata]
Avser kostnader för löpande underhåll av Sektionens lokaler samt mindre investeringar. Intäkter i samband med utlåning av Sektionens lokaler samt utrustning som inte innefattas av Ljud och Ljus ska läggas här

\item[FVU02, Trivsel]
Tilldelat Sektionens Hustomtar för att öka trivseln i Edekvata genom exempelvis inköp av tavlor, växter, högtidspynt och dylikt.

\end{description}

\subsection*{Nöjesutskottet}
\titlerule[0.5pt]
\begin{description}[style=multiline, leftmargin=60mm]

\item[NOJU01, NöjU allmänt]
Avser kostnader för Sektionens allmänna arrangemang som ej innefattas av övriga budgetposter.

\item[NOJU02, Biljard och spel]
Avser kostnader och intäkter för biljard och spel. Eftersom kostnaderna ska motsvaras av intäkterna är budgetposten nollbudgeterad.

\item[NOJU02, Tandem]
Avser  kostnader och intäkter i samband med Tandem. 500kr ska täcka underhåll av tandemcyklarna, och övriga kostnader skall kvitteras ut mot biljettintäkterna.

\item[NOJU02, Sångastriden]
Avser kostnader för Sektionens deltagande i Sångarstriden som till exempel kostnaden för dekor och dylikt.

\item[NOJU02, Programverksamhet]
Nollbudgeteras då saldot flyttats till NOJU01, Nöju Allmänt.

\item[NOJU02, Utedischo]
Avser intäkter från UtEDischot. På grund av att tacket för Utedischot 2016 sker under 2017 flyttas E-sektionens kostnad på 7500 kr för detta till budgeten för Utedischot 2017.

\item[NOJU03, Sporta med E]
Avser kostnader för idrottsverksamhet som till exempel hallhyra men även diverse inköp av idrottsredskap så som bollar etc. -- dock ej utrustning till hela lag eller liknande.

\end{description}

\subsection*{Källarmästeriet}
\titlerule[0.5pt]
\begin{description}[style=multiline, leftmargin=60mm]

\item[KM01, Gillen]
Avser kostnader och intäkter rörande KM:s verksamhet.

\item[KM01, Fast tillstånd]
Avser kostnaden för det fasta serveringstillståndet i Edekvata.

\end{description}

\subsection*{Informationsutskottet}
\titlerule[0.5pt]
\begin{description}[style=multiline, leftmargin=60mm]

\item[INFU01, Almuniverksamhet]
Till förfogande för de Alumniansvariga att använda till diverse Alumniaktiviteter. Arrangemang för alumner ska i största möjliga mån betalas av deltagarna.

\item[INFU01, Nolleguide]
Avser kostnaden för tryck av Nolleguiden.

\item[INFU01, Datordrift]
Avser kostnader för drift och underhåll samt uppgraderingar av program- och maskinvara till Sektionens datorutrustning. Inköp av ny datorutrustning och liknande bör läggas på dispositionsfonden.

\item[INFU01, HeHE]
Nollbudgeteras då HeHE inte längre trycks.

\end{description}

\newpage
\subsection*{Studierådet}
\titlerule[0.5pt]
\begin{description}[style=multiline, leftmargin=60mm]

\item[SRE01, CEQ-intäkter]
Avser intäkter från granskning av CEQ-enkäter. Denna intäkt måste spenderas av SRE och utgör grunden för SRE disponibelt.

\item[SRE01, SRE disponibelt]
Avser kostnader för SRE:s verksamhet som till exempel pluggkvällar, inspirationsföreläsningar, CEQ-priser etc.

\end{description}

\subsection*{Cafémästeriet}
\titlerule[0.5pt]
\begin{description}[style=multiline, leftmargin=60mm]

\item[CM01, LED]
Avser kostnader och intäkter gällande LED-café, inklusive avlöning.

\item[CM01, Tillsyn]
Avser kostnad för årlig tillsyn av miljöförvaltningen.

\item[CM01, LED-tack]
Samtlig funktionärsvård belastar SEK01, Funktionärsvård. På grund av detta är LED-tack nollbudgeterat.

\item[CM02, Mojter]
Avser intäkter från läsk- och kakköp direkt från Cafémästeriets förråd till självkostnadspris samt kostnaden för inköp av varorna. Eftersom dessa ska gå jämnt ut är budgetposten nollbudgeterad.

\end{description}

\subsection*{Näringslivsutskottet}
\titlerule[0.5pt]
\begin{description}[style=multiline, leftmargin=60mm]

\item[ARMU01, Sponsring]
Budgeterat belopp innefattar både nollningssponsring och övrig sponsring. Om ett företag sponsrar en viss produkt, till exempel T-shirtar till nollorna, bör även kostnaden för dessa belasta denna budgetpost.

\item[ARMU01, Teknikfokus]
Avser intäkter från Teknikfokus.

\end{description}

\subsection*{Nolleutskottet}
\titlerule[0.5pt]
\begin{description}[style=multiline, leftmargin=60mm]

\item[PHOS01, Nollning allmänt]
Budgeterat belopp ska användas för att introducera de nyantagna i studentvärlden.

\item[PHOS01, Phøset internt]
Avser kostnader för Phøsets interna verksamhet såsom kläder, skiphte, mat etc.

\item[PHOS01, Phaddertack]
Avser kostnader för att tacka phaddrar och andra nollningsaktiva för deras nollningsengagemang då de varken omfattas av Sektionens funktionärvård eller arbetsglädje.

\end{description}

\newpage
\subsection*{Långsiktiga ekonomiska mål 2017}

\subsubsection*{Dispositionsfonden}
Målet med dispositionsfonden är att ge styrelsen ett ekonomiskt underlag att kunna införskaffa nya inventarier och/eller förändra Sektionens verksamhet utanför verksamhetsårets rambudget. Det är dock viktigt att dispositionsfonden används “selektivt”, d.v.s. pengar som används ska gynna sektionens verksamhet i sin helhet och gärna göra det på en längre sikt. Den sittande styrelsen ska inte se det som att hela dispositionsfonden ska spenderas vart verksamhetsår, utan ska istället ses som ett verktyg som de kan använda vid behov. Det är också rekommenderat att inköp som skulle motsvara en majoritet av dispositionsfondens ingående årssaldo istället prioriteras att läggas på utrustningsfonden vid ett sektionsmöte.

\subsubsection*{Olycksfonden}
Målet med olycksfonden är att den ska fungera som en buffert för att kunna ersätta och/eller reparera funktionsodugliga inventarier som är nödvändiga för Sektionens ordinarie verksamhet.  Olyckfonden ska också användas i samband med olycksfall med personskador eller personlig utrustning.

\subsubsection*{Eget kapital}
Målet med eget kapital är att Sektionen ska kunna bedriva sin ordinarie verksamhet utan att få likviditetsproblem. Det egna kapitalet ska vara så stort så att det klarar av dem svängningar i bank/handkassa som uppstår under verksamhetsårets gång utan att kapital från andra fonder behöver röras. Årsbudget ska sättas på så sätt att det egna kapitalet är lika stort vid årets slut som vid årets början.

\subsubsection*{Utrustningsfonden}
Målet med utrustningsfonden är att ackumulera kapital för att kunna göra större nyinvesteringar eller ombyggnader. Från utrustningsfonden ska både styrelsen och ordinarie medlemmar kunna äska pengar från Sektionen till olika projekt, inköp, etc.

\subsubsection*{Jubileumsfonden}
Målet med jubileumsfonden att ackumulera kapital till att delvis bekosta ett jubileum vart femte år. Tanken är att göra detta genom att varje år avsätta en liten del av resultatet, lämpligtvis en femtedel av målet för jubileumsfonden, om så är möjligt.

\subsubsection*{Årsresultat}
Målet är att Sektionens årsresultat ska budgeteras på sådant sätt att fonderna och eget kapital klarar av att uppfylla sina mål både på kort och lång sikt. Sektionen ska sträva mot att endast gå så mycket plus som anses nödvändigt, för att ge tillbaka så mycket som möjligt till våra medlemmar.

\newpage
\subsubsection*{Konkretisering av målen och hur de uppnås}
Målet för \textbf{dispositonsfonden} är att det vid verksamhetsårets början ska finnas en summa på \SI{50000}{kr}.

Målet för \textbf{olycksfonden} är att det vid verksamhetsårets början ska finnas en summa på \SI{100000}{kr}.

Det långsiktiga målet för \textbf{eget kapital} är att summan vid verksamhetsårets uppgå till \SI{500000}{kr} men för att så stor del av Sektionens kapital som möjligt ska komma de nuvarande medlemmarna till gagn, till exempel genom att läggas på utrustningsfonden, bör det byggas upp över tid. I detta förslag på 5 års sikt.

Målet för \textbf{jubileumsfonden} är att det vid verksamhetsårets början ska finnas en summa på \SI{50000}{kr}, d.v.s. att man varje år avsätter ungefär \SI{10000}{kr}.

Resterande medel läggs på \textbf{utrustningsfonden}.

\begin{table}[H]
\begin{center}
\begin{tabularx}{0.9\textwidth}{X r r r r c}
    & \textbf{2016} & \textbf{2017} & \textbf{2018} & \textbf{2019} & \textbf{Prioritet} \\
    \hline
    Dispositionsfonden & \SI{120 000}{kr} & \SI{50 000}{kr} & \SI{50 000}{kr} & \SI{50 000}{kr} & 3 \\
    Olycksfonden & \SI{80 000}{kr} & \SI{100 000}{kr} & \SI{100 000}{kr} & \SI{100 000}{kr} & 1 \\
    Eget kapital & \SI{350 000}{kr} & \SI{400 000}{kr} & \SI{425 000}{kr} & \SI{450 000}{kr} & 2 \\
    Utrustningsfonden & \SI{387 000}{kr} & & & & 5 \\
    Jubileumsfonden & \SI{30 000}{kr} & \SI{45 000}{kr} & \SI{10 000}{kr} & \SI{20 000}{kr} & 4 \\
    Mål för årsresultat & \SI{100 000}{kr} & \SI{100 000}{kr} & \SI{80 000}{kr} & \SI{70 000}{kr} & \\
\end{tabularx}
\end{center}
\end{table}

\newpage
\end{document}
