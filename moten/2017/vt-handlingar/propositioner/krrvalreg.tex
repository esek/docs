\documentclass[../_main/handlingar.tex]{subfiles}

\begin{document}
\proposition{Flytta policybeslutet ``Närvaro vid Sektionsmöte'' till reglementet och uppdatera personval}

Styrelsen anser att policybeslutet ``Närvaro vid Sektionsmöte'' bör ligga i reglementet då det behandlar kandiderings- och rösträtt vid sektionsmötena. Vi vill också passa på att förtydliga texten. Att en medlem har rösträtt behöver inte stå med, det står redan i stadgan.

Därför yrkar styrelsen på

\begin{attsatser}
    \att ta bort policybeslutet ``Närvaro vid Sektionsmöte'',

    \att i reglementet lägga till \S4:G:\par
        \subsection*{4:G Kandiderings- och rösträtt}
        Personer som ej är fysiskt närvarande i lokalen vid Sektionsmötet har ej rösträtt, men får kandidera till poster på alternativa sätt, till exempel via videolänk, om Talmannen finner det lämpligt.

        Motkandidering till en av valberedningen nominerad styrelsepost skall meddelas valberedningen senast 3 läsdagar före aktuellt val. Är posten vakant kan motkandidering ske direkt på mötet.

    \newpage

    \att i reglementet ändra \S12 till:\par
        \subsection{12:A Personval}

        \hl{Enligt stadgan \S12:6 avgör lotten vid lika antal röster i votering i personval och där ej annat stadgats eller Reglementet föreskriver annorlunda gäller enkel majoritet.}

        \begin{alphlist}
        \item En kandidat, en ska \hl{tillsättas}\par
        \hl{Görs} med acklamation om inte sluten votering begärs. Om acklamationen ej blir enhällig blir det automatiskt sluten votering\hl{. \ \ \ }

        \item Mindre eller lika många kandidater \hl{än vad} som \hl{ska} tillsättas\hl{, flera ska tillsättas}\par
        Kandidaterna väljes en och en enligt fall \hl{a)}. Med acklamation kan \hl{mötet} bestämma att välja alla i klump.

        \item Flera kandidater, en \hl{ska tillsättas}\par
        \hl{
        Görs med sluten votering.
        Om någon kandidat erhåller majoriteten av rösterna är personen vald.
        I annat fall stryks den kandidaten som erhållit minsta röstetal.
        Vid lika antal röster avgör lotten.
        Efter det upprepas proceduren med de kvarvarande kandidaterna enligt passande fall.
        }

        \item Fler kandidater än vad som \hl{ska} tillsättas, \hl{flera ska tillsättas}\par
        \hl{
        Görs med sluten votering.
        Varje röstberättigad person får maximalt lika många röster som det finns platser kvar att tillsättas.
        Om någon kandidat erhåller majoriteten av rösterna är personen vald, och valproceduren upprepas med de kvarvarande kandidaterna enligt passande fall.
        I annat fall stryks den kandidaten som erhållit minsta röstetal.
        Vid lika antal röster avgör lotten.
        Efter det upprepas proceduren med de kvarvarande kandidaterna enligt passande fall.
        }
        \end{alphlist}

        \subsection{12:B Sakfrågor}

        \hl{Görs med acklamation} om inte votering begärs\hl{. }Först behandlar man alla yrkanden och därefter de tilläggsyrkanden som inte har automatiskt har avslagits. Talmannen ställer lämpliga yrkanden emot varandra för att slutligen få ett yrkande kvar som Sektionsmötet ska ta ställning till. När detta ej är möjligt använd lämpligt fall \hl{för personval} för att få ett slutgiltigt beslut. Därefter behandlas de tilläggsyrkande som finns kvar på samma sätt.

    \changenote

\end{attsatser}

\begin{signatures}{1}
    \ist
    \signature{\ordf}{Ordförande}
\end{signatures}

\end{document}
