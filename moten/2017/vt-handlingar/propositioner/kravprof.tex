\documentclass[../_main/handlingar.tex]{subfiles}

\begin{document}
\proposition{Öppna kravprofiler för valberedning}

Styrelsen vill att valberedningens kravprofiler för funktionärsposter ska vara öppna för alla att läsa, samt att kravprofilerna skrivs till nästkommande år av sittande styrelse och valberedning tillsammans.

Kravprofilerna ska skrivas ungefär som en jobbannons. En bra beskrivning av vad posten innebär, vad som förväntas av en person på posten, och bra egenskaper för en person på posten. Men precis som i en jobbannons bör inga konkreta frågor (och eventuella svar) finnas, med undantag för då det finns formella krav på innehavaren av en post, till exempel att en firmatecknare måste vara minst 20 år gammal. Det är sedan upp till valberedningen att ställa relevanta frågor utifrån kravprofilen för att bedöma den sökandes lämplighet för posten.

Styrelsen anser att detta har en rad med fördelar:
\begin{dashlist}
    \item De som suttit/sitter i valberedningen får ingen orättvis fördel av att ha sett kravprofilerna, och kan ej då heller anklagas för att ha tagit nytta av dem.
    \item Kravprofilerna hålls uppdaterade och relevanta med hjälp av styrelsens insamlade kunskap under sitt år.
    \item Kravprofilerna fungerar som en extra utförlig postbeskrivning, som sektionsmedlemmarna kan läsa vid intresse av att söka en post.
    \item Kandidater får bättre möjlighet att förbereda sig inför sin intervju.
\end{dashlist}

Då styrelsen tillsammans med valberedningen först i slutet av detta verksamhetsår kommer skriva kravprofiler tillsammans, är det orimligt att kräva att ändringen ska vara helt genomförd innan VM/17.

Det finns lite småfel här och där i reglementet som denna proposition också fixar till.

Med anledning som ovan yrkar styrelsen på

\begin{attsatser}
    \att i reglementet ändra \S7:A till:\par
    %\begin{itshape}
        \subsection*{7:A Tillvägagångssätt}
        Valberedningen ansvarar för planeringen av hela \hl{valberedningsförfarandet}. Det är dock Styrelsen som ansvarar för själva genomförandet av Expot och Valmötet. Officiella nomineringstiden skall vara minst 8 läsdagar, men bör vara längre. Nomineringar görs \hl{via hemsidan senast 23:59 den sista} nomineringsdagen. Valberedningens förslag skall anslås senast \hl{23:59 8 läsdagar} innan \hl{mötet. }Eventuella motkandidater anslås senast \hl{23:59 2} läsdagar \hl{innan mötet}.

        \hl{Valberedningen kan, om de så önskar, adjungera personer till möten eller intervjuer där de finner det lämpligt, förutom då de skall besluta om nomineringar.}
    %\end{itshape}

    \changenote

    \newpage

    \att i reglementet lägga till \S7:B (så att nuvarande \S7:B blir \S7:C o.s.v.) med följande innehåll:
    %\begin{itshape}
        \subsection*{7:B Kravprofiler}
        För varje post som valberedningen behandlar ska det finnas kravprofiler. De uppdateras inför varje kommande verksamhetsår av sittande valberedning och styrelse, och ska finnas öppna för att läsas på hemsidan. Det är utifrån kravprofilerna som valberedningen ska ställa relevanta frågor i sina intervjuer, och bedömda den sökandes lämplighet för posten.

        En god kravprofil innehåller en beskrivning av vad posten innebär, vad som förväntas av en person på posten, och bra egenskaper för en person på posten. Dock bör inga konkreta intervjufrågor (och eventuella svar) finnas med, med undantag för då det finns formella krav på innehavaren av en post.
    %\end{itshape}

    \att i reglementet ändra \S7:B (\S7:C efter förra att-satsen trädit i kraft) till:\par
    %\begin{itshape}
        \subsection*{7:C Valberedningens skyldigheter}
        Det åligger \hl{V}alberedningens Ordförande
        \begin{attlist}
            \item kalla \hl{till och} leda valberedningens arbete och möten,
            \item hålla Styrelsen informerad om hur arbetet fortskrider, samt
            \item tillsammans med Styrelsen planera Expot och \hl{V}almötet.
        \end{attlist}
        Det åligger \hl{alla} valberedningens medlemmar
        \begin{attlist}
            \item inte öppet diskutera hur arbetet inom valberedningen fortlöper,
            \item \hl{bara föra vidare information som framkommer under intervjuer med kandidater till Sektionsmötet}, samt
            \item närvara på Sektionsmöten som innehåller val som förberetts av valberedningen.
        \end{attlist}
        Det åligger \hl{hela} valberedningen
        \begin{attlist}
            \item \hl{i slutet av verksamhetsåret, i samråd med styrelsen, uppdatera och publicera kravprofilerna för nästa år,}
            \item använda \hl{S}ektionens informationskanaler för att nå ut med valinformation,
            \item muntligen informera studenter i årskurs 1 och 2, som är ordinarie medlemmar i \hl{S}ektionen om funktionärsvalen i samband med en föreläsning,
            \item kontinuerligt offentliggöra alla inkomna nomineringar fram tills det att nomineringstiden gått ut,
            \item genomföra intervjuer med personer som kandiderar till Styrelsepost, Valberedningens Ordförande, Revisorer och Inspektor samt då valberedningen finner det lämpligt,
            \item i samband med att valberedningens nomineringsförslag offentliggörs, även offentliggöra en lista över alla inkomna nomineringar där det framgår hur de nominerade ställer sig till en kandidatur, \hl{samt}
            \item offentliggöra officiella motkandidater fram till två dagar innan mötet för frågans avgörande\hl{.}
        \end{attlist}
    %\end{itshape}

    \changenote

    \newpage

    \att kravprofilerna för valen som utförs detta år öppnas upp i den mån det är möjligt, men att det inte är ett krav.
\end{attsatser}

\begin{signatures}{1}
    \ist
    \signature{\ordf}{Ordförande}
\end{signatures}

\end{document}
