\documentclass[../_main/handlingar.tex]{subfiles}

\begin{document}
\proposition{Ändring av hur Sektionen väljer Ph\o s}

Sektionen har sedan ett tag tillbaka haft traditionen att Phøset väljs i grupp. Reglementet säger både att Øverphøset ``ansvarar för rekryteringen av phøsare'' och att Co-phøs väljs individuellt på valmötet, vilket styrelsen anser vara en motsägelse. Tolkningen att Co-phøs väljs individuellt på valmötet har gjorts genom åren, men i praktiken så har det blivit så att valmötet har valt Phøs i grupp - av tradition.

Styrelsen anser att traditionen är problematisk. Vi vill att alla medlemmar ska kunna söka samt ha en rättvis och rimlig chans att bli Co-phøs. Om Co-phøsen blir valda mer eller mindre per automatik när dess Øverphøs blir valt, sker ingen form av valprocess för dem, och ingen annan får en chans.

Som styrelsen ser det finns två alternativ till lösningar på problemet, som båda tar bort motsägelsen i reglementet. Antingen
\begin{itemize}
    \item[1)] Att valberedningen och valmötet behandlar val av Co-phøs individuellt. Att Øverphøset ``ansvarar för rekryteringen av phøsare'' tas bort från reglementet.
\end{itemize}
eller
\begin{itemize}
    \item[2)] Att Øverphøset och valberedningen tillsammans rekryterar Co-phøs, istället för att de väljs på valmötet. Detta klargörs för posten Co-phøs i reglementet.
\end{itemize}

Problemet som ses med alternativ 1 är att i praktiken så sker ingen ändring av tolkningen av styrdokumenten. Att Øverphøset ``ansvarar för rekryteringen av phøsare'' har tidigare ändå ignorerats. Därför anser styrelsen att alternativ 2 är kommer göra störst nytta för Sektionen.

Styrelsens förslag är alltså att Co-phøs väljs av styrelsen på rekommendation  av Øverphøset och valberedningen. Detta anser vi få med sig en rad fördelar:

\begin{dashlist}
    \item Att det blir avsevärt svårare och mycket mer osannolikt att Phøset bildar en färdig grupp innan valet.
    \item Att ingen som ensam kandidat till varken Øverphøs eller Co-phøs behöver ställa sig upp mot en färdig grupp på valmötet, vilket vi tror kommer ge fler sökande till båda posterna.
    \item Att valberedningen tar hjälp av Øverphøset i valberedningen av Co-phøs.
    \item Att det fortfarande ger Øverphøset och valberedningen en god chans att ``pussla ihop'' ett Phøs som fungerar bra ihop, både kompetensmässigt och socialt.
\end{dashlist}

Styrelsen vill också passa på att uppdatera postbeskrivningarna så de är bättre skrivna och reflekterar hur NollU arbetat de senaste åren.

\newpage
Med anledning som ovan yrkar styrelsen på

\begin{attsatser}
    \att i reglementet under \S10:2:L ändra från:\par
    \begin{emptylist}
        \item Øverphøsare (u)
            \begin{dashlist}
                \item har det övergripande ansvaret för nollningen,
                \item ansvarar för nollningsaktiviteter och nolleuppdrag,
                \item ansvarar för rekryteringen av phøsare,
                \item aktivt deltaga i TLTHs gemensamma planering inför nollningen.
            \end{dashlist}
        \item Co-phøsare (5)
            \begin{dashlist}
                \item Bistår Øverphøsaren i dennes arbete,
                \item En Co-phøsare ansvarar för redovisningen av Nollningen.
            \end{dashlist}
        \item Övergudphadder (2)
            \begin{dashlist}
                \item ansvarar för phadderverksamheten,
                \item ansvarar för rekryteringen av phaddrarna.
                \item väljs av styrelsen kalenderår med rekomendation av Øverphøset och dess Co-phøsare.
            \end{dashlist}
    \end{emptylist}
    till:
    \begin{emptylist}
        \item Øverphøs (u)
            \begin{dashlist}
                \item Har det övergripande ansvaret för nollningen.
                \item Ansvarar för nollningsaktiviteter och nolleuppdrag.
                \item Ansvarar för rekryteringen av Co-phøs och Øvergudsphaddrar.
                \item Deltar aktivt i TLTH:s gemensamma planering inför nollningen.
            \end{dashlist}
        \item Co-phøs (5)
            \begin{dashlist}
                \item Bistår Øverphøset i dennes arbete.
                \item Ett Co-phøs ansvarar för den ekonomiska redovisningen av nollningen.
                \item Ett Co-phøs ansvarar för rekryteringen av phaddrarna.
                \item Väljs av styrelsen på rekommendation av Øverphøset och avgående valberedning.
            \end{dashlist}
        \item Øvergudsphadder (2)
            \begin{dashlist}
                \item Ansvarar tillsammans med ett Co-phøs för phadderverksamheten.
                \item Väljs av styrelsen på rekommendation av Øverphøset, Co-phøsen och avgående valberedning.
            \end{dashlist}
    \end{emptylist}
\end{attsatser}

\begin{signatures}{2}
    \ist
    \signature{\ordf}{Ordförande}
    \signature{Niklas Gustafson}{Øverphøs}
\end{signatures}

\end{document}
