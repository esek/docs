\documentclass[../_main/handlingar.tex]{subfiles}

\begin{document}
\proposition{Införande av projektfunktionärer}

Ibland jobbar personer i sektionen med projekt som ligger utanför en funktionärspost, till exempel med att renovera lokaler eller arrangera event som jubileum eller examensbanketter. Om de inte har en funktionärspost redan får de inga funktionärsprivilegier, vilket vi i styrelsen anser att de i vissa fall förtjänar att ha. Som lösning på detta vill vi införa en ny ``allmän'' funktionärspost som kan tillsättas av styrelsen eller sektionsmötet i samband med projekt som drivs på sektionen.

Därför yrkar styrelsen på

\begin{attsatser}
    \att i reglementet, under \S10:2:P Övriga funktionärer, lägga till\par
    %\begin{itshape}
        Projektfunktionär (e.a.)
        \begin{tightdashlist}
            \item har ett projekt med beslutsuppföljning på sektions- eller styrelsemöte, och väljs in som funktionär för att få funktionärsprivilegier.
            \item har en mandatperiod som är maximalt ett år lång, och är kalenderår ifall inget annat bestäms.
        \end{tightdashlist}
    %\end{itshape}
\end{attsatser}

\begin{signatures}{1}
    \ist
    \signature{\ordf}{Ordförande}
\end{signatures}

\end{document}
