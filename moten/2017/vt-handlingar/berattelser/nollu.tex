\documentclass[../_main/handlingar.tex]{subfiles}

\begin{document}
\berattelse{Nolleutskottet}
NollU har ägnat sig åt Nollning.

Under våren valdes grupphaddrar, där det inkluderas phusknollor, uppdragsphaddrar och internationella phaddrar för att ta emot nollorna på bästa möjliga sätt. I år blev det fler internationella phaddrar än tidigare år vilket var lyckat!

Phøset har varit på många möten tillsammans andra phøs, kåren, programledningen, matematiklärare och SI-ledare samt SVL. Vi anordnade flera aktiviteter med de andra phøsen som t.ex. ett gemensamt eftersläpp för nollningsaktiva efter temasläppet och phadderkickoff. Vi deltog aktivt i flera utbildningar och planerade nollningen in i minsta detalj. Mycket fokus har lagts på ekonomi, mångfald bland aktiviteter och förbättrad attityd till studier.

Under sommaren jobbade vi i phøset intensivt med att planera inför nollningen, både i början och i slutet av lovet. Närmare inpå nollningen bodde vi ihop och jobbade varje dag med att fixa lite allt möjligt som behövdes inför att alla nollor skulle komma. Mycket fokus lades på praktiska moment, men även en del planering.

E-sektionen deltog aktivt i de gemensamma aktiviteterna som kåren anordnade och vann ännu en gång FlyING efter ett mycket snyggt och temaenligt hopp i Helsingborg.

Nya saker för i år var bl.a. följande:
\begin{dashlist}
 \item Uppdragsgrupperna F.I.S.S, Philm och Øverphøs0lympiaden
 \item Ändrat koncept för pluggkvällar samt införskaffande av pluggphaddrar
 \item Nolledans på E
 \item Informationsutskick via appen Nollekollen
 \item Sittning på Nation (Helsingkrona) för uppdragsgrupper
 \item Välkomstsittning/kräftskiva på onsdagen i lv. 0
 \item Brunch
 \item MVP (Most valuable phadder)
 \item Phadder-drive
 \item LED-phaddrar
 \item Nollekampen
\end{dashlist}
Nu efter nollningen har phøset arbetat med att skriva testamente och förbereda en bra överlämning till nästa års phøs, samt att utvärdera hur allt under året gick.

I övrigt var det väldigt roligt att se hur mycket tid och engagemang alla phaddrar la ner på sina phaddergrupper, samt hur taggade alla uppdrags- och pluggphaddrar var på att hjälpa till under nollningen! Det har varit fantastiskt att ha varit phøs och øvergudsphaddrar just i år med allt tagg och all kämparglöd från både nya och gamla sektionsmedlemmar. Vi vill också tacka hela E-sektionen med alla utskott och frivilliga för all hjälp och stöd vi fått från er. Utan er hade Nollningen inte varit möjlig, och vi vill att ni ska veta hur mycket alla som ställer upp betyder för oss och för nollorna!
\begin{signatures}{1}
    \mvh
    \signature{Molly Rusk}{Øverphøs 2016}
\end{signatures}

\end{document}
