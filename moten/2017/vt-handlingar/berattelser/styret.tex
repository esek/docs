\documentclass[../_main/handlingar.tex]{subfiles}

\begin{document}
\berattelse{Styrelsen}
2016 började för styrelsens del med KPL vilket var väldigt roligt och ett bra sätt att lära känna varandra ytterligare. Helgen efter KPL anordnades det årliga Skiphtet för alla som varit funktionärer under det föregående året samt alla som nyligen gått på sina poster, det lektes bland annat massor med lekar och de fina utskottstavlor som alltjämt pryder Diplomats väggar målades.

Styrelsen ägnade alla hjärtans dag åt styrelseutbildning på kåren och har även haft ekonomiutbildning med förvaltningschefen samt årets revisorer. Styrelsemötena har fungerat på sedvanligt sätt med möten de flesta torsdagar. Vi har även haft längre kvällsmöten för större diskussioner och inför sektionsmötena.

Direkt efter vårterminsmötet satte arbetet igång med att realisera de beslut som togs. Uppdaterade styrdokument kom snabbt på plats vilka även har godkänts av kårfullmäktige. Under sommaren hjälpte delar av styrelsen till med renoveringarna av HK/BlåDörren, Diplomat samt Arkivet. Veckan innan nollningens början träffades styrelsen för att planera sitt arbete under nollningen. För styrelsens del var arbetet ganska koncentrerat till den första veckan och bestod bland annat av att medverka under Camp S:t Hans och under ett par tillfällen laga mat åt nollorna. För styrelsens del avslutades nollningen med att styrelserna från Chalmers och KTH kom på besök i samband med Nollegasquen vilket var väldigt roligt.

Efter nollningen gick hösten verkligen gått undan. På kvällsmötena, som hölls efter nollningen, har styrelsen framför allt diskuterat punkter att ta upp på höstterminsmötet och tillsammans förberett propositioner, budget och verksamhetsplan. Tillsammans med valberedningen höll styrelsen i expot dit framför allt många ettor kom och informerades om Sektionens utskott och funktionärsposter.

Styrelsen anordnade även ett funktionärstack innehållandes Laserdome och sittning på Blekingska Nation. Detta arrangemang var mycket uppskattat av Sektionens funktionärer och liknande tack blir förhoppningsvis årligen återkommande.

När årets slut närmade sig blev styrelsemedlemmarna allt tröttare men vi kämpade ända in i kaklet och avslutade året med ett trevligt pizzabak tillsammans med den nyvalda styrelsen och sedermera även KPL.
\begin{signatures}{1}
    \mvh
    \signature{Fredrik Peterson}{Ordförande}
\end{signatures}

\end{document}
