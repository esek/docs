\documentclass[../_main/handlingar.tex]{subfiles}

\begin{document}
\berattelse{Näringslivsutskottet}
ENU anordnade i början av verksamhetsåret en CV-fotografering och kort därefter anordnades en lunchföreläsning tillsammans med D-sektionens näringslivsutskott.

``Lunch med en ingenjör'' sjösattes och blev ett mycket lyckat event där studenter och verksamma ingenjörer lunchar tillsammans. Eventet har varit efterfrågat av såväl närvarande företag som andra som inte var med.

Under årets nollning anordnade näringslivsutskottet tre lunchföreläsningar. I år har vi, där det ekonomiskt har varit möjligt, caterat mat. Detta underlättar arbetet för ENU och uppskattas av deltagarna. Vidare har utskottet anordnat flera monterevent där företagen står i foajén och pratar med förbi gående studenter.

Efter nollningen har det varit lugnare, med enbart några monterevent. Lugnet beror bland annat på att vissa aktiviteter som var planerade till hösten har blivit senarelagda till våren, och att vissa aktiviteter inte blev av. De förra mikrovågsugnarna i Edekvata var slitna och började gå sönder och ENU ordnade med sponsring till de nya som är uppsatta i Edekvata nu.

Årets upplaga av Teknikfokus fortsatte i samma goda spår som tidigare år med många företag och besökare. Mässan drog in mycket pengar till E-sektionen och är en grundpelare som täcker en betydlig del av Sektionens utgifter under året. Det är av stor betydelse att engagemanget från E-sektionens sida ökar till nästkommande år för att mässan och samarbetet med D-sektionens ska vara fortsatt fruktsamma.

Det råder en påtaglig konjunktur bland hårdvaruföretag, vilket har märkts under året då flera företag har tagit kontakt med ENU och hört sig för hur de kan synas bland studenterna. Samarbeten tar tid att bygga upp och sker inte över en dag. Därmed är det viktigt att till nästa år fortsätta bygga vidare på dessa.
\begin{signatures}{1}
    \mvh
    \signature{Johannes Koch}{Näringslivsutskottets ordförande 2016}
\end{signatures}

\end{document}
