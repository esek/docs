\documentclass[10pt]{article}
\usepackage[utf8]{inputenc}
\usepackage[swedish]{babel}

\def\mo{Erik Månsson}
\def\ms{Johan Karlberg}
\def\ji{Albin Nyström Eklund}
%\def\jii{}

\def\doctype{Protokoll} %ex. Kallelse, Handlingar, Protkoll
\def\mname{styrelsemöte} %ex. styrelsemöte, Vårterminsmöte
\def\mnum{S14/17} %ex S02/16, E1/15, VT/13
\def\date{2017-05-19} %YYYY-MM-DD
\def\docauthor{\ms}

\usepackage{../e-mote}
\usepackage{../../../e-sek}

\begin{document}
\showsignfoot

\heading{{\doctype} för {\mname} {\mnum}}

%\naun{}{} %närvarane under
%\nati{} %närvarande till och med
%\nafr{} %närvarande från och med
\section*{Närvarande}
\subsection*{Styrelsen}
\begin{narvarolista}
\nv{Ordförande}{Erik Månsson}{E14}{}
\nv{Kontaktor}{Johan Karlberg}{E14}{}
\nv{Förvaltningschef}{Sophia Grimmeiss Grahm}{BME14}{}
\nv{Cafémästare}{Daniel Bakic}{E15}{\nafr{8}}
%\nv{Øverphøs}{Niklas Gustafson}{E15}{}
%\nv{SRE-ordförande}{Pontus Landgren}{E14}{}
\nv{ENU-ordförande}{Josefine Sandström}{E14}{}
\nv{Sexmästare}{Linnea Sjödahl}{BME15}{}
\nv{Krögare}{Markus Rahne}{BME14}{}
\nv{Entertainer}{Albin Nyström Eklund}{BME16}{\nati{12}}
\end{narvarolista}

\subsection*{Ständigt adjungerande}
\begin{narvarolista}
%\nv{Kårordförande}{Linus Hammarlund}{}{}
\nv{Kårrepresentant}{Agnes Sörliden}{}{}
%\nv{Aktivietssamordnare}{Lovisa Majtorp}{}{}
%\nv{Valberedningens ordförande}{Elin Magnusson}{}{}
%\nv{Skattmästare}{Olle Oswald}{}{}
%\nv{Kårrepresentant}{Daniel Damberg}{}{}
%\nv{Kårrepresentant}{John Alvén}{}{}
%\nv{Talman}{Fredrik Peterson}{E14}{}
%\nv{Elektras Ordförande}{Elisabeth Pongratz}{}{}
%\nv{Inspektor}{Monica Almqvist}{}{}
\end{narvarolista}


\subsection*{Adjungerande}
\begin{narvarolista}
\nv{Vice SRE-ordförande}{Edvard Carlsson}{E16}{}
\end{narvarolista}

\section*{Protokoll}
\begin{paragrafer}
\p{1}{OFMÖ}{\bes}
Ordförande {\mo} förklarade mötet öppnat 12:14.

\p{2}{Val av mötesordförande}{\bes}
{\valavmo}

\p{3}{Val av mötessekreterare}{\bes}
{\valavms}

\p{4}{Val av justeringsperson}{\bes}
{\valavj}

\p{5}{Godkännande av tid och sätt}{\bes}
{\tosg}

\p{6}{Adjungeringar}{\bes}
%{\ingaadj}
Edvard Carlsson adjungerades

\p{7}{Godkännande av dagordningen}{\bes}
%Dagordningen godkändes.
Albin \ypa lägga till \S11``Kavajer till Vice Entertainer''.

Erik \ypa lägga till \S12 ``Elektrotjejer söndag''.
%Föredragningslistan godkändes med yrkandet.

Föredragningslistan godkändes med samtliga yrkanden.

\p{8}{Föregående mötesprotokoll}{\bes}
%\latillprot{S00/17}
\ingaprot

\p{9}{Fyllnadsval och entledigande av funktionärer}{\bes}
\begin{fyllnadsval} %"Inga fyllnadsval." fylls i automatiskt
\fval{Johan Wendt}{Överbanan}
\fval{Isabella Hansen}{Näringslivskontakt}
\fval{Fanny Månefjord}{BME2ansvarig}
\fval{Ellen Nilsson}{BME3ansvarig}
\fval{Jonatan Kronander}{E3ansvarig}
\fval{Edvard Carlsson}{E2ansvarig}
\entl{Oscar Uggla}{Macapär}
\entl{Oscar Uggla}{Vice Krögare}
\entl{Edvard Carlsson}{Vice SRE-ordförande}
\entl{David Uhler Brand}{Vice Förvaltningschef}
\entl{Anders Nilsson}{Teknokrat}
\end{fyllnadsval}

\p{10}{Rapporter}{}
\begin{paragrafer}
\subp{A}{Hur mår alla?}{\info}
Punkten protokollfördes ej.
\subp{B}{Utskottsrapporter}{\info}
LEDs verksamhet har flutit på bra, i onsdags sålde vi för ca 7400kr vilket är rekord för sommartid. Ulla och mitt centrala team och ett fåtal dioder har hjälpt till med storstädningen inför sommaren och vi är så gott som klara med det. I torsdags höll jag tillsammans med fyra andra i Caféfesten, det var ca 90-100st som var där och fick avnjuta god mat, god dryck och gott sälskap. Tips för nästa år är att skaffa arbetare som hjälper till. Nu är det bara lite lagerinventering och halvårsrevidering som ska göras, utöver det kan vi i CM ta ett välförtjänt sommarlov.

KM förbereder för terminens sista gille och ska sedan ta ett välförtjänt sommaruppehåll. Har mer eller mindre spikat nollningspubarna och gått ut med detta till källarmästarna. Vi har även redan börjat ta arbetare för Regattan och UteDischot.

Förra veckan hade ENU en afterwork med Ericsson som blev lyckad. Tetrapak vill ha två lunchföreläsningar till hösten, en under nollningen och en senare för 3-5. De har också pratat om hackathon som Josefine ska kolla upp mer om och kommer bjuda på smaksatt vatten under nollningen. Hade en sommaravlutning på Mop i onsdags.

Nöju har sålt glass. Det gick bra. Vi gick 100 kr plus. Spelemännen hittade 400 kr i gamla sedlar. De ska lösa det. Spelkvällen var i övrigt lyckat. Nöju gör inte fler event under våren utan har i nuläget fullt fokus på utedischot. Erik har arbetat fram utedischo-märket med begränsade färger. Idag hörde premiemax av sig och sa att vi kan göra märket med alla färger och toningar. Så Eriks arbete var onödigt.

Vi i sexet höll vår sista sittning i veckan - Norgesittningen - vilken var väldigt lyckad. Vi har börjat planera inför nollningen, där alla sittningar är spikade. Har haft en del möten med andra sexmästare inför de intersektionella sittningarna.

I SRE pågår fortfarande arbete med CEQ. Jag själv har även börjat jobba lite med att få till en bra överlämning till min efterträdare Edvard. I övrigt händer det inte så mycket i utskottet.

NollU jobbar på för att nollningen ska bli bra.

FVU jobbar på bra.

I InfU har redaktionen haft lite problem med att google filtrerar bort utskicket. Chefredaktören har bett sektionsmedlemmar att markera utskicket som ej skräppost. DDG har grillat på det sista mötet innan sommaren. Picasso har gjort ett snyggt utkast på nollEguiden.
\subp{C}{Ekonomisk rapport}{\info}
Sophia meddelade att det är cirka \SI{800000}{kr} på kontot, det går bra. Inga stora utgifter det senaste. Fakturor från 2016 har kommit in, vilket är oväntat.
\subp{D}{Kåren informerar}{\info}
Kåren har det lugnt och skönt på kontoret. De taggar överlämning till sommaren. Imorgon är det JFF.

Nollegenneralen letar jobbare till nollningen.

Kårordförande uppmanar folk om att fylla i enkät om nollningsperspektiv.
\end{paragrafer}
\p{11}{Kavajer till Vice Entertainer}{}
På grund av att Albin förmodligen ska byta till M-sektionen vill Albin att Vice Entertainer ska få brodyr till kavajer för att representera utskottet bättre under nollningen när Albin inte kan närvara. Kavajerna står Vice Entertainer för själv och sektionen ska stå för kostnaden av brodyr.

Erik \ypa avsätta \SI{1500}{kr} till brodyr på kavajer till Vice Entertainer med beslutsuppföljning till S16/17. Med Erik som ansvarig

F-sektionen invaderade mötet och gav oss presenter.

\p{12}{Elektrotjejer söndag}{\dis}
Erik läste upp från mail skickat mellan Erik och Elektra-ordförande.

Mötet diskuterade ämnet.

\Mba inte hålla någon aktivitet söndag läsvecka -1.

Erik pratade om att Elektra har hållit ett event idag, ett event som egentligen borde hållas från sektionens sida enligt honom.

Mötet diskuterade detta.

\p{13}{Nästa styrelsemöte}{\bes}
{\Mba} nästa styrelsemöte ska äga rum 2017-05-24 12:10 i E:1426.

\p{14}{Beslutsuppföljning}{\bes}
{\Ibfu}

\p{15}{Övrigt}{\dis}
Erik påminnde styrelsen om att inte glömma att städa.

Edvard får gå på JFF.
\p{16}{OFMA}{\bes}
{\mo} förklarade mötet avslutat 12:44.

\end{paragrafer}

\newpage
\hidesignfoot
\begin{signatures}{3}
\signature{\mo}{Mötesordförande}
\signature{\ms}{Mötessekreterare}
\signature{\ji}{Justerare}
\end{signatures}
\end{document}
