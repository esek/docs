\documentclass[10pt]{article}
\usepackage[utf8]{inputenc}
\usepackage[swedish]{babel}

\def\mo{Fredrik Peterson}
\def\ms{Johan Karlberg}
\def\ji{David Uhler Brand}
\def\jii{Axel Voss}

\def\doctype{Protokoll} %ex. Kallelse, Handlingar, Protkoll
\def\mname{Höstterminsmötet} %ex. styrelsemöte, Vårterminsmöte
\def\mnum{HT/17} %ex S02/16, E1/15, VT/13
\def\date{2017-11-14} %YYYY-MM-DD
\def\docauthor{\ms}

\usepackage{../e-mote}
\usepackage{../../../e-sek}

\begin{document}
\showsignfoot

\heading{{\doctype} för {\mname} {\mnum}}

%\naun{}{} %närvarane under
%\nati{} %närvarande till och med
%\nafr{} %närvarande från och med
\section*{Närvarande}
\subsection*{Styrelsen}
\begin{narvarolista}
  \nv{Ordförande}{Erik Månsson}{E14}{}
  \nv{Kontaktor}{Johan Karlberg}{E14}{}
  \nv{Förvaltningschef}{Sophia Grimmeiss Grahm}{BME14}{\nafr{5}}
  \nv{Cafémästare}{Daniel Bakic}{E15}{\naun{1}{15}, \S17D - \S19}
  \nv{Øverphøs}{Niklas Gustafson}{E15}{}
  \nv{SRE-ordförande}{Edvard Carlsson}{E16}{\naun{1}{15}, \S17D - \S19}
  \nv{ENU-ordförande}{Josefine Sandström}{E14}{\nafr{5}}
  \nv{Sexmästare}{Linnea Sjödahl}{BME15}{}
  \nv{Krögare}{Markus Rahne}{BME14}{}
  \nv{Entertainer}{Albin Nyström Eklund}{BME16}{\naun{1}{15}, \S17D - \S19}
\end{narvarolista}

\subsection*{Ständigt adjungerande}
\begin{narvarolista}
    \nv{Talman}{Fredrik Peterson}{E14}{}
\end{narvarolista}

\subsection*{Medlemmar}
\begin{narvarolista}
    \nv{}{Emil Pedersen Lundh}{E17}{}
    \nv{}{Isabella Hansen}{E16}{\nati{16A}}
    \nv{}{Philip Johansson}{E16}{\nati{16A}}
    \nv{}{Jennie Karlsson}{E16}{\nati{16A}}
    \nv{}{Sonja Kenari}{E16}{}
    \nv{}{Johan Vikstrand}{E17}{}
    \nv{}{Tea Kerac}{BME17}{}
    \nv{}{Stephanie Bol}{BME17}{}
    \nv{}{Amanda Gustafsson}{BME17}{}
    \nv{}{Axel Sandqvist}{E17}{\nati{14}}
    \nv{}{Julius Dahlgren}{BME17}{}
    \nv{}{Gustav Jönemo}{E17}{}
    \nv{}{Fanny Månefjord}{BME16}{}
    \nv{}{Lina Samnegård}{BME16}{\nati{13}}
    \nv{}{Jessica Kågeman}{BME16}{}
    \nv{}{Elin Johansson}{BME16}{}
    \nv{}{Rasmus Sobel}{BME16}{}
    \nv{}{Filip Winzell}{BME16}{\nati{17A}}
    \nv{}{Andreas Bennström}{BME16}{}
    \nv{}{David Uhler Brand}{E14}{}
    \nv{}{Emil Bergström}{E17}{}
    \nv{}{Johan Siwerson}{E17}{}
    \nv{}{Mia Cicovic}{BME16}{}
    \nv{}{Henrik Ramström}{E16}{\naun{1}{10}, \S17E - \S19}
    \nv{}{Sanna Nordberg}{BME16}{}
    \nv{}{Matilda Dahlström}{BME16}{}
    \nv{}{Sophia Carlsson}{BME17}{}
    \nv{}{Adam Belfrage}{BME17}{}
    \nv{}{Johan Wendt}{E16}{\nati{15}}
    \nv{}{Davida Åström}{BME17}{}
    \nv{}{Alexander Wik}{BME17}{}
    \nv{}{Mattias Lundström}{E17}{}
    \nv{}{Johannes Larsson}{E16}{}
    \nv{}{Jakob Pettersson}{E17}{}
    \nv{}{Moa Rönnlund}{E17}{}
    \nv{}{Amanda Zarkout}{E17}{}
    \nv{}{William Marnfeldt}{E17}{}
    \nv{}{Carl Spångberg}{E17}{}
    \nv{}{Saga Junwik}{E16}{}
    \nv{}{Malin Heyden}{E16}{}
    \nv{}{Anton Jigsved}{BME16}{}
    \nv{}{Amelie Bäck}{E15}{\nati{16A}}
    \nv{}{Nicklas Norborg Persson}{E14}{\nati{16A}}
    \nv{}{Jonathan Benitez}{E17}{}
    \end{narvarolista}
    \begin{narvarolista}
    \nv{}{Anders Nilsson}{E13}{}
    \nv{}{Tor Hammarbäck}{E17}{}
    \nv{}{Emil Harvig}{BME14}{\nati{15}}
    \nv{}{Axel Voss}{E15}{}
    \nv{}{Axel Sondh}{E15}{\nafr{5}}
    \nv{}{Pontus Landgren}{E14}{\nafr{17A}}
    \nv{}{Tove Börjeson}{E17}{\nafr{16D}}
\end{narvarolista}

\subsection*{Adjungerande}
\begin{narvarolista}
\nv{Kårkontakt}{Caroline Svensson}{}{}
\end{narvarolista}
\newpage
\section*{Protokoll}

\begin{paragrafer}
\p{1}{TaFMÖ}{}
Talman {\mo} förklarade mötet öppnat 17:26.

\emph{Vid mötets början var 55 personer närvarande.}

\p{2}{Val av mötesordförande}{}
Talman {\mo} valdes.

\p{3}{Val av mötessekreterare}{}
Kontaktor {\ms} valdes.

\p{4}{Godkännande av tid och sätt}{}
Tid och sätt godkändes.

\p{5}{Val av två justeringspersoner}{}
Axel Voss nominerade sig själv.

Markus nominerade David Uhler Brand och Anders Nilsson.

Anders Nilsson godtog inte sin nominering.

David Uhler Brand måste gå tidigt.

Johan Karlberg nominerade Henrik Ramström.

Henrik Ramström godtog inte sin nominering.

David Uhler Brand nominerar sig själv.

\textbf{David Uhler Brand och Axel Voss valdes till justerare.}
\p{6}{Adjungeringar}{}
\emph{Caroline Svensson adjungerades.}

\p{7}{Godkännande av dagordningen}{}
%Viktor Persson \ypa lägga till \S17.5 ``Behandling av sen motion: Utökning av antal sökande till posten Co-phøsare''.

\textbf{Mötet beslutade att godkänna föredragningslistan.}

\p{8}{Föregående sektionsmötesprotokoll}{}
\textbf{Mötet beslutade att lägga till protokollet för E01/17 till handlingarna.}

\p{9}{Meddelanden}{}
Anders Nilsson förklarade vad Fullmäktigevalet innebär och att alla skall rösta.

Emil Harvig meddelade att valnämden kommer att stå på ARKAD. Man kan rösta där.
\p{10}{Beslutsuppföljning}{}
Emil Harvig presenterade beslutsuppföljningen av \emph{Fanbärare behöver längre påle.}

Markus Rahne frågade vad det är för träslag. Emil Harvig svarade furu.

David Uhler Brand frågade hur lång pålen blev. \SI{2.4}{m}, lite tyngre men den är bra.

Axel Voss frågade om har E-Sektionen har den längsta pålen. Emil Harvig svarade nej.

\textbf{Mötet beslutade att stryka Fanbärare behöver längre påle från beslutsuppföljningen.}

Erik Månsson presenterade beslutsuppföljningen av \emph{Införandet av arbetskläder för utlåning till funktionärer}.

\textbf{Mötet beslutade att stryka Införandet av arbetskläder för utlåning till funktionärer från beslutsuppföljningen.}

Erik Månsson presenterade beslutsuppföljningen av \emph{Uppdatering av övervakningspolicyn}.

Filip Larsson frågade hur Edekvata övervakades innan? Erik svarade med kameror.

\textbf{Mötet beslutade att stryka Uppdatering av övervakningspolicyn från beslutsuppföljningen samt att ta bort policybeslutet Övervakningspolicy.}

Erik Månsson presenterade beslutsuppföljningen av \emph{Återremittering av
``Utökning av antal sökande till posten Co-phøsare''}.

Erik Månsson \ypa stryka \emph{``Utökning av antal sökande till posten Co-phøsare''} från beslutsuppföljningen.

Filip Larsson frågade hur många poster det skulle utökas till? Från 5 till 6 Co-phøsare.

Rasmus Sobel frågade vad de nuvarande Phøset tycker. De anser att det inte behövs.

\textbf{Mötet beslutade att bifalla yrkandet.}

Albin Nyström Eklund presenterade beslutsuppföljningen av \emph{Inköp av banandräkter}.

\textbf{Mötet beslutade att stryka Inköp av banandräkter från beslutsuppföljningen.}

Sanna Nordberg och Matilda Dahlström presenterade beslutsuppföljningen av \emph{Förbättrad förvaring av sektionens lager}.

David Uhler Brand frågade om den gamla hyllan är slängd?

Sanna Nordberg svarade att de gamla hyllorna i Pump är slängda.

\textbf{Mötet beslutade att stryka Förbättrad förvaring av sektionens lager från beslutsuppföljningen.}

Erik Månsson presenterade beslutsuppföljningen av \emph{Representationsklädsel åt Inspektorn}.

Daniel Bakic frågade om den sitter lika bra på Erik Månsson som Inspektorn? Erik Månsson svarade att den sitter på bättre på Inspektorn.

Markus Rahne frågade om Erik Månsson kan tänka sig bära den någon dag? Svaret var ja.

\textbf{Mötet beslutade att stryka Representationsklädsel åt Inspektorn från beslutsuppföljningen.}

Erik Månsson presenterade beslutsuppföljningen av \emph{Uppgradering av ljudsystemet i Edekvata}.

\textbf{Mötet beslutade att stryka Uppgradering av ljudsystemet i Edekvata från beslutsuppföljningen.}

Erik Månsson presenterade beslutsuppföljningen av \emph{Äskning av pengar till mentorsprogram}.

\textbf{Mötet beslutade att skjuta upp Äskning av pengar till mentorsprogram till VT/18.}

Erik Månsson presenterade beslutsuppföljningen av \emph{Inköp av PA-toppar}.

Sonja Kenari frågad vad PA-toppar är? Erik Månsson svarade att det är högtalare.

\textbf{Mötet beslutade att stryka Inköp av PA-toppar från beslutsuppföljningen.}

\p{11}{Utskottsrapporter}{}
Mötet undrade vad NollU har gjort mer än det som står i utskottsrapporten.

Niklas Gustafson presenterade sin rapport.

\p{12}{Uppföljning av verksamhetsplan}{}
Erik Månsson berättade vad verksamhetsplanen är.

\p{13}{Ekonomisk rapport}{}
Sophia Grimmeiss Grahm gav en rapport för Sektionens ekonomi.

\p{14}{Uttag ur Sektionens fonder sedan förra terminsmötet}{}
Sophia Grimmeiss Grahm rapporterade om uttagen ur Sektionens fonder sedan förra terminsmötet.

\p{15}{Resultatrapport från första halvan av verksamhetsåret}{}
Sophia Grimmeiss Grahm presenterade resultatrapporten från första halvan av verksamhetsåret.

\textbf{Erik Månsson \ypa ajournera mötet i 30 minuter.}

\textbf{Mötet beslutade att bifalla yrkandet.}

\emph{Mötet ajournerades 18:08 och återupptogs 18:54.}

\p{16}{Behandling av motioner}{}
    \begin{paragrafer}
        \subp{A}{Döp om Kontaktor till Kommunikator}{}
        William Marnfeldt presenterade motionen.

        Erik Månsson presenterade styrelsens svar på motionen.

        Mötet diskuterade frågan.

        \textbf{Mötet beslutade att avslå motionen i sin helhet.}

        \textbf{Mötet beslutade att bifalla styrelsens yrkande om att William Marnfeldt ska erkänna att han hade fel och ge Johan Karlberg en puss på kinden.}
        \subp{B}{Ändra namnet på likabehandlingsombudet}{}
        Fanny Månefjord presenterade motionen.

        Erik Månsson presenterade styrelsens svar på motionen.

        Mötet diskuterade frågan.

        \textbf{Mötet beslutade att bifalla motionen i sin helhet.}

        \subp{C}{Namnbyte av funktionärspost}{}
        Fredrik Peterson presenterade motionen.

        Mötet diskuterade frågan.

        \textbf{Mötet beslutade att avslå motionen i sin helhet.}

        \subp{D}{Veckoliga Studiekvällar med tilltugg}{}
        William Marnfeldt och Filip Larsson presenterade motionen.

        Erik Månsson presenterade styrelsens svar på motionen.

        Fanny Månefjord frågade om de ska få \SI{10000}{kr} även om de äskar pengar från SVL. Erik Månsson svarade ja.

        Adam Belfrage frågade om det är tänkt att det skall ske samma tid som under nollningen. William svarde att det är tanken.

        Malin Heyden frågade vilka ska laga maten.  William och Filip kommer att starta projektet och folk får vara med svarade William.

        David Uhler Brand frågade om hur uppskattningen till \SI{10000}{kr} bestämdes. I samarbete med SRE svarade William.

        \textbf{Mötet beslutade att bifalla motionen i sin helhet.}

    \end{paragrafer}
\subp{17}{Behandling av propositioner}{}
    \begin{paragrafer}
        \subp{A}{Budgetförslag för 2018}{}
        Erik Månsson presenterade propositionen.

        Anders Nilsson frågade om styrelsen har gett någon tanke på hur sektionsmedlemmarna kan spendera pengar. Erik svarade ja.

        Fredrik Peterson noterade att funktionärsvården är ökad och frågade om styrelsen har förslag på hur de ska användas. Erik Månsson svarade att bland annat kick-off och kick-out är något som skall prioriteras mer.

        Fredrik Peterson frågade varför tryck till Vice-kavajer budgeteras till styrelsen. Erik svarade att målet är att Vice skall komma närmre styrelsen.

        Mötet diskuterade frågan.

        \textbf{Mötet beslutade att bifalla propositionen i sin helhet.}

        \emph{Mötet ajournerades 19:41 och återupptogs 19:50.}

        \subp{B}{Verksamhetsplansförslag för 2018}{}
        Erik Månsson presenterade propositionen.

        Erik svarade på frågor.

        \textbf{Mötet beslutade att bifalla propositionen i sin helhet.}

        \subp{C}{Flytta posten alumniansvarig till ENU}{}
        Josefine Sandström presenterade propositionen.

        Mötet diskuterade frågan.

        \textbf{Styrelsen \ypa ersätta den sista att-satsen med att \emph{ändringarna börjar att gälla verksamhetsåret 2018.}}

        \textbf{Mötet beslutade att bifalla propositionen med ändringsyrkandet.}

        \subp{D}{Barmästare i E6}{}
        Linnea Sjödahl presenterade propositionen.

        \textbf{Styrelsen \ypa ersätta den sista att-satsen med att \emph{ändringarna börjar att gälla verksamhetsåret 2018.}}

        Fanny Månefjord frågade hur nomineringar hanteras. Fredrik Peterson svarade att det är upp till Valberedningen.

        Niklas Gustafson ansåg att nomineringen fortfarande är relevant.

        Adam Belfrage frågade vad vice är ansvariga för idag. Linnea svarade att de framförallt är ansvariga för planering och mycket arbete i baren.

        Pontus Landgren påpekade att vad Valberedningen har ställt för frågor är irrelevant då de inte får diskutera vilka frågor som har ställts.

        \textbf{Mötet beslutade att bifalla propositionen med ändringsyrkandet.}

        \subp{E}{Omstrukturering av CM}{}
        Daniel Bakic presenterade propositionen.

        Fanny Månefjord frågade vem som ska ansvara för att få in funktionärerna och vad som kommer att hända om posterna inte fylls.

        Daniel Bakic svarade att de kommer att hanterar brist på ansvariga likt hur de har gjort denna läsperioden, stänga tidigare och liknande.

        Mötet diskuterade frågan.

        Emil Pedersen Lund berättade att F-sektionen har samma upplägg. Det fungerar ganska bra.

        \textbf{Mötet beslutade att bifalla propositionen i sin helhet.}

        \subp{F}{Namnbyte av Flickor på Teknis}{}
        Erik Månsson presenterade propositionen.

        \textbf{Mötet beslutade att bifalla propositionen i sin helhet.}

        \subp{G}{Uppdatering av postbeskrivning i InfU}{}
        Johan Karlberg presenterade propositionen.

        \textbf{Mötet beslutade att bifalla propositionen i sin helhet.}

        \subp{H}{Uppdatering av postbeskrivning i Nöjesutskottet}{}
        Albin Nyström Eklund presenterade propositionen.

        Adam Belfrage frågade vad de som har sökt Speleman skall göra. Albin svarade att Fritidsledare tar över det ansvaret och att det är ett alternativ.

        \textbf{Styrelsen \ypa ersätta den sista att-satsen med att \emph{ändringarna börjar att gälla verksamhetsåret 2018.}}

        \textbf{Mötet beslutade att bifalla propositionen med ändringsyrkandet.}

    \end{paragrafer}

\p{18}{Övrigt}{}
\textbf{Rasmus Sobel \ypa ta 15 minuters paus.}

Mötet diskuterade kavajer och brodyr till Vice.

William Marnfeldt frågade om det blir ett luciatåg i år?

Tove Börjeson svarade möjligtvis, hon skall kolla.

Mötet diskuterade Vice-Kavajer.

\textbf{Mötet beslutade att avslå yrkandet.}
\p{19}{TaFMA}{}
Talman {\mo} förklarade mötet avslutat 20:48.

\emph{Vid mötets slut var 53 personer närvarande.}

\end{paragrafer}

%\newpage
\hidesignfoot
\begin{signatures}{4}
\signature{\mo}{Mötesordförande}
\signature{\ms}{Mötessekreterare}
\signature{\ji}{Justerare}
\signature{\jii}{Justerare}
\end{signatures}
\end{document}
