\documentclass[10pt]{article}
\usepackage[utf8]{inputenc}
\usepackage[swedish]{babel}

\def\mo{Johan Westerlund}
\def\ms{Erik Månsson}
\def\ji{Anders Nilsson}
\def\jii{Pontus Landgren}

\def\doctype{Protokoll} %ex. Kallelse, Handlingar, Protkoll
\def\mname{Höstterminsmötet} %ex. styrelsemöte, Vårterminsmöte
\def\mnum{HT/17} %ex S02/16, E1/15, VT/13
\def\date{2017-11-14} %YYYY-MM-DD
\def\docauthor{\ms}

\usepackage{../e-mote}
\usepackage{../../../e-sek}

\begin{document}
\showsignfoot

\heading{{\doctype} för {\mname} {\mnum}}

%\naun{}{} %närvarane under
%\nati{} %närvarande till och med
%\nafr{} %närvarande från och med
\section*{Närvarande}
\subsection*{Styrelsen}
\begin{narvarolista}
  \nv{Ordförande}{Erik Månsson}{E14}{}
  \nv{Kontaktor}{Johan Karlberg}{E14}{}
  \nv{Förvaltningschef}{Sophia Grimmeiss Grahm}{BME14}{}
  \nv{Cafémästare}{Daniel Bakic}{E15}{}
  \nv{Øverphøs}{Niklas Gustafson}{E15}{\nafr{10A}}
  \nv{SRE-ordförande}{Edvard Carlsson}{E16}{}
  \nv{ENU-ordförande}{Josefine Sandström}{E14}{}
  \nv{Sexmästare}{Linnea Sjödahl}{BME15}{}
  \nv{Krögare}{Markus Rahne}{BME14}{}
  \nv{Entertainer}{Albin Nyström Eklund}{BME16}{}
\end{narvarolista}

\subsection*{Ständigt adjungerande}
\begin{narvarolista}
    \nv{Talman}{Fredrik Peterson}{E14}{}
\end{narvarolista}

\subsection*{Medlemmar}
\begin{narvarolista}
    %\nv{Post}{Namn}{Klass}{}
\end{narvarolista}

\newpage
\section*{Protokoll}

\begin{paragrafer}
\p{1}{TaFMÖ}{}
Talman {\mo} förklarade mötet öppnat 17:26.

\emph{Vid mötets början var 55 personer närvarande.}

\p{2}{Val av mötesordförande}{}
Talman {\mo} valdes.

\p{3}{Val av mötessekreterare}{}
Kontaktor {\ms} valdes.

\p{4}{Godkännande av tid och sätt}{}
Tid och sätt godkändes.

\p{5}{Val av två justeringspersoner}{}
Axel Voss nominerade sig själv.

Markus nominerar David och Anders.

De godtar inte sina nominering.

Johan nominerar Henrik Ramström.

Han godtar inte sin nominering.

David Brand nominerar sig själv.

\valavj
\p{6}{Adjungeringar}{}
\emph{Caroline Svensson}

\p{7}{Godkännande av dagordningen}{}
%Viktor Persson \ypa lägga till \S17.5 ``Behandling av sen motion: Utökning av antal sökande till posten Co-phøsare''.

\textbf{\Mba godkänna föredragningslistan.}

\p{8}{Föregående sektionsmötesprotokoll}{}
\textbf{\Mba lägga till protokollet för E01/17 till handlingarna.}

\p{9}{Meddelanden}{}
Anders Nilsson förklarade vad fm är och att alla skall rösta.

Emil; imorgon står de även på arkad, dvs valnämnden på kåren.
\p{10}{Beslutsuppföljning}{}
Emil Harvig presenterade beslutsuppföljningen av \emph{Fanbärare behöver längre påle.}.

Markus Rahne; vad är det för träslag. Furu.

David; hur lång blev pålen, vad är höjdskildnaden? 2.4 m. Lite tyngre men den är bra.

Axel Voss; har e-sek den längsta pålen nu? Nej.

\textbf{\Mba bifalla att-satserna i beslutsuppföljningen.}

Erik Månsson presenterade beslutsuppföljningen av \emph{Införandet av arbetskläder för utlåning till funktionärer}.

\textbf{\Mba bifalla att-satserna i beslutsuppföljningen.}

Erik Månsson presenterade beslutsuppföljningen av \emph{Uppdatering av övervakningspolicyn}.

Filip Larsson; Hur övervakades edekvata innan? Med kameror.

\textbf{\Mba bifalla att-satserna i beslutsuppföljningen.}

Erik Månsson presenterade beslutsuppföljningen av \emph{Återremittering av
``Utökning av antal sökande till posten Co-phøsare''}.

Erik Månsson \ypa stryka ``Utökning av antal sökande till posten Co-phøsare'' från besultsuppföljningen.

Filip; hur många poster skulle det utökas till? 5 till 6.

Rasmus Sobel; vad tycker phöset? Nej.

\textbf{\Mba bifalla yrkandet.}

Albin Nyström Eklund presenterade beslutsuppföljningen av \emph{Inköp av banandräkter}.

\textbf{\Mba bifalla att-satserna i beslutsuppföljningen.}

Sanna Nordberg och Matilda Dahlström presenterade beslutsuppföljningen av \emph{Förbättrad förvaring av sektionens lager}.

David; är den gamla hyllan slängd?

Sanna; Naeee?? De hyllorna i pump är iaf slängda

\textbf{\Mba bifalla att-satserna i beslutsuppföljningen.}

Erik Månsson presenterade beslutsuppföljningen av \emph{Representationsklädsel åt Inspektorn}.
Daniel; Sitter den lika bra på dig som inspektorn? Bättre på inspektorn.

Markus; Kan du se dig själv bära den någon dag? Ja
\textbf{\Mba bifalla att-satserna i beslutsuppföljningen.}

Erik Månsson presenterade beslutsuppföljningen av \emph{Uppgradering av ljudsystemet i Edekvata}.

\textbf{\Mba bifalla att-satserna i beslutsuppföljningen.}

Erik Månsson presenterade beslutsuppföljningen av \emph{Äskning av pengar till mentorsprogram}.

\textbf{\Mba bifalla att-satserna i beslutsuppföljningen.}

Erik Månsson presenterade beslutsuppföljningen av \emph{Inköp av PA-toppar}.

David; vilka toppar köptes in? Vet ej.

Sonja; vad är PA-toppar? Högtalare.
\textbf{\Mba bifalla att-satserna i beslutsuppföljningen.}

\p{11}{Utskottsrapporter}{}
Vad har NollU gjort?

Niklas presenterade sin rapport.
%\emph{Ingen hade några kommentarer på utskottsrapporterna.}

\p{12}{Uppföljning av verksamhetsplan}{}
Erik Månsson berättade vad verksamhetsplanen är.

Daniel; det är ganska självklara saker.

\emph{Ingen hade några kommentarer på uppföljningen av verksamhetsplanen.}

\p{13}{Ekonomisk rapport}{}
Sophia Grimmeiss Grahm gav en rapport för Sektionens ekonomi.

Daniel; Vad är iZettle. Det är vårt kortbetalningssystem.

\p{14}{Uttag ur Sektionens fonder sedan förra terminsmötet}{}
Sophia Grimmeiss Grahm berättade om uttagen ur Sektionens fonder sedan förra terminsmötet.

\p{15}{Resultatrapport från första halvan av verksamhetsåret}{}
Sophia Grimmeiss Grahm presenterade resultatrapporten från första halvan av verksamhetsåret.

Erik Månsson \ypa ajournera mötet i 30 minuter.

\textbf{\Mba bifalla yrkandet.}

\emph{Mötet ajournerades 18:08 och återupptogs 18:54.}

\p{16}{Behandling av motioner}{}
    \begin{paragrafer}
        \subp{A}{Döp om Kontaktor till Kommunikator}{}
        William Marnfeldt presenterade motionen.

        Erik Månsson presenterade styrelsens svar på motionen.

        Mötet diskuterade detta.

        \Mba avslå motionen i sin helhet.

        \Mba bifalla det sista att i styrelsens svar.

        \subp{B}{Ändra namnet på likabehandlingsombudet}{}
        Fanny Månefjord presenterade motionen.

        Erik Månsson presenterade styrelsens svar på motionen.

        Mötet diskuterade frågan.

        \textbf{\Mba bifalla motionen i sin helhet.}

        \subp{C}{Namnbyte av funktionärspost}{}
        Fredrik Peterson presenterade motionen.

        Erik Månsson presenterade styrelsens svar på motionen.

        Carl; är det inte patent på det namnet? Fredrik; idk.

        David; Om man inte nytjar namnet i likande situationer så är det lugnt.

        David; När bytte vi sist namn? Fredrik; 2015.

        William; vad är Umph-meister? DJ

        Mötet diskuterade frågan.
        \Mba ta attsatserna i klump.

        \textbf{\Mba avslå motionen i sin helhet.}

        \subp{D}{Veckoliga Studiekvällar med tilltugg}{}
        William Marnfeldt och Filip Larsson presenterade motionen.

        Erik Månsson presenterade styrelsens svar på motionen.

        \Mba ta attsatserna i klump.
        Fanny, ska de få 10k även om de äskar?

        Adam; Ska det ske samma tid som under nollningen? typ

        Malin; vilka ska laga maten?  William och Filip

        David; vad kommer 10 000 ifrån? Samarbete med SRE.

        \textbf{\Mba bifalla motionen i sin helhet.}

    \end{paragrafer}
\subp{17}{Behandling av propositioner}{}
    \begin{paragrafer}
        \subp{A}{Budgetförslag för 2018}{}
        Erik Månsson presenterade propositionen.

        Anders; har man gett någon tanke på hur sektionsmedlemmarna kan spendera pengar.

        Erik; ja.

        Fredrik; funkvård är ökad. Konkreta tankar på hur de ska användas.

        Erik; bland annat kick-off och kick-out.

        Fredrik; kläder under styrelsen. Personerna i fråga är inte med i styrelsen.

        Erik; Vice skall komma närmre styrelsen.

        Mötet diskuterade frågan.

        \textbf{\Mba bifalla propositionen i sin helhet.}

        \emph{Mötet ajournerades 19:41 och återupptogs 19:50.}

        \subp{B}{Verksamhetsplansförslag för 2018}{}
        Erik Månsson presenterade propositionen.

        Erik svarade på frågor.

        \textbf{\Mba bifalla propositionen i sin helhet.}

        \subp{C}{Flytta posten alumniansvarig till ENU}{}
        Josefine Sandström presenterade propositionen.

        Mötet diskuterade frågan.

        Styrelsen \ypa strycka sista att och att istället det ska börja gälla nästa år.

        \textbf{\Mba bifalla propositionen med tilläggsyrkandet.}

        \subp{D}{Barmästare i E6}{}
        Linnea Sjödahl presenterade propositionen.

        Styrelsen \ypa strycka sista att och att istället det ska börja gälla nästa år.

        Fanny; Nomineringar; det är vbs ansvar.

        Niklas; nomineringen är fortfarande relevant.

        Adam; Vad är vice ansvariga för idag? Planering och mycket i bar.

        Sanna; har de frågat hur de hanterar en bar? Nej.

        Amanda; De som är nominerade, vad händer nu? De kandiderar på valmötet. Det är upp till VB.

        Pontus; Vad VB har ställt för frågor osv är orelevant då de inte får diskutera vilka frågor som har ställts.

        David; Valbereds vice? Fredrik; ja.

        Styrelsen \ypa strycka sista att och att istället det ska börja gälla nästa år.

        \textbf{\Mba bifalla propositionen med tilläggsyrkandet.}

        \subp{E}{Omstrukturering av CM}{}
        Daniel Bakic presenterade propositionen.

        Fanny; vem ska ansvara för att få in funktionärerna.
        Daniel; Så som vi har gjort denna LP.

        Mötet diskuterade frågan.

        Emil P Lund; F-sektionen har samma upplägg. Det fungerar ganska bra.

        \textbf{\Mba bifalla propositionen i sin helhet.}

        \subp{F}{Namnbyte av Flickor på Teknis}{}
        Erik Månsson presenterade propositionen.

        \textbf{\Mba bifalla propositionen i sin helhet.}

        \subp{G}{Uppdatering av postbeskrivning i InfU}{}
        Johan Karlberg presenterade propositionen.

        \textbf{\Mba bifalla propositionen i sin helhet.}

        \subp{H}{Uppdatering av postbeskrivning i Nöjesutskottet}{}
        Albin Nyström Eklund presenterade propositionen.

        Adam; De som har sökt speleman nu, vad gör de?

        Söker annan post.

        Filip; karnevalsmalaj ansvar för överlämningen.

        Albin \ypa att det skall börja gälla vid årsskiftet.

        \textbf{\Mba bifalla propositionen i sin helhet.}

    \end{paragrafer}

\p{18}{Övrigt}{}
Rasmus \ypa 15 minuters paus.

Tor; fler borde få "kavajer med märken".

Henrik Ramström; märken inom utskott för att få utmärkelse.

Sanna; Det är konstigt att vice får kavaj.

Erik; bli funktionär och gör saker.

Erik; Vice, Vice skall komma närmre styret.

William; Blir det luciatåg i år?

Tove; Möjligtvis. Hon skall kolla.

Henrik Ramström; Håller med sanna.

Albin; vice-kavaj kommer från i år då nollu då de skulle ta större plats. Det var nice.

Sophia; samma problem finns inom FVU med skattm

Philip; finns det riktlinjer hur frack/kavaj ska se ut?

William; Undrade hur mycket brodyren kostade, Erik svarade

Jacob; pointerade att vi diskuterar vilken post och inte vilken individ som ska få kavaj

Sobel; vill att det ska vara så som det bestämdes i budgeten där alla vice har möjlighet till kavaj

Erik; vi testar nästa år.

Mötet diskuterade Vice-Kavajer.

\Mba avslå yrkandet.
\p{19}{TaFMA}{}
Talman {\mo} förklarade mötet avslutat 20:48.

\emph{Vid mötets slut var 51 personer närvarande.}

\end{paragrafer}

%\newpage
\hidesignfoot
\begin{signatures}{4}
\signature{\mo}{Mötesordförande}
\signature{\ms}{Mötessekreterare}
\signature{\ji}{Justerare}
\signature{\jii}{Justerare}
\end{signatures}
\end{document}
