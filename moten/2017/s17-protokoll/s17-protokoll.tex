\documentclass[10pt]{article}
\usepackage[utf8]{inputenc}
\usepackage[swedish]{babel}

\def\mo{Erik Månsson}
\def\ms{Johan Karlberg}
\def\ji{Sophia Grimmeiss Grahm}
%\def\jii{}

\def\doctype{Protokoll} %ex. Kallelse, Handlingar, Protkoll
\def\mname{styrelsemöte} %ex. styrelsemöte, Vårterminsmöte
\def\mnum{S17/17} %ex S02/16, E1/15, VT/13
\def\date{2017-08-31} %YYYY-MM-DD
\def\docauthor{\ms}

\usepackage{../e-mote}
\usepackage{../../../e-sek}

\begin{document}
\showsignfoot

\heading{{\doctype} för {\mname} {\mnum}}

%\naun{}{} %närvarane under
%\nati{} %närvarande till och med
%\nafr{} %närvarande från och med
\section*{Närvarande}
\subsection*{Styrelsen}
\begin{narvarolista}
\nv{Ordförande}{Erik Månsson}{E14}{}
\nv{Kontaktor}{Johan Karlberg}{E14}{}
\nv{Förvaltningschef}{Sophia Grimmeiss Grahm}{BME14}{}
%\nv{Cafémästare}{Daniel Bakic}{E15}{}
%\nv{Øverphøs}{Niklas Gustafson}{E15}{}
\nv{SRE-ordförande}{Edvard Carlsson}{E16}{}
\nv{ENU-ordförande}{Josefine Sandström}{E14}{}
\nv{Sexmästare}{Linnea Sjödahl}{BME15}{}
%\nv{Krögare}{Markus Rahne}{BME14}{}
\nv{Entertainer}{Albin Nyström Eklund}{BME16}{}
\end{narvarolista}

\subsection*{Ständigt adjungerande}
\begin{narvarolista}
%\nv{Kårordförande}{Linus Hammarlund}{}{}
%\nv{Kårrepresentant}{Anders Nilsson}{}{}
\nv{Kårrepresentant}{Caroline Svensson}{}{}
\nv{Kårrepresentant}{Agnes Sörliden}{\nafr{9}}{}
%\nv{Valberedningens ordförande}{Elin Magnusson}{}{}
%\nv{Skattmästare}{Olle Oswald}{}{}
%\nv{Kårrepresentant}{Daniel Damberg}{}{}
%\nv{Kårrepresentant}{John Alvén}{}{}
%\nv{Talman}{Fredrik Peterson}{E14}{}
%\nv{Elektras Ordförande}{Elisabeth Pongratz}{}{}
%\nv{Inspektor}{Monica Almqvist}{}{}
\end{narvarolista}

\begin{comment}
\subsection*{Adjungerande}
\begin{narvarolista}
%\nv{Post}{Namn}{Klass}{}
\end{narvarolista}
\end{comment}

\section*{Protokoll}
\begin{paragrafer}
\p{1}{OFMÖ}{\bes}
Ordförande {\mo} förklarade mötet öppnat 12:16.

\p{2}{Val av mötesordförande}{\bes}
{\valavmo}

\p{3}{Val av mötessekreterare}{\bes}
{\valavms}

\p{4}{Val av justeringsperson}{\bes}
{\valavj}

\p{5}{Godkännande av tid och sätt}{\bes}
{\tosg}

\p{6}{Adjungeringar}{\bes}
{\ingaadj}

%Förnamn Efternamn adjungerades

\p{7}{Godkännande av dagordningen}{\bes}
%Dagordningen godkändes.
Erik \ypa lägga till \S13 ``Flyingbiljetter''.

Föredragningslistan godkändes med yrkandet.
%Föredragningslistan godkändes med samtliga yrkanden.

\p{8}{Föregående mötesprotokoll}{\bes}
\latillprot{S14/17}

\latillprot{S15/17}

\latillprot{S16/17}
%\ingaprot

\p{9}{Fyllnadsval och entledigande av funktionärer}{\bes}
\begin{fyllnadsval} %"Inga fyllnadsval." fylls i automatiskt
%\fval{Namn}{Post}
%\entl{Namn}{Post}
\end{fyllnadsval}

\p{10}{Rapporter}{}
\begin{paragrafer}
\subp{A}{Hur mår alla?}{\info}
Punkten protokollfördes ej.
\subp{B}{Utskottsrapporter}{\info}
E6 har haft en väldigt intensiv vecka med tre sittningar - Välkomst, Sjungbok och sittningen med F. Välkomst gick inte så bra som de hade hoppats, Jessica (preferens) var sjuk och de började för sent så maten blev väldigt försenad. De lärde sig av sitt misstag dock och de andra två sittningarna, särskilt Sjungbok, gick väldigt bra!

Samarbetet med andra sexmästerier är alltid något de kan jobba på och även om sittningen med F gick smidigt kunde arbetet absolut varit mer tidseffektivt, vilket de tar med sig.

Utedischot gik bra! De sålde för \SI{50000}{kr} mer än förra året. NöjU har inte så mycket nu de närmaste veckorna.

SRE har haft sin workshop, de var ont om folk så de fick lite stöd från kåren. De flesta Edvard har pratat med var nöjda. Pluggkvällen gick bra. Edvards första SRX-möte gick bra.

ENU grillade under svep förra torsdagen, de fick lite hjälp av phaddrar också precis som Edvard. Nu förbereder de sig för gästföreläsningar.

FVU har bokfört lite och sålt några fler overaller, men annars är det rätt lugnt.

Erik informerar om att mentorsutbildningen är ikväll.

InfU har inte gjort något speciellt.
\subp{C}{Ekonomisk rapport}{\info}
\SI{730000}{kr} på konto men det ändras ganska mycket från dag till dag.
\subp{D}{Kåren informerar}{\info}
Agnes berättade att de hade sitt första styrelsemöte i måndags. De diskuterade stödmedlemskap, det var den stora frågan på mötet. I övrigt planerade de in datum.

Caroline berättade att de har fått en balmästare, så balen kommer att bli av. Det behövs fortfarande en balgrupp.

Bilder från utedischot och draggningen finns nu på Facebook.
\end{paragrafer}

\p{11}{Eventuellt samarbete mellan E-shop och LED-café}{\dis}
Punkten sköts upp på grund av att Daniel inte var närvarande.
\p{12}{Hälsa på Ulla}{\dis}
Punkten sköts upp på grund av att Daniel inte var närvarande.

\p{13}{Flyingbiljetter}{\dis}
Det har blivit fel med priserna på flyingbiljetter. Ett pris sades från början som sedan ändrades. E, V och A hann att sälja biljetterna för mindre innan det uppdaterade priset kom.
\newpage
Det var sagt innan att nollningsansvarig på kåren skulle ge information om priset på ett bestämt datum, fel information kom istället från FlyING-generalen och nollningsansvarig från kåren släppte ingen information.

ING har ansvaret för allt som händer i Helsingborg och kåren sköter bussarna.

Enligt jurister som E-phøs pratat med är det är olagligt att ta ytterligare 60kr för biljetten i efterhand. Därför (och p.g.a. andra anledningar) har E helt uteslutet alternativet att göra detta.

Styrelsens åsikt är att kåren bör ta ansvar.
\p{13}{Nästa styrelsemöte}{\bes}
{\Mba} nästa styrelsemöte ska äga rum 2017-09-07 12:10 i E:1407.

\p{14}{Beslutsuppföljning}{\bes}
{\Ibfu}

\p{15}{Övrigt}{\dis}
Sophia berättade att det börjar bli dags att fakturera F-sektionen för vårat F.

Mötet diskuterade olika förslag på hur fakturan ska överges.
\p{16}{OFMA}{\bes}
{\mo} förklarade mötet avslutat 12:42.

\end{paragrafer}

%\newpage
\hidesignfoot
\begin{signatures}{3}
\signature{\mo}{Mötesordförande}
\signature{\ms}{Mötessekreterare}
\signature{\ji}{Justerare}
\end{signatures}
\end{document}
