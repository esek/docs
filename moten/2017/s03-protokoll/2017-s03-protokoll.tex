\documentclass[10pt]{article}
\usepackage[utf8]{inputenc}
\usepackage[swedish]{babel}

\def\mo{Erik Månsson}
\def\ms{Johan Karlberg}
\def\ji{Niklas Gustafson}
%\def\jii{}

\def\doctype{Protokoll} %ex. Kallelse, Handlingar, Protkoll
\def\mname{styrelsemöte} %ex. styrelsemöte, Vårterminsmöte
\def\mnum{S03/17} %ex S02/16, E1/15, VT/13
\def\date{2017-02-02} %YYYY-MM-DD
\def\docauthor{\ms}

\usepackage{../e-mote}
\usepackage{../../../e-sek}

\begin{document}
\showsignfoot

\heading{{\doctype} för {\mname} {\mnum}}

%\naun{}{} %närvarane under
%\nati{} %närvarande till och med
%\nafr{} %närvarande från och med
\section*{Närvarande}
\subsection*{Styrelsen}
\begin{narvarolista}
\nv{Ordförande}{Erik Månsson}{E14}{}
\nv{Kontaktor}{Johan Karlberg}{E14}{}
\nv{Förvaltningschef}{Sophia Grimmeiss Grahm}{BME14}{}
\nv{Cafémästare}{Daniel Bakic}{E15}{}
\nv{Øverphøs}{Niklas Gustafson}{E15}{}
\nv{SRE-ordförande}{Pontus Landgren}{E14}{}
\nv{ENU-ordförande}{Josefine Sandström}{E14}{\nafr{4}}
\nv{Sexmästare}{Linnea Sjödahl}{BME15}{}
\nv{Krögare}{Markus Rahne}{BME14}{}
\nv{Entertainer}{Albin Nyström Eklund}{BME16}{}
\end{narvarolista}

\begin{comment}
\subsection*{Ständigt adjungerande}
\begin{narvarolista}
%\nv{Kårordförande}{Linus Hammarlund}{}{}
\nv{Kårrepresentant}{Jacob Karlsson}{}{}
\nv{Aktivietssamordnare}{Lovisa Majtorp}{}{}
%\nv{Valberedningens ordförande}{Elin Magnusson}{}{}
%\nv{Skattmästare}{Olle Oswald}{}{}
%\nv{Kårrepresentant}{Daniel Damberg}{}{}
%\nv{Kårrepresentant}{John Alvén}{}{}
%\nv{Talman}{Fredrik Peterson}{E14}{}
%\nv{Elektras Ordförande}{Elisabeth Pongratz}{}{}
%\nv{Inspektor}{Monica Almqvist}{}{}
\end{narvarolista}
\end{comment}
\begin{comment}
\subsection*{Adjungerande}
\begin{narvarolista}
%\nv{Post}{Namn}{Klass}{}
\end{comment}
\end{narvarolista}

\section*{Protokoll}
\begin{paragrafer}
\p{1}{OFMÖ}{\bes}
Ordförande {\mo} förklarade mötet öppnat 12:18.

\p{2}{Val av mötesordförande}{\bes}
{\valavmo}

\p{3}{Val av mötessekreterare}{\bes}
{\valavms}

\p{4}{Val av justeringsperson}{\bes}
{\valavj}

\p{5}{Godkännande av tid och sätt}{\bes}
{\tosg}

\p{6}{Adjungeringar}{\bes}
{\ingaadj}
%Förnamn Efternamn adjungerades

\p{7}{Godkännande av dagordningen}{\bes}
Dagordningen godkändes.
%Fredrik \ypa att lägga till \S18b ``Teknikfokus utnyttjande av LED-café''.
%Föredragningslistan godkändes med yrkandet.
%Föredragningslistan godkändes med samtliga yrkanden.

\p{8}{Föregående mötesprotokoll}{\bes}
\latillprot{S01/17}
%\ingaprot

\p{9}{Fyllnadsval och entledigande av funktionärer}{\bes}
\begin{fyllnadsval} %"Inga fyllnadsval." fylls i automatiskt
\entl{Alexander Arvebratt}{Diod}
\entl{Artur Lidström}{Umphmeister}
\end{fyllnadsval}

\p{10}{Rapporter}{}
\begin{paragrafer}
\subp{A}{Hur mår alla?}{\info}
Punkten protokollförs ej.

\subp{B}{Utskottsrapporter}{\info}
FvU har bokfört lite, kickoff är på gång och Sophia har samtalat lite med phøset om ekonomi och lokalbokningar.\\
InfU har planerat sammarbetet mellan vice och ordförande samt varit på det första kollegiemötet.\\
ENU meddelade att de flesta av eventgrupperna har kommit igång och att det snart är CV-fotografering.\\
NöjU har blivit nekade att åka på DÖMD. Därav planeras andra coola events, så som ``Agent 00E'' som ska hållas den 1 april. Den årliga bowlingturneringen återkommer i år också. Spelkvällar är på gång.\\
SRE har sedan S02 utbildat de nya årskursrepresentanterna i CEQ-granskning. Granskat några av CEQ:er, har en del kvar. Lottat fram vinnare i tävlingen, de tillkännages snart. Har även varit på möte angående uppstart nollning och programledning E. Preliminär pluggkväll 22/2.\\
CMs verksamhet har gått bra, de har sålt slut på maten varje dag och har fått in fler dioder. De ska sälja sallad till FPT och på Fettisdagen ska de fixa semlor.\\
E6 ska anordna sittning för Athena (kvinnligt nätverk vid LTH) den 2/2.\\
KM har spikat datum för gillen och varit på alkoholutbildning.\\
NollU planerat sitt temasläpp.\\
\subp{C}{Ekonomisk rapport}{\info}
Sophia meddelar att det inte händer så mycket ännu, det är mest saker kvar från förra året och saker från CM. Det ser bra ut.
\subp{D}{Kåren informerar}{\info}
Det har inte varit några möten på kåren sedan sist. Enkäten ``Hur mår teknologen'' har gått ut till 38000 personer, alla teknologer i Sverige. Lovisa meddelar mötet om vad hon gör på kåren, hon är aktivietssamordnare. Till helgen ska Lovisa på skipthe.
Rapporten från Universitetskanslersämbete har kommit ut, inga rekomendationer på minskningar i bidrag till kåren kunde ses.
\end{paragrafer}

\p{11}{Examensbankett}{\bes}
Punkten sköts upp till nästa möte.

\p{12}{Jubileum}{\bes}
Erik \ypa sätta Anders Nilsson ansvarig för årets Jubileum och sätta det på beslutsuppföljning till möte 10.\\
\Mbaby
\p{13}{Serveringsansvariga}{\dis}
Sophia föreslog att krögartrion och sexmästartrion är serveringsansvariga på sektionen, mötet var eniga om att det var en bra idé.
\p{14}{Delning av testamente}{\dis}
Sophia ville att alla testamenten ska vara tillgängliga för alla i styrelsen.\\
Mötet var överens om att det var en bra idé.\\
Erik gjorde en mapp i driven direkt.
\p{15}{Attestering}{\dis}
Erik \ypa ta bort möjligheten att attestera muntligt på samtliga dokument ska tas bort.\\
\Mbaby\\
Mötet var överens om att skriva riktlinjer för att man inte ska kunna fiffla pengar är rimligt.\\
Sophia ska kolla till nästa möte vad som står nu och hur vi ska utveckla det.
\p{16}{Nästa styrelsemöte}{\bes}
{\Mba} nästa styrelsemöte ska äga rum 2017-02-09 12:10 i E:1426.
\p{17}{Beslutsuppföljning}{\bes}
{\Ibfu}
\p{18}{Övrigt}{\dis}
Erik tyckte att folk ska gå på styrelseutbildningen, den var bra förra året.\\
Sophia sa att vi borde bestämma hur mycket vi ska ha till kickoff och kickout. Hon föreslog samma som förra året, mötet är överens.\\
Sophia \ypa lägga 30 kr per person och tillfälle till kickoff och kickout.\\
\Mbaby \\
Temasläppet är den 22/4 sa Niklas.\\
Pontus sa att det finns två förslag till nya PA-toppar, det kommer att landa på 9-12 tusen. Han föreslog att vi väntar på ett sektionsmöte för att ta beslut.
PA-topparna används när NöjU säljer glass sa Niklas.\\
Mötet är överns om att ansvariga får skriva en motion till ett sektionsmöte.\\
Pontus frågade var han kan få tag på matbiljetter, Sophia kan fixa fram sånt.\\
Albin undrade hur det går med folks djurdräckter. Det verkar gå bra, vissa ska till klädkammaren på AF-borgen resten ska köpa.\\
Pontus vill ha kontaktforumlär som går direkt till honom och så vill han lägga till "i am not a robot" för att undvika spam från sagda och befintliga formulär.\\
Erik upplyste mötet om att matbiljetter får användas för att köpa alkohol,det är lagligt.\\
Mötet ansåg, efter omröstning, att matbiljetterna bör designas om där den nya designen inte ska förhindra alkoholköp.\\
Mötet är överens om att man får byta ut gamla biljetter mot de nya.
\p{19}{OFMA}{\bes}
{\mo} förklarade mötet avslutat 12:52.

\end{paragrafer}

\newpage
\hidesignfoot
\begin{signatures}{3}
\signature{\mo}{Mötesordförande}
\signature{\ms}{Mötessekreterare}
\signature{\ji}{Justerare}
\end{signatures}
\end{document}
