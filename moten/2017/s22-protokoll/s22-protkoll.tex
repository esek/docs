\documentclass[10pt]{article}
\usepackage[utf8]{inputenc}
\usepackage[swedish]{babel}

\def\mo{Erik Månsson}
\def\ms{Johan Karlberg}
\def\ji{Daniel Bakic}
%\def\jii{}

\def\doctype{Protokoll} %ex. Kallelse, Handlingar, Protkoll
\def\mname{styrelsemöte} %ex. styrelsemöte, Vårterminsmöte
\def\mnum{S22/17} %ex S02/16, E1/15, VT/13
\def\date{2017-10-05} %YYYY-MM-DD
\def\docauthor{\ms}

\usepackage{../e-mote}
\usepackage{../../../e-sek}

\begin{document}
\showsignfoot

\heading{{\doctype} för {\mname} {\mnum}}

%\naun{}{} %närvarane under
%\nati{} %närvarande till och med
%\nafr{} %närvarande från och med
\section*{Närvarande}
\subsection*{Styrelsen}
\begin{narvarolista}
\nv{Ordförande}{Erik Månsson}{E14}{}
\nv{Kontaktor}{Johan Karlberg}{E14}{}
\nv{Förvaltningschef}{Sophia Grimmeiss Grahm}{BME14}{}
\nv{Cafémästare}{Daniel Bakic}{E15}{}
%\nv{Øverphøs}{Niklas Gustafson}{E15}{}
\nv{SRE-ordförande}{Edvard Carlsson}{E16}{}
\nv{ENU-ordförande}{Josefine Sandström}{E14}{}
\nv{Sexmästare}{Linnea Sjödahl}{BME15}{}
\nv{Krögare}{Markus Rahne}{BME14}{}
\nv{Entertainer}{Albin Nyström Eklund}{BME16}{}
\end{narvarolista}

\subsection*{Ständigt adjungerande}
\begin{narvarolista}
%\nv{Kårordförande}{Linus Hammarlund}{}{}
%\nv{Kårrepresentant}{Anders Nilsson}{}{}
\nv{Kårrepresentant}{Caroline Svensson}{}{}
\nv{Kårrepresentant}{Agnes Sörliden}{}{}
\nv{Valberedningens ordförande}{Christian Benson}{}{}
%\nv{Skattmästare}{Olle Oswald}{}{}
%\nv{Kårrepresentant}{Daniel Damberg}{}{}
%\nv{Kårrepresentant}{John Alvén}{}{}
%\nv{Talman}{Fredrik Peterson}{E14}{}
%\nv{Elektras Ordförande}{Elisabeth Pongratz}{}{}
%\nv{Inspektor}{Monica Almqvist}{}{}
\end{narvarolista}

\begin{comment}
\subsection*{Adjungerande}
\begin{narvarolista}
%\nv{Post}{Namn}{Klass}{}
\end{narvarolista}
\end{comment}

\section*{Protokoll}
\begin{paragrafer}
\p{1}{OFMÖ}{\bes}
Ordförande {\mo} förklarade mötet öppnat 12:13.

\p{2}{Val av mötesordförande}{\bes}
{\valavmo}

\p{3}{Val av mötessekreterare}{\bes}
{\valavms}

\p{4}{Val av justeringsperson}{\bes}
{\valavj}

\p{5}{Godkännande av tid och sätt}{\bes}
{\tosg}

\p{6}{Adjungeringar}{\bes}
{\ingaadj}

%Förnamn Efternamn adjungerades

\p{7}{Godkännande av dagordningen}{\bes}
%Dagordningen godkändes.
Erik \ypa lägga till \S12 ``HT-möte''.

Albin \ypa lägga till \S13 ``Sångastriden''.
%Föredragningslistan godkändes med yrkandet.

Föredragningslistan godkändes med samtliga yrkanden.
\p{8}{Föregående mötesprotokoll}{\bes}
\latillprot{S21/17}
%\ingaprot

\p{9}{Fyllnadsval och entledigande av funktionärer}{\bes}
\begin{fyllnadsval} %"Inga fyllnadsval." fylls i automatiskt
\fval{William Marnfeldt}{Årskurs E-1 ansvarig}
\fval{Måns Lindeberg}{SRE-ledamot}
\fval{Nelly Ostréus}{Årskurs BME-1 ansvarig}
\fval{Axel Sandqvist}{Årskurs E-1 ansvarig}
\fval{Saga Ekman}{Valberedningsledamot}
\fval{Axel Sondh}{Valberedningsledamot}
\fval{Johannes Larsson}{Macapär}

\entl{Odinn Dånsjö}{Vice cafémästare}
\entl{Johannes Larsson}{Kodhackare}
\end{fyllnadsval}

\p{10}{Rapporter}{}
\begin{paragrafer}
\subp{A}{Hur mår alla?}{\info}
Punkten protokollfördes ej.
\subp{B}{Utskottsrapporter}{\info}
KM har stakat ut hösten och kan lova att det blir gillen även denna terminen. Julgillet är spikat till 9 dec. De har kikat intresse för whiskyprovning och det ser lovande ut.

ENU går bra. Josefine var på möte i morse med Adsensus som vill skriva avtal med E-sektionen. Arm och Academic work kommer till foajén nästa vecka.

Nöjesutskottet kommer inom kort ha möte och diskutera igenom vad de kommer göra under hösten. Tanken är spelkvällar, filmkvällar och bastukvällar. Eventuellt kommer vi även se över möjligheterna att dra iväg på något annat evenemang i Linköping istället för DÖMD då det finns flera andra lockande evenemang.

SRE har haft introduktionsmöte för de fyra nyvalda, mest för att de skulle få träffa alla. SRE börjar jobba mer efter tentaperioden. Edvard har varit på ekonomiutbildning med SRX.

Sexmästeriet har vilat upp sig efter Nollegasquen och har haft ett möte där de pratade igenom hur nollningen kändes och också vad de ska göra i höst. De ska hålla i ET-slasquen i oktober och funderar på vad för andra event vi vill göra.

CM sålde kanelbullar igår. De jobbar med omstrukturering i LED för att klara sig utan en anställd.
\subp{C}{Ekonomisk rapport}{\info}
Sophia meddelade att ekonomin går bra. Sektionen har \SI{700000}{kr} på kontot.
\subp{D}{Kåren informerar}{\info}
Val till Fullmäktige har öppnat. Nominering och kandidering är i oktober och valet är i november. Kåren har kommit igång med sitt arbete och sin verksamhetsplan. De vill ha input på hur arbetet upplevs, därför kommer det komma enkäter och annan undersökning för att bilda en uppfattning om hur andra upplever kårens verksamhet. Speciellt vill de bilda en uppfattning om hur Kårens relation mot Sektioner bör vara, vem ska göra vad. Ytterligare en fråga är nollning utan arvorderade ansvariga.

Ytterligare meddelade Caroline att om ingen ansvarig till Vårterminsnollningen hittas innan 20 oktober så kommer ansvaret att läggas på heltidarna och där efter så får de avgöra om de kommer att utföra detta.

Under inläsningsveckan bjuder kåren på tentafrukost runt på campus.
\end{paragrafer}

\p{11}{Planering av storstädning}{\dis}
Sophia vill att styrelsen hittar datum för storstädning.

Styrelsen kom överens om tisdagen den 17 oktober och de flesta kan närvara.
\p{12}{HT-möte}{\dis}
Erik har lagt material till mötet i driven.

Mötet diskuterade vad ansvariga ska förbereda inför HT-mötet.

Mötet diskuterade eventuella propositioner som styrelsen vill lägga.

Christian berättade om ett postförslag han har hört, E-spelman. Främst för att vi saknar ett stort lan.

Josefine sa att Tetra-Pak har hört av sig och vill arrangera ett ``Hackaton''.
\p{13}{Sångastriden}{\dis}
Sångastridsgeneralen har skickat till stridsropets mail, så Albin har inte fått någon information ännu. Albin sa att han är villig att dra i spexdelen men inte i kördelen. Albin har tänkt att dra igång detta och hoppas på att folk som är intresserade kommer.
\p{14}{Nästa styrelsemöte}{\bes}
{\Mba} nästa styrelsemöte ska äga rum 2017-10-11 12:10 i E:1124.

\p{15}{Beslutsuppföljning}{\bes}
{\Ibfu}

\p{16}{Övrigt}{\dis}
Glöm inte städschemat.

Markus bjuder på middag nästa vecka, detta på grund utav att han har massa vin.

Josefine vill hänga med D-styret.

Daniel vill att vi ska anordna TDS.
\p{17}{OFMA}{\bes}
{\mo} förklarade mötet avslutat 12:52.

\end{paragrafer}

\newpage
\hidesignfoot
\begin{signatures}{3}
\signature{\mo}{Mötesordförande}
\signature{\ms}{Mötessekreterare}
\signature{\ji}{Justerare}
\end{signatures}
\end{document}
