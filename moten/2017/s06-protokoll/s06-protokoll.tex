\documentclass[10pt]{article}
\usepackage[utf8]{inputenc}
\usepackage[swedish]{babel}

\def\mo{Erik Månsson}
\def\ms{Johan Karlberg}
\def\ji{Albin Nyström Eklund}
%\def\jii{}

\def\doctype{Protokoll} %ex. Kallelse, Handlingar, Protkoll
\def\mname{styrelsemöte} %ex. styrelsemöte, Vårterminsmöte
\def\mnum{S06/17} %ex S02/16, E1/15, VT/13
\def\date{2017-02-23} %YYYY-MM-DD
\def\docauthor{\ms}

\usepackage{../e-mote}
\usepackage{../../../e-sek}

\begin{document}
\showsignfoot

\heading{{\doctype} för {\mname} {\mnum}}

%\naun{}{} %närvarane under
%\nati{} %närvarande till och med
%\nafr{} %närvarande från och med
\section*{Närvarande}
\subsection*{Styrelsen}
\begin{narvarolista}
\nv{Ordförande}{Erik Månsson}{E14}{}
\nv{Kontaktor}{Johan Karlberg}{E14}{}
%\nv{Förvaltningschef}{Sophia Grimmeiss Grahm}{BME14}{}
\nv{Cafémästare}{Daniel Bakic}{E15}{}
\nv{Øverphøs}{Niklas Gustafson}{E15}{\nati{11}}
\nv{SRE-ordförande}{Pontus Landgren}{E14}{}
\nv{ENU-ordförande}{Josefine Sandström}{E14}{}
\nv{Sexmästare}{Linnea Sjödahl}{BME15}{}
%\nv{Krögare}{Markus Rahne}{BME14}{}
\nv{Entertainer}{Albin Nyström Eklund}{BME16}{}
\end{narvarolista}

\begin{comment}
\subsection*{Ständigt adjungerande}
\begin{narvarolista}
%\nv{Kårordförande}{Linus Hammarlund}{}{}
%\nv{Kårrepresentant}{Jacob Karlsson}{}{}
\nv{Aktivietssamordnare}{Lovisa Majtorp}{}{}
%\nv{Valberedningens ordförande}{Elin Magnusson}{}{}
%\nv{Skattmästare}{Olle Oswald}{}{}
%\nv{Kårrepresentant}{Daniel Damberg}{}{}
%\nv{Kårrepresentant}{John Alvén}{}{}
%\nv{Talman}{Fredrik Peterson}{E14}{}
%\nv{Elektras Ordförande}{Elisabeth Pongratz}{}{}
%\nv{Inspektor}{Monica Almqvist}{}{}
\end{narvarolista}
\end{comment}

\begin{comment}
\subsection*{Adjungerande}
\begin{narvarolista}
%\nv{Post}{Namn}{Klass}{}
\end{narvarolista}
\end{comment}

\section*{Protokoll}
\begin{paragrafer}
\p{1}{OFMÖ}{\bes}
Ordförande {\mo} förklarade mötet öppnat 12:14.

\p{2}{Val av mötesordförande}{\bes}
{\valavmo}

\p{3}{Val av mötessekreterare}{\bes}
{\valavms}

\p{4}{Val av justeringsperson}{\bes}
{\valavj}

\p{5}{Godkännande av tid och sätt}{\bes}
{\tosg}

\p{6}{Adjungeringar}{\bes}
{\ingaadj}

%Förnamn Efternamn adjungerades

\p{7}{Godkännande av dagordningen}{\bes}
Dagordningen godkändes.
%Fredrik \ypa att lägga till \S18b ``Teknikfokus utnyttjande av LED-café''.
%Föredragningslistan godkändes med yrkandet.
%Föredragningslistan godkändes med samtliga yrkanden.

\p{8}{Föregående mötesprotokoll}{\bes}
\latillprot{S05/17}
%\ingaprot

\p{9}{Fyllnadsval och entledigande av funktionärer}{\bes}
\begin{fyllnadsval} %"Inga fyllnadsval." fylls i automatiskt
\fval{Jonas Andersson}{Diod}
\fval{Johan Wendt}{Diod}
\fval{Stephanie Mirsky}{Picasso}
\fval{Ariton Karamani}{Umphmeister}
\entl{Oskar Berg}{Diod}
\end{fyllnadsval}

\p{10}{Rapporter}{}
\begin{paragrafer}
\subp{A}{Hur mår alla?}{\info}
Punkten protokollförs ej.
\subp{B}{Utskottsrapporter}{\info}
DDG-jobbar, det går bra med den nya hemsidan. Anders har även gjort en hemsida till jubileumet.\\

LEDs verksamhet flyter på bra. De ska väldigt snart påminna E-sektionen att de behöver Dioder för nästa läsperiod. De hade möte med Teknikfokus-ansvariga i måndags och pratade om vad som gäller inför 2 mars då Teknikfokus gruppen hyr LED under morgonen fram till 10. Nu på måndag ska de göra sallader till FPT som ska hålla i lunchföreläsning, vilket kommer ge en bra intäkt. Under tiden ska de även förbereda oss inför fettisdagen dagen därpå då vi ska sälja semlor, så kom och köp! \\

KM har jobbat hela veckan inför första gillet nu på fredag. De hade drinkskola för de som skall jobba och har annars jobbat för att få till alla detaljer inför vad som förhoppningsvis kommer bli en fantastisk kväll och premiärgille. Kapsyltömmarnatten, KMs kick-off kommer även den hållas nu på lördag där tanken är att samla ihop hela utskottet inför årets som kommer med lekar, mat och för att undersöka vem som är bäst på att kasta kapsyler.

KM kommer även hålla pub på tisdag för Teknikfokus men utan huvudattraktionen Accenture då dom blandade ihop datumen. Vad som ersätter är just nu oklart.\\

NollU har haft phadderintervjuer hela veckan och har inte hunnit med så mycket mer. De har även börjat spåna på nolleguide, logga och kläder. Behövt lägga in extramöte för att diskutera nollningsschema inför nästa ÖPK möte.\\

Lunch med ingenjör denna veckan, enligt Rebecca går ENU 6000 kr med vinst. Bearingpoint lunchföreläsning 20 mars. Svep branschkväll 23/3 eller 29/3 tillsammans med D och F (mjukvaruinriktat för årskurs 4-5). Börjat mejla lite angående event under nollning.\\

NöjU har varit på kollegiemötet vilket varit nyttigt. Kåren arbetar i vår med:
\begin{itemize}
  \item [--] Sexmästarkollegiet arrangerar pubrunda 6 april.
  \item [--] Corneliusbalen 8 april.
  \item [--] Las Gegas är ett nytt F1-röj liknande event som drar igång i maj.
  \item [--] Tandem 12-13 maj.
  \item [--] ET-slasque 18 mars.
\end{itemize}
Utöver det har NöjU inte gjort mycket. Den 1 mars kommer möte angående Agent00E äga rum som eventuellt kommer bli ett kårevent istället. \\

E6 meddelar att skiphtet gick väldigt bra, alla som gick på sittningen och som jobbade verkade nöjda. Vi planerar för fullt för Teknikfokus som är nästa vecka.
Hade kollegiemöte i veckan.
De försöker också komma på ett bra system med att informera jobbarna om vilka sittningar de ska jobba på och allmänt organisera utskottet lite bättre.\\

SRE har arrangerat pluggkväll denna veckan och det gick bra. I övrigt flyter arbetet i SRE på som vanligt. Kollegien har haft möte och SRE undersöker möjligheter att få kontakt med utbytesstudenter inom sektionen.
\subp{C}{Ekonomisk rapport}{\info}
Sophia hälsar att ekonomin mår bra, en inbetalning från D-sektionen ska korrigeras.
\subp{D}{Kåren informerar}{\info}
Kårens verksamhet är igång som vanligt.\\
Las Gegas är ett nytt projekt, som ``Hål i väggen''. Tanken är att det är ett helgevent likt F1-röj. Det är en del jobb med Las Gegas, men Lovisa tror att det blir riktigt bra.
\end{paragrafer}
\p{11}{Tankar och idéer på hur vi ska utveckla styrelsearbetet}{\dis}
Mötet diskuterade ämnet.\\
Exempel på vad vi kan göra är:
\begin{itemize}
  \item Samla alla utskott mer, det vill säga alla vice och ordförande, för att prata om vad vi gör.
  \item Sprida korta sammanfattningar från styrelsemöten på sociala medier.
  \item Event där folk kan skriva motioner.
\end{itemize}
\p{12}{Attestering}{\dis}
Erik ansåg att riktlinjerna bör förtydligas ytterligare, till exempel byta ut ``kan bli'' mot ``blir'' osv. Även att kortinnehavare måste skriva på kontraktet annars blir denne återbetalningsskyldig.
\p{13}{Nästa styrelsemöte}{\bes}
{\Mba} nästa styrelsemöte ska äga rum 2017-03-09 12:10 i 1426.
\p{14}{Beslutsuppföljning}{\bes}
{\Ibfu}
\p{15}{Övrigt}{\dis}
Kontaktorn meddelar att events på facebook ska innehålla både engelska och svenska.\\
Rektorn har tagit ett beslut om vem som är ansvarig på psykisk hälsa, nu är områdesansvariga i ledningsgruppen grundutbildning ansvariga.\\
Anders har gett Daniel en bra idé om present till Ulla.\\
De pratar även om att samla ihop gamla cafe-mästare och fira lite.\\
Våra tröjor är nu beställda.\\
Erik ska utlysa alla vakanta poster.\\
Daniel säger att alla borde testa att jobba i cafét.\\
Styrelsen kommer att bli anmodade till Corneliusbalen.\\
\p{16}{OFMA}{\bes}
{\mo} förklarade mötet avslutat 12:49.

\end{paragrafer}

\newpage
\hidesignfoot
\begin{signatures}{3}
\signature{\mo}{Mötesordförande}
\signature{\ms}{Mötessekreterare}
\signature{\ji}{Justerare}
\end{signatures}
\end{document}
