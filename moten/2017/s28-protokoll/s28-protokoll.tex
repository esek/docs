\documentclass[10pt]{article}
\usepackage[utf8]{inputenc}
\usepackage[swedish]{babel}

\def\mo{Erik Månsson}
\def\ms{Johan Karlberg}
\def\ji{Markus Rahne}
%\def\jii{}

\def\doctype{Protokoll} %ex. Kallelse, Handlingar, Protkoll
\def\mname{styrelsemöte} %ex. styrelsemöte, Vårterminsmöte
\def\mnum{S28/17} %ex S02/16, E1/15, VT/13
\def\date{2017-11-30} %YYYY-MM-DD
\def\docauthor{\ms}

\usepackage{../e-mote}
\usepackage{../../../e-sek}

\begin{document}
\showsignfoot

\heading{{\doctype} för {\mname} {\mnum}}

%\naun{}{} %närvarane under
%\nati{} %närvarande till och med
%\nafr{} %närvarande från och med
\section*{Närvarande}
\subsection*{Styrelsen}
\begin{narvarolista}
\nv{Ordförande}{Erik Månsson}{E14}{}
\nv{Kontaktor}{Johan Karlberg}{E14}{}
\nv{Förvaltningschef}{Sophia Grimmeiss Grahm}{BME14}{}
\nv{Cafémästare}{Daniel Bakic}{E15}{}
\nv{Øverphøs}{Niklas Gustafson}{E15}{}
\nv{SRE-ordförande}{Edvard Carlsson}{E16}{}
\nv{ENU-ordförande}{Josefine Sandström}{E14}{}
\nv{Sexmästare}{Linnea Sjödahl}{BME15}{}
\nv{Krögare}{Markus Rahne}{BME14}{}
\nv{Entertainer}{Albin Nyström Eklund}{BME16}{}
\end{narvarolista}

\subsection*{Ständigt adjungerande}
\begin{narvarolista}
%\nv{Kårordförande}{Linus Hammarlund}{}{}
%\nv{Kårrepresentant}{Anders Nilsson}{}{}
%\nv{Kårrepresentant}{Caroline Svensson}{}{\nafr{9}}
\nv{Kårrepresentant}{Agnes Sörliden}{}{}
%\nv{Valberedningens ordförande}{Elin Magnusson}{}{}
%\nv{Skattmästare}{Olle Oswald}{}{}
%\nv{Kårrepresentant}{Daniel Damberg}{}{}
%\nv{Kårrepresentant}{John Alvén}{}{}
%\nv{Talman}{Fredrik Peterson}{E14}{}
%\nv{Elektras Ordförande}{Elisabeth Pongratz}{}{}
%\nv{Inspektor}{Monica Almqvist}{}{}
\nv{Inköps- och lagerchef och Cafémästare 2018}{Elin Johansson}{BME16}{}
\end{narvarolista}

\subsection*{Adjungerande}
\begin{narvarolista}
%\nv{Kontaktor 2018}{Axel Voss}{E15}{}
\nv{Sexmästare 2018}{Alexander Wik}{BME17}{}
\nv{Entertainer 2018}{Adam Belfrage}{BME17}{\nafr{9}}
%\nv{Krögare 2018}{Malin Heyden}{E16}{}
%\nv{ENU-ordförande 2018}{Isabella Hansen}{E16}{}
\nv{Øverphøs 2018}{Andreas Bennström}{BME16}{}
\nv{Förvaltningschef 2018}{Magnus Lundh}{E15}{}
\nv{SRE-Ordförande 2018}{Fanny Månefjord}{BME16}{}
\nv{Likabehandlingsombud}{Lina Samnegård}{BME16}{}
\nv{Skydds. likabehandlingsansvar 2018}{Ebba Lindgärde}{BME17}{}
\nv{Skydds. likabehandlingsansvar 2018}{Emma Hjörneby}{BME17}{\nafr{10A}}
\end{narvarolista}

\section*{Protokoll}
\begin{paragrafer}
\p{1}{OFMÖ}{\bes}
Ordförande {\mo} förklarade mötet öppnat 12:11.

\p{2}{Val av mötesordförande}{\bes}
{\valavmo}

\p{3}{Val av mötessekreterare}{\bes}
{\valavms}

\p{4}{Val av justeringsperson}{\bes}
{\valavj}

\p{5}{Godkännande av tid och sätt}{\bes}
{\tosg}

\p{6}{Adjungeringar}{\bes}
%{\ingaadj}
Adam Belfrage adjungerades.

Alexander Wik adjungerades.

Andreas Bennström adjungerades.

Fanny Månefjord adjungerades.

Lina Samnegård adjungerades.

Ebba Lindgärde adjungerades.

Emma Hjörneby adjungerades.
\p{7}{Godkännande av dagordningen}{\bes}
Dagordningen godkändes.
%Albin \ypa lägga till \S13 ``Tackphest till SåS.''.

%Föredragningslistan godkändes med yrkandet.
%Föredragningslistan godkändes med samtliga yrkanden.

\p{8}{Föregående mötesprotokoll}{\bes}
\latillprot{S27/17}
%\ingaprot

\p{9}{Fyllnadsval och entledigande av funktionärer}{\bes}
\begin{fyllnadsval} %"Inga fyllnadsval." fylls i automatiskt
\fval{Philip Johansson}{Projektgrupp Teknikfokus}
\fval{Sonja Kenari}{Projektgrupp Teknikfokus}
\fval{Henrik Stålbom}{Projektgrupp Teknikfokus}
\fval{Isabella Hansen}{Projektgrupp Teknikfokus}
%\entl{Namn}{Post}
\end{fyllnadsval}

\p{10}{Rapporter}{}
\begin{paragrafer}
\subp{A}{Hur mår alla?}{\info}
Punkten protokollfördes ej.
\subp{B}{Utskottsrapporter}{\info}
Edvard har varit på SRX-möte.

Markus har suttit och sålt biljetter till julgillet i veckan, nästan alla biljetter såldes. De hade KM tackphest igår.

Verksamheten i LED rullar på. Daniel och Elin skall på kollegiemöte nästa fredag.

Sexmästeriet håller på att planera inför Afternoon Tea och IKEA-sittningen, som är de sista eventen vi kommer att hålla i innan vi går av.

Macapärerna testade gillemode när ENU hade pub i tisdags.

NollU har haft möte angående testamente, Skiphte och städning.

Nöju har fixat inför bastukvällen. Det har köpts in två nya sällskapsspel för ``biljardpengarna''. Spelkväll kommer vara i sista veckan. Kick-out för utskottet kommer troligen vara 8 december.

Det är rätt lugnt i ENU för tillfället. Isabella och Josefine har fakturerat för Lunch med ingenjör. Josefine håller också igång mejlkontakten med Cybercom som ville sponsra saker till LED och Adsensus som ville hålla en utbildning på E-sektionen. Teknikfokus höll i pub i tisdags.

Sophia, Olle och Mange har bokfört. Sophia har också hängt mycket i iDét för att reda ut hur mycket pengar vi ska få av data. Tyvärr ser det ut som att vinstdelningen från Teknikfokus inte kommer förrän nästa år.
\subp{C}{Ekonomisk rapport}{\info}
Sophia meddelade att ekonomin mår bra, vi har pengar. De skall försöka att få klart allt för 2017 detta året.
\subp{D}{Kåren informerar}{\info}
I helgen var det F1-Röj. Det var möte i fullmäktige, där de tog beslut om IT-strukturen på kåren. Fullmäktige skall nu på söndag välja nya heltidare.
\end{paragrafer}

\p{11}{E-sektionens arbete mot trakasserier}{\dis}
Fanny sammanfattade lite om vilka resurser och krav som finns. Det finns en rättighetslista som universitetet har. Där står alla rättigheter och skyldigheter som studenter har. Det står även lite om studiemiljö. Det står att vi ska motverka diskriminering, allstå vi ska arbeta i förebyggande syfte.

Mötet diskuterade eventuella problem som finns idag.

Fanny sa att våra åsikter, de som sitter i rummet, är inte representativa för gemene sektionsmedlem. Vi är engagerade det kanske finns folk som vill vara engagerade men inte vågar.

Det är svårt att få åsikter från folk som inte är engagerade.

F-sektionen har en rapport som följd av jodel, E-Sektionen skall ta del av den.

Albin lyfte problem som kan förekomma på nollningen och de som vill vara aktiva men inte får en post på VM.

Synlighet är bra.

Mötet diskuterade ämnet.

Niklas lyfte att Sektionen kanske kan ha ett diskussionsmöte.

Sophia tyckte att det är bra att göra en större grej av det så att vi visar att vi bryr oss.

Mötet diskuterade ämnet vidare.

Agnes berättade att TLTH kommer att tillsätta en arbetsgrupp som kommer att titta på hur de skall stänga av funktionärer.
\p{12}{Regler för pepparkakshustävlingen}{\dis}
Johan \ypa pepparkakshusen skall vara ätbara, superlim är inte ätbart.

Markus \ypa Cafémästaren skall äta och kolla för att det är ätbart.

Niklas \ypa Kontaktorn bestämmer vad Cafémästare skall äta.

Johan jämkade sig med samtliga tilläggsyrkanden.

\Mba bifalla samtliga yrkanden.
\p{14}{Nästa styrelsemöte}{\bes}
{\Mba} nästa styrelsemöte ska äga rum 2017-12-07 12:10 i E:1124.

\p{15}{Beslutsuppföljning}{\bes}
Erik \ypa stryka Inköp av förstärkare från beslutsuppföljningen, den kostade \SI{4845}{kr}.

\Mbaby

\p{16}{Övrigt}{\dis}
Daniel bad alla att fixa testamenten till samtliga utskottsordföranden.
\p{17}{OFMA}{\bes}
{\mo} förklarade mötet avslutat 12:59.

\end{paragrafer}

\newpage
\hidesignfoot
\begin{signatures}{3}
\signature{\mo}{Mötesordförande}
\signature{\ms}{Mötessekreterare}
\signature{\ji}{Justerare}
\end{signatures}
\end{document}
