
\documentclass[10pt]{article}
\usepackage[utf8]{inputenc}
\usepackage[swedish]{babel}

\def\mo{Erik Månsson}
\def\ms{Johan Karlberg}
\def\ji{Niklas Gustafson}
%\def\jii{}

\def\doctype{Protokoll} %ex. Kallelse, Handlingar, Protkoll
\def\mname{styrelsemöte} %ex. styrelsemöte, Vårterminsmöte
\def\mnum{S11/17} %ex S02/16, E1/15, VT/13
\def\date{2017-04-27} %YYYY-MM-DD
\def\docauthor{\ms}

\usepackage{../e-mote}
\usepackage{../../../e-sek}
\begin{document}

\showsignfoot

\heading{{\doctype} för {\mname} {\mnum}}

%\naun{}{} %närvarane under
%\nati{} %närvarande till och med
%\nafr{} %närvarande från och med
\section*{Närvarande}
\subsection*{Styrelsen}
\begin{narvarolista}
\nv{Ordförande}{Erik Månsson}{E14}{}
\nv{Kontaktor}{Johan Karlberg}{E14}{}
%\nv{Förvaltningschef}{Sophia Grimmeiss Grahm}{BME14}{}
\nv{Cafémästare}{Daniel Bakic}{E15}{}
\nv{Øverphøs}{Niklas Gustafson}{E15}{\nafr{7}}
\nv{SRE-ordförande}{Pontus Landgren}{E14}{}
\nv{ENU-ordförande}{Josefine Sandström}{E14}{}
\nv{Sexmästare}{Linnea Sjödahl}{BME15}{}
\nv{Krögare}{Markus Rahne}{BME14}{\nafr{7}}
\nv{Entertainer}{Albin Nyström Eklund}{BME16}{\nati{14}}
\end{narvarolista}

\subsection*{Ständigt adjungerande}
\begin{narvarolista}
%\nv{Kårordförande}{Linus Hammarlund}{}{}
%\nv{Kårrepresentant}{Jacob Karlsson}{}{}
\nv{Aktivietssamordnare}{Lovisa Majtorp}{}{}
%\nv{Valberedningens ordförande}{Elin Magnusson}{}{}
%\nv{Skattmästare}{Olle Oswald}{}{}
%\nv{Kårrepresentant}{Daniel Damberg}{}{}
%\nv{Kårrepresentant}{John Alvén}{}{}
%\nv{Talman}{Fredrik Peterson}{E14}{}
%\nv{Elektras Ordförande}{Elisabeth Pongratz}{}{}
%\nv{Inspektor}{Monica Almqvist}{}{}
\end{narvarolista}

\subsection*{Adjungerande}
\begin{narvarolista}
\nv{Teknokrat}{Anders Nilsson}{E13}{}
\end{narvarolista}

\section*{Protokoll}
\begin{paragrafer}
\p{1}{OFMÖ}{\bes}
Ordförande {\mo} förklarade mötet öppnat 12:13.

\p{2}{Val av mötesordförande}{\bes}
{\valavmo}

\p{3}{Val av mötessekreterare}{\bes}
{\valavms}

\p{4}{Val av justeringsperson}{\bes}
{\valavj}

\p{5}{Godkännande av tid och sätt}{\bes}
{\tosg}

\p{6}{Adjungeringar}{\bes}
Anders Nilsson adjungerades.

\p{7}{Godkännande av dagordningen}{\bes}
%Dagordningen godkändes.

Linnea \ypa lägga till \S12 ``Inköp av bestick''.

Erik \ypa att lägga till \S13 ``Mentorsprogram''.

Albin \ypa att lägga till \S14 ``Pengar till målning av cylindern utanför LED''.

%Föredragningslistan godkändes med yrkandet.
Föredragningslistan godkändes med samtliga yrkanden.
\p{8}{Föregående mötesprotokoll}{\bes}
\latillprot{S10/17}
%\ingaprot

\p{9}{Fyllnadsval och entledigande av funktionärer}{\bes}
\begin{fyllnadsval} %"Inga fyllnadsval." fylls i automatiskt
\fval{Saga Juniwik}{Vice Krögare}
%\entl{Namn}{Post}
\end{fyllnadsval}

\p{10}{Rapporter}{}
\begin{paragrafer}
\subp{A}{Hur mår alla?}{\info}
Punkten protokollfördes ej.
\subp{B}{Utskottsrapporter}{\info}
LEDs verksamhet flyter på bra, lite ont om kunder dock. På fredag ska de fira att Ulla varit en del av sektionen i 10 år, det ska bli kul! Styrelsen köper även present i form av en korg ostdelikatesser och sektionen står för mat och tårta till själva firandet.

InfU har haft kick-off det var kul. Allt annat går bra.

Efter tre veckors paus är det dags för gille igen. Utskottet går bra, förberedelserna inför jubileumet rullar på och de har hittat en ny Vice Krögare till hösten.

NollU mår bra.

ENU mår bra. De har marknadsfört Accenture-night på facebook och ett rekryteringsevent åt Academic Work. Ericsson är väldigt taggade och det ser ut som att det kan bli några evenemang med dem till hösten och förhoppningsvis nu i vår om vi hinner. Mycket mejlande.

NöjU har i veckan planerat evenemang som ska genomföras under maj månad. Däribland E-lounge, sektionsmålning med grillning, spelkväll. Idrottsförmännen är med i planering av buffelhornet men det ser dåligt ut med anmälningar. UteDischots tygmärke och armband ska beställas inom kort.

E6 har haft fullt upp med både Temasläppssittning tillsammans med D6 och pubrunda med F6 den här veckan. Båda eventen var välbesökta och flöt på bra. Samarbetena med de andra sexmästerierna gick väldigt bra och alla verkade tycka det var väldigt kul!

SRE tillsammans med SRD stod igår och marknadsförde Speak up days ett evenemang för att belysa studentinflytande på LTH. I övrigt flyter den vanliga verksamheten med kursutvärderingar på som vanligt.
\subp{C}{Ekonomisk rapport}{\info}
Sophia var inte närvarande, men Erik meddelade att ekonomin mår bra.
\subp{D}{Kåren informerar}{\info}
Lovisa informerade om att det inte kommer att vara någon el i kårhuset på lördag, detta betyder att det inte finns tillgång till lokalerna. Hon informerade även om 2 enkäter som är ute nu, en från ``Speak up days
'' och en om nollningen.
\end{paragrafer}
\newpage
\p{11}{VT/17}{\dis}
Erik har bokat rulle så att vi kan köpa mat.

Linnea och Niklas kommer inte kunna närvara på hela mötet, men deras Vice eller någon annan i utskottet som har koll kommer att närvara, detta då de har andra möten.
\p{12}{Inköp av bestick}{\bes}
Linnea:s hovmästare har kollat vad det kostar att köpa in 150 uppsättningar med bestick. Det kommer att kosta \SI{15000}{kr} att komplettera de redan befintliga uppsättningarna och \SI{10000}{kr} att köpa helt nytt från t.ex. Ikea. Linnea vill att inköpet skall belasta olycksfonden, detta då bestick inte fyllts på som det borde årligen.

Anders kollade på Ikeas hemsida lite snabbt, han hittade något för \SI{2500}{kr}.

Mötet anser att det är mer rimligt att köpa in en ny serie av bestick istället för att komplettera de gamla.

Punkten skjuts upp till ett senare möte.
\p{13}{Mentorsprogram}{\info}
Erik frågade om någon i styrelsen vill vara mentor. Han önskar även att alla ska kolla om det finns folk som är sugna på att söka till mentor i respektive utskott.

Mentorerna kommer att få någon typ av föreläsningen för att få något att stå på.

\p{14}{Pengar till målning av cylindern utanför LED}{\bes}
Det är \SI{1000}{kr} i budget från jubileumsbudgeten. Det som ska köpas är färg och tillbehör till korvgrillning.

Erik \ypa avsätta \SI{2000}{kr} från dispositionsfonden med beslutsuppföljning med Albin som ansvarig till S14/17.

\Mbaby
\p{15}{Nästa styrelsemöte}{\bes}
{\Mba}nästa styrelsemöte ska äga rum 2017-04-30 11:11 i stadsparken.
{\Mba}nästa styrelsemöte efter det ska äga rum 2017-05-04 12:10 i E:1426.
\p{16}{Beslutsuppföljning}{\bes}
Kostnaden för ``Inköp av skärmar till baren'' uppgick till \SI{7148}{kr}.

Kostnaden för ``Inköp av Raspberry Pi:s'' uppgick till \SI{3493}{kr}.

Den extra pi:n som köptes blev använd snabbt när Tesla dog.

Anders \ypa stryka \emph{Inköp av Raspberry Pi:s} från beslutsuppföljningen.

\Mbaby

Anders \ypa stryka \emph{Inköp av skärmar till baren} från beslutsuppföljningen.

\Mbaby

Anders \ypa skjuta upp \emph{55-års jubileum} till S15/17.

\Mbaby

\p{17}{Övrigt}{\dis}
Anders vill ha hjälp på lördagen med att hålla fördrinken. Det som behövs är någon som ser till så att det ser bra ut och att ställa fram lite saker. Hjälpen är behövd från 16.00 till 19.00. Fördrinken är vid 17.
\p{18}{OFMA}{\bes}
{\mo} förklarade mötet avslutat 12:55.

\end{paragrafer}

%\newpage
\hidesignfoot
\begin{signatures}{3}
\signature{\mo}{Mötesordförande}
\signature{\ms}{Mötessekreterare}
\signature{\ji}{Justerare}
\end{signatures}
\end{document}
