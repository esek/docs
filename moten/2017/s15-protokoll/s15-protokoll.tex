\documentclass[10pt]{article}
\usepackage[utf8]{inputenc}
\usepackage[swedish]{babel}

\def\mo{Erik Månsson}
\def\ms{Johan Karlberg}
\def\ji{Linnea Sjödahl}
%\def\jii{}

\def\doctype{Protokoll} %ex. Kallelse, Handlingar, Protkoll
\def\mname{styrelsemöte} %ex. styrelsemöte, Vårterminsmöte
\def\mnum{S15/17} %ex S02/16, E1/15, VT/13
\def\date{2017-05-24} %YYYY-MM-DD
\def\docauthor{\ms}

\usepackage{../e-mote}
\usepackage{../../../e-sek}

\begin{document}
\showsignfoot

\heading{{\doctype} för {\mname} {\mnum}}

%\naun{}{} %närvarane under
%\nati{} %närvarande till och med
%\nafr{} %närvarande från och med
\section*{Närvarande}
\subsection*{Styrelsen}
\begin{narvarolista}
\nv{Ordförande}{Erik Månsson}{E14}{}
\nv{Kontaktor}{Johan Karlberg}{E14}{}
\nv{Förvaltningschef}{Sophia Grimmeiss Grahm}{BME14}{}
\nv{Cafémästare}{Daniel Bakic}{E15}{}
\nv{Øverphøs}{Niklas Gustafson}{E15}{}
\nv{SRE-ordförande}{Pontus Landgren}{E14}{}
\nv{ENU-ordförande}{Josefine Sandström}{E14}{}
\nv{Sexmästare}{Linnea Sjödahl}{BME15}{}
\nv{Krögare}{Markus Rahne}{BME14}{}
\nv{Entertainer}{Albin Nyström Eklund}{BME16}{}
\end{narvarolista}




\subsection*{Adjungerande}
\begin{narvarolista}
\nv{Teknokrat}{Anders Nilsson}{E13}{}
\nv{Vice SRE-ordförande}{Edvard Carlsson}{E16}{}
\end{narvarolista}


\section*{Protokoll}
\begin{paragrafer}
\p{1}{OFMÖ}{\bes}
Ordförande {\mo} förklarade mötet öppnat 12:14.

\p{2}{Val av mötesordförande}{\bes}
{\valavmo}

\p{3}{Val av mötessekreterare}{\bes}
{\valavms}

\p{4}{Val av justeringsperson}{\bes}
{\valavj}

\p{5}{Godkännande av tid och sätt}{\bes}
{\tosg}

\p{6}{Adjungeringar}{\bes}
%{\ingaadj}

Anders Nilsson adjungerades.

Edvard Carlsson adjungerades.
\p{7}{Godkännande av dagordningen}{\bes}
%Dagordningen godkändes.
Erik \ypa lägga till \S11 ``Sommarfoto''.

Erik \ypa lägga till \S12 ``Inköp av SSD till Buzzard''.
%Föredragningslistan godkändes med yrkandet.

Föredragningslistan godkändes med samtliga yrkanden.

\p{8}{Föregående mötesprotokoll}{\bes}
\latillprot{S13/17}
%\ingaprot

\p{9}{Fyllnadsval och entledigande av funktionärer}{\bes}
\begin{fyllnadsval} %"Inga fyllnadsval." fylls i automatiskt
%\fval{Namn}{Post}
\entl{Linnea Hellholm}{Alumniansvarig BME}
\entl{Elin Branzell}{Alumniansvarig BME}
\entl{Edvard Carlsson}{Källarmästare}
\end{fyllnadsval}

\p{10}{Rapporter}{}
\begin{paragrafer}
\subp{A}{Hur mår alla?}{\info}
Punkten protokollfördes ej.
\subp{B}{Utskottsrapporter}{\info}
FVU har börjat planera lite inför sommarens ommålning av edekvata. De har även bokför en massa.

DDG har haft sista mötet innan sommaren, de grillade vid sjön Sjøn. Mojt är nästan färdiginstallerad med hjälp av bland annat teknokrater och macapärer.

KM har haft alla gillen för terminen och tar nu en välförtjänt paus. Har börjat spika nollningens pubar. Haft möte med ÖPK och utbytt idéer rörande aktivitet mellan draggningen och INGvasion.

Vice Entertainer har fått in sina kavajer. De passade bra. Annars jobbar utskottet med utedischot och planera inför nollningen.

I och med att E6 hållit vår sista sittning för terminen har vi inte haft så mycket att göra. Haft en del möten om nollningen och vi håller på att planera sittningarna vi ska ha då.

SRE-ordförande meddelade att: alla kurser är i princip utvärderade nu, framgångsrika möten på ett flertal kurser där studenterna har önskat förändringar. Har varit lite svårigheter med att hitta lämpliga mötestider vilket har gjort att det dragit ut lite på tiden. Just nu pågår diskussioner om ett utbytesmingel i höst, har ställt frågan till D-sek om de är intresserade och SVL vill i så fall sätta datum på detta innan sommaren. Jag själv har också skrivit ett utkast till ett nytt testamente då de befintliga är inaktuella och inte tar hänsyn till den nya organisationen etc. Vill även ta tillfället i akt och tacka så mycket för det här halvåret, ifrån min sida har det varit givande och roligt!

NollU jobbar inför nollningen.

CM säljer mat.
\subp{C}{Ekonomisk rapport}{\info}
Sophia informerade om utskottens inkomster samt utgifter.
\subp{D}{Kåren informerar}{\info}
Ingen kårrepresentant närvarade.
\end{paragrafer}

\p{11}{Sommarfoto}{\dis}
Styrelsen vill ta ett sommarkort. Fredag föreslogs som dag att ta ett kort.

\p{12}{Inköp av SSD till Buzzard}{\bes}
Erik \ypa att avsätta \SI{1000}{kr} till köp av en SSD med beslutsuppföljning till S16/17 med Erik Månsson som ansvarig.

\Mbaby

\p{13}{Nästa styrelsemöte}{\bes}
{\Mba} nästa styrelsemöte ska äga rum 2017-08-24 12:10 i E:1426.

\p{14}{Beslutsuppföljning}{\bes}
%{\Ibfu}
Genomförande av jubileet gick bra. Ekonomin är svår att ge ett resultat på nu då det inte är bokfört ännu. Så som det ser ut nu så ska det inte vara något som ändrar sig från budgeten.

Erik \ypa skjuta upp \emph{55-års jubileum} till S16/17.

\Mbaby

Erik \ypa skjuta upp \emph{Examensbankett} till S16/17

\Mbaby
\p{15}{Övrigt}{\dis}
Pontus avslutar sin mandatperiod med hybris.

Pontus tackar för sig och hoppades att allt skulle gå bra efter sommaren.

Niklas berättade om ett info-möte för styrelser och phøs den 17 augusti på kåren.

\p{16}{OFMA}{\bes}
{\mo} förklarade mötet avslutat 12:46.

\end{paragrafer}

%\newpage
\hidesignfoot
\begin{signatures}{3}
\signature{\mo}{Mötesordförande}
\signature{\ms}{Mötessekreterare}
\signature{\ji}{Justerare}
\end{signatures}
\end{document}
