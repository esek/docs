\documentclass[10pt]{article}
\usepackage[utf8]{inputenc}
\usepackage[swedish]{babel}

\def\mo{Erik Månsson}
\def\ms{Johan Karlberg}
\def\ji{Pontus Landgren}
%\def\jii{}

\def\doctype{Protokoll} %ex. Kallelse, Handlingar, Protkoll
\def\mname{styrelsemöte} %ex. styrelsemöte, Vårterminsmöte
\def\mnum{S07/17} %ex S02/16, E1/15, VT/13
\def\date{2017-03-09} %YYYY-MM-DD
\def\docauthor{\ms}

\usepackage{../e-mote}
\usepackage{../../../e-sek}

\begin{document}
\showsignfoot

\heading{{\doctype} för {\mname} {\mnum}}

%\naun{}{} %närvarane under
%\nati{} %närvarande till och med
%\nafr{} %närvarande från och med
\section*{Närvarande}
\subsection*{Styrelsen}
\begin{narvarolista}
\nv{Ordförande}{Erik Månsson}{E14}{}
\nv{Kontaktor}{Johan Karlberg}{E14}{}
\nv{Förvaltningschef}{Sophia Grimmeiss Grahm}{BME14}{}
\nv{Cafémästare}{Daniel Bakic}{E15}{}
\nv{Øverphøs}{Niklas Gustafson}{E15}{\nati{14}}
\nv{SRE-ordförande}{Pontus Landgren}{E14}{}
\nv{ENU-ordförande}{Josefine Sandström}{E14}{}
%\nv{Sexmästare}{Linnea Sjödahl}{BME15}{}
\nv{Krögare}{Markus Rahne}{BME14}{\nati{17}}
\nv{Entertainer}{Albin Nyström Eklund}{BME16}{}
\end{narvarolista}

\subsection*{Ständigt adjungerande}
\begin{narvarolista}
%\nv{Kårordförande}{Linus Hammarlund}{}{}
\nv{Kårrepresentant}{Jacob Karlsson}{}{}
%\nv{Aktivietssamordnare}{Lovisa Majtorp}{}{}
%\nv{Valberedningens ordförande}{Elin Magnusson}{}{}
%\nv{Skattmästare}{Olle Oswald}{}{}
%\nv{Kårrepresentant}{Daniel Damberg}{}{}
%\nv{Kårrepresentant}{John Alvén}{}{}
%\nv{Talman}{Fredrik Peterson}{E14}{}
%\nv{Elektras Ordförande}{Elisabeth Pongratz}{}{}
%\nv{Inspektor}{Monica Almqvist}{}{}
\end{narvarolista}

\subsection*{Adjungerande}
\begin{narvarolista}
\nv{Teknokrat}{Anders Nilsson}{E13}{}
\end{narvarolista}

\section*{Protokoll}
\begin{paragrafer}
\p{1}{OFMÖ}{\bes}
Ordförande {\mo} förklarade mötet öppnat 12:12.

\p{2}{Val av mötesordförande}{\bes}
{\valavmo}

\p{3}{Val av mötessekreterare}{\bes}
{\valavms}

\p{4}{Val av justeringsperson}{\bes}
{\valavj}

\p{5}{Godkännande av tid och sätt}{\bes}
{\tosg}

\p{6}{Adjungeringar}{\bes}
%{\ingaadj}
Anders Nilsson adjungerades.

\p{7}{Godkännande av dagordningen}{\bes}
Dagordningen godkändes.
%Fredrik \ypa att lägga till \S18b ``Teknikfokus utnyttjande av LED-café''.
%Föredragningslistan godkändes med yrkandet.
%Föredragningslistan godkändes med samtliga yrkanden.

\p{8}{Föregående mötesprotokoll}{\bes}
\latillprot{S06/17}
%\ingaprot

\p{9}{Fyllnadsval och entledigande av funktionärer}{\bes}
\begin{fyllnadsval} %"Inga fyllnadsval." fylls i automatiskt
\fval{Niklas Karlsson}{Chefredaktör}
\fval{Ludvig Pettersson}{Näringslivskontakt}
%\entl{Namn}{Post}
\end{fyllnadsval}

\p{10}{Rapporter}{}
\begin{paragrafer}
\subp{A}{Hur mår alla?}{\info}
Punkten protokollfördes ej.
\subp{B}{Utskottsrapporter}{\info}
Erik har varit hos Ingrid och pratat om mentorskapsprogrammet till sektionen.

CM har haft mycket att göra, FPT var väldigt nöjda med salladerna och faktura är ivägskickad. De lyckades sälja 164 semlor under fettisdagen, vilket var bra enligt Daniel. Samarbetet mellan CM och Teknikfokus gick bra och det var inga konstigheter, Daniel ska ha uppföljning med Anna och Rasmus snart.

FvU har haft kick-off. Hustomtarna har också lagat en av sofforna i diplomat som var felmonterad.

InfU har haft ett utskottsmöte, främst diskuterades kick-off. Johan har fixat med inbjudningar till jubileet. Picasso är nu två vilket är skönt, det har varit ganska mycket att göra. Idag valde vi äntligen en Chefredaktör, Olle har ställt upp innan så att allt har fungerat men sedan någon vecka tillbaka har Niklas tagit mer ansvar.

KM har kommit igång med sin verksamhet på allvar. De har anordnat tre gillen och de har varit uppskattade. De har paus en vecka nu under inläsningsperiod men återkommer lagom till tentaveckan.

NollU håller på med kläder, loggorna och nollningsschema. Phaddrarna väljs efter tentaperioden.

ENU har haft kick-off. De anordnar en lunchföreläsning med BearingPoint på måndagen efter tentorna, branschkväll med SVEP 29 mars för äldre studenter med D och F. En grupp håller på att mejla med Ericsson, troligtvis vill Ericsson vara med på ett gille en fredag.

NöjU har sedan förra mötet arrangerat spelkväll 8/3. Första tandemmötet var igår och det planeras inför Agent 00E och en påskäggsjakt. UteDischot börjar även ta form och D-sektionen har valt en ansvarig för deras uppgifter.

SRE har sedan förra mötet haft kick-off och lunchmöte. I samband med lunchmötet har även posters på likabehandlingsombud, världsmästare och skyddsombud tryckts upp med syftet att uppmärksamma sektionens medlemmar på vilka personer det är som innehar posterna samt vad de har för uppdrag. I samband med internationella kvinnodagen har likabehandlingsombuden tillsammans D-sektionens Dito hållit ett evenemang i foajen vilket gick bra. En hel del kurser från LP2 har nu haft CEQ-möten och mer är att vänta.
\subp{C}{Ekonomisk rapport}{\info}
Det har inte hänt några stora grejer sa Sophia.
\subp{D}{Kåren informerar}{\info}
Kårstyrelsen hade möte igår och de har utsätt karnevalansvarig. I övrigt har det inte hänt något speciellt. Jakob berättade även lite om karnevalen.
\end{paragrafer}
\p{11}{Inköp till Edekvataköket}{\dis}{}
Markus tycker att till exempel en matberedare hade varit bra att köpa in, kanske att Sexet och KM skulle dela på den.\\
Mötet diskuterade hur detta skulle kunna budgeteras.\\
Markus ska kolla på det och komma med ett förslag till senare möte.
\p{12}{Äskning av pengar för skärmar till baren}{\bes}{}
Anders presenterade förslaget.\\
Albin frågade huruvida skärmarna kan användas till andra saker, Anders ser många möjligheter.\\
Albin ifrågasatte om skärmarna går att ersätta med posters istället.\\
Markus sa att posters inte skulle fungera, han har inte möjlighet att göra posters till alla gillen då det inte går att veta vilken dricka som finns tillgänglig.\\
Pontus påpekade hur nice det är med liveuppdatering.\\
Mötet ansåg att det räcker med 2 väggfästen då skärmarna förmodligen inte kommer att flyttas runt.\\
Daniel frågade om livslängden på sakerna.\\
Anders sa minst 5 år.\\
Mötet godkände äskningen med beslutsuppföljning till möte 10 med Anders som ansvarig samt att budgeten sätts till 8000kr och belastar dispositionsfonden.
\p{13}{Äskning av pengar för inköp av Raspberry Pis}{\bes}{}
Anders presenterade förslaget.\\
Pontus ifrågasatte hur många som behövs.\\
Albin tyckte inte att det ska köpas in för att ha ifall en annan går sönder.\\
Mötet anser att det är lämpligt att ha minst en extra för att använda som utvecklingskort, folk ska inte behöva ha egena saker att utveckla på.\\
Mötet är överens om att det ska köpas 4 st.\\
Mötet godkände äskningen med beslutsuppföljning till möte 9 med Anders som ansvarig samt att budgeten sätts till 3700kr och belastar dispositionsfonden.
\p{14}{Planering VT/17}{\dis}
Mötet är satt preliminärt till den 25 april, motionsstop den 5 april, deadline för handlingar är i så fall den 18 april.\\
Erik gick igenom vad som ska förberedas till mötet.\\
Erik tror att vi behöver minst 3 kvällsmöten innan mötet.\\
Det finns några nu som redan har förslag till propositioner och motioner.
\p{15}{Nästa styrelsemöte}{\bes}
{\Mba}nästa styrelsemöte ska äga rum 2017-03-23 12:10 i E:1426.
\p{16}{Beslutsuppföljning}{\bes}
{\Ibfu}
\p{17}{Övrigt}{\dis}
Erik funderade på present till Peter som ska sluta som vaktmästare.
\p{18}{OFMA}{\bes}
{\mo} förklarade mötet avslutat 13:15.

\end{paragrafer}
\hidesignfoot
\begin{signatures}{3}
\signature{\mo}{Mötesordförande}
\signature{\ms}{Mötessekreterare}
\signature{\ji}{Justerare}
\end{signatures}
\end{document}
