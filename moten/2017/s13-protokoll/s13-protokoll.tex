\documentclass[10pt]{article}
\usepackage[utf8]{inputenc}
\usepackage[swedish]{babel}

\def\mo{Erik Månsson}
\def\ms{Johan Karlberg}
\def\ji{Niklas Gustafson}
%\def\jii{}

\def\doctype{Protokoll} %ex. Kallelse, Handlingar, Protkoll
\def\mname{styrelsemöte} %ex. styrelsemöte, Vårterminsmöte
\def\mnum{S13/17} %ex S02/16, E1/15, VT/13
\def\date{2017-05-11} %YYYY-MM-DD
\def\docauthor{\ms}

\usepackage{../e-mote}
\usepackage{../../../e-sek}

\begin{document}
\showsignfoot

\heading{{\doctype} för {\mname} {\mnum}}

%\naun{}{} %närvarane under
%\nati{} %närvarande till och med
%\nafr{} %närvarande från och med
\section*{Närvarande}
\subsection*{Styrelsen}
\begin{narvarolista}
\nv{Ordförande}{Erik Månsson}{E14}{}
\nv{Kontaktor}{Johan Karlberg}{E14}{}
\nv{Förvaltningschef}{Sophia Grimmeiss Grahm}{BME14}{}
\nv{Cafémästare}{Daniel Bakic}{E15}{}
\nv{Øverphøs}{Niklas Gustafson}{E15}{}
\nv{SRE-ordförande}{Pontus Landgren}{E14}{}
\nv{ENU-ordförande}{Josefine Sandström}{E14}{}
\nv{Sexmästare}{Linnea Sjödahl}{BME15}{}
\nv{Krögare}{Markus Rahne}{BME14}{}
\nv{Entertainer}{Albin Nyström Eklund}{BME16}{}
\end{narvarolista}

\subsection*{Ständigt adjungerande}
\begin{narvarolista}
%\nv{Kårordförande}{Linus Hammarlund}{}{}
%\nv{Kårrepresentant}{Jacob Karlsson}{}{}
\nv{Aktivietssamordnare}{Lovisa Majtorp}{}{}
%\nv{Valberedningens ordförande}{Elin Magnusson}{}{}
%\nv{Skattmästare}{Olle Oswald}{}{}
%\nv{Kårrepresentant}{Daniel Damberg}{}{}
%\nv{Kårrepresentant}{John Alvén}{}{}
%\nv{Talman}{Fredrik Peterson}{E14}{}
%\nv{Elektras Ordförande}{Elisabeth Pongratz}{}{}
%\nv{Inspektor}{Monica Almqvist}{}{}
\end{narvarolista}

\begin{comment}
\subsection*{Adjungerande}
\begin{narvarolista}
%\nv{Post}{Namn}{Klass}{}
\end{narvarolista}
\end{comment}

\section*{Protokoll}
\begin{paragrafer}
\p{1}{OFMÖ}{\bes}
Ordförande {\mo} förklarade mötet öppnat 12:14.

\p{2}{Val av mötesordförande}{\bes}
{\valavmo}

\p{3}{Val av mötessekreterare}{\bes}
{\valavms}

\p{4}{Val av justeringsperson}{\bes}
Albin nominerar Niklas.

Niklas godtog nomineringen.

{\valavj}

\p{5}{Godkännande av tid och sätt}{\bes}
{\tosg}

\p{6}{Adjungeringar}{\bes}

{\ingaadj}

%Förnamn Efternamn adjungerades
\newpage
\p{7}{Godkännande av dagordningen}{\bes}
%Dagordningen godkändes.
Erik \ypa att lägga till \S13``Extrainsatt sektionsmöte''.

Niklas \ypa lägga till \S14``Elektrotjejer söndag''.
%Föredragningslistan godkändes med yrkandet.

Föredragningslistan godkändes med samtliga yrkanden.

\p{8}{Föregående mötesprotokoll}{\bes}
\latillprot{S11/17}
%\ingaprot

\p{9}{Fyllnadsval och entledigande av funktionärer}{\bes}
\begin{fyllnadsval} %"Inga fyllnadsval." fylls i automatiskt
%\fval{Namn}{Post}
\entl{Saga Juniwik}{Källarmästare}
\entl{Rebecka Gerdtham}{Sexig}
\end{fyllnadsval}

\p{10}{Rapporter}{}
\begin{paragrafer}
\subp{A}{Hur mår alla?}{\info}
Punkten protokollfördes ej.
\subp{B}{Utskottsrapporter}{\info}
Sophia och Olle har bokfört. Arkivarierna letade upp spännande saker i arkivet inför jubileumet. Hustomtarna har planerat vad de vill göra.

I InfU jobbar DDG på självmant, så mycket mer har inte hänt.

ENU anordnar After work med Ericsson ikväll! Det utlovas klägg och öl/ cider/ alkoholfritt. Josefine mejlat en del. Axis ville ha en invigning med mikrovågsugnarna men det kommer antagligen inte hinnas med innan tentorna.

NöjU har målat cylindern. Det blev väldigt fint. De har även håll i glassförsäljning. Albin har haft första mötet med ordförande D om utedischot. Det känndes bra, han sköter sin del och jag är snart klar med min. Beställer förhoppningsvis märke och armband ikväll beroende på hur samarbetsvänlig ordförande Erik är.

E6 har hållit i Temasläppssittningen (med D6), Pubrunda (med F6), SSS (Sektionsstyrelsesittning) och hjälpt till inför jubileumsbalen. Det har gått bra men varit mycket jobb. Samarbetena med andra sexmästerier har gått väldigt bra och varit kul.

SRE jobbar endast med CEQ:er just nu, många möten och utvärderingar av kurser.

CM flyter på, en doktorand har beställt ett flertal mackor till ett föredrag hen ska hålla.

KM gick bra under jubileumsveckan.

NollU arbetar klart nollningschemat.
\subp{C}{Ekonomisk rapport}{\info}
Sophia meddelade att ekonomin mår bra, det är mycket pengar från jubileet både inkomster och utgifter.
\subp{D}{Kåren informerar}{\info}
Kåren har inget speciellt att meddela. Las Gegas var nice.
\end{paragrafer}
\newpage
\p{11}{Mentorsprogrammet}{\dis}
Det är just nu cirka 30 mentorer, det är fler BME än E. Det ser ganska ljust ut just nu, Erik och Pontus tror att det kommer att gå bra. Den utbildningen som ska hållas kommer att vara 2 timmar.
\p{12}{Inköp av bestick}{\bes}
Linnea meddelade att sexet klarar av kostnaden på E6:s budget.

\p{13}{Extrainsatt sektionsmöte}{\bes}
Erik \ypa godkänna sitt eget beslut om att lägga det extrainsatta sektionsmötet den 18 maj med anledning av att välja SRE-ordförande.

\Mbaby

\p{14}{Elektrotjejer söndag}{\dis}
På andra sektioner så träffas tjejerna på söndagen innan nollningen. Det har varit uppskattat av folk.

Tidigare år har programledningen velat att Elektra ska ta hand om det men då har sektionen gått emot då Elektra inte är en del av sektionen. I år har Elektra inte velat ta på sig det ansvaret.

Pontus berättade att de har snackat om detta i programledningen. En sån här träff sker innan terminsstart och det är det största hindret som programledningen ser. Problemet med detta är att folk som vill vara med kanske inte har möjligheten att delta.

Mötet anser att detta är en bra sak. Att välkomna nya till sektionen och få dem att stanna. Om Elektra inte vill så ser en del av mötet det inte konstigt att sektionen tar över uppdraget.

Sektionen kommer att föra en diskussion med Elektra, Erik kommer att ta tag i det.

\p{15}{Nästa styrelsemöte}{\bes}
{\Mba} nästa styrelsemöte ska äga rum 2017-05-19 12:10 i E:1426.

\p{16}{Beslutsuppföljning}{\bes}

Kostnaden för ``Målning av cylindern'' uppgick till \SI{842}{kr}, budget var \SI{2000}{kr}.

Albin \ypa stryka ``Målning av cylindern'' från beslutsuppföljningen.

\Mbaby
%{\Ibfu}

\p{17}{Övrigt}{\dis}
Niklas berättade att den 3:e juni har NollU tänkt att hålla en sittning för KM och E6, de som vill är välkomna.

Daniel påpekade att vi snart måste börja diskutera hur vi göra med LED när Ulla slutar.

Niklas berättade NollU behöver ha ett möte med styrelsen. Måndagen den 22:a kl 15.00 är nu inbokat.

\p{18}{OFMA}{\bes}
{\mo} förklarade mötet avslutat 13:04.

\end{paragrafer}

\newpage
\hidesignfoot
\begin{signatures}{3}
\signature{\mo}{Mötesordförande}
\signature{\ms}{Mötessekreterare}
\signature{\ji}{Justerare}
\end{signatures}
\end{document}
