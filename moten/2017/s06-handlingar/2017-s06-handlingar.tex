\documentclass[10pt]{article}
\usepackage[utf8]{inputenc}
\usepackage[swedish]{babel}

\def\doctype{Handlingar} %ex. Kallelse, Handlingar, Protkoll
\def\mname{styrelsemöte} %ex. styrelsemöte, Vårterminsmöte
\def\mnum{S06/17} %ex S02/16, E1/15, VT/13
\def\date{2017-02-23} %YYYY-MM-DD
\def\docauthor{Erik Månsson}

\usepackage{../e-mote}
\usepackage{../../../e-sek}

\begin{document}

\heading{{\doctype} till {\mname} {\mnum}}

\section*{Information om punkterna}

\begin{paragrafer}

\p{11}{Tankar och idéer på hur vi ska utveckla styrelsearbetet}{\dis}
Efter TLTH:s styrelseutbildning lyfter jag frågan om styrelsen har tankar och idéer på hur vi vill utveckla och effektivisera styrelsearbetet och även sektionen som helhet. Finns det saker vi borde lägga till, ta bort eller ändra på?

\rnamnpost{Pontus Landgren}{SRE-ordförande}

\p{12}{Attestering}{\dis}
Läs de föreslagna riktlinjerna i driven.

\end{paragrafer}

\begin{signatures}{1}
\ist
\signature{\docauthor}{Ordförande}
\end{signatures}

\end{document}
