\documentclass[10pt]{article}
\usepackage[utf8]{inputenc}
\usepackage[swedish]{babel}

\def\mo{Fredrik Peterson}
\def\ms{Erik Månsson}
\def\ji{Johan Persson}
%\def\jii{}

\def\doctype{Protokoll} %ex. Kallelse, Handlingar, Protkoll
\def\mname{styrelsemöte} %ex. styrelsemöte, Vårterminsmöte
\def\mnum{S06/16} %ex S02/16, E1/15, VT/13
\def\date{2016-02-25} %YYYY-MM-DD
\def\docauthor{\ms}

\usepackage{../e-mote}
\usepackage{../../e-sek}

\begin{document}
\showsignfoot

\heading{{\doctype} för {\mname} {\mnum}}

%\naun{}{} %närvarane under
%\nati{} %närvarande till och med
%\nafr{} %närvarande från och med
\section*{Närvarande}
\subsection*{Styrelsen}
\begin{narvarolista}
\nv{Ordförande}{Fredrik Peterson}{E14}{}
\nv{Kontaktor}{Erik Månsson}{E14}{}
\nv{Förvaltningschef}{Anders Nilsson}{E13}{}
\nv{Cafémästare}{Stephanie Mirsky}{E13}{}
\nv{Øverphøs}{Molly Rusk}{BME14}{}
\nv{SRE-ordförande}{Johan Persson}{E13}{}
%\nv{ENU-ordförande}{Johannes Koch}{E13}{\nafr{}}
\nv{Sexmästare}{Martin Gemborn Nilsson}{E14}{}
\nv{Krögare}{Malin Lindström}{BME14}{}
\nv{Entertainer}{Dalia Khairallah}{E15}{}
\end{narvarolista}

\subsection*{Ständigt adjungerande}
\begin{narvarolista}
%\nv{Valberedningens ordförande}{Elin Magnusson}{}{}
%\nv{Skattmästare}{Sophia Grimmeiss Grahm}{}{}
\nv{Kårrepresentant}{Daniel Damberg}{}{}
\nv{Kårrepresentant}{John Alvén}{}{}
%\nv{Talman}{Johan Westerlund}{E11}{}
%\nv{Elektras Ordförande}{Elisabeth Pongratz}{}{}
%\nv{Inspektor}{Monica Almqvist}{}{}
\end{narvarolista}

\begin{comment}
\subsection*{Adjungerande}
\begin{narvarolista}
%\nv{Post}{Namn}{Klass}{}
\end{narvarolista}
\end{comment}

\section*{Protokoll}
\begin{paragrafer}
\p{1}{OFSÖ}{\bes}
Ordförande {\mo} förklarade mötet öppnat 12:20.

\p{2}{Val av mötesordförande}{\bes}
\valavmo

\p{3}{Val av mötessekreterare}{\bes}
\valavms

\p{4}{Tid och sätt}{\bes}
\tosg

\p{5}{Adjungeringar}{\bes}
\ingaadj

%Förnamn Efternamn adjungerades

\p{6}{Val av justeringsperson}{\bes}
\valavj

\p{7}{Föredragningslistan}{\bes}
Fredrik \ypa att lägga till \S11.5 ``Kårens värdegrund''.

Molly \ypa att lägga till \S11.75 ``Phadderkickoff i läsvecka -1''.

Föredragningslistan godkändes med samtliga yrkanden.

\p{8}{Föregående mötesprotokoll}{\bes}
\latillprot{S10/16, S11/16 och S12/16}
%\ingaprot

\p{9}{Fyllnadsval/Entledigande av funktionärer}{\bes}
\begin{fyllnadsval} %"Inga fyllnadsval." fylls i automatiskt
%\fval{}{}
\entl{Henrik Fryklund}{Världsmästare}
\end{fyllnadsval}

\p{10}{Rapporter}{}
\begin{paragrafer}
\subp{A}{Check in}{\info}
Erik mår bra, men har galet mycket tentaplugg att göra.

Anders har också för mycket att göra, och för lite tid.

Martin har också mycket att göra med plugget. Funderar på att lösa det genom att inte göra tentorna. Ska hålla sitting med W innan ET.

Stephanie mår bra. Caféfesten gick bra och folk hade skoj! En L:are kom igår och ville bli diod.

Malin mår bra. Ska ha gille på fredag. Pubrunda idag

Johan håller på med CEQ-möte just nu

Molly har mycket att göra, både med Phöset och i skolan. Ska köra phadderkickoff ikväll. Har fixat nollningskontrakt men långt ifrån alla har skrivit på, vilket är segt.

Dalia mår bra, lite trött bara. Har inte så mycket att göra i skolan, har bara en tenta. Det händer inte så mycket med NöjU just nu. Planeringen till UtEDishot går jättebra.

Fredrik har mycket att göra och tänker lösa problemet som Martin - genom att inte göra alla tentor. Har fixat med funktionärsutvärderingen

\subp{B}{Kåren informerar}{\info}

Kåren har inte så mycket att rapportera. Hade FM i tisdags där det beslutades om en ny likabehandlingspolicy.

\subp{C}{Ekonomi}{\info}

Anders pratade lite om ekonomin, vilken mår bra.

\end{paragrafer}

\p{11}{Inköp av maskotdräkt}{\bes}
Fredrik vill köpa in en maskotdräkt som kostar ungefär 300 USD, där tull och moms tillkommer så det blir ungefär 3500 kr.

Mötet var positivt till detta.

Beslutet senarelades.

\p{11.5}{Kårens värdegrund}{}
Från mötet från värdegrunsworkshopen har det kommit upp en fråga.

Vilket samhällsansvar ska kåren ha i värdegrunden? Ska kåren ta samhällsansvar.

Martin tyckte att det var lite konstigt att det skulle stå i styrdokumenten om att man kan/ska ta samhällsansvar.

Daniel sa att det handlar om var kårens pengar tar vägen. Just nu är det oklart om kåren har rätt att t.ex. donera pengar till samhällsproblem.

Fredrik tyckte att det inte ska stå något om detta i värdegrunden, eftersom det inte passar in i kårens mål och visionsdokument.

Malin tycker inte heller att det passar in i värdegrunden.

Fredrik berättade lite om LTH:s och KTH:s värdegrund.

Erik undrade om det handlar att donera pengar eller...

John sa att det handlar om kursen ska kunna lägga resurser på evenemang som handlar om samhällsfrågor.

Erik sa att donera pengar är lite konstigt men att lägga resurser/pengar på evenemang som ..

\p{11.75}{Phadderkickoff i läsvecka -1}{}
Malin sa att det kommer ligga en phadderkickoff på fredagen vecka läsvecka -1 som hon vill att hela styrelsen ska vara med på. Där kommer vara lite viktig information styrelsen bör ta del av. Eftersläpp på kvällen!

\p{12}{Nästa styrelsemöte}{\bes}
\Mba nästa styrelsemöte preliminärt ska äga rum 2016-06-08 18:00 i E-huset. De som inte kan närvara fysiskt kan vara med på Skype istället.

\p{13}{Beslutsuppföljning}{\bes}
\Ibfu

\p{14}{Övrigt}{\dis}
Malin sa att vi behöver lite nya sopborstar till Edekvata, och att det är stökigt i köket.

Malin sa att tangentbordet är konstigt HK. Erik sa attt hon skulle fråga Macapärerna om hjälp.

Malin frågade om vi har en stavmixer, vilken ingen trodde att vi har.

Martin sa att det kommer behövas mer bleck till nollningen.

Stephanie frågade hur oroliga vi behöver vara om D-sektionen vakantsätter sin Teknikfokusansvarig.

Mötet pratade om att näringslivsutskotten får samarbeta för att få ihop Teknikfokus.

Fredrik tycker att det inte är något problem, eftersom han gärna ser att näringslivsutskotten får bättre kontakt med Teknikfokus.

Fredrik bad Cölen att köpa en blomma till Ulla som vi gemensamt ska gå ner och lämna nu.

\p{15}{OFSA}{\bes}
{\mo} förklarade mötet avslutat 12:59.

\end{paragrafer}

\newpage
\hidesignfoot
\begin{signatures}{3}
\signature{\mo}{Mötesordförande}
\signature{\ms}{Mötessekreterare}
\signature{\ji}{Justerare}
\end{signatures}
\end{document}
