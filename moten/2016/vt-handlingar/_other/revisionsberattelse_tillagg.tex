\documentclass[../_main/handlingar.tex]{subfiles}

\begin{document}
\subsection{Tillägg till revisionsberättelse för E-sektionen 2015}
Vid genomgång av protokoll har några punkter uppmärksammats. Styrelseprotokollen är överlag lätta att följa och det är lätt att förstå vilka beslut som har fattats. Sektionsmötesprotokollen däremot har sedan ett antal år tillbaka varit relativt svårtolkade. Enligt stadgarna så behöver endast beslutsprotokoll föras på sektionsmöten. Med det sagt menas det inte att det är fel om det även är med diskussion i protokollet. Dock kan detta leda till att protokollen blir svårlästa och att det som är viktigt försvinner i mängden. Likaså är formen som E-sektionens sektionsmötesprotokoll är skrivna på oläslig om man inte läser handlingarna parallellt. En ytterligare förbättring som också skulle göra protokollen mer tillgängliga är ett system för hur man numrerar yrkanden, något som kan vara extra värdefullt när yrkandena blir många. Vi skulle vilja rekommendera att man under nästkommande verksamhetsår ser över formen för hur protokoll skrivs med syfte att göra de mer tillgängliga (revisorssuppleanterna hjälper gärna till ifall stöd och input önskas).

Efter granskning av ekonomin har inga oegentligheter uppdagats. Styrelsen har tagit till sig av de rekommendationer som gavs efter föregående bokslut och tagit fram internbudgetar för de olika resultatenheterna. De har också varit noggranna med att följa budgetriktlinjerna vilket underlättat granskningen betydligt. Utöver detta har rutinerna för lagerhantering och bokföring setts över och förbättrats vilket lett till att hela styrelsen nu har en bättre insikt i det ekonomiska läget, framförallt för det egna utskottet.

De evenemang som anordnas i samarbete med D-sektionen, Teknikfokus och Utedischot, har inte följt budget. Detta är ett återkommande problem då D-sektionen detaljstyr budgeten för dessa evenemang, utan att sedan följa denna. Vi rekommenderar att man inför nästa år sätter sig ned med D-sektionen och tar fram en lösning för att detta inte ska fortsätta vara fallet i framtiden.

Nedan lyfter vi fram några av de resultatenheter som skiljer sig mest från budget eller på annat sätt är särskilt intressanta:

\textbf{NOLLU01} - Resultatenheten gick 15000 kronor mindre minus än budgeterat. Tycker sektionen fortfarande att man vill lägga 40000 kronor på nollningen så bör nästkommande phös fundera på hur man skulle kunna spendera dessa pengar så att de går nollorna till gagn. Finns det exempelvis något idag som man skulle kunna bjuda på istället eller subventionera för nollor?

\textbf{CM01} - Resutatenheten gick back istället för 71000 kronor plus. Vi kan inte komma på några andra anledningar till resultatet än de som nämns i verksamhetsberättelsen. Vår uppfattning är att CM har funderat igenom rutiner och kommit på åtgärder för hur man ska undvika överraskande resultat i framtiden och kunna ha bättre kontinuerlig koll på hur det går.

\newpage

\textbf{KM01 + SEX01} - Det har visat sig att kostnaden för serveringstillståndet är betydligt högre än vad som anges i budgeten vilket bör åtgärdas inför nästkommande budgetering. Trots detta har båda dessa resultatenheter överstigit budgeten avsevärt. Då detta i princip enbart beror på den vinst vi gör på alkoholförsäljningen, vilken vi måste göra enligt svensk lag, kan vi inte ha kvar budgeten som den är i dag om vi vill bibehålla den verksamhet vi har. Därför bör budgeten uppdateras för att motsvara verkligheten bättre. Med tanke på svensk alkohollag bör man också se över hur många enheter man beräknar på en flaska, en box eller ett fat så att man inte riskerar att gå back på dessa.

Slutligen vill vi bara säga bra jobbat till styrelsen och samtliga funktionärer under 2015, tack för ett bra samarbete!

\begin{signatures}{2}
    \mvh
    \signature{Sara Gunnarsson}{Revisor 2015}
    \signature{Kajsa Eriksson Rosenqvist}{Revisor 2015}
\end{signatures}

\end{document}
