\documentclass[../_main/handlingar.tex]{subfiles}

\begin{document}
\stadgeandring{Revidering av stadgar och reglemente ångående SRE}
\begin{itshape}
Styrelsen 2015 har skrivit fel i sitt svar på motionen, där de råkat skriva ``reglementet'' istället för ``stadgan''. Det har nu rättats till, tillsammans med några andra små stavfel. Det har också gjorts ett förtydligande av vilket år undertecknade satt på sin post.

Motionen biföll i sin helhet med det ytterligare yrkandet nedan under HT/15. Därför har att-satserna om reglementesändringarna redan trätt i kraft.
\end{itshape}

\subsection*{Motion: Revidering av stadgar och reglemente ångående SRE}

Studierådet har under det gångna året noggrant granskat samtliga beskrivningar av SRE i E-sektionens stadgar och reglemente och upptäckt att alltför mycket tyvärr inte alls överensstämmer med utskottets utformning och verksamhet idag. Med syftet att lyfta fram detta utskott till nutid samt underlätta för kommande utskottsordföranden och funktionärer önskar vi i studierådet göra de förändringar denna proposition behandlar.

E-sektionens stadgar gällande studierådet efterlevs i hög grad inte och överensstämmer inte alls med övriga utskotts beskrivningar genom en för stor detaljrikedom på ett vis som snarare hör hemma i reglementet. Att verksamheten bedrivs på ett sätt som inte överensstämmer med stadgarna gör att dessa förlorar sin legitimitet och därför bör de berörda delarna kraftigt revideras, eller som vi föreslår helt strykas, för att undvika detta. En lägre grad av detaljstyrning i stadgarna ger även utskottet och dess medlemmar en högre grad av frihet vilket underlättar en för utskottet viktig anpassning av verksamheten efter rådande krav som ställs från sektionen och av LTH:s instanser där SRE finns representerade.

Vad gäller valet av studierådets funktionärer väljs majoriteten av dessa i nuläget på vårterminsmötet. Då SRE-ordförande numera väljs in, tillsammans med resterande styrelseposter, på valmötet, hade det av praktiska skäl varit önskvärt att även resterande SRE-poster väljs in på hösten. Detta leder till ett bättre samarbete inom utskottet vilket också gynnar utskottets arbete. När det gäller årskursrepresentanter måste dessa sitta på mandattiden läsår med anledning av posternas funktion, det vill säga ``Årskurs E/BME - (1-3) ansvarig'', men av skälet att inga årskurs 1-ansvariga har möjlighet att väljas in vid ett Vårterminsmöte samt vikten av att snabbt få in årskursrepresentanter vid behov i SRE för att möjliggöra ett effektivt arbete i programledningarna bör dessa poster väljas av styrelsen på läsår. En reglementesändring behöver göras under beskrivningen av antal SRE-ordförande för att få ett gemensamt utseende på samtliga utskottsordförande. Då SRE-ordförande är utskottsordförande skall denna även benämnas med ``(u)'' i antal.

Därför yrkar studierådet på
\begin{attsatser}
  \att i stadgarna stryka \S9:12:1 Sammansättning\par
  \begin{itshape}
    Studierådet består av\\
    a) studierådsordförande,\\
    b) vice studierådsordförande,\\
    c) sekreterare, samt\\
    d) funktionärer enligt Reglementet.
  \end{itshape}

  \newpage
  \att i stadgarna stryka \S9:12:2 Mandattid\par
  \emph{För studierådet är mandattiden 1/7-30/6. Representanter i högskolans organ har den mandattid som respektive organ fastställt.}

  \att i stadgarna stryka \S9:12:3 Skyldigheter\par
  \emph{Det åligger Studierådet att sköta utbildningbevakningen och representationen i högskolans upprättade organ, samt vad som där äga sammanhang.}

  \att i stadgarna stryka \S9:12:4 Stormöte-SRE\par
  \begin{itshape}
    Minst en (1) gång per läsår skall Studierådet hålla ett Stormöte-SRE.\\
    Till Stormötet-SRE har alla sektionens medlemmar rätt att närvara med yttrande och yrkanderätt.\\
    Rösträtt tillkommer dem som uppfyller \S2:1:1.
  \end{itshape}

  \att i stadgarna stryka \S9:12:5 Utlysande\par
  \begin{itshape}
    Kallelse och föredragningslista till Stormöte-SRE skall av Studierådet uppsättas på sektionens anslagstavla senast elva (11) läsdagar före Stormötet-SRE.\\
    Möteshandlingar skall finnas tillgängliga på sektionens anslagstavlor senast fem (5) läsdagar innan mötet.
  \end{itshape}

  \att i stadgarna under \S2:1:3 Rättigheter ändra följande att-sats från:\par
  \emph{att deltaga vid Sektionsmöte och stormöten-SRE, med rösträtt, yrkanderätt och yttranderätt,}\par
  till:\par
  \emph{att deltaga vid Sektionsmöte med rösträtt, yrkanderätt och yttranderätt,}

  \att i stadgarna \S9:12 Studierådet, SRE lägga till\par
  \emph{Det åligger Studierådet att sköta utbildningsbevakningen och representationen i LTH:s instanser, samt vad som där äga sammanhang.}

  \newpage
  \att i reglementet ändra \S10:2:M Funktionärerna i Studierådet, SRE, under Årskurs E-1 ansvarig, Årskurs E-2 ansvarig, Årskurs E-3 ansvarig, Årskurs BME-1 ansvarig, Årskurs BME-2 ansvarig samt Årskurs BME-3 ansvarig, lägga till punkten\par
  \emph{- väljs per läsår av styrelsen på rekommendation av SRE-ordföranden.}

  \att de SRE-funktionärer som väljs in på Vårterminsmötet 2016 sitter under halvmandattid (1/7-31/12).

  \att ändra texten under \S10:2:M i reglementet från ``SRE-Ordförande (1)'' till ``SRE-Ordförande (u)''.
\end{attsatser}

\begin{signatures}{4}
    \mvh
    \signature{Sofia Karlén}{SRE-ordförande 2015}
    \signature{Fredrik Peterson}{Studierådet 2015}
    \signature{Johan Persson}{Studierådet 2015}
    \signature{Molly Rusk}{Vice SRE-ordförande 2015}
\end{signatures}

\subsection*{Ytterligare yrkanden}

\begin{attsatser}
    \att i stadgan \S12:5 Högskoleorgan ändra från\par
    \begin{itshape}
      Representanter i högskolans organ nomineras av Stormöte-SRE.\\
      Nominering av representanter förbereds av Studierådet.\\
      Fyllnadsnominering görs av Studierådet.
    \end{itshape}\par
    till\par
    \begin{itshape}
      Nominering av representanter görs av Studierådet.\\
      Fyllnadsnominering görs av Studierådet.
    \end{itshape}
\end{attsatser}


\end{document}
