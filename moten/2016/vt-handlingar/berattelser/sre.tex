\documentclass[../_main/handlingar.tex]{subfiles}

\begin{document}
\berattelse{Studierådet 2015}

SRE har censurerat de CEQ-enkäter som E och BME fyllt i och har därefter representerat och fört fram åsikterna från respektive klass för respektive kurs vid CEQ-möten.

SRE:s ordförande har delat med sig av E-sektionens aktuella studiebevakning under SRX-möten (Studierådsordförandekollegiets möten), samlat inspiration från övriga sektionernas studierådsordförande samt diskuterat studierådens gemensamma mål under TLTH.

Studentrepresentanter i utbildningsnämnd A (UNA) och institutionsstyrelserna (INST) för elektro- och informationsteknik, biomedicinsk teknik, reglerteknik samt datavetenskap har tillsatts.

SRE har nominerat två lämpliga lärare till LTH:s och senare LU:s pedagogiska pris.

SRE har deltagit i programledningsmöten (PLE respektive PLBME ) och fört fram studenternas åsikter angående kurser och program samt fått information om, liksom diskuterat, programmens aktuella status (exempelvis vad gäller kursändringar och utbytesstudier).

SRE har infört en officiell logga!! Utskottet har även kommit överens om en officiell utskottströja, en blå hoodie med vår nya logga på ryggen, som just nu finns i tolv exemplar. Detta med syfte att synas bättre samtidigt som vi stärker gemenskapen inom utskottet.

SRE har genomfört en åsiktshörna i E-foajen med syftet att med hjälp av egenskrivna enkäter förse PL med direkta åsikter, framförallt angående programupplägg och studiemiljö som inte CEQ täcker.

SRE har under nollningen 2015:
\begin{dashlist}
\item planerat och genomfört fyra mycket välbesökta pluggkvällar samt en sykväll.
\item för första gången hållit i en Workshop för nollorna i två delar, en informerande angående varför studiebevakning är viktigt och en gruppövning i diverse frågor från likabehandling till studieteknik samt ”Att plugga på LTH”.
\end{dashlist}

SRE har rekryterat flera nya funktionärer under hösten. Av denna anledning har SRE-ordförande hållit i en liten utbildning i bland annat arbetsrapportläsning och CEQ-censurering.

SRE har stärkt relationerna med programledningarna för E och BME, programplanerare Åsa Vestergren samt studievägledare Ingrid Holmberg, vilket är mycket viktigt för hjälp, kontinuerlig sponsring samt allmän uppmuntran i SRE:s arbete.

\begin{signatures}{1}
    \mvh
    \signature{Sofia Karlén}{SRE-ordförande 2015}
\end{signatures}

\end{document}
