\documentclass[../_main/handlingar.tex]{subfiles}

\begin{document}
\berattelse{Nolleutskottet 2015}

NollU har ägnat sig åt nollning.

Under våren valdes 62 phaddrar varav 2 phusknollor, 20 uppdragsphaddrar och 4 internationella phaddrar för att ta emot nollorna på bästa möjliga sätt. En nyhet var att man sökte i mindre grupper och så slog phøset ihop dessa grupper till större grupper. Detta ledde till att antalet phaddergrupper minskade till totalt 10st exkl. internationella gruppen.

Phøset har varit på många möten tillsammans andra phøs, kåren, programledningen, matematiklärare och SI-ledare samt SVL. Vi anordnade flera aktiviteter med de andra phøsen som t.ex. ett gemensamt eftersläpp för nollningsaktiva efter temasläppet och phadderkickoff. Vi deltog aktivt i flera utbildningar och planerade nollningen in i minsta detalj. Mycket fokus har lagts på ekonomi, kontinuerlig utvärdering och att öka samarbetet med D-sektionen.

Under sommaren tog NollU lite sommarlov och åkte på semester i Karlstad där solen alltid skiner.

E-sektionen deltog aktivt i de gemensamma aktiviteterna som kåren anordnade. Helt magiskt presterade sektionen utöver det naturliga och vann både Livbojjen och Bildörren, något som aldrig har gjorts tidigare av någon sektion! Efter årets nollningen kan man stolt säga att E-sektionen är den mest taggade sektionen som aldrig får nog av att bada!

En nyhet i år var den nya uppdragsgruppen Spexet som levererade ett fenomenalt spex innan NollEGasquen som vi hoppas på kommer att bli en fin tradition.

Nu efter nollningen har phøset arbetat med att skriva testamente och förbereda en bra överlämning till nästa års phøs.

Avslutningsvis vill vi säga att det var inte en enda dag under nollningen som vi inte kände att vi hade hjälp och stöd från hela sektionen. Det har varit otroligt skönt och vi är glada och stolta över den fina gemenskapen som finns på E-sektionen samt att den gemenskapen nu har växt med årets 130 nya ettor.

\begin{signatures}{1}
    \mvh
    \signature{Henrik Fryklund}{Øverphøs 2015}
\end{signatures}

\end{document}
