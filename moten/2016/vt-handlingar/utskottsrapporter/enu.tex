\documentclass[../_main/handlingar.tex]{subfiles}

\begin{document}
\utskottsrapport{Näringslivsutskottet}

Näringslivsutskottet har delat in sig i två grupper med två olika huvudsyften; PR och Företagskontakt. PR-gruppen sköter marknadsföring i form av affischer, bilder till tv-skärmarna, anmälningsformulär m.m. Företagsgruppen ringer och mailar företag i syfte att hitta nya samarbeten med näringslivet. Eventgruppen från tidigare år är borttagen och ersatt med att ENUs funktionärer i år håller i olika projekt från start till mål.

ENU började sitt utskottsår med en mycket uppskattad kick-off som vice och ordförande höll i. Årets första event var CV-fotografering för sektionens medlemmar. Eventet var både lyckat och uppskattat av sektionens medlemmar. Fotograferingen kombinerades senare med CV-granskning som Academic Work höll i. E- och D-sektionens näringslivsutskott har också anordnat en lunchföreläsning tillsammans med Tetra Pak för mjuvaruintresserade studenter i de högre årskullarna. ENU har även lanserat eventet ``Lunch med en ingenjör'' som startade vecka 14.

ENU har även skaffat tröjor med ENUs logga på ryggen för att kunna göra ett enhetligt intryck inför företagen. Utskottet har haft flera ``stormöten'' där alla utskottets funktionärer har deltagit och utformat verksamheten.

Teknikfokus 2016 var större än tidigare mässor, med 29 betalande företag, varav ett var medicintekniskt. Förberedelserna inför mässan flöt på bra och mässan fick mycket bra kritik från företagen. Tyvärr så var funktionärsdeltagandet från E-sektionens sida väldigt lågt, vilket är något som måste förbättras till kommande år.

\begin{signatures}{1}
    \mvh
    \signature{Johannes Koch}{ENU-ordförande}
\end{signatures}

\end{document}
