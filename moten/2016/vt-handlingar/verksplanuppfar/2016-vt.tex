\documentclass[../_main/handlingar.tex]{subfiles}

\begin{document}
\verksplanuppf{VT 2016}
I höstas infördes Sektionens första (?) verksamhetsplan och hur man jobbar med denna, dess uppföljning samt innehållet i den kommer troligtvis att utvecklas med tiden om inslaget kommer att finnas kvar inom vår verksamhet. I det stora hela kan man säga att styrelsen inte på något sätt aktivt har jobbat med verksamhetsplanen. Konstateras kan dock att många delmål för 2016 trots detta redan har uppfyllts. Nedan gås kort igenom hur de olika utskotten ser på dem delmål som satts upp för året och hur arbetet hittills har gått med dessa; märkväl utan någon egentlig medvetenhet om dess innehåll.

\subsubsection*{Styrelsen}
Att utvärdera styrelsen och Sektionens struktur är ett arbete som skulle ta mycket tid om det skulle bli ordentligt gjort och jag skulle vilja påstå att det finns saker som behöver göras innan det blir aktuellt. Kontantfria betalningsalternativ har införts. Teknikfokus finns på agendan och likaså att synliggöra styrelsens arbete, något som redan har jobbats en del med. Delar av styrelsen arbetar med att förslag för hur vi ska tacka och visa uppskattning till våra funktionärer. Samarbetet mellan utskottsordförandena fungerar för det mesta väldigt bra. Tankar om överlämning har tänkts och förhoppningsvis kommer dessa att konkretiseras under hösten. De långsiktiga planerna skulle det behöva jobbas med, viktigt att tänka på i detta sammanhang är att inte låsa Sektionens verksamhet utan att den fortfarande måste vara väldigt lättanpassad efter rådande omständigheter.

\subsubsection*{Informationsutskottet}
Informationsspridningen på Sektionen har fungerat bra och vi upplever att vi når ut till alla. Vi tittar just nu på att uppdatera datorerna i HK, fixa backupen och utvärdera vad vi ska göra med switchen i HK. Vi har en diskussion om hur vi ska utveckla Alumniverksamheten. En utvärdering av informationskanaler kommer senare under året.

Nu i början av året har Kontaktorn haft en del kontakt med andra sektioner och KTH, och siktar på att fortsätta kontakten med fler sektioner och skolor senare under året.

\subsubsection*{Källarmästeriet}
Vi har arbetat mycket med marknadsföring av gillen. Vi har fortsatt att sätta upp planscher i E-huset och information på TV-skärmarna. Vi har dessutom satt upp planscher i de andra husen, varit med i kårnytt och börjat göra evenemang på facebook, till de flesta arrangemang som anordnas. Det verkar ha gett god effekt då fler från andra sektioner har hittat till gillena. Dessutom jobbas det för att ha fler gillen med andra sektioner som indirekt marknadsför gillena.

Ingen speciell utvärdering av ansvarsfördelningen har gjort hittills, men kommer göras så småningom. Dock fungerar strukturen bra som den är nu.

\subsubsection*{Nolleutskottet}
Vi har under vårt arbete hittills jobbat mycket med att få en mångfald av aktiviteter. Detta har gjorts genom att införa nya samt ändra gamla uppdrag och evenemang som passar flera olika personer. För att öka spridningen på aktiviteter för nollorna har mycket av arbetet gjorts tillsammans med andra utskott vilket gett nya möjligheter i arbetet.

Vi har tagit även kontakt med Sektionens världsmästare för att jobba för att bättre integrera internationella studenter under nollningen och kommer ha våra internationella phaddrar till hjälp för detta.

Vi har via dels det gamla phøset men även andra på Sektionen fått respons från vad folk tyckt om tidigare års nollningar och tagit denna feed-back till oss för att göra nollningen 2016 så bra den bara kan bli.

En renhållningsplan för lokalerna under nollningen har ännu inte hunnits med men vi kommer att titta noggrannare på detta under våren innan nollningen börjar.

\subsubsection*{Cafémästeriet}
Cafémästeriet jobbar kontinuerligt för att minska svinn så att så lite som möjligt ska kastas och detta kommer fortgå. Likaså gäller utvecklingen av LED:s sortiment, det jobbas kontinuerligt med detta, eventuellt kommer vi åka till en så kallad inspirationsmässa från vår leverantör. Mojtarna har inte arbetats med då dessa inte längre finns kvar.

\subsubsection*{Förvaltningsutskottet}
Förvaltningsutskottet har arbetat med styrelsen för en kortsiktig plan för Edekvata men en direkt långsiktig plan saknas, istället har vi valt att lägga fram en proposition som föreslår hur lokalerna ska kunna jobbas med i längden.

\subsubsection*{Studierådet}
Studierådet fortsätter arbetet med att förbättra synligheten inom Sektionen, med stor fokus på att visa upp oss under nollningen så att nya studenter vet varför vi finns och vad vi kan göra. Vi arbetar även med att ha fler pluggkvällar jämnt utspridda under året och att under nollningen ändra på pluggkvällarnas struktur för att ge bättre studiero. Diskussioner om hur svarsfrekvensen på CEQ ska ökas hålls.

\subsubsection*{Sexmästeriet}
Att de de flesta sittningar är under nollningen beror mest på att Sexmästeriet har ganska mycket uppstartsjobb då det är ett stort utskott och att de personer som i största utsträckning ska ska driva utskottet framåt har mycket att lära och ska på ett antal utbildningar innan det går smidigt och enkelt att hålla i evenemang. Många event som t.ex. Skipthet, Teknikfokus och Flickor på Teknis ligger även under våren vilket tar en del fokus från att hålla sittningar för Sektionen. Även det stora problemet med att hitta datum som passar bra har sannolikt en stor inverkan på saken i fråga.

Den 4/1 anordnade E6 tillsammans med 5 andra sexmästerier en stor sittning för sektions-medlemmen. E6 har även för avsikt att under resterande delen av våren, om datumen tillåter, försöka fokusera lite mer på evenemang direkt riktade mot E-sektions-medlemmen. Det blir förhoppningsvis en sittning 4/5 och även ett mindre samarbete med Nöju finns på ritbordet.

Att hålla rent i Pump är något som E6 bör fortsätta och förbättra sig på att göra. Mitt bland alla andra event är dock detta något som enkelt glöms bort. Tyvärr är det lite ont om plats i Pump och det står endel saker på golvet, dock är den ruttna löken i alla fall utkastad! I hyllorna är det dock ganska bra ordning, tacka E6-15 för det!

E6 har ännu inte hållt i något större event i Edekvata. Dock har vi använt köket endel och till största del städat bra efter oss. De gånger det inte har varit perfekt gjort har det i största mån gjorts i efterhand. En sak som E6 skulle kunna bli bättre på är att tömma diskstället.

\subsubsection*{Nöjesutskottet}
Nöjesutskottet har jobbat och jobbar fortfarande på att utveckla beskrivningen av posterna i utskottet. Fortfarande lite diffusa arbetsuppgifter men vi jobbar på att förbättra detta. Vi har under tiden lyckats ha flera olika sorters evenemang och kommer fortsatt att försöka öka mångfalden på evenemangen som anordnas av sektionen. Utvärdering av Utedischot är inte relevant ännu vi siktar på att göra en ordentlig och bra utvärdering av Utedischot 2016.

\subsubsection*{Näringslivsutskottet}
ENU har haft flera samarbeten med befintliga företag och tillsammans med dessa anordnat liknande event som föregående år. Tillsammans med D-sektionens näringslivsutskott har ENU även anordnat en lunchföreläsning med Tetra Pak som riktade sig mot mjukvaruintresserade studenter. ENU har kontaktat och kontaktar kontinuerligt både företag vi tidigare har haft samarbeten och företag som vi önskar inleda samarbeten med.

Införandet av ett mentorskapsprogram har diskuterats på utskottsmöte där det beslutades att vi inte ska införa mentorskapsprogram under detta verksamhetsåret. Istället har ENU lanserat ``Lunch med en ingenjör'' där studenterna kan ställa sina frågor till verksamma ingenjörer.

Både prissättning och spridningen av evenemang under året har utvärderats och setts över. Spridningen av evenemangen anser vi inte ha så stor inverkan på i nuläget, då vi hellre ser att ett event blir av vid en mindre väl val tid än att det inte blir av alls.

\newpage
\begin{signatures}{10}
    \mvh
    \signature{Fredrik Peterson}{Ordförande}
    \signature{Erik Månsson}{Kontaktor}
    \signature{Anders Nilsson}{Förvaltningschef}
    \signature{Stephanie Mirsky}{Cafémästare}
    \signature{Molly Rusk}{Øverphøs}
    \signature{Johan Persson}{SRE-ordförande}
    \signature{Johannes Koch}{ENU-ordförande}
    \signature{Martin Gemborn Nilsson}{Sexmästare}
    \signature{Malin Lindström}{Krögare}
    \signature{Dalia Khairallah}{Entertainer}
\end{signatures}

\end{document}
