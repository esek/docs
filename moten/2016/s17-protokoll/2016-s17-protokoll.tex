\documentclass[10pt]{article}
\usepackage[utf8]{inputenc}
\usepackage[swedish]{babel}

\def\mo{Fredrik Peterson}
\def\ms{Erik Månsson}
\def\ji{Johan Persson}

\def\doctype{Protokoll} %ex. Kallelse, Handlingar, Protkoll
\def\mname{styrelsemöte} %ex. styrelsemöte, Vårterminsmöte
\def\mnum{S17/16} %ex S02/16, E1/15, VT/13
\def\date{2016-09-08} %YYYY-MM-DD
\def\docauthor{\ms}

\usepackage{../e-mote}
\usepackage{../../../e-sek}

\begin{document}
\showsignfoot

\heading{{\doctype} för {\mname} {\mnum}}

%\naun{}{} %närvarane under
%\nati{} %närvarande till och med
%\nafr{} %närvarande från och med
\section*{Närvarande}
\subsection*{Styrelsen}
\begin{narvarolista}
\nv{Ordförande}{Fredrik Peterson}{E14}{}
\nv{Kontaktor}{Erik Månsson}{E14}{}
\nv{Förvaltningschef}{Anders Nilsson}{E13}{}
\nv{Cafémästare}{Stephanie Mirsky}{E13}{}
\nv{SRE-ordförande}{Johan Persson}{E13}{}
\nv{ENU-ordförande}{Johannes Koch}{E13}{}
\nv{Sexmästare}{Martin Gemborn Nilsson}{E14}{\nafr{10A}}
\nv{Krögare}{Malin Lindström}{BME14}{}
\nv{Entertainer}{Dalia Khairallah}{E15}{}
\end{narvarolista}

\subsection*{Ständigt adjungerande}
\begin{narvarolista}
\nv{Kårordförande}{Linus Hammarlund}{}{}
\end{narvarolista}

\subsection*{Adjungerande}
\begin{narvarolista}
\nv{Cophøs}{Viktor Persson}{E14}{}
\end{narvarolista}

\section*{Protokoll}
\begin{paragrafer}
\p{1}{OFSÖ}{\bes}
Ordförande {\mo} förklarade mötet öppnat 12:11.

\p{2}{Val av mötesordförande}{\bes}
\valavmo

\p{3}{Val av mötessekreterare}{\bes}
\valavms

\p{4}{Tid och sätt}{\bes}
\tosg

\p{5}{Adjungeringar}{\bes}
Viktor Persson adjungerades.

\p{6}{Val av justeringsperson}{\bes}
\valavj

\p{7}{Föredragningslistan}{\bes}
Malin \ypa att lägga till \S11.5 ``Inköp av stekbord''.

Föredragningslistan godkändes med yrkandet.

\p{8}{Föregående mötesprotokoll}{\bes}
\ingaprot

\p{9}{Fyllnadsval/Entledigande av funktionärer}{\bes}
\begin{fyllnadsval}
\fval{Jonas Thurborg (E15)}{Diod}
\fval{Seif Sharif (E15)}{Diod}
\fval{Rebecca Ritter}{Diod}
\fval{Kevin Tan (E16)}{Diod}
\fval{Jonathan Jakobsson (E15)}{Diod}
\end{fyllnadsval} %fixat och klart på hemsidan

\p{10}{Rapporter}{}
\begin{paragrafer}
\subp{A}{Check in}{\info}
Anders har det bra och har fullt upp med ekonomin. Revisorerna har börjat gå igenom bokföringen.

Johannes håller på med lunchföreläsningar.

Viktor sa att det går bra för NollU.

Stephanie sa att LED går bra, har fått ihop bra med dioder. I morse hade kylarna lagt av, vilket fixades av PH. Som tur blev inget förstört.

Malin sa att KM mår bra. De har fått sina märken till drinkarna. Stekbordet har gått sönder.

Dalia sa att det går bra för NöjU också. Det var tyvärr inte så många som kom på brännbollshänget. De håller på att fixa inför phadderolympiaden.

SRE går jävligt bra. De håller på med CEQ-enkäterna.

Det går bra för E6, sittningen igår gick superbra.

Erik sa att det går bra för InfU, men att han tyvärr fått nollesjukan.

Fredrik har varit på FlyING och representerat Sektionen på bästa möjliga sätt.

\subp{B}{Kåren informerar}{\info}
Linus presenterade sig själv.

Linus informerade om att kårens servrar har wipeats och att de börjar få upp systemen igen. Fem sektioner tappade sina hemsidor och mailadresser. Heltidarna kan inte jobba på kårens datorer än.

Kåren har haft sitt första kårstyrelsemöte. De har uppdaterat riktlinjerna för rulle och får in nya rulle nästa vecka.

Nollningen går bra!

\subp{C}{Ekonomi}{\info}
Anders rapporterade om ekonomin. Vi omsätter mycket pengar just nu och har mycket utlägg.

\end{paragrafer}

\p{11}{Inköp av dammsugare}{\bes}
Fredrik tycker att vi bör ha en fungerande dammsugare och presenterade några olika alternativ som vi kan köpa in.

Fredrik förslår att köpa in en dammsugare till en kostnad av max 1000kr.

Martin sa att vår dammsugare fungerar men bara saknar munstycke och föreslog att bara in reservdelar istället.

Fredrik sa att vi kan köpa in ett nytt munstycke men sa också att den tar tappat mycket kraft.

\emph{Punkten senarelades.}

\p{11.5}{Inköp av stekbord}{}
Malin informerade om att Sektionens stekbord är trasigt. Hon gav några olika förslag på stekbord som vi kan köpa in.

Hon sa också att vi kan få ett gratis stekbord från Sibylla i Oskarshamn men att det förmodligen är för stort.

\textbf{\Mba köpa in ett stekbord till en kostnad av maximalt 9000kr vilken belastar dispositionsfonden samt att lägga beslutet på beslutsuppföljningen till styrelsemöte 20 med Malin som ansvarig.}

\p{12}{PA-systemet (trasiga sladdar)}{\dis}
Anders sa att det går sönder för mycket sladdar på Sektionen - 2-3 kablar till umphboxarna och en sladd till PA-topparna. Han sa att budgeten för ljud och ljus är övertrasserad redan. Han tycker att vi måste försöka hålla rätt på grejorna och att om det fortsätter misskötas måste vi skriva ut dem istället för att man kan låna när man vill.

Han uppmanade alla att informera sina utskott om att vara försiktiga med vår utrustning till ljud. Han sa att i våras fungerade det bra så han hoppas att det kan fortsätta så.

Erik ville pointera att det är viktigt att man säger till direkt om man råkat ta sönder något, så att nästa person som vill använda utrustningen kan göra det utan bekymmer.

\p{13}{Nästa styrelsemöte}{\bes}
\Mba nästa styrelsemöte ska äga rum 2016-09-15 12:10 i E:1426.

\p{14}{Beslutsuppföljning}{\bes}
\Ibfu

\p{15}{Övrigt}{\dis}
Mötet diskuterade svansen av Spritbolaget.

Stephanie sa att i måndag morse var caféet stökigt och saker var sönder/skitiga. Hon sa att detta är ett stort problem eftersom vi när som helst kan bli inspekterade. Hon sa också att vi inte ska glömma att det är Ullas arbetsplats och uppmanade alla att uppmana sina utskott om att hålla rent.

Malin och Martin sa att det är jättebra att CM har städats. Martin sa också att lite av det förmodligen är sexets fel men sa också att det varit mycket saker i CM som gjort det svårt för sexet att städa ordentligt.

Anders vill pointera att det är viktigt att hålla rent LED inte bara för inspektion och hygien, utan också för att vi inte äger lokalen själva utan huset låter oss ha den.

Anders sa att stolar i LED och foajén har använts som dörrstopp och ibland ställts utomhus, vilket PH inte har tyckt om.

\newpage
\p{16}{OFSA}{\bes}
{\mo} förklarade mötet avslutat 12:57.

\end{paragrafer}

\hidesignfoot
\begin{signatures}{3}
\signature{\mo}{Mötesordförande}
\signature{\ms}{Mötessekreterare}
\signature{\ji}{Justerare}
\end{signatures}
\end{document}
