\documentclass[10pt]{article}
\usepackage[utf8]{inputenc}
\usepackage[swedish]{babel}

\def\mo{Fredrik Peterson}
\def\ms{Erik Månsson}
\def\ji{Malin Lindström}

\def\doctype{Protokoll} %ex. Kallelse, Handlingar, Protkoll
\def\mname{styrelsemöte} %ex. styrelsemöte, Vårterminsmöte
\def\mnum{S14/16} %ex S02/16, E1/15, VT/13
\def\date{2016-06-09} %YYYY-MM-DD
\def\docauthor{\ms}

\usepackage{../e-mote}
\usepackage{../../e-sek}

\begin{document}
\showsignfoot

\heading{{\doctype} för {\mname} {\mnum}}

\section*{Närvarande}
\subsection*{Styrelsen}
\begin{narvarolista}
\nv{Ordförande}{Fredrik Peterson}{E14}{}
\nv{Kontaktor}{Erik Månsson}{E14}{}
\nv{Förvaltningschef}{Anders Nilsson}{E13}{}
\nv{Cafémästare}{Stephanie Mirsky}{E13}{}
\nv{SRE-ordförande}{Johan Persson}{E13}{}
\nv{ENU-ordförande}{Johannes Koch}{E13}{}
\nv{Sexmästare}{Martin Gemborn Nilsson}{E14}{}
\nv{Krögare}{Malin Lindström}{BME14}{}
\end{narvarolista}

\section*{Protokoll}
\begin{paragrafer}
\p{1}{OFSÖ}{\bes}
Ordförande {\mo} förklarade mötet öppnat 18:10.

\p{2}{Val av mötesordförande}{\bes}
\valavmo

\p{3}{Val av mötessekreterare}{\bes}
\valavms

\p{4}{Tid och sätt}{\bes}
\tosg

\p{5}{Adjungeringar}{\bes}
\ingaadj

\p{6}{Val av justeringsperson}{\bes}
\valavj

\p{7}{Föredragningslistan}{\bes}
Föredragningslistan godkändes.

\p{8}{Föregående mötesprotokoll}{\bes}
\latillprot{S13/16}

\p{9}{Fyllnadsval/Entledigande av funktionärer}{\bes}
\begin{fyllnadsval} %"Inga fyllnadsval." fylls i automatiskt
\fval{Anna Goos}{Teknikfokusansvarig}
\entl{My Reimer}{Alumniansvarig E}
\end{fyllnadsval}

\p{10}{Rapporter}{}
\begin{paragrafer}
\subp{A}{Check in}{\info}
Martin har haft sittning som gick bra, typ. Kylskåpen fungderade inte, proppen gick och de hade lite annat strul också. Idag har han bokfört med Anders, vilket gick ganska bra.

Anders har mest hållt på med att bygga om i Edekvata. Han ligger lite efter i bokföringen.

Johan sa inte så mycket.

Stephanie hade inte heller så mycket att säga. Ligger i fas med bokföringen.

Johannes håller på att maila företag.

Erik mår fint, han tycker det är gött med sommarlov.

Malin har frostat av frysen i KM. Har kollat på lite inköp till KM och har köpt en streckkodsläsare.

Fredrik sa att fasadskylten är äntligen uppsatt. Annars är det mest renovering som gäller.

\subp{B}{Kåren informerar}{\info}
\emph{Kåren informerade inte.}

\subp{C}{Ekonomi}{\info}
\emph{Ekonomin diskuterades inte.}

\end{paragrafer}

\p{11}{Inköp av maskotdräkt}{}
Fredrik har mailat med leverantören av maskotdräkten i fråga och har kommit fram till att det rör sig om ett inköp om cirka 300 USD. Han tycker att det verkar vara en seriös leverantör.

Fredrik föreslog att besluta att köpa in maskotdräkten med en budget på 3800kr.

Mötet diskuterade.

\textbf{\Mba köpa in en maskotdräkt till en kostnad på max 4000kr som belastar dispositionsfonden samt lägga detta på beslutsuppföljningen till styrelsemöte 16.}

\p{12}{Inköp av nattfack}{}
Anders informerade om att det behövs ett nytt nattfack till HK, eftersom det gamla nu ligger på tippen.

Anders \ypa köpa in ett nytt nattfack till en kostnad av max 3000kr som belastar dispositionsfonden samt lägga detta på beslutsuppföljningen till styrelsemöte 16.

\textbf{\Mba bifalla yrkandet.}

\p{13}{Namn på rum}{}
Malin undrar vilka namn de nya rummen kommer ha.

Mötet var enade om att behålla de gamla namnen.

\p{14}{Nästa styrelsemöte}{\bes}
\Mba nästa styrelsemöte ska äga rum 2016-08-18.

\p{15}{Beslutsuppföljning}{\bes}
\Ibfu

\p{16}{Övrigt}{\dis}
\emph{Inget övrigt diskuterades.}

\p{17}{OFSA}{\bes}
{\mo} förklarade mötet avslutat 18:28.

\end{paragrafer}

%\newpage
\hidesignfoot
\begin{signatures}{3}
\signature{\mo}{Mötesordförande}
\signature{\ms}{Mötessekreterare}
\signature{\ji}{Justerare}
\end{signatures}
\end{document}
