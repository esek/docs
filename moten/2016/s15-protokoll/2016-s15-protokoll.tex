\documentclass[10pt]{article}
\usepackage[utf8]{inputenc}
\usepackage[swedish]{babel}

\def\mo{Fredrik Peterson}
\def\ms{Erik Månsson}
\def\ji{Stephanie Mirsky}

\def\doctype{Protokoll} %ex. Kallelse, Handlingar, Protkoll
\def\mname{styrelsemöte} %ex. styrelsemöte, Vårterminsmöte
\def\mnum{S15/16} %ex S02/16, E1/15, VT/13
\def\date{2016-08-18} %YYYY-MM-DD
\def\docauthor{\ms}

\usepackage{../e-mote}
\usepackage{../../../e-sek}

\begin{document}
\showsignfoot

\heading{{\doctype} för {\mname} {\mnum}}

%\naun{}{} %närvarane under
%\nati{} %närvarande till och med
%\nafr{} %närvarande från och med
\section*{Närvarande}
\subsection*{Styrelsen}
\begin{narvarolista}
\nv{Ordförande}{Fredrik Peterson}{E14}{}
\nv{Kontaktor}{Erik Månsson}{E14}{}
\nv{Förvaltningschef}{Anders Nilsson}{E13}{}
\nv{Cafémästare}{Stephanie Mirsky}{E13}{}
\nv{Øverphøs}{Molly Rusk}{BME14}{}
\nv{SRE-ordförande}{Johan Persson}{E13}{}
\nv{ENU-ordförande}{Johannes Koch}{E13}{}
\nv{Sexmästare}{Martin Gemborn Nilsson}{E14}{\nafr{10A}}
\nv{Krögare}{Malin Lindström}{BME14}{\nafr{10A}}
\nv{Entertainer}{Dalia Khairallah}{E15}{}
\end{narvarolista}

\subsection*{Ständigt adjungerande}
\begin{narvarolista}
\nv{Talman}{Johan Westerlund}{E11}{}
\end{narvarolista}

\section*{Protokoll}
\begin{paragrafer}
\p{1}{OFSÖ}{\bes}
Ordförande {\mo} förklarade mötet öppnat 13:00.

\p{2}{Val av mötesordförande}{\bes}
\valavmo

\p{3}{Val av mötessekreterare}{\bes}
\valavms

\p{4}{Tid och sätt}{\bes}
\tosg

\p{5}{Adjungeringar}{\bes}
\ingaadj

\p{6}{Val av justeringsperson}{\bes}
\valavj

\p{7}{Föredragningslistan}{\bes}
Föredragningslistan godkändes.

\p{8}{Föregående mötesprotokoll}{\bes}
\latillprot{S14/16}

\p{9}{Fyllnadsval/Entledigande av funktionärer}{\bes}
\begin{fyllnadsval} %"Inga fyllnadsval." fylls i automatiskt
\entl{Linus Bergman (elt15lbe)}{Fritidsledare}
\entl{Emma Anzén}{Valberedningsledamot}
\entl{Henrik Fryklund}{Sexig}
\end{fyllnadsval}

\p{10}{Rapporter}{}
\begin{paragrafer}
\subp{A}{Check in}{\info}
Mötet pratade om hur alla mår, vad de gjort över sommaren och hur det går med nollningsförberedelserna. Överlag går det superbra!

Dalia behöver hjälp med att rigga upp UtEDischot.

\subp{B}{Kåren informerar}{\info}
\emph{Kåren informerade inte.}

\subp{C}{Ekonomi}{\info}
\emph{Punkten togs inte upp.}

\end{paragrafer}

\p{11}{Inköp av maskotdräkt}{\dis}
Fredrik informerade om att det inte blir någon maskotdräkt till denna nollningen eftersom det inte gick ihop med vår leverantör.

Styrelsen var överrens om att ändå köpa in en maskotdräkt vid senare tillfälle.

\p{12}{Inköp av köksredskap}{\bes}
Martin ville köpa in
\begin{dashlist}
    \item 5 bläck
    \item 2 elvispar
    \item inte en vitlökspress
\end{dashlist}
till en kostnad av max 2300kr och yrkade på att detta ska belasta dispositionsfonden.

\textbf{\Mba bifalla yrkandet.}

\p{13}{Nästa styrelsemöte}{\bes}
\Mba nästa styrelsemöte ska äga rum 2016-09-01 12:10 i E:1426.

\p{14}{Beslutsuppföljning}{\bes}
\Ibfu

\p{15}{Övrigt}{\dis}
Johan ville bestämma datum för hösttermins- och valmöte.

Mötet bestämde preliminärt att lägga HT 2016-11-22 och VM 2016-11-29.

Johan erbjöd sig att sälja en portabel högtalare vilket styrelsen var positiv till. Erik tar tag i det.

Johan har pratat med Cederwall om att fixa skyltarna med utexaminerade i entrén.

Molly söker jobbare till eftersläppet på fredag. Det löste sig direkt.

\p{16}{OFSA}{\bes}
{\mo} förklarade mötet avslutat 13:32.

\end{paragrafer}

\newpage
\hidesignfoot
\begin{signatures}{3}
\signature{\mo}{Mötesordförande}
\signature{\ms}{Mötessekreterare}
\signature{\ji}{Justerare}
\end{signatures}
\end{document}
