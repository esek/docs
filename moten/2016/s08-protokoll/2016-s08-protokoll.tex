\documentclass[10pt]{article}
\usepackage[utf8]{inputenc}
\usepackage[swedish]{babel}

\def\mo{Fredrik Peterson}
\def\ms{Erik Månsson}
\def\ji{Anders Nilsson}

\def\doctype{Protokoll} %ex. Kallelse, Handlingar, Protkoll
\def\mname{styrelsemöte} %ex. styrelsemöte, vårterminsmöte
\def\mnum{S08/16} %ex S02/16, E1/15, VT/13
\def\date{2016-04-07} %YYYY-MM-DD
\def\docauthor{\ms}

\usepackage{../e-mote}
\usepackage{../../../e-sek}

\begin{document}
\showsignfoot

\heading{{\doctype} för {\mname} {\mnum}}

%\naun{}{} %närvarane under
%\nati{} %närvarande till och med
%\nafr{} %närvarande från och med
\section*{Närvarande}
\subsection*{Ordinarie}
\begin{narvarolista}
\nv{Ordförande}{Fredrik Peterson}{E14}{}
\nv{Kontaktor}{Erik Månsson}{E14}{}
\nv{Förvaltningschef}{Anders Nilsson}{E13}{}
\nv{Cafémästare}{Stephanie Mirsky}{E13}{}
\nv{Øverphøs}{Molly Rusk}{BME14}{}
\nv{SRE-ordförande}{Johan Persson}{E13}{}
\nv{ENU-ordförande}{Johannes Koch}{E13}{\nafr{10A}}
\nv{Sexmästare}{Martin Gemborn Nilsson}{E14}{}
\nv{Krögare}{Malin Lindström}{BME14}{}
\nv{Entertainer}{Dalia Khairallah}{E15}{}
\end{narvarolista}

\subsection*{Ständigt adjungerande}
\begin{narvarolista}
%\nv{Valberedningens ordförande}{Elin Magnusson}{}{}
\nv{Skattmästare}{Sophia Grimmeiss Grahm}{}{}
\nv{Kårrepresentant}{Daniel Damberg}{}{}
\nv{Kårrepresentant}{John Alvén}{}{}
%\nv{Talman}{Johan Westerlund}{E11}{}
%\nv{Elektras Ordförande}{Elisabeth Pongratz}{}{}
%\nv{Inspektor}{Monica Almqvist}{}{}
\end{narvarolista}

\begin{comment}
\subsection*{Adjungerande}
\begin{narvarolista}
%\nv{Post}{Namn}{Klass}{}
\end{narvarolista}
\end{comment}

\section*{Protokoll}
\begin{paragrafer}
\p{1}{OFSÖ}{\bes}
Ordförande {\mo} förklarade mötet öppnat 12:15.

\p{2}{Val av mötesordförande}{\bes}
\valavmo

\p{3}{Val av mötessekreterare}{\bes}
\valavms

\p{4}{Tid och sätt}{\bes}
\tosg

\p{5}{Adjungeringar}{\bes}
\ingaadj

\p{6}{Val av justeringsperson}{\bes}
\valavj

\p{7}{Föredragningslistan}{\bes}
Erik \ypa lägga till \S11.5 ``Motioner/Propositioner''.

Fredrik \ypa lägga till \S11.75 ``Subventionering av medaljer''.

\textbf{Föredragningslistan godkändes med samtliga yrkanden.}

\p{8}{Föregående mötesprotokoll}{\bes}
\latillprot{S07/16}
%\ingaprot

\p{9}{Fyllnadsval/Entledigande av funktionärer}{\bes}
\begin{fyllnadsval} %"Inga fyllnadsval." fylls i automatiskt
%\fval{Namn}{Post}
%\entl{Namn}{Post}
\end{fyllnadsval}

\p{10}{Rapporter}{}
\begin{paragrafer}
\subp{A}{Check in}{\info}
Erik mår bra. Han har köpt frack till Vårbalen på KTH. Har jobbat med handlingar och propositioner.

Johan mår bra, har planerat in nästa pluggkväll och håller på med CEQ:er och sånt.

Dalia mår jättebra. Hon har köpt två klänningar som hon ska ha på KTH:s vårbal. Håller på att planera Utedischot.

Molly mår jättebra. NollU håller på att filma temasläppsfilmen och att förbereda inför temasläppet. Har skrivit färdigt nollekontraktet.

Stephanie mår bra, Caféet mår också bra.

Martin mår bra. Håller på med FpT.

Anders har mycket att göra, annars är det bra.

Malin mår också bra. Ska ha phaddergille på lördag och nästa vecka gille med K-sektionen.

Johannes vecka har varit bra. Var nära på att missa styrelsemötet men Hanna räddade honom. Just nu håller ENU ``Lunch med en ingenjör''.

Fredrik är det bra med. Han jobbar just nu mycket med VT-mötet.

\subp{B}{Kåren informerar}{\info}
Posten har valt en massa poster i helgen. Några poster är fortfarande vakanta.

Temat för årets nollning är En Rebellisk Nollning!

\subp{C}{Ekonomi}{\info}
Anders informerar att ekonomin rullar på, inga konstigheter. Vi har haft många inbetalningar sedan förra mötet.

\end{paragrafer}

\p{11}{Sektionskalender}{\dis}
Fredrik vill ha en gemensam kalender för hela Sektionens verksamhet. Erik kommer fixa en sådan.

Martin vill att Styrelsekalendern är lila istället, vilket Erik vägrade.

\p{11.5}{Motioner/Propositioner}{}
Erik sa att Macapärerna har lagt fram en motion om inköp av en backuplösning.

Styrelsen var positiv till motionen.

Anders föreslog att det ska läggas på beslutsuppföljningen, vilket mötet höll med om. Erik kommer fråga Macapärerna om de vill ändra sin motion, annars kommer styrelsen att yrka på detta. I övrigt vill styrelsen bifalla motionen i sin helhet.

Erik berättade om sin proposition om FpT-ansvarig som endast handlar om att rätta till deras mandatperiod.

Styrelsen var positiv till propositionen.

\p{11.75}{Subventionering av medaljer}{}
Fredrik vill subventionera årets medaljer till Sektionen. Om vi räknar högt rör det sig om ungefär 2000kr, som han vill ta från dispositionsfonden.

Fredrik \ypa avsätta 2000kr från dispositionsfonden till subventionering av medaljer.

\textbf{\Mba bifalla yrkandet.}

\p{12}{Internationalisering}{\dis}
Kåren har tittat på hur man kan jobba med internationalisering och har föreslagit en policy som styrelsen ska ge sina tankar på.

Mötet diskuterade hur man kan jobba med internationalisering på Sektionen.

Mötet kom fram till att vi borde skriva information på engelska där det kan beröra utbytesstudenter, och diskutera mer om detta vid ett senare möte.

Fredrik berättade sina tankar om policyn, vilka står i handlingara, och
mötet diskuterade.

Fredrik skriver ihop ett remissvar, som han kommer presentera nästa vecka.

\p{13}{Nästa styrelsemöte}{\bes}
\Mba nästa styrelsemöte ska äga rum 2016-04-14 12:10 i E:1426.

\p{14}{Beslutsuppföljning}{\bes}
\Ibfu

\p{15}{Övrigt}{\dis}
Dalia frågade om en Vice-utskottsordförande kan gå på styrelsemöte istället för utskottsordföranden. Anders sa att det går bra, men de har ingen rösträtt.

Molly ser gärna att styrelsen kommer på phadderutbildningen på lördag.

Fredrik påminde om Vårterminsmötemötet på söndag.

\p{16}{OFSA}{\bes}
{\mo} förklarade mötet avslutat 12:59.

\end{paragrafer}

%\newpage
\hidesignfoot
\begin{signatures}{3}
\signature{\mo}{Mötesordförande}
\signature{\ms}{Mötessekreterare}
\signature{\ji}{Justerare}
\end{signatures}
\end{document}
