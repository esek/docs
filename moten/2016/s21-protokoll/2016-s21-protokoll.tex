\documentclass[10pt]{article}
\usepackage[utf8]{inputenc}
\usepackage[swedish]{babel}

\def\mo{Fredrik Peterson}
\def\ms{Erik Månsson}
\def\ji{Stephanie Mirsky}
%\def\jii{}

\def\doctype{Protokoll} %ex. Kallelse, Handlingar, Protkoll
\def\mname{styrelsemöte} %ex. styrelsemöte, Vårterminsmöte
\def\mnum{S21/16} %ex S02/16, E1/15, VT/13
\def\date{2016-10-13} %YYYY-MM-DD
\def\docauthor{\ms}

\usepackage{../e-mote}
\usepackage{../../../e-sek}

\begin{document}
\showsignfoot

\heading{{\doctype} för {\mname} {\mnum}}

%\naun{}{} %närvarane under
%\nati{} %närvarande till och med
%\nafr{} %närvarande från och med
\section*{Närvarande}
\subsection*{Styrelsen}
\begin{narvarolista}
\nv{Ordförande}{Fredrik Peterson}{E14}{}
\nv{Kontaktor}{Erik Månsson}{E14}{}
\nv{Förvaltningschef}{Anders Nilsson}{E13}{}
\nv{Cafémästare}{Stephanie Mirsky}{E13}{}
\nv{Øverphøs}{Molly Liljebjörn Rusk}{BME14}{}
\nv{Sexmästare}{Martin Gemborn Nilsson}{E14}{}
\nv{Krögare}{Malin Lindström}{BME14}{}
\end{narvarolista}

\subsection*{Ständigt adjungerande}
\begin{narvarolista}
\nv{Kårordförande}{Linus Hammarlund}{}{}
\end{narvarolista}

\section*{Protokoll}
\begin{paragrafer}
\p{1}{OFSÖ}{\bes}
Ordförande {\mo} förklarade mötet öppnat 12:16.

\p{2}{Val av mötesordförande}{\bes}
\valavmo

\p{3}{Val av mötessekreterare}{\bes}
\valavms

\p{4}{Tid och sätt}{\bes}
\tosg

\p{5}{Adjungeringar}{\bes}
\ingaadj

\p{6}{Val av justeringsperson}{\bes}
\valavj

\p{7}{Föredragningslistan}{\bes}
Föredragningslistan godkändes.

\p{8}{Föregående mötesprotokoll}{\bes}
\latillprot{S19/16 och S20/16}

\p{9}{Fyllnadsval/Entledigande av funktionärer}{\bes}
\begin{fyllnadsval} %"Inga fyllnadsval." fylls i automatiskt
    \entl{Tom Johansson}{Chefredaktör}
    \entl{Olle Osvald}{Redaktör}
    \fval{Olle Osvald}{Chefredaktör}
    \fval{Henrik Ramström}{Valberedningsledamot}
\end{fyllnadsval}

\p{10}{Rapporter}{}
\begin{paragrafer}
\subp{A}{Check in}{\info}
Erik mår bra, han börjar komma ikapp i tentaplugget. Han har pratat med HeHE-redaktionen om att få igång utskicken igen.

Martin mår bra. Han ska ha möte med Sexet för att planera upp nästa läsperiod.

Malin ska med KM hålla gille på fredag och vid ett senare tillfälle hålla i en ölprovning.

Stephanie mår bra. LED-café har fått svar från Miljöinspektionen efter den senaste inspektionen och det såg relativt bra ut.

NollU går det bra för. Molly har fixat alla utlägg. De har börjat prata med kandidater till nästa års Phøs.

Anders har fortsatt bokföra. Halvårsbokslutet är klart. All försäljning från under nollningen är bokförd.

Med Fredrik är det fint. Han jobbar på med funktionärstacket vilket han skrapat fram lite mer pengar till, typ 12000kr. Han har kollat på om vi kan spela Laserdome i Malmö.

\subp{B}{Kåren informerar}{\info}
Linus informerade om att nya Rulle (VI) nu är bokningsbar. I hela nästa vecka kör kåren tentafrukost. Alla fakturor från Kåren gällande nollningsaktviteter skickas ut snart.

\subp{C}{Ekonomi}{\info}
Anders informerade om ekonomin, den mår bra.

\end{paragrafer}

\p{11}{Riktlinje för kassadifferenser}{\bes}
Anders presenterade ett nytt förslag till riktlinje som handlar om kassadifferenser (se handlingarna).

\textbf{\Mba anta den föreslagna riktlinjen \emph{Hantering av kassadifferanser}.}

\p{12}{Pengar till utskottsavslutning}{}
Malin frågade om vi kan besluta om hur mycket pengar som kan användas för utskottsavslutningen, eftersom hon tänkt hålla i den för KM på lördag.

Fredrik ville gärna vänta med detta till efter tentaveckan så vi vet hur mycket funktionärstacket kommer kosta, men sa också att eftersom KM tänkte hålla i det på lördag får vi lösa det nu ändå.

Fredrik kollade upp hur mycket pengar som finns kvar under budgetposten, och kom fram till att det rör sig om ungefär 30kr per person.

Malin frågade om lite av kostnaden kan läggas på utskottets budget, vilket Anders avrådde från.

Molly frågade om man fick använda pengarna för kickoffen i våras ifall man inte hade någon kickoff, varpå Fredrik svarade att det inte är möjligt.

\p{13}{Äskning av pengar för backad examensbankett}{\bes}
Fredrik presenterade My Reimers äskning för backad examensbankett.

Mötet diskuterade.

\textbf{\Mba bevilja äskandet.}

Styrelsen resonerade att medan examensbanketten inte arrangerades strikt inom ramarna för den ordinarie verksamheten bedömdes ändå evenemanget bidra med stort värde för sektionsmedlemmar. På grund av detta beslutade styrelsen att bevilja äskandet men var också eniga om att man i framtiden bör vara tydligare med vad som gäller för evenemang, likt detta, som befinner sig i gränslandet mellan sektions- och privat verksamhet.

\p{14}{Nästa styrelsemöte}{\bes}
\Mba nästa styrelsemöte ska äga rum 2016-11-03 12:10 i E:1426.

\p{15}{Beslutsuppföljning}{\bes}
Fredrik \ypa skjuta upp beslutsuppföljningen av \emph{Inköp av dammsugare} till nästa styrelsemöte eftersom den inte kommit än.

\textbf{\Mba bifalla yrkandet.}

\p{16}{Övrigt}{\dis}
Molly påminde alla att fylla i Doodlen till TDS.

\p{17}{OFSA}{\bes}
{\mo} förklarade mötet avslutat 12:50.

\end{paragrafer}

%\newpage
\hidesignfoot
\begin{signatures}{3}
\signature{\mo}{Mötesordförande}
\signature{\ms}{Mötessekreterare}
\signature{\ji}{Justerare}
\end{signatures}
\end{document}
