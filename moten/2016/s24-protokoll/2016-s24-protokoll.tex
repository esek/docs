\documentclass[10pt]{article}
\usepackage[utf8]{inputenc}
\usepackage[swedish]{babel}

\def\mo{Fredrik Peterson}
\def\ms{Erik Månsson}
\def\ji{Dalia Khairallah}
%\def\jii{}

\def\doctype{Protokoll} %ex. Kallelse, Handlingar, Protkoll
\def\mname{styrelsemöte} %ex. styrelsemöte, Vårterminsmöte
\def\mnum{S24/16} %ex S02/16, E1/15, VT/13
\def\date{2016-11-24} %YYYY-MM-DD
\def\docauthor{\ms}

\usepackage{../e-mote}
\usepackage{../../../e-sek}

\begin{document}
\showsignfoot

\heading{{\doctype} för {\mname} {\mnum}}

%\naun{}{} %närvarane under
%\nati{} %närvarande till och med
%\nafr{} %närvarande från och med
\section*{Närvarande}
\subsection*{Styrelsen}
\begin{narvarolista}
\nv{Ordförande}{Fredrik Peterson}{E14}{}
\nv{Kontaktor}{Erik Månsson}{E14}{}
\nv{Förvaltningschef}{Anders Nilsson}{E13}{}
\nv{Cafémästare}{Stephanie Mirsky}{E13}{}
\nv{Øverphøs}{Molly Lilljebjörn Rusk}{BME14}{}
\nv{SRE-ordförande}{Johan Persson}{E13}{}
\nv{ENU-ordförande}{Johannes Koch}{E13}{}
\nv{Sexmästare}{Martin Gemborn Nilsson}{E14}{}
\nv{Krögare}{Malin Lindström}{BME14}{}
\nv{Entertainer}{Dalia Khairallah}{E15}{}
\end{narvarolista}

\subsection*{Ständigt adjungerande}
\begin{narvarolista}
    \nv{Kårrepresentant}{Jacob Karlsson}{}{}
\end{narvarolista}

\section*{Protokoll}
\begin{paragrafer}
\p{1}{OFSÖ}{\bes}
Ordförande {\mo} förklarade mötet öppnat 12:15.

\p{2}{Val av mötesordförande}{\bes}
{\valavmo}

\p{3}{Val av mötessekreterare}{\bes}
{\valavms}

\p{4}{Tid och sätt}{\bes}
{\tosg}

\p{5}{Adjungeringar}{\bes}
{\ingaadj}

\p{6}{Val av justeringsperson}{\bes}
{\valavj}

\p{7}{Föredragningslistan}{\bes}
Föredragningslistan godkändes.

\p{8}{Föregående mötesprotokoll}{\bes}
\ingaprot

\p{9}{Fyllnadsval/Entledigande av funktionärer}{\bes}
\begin{fyllnadsval} %"Inga fyllnadsval." fylls i automatiskt
    \fval{Ellen Nilsson}{Årskurs BME-2 ansvarig}
    \fval{Linnea Wenäll}{Årskurs BME-1 ansvarig}
\end{fyllnadsval}

\p{10}{Rapporter}{}
\begin{paragrafer}
\subp{A}{Check in}{\info}
Anders mår bra. Han har inte hunnit bokföra så mycket sedan HT/16, men han har hunnit titta på den mesta av bokföringen som fattades från nollningen.

Martin är inte trött, men har en dålig känsla. E6 har sålt många biljetter till brunchen!

Erik är också trött, han har fullt upp med mötena.

Johan sa att SRE har släppt vilka som vann CEQ-utlottningen.

Johannes sa att det är bra med honom, ENU gör inte så mycket just nu.

Molly är trött.

Malin är också trött. KM har fått in över 100 anmälningar till julgillet! Hon sa att snart är det dags för mer jul.

Dalia är också trött, men är taggad. Hon ska ha lunchmöte imorgon med NöjU för att planera ett sista event för året.

Fredrik är också trött, lite förkyld, annars är det rätt bra. Har jobbat med överlämning och testamente.

\subp{B}{Kåren informerar}{\info}
Jacob uppmuntrade alla att rösta i FM-valet. Just nu leder W i det relativa deltagarantalet med ungefär 38\%.

Anders frågade hur stor andel av TLTH:s medlemmar som röstat.

Jacob svarade att just nu är det 18.6\%

\subp{C}{Ekonomi}{\info}
Anders informerade om ekonomin.

Fredrik sa att Anders vill att alla ska lämna in sina kvitton så snabbt som möjligt.

\end{paragrafer}

\p{11}{Inköp av kokplatta}{\bes}
Malin vill köpa in en ny kokplatta till köket, särskilt med tanke på det kommande julgillet som KM håller i.

Martin vill också köpa in en bra kokplatta.

Mötet diskuterade och kom fram till att köpa en rejäl kokplatta som håller länge.

Fredrik frågade om man vill ha en induktionsplatta eller vanlig platta.

Mötet kom efter diskussion fram till att köpa in en enkel induktionsplatta.

Beslutet senarelades. Malin och Martin tar fram några förslag till nästa möte.

\p{12}{Förlängda öppettider i LED}{\dis}
Fredrik pratade om att caféet vill testa att ha öppet till 16:00 resten av läsperioden.

Johannes menade på att det är rimligare att det är öppet till 15:30, så man bara täcker föreläsningspausen.

Dalia var skeptisk till att förlänga arbetstiderna när vi har svårt att få ihop dioder redan.

Anders sa att många i Sektionen har önskat att caféet har öppet en halvtimme längre, och att det skulle vara värt att testa.

Mötet fortsatte diskutera.

Mötet var generellt positivt till att testa att förlänga arbetstiderna.

\p{13}{Inspektorklädsel}{\dis}
Inspektor Monica har önskat om att få ha en Sektionell högtidsklädsel, gärna redan till Sångarstriden.

Fredrik sa att det förmodligen är svårt att fixa till Sångarstriden, men är positiv till att fixa till nästa år.

Johannes undrade om plagget ska vara personligt eller gå vidare till nästa Inspektor.

Mötet kom fram till att det borde gå vidare till nästa Inspektor.

Fredrik tittar på förslag till framtida möte.

\p{14}{Överlämningsmiddag}{\dis}
Fredrik vill gärna ha en middag med den kommande styrelsen, och föreslog någon dag i sista veckan - den 12-15:e december.

Mötet var positivt till detta.

Fredrik gör en doodle när nya styrelsen är vald.

\p{15}{Nästa styrelsemöte}{\bes}
{\Mba} nästa styrelsemöte ska äga rum 2016-12-01 12:10 i E:1426.

\p{16}{Beslutsuppföljning}{\bes}

Malin \ypa skjuta upp beslutsuppföljningen av \emph{Inköp av stekbord} till S26.

\textbf{\Mba bifalla yrkandet.}

Fredrik sa att vi äntligen har fått vår dammsugare och att den fungerar bra. Kostnaden uppgick till 1253kr.

Fredrik \ypa stryka beslutsuppföljningen av \emph{Inköp av dammsugare}.

\textbf{\Mba bifalla yrkandet.}

Fredrik \ypa skjuta upp beslutsuppföljningen av \emph{Inköp av tevattenkran} till S25.

\textbf{\Mba bifalla yrkandet.}

Fredrik tycker inte att vi borde köpa in något maskotdräkt eftersom han, efter lite forskning, inte tycker att de ser så bra ut som man kanske hade velat.

Fredrik \ypa stryka beslutsuppföljningen av \emph{Inköp av maskotdräkt}.

\textbf{\Mba bifalla yrkandet.}

\p{17}{Övrigt}{\dis}
Mötet pratade om maten till Sektionsmötena.

\p{18}{OFSA}{\bes}
{\mo} förklarade mötet avslutat 12:49.

\end{paragrafer}

%\newpage
\hidesignfoot
\begin{signatures}{3}
\signature{\mo}{Mötesordförande}
\signature{\ms}{Mötessekreterare}
\signature{\ji}{Justerare}
\end{signatures}
\end{document}
