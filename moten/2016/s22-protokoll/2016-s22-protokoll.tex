\documentclass[10pt]{article}
\usepackage[utf8]{inputenc}
\usepackage[swedish]{babel}

\def\mo{Fredrik Peterson}
\def\ms{Erik Månsson}
\def\ji{Malin Lindström}

\def\doctype{Protokoll} %ex. Kallelse, Handlingar, Protkoll
\def\mname{styrelsemöte} %ex. styrelsemöte, Vårterminsmöte
\def\mnum{S22/16} %ex S02/16, E1/15, VT/13
\def\date{2016-11-03} %YYYY-MM-DD
\def\docauthor{\ms}

\usepackage{../e-mote}
\usepackage{../../../e-sek}

\begin{document}
\showsignfoot%

\heading{{\doctype} för {\mname} {\mnum}}

%\naun{}{} %närvarane under
%\nati{} %närvarande till och med
%\nafr{} %närvarande från och med
\section*{Närvarande}
\subsection*{Styrelsen}
\begin{narvarolista}
\nv{Ordförande}{Fredrik Peterson}{E14}{}
\nv{Kontaktor}{Erik Månsson}{E14}{}
\nv{Förvaltningschef}{Anders Nilsson}{E13}{}
\nv{Cafémästare}{Stephanie Mirsky}{E13}{}
\nv{Øverphøs}{Molly Liljebjörn Rusk}{BME14}{}
\nv{SRE-ordförande}{Johan Persson}{E13}{}
\nv{ENU-ordförande}{Johannes Koch}{E13}{}
\nv{Sexmästare}{Martin Gemborn Nilsson}{E14}{}
\nv{Krögare}{Malin Lindström}{BME14}{}
\nv{Entertainer}{Dalia Khairallah}{E15}{}
\end{narvarolista}

\subsection*{Ständigt adjungerande}
\begin{narvarolista}
\nv{Skattmästare}{Sophia Grimmeiss Grahm}{}{}
\nv{Kårrepresentant}{Jacob Karlsson}{}{}
\nv{Kårordförande}{Linus Hammarlund}{}{\nafr{10B}}
\end{narvarolista}

\section*{Protokoll}
\begin{paragrafer}
\p{1}{OFSÖ}{\bes}
Ordförande {\mo} förklarade mötet öppnat 12:13.

\p{2}{Val av mötesordförande}{\bes}
\valavmo%

\p{3}{Val av mötessekreterare}{\bes}
\valavms%

\p{4}{Tid och sätt}{\bes}
\tosg%

\p{5}{Adjungeringar}{\bes}
\ingaadj%

\p{6}{Val av justeringsperson}{\bes}
\valavj%

\p{7}{Föredragningslistan}{\bes}
Stephanie \ypa att lägga till \S12.5 ``Tevattenkran''.

Föredragningslistan godkändes med yrkandet.

\p{8}{Föregående mötesprotokoll}{\bes}
\latillprot{S21/16}

\p{9}{Fyllnadsval/Entledigande av funktionärer}{\bes}
\begin{fyllnadsval} %"Inga fyllnadsval." fylls i automatiskt
\fval{Edvard Carlsson (ed8030ca-s)}{Årskurs E-1 ansvarig}
\fval{Daniel Charmi (elt14dch)}{Projektgrupp Teknikfokus}
\fval{Johan Sievert Lindeskog}{Diod}
\fval{Viktor Drakfelt}{Diod}
\fval{Joel Bill}{Diod}
\fval{Odinn Dånsjö}{Diod}
\fval{Filip Johansson}{Diod}
\fval{Johanna Wikström}{Diod}
\fval{Elin Johansson}{Diod}
\fval{Lena Ilievska}{Diod}
\fval{Seif Sharif}{Diod}
\fval{Jonathan Jakobsson}{Diod}
\end{fyllnadsval}

\p{10}{Rapporter}{}
\begin{paragrafer}
\subp{A}{Check in}{\info}
Erik har jobbat mycket med mötet och expot. Hans tentor gick bra!

Anders har haft fullt upp med sektionsarbete efter tentaveckan, särskilt med budget.

Johan är det bra med. SRE har inte gjort så mycket förutom CEQ-censurering. Påven har varit på besök.

Martin sa att E6 ska hålla en brunch snart och G.E.M.B.O.R.N.-sittningen den 16:e december. Han har inte lagat sin cykel.

Stephanie sa att det går bra i LED förutom att tevattenkranen är sönder. CM har fått in nya Dioder.

Malin mår bra! De ska ha ølprovning imorgon.

Johannes mår bra! ENU gör inte så jättemycket nu. Han håller på att fakturera lite nu.

Dalia sa att det inte hänt så mycket i NöjU heller. Hon håller på att fixa testamente.

Molly sa att det är både bra med henne och NollU. De ska ha två kvällsmöte nästa vecka om testamente.

Sophia är det bra med!

Fredrik är det också bra med! Han har mest jobbat med mötena på sistone.

\subp{B}{Kåren informerar}{\info}
Kåren letar jobbare till Arkadgasquen. I vinter kommer kåren få en ny kanslist.

\subp{C}{Ekonomi}{\info}
\emph{Punkten togs inte upp.}

\end{paragrafer}

\p{11}{Kvarvarande utskottsarrangemang}{\dis}
Erik sa att InfU inte har något kvar att arrangera.

Anders sa att FVU bokför mest.

Johan sa att SRE inte har något speciellt att göra. Han ville också tillägga att de har gjort en liten groda, nämligen att utlysa priser genom ett lotteri till de som fyllt i CEQ-enkäter. Speciellt riktar sig utlottningen till de nyantagna - synd bara att de inte har några CEQ-enkäter att fylla i.

Stephanie sa att CM kommer börja sälja lussekatter i LED vid första advent.

Malin sa att KM har fyra pubar kvar.

Johannes sa att Academic Work kommer hit på tisdag, och han tror att det kommer vara den sista aktiviteten ENU har för i år.

Dalia sa att NöjU kommer ha sporta med E varje söndag. Det blir förmodligen ingen bowlingturnering eftersom det skulle krockat för mycket med andra aktiviteter. Hon vill hålla skiphte för nya NöjU.

Molly sa att NollU mest kommer hålla på med utvärdering och testamente.

Fredrik sa att han mest kommer jobba med möterna som finns kvar. Han ska försöka få upp examenstavlorna på väggen i foajén innan han går av för året.

\p{12}{Funktionärstack}{\dis}
Fredrik summerade kort om hur funktionärstacket är planerat. Han föreslog att styrelsen ska ha något spex på kvällen.

Dalia tyckte att styrelsen skulle ha spex på phaddertacket också.

Johannes föreslog att styrelsen kör en utdelning av ``medaljer''.

\p{12.5}{Tevattenkran}{}
Stephanie sa att tevattenkranen är sönder och hon behöver beställa en ny.

Stephanie \ypa att köpa in en ny tevattenkran till en kostnad av maximalt 600kr vilken belastar dispositionsfonden samt att lägga detta på beslutsuppföljningen till S24 med sig själv som ansvarig.

\textbf{\Mba bifalla yrkandet.}

\p{13}{Nästa styrelsemöte}{\bes}
\Mba nästa styrelsemöte ska äga rum 2016-11-17 12:10 i E:1426.

\p{14}{Beslutsuppföljning}{\bes}
\textbf{\Mba skjuta upp \emph{Inköp av stekbord} till S24.}

\textbf{\Mba skjuta upp \emph{Inköp av dammsugare} till S24.}

\p{15}{Övrigt}{\dis}
Dalia sa att förra året gjorde Alexander och Emil en julkalender. Hon sa att Erik har gått med på att filma och klippa, och söker nu efter folk som vill vara med.

Mötet diskuterade huruvida HK ska användas som mötesrum eller inte, och kom fram till att alltid boka Arkivet före HK.

Johan frågade Erik om mackan som han åt imorse var god. Erik svarade att den var en av de bättre han ätit.

\p{16}{OFSA}{\bes}
{\mo} förklarade mötet avslutat 12:38.

\end{paragrafer}

\newpage
\hidesignfoot
\begin{signatures}{3}
\signature{\mo}{Mötesordförande}
\signature{\ms}{Mötessekreterare}
\signature{\ji}{Justerare}
\end{signatures}
\end{document}
