\documentclass[../_main/handlingar.tex]{subfiles}

\begin{document}
\verksplanuppf{HT 2016}

\subsubsection*{Styrelsen}
Styrelsen har jobbat med en del av de punkter som angetts i 2016s års verksamhetsplan. I budgeten har ett par förslag lagt för att förbättra funktionärernas tack eller rättigheter. Tillsammans med budgeten kommer även en ekonomisk vision för Sektionen att presenteras. Ytterligare delar av Edekvata renoverades under sommaren och även om en plan fortfarande saknas så var det ett steg på vägen och någon som gjorde det mer attraktivt att sitta och plugga i våra lokaler.  Samarbetet mellan utskottsordförandena tycker vi har fungerat bra och vi har ändrat lite i städrutinerna även om vi valt att inte lägga ut detta på någon annan mot betalning.

\subsubsection*{Informationsutskottet}
Informationsutskottet har sedan förra mötet lyckats bra med att jobba mot delmålen. Relationen med våra vänsektioner är bibehållen - KTH och Chalmers var och besökte oss under nolleqasquen, och bjöd i gengäld oss tillbaka på deras tillställningar. DDG har jobbat för att Sektionens datorsystem och hemsida hållts uppdaterad.

Alumniverksamheten ligger visserligen på is i år, men en proposition om att utvecka alumniverksamheten gick igenom på VT/16 vilket verkar ha gett resultat i form av sektionsmedlemmar som är intresserade av att jobba med alumniverksamheten.

Sektionens informationskanaler är något som ständigt borde kollas utvärderas och varje gång någon har velat nå ut till Sektionen har vi varit noggranna med att försöka nå ut till så många som möjligt på bästa möjliga sätt.

\subsubsection*{Källarmästeriet}
Källarmästeriet har fortsatt jobbat på sin marknadsföring med affischering i E-huset och de andra husen på LTH. Nu gör vi dessutom evenemang på facebook till alla gillen.

Vi har jobbat för att få en jämn arbetsfördelning i krögartrion och försökt involvera källarmästarna mer i arbetet. Både genom workshop där de har fått brainstorma och komma med idéer till gillen och genom att göra enkät och fråga hur de tycker att det fungerar.

Inom krögartrion har vi jobbat för att få fördelade arbetsuppgifter genom att dela upp olika huvudansvarsområden. Vi har också strukturerat om lite för att få en så lik arbetsbelastning som möjligt. Arbetet har gått mycket bra och vi har försökt dela upp arbetsuppgifterna utefter vem som anses mest lämpad för en viss uppgift.

Cölen har fått jobba mycket självständigt för att inte belasta krögartrion så mycket. Med kontinuerlig kontakt har detta samarbete fungerat mycket bra från båda håll.

Jag anser därför att källarmästeriet har jobbat för sina mål på verksamhetsplanen och lyckats göra framsteg.

\newpage
\subsubsection*{Nolleutskottet}
NollU har under året jobbat för att integrera de internationella studenterna i sektionen med mycket stor hjälp av de internationella phaddrarna som hållit i evenemang med de nya studenterna, bl.a. en intersektionell sittning med V.

NollU har även jobbat med en mångfald av aktiviteter under nollningen med mål att det ska finnas något för alla olika personligheter. Detta har framförallt varit mycket lyckat med både nya kårevenemang och nya uppdrag.

En del förbättringar har gjorts med utvärderingar från tidigare års nollor och phaddrar har gjorts, men många evenemang har ändrats utan detta som utgångspunkt.

En plan för renhållning av lokalerna som använts under nollningen har inte gjorts då vi inte känt något större behov av detta. Med undantag för skrubbning av campus efter första målningen har all städning fungerat felfritt.

\subsubsection*{Cafémästeriet}
Arbetet för att minska mängden svinn i LED är något som kontinuerligt fortsatt sedan våren. Detta genom att bland annat ha förbättrat hur det är organiserat i förvaringsutrymmena, då det på så vis enklare går att se huruvida något hinner att gå ut.

Då mojterna avskaffades inför verksamhetsår 2016 har dessa inte brukats alls.
Sortimentet utvecklas och utvärderas konstant. dels genom förslag som inte bara funktionärerna ger men även kunder, samt så deltog vi i en matmässa som vår leverantör (Martin \& Servera) anordnade.

Överlag har cafémästeriet fortfarande en bra fungerande verksamhet, även om priserna behövts att höjas på grund av ökade priser från leverantörerna.

\subsubsection*{Förvaltningsutskottet}
Sedan förra mötet har fvu jobbat tillsammans med styrelsen för att renovera delar av våra lokaler. Vi har även jobbat för att tydliggöra sektionens ekonomin planering för sektionen inför nästa verksamhetsår.

\subsubsection*{Studierådet}
För att öka synligheten har studierådet försökt vara mer aktiva under nollningen via workshop och pluggkvällar. Utskottet har också jobbat för att sprida vetskapen om den rättighetslista som tagits fram av universitetet och som gäller alla studenter. Studierådet har försökt få med fler äldre studenter i utskottet, och har lyckats till en viss del med detta.

För att öka svarsfrekvensen på CEQ-enkäterna har SRE utlyst en tävling där priser lottas ut till de som fyller i alla sina CEQ.

\subsubsection*{Sexmästeriet}
Evenemangen har under året spridits ut bra. Efter uppstarten i början av årets så hölls 3 (4 om Skiphtet inkluderas) sittningar under våren som sektionens medlemmar kunde gå på. Under nollningen var det tätt men event och nu i läsperiod 4 ska ytterligare ett par evenemang hållas. Överlag har bra ordning i Sexmästeriets förrår hållits då flera städningar gjorts. Under nollningen var det inte helt perfekt men det kan förklaras med att det var mycket folk i rörelse i förrådet samt att áll äppeljuice förvarades där. På för utskotten gemensamma arbetsytor har även där överlag bra ordning eftersträvas. De gånger ordningen varit under kritik har det åtgärdats.

\subsubsection*{Nöjesutskottet}
Vi har jobbat i utskottet för att förbättra postbeskrivningar men det är ännu inte helt perfekt. Jobbar fortfarande på detta.

Testamente har skrivits för UtEDishot så nästa år kommer det att vara enkelt och smidigt för de som tar över det ärofyllda uppdraget att arrangera festen.

\subsubsection*{Näringslivsutskottet}
Sektionens näringslivsutskott har under året anordnat en mängd aktiviteter som har varit mycket uppskattade. Under nollningsperioden hölls merparten av höstens aktiviteter med lunchföreläsningar med bland annat Ericsson och Actic. Samarbeten med befintliga företag har behållits samtidigt som både Ericsson och Axis har visat ökat intresse att synas för studenterna på E och BME. Prissättningar på aktiviteter som erbjuds i förhållande till andra sektioners prissättning har setts över och vägts samman med vad näringslivsutskottet har som syfte för sektionens medlemmar. Evenemangen under våren har varit jämnt utspridda men höstterminens aktiviteter koncentreras fortfarande under nollningsperioden, vilket kan vara något att tänka på till nästkommande år.

\newpage
\begin{signatures}{10}
    \mvh
    \signature{Fredrik Peterson}{Ordförande}
    \signature{Erik Månsson}{Kontaktor}
    \signature{Anders Nilsson}{Förvaltningschef}
    \signature{Stephanie Mirsky}{Cafémästare}
    \signature{Molly Rusk}{Øverphøs}
    \signature{Johan Persson}{SRE-ordförande}
    \signature{Johannes Koch}{ENU-ordförande}
    \signature{Martin Gemborn Nilsson}{Sexmästare}
    \signature{Malin Lindström}{Krögare}
    \signature{Dalia Khairallah}{Entertainer}
\end{signatures}

\end{document}
