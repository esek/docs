\documentclass[../_main/handlingar.tex]{subfiles}

\begin{document}
\utskottsrapport{Informationsutskottet}

Sedan VT/16 har InfUs verksamhet kommit igång ännu bättre. HeHE har fått en ny Chefredaktör som tagit upp arbetet med HeHE och skickar nu ut veckobladet varje vecka igen vilket har uppskattats av många sektionsmedlemmar. DDG har jobbat en del med hemsidan och fixat med bl.a. alkoholhanteringssystemet och biljardaccesssystemet. De har också fixat E-vote till flera systersektioner som alla är mycket nöjda med hur det fungerar. Under nollningen hjälpte ett gäng kodhackare och macapärerna till med att hålla datorstugan vilket gick bra. DDG:s engagemang har ökat och vi har till nästa år många som är taggade på att vara med!

Vår fotograf har varit på ett antal evenemang och fotograferat väldigt fina bilder åt oss. Nytt för i år är att vi har lagt ut bilderna på facebook vilket, om man ser på reaktionerna vi fått, har varit mycket populärt bland medlemmarna. Picasson har gjort ett fortsatt strålande jobb med att fixa affischer till alla utskotten och styrelsen, inte minst under nollningen. Tillsammans med NollU fick vi fram årets NollEguide som blev riktigt bra.

Tyvärr har vi saknat Lastgammal/Alumniansvarig under större delen av året och har därför inte haft någon alumniverksamhet. På den ljusa sidan verkar propositionen om utökandet av alumniverksamheten ha gett resultat och det finns i skrivande stund flera som är intresserade av att ta på sig uppdraget nästa år.

Jag själv har spenderat mycket av min tid på att hjälpa till här och där under nollningen. Jag har fortsatt försöka skriva utförliga protokoll och försökt få ut dem i tid, även fast det inte alltid blir som man tänkt. I skrivande stund håller jag (bokstavligt talat) på med att sätta ihop handlingarna till terminsmötet och valmötet, som med de nya mallarna blivit mycket smidigare. Under nollingen försökte jag få igång vår Instagram ordentligt som nu har ~200 följare. Inte supermycket, men jag hoppas på att den fortsätter växa nästa år. Utöver det har jag mest jobbat med att hålla alla andra informationskanaler uppdaterade.

Jag har också fixat inbjudningarna till Nolleqasquen, där stavning och korrekturläsning var extra viktigt eftersom de skickades till så många fina gäster. Jag såg också till att välkomna våra vänstyrelser från Chalmers och KTH hit, vilket var väldigt trevligt och bådar gott för framtida år tillsammans.

\begin{signatures}{1}
    \mvh
    \signature{\sekr}{Kontaktor}
\end{signatures}

\end{document}
