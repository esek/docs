\documentclass[../_main/handlingar.tex]{subfiles}

\begin{document}
\utskottsrapport{Nolleutskottet}

Sedan vårterminsmötet har NollU hållit i dels ytterligare en phadderutbildning, ett temasläpp samt nollningen. E-sektionens nollning har även i år fungerat väldigt bra, där vi valt att fokusera på att ha en större mångfald av aktiviteter för att det ska finnas något för alla. Alternativa aktiviteter för internationella studenter har i år funnits där det tidigare år inte haft något inplanerat.

Större satsningar på studier har även gjorts genom bl.a. ändrat koncept av pluggkvällar som gjorts tillsammans med studierådet, samt införande av pluggphaddrar vid dessa tillfällen. Även en positiv attityd till skolan från Sektionen till de nyantagna från framförallt phaddrar och nollningsaktiva har bidragit till en förbättrad syn på studier. Phøset har jobbat nära studievägledningen som deltagit framförallt i läsvecka 0, men som haft mycket inblandning i mottagandet även utanför nollningsaktivtieterna.

Under året har Nolleutskottet samarbetat mycket med Sektionens alla utskott som hjälpt till enormt mycket med nollningens alla aktiviteter.

Samarbete med andra sektioner har byggts upp under året och många av nollningsaktiviteterna har anordnats tillsammans med hela TLTH för att nollorna ska lära känna folk från andra sektioner. Förutom kårevenemangen har även evenemang som gillen, sittningar, bruncher och lekar hållits med andra sektioner under nollningen.

Vi har även infört ett nytt system för att få ut information till nollor: Nollekollen! Denna app har gjort det lättare att både innan och under nollningen nå ut med information till nollor om Sektionen och nollningsaktiviteter och vi hoppas att det kommer fortsätta att användas kommande år.

\begin{signatures}{1}
    \mvh
    \signature{Molly Liljebjörn Rusk}{Øverphøs}
\end{signatures}

\end{document}
