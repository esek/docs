\documentclass[../_main/handlingar.tex]{subfiles}

\begin{document}
\utskottsrapport{Sexmästeriet}

Sedan vårterminsmötet har Sexmästeriet gjort en hel del saker. Först en stor intersektionell sittning i gasque tillsammans med flertalet andra sexmästerier. Efter det en sittning i Edekvata, främst riktad mot sektionens egna medlemmar, och till sist en sittning med W på Lopthet innan sommaren.

Under sommaren gjordes sedan en hel del planering inför nollningen så att fokus under nollningen kunde läggas på själva genomförandet. Första veckan hölls en välkomstsittning i foajén, sjungboksinsångning med eftersläpp samt en intersektionell sittning med sexmästerierna från F-,E-,K- och ING-sektionen. Vidare fortsatte nollning med en Vett och Etikett sittning samt en sittning med A- respektive D-sektionen.

Nollningen avslutades sedan med, vad som på grund av stavfel blev, en Nolleqasque. Nolleqasquen planerades noggrant och genomfördes sedan med bravur av utskottets alla medlemmar.

Innan julen anländer kommer en brunch, 26/11, samt en sista sittning, 16/12, att hållas. Detta innan det är dags för de flesta att lämna Lund för denna gång samt för E6-16 att avsluta året och lämna över till nästa års Sexmästeri.

De ekonomiska målen ser ut att gå okej. All ekonomin från nollningen är inte helt klar men det ser ut som att det kommer landa på några tusen under de 20 000 kr som är budgeterade. Då ska det dock tilläggas att årets Sexmästeri har belastats med ca 4000 kr i lagerdifferenser från föregående år.

\begin{signatures}{1}
    \mvh
    \signature{Martin Gemborn Nilsson}{Sexmästare}
\end{signatures}

\end{document}
