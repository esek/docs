\documentclass[../_main/handlingar.tex]{subfiles}

\begin{document}
\utskottsrapport{Källarmästeriet}

Under nollningen anordnade Källarmästeriet många gillen. Några av gillena var i samarbete med andra utskott på Sektionen såsom NöjU och ENU men även med andra sektioner såsom maskin- och ING-sektionen. Framförallt samarbetet med ING-sektionen var väldigt lyckat och Edekvata var fyllt till max med folk. Förutom att hålla i gillen under nollningen så hjälpte vi till med försäljning av klägg på både UtE-Dischot och Regattan.

Under hösten har gillena flutit på som vanligt och hållits varje vecka. Dessutom har vi som ny aktivitet planerat att det ska vara ølprovning, något som vi ser fram emot och hoppas kan fortsätta hållas för kommande generationer.

Cølen arbetar kontinuerligt med att hålla alkohollagret snyggt och fyllt med ett bra varierande utbud. De sköter det jättebra och är väldigt självgående, deras arbete är mycket uppskattat. Dessutom har de tilldelats ansvaret för utbud och presentation på ølprovningen.

De ekonomiska målen ser ut att uppnås och stora förändringar i den ekonomiska redovisningen har skett i höst, till det bättre. Uppdateringar av alkoholhanteringssystemet har skett och underlättar arbetet för alla i utskottet.

Datumet till julgillet är nu spikat till 10 december. Vi i krögartrion arbetar nu för att göra en så bra överlämning som möjligt för nästa gäng som ska ta över.

\begin{signatures}{1}
    \mvh
    \signature{Malin Lindström}{Krögare}
\end{signatures}

\end{document}
