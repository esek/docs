\documentclass[../_main/handlingar.tex]{subfiles}

\begin{document}
\utskottsrapport{Näringslivsutskottet}

Under årets nollning anordnade näringslivsutskottet tre lunchföreläsningar. I år har vi, där det ekonomiskt har varit möjligt, caterat mat. Detta underlättar arbetet för ENU och uppskattas av deltagarna. Vidare har utskottet anordnat flera monterevent där företagen står i foajén och pratar med förbigående studenter.

Efter nollningen har det varit lugnare, med enbart några monterevent. Lugnet beror bland annat på att vissa aktiviteter som var planerade till hösten har blivit senarelagda till våren, och att vissa aktiviteter inte blev av. De förra mikrovågsugnarna i edekvata var slitna och började gå sönder och ENU ordnade med sponsring till de nya som är uppsatta i Edekvata nu.

Planeringen inför Teknikfokus är i full gång och i år har Sektionen två medlemmar i projektgruppen. Dagarna innan Teknikfokus behövs arbetare och då hoppas vi att E-sektionen bidrar med många och flitiga medlemmar.

Det råder en påtaglig konjunktur bland hårdvaruföretag, vilket har märkts under året då flera företag har tagit kontakt med ENU och hört sig för hur de kan synas bland studenterna. Samarbeten tar tid att bygga upp och sker inte över en dag. Därmed är det viktigt att till nästa år fortsätta bygga vidare på dessa.

\begin{signatures}{1}
    \mvh
    \signature{Johannes Koch}{ENU-ordförande}
\end{signatures}

\end{document}
