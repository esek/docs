\documentclass[../_main/handlingar.tex]{subfiles}

\begin{document}
\proposition{Uppdatering av alkoholpolicyn}

I E-sektionens budgetriktlinjer har det tidigare stått skrivit under ett antal konton att kostnaderna inte får läggas på alkohol. Under 2016 har vi i styrelsen flertalet gånger diskuterat detta och i vissa fall även känt oss begränsade av detta. Styrelsen anser att i de allra flesta fall har denna begränsning varit bra men vi tycker att det i vissa fall ska finnas möjlighet att utan att bryta mot riktlinjerna köpa in alkohol. Två exempel på fall när vi anser att detta hade varit lämpligt är när gåvor ges eller att Sektionen till exempel under ett funktionärstack har möjlighet att bjuda på en mindre mängd alkoholhaltig dryck till maten. För att tydligt beskriva vid vilka tillfällen det skulle vara lämpligt att bjuda och eller ge bort alkohol anser vi att Sektionens alkohol- och drogpolicy bör uppdateras.

Med anledning av detta yrkar styrelsen på

\begin{attsatser}
    \att sist i Policybeslutet \emph{Alkohol- och drogpolicy} lägga till\par
    \begin{itshape}
        \subsection*{4. Alkohol som tack eller gåva}
        Vid funktionärstack dit alla Sektionens funktionärer är inbjudna kan funktionärerna bjudas på en mindre mängd alkoholhaltig dryck till maten. Alkoholen ska absolut inte vara den stora begivenheten för tacket. Vid tillfällen när Sektionen vill förära en person eller förening med en gåva kan denna bestå av alkoholhaltig dryck. Varje tillfälle vare sig det gäller tack eller gåva ska i förväg godkännas av styrelsen.
    \end{itshape}
\end{attsatser}

\begin{signatures}{1}
    \ist
    \signature{\ordf}{Ordförande}
\end{signatures}

\end{document}
