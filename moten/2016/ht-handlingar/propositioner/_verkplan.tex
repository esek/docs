\documentclass[../_main/handlingar.tex]{subfiles}

\begin{document}
\proposition{Verksamhetsplansförslag för 2017}

Styrelsen yrkar på att

\begin{attsatser}
    \att antaga den bifogade verksamhetsplanen för 2017.
\end{attsatser}

\begin{signatures}{1}
    \ist
    \signature{\ordf}{Ordförande}
\end{signatures}

\subsection*{Verksamhetsplan 2017}

Syftet med verksamhetsplanen är att skapa en tydligare struktur och mer operativt långsiktigt arbete. Verksamhetsplanen innehåller övergripande mål för Sektionen och delmål för styrelsen och Sektionens utskott för nästkommande år. Verksamhetens mål skall redovisas på varje terminsmöte.

\emph{Budgeten och verksamhetsplanen ska komplettera varandra.}

\subsubsection*{E-sektionen}
\emph{E-sektionen ska verka för att:}
\begin{dashlist}
    \item Främja medlemmarnas studietid
    \item Alla medlemmar trivs
    \item Inkludera alla medlemmar
    \item Sektionens verksamhet präglas av demokrati och transparens
    \item Ha en ständigt framåtsträvande verksamhet
    \item Varje funktionärspost har ett ansvar och kan göra skillnad
    \item Hålla en god och ansvarsfull ekonomi
    \item Anpassa Sektionens verksamhet efter medlemmarnas behov
\end{dashlist}

\subsubsection*{Styrelsen}
Styrelsen ska arbeta för att göra medlemmarnas studietid så bra som möjligt. Mer specifikt innebär detta förutom att vara ansvariga för den dagliga verksamheten att arbeta med mer långsiktiga frågor som till exempel lokaler och ekonomi.

\emph{Delmål 2017:}
\begin{dashlist}
    \item Hjälpa CM att se över hur man ska jobba för högre lönsamhet och hålla noggrann koll på LEDs ekonomi
    \item Jobba för att involvera Sektionen och dess medlemmar mer i arbetet med Teknikfokus
    \item Se över och utvärdera Sektionens funktionärsposter
    \item Under året se över Sektionens kostnader och intäkter
    \item Panta mera \scalebox{0.5}{\recycle}
\end{dashlist}

\subsubsection*{Cafémästeriet}
Cafémästeriets verksamhet fungerar i dagens läge överlag väldigt bra. Utskottet bör jobba mot att caféet ska vara fortsatt konkurrenskraftigt och bibehålla studentvänliga priser.

\emph{Delmål 2017:}
\begin{dashlist}
    \item Jobba för att fortsätta att minska svinn både i LED och i Cafémästeriets förråd
    \item Utvärdera caféets sortimentet
    \item Fortsätta med att göra kvartalsbokslut för att hålla koll på ekonomin samt för att få en överblick över caféets kostnader
    \item Utvärdera möjligheterna att utöka caféets öppettider
    \item Utvärdera posterna och dess beskrivningar
\end{dashlist}

\newpage

\subsubsection*{Sexmästeriet}
Sektionens medlemmar ska under året ha möjlighet att gå på prisvärda sittningar och evenemang som håller hög kvalité. Detta bör vara Sexmästeriets huvudsakliga mål att arbeta mot.

\emph{Delmål 2017:}
\begin{dashlist}
    \item Fördela evenmangen så jämnt som möjligt under året
    \item Arbeta för att hålla fortsatt god ordning i Sexmästeriets förråd (Pump)
    \item Jobba för att minimera svinn i vinlagret
\end{dashlist}

\subsubsection*{Näringslivsutskottet}
Näringslivsutskottet knyter samman sektionens medlemmar med näringslivet. Utskottet bör anordna en god blandning aktiviteter som på olika sätt främjar sektionens medlemmars chanser inför arbetslivet. Utskottet har också i uppgift att dra in pengar till sektionen via olika evenemang eller sponsring.

\emph{Delmål 2017:}
\begin{dashlist}
    \item Bibehålla samarbetet med befintliga företag samt aktivt söka nya samarbeten med organisationer och företag som är relevanta för sektionens medlemmar
    \item Se över prissättning och syfte med aktiviteter för att nå en god blandning av vinstdrivande och icke-vinstdrivande aktiviteter
    \item Utvärdera om Sektionens hemsida som vänder sig till företag behöver uppdateras
\end{dashlist}

\subsubsection*{Förvaltningsutskottet}
Sektionens medlemmar ska ha tillgång till fräscha uppehållslokaler ämnade både för studier och studiesociala aktiviteter. I nuläget prioriteras ekonomin alltid högre än Sektionens lokaler. Målet ska vara att se till att underhåll av Edekvata görs löpande utan att ekonomin prioriteras ned.

\emph{Delmål 2017:}
\begin{dashlist}
    \item Jobba för att få Vice Förvaltningschefen och Hustomtarna att jobba för en löpande framtidsplanering och underhåll av lokalerna.
    \item Jobba för att effektivisera Sektionens bokföring för att göra den mindre tidskrävande och mer intuitiv.
    \item Jobba för att effektivt utbilda berörda funktionärer i Sektionens ekonomi och bokföring.
    \item Utvärdera utskottets nya struktur och dess poster.
\end{dashlist}

\newpage

\subsubsection*{Informationsutskottet}
Informationsutskottet är ett spritt uttskott som har funktionärer med många olika arbetsuppgifter. 2016 infördes en ny post - Vice Kontaktor - som infördes för att ha en person som jobbar med att knyta samman utskottet. Vidare har utskottet fått fler alumniansvariga, teknokraterna har kommit till, men Ekiperingsexperterna har fallit bort. Därför bör utskottet nästa år jobba mycket med att komma igång med sin nya sammansättning.

\emph{Delmål 2017:}
\begin{dashlist}
    \item Utvärdera utskottets nya struktur och dess poster
    \item Få igång Vice Kontaktorns samarbete med Kontaktorn och resten av utskottet
    \item Arbeta för fortsatt god informationsspridning genom kontinuerlig utvärdering av informationskanalerna
    \item Jobba för att hålla Sektionens datorsystem och hemsida uppdaterade
    \item Utvärdera skicket på Sektionens tekniska utrustning för att se om något behöver bytas ut i förebyggande syfte
\end{dashlist}

\subsubsection*{Källarmästeriet}
Källarmästeriet bör jobba för att gillena ska vara fortsatt attraktiva för teknologer. Utskottet bör också fortsätta marknadsföra gillena till övriga Teknologkåren.

\emph{Delmål 2017:}
\begin{dashlist}
    \item Försöka locka fler teknologer till att komma på gillena
    \item Uppmuntra samarbete såväl inom Sektionen som med andra sektioner
    \item Jobba för att fortsatt ha ett billigt utbud av mat och dryck
    \item Jobba för att få så lite svinn som möjligt av öl och cider i alkohollagret
\end{dashlist}

\subsubsection*{Nöjesutskottet}
Nöjesutskottet fungerar ganska bra. Dock är arbetsbelastningen fortfarande inte helt okej mellan alla medlemmar i utskottet. Vissa poster är alldeles för diffusa och behöver en viss struktur. Det hade även varit bra att försöka marknadsföra Nöjesutskottets event lite mer så att det ska locka fler människor.

\emph{Delmål 2017:}
\begin{dashlist}
    \item Ha en jämnare arbetsbelastning mellan utskottets medlemmar
    \item Förbättra postbeskrivningar så att alla i utskottet vet vad de ska göra
    \item Marknadsföra eventen bättre så att fler på Sektionen deltar i de aktiviteter som NöjU arrangerar
\end{dashlist}

\newpage

\subsubsection*{Nolleutskottet}
NollUs mål är att alla nyantagna studenter på E- och BME-programmet ska få ett så bra mottagande, och därmed en så bra start på sin studietid, som möjligt. Detta ska göras dels genom att arrangera en mängd olika studiesociala aktiviteter, där det helst ska finnas något för alla, men även genom att ge en så positiv attityd till studier som möjligt med hjälp av phaddrar och äldre studenter.

\emph{Delmål 2017:}
\begin{dashlist}
    \item Arbeta för att integrera internationella studenter ännu mer i Sektionen
    \item Arbeta för att det ska fortsätta finnas en mångfald av aktiviteter
    \item Arbeta för att få fram förslag och förbättringar kring nollningen från Sektionens medlemmar
    \item Arbeta för att förmedla en positiv attityd till studier, däribland att gå på övningarna
    \item Utvärdera arbetsbördan på utskottets funktionärer
    \item Arbeta för att informera mer om Sektionens utskott redan under nollningen
\end{dashlist}

\subsubsection*{Studierådet}
Studierådet ska verka för att vara ett synligt utskott. Arbetet skall göras mer tillgängligt för sektionens medlemmar för att visa förändringar som har genomförts. Utskottet ska verka för att medlemmarna är medvetna om deras möjligheter att påverka och förbättra sin utbildning.

\emph{Delmål 2017:}
\begin{dashlist}
    \item Arbeta för att synliggöra utskottets arbete och resultat till sektionens medlemmar
    \item Anordna pluggkvällar kontinuerligt under hela året samt utvärdera deras struktur och syfte
    \item Arbeta för att ha minst en representant från varje årskurs, inklusive årskurs fyra och fem
    \item Arbeta för att öka svarsfrekvensen på CEQ-enkäterna
\end{dashlist}

\newpage
\end{document}
