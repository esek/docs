\documentclass[../_main/handlingar.tex]{subfiles}

\begin{document}
\proposition{Uppdatering av utskottsbeskrivningar}

Utskottsbeskrivningarna i reglementet är på sina håll utdaterade och inte enhetliga. Styrelsen har gjort ett förslag till uppdatering och yrkar därför på

\begin{attsatser}
    \att ändra innehållet under \S9:2 Utskottsbeskrivningar i reglementet till:\par
        \subsubsection*{9:2:A Förvaltningsutskottet, FVU}
        Förvaltningsutskottet har till uppgift att arbeta och sköta Sektionens administrativa arbete, lokaler, inventarier och bevara Sektionens historia.

        Det åligger utskottet att
        \begin{tightdashlist}
        \item förvalta Sektionens ekonomi, bokföring och juridiska dokument.
        \item förvalta Sektionens lokaler, inventarier och fordon.
        \item förvalta Sektionens arkiv.
        \item ansvara för E-shops försäljning, lagerföring och inköp av Sektionens reklamprodukter såsom ouveraller, märken, pins, sångböcker, tröjor m.m..
        \end{tightdashlist}

        \subsubsection*{9:2:B Informationsutskottet, InfU}
        Informationsutskottet har till uppgift att ansvara för informationsspridningen på Sektionen, Sektionens tekniska utrustning och Sektionens alumniverksamhet.

        Det åligger utskottet att
        \begin{tightdashlist}
            \item se till att Sektionens hemsida fungerar bra och har uppdaterad information.
            \item se till att Sektionens tekniska utrustning fungerar.
            \item skriva och publicera nollEguiden och HeHE.
            \item under nollningen utbilda nyantagna studenter i LTH:s och Sektionens datorsystem.
            \item arrangera minst ett alumnievenemang under året.
        \end{tightdashlist}

        \subsubsection*{9:2:C Näringslivsutskottet, ENU}
        Näringslivsutskottet har till uppgift att vara kopplingen mellan näringslivet och Sektionens medlemmar.

        Det åligger utskottet att
        \begin{tightdashlist}
        \item anordna aktiviteter som främjar Sektionens medlemmar inför arbetslivet.
        \item vårda och utveckla samarbeten med företag.
        \item tillgodose Sektionens behov av sponsring.
        \item anordna en arbetsmarknadsmässa för Sektionens medlemmar.
        \item hålla ständig kontakt med övriga sektioners motsvarande utskott.
        \end{tightdashlist}

        \subsubsection*{9:2:D Nöjesutskottet, NöjU}
        Nöjesutskottet har till uppgift att arrangera nöjes-, fritids- och idrottsaktiviteter för att tillgodose medlemmarnas behov. Utskottet ska se till att det finns ett varierat utbud av aktiviteter vilket kan innebära allt från till exempel spel- och filmkvällar till glassförsäljning och idrottsturneringar.

        Det åligger utskottet att
        \begin{tightdashlist}
        \item arrangera UtEDischot tillsammans med D-sektionen.
        \item organisera Sektionens bidrag till Sångarstriden.
        \item organisera Sektions deltagande i Tandemstafetten.
        \item arrangera idrottsaktiviteter för Sektionens medlemmar.
        \item arrangera andra roliga, nöjesarrangemang så att medlemmarnas behov tillgodoses.
        \end{tightdashlist}

        \subsubsection*{9:2:E Källarmästeriet, KM}
        Källarmästeriet har till uppgift att tillgodose Sektionen med gillen samt att sköta øl- och spritförrådet.

        Det åligger utskottet att
        \begin{tightdashlist}
        \item regelbundet arrangera gillen under läsperioderna.
        \item arrangera gillen under nollningsperioden.
        \item arrangera ett Julgille med julmat i december månad.
        \item sköta inköp samt lagerhållning av dryck avsett för utskottets verksamhet.
        \item löpande kontrollera vinstmarginaler på øl- och spritförrådets varor.
        \end{tightdashlist}

        \subsubsection*{9:2:F Cafémästeriet, CM}
        Cafémästeriet har till uppgift att tillgodose Sektionens medlemmar med möjligheten att handla enklare mat och dryck för studentvänliga priser, samtidigt som konkurrenskraft gentemot andra caféer bibehålls.

        Det åligger utskottet att
        \begin{tightdashlist}
        \item sköta driften av LED-café.
        \item ansvara för Sektionens inköp och beställningar till LED-café.
        \item hålla ordning och sköta inventering av inköpta varor.
        \item hantera den ekonomiska uppföljningen av samtlig försäljning i LED-café genom löpande kontroll av inköpspris samt redovisning för att upptäcka resultatförändringar.
        \end{tightdashlist}

        \subsubsection*{9:2:G Sexmästeriet, E6}
        Sexmästeriet har som huvudsaklig uppgift att tillgodose Sektionen med festarrangemang.

        Det åligger utskottet att
        \begin{tightdashlist}
        \item under nollningen arrangera sittningar med fokus på de nyantagna, däribland en Vett- \& Etikettsittning.
        \item arrangera en Nollegasque i form av en bal i anslutning till Nollningen.
        \item arrangera minst två evenemang under vårterminen och två under höstterminen som är öppna för alla Sektionens medlemmar.
        \item verka för att evenemang med andra sektioner hålls, för att främja intersektionella relationer.
        \end{tightdashlist}

        \subsubsection*{9:2:H Nolleutskottet, NollU}
        Nolleutskottet har till uppgift att arrangera mottagandet av nyantagna studenter.

        Det åligger utskottet att
        \begin{tightdashlist}
            \item rekrytera, utbilda och organisera phaddrar till nollningen.
            \item arrangera nollningsaktiviteter som får de nyantagna att känna sig välkomna till Sektionen.
            \item i samråd med studievägledningen och Studierådet arrangera nollningsaktiviteter som främjar studierna vid högskolan.
            \item inom utskottet utse en ekonomiansvarig.
        \end{tightdashlist}

        \subsubsection*{9:2:I Studierådet, SRE}
        Studierådet har till uppgift att utföra studiebevakning för Sektionens medlemmar. Detta innebär att genomföra kursutvärderingar och att i övrigt arbeta för en bättre studiesituation.

        Det åligger utskottet att
        \begin{tightdashlist}
        \item arrangera studiefrämjande aktiviteter.
        \item föra sektionsmedlemmarnas talan i programledningar och instutionsstyrelser.
        \item utföra CEQ-censurering samt deltaga på tillhörande möten.
        \end{tightdashlist}
\end{attsatser}

\begin{signatures}{1}
    \ist
    \signature{\ordf}{Ordförande}
\end{signatures}

\end{document}
