\documentclass[../_main/handlingar.tex]{subfiles}

\begin{document}
\proposition{Borttagandet av Jämlikhetspolicyn}
E-sektionen är en förening inom Teknologkåren vid Lunds Tekniska Högskola (TLTH). Detta innebär bland annat att en del av de styrdokument som gäller TLTH även behöver följas av oss som förening.
I dagsläget har vi ett utdaterat policybeslut gällande jämlikhet, och TLTH har nyligen antagit en uppdatering av deras policy om likabehandling som täcker vårt egna policybeslut. Detta gör alltså vår egen policy överflödig.
Med anledning av detta och då likabehandlingsombudens uppgifter är välbeskrivna i reglementet anser vi inte längre att detta policybeslut fyller något syfte. Därför tycker vi att den bör tas bort, och vi vill även genomföra några följduppdateringar i reglementet.

Med anledning av ovanstående yrkar styrelsen på

\begin{attsatser}
    \att ta bort policybeslutet \emph{Jämlikhetspolicy}
    \att under \S10:2:M i reglementet ändra postbeskrivningen för Likabehandlingsombud från\par
    {\it
        \begin{tightdashlist}
            \item ansvarar för att bevaka sektionens verksamhet från ett likabehandlingsperspektiv
            \item ska verka för en jämlik studiemiljö
            \item ska uppmärksamma styrelsen på situationer och miljöer som skulle kunna upplevas som kränkande av studenter vid sektionen eller bryta mot Sektionens likabehandlingspolicy
            \item ska hålla sektionen informerad om sektionens likabehandlingspolicy
            \item ska fungera som kontaktperson och hjälp för medlemmar som anser sig särbehandlade av personer kopplade till högskolan och programmet på grund av någon av diskrimineringsgrunderna listade i jämlikhetspolicyn
            \item ska även fungera som kontaktperson och hjälp för medlemmar som anser sig särbehandlade av personer kopplade till högskolan och programmet på andra grunder än diskrimineringsgrunderna
        \end{tightdashlist}
    }\par
    till\par
    {\it
    \begin{tightdashlist}
        \item ansvarar för att bevaka Sektionens verksamhet från ett likabehandlingsperspektiv
        \item ska verka för en jämlik studiemiljö
        \item ska uppmärksamma styrelsen på situationer och miljöer som skulle kunna upplevas som kränkande av studenter vid Sektionen
        \item ska hålla Sektionen informerad om TLTH:s policy för likabehandling
        \item ska fungera som kontaktperson och hjälp för medlemmar som anser sig särbehandlade av personer kopplade till högskolan eller programmet
    \end{tightdashlist}
    }
\end{attsatser}

\begin{signatures}{2}
    \ist
    \signature{\ordf}{Ordförande}
    \signature{\sekr}{Kontaktor}
\end{signatures}

\end{document}
