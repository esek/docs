\documentclass[../_main/handlingar.tex]{subfiles}

\begin{document}
\proposition{Införandet av arbetskläder för utlåning till funktionärer}

Hos de utskott som i sin huvudsakliga verksamhet representerar Sektionen utåt eller serverar mat och alkohol finns ett behov av en enhetlig klädsel. Även för de utskott som i huvudsak bara hanterar mat kan man anse att ett behov av någon form av arbetsklädsel finns.

I nuläget betalar utskottets medlemmar för dessa kläder själva. För att utskottets medlemmar, som ödmjukt ger av sin tid och sitt engagemang till Sektion, inte ska behöva betala dessa kläder bör möjlighet till utlåning av, för utskottet relevanta, kläder finnas. Att mer eller mindre tvinga funktionärer till att lägga ut pengar för att få ge av sitt engagemang är något som vi anser att Sektionen bör undvika så gott det går.

Vi anser därför att inköp av kläder till berörda utskott bör göras. Tanken är att det i budgeten för varje år ska avsättas en mindre summa pengar för att vid behov underhålla/köpa nya kläder. Förslaget hindrar inte att utskottets medlemmar själva betalar för sina kläder om de önskar att själva behålla och/eller hantera dessa på eget valt sätt.

Med anledning av ovanstående yrkar styrelsen på

\begin{attsatser}
    \att avsätta 6000kr från utrustningsfonden för att i varierande storlekar köpa in följande plagg:
    \begin{dashlist}
        \item 5-10 förkläden med tryck till Cafémästeriet
        \item 5-10 pikétröjor med tryck till Källarmästeriet
        \item 5-10 pikétröjor med tryck till Näringslivsutskottet
        \item 5-10 kavajer till Sexmästeriet (utan tryck)
    \end{dashlist}

    \att anta den bifogade föreslagna policyn \emph{Policy för utlåning och hantering av kläder}, samt
    \att detta läggs på beslutsuppföljningen till VT/17 där undertecknad står som ansvarig.
\end{attsatser}

\begin{signatures}{1}
    \ist
    \signature{Martin Gemborn Nilsson}{Sexmästare}
\end{signatures}

\newpage
\section*{Policybeslut: Policy för utlåning och hantering av arbetskläder}
Arbetskläder ska i största möjliga mån finnas tillgängliga för utlåning till medlemmar av de utskott som i sin huvudsakliga verksamhet representerar Sektionen utåt och/eller arrangerar evenemang där alkohol mat servera. Detta ses som en service till personer/organisationer/företag som på olika sätt betalar pengar till Sektion. Att kunna göra valet att inte lägga ut egna pengar på eller riskera att förstöra sina egna kläder, är något som den som ger av sitt engagemang till Sektionen bör ha möjlighet till.

\begin{dashlist}
    \item Följande gäller för kläder som av Sektionen lånas ut till funktionärer:
    \item Kläderna används bara under de tider som personen i fråga jobbar för det berörda utskottet.
    \item Personen i fråga, som lånat kläderna, står själv för tvätt av dessa.
    \item Eventuell kostnad för kläder som tappas bort eller förstörs i sammanhang som inte har med Sektionens verksamhet att göra står personen som lånat kläderna för.
    \item Utskottets ordförande ansvarar för hantering och utlåning av kläderna.
    \item Om tillräckligt med kläder finns och det i övrigt anses lämpligt är det ok att låna ut ett plagg över längre tid, till exempel en mandatperiod. Övriga punkter i policyn gäller fortsatt.
\end{dashlist}

\newpage
\end{document}
