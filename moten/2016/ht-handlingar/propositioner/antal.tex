\documentclass[../_main/handlingar.tex]{subfiles}

\begin{document}
\proposition{Tydliggörande av antalet funktionärer för en post}

I reglementet är det oklart vad som menas med att antalet för en post är t.ex. 2-5. För att göra det hela enklare vill vi ta bort minimumantalet och bara skriva ut maxantalet.

\begin{attsatser}
    \att för alla poster som beskrivs under \S10:2 i reglementet där antalet för posten skrivs som (s-t) ändra (s-t) till bara (t), d.v.s. att bara ha med maxantalet på posten, samt
    \att under \S10:2:B Förklaringar ändra\par
        \begin{itshape}
            \begin{itemize}
                \item[(n)] betyder extakt n stycken,
                \item[(s-t)] betyder mellan s stycken till t stycken, samt
            \end{itemize}
        \end{itshape}\par
        till\par
        \begin{itshape}
            \begin{itemize}
                \item[($n$)] $n \in \mathbb{N}_0$, betyder upp till $n$ stycken, samt
            \end{itemize}
        \end{itshape}
\end{attsatser}

\begin{signatures}{1}
    \ist
    \signature{\sekr}{Kontaktor}
\end{signatures}

\end{document}
