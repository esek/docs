\documentclass[../_main/handlingar.tex]{subfiles}

\begin{document}
\proposition{Uppdatering av övervakningspolicyn}

Material från Sektionens övervakningskameror ska hanteras på sådant sätt att den värnar om den personliga integriteten hos våra medlemmar. Därför ska inte materialet vara tillgängligt till fler än nödvändigt samt att materialet måste hanteras korrekt. Vi har uppmärksammat brister i den nuvarande policyn och hur Sektionen faktiskt hanterar materialet.

Därför yrkar styrelsen på

\begin{attsatser}
    \att uppdatera policybeslutet \emph{Övervakningspolicy} till det bifogade förslaget,
    \att övervakningssystemet tas ur bruk tills dess att det uppfyller kraven i policybeslutet, samt
    \att en beslutsuppföljning sker på VT/17 med sittande styrelsen som ansvarig.
\end{attsatser}

\begin{signatures}{3}
    \ist
    \signature{\ordf}{Ordförande}
    \signature{Anders Nilsson}{Förvaltningschef}
    \signature{\sekr}{Kontaktor}
\end{signatures}

\newpage
\section*{Policybeslut: Övervakningspolicy}
\emph{Denna policy finns till för att värna om den personliga integriteten av personer som vistas i Sektionens lokaler.}

Övervakningsmaterial får endast lagras och hanteras i syfte att utreda brott eller olyckor som inträffat i Sektionens lokaler, all annan användning är förbjuden. Materialet som lagrats skall kontinuerligt tas bort efter 14 dagar. Material får inte lagras längre än 14 dagar eller flyttas till andra maskiner eller medier, såvida inte materialet används i en pågående utredning. Ljudupptagning är förbjudet.

Övervakningsmaterial som lagras skall krypteras och dess dekrypteringsnycklar ska endast innehas av Sektionens firmatecknare och utskottsordföranden för informationsutskottet. Nycklarna skall bytas vid överlämning av nyckelinnehavarnas poster. Nyckelinnehavarna äger rätt att, då tillräckliga skäl föreligger, dekryptera och ta del av relevant material. Varje sådan handling skall redovisas på nästkommande styrelsemöte. Protokollet för det styrelsemötet ska innehålla följande:
\begin{dashlist}
    \item Vem som tagit del av övervakningsmaterialet och när den tagit del av materialet.
    \item Vilken/vilka tidsperioder materialet berör och i vilket syfte de används.
    \item Om kopior av materialet skapats och i vilket syfte.
\end{dashlist}

Vid underhåll av övervakningssystemet får berörda funktionärer på uppdrag av nyckelinnehavarna endast ta del av material den utsträckning det krävs för att kontrollera att kamerorna fungerar och är korrekt inställda.

\end{document}
