\documentclass[../_main/handlingar.tex]{subfiles}

\begin{document}
\beslutsuppfoljning{Uppfräschning av gamla arkivet}

På vårterminsmötet 2016 beslutades att Arkivet skulle renoveras. Renoveringen skulle framförallt behandla inköp av ny inredning som skulle passa till phösets verksamhet under nollningen och till mötesrum under resterande terminen . Arbetet skulle vara färdigt innan nollningen till en kostnad av maximalt 10 000 kr.

Innan terminensslut disskuterade jag och phöset vilka inventarier som skulle passa till rummet samt planer på ny belysning. Arbetet började med att jag och arkivarierna tömde ut resterande saker från arkivet och strax därefter sattes en whiteboard upp. De mesta av de nya inventarierna köptes in i samband med renoveringen av HK/BD och Diplomat och blev färdigställandet försenat pga. husets fönsterbyte. Men precis innan nollningen byggde phöset ihopa möblerna och någon vecka in fick vi fixat belysningen.

I efterhand kan konstateras att resultat varit mycket lyckad. Dock så saknas några komponenter i rummet pga. förseningarna. Det som skulle behöva fixas nu i efterhand är bla. nya stolar och teknik för att kunna optimera rummet för dess avsedda verksamhet.

Med anledning av ovanstående yrkar vi på

\begin{attsatser}
    \att skjuta upp beslutsuppföljningen av \emph{Uppfräschning av gamla arkivet} till VT/17.
\end{attsatser}

\begin{signatures}{1}
    \ist
    \signature{Anders Nilsson}{Förvaltningschef 2016}
\end{signatures}

\end{document}
