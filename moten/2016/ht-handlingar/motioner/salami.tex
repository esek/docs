\documentclass[../_main/handlingar.tex]{subfiles}

\begin{document}
\motion{Införandet av obligatorisk försäljning av salami för sektionens medlemmar}

Sektionen, och då särskilt sektionens café, har stagnerat från ett vinstperspektiv. För att fortsatt vara en attraktiv sektion för studenterna vid Elektroteknik- och Medicin\&Teknik-programmen behöver sektionen mer likvida medel och därmed göra drastiska ändringar i sin verksamhet. Sektionen besitter i dagsläget outnyttjad och gratis arbetskraft, vilket kan bidra till att sektionen lyfter sin vinst mot nya höjder. Salamiförsäljning är en väl beprövad metod för att dra in pengar bland Sveriges föreningar och E-sektionen bör inte vara sen att hänga på trenden. En salami säljs för 170 kr, och sektionen kommer tjäna upp till 65 kr per såld korv.* Ett räkneexempel: Sektionens medlemmar säljer tillsammans 100 000 salamikorvar, (och det är många korvar), vilket motsvarar en vinst på ca 40 000 miljarder kr (och det är mycket pengar).

Med detta i beaktelse yrkar motionären på
\begin{attsatser}
    \att sektionens medlemmar ska via dörrknackning sälja minst tio (10) st salamikorvar per läsperiod
    \att misslyckande att nå upp till lägstanivån tio (10) salami, ska berörd medlem tvingas köpa salami själv
    \att 10\% av eventuell vinst från försäljning ska placeras i en fond för framtida uppfödning av salami-grisar
\end{attsatser}

\begin{signatures}{1}
    Vänlig hälsning
    \signature{Tjorven ``Korven'' Svensson}{}
\end{signatures}

\end{document}
