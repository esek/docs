\documentclass[../_main/handlingar.tex]{subfiles}

\begin{document}
\motion{Styrelseresa till Bahamas}

Bahamas, formellt Samväldet Bahamas, är en stat bestående av en kedja av ett stort antal öar i Västindien mellan Florida och Kuba, varav 30 är bebodda.

Sedan självständigheten har Bahamas utvecklats genom turism och internationella banker och investmentföretag. Jordbruket är obetydligt och man måste importera livsmedel. Fisket ger bl.a. skaldjur. Genom liberala valuta- och skattegregler har Bahamas blivit ett skatteparadis, främst för nordamerikaner. Nassau är ett finanscentrum, särskilt för offshoreverksamhet.

Turistindustrin står för mer än 60\% av BNP och sysselsätter hälften av arbetskraften. Hur ekonomin ser ut framåt, är alltså mycket beroende på turismen, vilken i sin tur är beroende på ekonomin i USA, varifrån huvuddelen av turisterna kommer.

All elektricitet produceras av fossilt bränsle såsom råolja. Därför yrkar jag på

\begin{attsatser}
    \att tillsätta 50 000 svenska kronor för att skicka styrelsen till Bahamas.
    \att Styrelsen ska åka någon gång i december. Detta får mötet bestämma.
    \att kostnaden belastar utrustningsfonden.
\end{attsatser}

\begin{signatures}{1}
    \signature{Oddput Clementin}{Hedersstudernade}
\end{signatures}

\end{document}
