\documentclass[../_main/handlingar.tex]{subfiles}

\begin{document}
\section{Ekonomisk rapport}

E-sektionens ekonomi mår bra och Sektionen har en mycket god förmåga att betala alla sina kortfristiga skulder. Som sektion har vi i nuläget inga långfristiga skulder så som banklån eller liknande. Sektionens tillgångar uppgår till 822 345,86 kr enligt bokfört underlag t.o.m. 14/11 2016 då lager exkluderas. En stor del av dessa tillgångar är öronmärkta och ligger i Sektionens fiktiva fonder. De pengarna som ligger i fonderna är avsedda för olika ändamål så som inköp av ny utrustning, reparationer eller som ersättning vid skador med mera. Det bör tilläggas att det skulle vara mycket oklokt att spendera samtliga medel i fonderna eftersom det skulle lämna Sektionen ekonomiskt sårbar om vi skulle drabbas av en större oförutsedd utgift.

På nästa sida ser ni en balansrapport för Sektionen där ni kan se aktuella tillgångar och skulder. Dagsaktuella värden på våra bankkonton respektive handkassan samt övriga tillgångar redovisas på mötet.

Halvårsbokslutet som bifogats I handlingarna innefattar de transaktioner som har mellan transaktionsdatumen 20160101-20160630. Det är några saker som bör noteras. För det första har inga periodiseringar har gjorts vilket medför att:
\begin{dashlist}
\item Intäkten för Teknikfokus därmed inte är med i resultatet för första halvåret.
\item Kostnader för renoveringar på ca \SI{47000}{kr} är med i resultatet för första halvåret och inte flyttade till utrustningsfonden.
\end{dashlist}

Det har också gjort en tydlig revidering vid lager inventeringana för att se till att lagernas saldo ska ge en verklig bild av deras innehålls värde. Detta resulterade I att framförallt större negativa differanser under lager E-shop och lager Vin på ca \SI{16000}{kr} totalt.  Det bör också understrykas att Sektionens verksamhet skiljer sig markant mellan våren och hösten. Efter genomgång av halvårsbokslutet har styrelsen valt att inte revidera budgeten.

Sektionens resultat för första halvåret är \SI{-71929.39}{kr}.

\begin{signatures}{1}
    \mvh
    \signature{Anders Nilsson}{Förvaltningschef}
\end{signatures}

\end{document}
