\documentclass[10pt]{article}
\usepackage[utf8]{inputenc}
\usepackage[swedish]{babel}

\def\mo{Johan Westerlund}
\def\ms{Erik Månsson}
\def\ji{Elin Magnusson}
\def\jii{Pontus Landgren}

\def\doctype{Protokoll} %ex. Kallelse, Handlingar, Protkoll
\def\mname{Höstterminsmötet} %ex. styrelsemöte, Vårterminsmöte
\def\mnum{HT/16} %ex S02/16, E1/15, VT/13
\def\date{2016-11-22} %YYYY-MM-DD
\def\docauthor{\ms}

\usepackage{../e-mote}
\usepackage{../../../e-sek}

\begin{document}
\showsignfoot

\heading{{\doctype} för {\mname} {\mnum}}

%\naun{}{} %närvarane under
%\nati{} %närvarande till och med
%\nafr{} %närvarande från och med
\section*{Närvarande}
\subsection*{Styrelsen}
\begin{narvarolista}
    \nv{Ordförande}{Fredrik Peterson}{E14}{}
    \nv{Kontaktor}{Erik Månsson}{E14}{}
    \nv{Förvaltningschef}{Anders Nilsson}{E13}{\nafr{4}}
    \nv{Cafémästare}{Stephanie Mirsky}{E13}{Frånvarande under \S17A, \S17D-\S17E} %>17A 17A< >17D 17E<
    \nv{Øverphøs}{Molly Lilljebjörn Rusk}{BME14}{}
    \nv{SRE-ordförande}{Johan Persson}{E13}{}
    \nv{ENU-ordförande}{Johannes Koch}{E13}{}
    \nv{Sexmästare}{Martin Gemborn Nilsson}{E14}{}
    \nv{Krögare}{Malin Lindström}{BME14}{\nafr{4}}
    \nv{Entertainer}{Dalia Khairallah}{E15}{\nati{17}} %>17?
\end{narvarolista}

\subsection*{Medlemmar}
\begin{narvarolista}
    %\nv{Post}{Namn}{Klass}{}
    \nv{}{Daniel Johansson}{E14}{}
    \nv{}{Elin Magnusson}{BME12}{}
    \nv{}{Elin Branzell}{BME14}{}
    \nv{}{Viktor Persson}{E14}{}
    \nv{}{Christian Benson}{E14}{}
    \nv{}{Johan Karlberg}{E14}{}
    \nv{}{Pontus Landgren}{E14}{}
    \nv{}{Sophia Grimmeiss Grahm}{BME14}{}
    \nv{}{Rebecca Ritter}{E14}{\nati{17}} %>17
    \nv{}{Lina Samnegård}{BME16}{\nati{17E}} %>17E
    \nv{}{Elin Johansson}{BME16}{\nati{17E}} %>17E
    \nv{}{Linnea Wenäll}{BME16}{\nati{17E}} %>17E
    \nv{}{Fanny Månefjord}{BME16}{\nati{17E}} %>17E
    \nv{}{Isak Marklund}{E16}{}
    \nv{}{Markus Rahne}{BME14}{}
    \nv{}{David Uhler Brand}{E14}{}
    \nv{}{Louise Wedberg}{BME15}{}
    \nv{}{Linnea Sjödahl}{BME15}{}
    \nv{}{Björn Johansson}{E15}{}
    \nv{}{Madeleine Arkenius}{BME15}{}
    \nv{}{Axel Sondh}{E15}{}
    \nv{}{Sofia Rokkones}{BME15}{}
    \nv{}{Martin Alumets}{E15}{\nati{17E}} %>17E
    \nv{}{Jonatan Kronander}{E15}{\nati{17E}} %>17E
    \nv{}{Filip Kronström}{E15}{\nati{17E}} %>17E
    \nv{}{Magnus Lundh}{E15}{\nati{17E}} %>17E
    \nv{}{Emil Harvig}{BME14}{\nati{17.5}} %>17.5
    \nv{}{Elisabeth Klint}{BME14}{Frånvarande under \S17} % >17 17<
    \nv{}{Amanda Wallin}{BME14}{}
    \nv{}{Ester Randahl}{BME14}{}
    \nv{}{Rasmus Sobel}{BME16}{}
    \nv{}{Sanna Nordberg}{BME16}{\nati{16}} %>16
    \nv{}{Albin Nyström Eklund}{BME16}{\nati{17}} %>17
    \nv{}{Cornelia Sjöberg}{BME14}{Frånvarande under \S1-\S5, \S13} %5< >13 13<
    \nv{}{Linnea Hellholm}{BME14}{Frånvarande under \S1-\S5, \S13, \S17E-\S17N} %5< >13 13< >17E 17N<
    \nv{}{Daniel Bakic}{E15}{\nafr{4}} %4<
    \nv{}{Niklas Karlsson}{E14}{\nafr{4}} %4<
    \nv{}{Niklas Gustafson}{E15}{\nafr{16}} %16<
    \nv{}{Oscar Uggla}{E15}{\nafr{16}} %16<
\end{narvarolista}

\subsection*{Ständigt adjungerande}
\begin{narvarolista}
    \nv{Talman}{Johan Westerlund}{E11}{}
\end{narvarolista}

\newpage
\section*{Protokoll}

\begin{paragrafer}
\p{1}{TaFMÖ}{}
Talman {\mo} förklarade mötet öppnat 17:22.

\p{2}{Val av mötesordförande}{}
Talman {\mo} valdes.

\p{3}{Val av mötessekreterare}{}
Kontaktor {\ms} valdes.

\p{4}{Godkännande av tid och sätt}{}
Tid och sätt godkändes.

\p{5}{Val av två justeringspersoner}{}
\valavj

\p{6}{Adjungeringar}{}
\emph{\ingaadj}

\p{7}{Godkännande av dagordningen}{}
Viktor Persson \ypa lägga till \S17.5 ``Behandling av sen motion: Utökning av antal sökande till posten Co-phøsare''.

Fredrik Peterson tyckte inte att punkten skulle läggas till eftersom den är sen och medlemmarna inte fått en chans att läsa igenom och tänka ordentligt.

Molly Rusk sa att motionen från början var tänkt att läggas som en proposition, och att på grund av ett missförstånd i Phøset blev motionen sent lagd.

\textbf{\Mba bifalla yrkandet.}

\textbf{\Mba godkänna föredragningslistan.}

\p{8}{Föregående sektionsmötesprotokoll}{}
\textbf{\Mba lägga till protokollet för VT/16 till handlingarna.}

\p{9}{Meddelanden}{}
Malin Lindström uppmanade alla att gå på julgillet till 10:e december.

Martin Gemborn Nilsson sa att på lördag håller E6 i en brunch, vilken tydligen ska vara lika bra som både bakishäng och förfest.

David Uhler Brand ville uppmana alla att gå på F1-röj.

Malin Lindström sa att innan brunchen och eftersläppet kan man gå på gille.

Martin Gemborn Nilsson sa att den 16:e december kommer E6 hålla i G.E.M.B.O.R.N.-sittningen.

\p{10}{Beslutsuppföljning}{}
Fredrik Peterson presenterade beslutsuppföljningen av \emph{Upprustning av E-husets yttre fasadskylt}.

\textbf{\Mba bifalla att-satserna i beslutsuppföljningen.}

Daniel Johansson presenterade beslutsuppföljningen av \emph{Inköp av backuplösning till Sektionens servrar}. Han pointerade att det är fel i handlingarna - NAS:en kostade \SI{3357.50}{kr}, inte \SI{357.50}{kr}.

\textbf{\Mba bifalla att-satserna i beslutsuppföljningen.}

Fredrik Peterson presenterade beslutsuppföljningen av \emph{Renovering och ombyggnad av HK och BD}.

\textbf{\Mba bifalla att-satsen i beslutsuppföljningen.}

Johannes Koch presenterade beslutsuppföljningen av \emph{Uppfräschning av Diplomat}.

\textbf{\Mba bifalla att-satsen i beslutsuppföljningen.}

Johan \ypa varje gång ett beslut tas ska det automatiskt tas med acklamation ifall ingen begär något annat.

\textbf{\Mba bifalla yrkandet.}

Anders Nilsson presenterade beslutsuppföljningen av \emph{Uppfräschning av gamla arkivet}.

\textbf{\Mba bifalla att-satsen i beslutsuppföljningen.}

Anders Nilsson berättade hur det går med \emph{Inköp av ny spisfläkt till Edekvataköket}. Han sa att p.g.a. krav från Akademiska Hus skulle kostnaden uppskattningsvis skulle bli mycket större än vad beslutet tillät.

Anders Nilsson \ypa stryka \emph{Inköp av ny spisfläkt till Edekvataköket} från beslutsuppföljningen.

\textbf{\Mba bifalla yrkandet.}

\p{11}{Utskottsrapporter}{}
\emph{Ingen hade några kommentarer på utskottsrapporterna.}

\p{12}{Uppföljning av verksamhetsplan}{}
\emph{Ingen hade några kommentarer på uppföljningen av verksamhetsplanen.}

\p{13}{Ekonomisk rapport}{}
Anders Nilsson gav en rapport för Sektionens ekonomi.

\p{14}{Uttag ur Sektionens fonder sedan förra terminsmötet}{}
Anders Nilsson berättade om uttagen ur Sektionens fonder sedan förra terminsmötet.

\p{15}{Resultatrapport från första halvan av verksamhetsåret}{}
Anders Nilsson presenterade resultatrapporten från första halvan av verksamhetsåret.

Fredrik Peterson \ypa ajournera mötet i 10 minuter.

\textbf{\Mba bifalla yrkandet.}

\emph{Mötet ajournerades 18:23 och återupptogs 18:33.}

\p{16}{Behandling av motioner}{}
    \begin{paragrafer}
        \subp{A}{Styrelseresa till Bahamas}{}
        Johan Westerlund presenterade motionen.

        Fredrik Peterson presenterade styrelsens svar på motionen.

        Albin Nyström Eklund yrkade på
        \begin{attsatser}
            \att styrelsen ska åka till Qvidinge istället, samt
            \att sänka budgeten till 1500kr.
        \end{attsatser}

        \textbf{\Mba bifalla motionen med Albins motyrkanden.}

        \subp{B}{Införandet av obligatorisk försäljning av salami för sektionens medlemmar}{}
        Johannes Koch presenterade motionen.

        \textbf{\Mba avslå motionen i sin helhet.}

        \subp{C}{Låt Nolleqasquen ha sitt rätta namn}{}
        Daniel Bakic presenterade motionen.

        Fredrik Peterson presenterade styrelsens svar på motionen.

        \textbf{\Mba avslå motionen i sin helhet.}

        \subp{D}{Låt Mongomästaren ha sitt rätta namn}{}
        Daniel Bakic presenterade motionen och uppmuntrade att avslå motionen.

        Fredrik Peterson presenterade styrelsens svar på motionen.

        Daniel Bakic fick frågan om hur fulla de var, varpå han svarade att de var ganska fulla.

        \textbf{\Mba avslå motionen i sin helhet.}

        \subp{E}{Diskhomästare}{}
        Oscar Uggla presenterade motionen.

        Fredrik Peterson presenterade styrelsens svar på motionen.

        \textbf{\Mba avslå motionen i sin helhet.}

        \subp{F}{Fanbärare behöver längre påle}{}
        Emil Harvig presenterade motionen.

        Fredrik Peterson presenterade styrelsens svar på motionen.

        \textbf{\Mba bifalla motionen i sin helhet.}

        Johan Persson \ypa ajournera mötet i 10 minuter.

        \textbf{\Mba bifalla yrkandet.}

        \emph{Mötet ajournerades 19:02 och återupptogs 19:13.}

    \end{paragrafer}
\subp{17}{Behandling av propositioner}{}
    \begin{paragrafer}
        \subp{A}{Budgetförslag för 2017}{}
        Anders Nilsson presenterade propositionen.

        Erik Månsson \ypa flytta budgetriktlinjen för Ljud och Ljus som ligger under FVU02 till INFU01 för att överrensstämma med budgetförslaget och reglementet.

        Styrelsen jämkade sig med Eriks yrkande.

        \textbf{\Mba bifalla propositionen i sin helhet med Eriks tilläggsyrkande.}

        \subp{B}{Verksamhetsplansförslag för 2017}{}
        Fredrik Peterson presenterade propositionen.

        \textbf{\Mba bifalla propositionen i sin helhet.}

        \subp{C}{Borttagandet av Jämlikhetspolicyn}{}
        Fredrik Peterson presenterade propositionen.

        \textbf{\Mba bifalla propositionen i sin helhet.}

        \subp{D}{Införandet av arbetskläder för utlåning till funktionärer}{}
        Martin Gemborn Nilsson presenterade propositionen.

        \textbf{\Mba bifalla propositionen i sin helhet.}

        \subp{E}{Uppdatering av alkoholpolicyn}{}
        Fredrik Peterson presenterade propositionen.

        \textbf{\Mba bifalla propositionen i sin helhet.}

        Pontus Landgren \ypa ajournera mötet i 5 minuter.

        \textbf{\Mba bifalla yrkandet.}

        \emph{Mötet ajournerades 19:57 och återupptogs 20:04.}

        \subp{F}{Uppdatering av medaljpolicyn}{}
        Fredrik Peterson presenterade propositionen. Han nämnde att styrelsen fick ta hjälp av wikin för att kunna ta fram vilken medalj som är vilken.

        David Uhler Brand ville pointera att om vi hade lagt ner wikin på VT/16 hade styrelsen inte kunnat lösa vilken medalj som är vilken.

        \textbf{\Mba bifalla propositionen i sin helhet.}

        \subp{G}{Uppdatering av övervakningspolicyn}{}
        Anders Nilsson presenterade propositionen.

        \textbf{\Mba bifalla propositionen i sin helhet.}

        \subp{H}{Kompletterande revidering av reglemente gällande SRE}{}
        Johan Persson presenterade propositionen.

        \textbf{\Mba bifalla propositionen i sin helhet.}

        \subp{I}{Uppdatering av utskottsbeskrivningar}{}
        Fredrik Peterson presenterade propositionen.

        \textbf{\Mba bifalla propositionen i sin helhet.}

        \subp{J}{Ändring av antalet funktionärer i FVU}{}
        Anders Nilsson presenterade propositionen.

        \textbf{\Mba bifalla propositionen i sin helhet.}

        \subp{K}{Ändring av postbeskrivningar och antal funktionärer i CM}{}
        Stephanie Mirsky presenterade propositionen.

        \textbf{\Mba bifalla propositionen i sin helhet.}

        \subp{L}{Tydliggörande av antalet funktionärer för en post}{}
        Erik Månsson presenterade propositionen.

        \textbf{\Mba bifalla propositionen i sin helhet.}

        \subp{M}{Inköp av ny huvudswitch}{}
        Erik Månsson presenterade propositionen.

        \textbf{\Mba bifalla propositionen i sin helhet.}

        \subp{N}{Inköp av nya datorer}{}
        Erik Månsson presenterade propositionen.

        \textbf{\Mba bifalla propositionen i sin helhet.}

    \end{paragrafer}
\p{17.5}{Behandling av sen motion: Utökning av antal sökande till posten Co-phøsare}{}
\emph{Motionen hittas i de sena handlingarna.}

Viktor Persson presenterade propositionen.

Erik Månsson begärde sluten votering.

Johan Westerlund yrkade på
\begin{attsatser}
    \att återremittera motionen, och
    \att dra streck i debatten.
\end{attsatser}

\textbf{\Mba bifalla yrkandet att dra streck i debatten.}

Niklas Gustafson \ypa stryka att-satsen om att öka antalet Co-phøsare och yrkade istället på att öka maxantalet Øvergudsphaddrar från 2 till 3.

Viktor Persson \ypa ändringen ska börja gälla nästa verksamhetsår.

Motionärerna jämkade sig med Viktors yrkande.

\textbf{\Mba återremittera motionen, och att ärendet läggs på beslutsuppföljningen till VT/17 med Phøset 2016 som ansvariga.}

\p{18}{Övrigt}{}
Malin Lindström sa att anmälan till julgillet öppnar imorgon.

Elin Magnusson påminde om att motkandidera till styrelseposter.

\p{19}{TaFMA}{}
Talman {\mo} förklarade mötet avslutat 21:55.

\end{paragrafer}

%\newpage
\hidesignfoot
\begin{signatures}{4}
\signature{\mo}{Mötesordförande}
\signature{\ms}{Mötessekreterare}
\signature{\ji}{Justerare}
\signature{\jii}{Justerare}
\end{signatures}
\end{document}
