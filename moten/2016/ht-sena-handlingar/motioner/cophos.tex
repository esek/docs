\documentclass[../_main/handlingar.tex]{subfiles}

\begin{document}
\motion{Utökning av antal sökande till posten Co-phøsare}

Nollningen är ett tillfälle då hela sektionen samarbetar för att sammanföra nya studenter med LTH i syfte av att bidra till en bättre studiemiljö samt för att bygga upp och behålla den starka gemenskap vi har. Detta är något som de flesta är medvetna om och något som ännu fler uppskattar. Men något som ofta förbises är att nollningen inte har blivit såhär på ett år utan att det ständigt pågår en utveckling, där nollningen förbättras för varje iteration. Men med denna utveckling har även arbetsbördan för de ansvariga ökat. Detta medför att phøs-poster, i synnerhet posten Øverphøs, har mycket hög arbetsbelastning under förberedelse av nollningen, till en sådan nivå att det är värt att nämna funktionärernas välmående.

Det är dock inte bara på E-sektionen som phøs har mycket arbetsbelastning. Detta är något som även har uttryckts på andra sektioner där antalet phøs har varit densamma- eller även fler än vad det är på E-sektionen. Det är även värt att tillägga att E-sektionen traditionellt sett har den längsta nollningen av sektionerna på LTH.

För att kunna fortsätta förbättra nollningen, minska arbetsbelastningen för phøsare, i syfte av en ännu bättre studiemiljö och för att bygga upp och bibehålla vår starka gemenskap är det lämpligt att öka antalet tillåtna sökande till posten co-phøsare.

Därför yrkar jag på
\begin{attsatser}
    \att i reglementet under punkt 10:2:L ändra högsta antal sökande från\par
        \textit{Co-phøsare (2-5)}
    \par
    till\par
        \textit{Co-phøsare (2-6)}
    \att utforma antalet enligt propositionen \textit{Tydliggörande av antalet funktionärer för en post} om den biföll
\end{attsatser}

\begin{signatures}{5}
    \mvh
    \signature{Viktor Persson}{Co-phøsare}
    \signature{Linnea Hellholm}{Co-phøsare}
    \signature{Molly Rusk}{Øverphøs}
    \signature{Elin Branzell}{Co-phøsare}
    \signature{Christian Benson}{Co-phøsare}
\end{signatures}

\end{document}
