\documentclass[10pt]{article}
    \usepackage[utf8]{inputenc}
    \usepackage[swedish]{babel}
    
    \def\doctype{Handlingar} %ex. Kallelse, Handlingar, Protkoll
    \def\mname{Styrelsemöte} %ex. styrelsemöte, Vårterminsmöte
    \def\mnum{S13/19} %ex S02/16, E1/15, VT/13
    \def\date{2019-05-14} %YYYY-MM-DD
    \def\docauthor{Edvard Carlsson}
    
    \usepackage{../e-mote}
    \usepackage{../../../e-sek}
    
    \begin{document}
    
    \heading{{\doctype} till {\mname} {\mnum}}
    
    \section*{Äskning av pengar för specialiseringskväll}
    
 	Den 16 maj kommer alumniverksamheten i samarbete med SRE att anordna en specialiseringskväll. Till denna kväll är både alumner samt studenter som läser sitt 4e eller 5e år på E-sektionen inbjudna från diverse specialiseringar. Under kvällen kommer varje specialisering som har en representant att ha en egen station där studenterna som läser år 1-3 kan gå runt och mingla, ställa frågor och få en bättre uppfattning om vad just de vill välja för specialisering. Till denna kväll är det tänkt att bjuda alla anmälda studenter och alumner på mat. Vi har räknat på maximalt 50 st portioner mat (till alumner och anmälda studenter) och har satt en budget på 20 kr per portion. För att slippa kallsvettas av nervositet i kassan på ICA sätter vi därför budgeten på 1500 kr. Vi anser att detta är ett nyttigt evenemang för sektionen som ger studenterna bättre förutsättningar att göra sitt val av specialisering och med gratis mat till alla anmälda blir evenemanget mer avslappnat och trivsamt. För att slippa bränna hela alumniverksamhetens eller SREs årsbudget på en kväll anser vi det vara rimligt att äska pengar till detta medlemsnyttiga evenemang. 
     
    Därför yrkar jag på 

    \begin{attsatser}
        \att inhandla mat till specialiseringskvällen 16/5,
        \att budgeten sätts till \SI{1500}{kr},
        \att kostnaden belastar dispositionsfonden, samt
        \att detta läggs på beslutsuppföljningen till S14/19 med undertecknad som ansvarig. 
    \end{attsatser}

    \begin{signatures}{1}
    \textit{\ist}
    \signature{Jakob Pettersson}{ENU-ordförande}
    \end{signatures}
    
    \end{document}
    