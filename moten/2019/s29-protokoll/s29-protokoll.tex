\documentclass[10pt]{article}
\usepackage[utf8]{inputenc}
\usepackage[swedish]{babel}

\def\mo{Edvard Carlsson}
\def\ms{Mattias Lundström}
\def\ji{Henrik Ramström}
%\def\jii{}

\def\doctype{Protokoll} %ex. Kallelse, Handlingar, Protkoll
\def\mname{Styrelsemöte} %ex. styrelsemöte, Vårterminsmöte
\def\mnum{S29/19} %ex S02/16, E1/15, VT/13
\def\date{2019-12-09} %YYYY-MM-DD
\def\docauthor{\ms}

\usepackage{../e-mote}
\usepackage{../../../e-sek}

\begin{document}
\showsignfoot

\heading{{\doctype} för {\mname} {\mnum}}

%\naun{}{} %närvarane under
%\nati{} %närvarande till och med
%\nafr{} %närvarande från och med
\section*{Närvarande}
\subsection*{Styrelsen}
\begin{narvarolista}
\nv{Ordförande}{Edvard Carlsson}{E16}{}
\nv{Kontaktor}{Mattias Lundström}{E17}{}
\nv{Förvaltningschef}{Henrik Ramström}{E16}{}
\nv{Cafémästare}{Jonathan Benitez}{E17}{} 
\nv{Sexmästare}{Theo Nyman}{BME18}{}
\nv{Krögare}{Davida Åström}{BME17}{}
\nv{Entertainer}{Saga Åslund}{BME18}{}
\nv{SRE-ordförande}{Lina Samnegård}{BME16}{}
\nv{ENU-ordförande}{Jakob Pettersson}{E17}{}
\nv{Øverphøs}{Stephanie Bol}{BME17}{}
\end{narvarolista}


\subsection*{Ständigt adjungerande}
\begin{narvarolista}
%\nv{}{}{}{}
%\nv{Skattmästare}{Daniel Bakic}{E15}{\nafr{10}}
%\nv{Vice Krögare}{Klara Indebetou}{BME17}{}
%\nv{Vice Krögare}{Hjalmar Tingberg}{BME16}{}
\nv{Kårrepresentant}{Ivar Vänglund}{}{\nafr{6}}
%\nv{Kårrepresentant}{Martin Bergman}{}{}
%\nv{Valberedningens ordförande}{Axel Voss}{E15}{\nafr{10b}}
%\nv{Fullmäktigeledamot}{Magnus Lundh}{E15}{\nafr{12}}
\nv{Chefredaktör}{Emil Eriksson}{E18}{}
%\nv{Inspektor}{Monica Almqvist}{}{}
%\nv{Vice Förvaltningschef}{Rasmus Sobel}{BME16}{}
%\nv{Inköps- och lagerchef}{Frida Pilcher}{E18}{}


\end{narvarolista}

%\begin{comment}
\subsection*{Adjungerande}
\begin{narvarolista}
%\nv{post}{namn}{klass}{nati/nafr/tom}
%\nv{Likabehandlingsombud}{Jonna Fahrman}{BME17}{}
%\nv{Likabehandlingsombud}{Hanna Bengtsson}{BME18}{}
%\nv{Projekfunktionär}{Emma Hjörneby}{BME17}{}
%\nv{Macapär}{Filip Larsson}{E17}{}
%\nv{Kodhackare}{Vincent Palmer}{E18}{}
\nv{Förvaltningschef Electus}{Rasmus Sobel}{BME16}{}
\nv{Sexmästare Electus}{Anna Hollsten}{BME19}{}
\nv{Entertainer Electus}{Amir Ghanaatifard}{E19}{}
\nv{Øverphøs Electus}{Sophia Carlsson}{BME17}{}
\nv{Cafemästare Electus}{Frida Pilcher}{E18}{}
\nv{ENU-ordförande Electus}{Fredrik Berg}{E17}{}
\nv{Krögare Electus}{Love Sjelvgren}{E18}{}
\nv{SRE-ordförande Electus}{Hanna Bengtsson}{BME18}{}
%\nv{Husstyrelserepresentant}{Joakim Magnusson Fredluend}{BME19}{}

\end{narvarolista}
%\end{comment}

\section*{Protokoll}
\begin{paragrafer}
\p{1}{OFMÖ}{\bes}
Ordförande {\mo} förklarade mötet öppnat kl 12.12

\p{2}{Val av mötesordförande}{\bes}
{\valavmo}

\p{3}{Val av mötessekreterare}{\bes}
{\valavms}

\p{4}{Val av justeringsperson}{\bes}
{\valavj}

\p{5}{Godkännande av tid och sätt}{\bes}
{\tosg}

\p{6}{Adjungeringar}{\bes}
%Adam Belfrage adjungerades.{}
%Hanna Bengtsson adjungerades. \\
%Jonna Fahrman adjungerades.
%Vincent Palmer adjungerades.\\
%Filip Larsson adjungerades. 

Anna Hollsten adjungerades. 

Amir Ghanaatifard adjungerades.

Sophia Carlsson adjungerades.

Fredrik Berg adjungerades.

Frida Pilcher adjungerades.

Rasmus Solel adjungerades.

Love Sjelvgren adjungerades.

Hanna Bengtsson adjungerades. 

%\textit{Inga adjungeringar.}


\p{7}{Godkännande av dagordningen}{\bes}

%Davida \ypa lägga till punkten ``Lophtet'' till dagordningen.\\
%Edvard \ypa lägga till punkten ``Ordensband'' til dagordningen.
%Fredrik \ypa att lägga till \S18b ``Teknikfokus utnyttjande av LED-café''.
%Jonathan \ypa ändra punkten §12 från att vara en beslutspunkt till diskussion. \\
%Föredragningslistan godkändes med yrkandet.
%Henrik \ypa lägga till punkten ``Faktura till F'' som §13.
%Jakob Pettersson \ypa tägga till punkten ''Øverphøs informerar'' som \S16.


Föredragningslistan godkändes med yrkandet.

\p{8}{Föregående mötesprotokoll}{\bes}
\latillprotgodkand{S26/19 och S27/19}
%\textit{\ingaprot}

\p{9}{Fyllnadsval och entledigande av funktionärer}{\bes}
\begin{fyllnadsval} %"Inga fyllnadsval." fylls i automatiskt
%\fval{Moa Rönnlund}{Halvledare}
%\entl{Fanny Månefjord}{Husstyrelserepresentant från och med 30 juni}
\fval{Adam Belfrage}{Källarmästare 2020}
\fval{Adam Lüning}{Sexig 2020}
\fval{Albin Lidbäck}{Källarmästare 2020}
\fval{Alfred Langerbeck}{Källarmästare 2020}
\fval{Alice Garnheim}{Källarmästare 2020}
\fval{Alvina Vernersson}{Källarmästare 2020}
\fval{Anna Nilsson}{Källarmästare 2020}
\fval{Anton Arvidsson}{Sexig 2020}
\fval{Arvid Hansson}{Sexig 2020}
\fval{Axel Sweger}{Sexig 2020}
\fval{Biborka Bihari}{Källarmästare 2020}
\fval{Carl Spångberg}{Källarmästare 2020}
\fval{David Karlsson}{Källarmästare 2020}
\fval{David Karlsson}{Källarmästare 2020}
\fval{Ebba Fritzell}{Sexig 2020}
\fval{Elin Dahlberg}{Källarmästare 2020}
\fval{Elliot de Wrang}{Källarmästare 2020}
\fval{Emil Bergman}{Källarmästare 2020}
\fval{Emma Amnemyr}{Sexig 2020}
\fval{Emma Hansson}{Sexig 2020}
\fval{Erica Elgcrona}{Källarmästare 2020}
\fval{Erik Chen }{Sexig 2020}
\fval{Erik Dahlberg}{Sexig 2020}
\fval{Erik Wickström}{Sexig 2020}
\fval{Evelina Persson}{Källarmästare 2020}
\fval{Fredrick Nilsson}{Sexig 2020}
\fval{Freja Sahlin}{Källarmästare 2020}
\fval{Henrik Ramström}{Källarmästare 2020}
\fval{Hilda Eliasson}{Sexig 2020}
\fval{Hugo Wikholm }{Sexig 2020}
\fval{Ida Gustafsson}{Källarmästare 2020}
\fval{Isabella Schreiter}{Källarmästare 2020}
\fval{Isak Tedenvall}{Källarmästare 2020}
\fval{Jack Johansson}{Sexig 2020}
\fval{Jacob Rinderud}{Sexig 2020}
\fval{Jakob Botvidsson}{Sexig 2020}
\fval{Jakob Pettersson}{Källarmästare 2020}
\fval{Jakob Wisth}{Sexig 2020}
\fval{Jennie Karlsson}{Källarmästare 2020}
\fval{Jennifer Zacke}{Källarmästare 2020}
\fval{Jessica Chen}{Källarmästare 2020}
\fval{Jimmy Szentes }{Sexig 2020}
\fval{Johan Halldin}{Källarmästare 2020}
\fval{Johannes Larsson}{Källarmästare 2020}
\fval{Jonathan Do}{Sexig 2020}
\fval{Jonna Fahrman}{Källarmästare 2020}
\fval{Josefin Karlsson}{Sexig 2020}
\fval{Klara Almgren }{Sexig 2020}
\fval{Klara Halling}{Sexig 2020}
\fval{Louise Rehme}{Sexig 2020}
\fval{Madeleine Nyman}{Källarmästare 2020}
\fval{Maja Eklund}{Sexig 2020}
\fval{Malin Heyden}{Källarmästare 2020}
\fval{Malin Larsson}{Sexig 2020}
\fval{Malva Persmark}{Sexig 2020}
\fval{Mattias Elmroth Nordlander}{Sexig 2020}
\fval{Max Johanson}{Källarmästare 2020}
\fval{Minna Molin}{Sexig 2020}
\fval{Miranda köhn}{Källarmästare 2020}
\fval{Moa Eberhardt }{Sexig 2020}
\fval{Moa Rönnlund}{Källarmästare 2020}
\fval{Nora Öhlin}{Sexig 2020}
\fval{Oliver Lindblom}{Källarmästare 2020}
\fval{Oliwer Ekdahl}{Källarmästare 2020}
\fval{Oscar Unosson}{Sexig 2020}
\fval{Oskar Siwerson}{Sexig 2020}
\fval{Petter Melander}{Sexig 2020}
\fval{Pontus Hallqvist}{Källarmästare 2020}
\fval{Richard Byström}{Källarmästare 2020}
\fval{Sofie Johannesson}{Källarmästare 2020}
\fval{Sophia Lennartsson}{Källarmästare 2020}
\fval{Stephanie Bol}{Källarmästare 2020}
\fval{Tilda Berglind}{Källarmästare 2020}
\fval{Tina Tabandeh}{Källarmästare 2020}
\fval{Tor Hammarbäck}{Källarmästare 2020}
\fval{Tove Hector}{Sexig 2020}
\fval{Viktor Andersson}{Källarmästare 2020}
\fval{Wilma Olsson}{Källarmästare 2020}

\end{fyllnadsval}

\p{10}{Rapporter}{}
\begin{paragrafer}
\subp{A}{Hur mår alla?}{\info}
%Punkten protokollfördes ej.


\subp{B}{Utskottsrapporter}{\info}
Edvard informerade att Testamentesskrivarkvällen gick bra. Utöver det har Edvard dragit igång Påverkansenkäten igen samt släppt utvärderingsenkät för styrelsen. 

Jonathan och CM har kört på med allmänt caféjobb. CM har sålt lussekatter, printat kaffekort, gjort i ordning iZettle, sålt kaffebomber till företag samt haft möte med CM20. 

Henrik har skrivit testamente och haft utbildning med Rasmus. Henrik har också haft möte med Källarmästarna i år och nästa år på D-sektionen. Utöver det har Henrik pratat med tillståndsenheten angående Julgillet och fixat julpynt i Edekvata tillsammans med hustomtarna. 
Till sist har Skattmästaren kommit ikapp med massa bokföring. 

Mattias och InfU har fortsatt sin verksamhet och påbörjat sin överläming. Förra veckan skrevs det testamente och den här veckan ska två event fotograferas och ett sista nummer av HeHE ska också släppas. 
I veckan väntar ett möte angående serverbytet och på onsdagkväll ska InfU-19 ha en lite kickout och myskväll tillsammans med InfU-20 med middag på Helsingkrona. ,
Utöver det har Mattias fixat FikaFika modulen på hemsidan, trixat med G Suite samt skrivit protokoll som väntar på justering. 

Davida meddelade att gillet aldrig gått så bra som i fredags. Riktiga mästare i köket som langade klägg på rekordfart. Lagkänslan var på topp och försäljningen gick rekordbra på grund av Alex som rockade köket och Johan som fullständigt krossade baren. 
Utöver det har Davida sålt biljetter till Julgillet och bokfört. Skrivit ut utgången alkohol och strukturerat upp planeringen av Julgillet. Gått igenom testamentet. 

Stephanie och NollU är färdig med sin överlämning och alla har fått sina testamenten. 

Jakob och ENU har haft möte med utskottet och arbetat med CV-granskningen vilket är sista eventet för i år. Utöver det meddelade Jakob att casekvällen med ALTEN var lyckad och att han fortsätter att jaga fakturor. Jakob har också haft möte med Alumni för att göra någon form av sammanställning av alumnistatistiken som sektionen kan skicka ut och publicera. 

Saga och NöjU meddelade att SEKTIONEN VANN SÅNGARSTRIDEN och att alla haft roligt. Saga är imponerad av alla inblandade! Utöver det har NöjU haft kick-out med middag på VGs vilket var mysigt. Nu börjar NöjU-veckan och NöjU-19 är taggade. 

Theo och Sexmästeriet har planerat överlämnning och årets sista sittning. De är taggade. Utöver det ska Theo ha överlämningsmöte på fredagen. 

Lina och Studieråtet har forsatt sin vanliga verksamhhet. I kollegiet har de jobbat vidare att planera tackbrunchen för årskursansvariga med nästan 100 anmälda. I veckan är det sista mötet för året. 




\subp{C}{Ekonomisk rapport}{\info}
Henrik informerade att ekonomin ser bra ut.


\subp{D}{Kåren informerar}{\info}

Ivar meddelade att Speak Up-Days är slut. På onsdag är det Puben Puben med tap-takeover från Brygghuset Finn. Flera sorters julöl kommer serveras. 

Ivar meddelade också att Kåren ber om externa ansökningar angående Kårens klimatarbete, mer information finns på Kårnytt.


\end{paragrafer}

\p{11}{Äskning av pengar för inköp av symaskin}{\bes}
Edvard presenterade äskningen för inköp av ny symaskin.

Det är en klassisk och nybörjarvänlig modell. 

Henrik \ypa på att kostnaden ska belasta dispositionsfonden istället för utrustningsfonden.

Edvard \js. 

Mötet \textbf{beslutade att} bifalla den framvaskade motionen. 

\p{12}{Informationsspridning och separat grupp för ENU}{\dis}
Jakob presenterade diskussionspunkten. 

Edvard och Mattias var positiv till idén. Höll med om att sektionens informationskanaler kan göras effektivare.

Mötet diskuterade informationsspridning. 

Jonathan tyckte att mötet kan skjuta på diskussionen till nästa år. 

\p{13}{Sammanslagning av Elektroteknik och Medicin och teknik på Chalmers}{\dis}

Edvard presenterade underlaget för diskussionen. Mötet diskuterade frågorna i ordning. 

Henrik menar att segregationen är mest studierelaterat mellan programmen, inte alls på sektionen och fritiden. Man är närmare med personer man delar sektion men jämfört med program som läser samma kurser.

Davida tyckte inte det är någon segregation alls på sektionen. När man ska hitta företagssammarbeten för sektionens näringsliv kan det skilja sig en del åt. Davida tyckte det är bra med bredd. 

Lina tyckte inte heller det är någon segregation. Vi går på samma campus samt pluggar och äter på samma ställe.

Saga poängeterade att när man ska starta en ny verksamhet är det betydligt lättare att haka på en befintlig. Främst om man studerar och ofta befinner sig på samma platser. 

Mötet konstaterade att E-sektionen har tillgång till sektions- och festlokal där båda programmen får plats och känner sig välkomna.

Gällande frågan om kapacitet och engagemang på sektionen upplevde Jakob att vi har för mycket engagemang!

Att sektionen heter \textit{E-sektionen} har aldrig upplevts som ett problem enligt de på mötet närvarande som går på BME-programmet. 

\p{14}{Överlämning av Google Drive}{\dis}
Mattias informerade om överlämning av Google Drive och mötet fick chans att ställa frågor.


\p{15}{Nästa styrelsemöte}{\bes}
\Mba nästa styrelsemöte ska äga rum 2019-12-16 17.10 i E:1124.

\p{16}{Beslutsuppföljning}{\bes}
%Edvard \ypa stryka ''Projektfunktionär: Vårbal'' från Beslutsuppföljningen. Liknande projekt uppmuntras.
%\Mbaby
%Davida \ypa skjuta upp ''Inköp av draghandtag till cykelvagn'' till nästa styrelsemöte.
%\Mbaby

Edvard \ypa stryka ''Testamentesskrivarkväll''från Beslutsuppföljningen. 

\Mbaby

Edvard \ypa skjuta upp ''Funktionärstacket'' till nästa styrelsemöte då man fortfarande väntar på kvitton. 

\Mbaby

\p{17}{Övrigt}{\dis}

Henrik tog upp att D-sektionen vill köpa nya plastbord som sektionerna just nu delar på. Anledningen är att de är ruttna. 

Edvard meddelade att det är städning av EKEA ikväll. 

Theo meddelade att funktionärsskiphte kommer vara på fredagen den 6:e mars istället. De har bokat Lophtet och det blir sittning i E-foajen likt förra året.   

Rasmus meddelade att det är Luciatåg i år igen. Vill man delta eller känner någon som kan vilja delta ska man höra av sig till Rasmus. 

\p{18}{Sammanfattning av mötet}{\info}
% \Fbs
% \Fbsup

Källarmästare och Sexiga har fyllnadsvalts inför nästa år. 

Mötet beslutade att köpa in en symaskin för 3000 kr. 

\p{19}{OFMA}{\bes}
{\mo} förklarade mötet avslutat kl. 13.12. 
\end{paragrafer}

%\newpage
\hidesignfoot
\begin{signatures}{3}
\signature{\mo}{Mötesordförande}
\signature{\ms}{Mötessekreterare}
\signature{\ji}{Justerare}
\end{signatures}
\end{document}
