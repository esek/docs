\documentclass[10pt]{article}
    \usepackage[utf8]{inputenc}
    \usepackage[swedish]{babel}
    
    \def\doctype{Handlingar} %ex. Kallelse, Handlingar, Protkoll
    \def\mname{Styrelsemöte} %ex. styrelsemöte, Vårterminsmöte
    \def\mnum{S10/19} %ex S02/16, E1/15, VT/13
    \def\date{2019-04-08} %YYYY-MM-DD
    \def\docauthor{Edvard Carlsson}
    
    \usepackage{../e-mote}
    \usepackage{../../../e-sek}
    
    \begin{document}
    
    \heading{{\doctype} till {\mname} {\mnum}}
    
    \section*{Äskning av pengar för inköp av vattendunkar}
    
 	Vatten är gratis, gott och sexigt. Därför är det viktgt att det vi kan förse våra medlemmar och gäster på evenemang rikligt med denna vara. I nuläget måste vi hyra vattendunkar när vi har evenemang där det krävs. Jag tycker det är bättre att vi har egna eftresom vi faktiskt har många evenmang som har behov av att dela ut stora mängder vatten utan tillgång till kranar. Utöver detta skulle de även kunna användas mer regelbundet av KM och Sexet. 
    
    Jag har tittat runt lite på nätet och hittat dessa:

\begin{tabular}{ l  l }
    
    \href{https://www.granngarden.se/plastdunk/p/1076530}{\textit{Plastdunk (länk)}}  & \SI{169}{kr}  \\
   
    \href{https://www.granngarden.se/avtappningskran-till-dunk/p/1072552}{\textit{Avtappningskran (länk)}} & \SI{79}{kr}  \\
   
\end{tabular}

    Jag yrkar 
    \begin{attsatser}
        \att köpa in 3 uppsättningar av ovan nämnda vattendunkar och tappkranar,
        \att budgeten sätts till \SI{800}{kr},
        \att kostnaden belastar dispositionsfonden, samt
        \att detta läggs på beslutsuppföljningen till S12/19 med undertecknad som ansvarig. 
    \end{attsatser}

    \begin{signatures}{1}
    \textit{\ist}
    \signature{\docauthor}{Ordförande}
    \end{signatures}
    
    \end{document}
    