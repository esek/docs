\documentclass[10pt]{article}
    \usepackage[utf8]{inputenc}
    \usepackage[swedish]{babel}
    
    \def\doctype{Sena handlingar} %ex. Kallelse, Handlingar, Protkoll
    \def\mname{Styrelsemöte} %ex. styrelsemöte, Vårterminsmöte
    \def\mnum{S17/19} %ex S02/16, E1/15, VT/13
    \def\date{2019-09-02} %YYYY-MM-DD
    \def\docauthor{Edvard Carlsson}
    
    \usepackage{../e-mote}
    \usepackage{../../../e-sek}
    
    \begin{document}
    
    \heading{{\doctype} till {\mname} {\mnum}}
    
    \section*{Äskning av pengar till inköp av ny skrivare}
    
    Tiden flyter på och arbetet fortgår. Detta trots att vi har en skrivare som ofta endast förvaltningschefen kan använda. Detta är ju inte helt optimalt, speciellt nu under nollningstider och skulle definitivt kunna förbättras. Därmed föreslår jag att vi snarast köper in en ny skrivare som är bättre anpassad för vårt arbete. 

Med sin moderna look och mått på 50x40x50 (BxDxH) så kommer den passa in perfekt där den gamla skrivaren stod, med den enda förändringen att den kommer printa mycket snabbare. Detta då HP Color LaserjetPro MFP M280nw är utrustad för att kunna erbjuda en hög utskriftshastighet. Nästan lite för snabb för sin 50-sidors dokumentmatare. 
Den klarar även av mobila utskrifter vilket kommer underlätta från att alla måste sitta och hålla på med USB sladden och den klarar även av en driftcykel på 40000 sidor (max) per månad vilket betyder att den kommer klara av vår nollning.

Det kanske viktigaste som är med den är användar recensionerna som har varit possitiva på alla ställen jag har letat!

Med allt detta sagt så yrkar jag därmed på



    \begin{attsatser}
        \att köpa in en HP Color LarserjetPro MFP M280nw tillsammans med patroner (\href{https://www.netonnet.se/art/dator-surfplatta/skrivare-scanner/laserskrivare/hp-color-laserjetpro-mfp-m280nw/248675.8937/}{\textit{länk}}),
        \att budget sätts till \SI{6000}{kr},
        \att kostnaden belastar dispositionsfonden, samt
        \att detta läggs på beslutsuppföljningen till S19/19 med undertecknad som ansvarig. 
    \end{attsatser}

    \begin{signatures}{1}
    \textit{\ist}
    \signature{Henrik Ramström}{Förvaltningschef}
    \end{signatures}
    



    \end{document}
    