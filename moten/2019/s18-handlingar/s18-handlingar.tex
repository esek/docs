\documentclass[10pt]{article}
    \usepackage[utf8]{inputenc}
    \usepackage[swedish]{babel}
    
    \def\doctype{Handlingar} %ex. Kallelse, Handlingar, Protkoll
    \def\mname{Styrelsemöte} %ex. styrelsemöte, Vårterminsmöte
    \def\mnum{S18/19} %ex S02/16, E1/15, VT/13
    \def\date{2019-09-09} %YYYY-MM-DD
    \def\docauthor{Edvard Carlsson}
    
    \usepackage{../e-mote}
    \usepackage{../../../e-sek}
    
    \begin{document}
    
    \heading{{\doctype} till {\mname} {\mnum}}
    
    \section*{Biljard}
    
 	Hejsan! Jag har lite synpunkter angående hur biljard används i nuläget. Jag la ned ungefär
65 timmar på att renovera biljard i somras med övriga sektionsmedlemmar. Jag gjorde
dessutom med optionen att biljard inte längre skulle användas som ett rum att lagra skräp i.
Jag förstår att under nollningen så behövs det användas till uppdragen genom förvaring men
det förvaras på ett extremt ohållbart sätt då det bara trycks in massa skit där i. Man kan
förvara det man vill ha på ett sätt som är bättre, kundvagnar eller martin-och-cervera vagnar
med märkningar till respektive uppdrag så kan man slänga resten. Det är lättare för
uppdragen i sin helhet och ser snyggare ut överlag. Sen hör inte grillkol eller annat skräp
hemma där. Dessutom att lägga trä på våra nya läderstolar repar dem. Jag tycker inte det är
så roligt att man inte visa någon som helst respekt mot oss som ändå har arbetat hårt för att
göra den här renoveringen möjlig och verkligen bara dumpar skit i det här utrymmet och går
tillbaka till gamla vanor på mindre en dag. Däremot är jag som sagt helt fine med att man
använder biljard som förvaringsställe under nollningen men på ett organiserat och städat
sätt.

   

    \begin{signatures}{1}
    \textit{\mvh}
    \signature{Adam Belfrage}{}
    \end{signatures}
    



    \newpage

    \section*{Äskning av pengar för inköp av nya izettle läsare}
    
 	Som vi alla vet så använder vi oss nu utav iZettle som vår betalningstjänst. Detta innebär att vi ligger i behov av att ha deras kortläsare till hands för att kunna bedriva vår dagliga verksamhet. 
Nu under nollningen har vi märkt att vissa iZettle kortläsare har varit dåliga och därmed ligger vi i absolut underkant till så många iZettle kortläsare som vi behöver. Detta har inneburit att vi ofta har behövt hyra från kåren, vilket tar upp dyrbar tid, ork, kapital och skulle enkelt kunna undvikas med en liten investering i nya kortläsare.

I dagsläget använder vi oss utav en kortläsare som heter iZettle reader 1. Denna har IZettle tyvärr slutat sälja och innebär att vi kommer istället behöva beställa iZettle reader 2 som är lite dyrare.

Då behovet av kortläsare är som störst nu under nollningen bör vi snarast beställa in nya och därmed så tycker jag att vi lägger detta på dispen istället för att vi väntar till sektionsmötet.


Med allt detta sagt så ser jag att vi har 3 alternativ:

\begin{enumerate}
  \item Vi kan strunta i att beställa in nya kortläsare och se oss falla ihop som sektion till att bli en hög med smuts (läses KTH).
  \item Vi kan beställa in en extra kortläsare nu och sen senare kanske köpa in en extra på t.ex. sektionsmötet eller någon annan gång som det kan tyckas passa. Detta kan tyvärr innebära extra fraktavgifter och att vi fortfarande ligger lite på gränsen till så många izettle kortläsare som vi vill ha.
   \item Vi beställer in 2st extra iZettle Reader 2. Dessa kommer kosta ungefär 1200 kr (tillsammans med en rabatt som vi får som långtidskund) med potentiellt lite frakt och därmed så yrkar jag på
\end{enumerate}

    \begin{attsatser}
        \att köpa in 2st nya kortläsare av typen iZettle Reader 2  (\href{https://www.izettle.com/se/kortterminal}{\textit{länk}}),
        \att budget sätts till \SI{1500}{kr},
        \att kostnaden belastar dispositionsfonden, samt
        \att detta läggs på beslutsuppföljningen till S20/19 med undertecknad som ansvarig. 
    \end{attsatser}

    \begin{signatures}{1}
    \textit{\ist}
    \signature{Henrik Ramström}{Förvaltningschef}
    \end{signatures}

    \end{document}
    