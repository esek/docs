\documentclass[10pt]{article}
    \usepackage[utf8]{inputenc}
    \usepackage[swedish]{babel}
    
    \def\doctype{Handlingar} %ex. Kallelse, Handlingar, Protkoll
    \def\mname{Styrelsemöte} %ex. styrelsemöte, Vårterminsmöte
    \def\mnum{S24/19} %ex S02/16, E1/15, VT/13
    \def\date{2019-11-04} %YYYY-MM-DD
    \def\docauthor{Edvard Carlsson}
    
    \usepackage{../e-mote}
    \usepackage{../../../e-sek}
    
    \begin{document}
    
    \heading{{\doctype} till {\mname} {\mnum}}
    
    \section*{Äskning av pengar för inköp av video-adapter}
    
 	Nya moderna labtops har sedan några år tillbaka börjat skippa stödet för inbyggd HDMI-utgång och istället börjat använda lösningar som mini displayport och USB-C. Som det ser ut är det USB-C som dominerar på marknaden och alla moderna PC-labtops har idag en USB-C utgång. De senaste åren har också Apple börjat använda USB-C som utgång på alla sina nya modeller. 

För att sektionsmedlemmar enkelt ska kunna koppla upp sig till videoskärmar, projektorer och annat yrkar jag


   \begin{attsatser}
        \att köpa in Plexgear Multiadapter för USB-C med följande specifikation.
HDMI-port (4K), Nätverksport, Minneskortläsare, 2 USB-A portar samt genomgående laddning för USB-C. (\href{https://www.kjell.com/se/produkter/dator-natverk/mac-tillbehor/plexgear-multiadapter-for-usb-c-p61629}{\textit{länk}}),
        \att budget sätts till \SI{799}{kr},
        \att kostnaden belastar dispositionsfonden, samt
        \att detta läggs på beslutsuppföljningen till S26/19 med undertecknad som ansvarig. 
    \end{attsatser}


alternativt

\begin{attsatser}
        \att köpa in Multiadapter USB-C till HDMImed följande specifikation.
HDMI-port, USB-A port samt genomgående laddning för USB-C. (\href{https://www.kjell.com/se/produkter/dator-natverk/kablar-adaptrar/usb/usb-adaptrar/multiadapter-usb-c-till-hdmi-p96544}{\textit{länk}}),
        \att budget sätts till \SI{500}{kr},
        \att kostnaden belastar dispositionsfonden, samt
        \att detta läggs på beslutsuppföljningen till S26/19 med undertecknad som ansvarig. 
    \end{attsatser}


    \begin{signatures}{1}
    \textit{\ist}
    \signature{Mattias Lundström}{Kontaktor}
    \end{signatures}
    

   
    \end{document}
    