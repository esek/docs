\documentclass[10pt]{article}
\usepackage[utf8]{inputenc}
\usepackage[swedish]{babel}

\def\mo{Edvard Carlsson}
\def\ms{Sonja Kenari}
\def\ji{Lina Samnegård}
%\def\jii{}

\def\doctype{Protokoll} %ex. Kallelse, Handlingar, Protkoll
\def\mname{Styrelsemöte} %ex. styrelsemöte, Vårterminsmöte
\def\mnum{S11/19} %ex S02/16, E1/15, VT/13
\def\date{2019-04-15} %YYYY-MM-DD
\def\docauthor{\ms}

\usepackage{../e-mote}
\usepackage{../../../e-sek}

\begin{document}
\showsignfoot

\heading{{\doctype} för {\mname} {\mnum}}

%\naun{}{} %närvarane under
%\nati{} %närvarande till och med
%\nafr{} %närvarande från och med
\section*{Närvarande}
\subsection*{Styrelsen}
\begin{narvarolista}
\nv{Ordförande}{Edvard Carlsson}{E16}{}
\nv{Kontaktor}{Sonja Kenari}{E15}{}
\nv{Förvaltningschef}{Henrik Ramström}{E16}{}
\nv{Cafémästare}{Jonathan Benitez}{E17}{}
\nv{Sexmästare}{Theo Nyman}{BME18}{}
\nv{Krögare}{Davida Åström}{BME17}{}
%\nv{Entertainer}{Saga Åslund}{BME18}{}
\nv{SRE-ordförande}{Lina Samnegård}{BME16}{}
\nv{ENU-ordförande}{Jakob Pettersson}{E17}{}
\nv{Øverphøs}{Stephanie Bol}{BME17}{Närvarade under §11}
\end{narvarolista}


\subsection*{Ständigt adjungerande}
\begin{narvarolista}
%\nv{Sigillbevarare}{Matilda Horn}{BME18}{\nati{17}}
%\nv{}{}{}{}
%\nv{Kårrepresentant}{Jacob Karlsson}{}{\nafr{3}}
%\nv{Valberedningens ordförande}{Elin Magnusson}{}{}
%\nv{Skattmästare}{Daniel Bakic}{E15}{}
%\nv{Vice Krögare}{Klara Indebetou}{BME17}{}
%\nv{Vice Krögare}{Hjalmar Tingberg}{BME16}{}
\nv{Kårrepresentant}{Filip Johansson}{}{}
%\nv{Kårrepresentant}{Anna Qvil}{}{}
%\nv{Valberedningens ordförande}{Axel Voss}{E15}{\nafr{10b}}
%\nv{Fullmäktigeledamot}{Magnus Lundh}{E15}{\nafr{12}}
%\nv{Chefredaktör}{Max Mauritsson}{BME16}{}
%\nv{Elektras Ordförande}{Elisabeth Pongratz}{}{}
%\nv{Inspektor}{Monica Almqvist}{}{}
%\nv{Valberedningens ordförande}{Axel Voss}{E15}{\nafr{11}}

\end{narvarolista}

%\begin{comment}
\subsection*{Adjungerande}
\begin{narvarolista}
%\nv{post}{namn}{klass}{nati/nafr/tom}
\nv{Likabehandlingsombud}{Hanna Bengtsson}{BME18}{}
\nv{Likabehandlingsombud}{Jonna Fahrman}{BME17}{\nafr{6}}
%\nv{Projektfunktionär}{Sophia Carlsson}{BME17}{}
%\nv{Projekfunktionär}{Emma Hjörneby}{BME17}{}
%\nv{}{}{}{}
\end{narvarolista}
%\end{comment}

\section*{Protokoll}
\begin{paragrafer}
\p{1}{OFMÖ}{\bes}
Ordförande {\mo} förklarade mötet öppnat kl.12.16.

\p{2}{Val av mötesordförande}{\bes}
{\valavmo}

\p{3}{Val av mötessekreterare}{\bes}
{\valavms}

\p{4}{Val av justeringsperson}{\bes}
{\valavj}

\p{5}{Godkännande av tid och sätt}{\bes}
{\tosg}

\p{6}{Adjungeringar}{\bes}
%Adam Belfrage adjungerades.{}
Hanna Bengtsson adjungerades. \\
Jonna Fahrman adjungerades


%\textit{Inga adjungeringar.}


\p{7}{Godkännande av dagordningen}{\bes}
%Theo \ypa lägga till sena handlingar till dagordningen.
%Davida \ypa lägga till punkten ``Lophtet'' till dagordningen.\\
%Edvard \ypa lägga till punkten ``Ordensband'' til dagordningen.
%Fredrik \ypa att lägga till \S18b ``Teknikfokus utnyttjande av LED-café''.
%Jonathan \ypa ändra punkten §12 från att vara en beslutspunkt till diskussion. \\
%Föredragningslistan godkändes med yrkandet.
%Föredragningslistan godkändes med samtliga yrkanden.

Dagordningen godkändes.


\p{8}{Föregående mötesprotokoll}{\bes}
\latillprot{S08/19 och S10/19}
%\textit{\ingaprot}

\p{9}{Fyllnadsval och entledigande av funktionärer}{\bes}
\begin{fyllnadsval} %"Inga fyllnadsval." fylls i automatiskt
%\fval{Moa Rönnlund}{Halvledare}
\fval{Simon Mahdavi}{Diod}
%\entl{Namn}{Post}
\end{fyllnadsval}

\p{10}{Rapporter}{}
\begin{paragrafer}
\subp{A}{Hur mår alla?}{\info}
Punkten protokollfördes ej.

\subp{B}{Utskottsrapporter}{\info}

CM är stängt denna veckan. Däremot ska LED städas denna veckan och datum till stora caféfesten ska spikas. Det ska även komma 
    fler koppar till caféet.

FVU har skött lite lokalunderhåll och bokfört. Henrik har även mailat lite med Kåren angående Ullas kontrakt.

InfU rullar på. Det blir en Git-workshop för kodhackarna denna veckan. Massa bilder uppe   
    från sektionsevent på Facebook så in och tagga varandra! Det finns även nytt HeHE ute som man 
    inte borde missa.

KM har haft ett helt vegetariskt gille. Påskgillet blir inställt p.g.a att det inte har ansökts om tillstånd.

NollU har haft möte med SVL. På tisdag är det phadderutbildning och i helgen var det skiphte.

ENU håller på att spika företag för Lunch med en Ingenjör. Jakob har även planerat in några möten med olika företag.

E6 har lagat mat åt BME-programledningen. Det är även en sittning inbokad med K-sektionen i slutet
    av maj! Det är hela 3 sittningar kvar under våren.

SRE har fått in alla ifyllda CEQ-enkäter som ska censurerar. CEQ-möten ska inplaneras då det har varit problem
    med några kurser.

\subp{C}{Ekonomisk rapport}{\info}
Det ser bra ut.

\subp{D}{Kåren informerar}{\info}
Det är JätteFunktionärsFesten den 4/5. Ett stort event som är gratis för alla som är funktionär på sektionen eller Kåren. Glöm inte att anmäla er!

Det finns även lediga poster på tlth.se/lediga. Kika in dessa och bli aktiv på en helt ny nivå.

Glöm heller inte att Teknologkåren fyller 35 år i maj. Håll utkik på Kårens informationskanaler för att 
    hålla dig uppdaterad om trevliga evenemang som händer under jubileumsveckan.

\end{paragrafer}

\p{11}{Styrelsen i NollEguiden}{\dis}
Styrelsen diskuterade om hur vi ska synas i NollEguiden och när vi ska ta bilder. 



\p{12}{Enkät från likabehandlingsombud}{\dis}
Likabehandlingsombuden vill skicka ut en enkät angående sexuella trakasserier på sektionen. Detta för att 
    få en uppfattning om vad som händer och vad man ska göra för att förhindra att denna typen av situationer uppstår. 

\p{13}{Do's and don'ts till HTM utifrån VTM}{\dis}
Styrelsen diskuterade om saker som gått bra och dåligt under VTM.

\p{14}{Nästa styrelsemöte}{\bes}
\Mba nästa styrelsemöte ska äga rum 2019-05-06 kl.12.10 i E:1123.

\p{15}{Beslutsuppföljning}{\bes}

\textit{Inga beslut att följa upp.}

\p{16}{Övrigt}{\dis}
Jonathan har fått svar från Ulla och hon är jättetaggad på att jobba igen. Mötet diskuterade 
    frågan och om hur det ska implementeras på bästa sätt.

Edvard informerar lite om hur kallelsen ska gå ut till nästa styrelsemöte.

Styrelsen borde informera sina funktionärer om förmåner som man har som funktionär på sektionen.

Sonja undrade vad som hände med styrelsesittningen. Edvard ska kolla upp detta.

Theo undrar hur man gör med funktionärer som bytar på program. Henrik svarade med att vi inte har några speciella regler för det på E-sektionen.

\p{17}{OFMA}{\bes}
{\mo} förklarade mötet avslutat kl. 13.03.
\end{paragrafer}

%\newpage
\hidesignfoot
\begin{signatures}{3}
\signature{\mo}{Mötesordförande}
\signature{\ms}{Mötessekreterare}
\signature{\ji}{Justerare}
\end{signatures}
\end{document}
