\documentclass[10pt]{article}
\usepackage[utf8]{inputenc}
\usepackage[swedish]{babel}

\def\mo{Edvard Carlsson}
\def\ms{Sonja Kenari}
\def\ji{Lina Samnegård}
%\def\jii{}

\def\doctype{Protokoll} %ex. Kallelse, Handlingar, Protkoll
\def\mname{Styrelsemöte} %ex. styrelsemöte, Vårterminsmöte
\def\mnum{S02/19} %ex S02/16, E1/15, VT/13
\def\date{2019-01-30} %YYYY-MM-DD
\def\docauthor{\ms}

\usepackage{../e-mote}
\usepackage{../../../e-sek}

\begin{document}
\showsignfoot

\heading{{\doctype} för {\mname} {\mnum}}

%\naun{}{} %närvarane under
%\nati{} %närvarande till och med
%\nafr{} %närvarande från och med
\section*{Närvarande}
\subsection*{Styrelsen}
\begin{narvarolista}
\nv{Ordförande}{Edvard Carlsson}{E16}{}
\nv{Kontaktor}{Sonja Kenari}{E15}{}
\nv{Förvaltningschef}{Henrik Ramström}{E16}{}
\nv{Cafémästare}{Jonathan Benitez}{E17}{}
\nv{Sexmästare}{Theo Nyman}{BME18}{}
%\nv{Krögare}{Davida Åström}{BME17}{}
\nv{Entertainer}{Saga Åslund}{BME18}{}
\nv{SRE-ordförande}{Lina Samnegård}{BME16}{}
\nv{ENU-ordförande}{Jakob Pettersson}{E17}{}
\nv{Øverphøs}{Stephanie Bol}{BME17}{}
\end{narvarolista}


\subsection*{Ständigt adjungerande}
\begin{narvarolista}
%\nv{Sigillbevarare}{Matilda Horn}{BME18}{\nati{17}}
%\nv{}{}{}{}
%\nv{Kårrepresentant}{Jacob Karlsson}{}{\nafr{3}}
%\nv{Valberedningens ordförande}{Elin Magnusson}{}{}
\nv{Skattmästare}{Daniel Bakic}{E15}{}
\nv{Kårrepresentant}{Philip Johansson}{}{}
\nv{Kårrepresentant}{Anna Qvil}{}{}
%\nv{Talman}{Erik Månsson}{E14}{}
%\nv{Elektras Ordförande}{Elisabeth Pongratz}{}{}
%\nv{Inspektor}{Monica Almqvist}{}{}
\end{narvarolista}

%\begin{comment}
\subsection*{Adjungerande}
\begin{narvarolista}
%\nv{Fotograf}{Alexander Wik}{BME17}{\nati{17}}
\nv{Sigillbevarare}{Matilda Horn}{BME18}{}
%\nv{Sångförman}{Adam Belfrage}{BME17}{\nati{17}}
\end{narvarolista}
%\end{comment}

\section*{Protokoll}
\begin{paragrafer}
\p{1}{OFMÖ}{\bes}
Ordförande {\mo} förklarade mötet öppnat kl.12.11.

\p{2}{Val av mötesordförande}{\bes}
{\valavmo}

\p{3}{Val av mötessekreterare}{\bes}
{\valavms}

\p{4}{Val av justeringsperson}{\bes}
{\valavj}

\p{5}{Godkännande av tid och sätt}{\bes}
{\tosg}

\p{6}{Adjungeringar}{\bes}
Matilda Horn adjungerades.
%Adam Belfrage adjungerades.{}
%Förnamn Efternamn adjungerades


\p{7}{Godkännande av dagordningen}{\bes}
Theo \ypa lägga till sena handlingar till dagordningen.


%Fredrik \ypa att lägga till \S18b ``Teknikfokus utnyttjande av LED-café''.

Föredragningslistan godkändes med yrkandet.
%Föredragningslistan godkändes med samtliga yrkanden.

\p{8}{Föregående mötesprotokoll}{\bes}
\latillprot{S01/19}
%\ingaprot

\p{9}{Fyllnadsval och entledigande av funktionärer}{\bes}
\begin{fyllnadsval} %"Inga fyllnadsval." fylls i automatiskt
\fval{Martin Ollén}{Diod}
\fval{Ester Pörtfors}{Diod}
\fval{Elina Yrlid}{E1-ansvarig}
\fval{Johan Siwerson}{Ekiperingsexpert}
\fval{Mia Cicovic}{HTF-ansvarig}

%\entl{Namn}{Post}
\end{fyllnadsval}

\p{10}{Rapporter}{}
\begin{paragrafer}
\subp{A}{Hur mår alla?}{\info}
Punkten protokollfördes ej.

\subp{B}{Utskottsrapporter}{\info}
Jonathan berättar att LED är igång. Mycket bättre nu när alla är tillbaka från semester. 34 nya förslag för namn på kaffebomberna.

För FVU går det bra. Fokus på inventering denna veckan. Lite problem med lagervärdet men borde vara löst nu. Börjat titta på nya inköp till sektionen. Klar med utlämning av bankkort och insamling av nycklar.

Går bra för InfU med uppstartsmöten och att få igång utskottet.

NollUs phadderinfo är flyttas till 14/2 och inte V.5 som tidigare informerat. Många möten denna veckan för att komma ikapp efter semester. Håller kontakten med SVL.

ENU svarat på mail om prisuppgifter och hittat lite spännande företag att kanske ha samarbete med i framtiden. Jakob har tillsammans med sin Vice uppdaterat prislistan för att företagen ska få tydlig överblick över vad vi kan erbjuda. ENU har även haft  sitt första möte för utskottet och gått igenom framtidsvisionen för året. En CV fotografering är planerad 14/2. Jakob har även haft möte med Ericsson angående eventuella samarbeten. 

NöjU haft årets första ``Sporta med E'' och spelkväll. Entertainer har även haft möten om massa grejer som Dreamhacke och utEDischo. Cykelfest med K-sektionen kommer eventuellt bli av i maj.

Sexet planerar mycket för helgen. Det kommer bli bra! 

SRE kollat på eventuella event för året. Nya år E1 ansvarig och HTF ansvarig har tillkommit till utskottet. Första SRX mötet kommer äga rum senare idag.



\subp{C}{Ekonomisk rapport}{\info}
Vi gör av med mer pengar än vad vi gjorde förra året, så det går bra.

\subp{D}{Kåren informerar}{\info}
Kårkontakterna nämner framtida styrelseutbildning. 

Första fullmäktigemötet kommer vara den 18/1-19. 

Fortsätt gärna nominera eller kandidera till Øverste och Nolleamiral så att vi kan ha en nollning på hela campus nästa år.

\subp{E}{Omvärldsrapporter}{\info}
Resan för att åka till Norge och representera E-sektionen måste planeras.

K- och W-sektionen kommer få nya lokaler. Styrelsen är inbjudna till invigning 6/2 kl.18.
\end{paragrafer}

\p{11}{Äskning av pengar till inköp av mikrofon}{\bes}
Theo vill äska pengar till fungerande mikrofoner som både sångförmännen kan använda samt Øverbananens band.

Henrik undrar om det ska läggas på FVUs budget och inte på dispositionsfonden.

Edvard informerar om att det är säkrare och naturligare att lägga på dispositionsfonden.

\Mbaby

\p{12}{Remiss: Policy för entledigande och avstängning}{\dis}
Styrelsen diskuterade den nya policyn från Kåren.

\p{13}{Budget för kickoffer}{\dis}
Diskussion om att återigen höja budgeten per funktionär. 
Henrik fixar ett Excel-dokument som alla utskottsordförande kan fylla i med hur många funktionärer de har så att vi kan få konkreta siffror på hur mycet vi kan öka och hur det ska gå till.

\p{14}{Medaljer till Skiphtet}{\dis}
Edvard \ypa vi ger en bidragsmedaljen till en kämpe på sektionen under skiphtesgasquen.

\Mbaby

\p{15}{Profilkläder}{\dis}
Alla utskott får i läxa att se om det finns intresse med profilkläder för utskottet som alla funktionärer själv står för kostnaderna. 
Det kommer ta tid att få dem.

\p{16}{Nästa styrelsemöte}{\bes}
\Mba nästa styrelsemöte ska äga rum 2019-02-06 kl. 12.10 i E:1123.

\p{17}{Beslutsuppföljning}{\bes}
%Henrik \ypa skuta upp ``Inköp av kylskåp'' till nästa styrelsemöte.
% Alexander \ypa stryka \emph{Kundvagnar} från beslutsuppföljningen.
%{\Ibfu}
Henrik \ypa stryka inköpa av kylskåp från beslutsuppföljningen.

\Mbaby

\p{18}{Övrigt}{\dis}
Kandidater för att vara likabehandlingsombud får komma på nästa styrelsemöte för att en ska väljas.

Edvard informerar om vår team drive.


\p{19}{OFMA}{\bes}
{\mo} förklarade mötet avslutat kl. 13.01. 
\end{paragrafer}

%\newpage
\hidesignfoot
\begin{signatures}{3}
\signature{\mo}{Mötesordförande}
\signature{\ms}{Mötessekreterare}
\signature{\ji}{Justerare}
\end{signatures}
\end{document}
