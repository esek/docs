\documentclass[10pt]{article}
\usepackage[utf8]{inputenc}
\usepackage[swedish]{babel}

\def\mo{Edvard Carlsson}
\def\ms{Mattias Lundström}
\def\ji{Jonathan Benitez}
%\def\jii{}

\def\doctype{Protokoll} %ex. Kallelse, Handlingar, Protkoll
\def\mname{Styrelsemöte} %ex. styrelsemöte, Vårterminsmöte
\def\mnum{S26/19} %ex S02/16, E1/15, VT/13
\def\date{2019-11-18} %YYYY-MM-DD
\def\docauthor{\ms}

\usepackage{../e-mote}
\usepackage{../../../e-sek}

\begin{document}
\showsignfoot

\heading{{\doctype} för {\mname} {\mnum}}

%\naun{}{} %närvarane under
%\nati{} %närvarande till och med
%\nafr{} %närvarande från och med
\section*{Närvarande}
\subsection*{Styrelsen}
\begin{narvarolista}
\nv{Ordförande}{Edvard Carlsson}{E16}{}
\nv{Kontaktor}{Mattias Lundström}{E17}{}
\nv{Förvaltningschef}{Henrik Ramström}{E16}{}
\nv{Cafémästare}{Jonathan Benitez}{E17}{}
\nv{Sexmästare}{Theo Nyman}{BME18}{}
\nv{Krögare}{Davida Åström}{BME17}{}
\nv{Entertainer}{Saga Åslund}{BME18}{}
\nv{SRE-ordförande}{Lina Samnegård}{BME16}{}
\nv{ENU-ordförande}{Jakob Pettersson}{E17}{}
\nv{Øverphøs}{Stephanie Bol}{BME17}{}
\end{narvarolista}


\subsection*{Ständigt adjungerande}
\begin{narvarolista}
%\nv{}{}{}{}
%\nv{Vice Krögare}{Klara Indebetou}{BME17}{}
%\nv{Vice Krögare}{Hjalmar Tingberg}{BME16}{}
\nv{Kårrepresentant}{Ivar Vänglund}{}{}
\nv{Kårrepresentant}{Martin Bergman}{}{}
%\nv{Valberedningens ordförande}{Axel Voss}{E15}{\nafr{10b}}
%\nv{Fullmäktigeledamot}{Magnus Lundh}{E15}{\nafr{12}}
\nv{Skattmästare}{Daniel Bakic}{E15}{}
\nv{Vice Förvaltningschef}{Rasmus Sobel}{BME16}{}
\nv{Chefredaktör}{Emil Eriksson}{E18}{}
%\nv{Inspektor}{Monica Almqvist}{}{}

\end{narvarolista}

%\begin{comment}
\subsection*{Adjungerande}
\begin{narvarolista}
%\nv{post}{namn}{klass}{nati/nafr/tom}
%\nv{Likabehandlingsombud}{Jonna Fahrman}{BME17}{}
%\nv{Likabehandlingsombud}{Hanna Bengtsson}{BME18}{}
%\nv{Projekfunktionär}{Emma Hjörneby}{BME17}{}
%\nv{Macapär}{Filip Larsson}{E17}{}
%\nv{Kodhackare}{Vincent Palmer}{E18}{}
\nv{Sigillbevarare}{Matilda Horn}{BME18}{}
\end{narvarolista}
%\end{comment}

\section*{Protokoll}
\begin{paragrafer}
\p{1}{OFMÖ}{\bes}
Ordförande {\mo} förklarade mötet öppnat kl 12:12.

\p{2}{Val av mötesordförande}{\bes}
{\valavmo}

\p{3}{Val av mötessekreterare}{\bes}
{\valavms}

\p{4}{Val av justeringsperson}{\bes}
{\valavj}

\p{5}{Godkännande av tid och sätt}{\bes}
{\tosg}

\p{6}{Adjungeringar}{\bes}
%Adam Belfrage adjungerades.{}
%Hanna Bengtsson adjungerades. \\
%Jonna Fahrman adjungerades.
%Vincent Palmer adjungerades.\\
%Filip Larsson adjungerades. 
Matilda Horn adjungerades. 


%\textit{Inga adjungeringar.}


\p{7}{Godkännande av dagordningen}{\bes}

%Davida \ypa lägga till punkten ``Lophtet'' till dagordningen.\\
%Edvard \ypa lägga till punkten ``Ordensband'' til dagordningen.
%Fredrik \ypa att lägga till \S18b ``Teknikfokus utnyttjande av LED-café''.
%Jonathan \ypa ändra punkten §12 från att vara en beslutspunkt till diskussion. \\
%Föredragningslistan godkändes med yrkandet.
%Henrik \ypa lägga till punkten ``Faktura till F'' som §13.
%Jakob Pettersson \ypa tägga till punkten ''Øverphøs informerar'' som \S16.

Davida \ypa lägga till punkten ''Funktionärströjor'' som \S13.

Edvard \ypa lägga till punkten ''Miljökollegie'' som \S14.

Edvard \ypa lägga till punkten ''Testamentesskrivarkväll'' som \S15. 

Edvard \ypa lägga till punkten ''Arbetsordning till Valmötet'' som punkt \S16. 

Föredragningslistan godkändes med samtliga yrkanden.

\p{8}{Föregående mötesprotokoll}{\bes}
%\latillprotgodkand{S14/19 \& S15/19}
\textit{\ingaprot}

\p{9}{Fyllnadsval och entledigande av funktionärer}{\bes}
\begin{fyllnadsval} %"Inga fyllnadsval." fylls i automatiskt
%\fval{Moa Rönnlund}{Halvledare}
%\entl{Fanny Månefjord}{Husstyrelserepresentant från och med 30 juni}

\fval{Jacob Forsell}{Halvledare}


\end{fyllnadsval}

\p{10}{Rapporter}{}
\begin{paragrafer}
\subp{A}{Hur mår alla?}{\info}
%Punkten protokollfördes ej.
Mötet mår i överlag bra. Första dagen Henrik inte har varit i skolan sedan Chalmers. 

\subp{B}{Utskottsrapporter}{\info}
Edvard har ordnat det sistna till funktionärstacket och haft kollegiemöte. 

Jonathan och CM har sålt kaffe till Arkad. En ny Halvledare har tillsats. Denna veckan är första som CM inte längre saknar dioder. 
Idag ska CM baka lussekatter och mysa. 

Henrik och FVU har under den senaste veckan fått svar på frågor angående sektionens alkoholtillstånd samt sökt alkoholtillstånd till Julgillet. 
Det har även varit HTM där FVU stolt presenterade sina ekonomiska rapporter. Utöver det har sektionen hyrt ut köket till Arkad och dessutom fixat glasskivorna till vägglamporna, ett bord i Vega samt kollat på att köpa in nya soffor.

Mattias och InfU's verksamhet fortsätter som vanligt. Haft möte med Macapärerna och diskuterat IT. Utöver det har Mattias sysslat med styrelsearbete i form av handlingar och protokoll från sektionsmöten.  

Stephanie och Nolleutskottet fortsätter med sitt efterarbete och skriver fortfarande på sina testamenten. Idag ska NollU köpa mer färg och i veckan ska Stephanie på ØPK-möte.

Jakob och ENU har markadsfört en inspirationskväll på Knightec som äger rum ikväll. Utöver det har Jakob haft möte med näringslivskontakter och deras överlämningsarbete. 

Saga och Nöju har haft Ölresan och UtEDischotack. Ölresan var lyckad och badtunnorna på tacket var uppskattat. I veckan skall Entertainertrion planera Kick Out. Till sist är det spelkväll på torsdag.

Theo och Sexmästeriet har sittning ikväll. Theo har pratat med Lunds Akademiska Vinsällskap om att gästa oss den 10/12 på Nobelsittningen. Utöver det fortsätter vanligt jobb inför sittningar. 

Lina och SRE har haft sin vanliga verksamhet. Planerar en till pluggkväll den här läsperioden och de sista CEQ-mötena skall snart planeras. 

\subp{C}{Ekonomisk rapport}{\info}

Henrik har inte mycket att tilläga sedan Höstterminsmötet. Henrik försöker som vanligt jaga in kvitton. 

\subp{D}{Kåren informerar}{\info}

Martin informerade att det är Tävling i Fest nu i veckan och att Sångarstriden har börjat sälja biljetter. Märken och pins finns att köpa i Expen. 

Ivar informerade om att man kan söka Øverste om man vill bidra mer till nollningen samt att kårstyrelsen ska ha möte i Helsingborg nästa måndag. 

\end{paragrafer}

\p{11}{Medaljutdelning på funktionärstacket}{\dis}
Matilda presenterade svar från bidragsmedaljens nomineringsenkät.

Mötet diskuterade medaljutdelning. 

\p{12}{Ton på Höstterminsmötet}{\dis}

Edvard tog upp att det kommit in synpunkter på dålig ton under Höstterminsmötet. 

Mötet diskuterade varför det kan ha uppkommit samt vad man kan göra för att undvika det i framtiden. 

Ett förslag är att styrelsen ska vara mer tydlig hur man presenterar sig själv på mötet. Poängtera att man framför sin personliga åsikt för att undvika problem med maktbalans och grupperingar. 


%Edvard - till sektionsmöte så kommer många med olika approach. Man får göra det tydligt. Maktfördelning? 
%Alla kanske inte alla har läst igenom. Göra det tydligt att man ska läsa ingenom handlingarna, och presentera sin åsikt som medlem och styrelse. 

%Tas med till överlämningen. Fundera hur man ska presentera sin psikt ifrån. 


\p{13}{Funktionärströjor}{\dis}

Davida informerade att KMs jobbare tidigare köpt tröjor privat men att det egentligen är onödigt då de oftast aldrig används annat än till jobb. 

Davida vill att KM ska kunna köpa tillbaka tröjorna som KMs jobbare använt i år och använda detta som fasta arbetskläder. Både för att spara på funktionärers pengar men också miljön. 

Mötet tyckte det var en väldigt bra idé. 

\p{14}{Miljökollegiet}{\dis}

Edvard presenterade diskussionsunderlaget om Miljökollegie från Kåren. 

Edvard tyckte att mycket egentligen kan göras i befintliga kollegie.

Davida poängterade att miljötänk faller på alla utskott och ofta på låg nivå. Exempelvis bränsle, matinköp och annat. Mycket som måste göras på detaljnivå.

Edvard tror det är svårt att få in en funktionärspost som kan göra något konkret. Vill man driva detta är lättare att sitta som exempelvis Vice Ordförande. 

Saga tänkte att det kanske är lättare om de befintliga kollegierna tillsammans kan ta fram miljöplaner och främja miljöarbete. 

Daniel Bakic sa att det var samma sak som Sektionsgrodan tidigare år och tror att det är svårt att ha en specifik funktionärspost för just detta. Mer effektivt att ta det inom utskottet. 

Edvard informerade att det faktiskt står i styrelsens verksamhetsplanen att förbättra sektionens miljöarbete.

Davida tyckte det är bra att det står i verksamhetsplanen och lyfte idén om en workshop om miljöarbete. 

Saga undrade om eventuellt Kåren kunde anordna en workshop. 

\p{15}{Testamentesskrivarkväll}{\info}

Edvard informerade att kårens testamentesskrivarkväll den 26:e november krockar med sektionens Valmöte. Edvard har istället planerat att ta en annan kväll för en egen testamentesskrivningskväll och samtidigt bjuda på lättare tilltugg. 

Förslag på datum är måndagen 2:a december.

\p{16}{Arbetsordning till Valmötet}{\bes}

Edvard presenterade förslaget att lägga fram en arbetsordning till Valmötets handlingar. 

\Mba att lägga fram en arbetsordning till Valmötets handlingar. 

\p{17}{Nästa styrelsemöte}{\bes}
\Mba nästa styrelsemöte ska äga rum 2019-11-25 12.10 i E:1123.

\p{18}{Beslutsuppföljning}{\bes}
%Edvard \ypa stryka ''Projektfunktionär: Vårbal'' från Beslutsuppföljning. Liknande projekt uppmuntras.
%\Mbaby
%Davida \ypa skjuta upp ''Inköp av draghandtag till cykelvagn'' till nästa styrelsemöte.
%\Mbaby
Mattias \ypa stryka ''Inköp av video-adapter'' från Beslutsuppföljning.

\Mbaby

Saga Åslund \ypa skjuta upp ''Fotovägg - Diplomat'' till nästa styrelsemöte.

\Mbaby

\p{19}{Övrigt}{\dis}

Henke informerade att det är massa skräp och bös i Blå Dörren. Se till att plocka bort innan Henrik rensar. 

Saga sa att styrelsen borde boka ett datum till städning av buren så att det överlämnas snyggt till nästa år. 

Edvard informerade att det är städvecka för InfU och CM. 


\p{20}{Sammanfattning av mötet}{\info}
% \Fbs
% \Fbsup

Jakob Forsell valdes som Halvledare.

Mötet beslutade att lägga fram en arbetsordning till Valmötets handlingar.

''Inköp av video-adapter'' \textbf{ströks} från Beslutsuppföljning.

''Fotovägg - Diplomat'' från Beslutsuppföljning \textbf{sköts upp} tills nästa styrelsemöte. 


\p{21}{OFMA}{\bes}
{\mo} förklarade mötet avslutat kl. 13:05.
\end{paragrafer}

%\newpage
\hidesignfoot
\begin{signatures}{3}
\signature{\mo}{Mötesordförande}
\signature{\ms}{Mötessekreterare}
\signature{\ji}{Justerare}
\end{signatures}
\end{document}
