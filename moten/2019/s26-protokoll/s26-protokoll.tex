\documentclass[10pt]{article}
\usepackage[utf8]{inputenc}
\usepackage[swedish]{babel}

\def\mo{Edvard Carlsson}
\def\ms{Mattias Lundström}
\def\ji{Jonathan Benitez}
%\def\jii{}

\def\doctype{Protokoll} %ex. Kallelse, Handlingar, Protkoll
\def\mname{Styrelsemöte} %ex. styrelsemöte, Vårterminsmöte
\def\mnum{S26/19} %ex S02/16, E1/15, VT/13
\def\date{2019-11-18} %YYYY-MM-DD
\def\docauthor{\ms}

\usepackage{../e-mote}
\usepackage{../../../e-sek}

\begin{document}
\showsignfoot

\heading{{\doctype} för {\mname} {\mnum}}

%\naun{}{} %närvarane under
%\nati{} %närvarande till och med
%\nafr{} %närvarande från och med
\section*{Närvarande}
\subsection*{Styrelsen}
\begin{narvarolista}
\nv{Ordförande}{Edvard Carlsson}{E16}{}
\nv{Kontaktor}{Mattias Lundström}{E17}{}
\nv{Förvaltningschef}{Henrik Ramström}{E16}{}
\nv{Cafémästare}{Jonathan Benitez}{E17}{}
\nv{Sexmästare}{Theo Nyman}{BME18}{}
\nv{Krögare}{Davida Åström}{BME17}{}
\nv{Entertainer}{Saga Åslund}{BME18}{}
\nv{SRE-ordförande}{Lina Samnegård}{BME16}{}
\nv{ENU-ordförande}{Jakob Pettersson}{E17}{}
\nv{Øverphøs}{Stephanie Bol}{BME17}{}
\end{narvarolista}


\subsection*{Ständigt adjungerande}
\begin{narvarolista}
%\nv{}{}{}{}
%\nv{Skattmästare}{Daniel Bakic}{E15}{\nafr{10}}
%\nv{Vice Krögare}{Klara Indebetou}{BME17}{}
%\nv{Vice Krögare}{Hjalmar Tingberg}{BME16}{}
%\nv{Kårrepresentant}{Ivar Vänglund}{}{}
%\nv{Kårrepresentant}{Martin Bergman}{}{}
%\nv{Valberedningens ordförande}{Axel Voss}{E15}{\nafr{10b}}
%\nv{Fullmäktigeledamot}{Magnus Lundh}{E15}{\nafr{12}}
\nv{Chefredaktör}{Erik Eriksson}{E17}{}
%\nv{Inspektor}{Monica Almqvist}{}{}
\nv{}
Matilda, Rasmus, Ivar, Martin 
Bakic

\end{narvarolista}

%\begin{comment}
\subsection*{Adjungerande}
\begin{narvarolista}
%\nv{post}{namn}{klass}{nati/nafr/tom}
%\nv{Likabehandlingsombud}{Jonna Fahrman}{BME17}{}
%\nv{Likabehandlingsombud}{Hanna Bengtsson}{BME18}{}
%\nv{Projekfunktionär}{Emma Hjörneby}{BME17}{}
%\nv{Macapär}{Filip Larsson}{E17}{}
%\nv{Kodhackare}{Vincent Palmer}{E18}{}
\end{narvarolista}
%\end{comment}

\section*{Protokoll}
\begin{paragrafer}
\p{1}{OFMÖ}{\bes}
Ordförande {\mo} förklarade mötet öppnat kl 12.12

\p{2}{Val av mötesordförande}{\bes}
{\valavmo}

\p{3}{Val av mötessekreterare}{\bes}
{\valavms}

\p{4}{Val av justeringsperson}{\bes}
{\valavj}

\p{5}{Godkännande av tid och sätt}{\bes}
{\tosg}

\p{6}{Adjungeringar}{\bes}
%Adam Belfrage adjungerades.{}
%Hanna Bengtsson adjungerades. \\
%Jonna Fahrman adjungerades.
%Vincent Palmer adjungerades.\\
%Filip Larsson adjungerades. 
Matilda 


%\textit{Inga adjungeringar.}


\p{7}{Godkännande av dagordningen}{\bes}

%Davida \ypa lägga till punkten ``Lophtet'' till dagordningen.\\
%Edvard \ypa lägga till punkten ``Ordensband'' til dagordningen.
%Fredrik \ypa att lägga till \S18b ``Teknikfokus utnyttjande av LED-café''.
%Jonathan \ypa ändra punkten §12 från att vara en beslutspunkt till diskussion. \\
%Föredragningslistan godkändes med yrkandet.
%Henrik \ypa lägga till punkten ``Faktura till F'' som §13.
%Jakob Pettersson \ypa tägga till punkten ''Øverphøs informerar'' som \S16.

Davida ypa lägga till diskussionpunkt om att köpa tillbaka folks funktionärströjor som punkt 13.

Edvard ypa lägga till miljökollegiet som punkt 14.

Edvard ypa lägga till testamenteskrivarkväll  15. 

Edvard YPA arbetsordning till s-- punkt 16. 

Föredragningslistan godkändes med yrkanden.

\p{8}{Föregående mötesprotokoll}{\bes}
%\latillprotgodkand{S14/19 \& S15/19}
\textit{\ingaprot}

\p{9}{Fyllnadsval och entledigande av funktionärer}{\bes}
\begin{fyllnadsval} %"Inga fyllnadsval." fylls i automatiskt
%\fval{Moa Rönnlund}{Halvledare}
%\entl{Fanny Månefjord}{Husstyrelserepresentant från och med 30 juni}
Jacob Forsell - halvledare. 


\end{fyllnadsval}

\p{10}{Rapporter}{}
\begin{paragrafer}
\subp{A}{Hur mår alla?}{\info}
%Punkten protokollfördes ej.
Mötet mår i överlag bra. Första dagen Henrik inte har varit i skolan sedan Chalmers. 

\subp{B}{Utskottsrapporter}{\info}


\subp{C}{Ekonomisk rapport}{\info}

Henrik har inte mycket att tilläga sedan HTM- försöker vi in kvitton. 

\subp{D}{Kåren informerar}{\info}
Martin - Tävling i sittning nu i veckan. Sås har börjat sälja biljetter. Märken och pins finns att köpa i Expen. 
Ivar - man kan söka överste om man vill bidra mer till nollningen. Kårstyrelsen ska ha möte i helsingborg nästa måndag. 



\subp{E}{Omvärldsrapport}{\info}

..
\end{paragrafer}

\p{11}{Medaljutdelning på funktionärstacket}{\dis}
Fått svar på enkät för bidragsmedaljen. Matilda presenterade ett urval av svaren. 

Mötet diskuterade medaljutdelning. 

\p{12}{Ton på Höstterminsmötet}{\dis}

Edvard tog upp klagomål på stämningen på HTM. Dålig ton på propositionen angående kontaktor/vice. 

Mötet diskuterade .. Styrelsen som grupp kanske ska poängtera att man har en personlig åsikt. 

Edvard - till sektionsmöte så kommer många med olika approach. Man får göra det tydligt. Maktfördelning? 
Alla kanske inte alla har läst igenom. Göra det tydligt att man ska läsa ingenom handlingarna, och presentera sin åsikt som medlem och styrelse. 

Tas med till överlämningen. Fundera hur man ska presentera sin psikt ifrån. 


\p{13}{funktionärströjor}{\dis}

Davida -- km har köpt in tröjor privat till alla som jobbar. Onödigt att köpa in en KM tröja som egentligenn aldrig använts igen tillskillnad från utskottshoddies. 

Davida -- rimlgit att köpa tillbaka tröjorna som KMs jobbare använt i år och använda detta som arbetskläder.

Mötet tyckte det var en väldigt bra idé. 



\p{14}{miljökollegiet}{\dis}
Edvard tycket det kan göras i OK egentligen. Henrik tyckte det är lite onödigt-- kanske skyddsombud

-- Stephanie tycker det är väldigt viktigt. Jakob undrade om det diskuterats på OK.

Saga - hör till alla utskott i sektionen. Viktigt att de kommer och rapporterar. Viktigt att man har koll hela sektionen. 
Davida - miljötänk faller ner på alla utskott i låg nivå. Ex bränsle, matinköp osv. Måste göras på detaljnivå. Handlar mest om detaljfrågor.

Edvard - Om det kmr någon somm vill driva detta är det lättare att sitta som vice ordförande. Kommer ha svårt att få in en fuktionärspost som kan göra något konkret. 
Henrik - projektfunktionär som sätter fram en plan för alla utskott? tror det är svårt att ha kontinuitet. 
Saga - kanske lättare om de egna kollegierna kan tillsammans ta fram en miljöplan inom dess egna kollegie. 

Bakic - samma sak med sektionsgrodan tidigare år. Svårigheter att ha det funktionspostspecifk. Lättare att ta det inom utskott. 

Stephanie - Har som postbeskrivningen att ta upp. Edvard svarade att det faktiskt står i verksamhetsplanen. 

Davida - Bra att det står i verksamhetsplan -- kanske anordna en workshop? 
Saga . kanske kåren kan anordna worksohp??




\p{15}{testamentekväll}{\info}
Edvard -- Kåren har den 26 men det krockar med vårt valmöte. Vi tar en kväll och bjuder på lättare tilltugg och bjuder in alla som ska skriva testamente. 

Förslag på datum är måndag den 2 december


\p{16}{arbetsordning}{\bes}

\Mba 
Mötet vill lägga fram en arbetsprdning till handlingarna i valmöte.t 

\p{17}{Nästa styrelsemöte}{\bes}
\Mba nästa styrelsemöte ska äga rum 2019-11-25 12.10 i E:1123.

\p{18}{Beslutsuppföljning}{\bes}
%Edvard \ypa stryka ''Projektfunktionär: Vårbal'' från Beslutsuppföljning. Liknande projekt uppmuntras.
%\Mbaby
%Davida \ypa skjuta upp ''Inköp av draghandtag till cykelvagn'' till nästa styrelsemöte.
%\Mbaby
Mattias stryka. 

Saga - skjuter upp till nästa styrelsemöte. 

\p{19}{Övrigt}{\dis}
Spex i helgen på funktionärstacket. Löser det på en förfest. 

Henke - Skit i blå dörren, se till att ta bort. 

Saga - Borde boka datum och städa buren så vi lämnar det snyggt. 

Edvard - städvecka för InfU och CM. 


\p{20}{Sammanfattning av mötet}{\info}
% \Fbs
% \Fbsup


\p{21}{OFMA}{\bes}
{\mo} förklarade mötet avslutat kl. xxxx
\end{paragrafer}

%\newpage
\hidesignfoot
\begin{signatures}{3}
\signature{\mo}{Mötesordförande}
\signature{\ms}{Mötessekreterare}
\signature{\ji}{Justerare}
\end{signatures}
\end{document}
