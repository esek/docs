\documentclass[10pt]{article}
\usepackage[utf8]{inputenc}
\usepackage[swedish]{babel}

\def\mo{Edvard Carlsson}
\def\ms{Sonja Kenari}
\def\ji{Davida Åström}
%\def\jii{}

\def\doctype{Protokoll} %ex. Kallelse, Handlingar, Protkoll
\def\mname{Styrelsemöte} %ex. styrelsemöte, Vårterminsmöte
\def\mnum{S08/19} %ex S02/16, E1/15, VT/13
\def\date{2019-03-27} %YYYY-MM-DD
\def\docauthor{\ms}

\usepackage{../e-mote}
\usepackage{../../../e-sek}

\begin{document}
\showsignfoot

\heading{{\doctype} för {\mname} {\mnum}}

%\naun{}{} %närvarane under
%\nati{} %närvarande till och med
%\nafr{} %närvarande från och med
\section*{Närvarande}
\subsection*{Styrelsen}
\begin{narvarolista}
\nv{Ordförande}{Edvard Carlsson}{E16}{}
\nv{Kontaktor}{Sonja Kenari}{E15}{}
\nv{Förvaltningschef}{Henrik Ramström}{E16}{}
\nv{Cafémästare}{Jonathan Benitez}{E17}{}
\nv{Sexmästare}{Theo Nyman}{BME18}{}
\nv{Krögare}{Davida Åström}{BME17}{}
\nv{Entertainer}{Saga Åslund}{BME18}{\nafr{11}}
\nv{SRE-ordförande}{Lina Samnegård}{BME16}{\nafr{9}}
\nv{ENU-ordförande}{Jakob Pettersson}{E17}{}
\nv{Øverphøs}{Stephanie Bol}{BME17}{}
\end{narvarolista}


\subsection*{Ständigt adjungerande}
\begin{narvarolista}
%\nv{Sigillbevarare}{Matilda Horn}{BME18}{\nati{17}}
%\nv{}{}{}{}
%\nv{Kårrepresentant}{Jacob Karlsson}{}{\nafr{3}}
%\nv{Valberedningens ordförande}{Elin Magnusson}{}{}
%\nv{Skattmästare}{Daniel Bakic}{E15}{}
%\nv{Vice Krögare}{Klara Indebetou}{BME17}{}
%\nv{Vice Krögare}{Hjalmar Tingberg}{BME16}{}
\nv{Kårrepresentant}{Philip Johansson}{}{}
\nv{Kårrepresentant}{Anna Qvil}{}{}
%\nv{Fullmäktigeledamot}{Magnus Lundh}{E15}{\nafr{12}}
%\nv{Chefredaktör}{Max Mauritsson}{BME16}{}
%\nv{Elektras Ordförande}{Elisabeth Pongratz}{}{}
%\nv{Inspektor}{Monica Almqvist}{}{}
%\nv{Valberedningens ordförande}{Axel Voss}{E15}{\nafr{11}}

\end{narvarolista}

%\begin{comment}
\subsection*{Adjungerande}
\begin{narvarolista}
%\nv{post}{namn}{klass}{nati/nafr/tom}
\nv{Projektfunktionär}{Sophia Carlsson}{BME17}{}
\nv{Projekfunktionär}{Emma Hjörneby}{BME17}{}
%\nv{}{}{}{}
\end{narvarolista}
%\end{comment}

\section*{Protokoll}
\begin{paragrafer}
\p{1}{OFMÖ}{\bes}
Ordförande {\mo} förklarade mötet öppnat kl.12.10.

\p{2}{Val av mötesordförande}{\bes}
{\valavmo}

\p{3}{Val av mötessekreterare}{\bes}
{\valavms}

\p{4}{Val av justeringsperson}{\bes}
{\valavj}

\p{5}{Godkännande av tid och sätt}{\bes}
{\tosg}

\p{6}{Adjungeringar}{\bes}
%Adam Belfrage adjungerades.{}
Sophia Carlsson adjungerades.\\
Emma Hjörneby adjungerades.
%\textit{Inga adjungeringar.}


\p{7}{Godkännande av dagordningen}{\bes}
%Theo \ypa lägga till sena handlingar till dagordningen.
Davida \ypa lägga till punkten ``Lophtet'' till dagordningen.\\
Edvard \ypa lägga till punkten ``Ordensband'' til dagordningen.
%Fredrik \ypa att lägga till \S18b ``Teknikfokus utnyttjande av LED-café''.

%Föredragningslistan godkändes med yrkandet.
Föredragningslistan godkändes med samtliga yrkanden.

\p{8}{Föregående mötesprotokoll}{\bes}
\latillprot{S06/19 och S07/19}
%\textit{\ingaprot}

\p{9}{Fyllnadsval och entledigande av funktionärer}{\bes}
\begin{fyllnadsval} %"Inga fyllnadsval." fylls i automatiskt
\fval{Moa Rönnlund}{Halvledare}
\fval{Elin Johansson}{Halvledare}
\fval{Hannes Björk}{Halvledare}
\fval{Paulina Sager}{Diod}
\fval{Elin Andersson}{Diod}
\fval{Vincent Palmer}{Diod}
\fval{Casper Schwerin}{Diod}
\fval{Johanna Bengtsson}{Diod}
\fval{Malin Heyden}{Diod}
\fval{Amelie Bäck}{Diod}
\fval{Anton Jigsved}{Diod}
\fval{Hjalmar Tingberg}{Diod}
\fval{Filip Larsson}{Diod}
\fval{Theo Nyman}{Diod}
\fval{Martin Ollén}{Diod}
\entl{Johan Siwerson}{Näringslivskontakt}
%\entl{Namn}{Post}
\end{fyllnadsval}

\p{10}{Rapporter}{}
\begin{paragrafer}
\subp{A}{Hur mår alla?}{\info}
Punkten protokollfördes ej.

\subp{B}{Utskottsrapporter}{\info}
CM har haft det lugnt under tentaperioden. Har sett över lite datum för caféfesten. Massa nya dioder som hoppat in och förhoppningsvis får vi dioder varje vecka framöver.

FVU har hanterat bokföringskaos och haft lite möten. Har också sett över externa betalningarna från Gillen, verkar vara något som behöver utvärderas framöver. Henrik har också sökt lite tillstånd.

InfU rullar på. Lite olika ambitionsnivåer i utskottet. Har försökt utvärdera alla olika funktionärers uppfattning av sina poster hittills för att se vilka förändringar som behöver göras. 

KM haft Gille som en del av Kårens pubrunda. Fick lite tekniska problem under kvällen men det löstes med analoga verktyg! På fredag är det Gille igen!

NollU är inne i en tuff period just nu med mycket att göra. Större delen av phadderprocessen är klar. Finns några platser kvar som internationell phadder samt uppdragsphadder, anmälan för dessa är öppna till måndag. Förhandlingar för schemat är igång.

ENU har fått nya mailadresser och strukturerat om lite så att vi kontinuerligt ska få kontakt med olika företag. Fortsatt kontakt med företag som vi försöker skapa samarbete med. Det har också varit en del möten inför den kommande halvan av terminen.

NöjU har hållit i kickboxning under tentaperioden och haft lite paus i andra event. DÖMD-kick off har hållits vilket blev kul!

Sexet har kommit igång med planering för event under resten av året. Intersektionell sittning i maj är potentiellt med K. Har också hjälp D-sektionen i baren under ET-slasque. Det är även sittning i Edekvata den 6/4 som E6 håller i!

SRE har haft fler CEQ-möten. Försökt skapa posters till utskottet genom samarbete med Picasso.

\subp{C}{Ekonomisk rapport}{\info}
Vi går lite back just nu men vi inväntar fler inbetalningar, annars ser det bra ut.

\subp{D}{Kåren informerar}{\info}
På söndag är det valfullmäktige där majoriteten av heltidare och kårstyrelsen kommer att väljas. \\
Kåren är även på jakt efter en ny inspektor samt proinspektor. Nominera någon som du tror hade passat utmärkt för posten!\\
8/4 är det testamentesskrivarkväll i Cornelis där man kan komma och skriva på sina testamenten tillsammans med andra oavsett hur långt på sin mandatperiod man suttit.\\
JätteFunktionärsFilmkväll den 11/4, öppet för alla som är funktionär på en sektion eller Kåren!
Den 4/5 är det även JätteFunktionärsFesten.


\end{paragrafer}

\p{11}{Lophtet}{\info}
Davida informerar lite kring angående eftersläppsregler som kommit in från AktU. 

\p{12}{Ordensband}{\dis}
Henrik informerar hur de tänkte förra året när de köptes in. Styrelsen diskuterade därefter de olika ordensbanden vi har och hur vi ska göra med försäljningen. Just nu har vi ett tjockare nytt ordensband som inte överensstämmer med det tunnare färgmässigt och även en rosett. Styrelsen kom överens om att avvakta med att sälja de nya ordensbanden eftersom deras färg inte stämmer och undersöka saken tydligare angående inskaffning av bredare ordensband till Nollegasquen så att alla har enhetliga band i slutändan. Detta för att vi inte ska ha tre olika alternativ till ordensband (två olika band och en rosett) som skiljer sig åt så pass mycket, utan att man utifrån ordensbanden tydligt ska kunna se att alla tillhör samma sektion.

\p{15}{Nästa styrelsemöte}{\bes}
\Mba nästa styrelsemöte ska äga rum 2019-04-01 kl.12.10 i E:1123.

\p{16}{Beslutsuppföljning}{\bes}
\textit{Inga beslut att följa upp.}


\p{17}{Övrigt}{\dis}
Stephanie frågar om balgruppen inte ska lägga fram en motion för att göra projektet till ordentliga poster. Balgruppen svarar med att de först vill se hur balen går innan de skickar in något.\\
De som ville ha biljett till Konglig Elektrosektion på KTHs vårbal i styrelsen har fått en. Upp till de som ska åka att fixa biljetter dit.

\p{18}{OFMA}{\bes}
{\mo} förklarade mötet avslutat kl. 13.02.
\end{paragrafer}

%\newpage
\hidesignfoot
\begin{signatures}{3}
\signature{\mo}{Mötesordförande}
\signature{\ms}{Mötessekreterare}
\signature{\ji}{Justerare}
\end{signatures}
\end{document}
