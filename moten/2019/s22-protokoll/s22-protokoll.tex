\documentclass[10pt]{article}
\usepackage[utf8]{inputenc}
\usepackage[swedish]{babel}

\def\mo{Edvard Carlsson}
\def\ms{Mattias Lundström}
\def\ji{Theo Nyman}
%\def\jii{}

\def\doctype{Protokoll} %ex. Kallelse, Handlingar, Protkoll
\def\mname{Styrelsemöte} %ex. styrelsemöte, Vårterminsmöte
\def\mnum{S22/19} %ex S02/16, E1/15, VT/13
\def\date{2019-10-07} %YYYY-MM-DD
\def\docauthor{\ms}

\usepackage{../e-mote}
\usepackage{../../../e-sek}

\begin{document}
\showsignfoot

\heading{{\doctype} för {\mname} {\mnum}}

%\naun{}{} %närvarane under
%\nati{} %närvarande till och med
%\nafr{} %närvarande från och med
\section*{Närvarande}
\subsection*{Styrelsen}
\begin{narvarolista}
\nv{Ordförande}{Edvard Carlsson}{E16}{}
\nv{Kontaktor}{Mattias Lundström}{E17}{}
\nv{Förvaltningschef}{Henrik Ramström}{E16}{}
\nv{Cafémästare}{Jonathan Benitez}{E17}{\nafr{9}} 
\nv{Sexmästare}{Theo Nyman}{BME18}{}
\nv{Krögare}{Davida Åström}{BME17}{}
\nv{Entertainer}{Saga Åslund}{BME18}{}
\nv{SRE-ordförande}{Lina Samnegård}{BME16}{}
\nv{ENU-ordförande}{Jakob Pettersson}{E17}{}
\nv{Øverphøs}{Stephanie Bol}{BME17}{}
\end{narvarolista}


\subsection*{Ständigt adjungerande}
\begin{narvarolista}
%\nv{}{}{}{}
\nv{Valberedningens Ordförande}{Axel Voss}{E15}{}
\nv{Skattmästare}{Daniel Bakic}{E15}{}
%\nv{Vice Krögare}{Klara Indebetou}{BME17}{}
%\nv{Vice Krögare}{Hjalmar Tingberg}{BME16}{}
\nv{Chefredaktör}{Emil Eriksson}{E17}{}
\nv{Kårrepresentant}{Ivar Vänglund}{}{}
\nv{Kårrepresentant}{Martin Bergman}{}{}
%\nv{Valberedningens ordförande}{Axel Voss}{E15}{\nafr{10b}}
%\nv{Fullmäktigeledamot}{Magnus Lundh}{E15}{\nafr{12}}
%\nv{Inspektor}{Monica Almqvist}{}{}

\end{narvarolista}

%\begin{comment}
\subsection*{Adjungerande}
\begin{narvarolista}
%\nv{post}{namn}{klass}{nati/nafr/tom}
%\nv{Likabehandlingsombud}{Jonna Fahrman}{BME17}{}
%\nv{Likabehandlingsombud}{Hanna Bengtsson}{BME18}{}
%\nv{Projekfunktionär}{Emma Hjörneby}{BME17}{}
%\nv{Macapär}{Filip Larsson}{E17}{}
%\nv{Kodhackare}{Vincent Palmer}{E18}{}
\nv{Köksmästare}{Love Sjelvgren}{E18}{\nafr{9}}
\nv{Projektgrupp F1 Röj}{Erik Nord}{}{}
\nv{Projektgrupp F1 Röj}{Tom Andersson}{}{}
\nv{Husstyrelserepresentant}{Joakim Fredlund}{}{}

\end{narvarolista}
%\end{comment}

\section*{Protokoll}
\begin{paragrafer}
\p{1}{OFMÖ}{\bes}
Ordförande {\mo} förklarade mötet öppnat kl 12.12

\p{2}{Val av mötesordförande}{\bes}
{\valavmo}

\p{3}{Val av mötessekreterare}{\bes}
{\valavms}

\p{4}{Val av justeringsperson}{\bes}
{\valavj}

\p{5}{Godkännande av tid och sätt}{\bes}
{\tosg}

\p{6}{Adjungeringar}{\bes}
%Adam Belfrage adjungerades.{}
%Hanna Bengtsson adjungerades. \\
%Jonna Fahrman adjungerades.
%Vincent Palmer adjungerades.\\
%Filip Larsson adjungerades. 
Joakim Fredlund adjungerades.\\
Tom Andersson adjungerades.\\
Erik Nord adjungerades.\\
Love Sjelvgren adjungerades. 


%\textit{Inga adjungeringar.}


\p{7}{Godkännande av dagordningen}{\bes}

%Davida \ypa lägga till punkten ``Lophtet'' till dagordningen.\\
%Edvard \ypa lägga till punkten ``Ordensband'' til dagordningen.
%Fredrik \ypa att lägga till \S18b ``Teknikfokus utnyttjande av LED-café''.
%Jonathan \ypa ändra punkten §12 från att vara en beslutspunkt till diskussion. \\
%Föredragningslistan godkändes med yrkandet.
%Henrik \ypa lägga till punkten ``Faktura till F'' som §13.
%Jakob Pettersson \ypa lägga till punkten ''Øverphøs informerar'' som \S16.

Axel Voss \ypa behandla punkt \S15 ''Funktionärstacket'' före punk \S11 ''Äskning av pengar för inköp av nya ljusslingor''.

Föredragningslistan godkändes med yrkandet.

\p{8}{Föregående mötesprotokoll}{\bes}
%\latillprotgodkand{S14/19 \& S15/19}
\textit{\ingaprot}

\p{9}{Fyllnadsval och entledigande av funktionärer}{\bes}
\begin{fyllnadsval} %"Inga fyllnadsval." fylls i automatiskt
%\fval{Moa Rönnlund}{Halvledare}
%\entl{Fanny Månefjord}{Husstyrelserepresentant från och med 30 juni}

\fval{Molly Lilljebjörn Rusk}{Alumniansvarig BME}
\fval{Jakob Botvidsson}{Årskursansvarig BME-1}
\fval{Elina Dahlberg}{Årskursansvarig BME-1}
\fval{Andrea Cicovic}{Representant från de nyintagna}

Joakim kandiderade till Husstyrelserepresentant och presenterade sig själv. 

\fval{Joakim Magnusson Fredlund}{Husstyrelserepresentant}

\end{fyllnadsval}

Love Sjelvgren ankom till mötet.

Theo \ypa adjungera Love.

\Mbaby

\begin{fyllnadsval}

\fval{Love Sjelvgren}{Redaktör}

\end{fyllnadsval}


%Andrea, taggad på VB. Helt rätt person enl. Voss. 
%Joakim kandiderar til Husstyrelserepresentant. Joakim presenterar sig själv. Henke tycker det låter bra!

%Theo yrkar att riva adjungerar och lägga till Love. 

\p{10}{Rapporter}{}
\begin{paragrafer}
\subp{A}{Hur mår alla?}{\info}

Davida mår bra och hon känner att hon har varit på semester. Jonathan tyckte inte han var smart då han pluggade till 03 igår. 

Bakic har blivit döv på vänstra örat och han misstänker inflammation. 
I övrigt mår mötet bra. 

%Punkten protokollfördes ej.

\subp{B}{Utskottsrapporter}{\info}
 
Jonathan och CM trion känner sig utbrända efter att ha öppet hela nollningen och fram tills nu. Nu känner de att har förtjänat och behöver vila och stänger därmed LED nästa vecka. 
Troligtvis kommer det också vara en miljöinspektion den här veckan. Utöver det meddelade Jonathan att de har diskuterat möjligheten till att ha en teprovning nästa vecka i och med att de har stängt. Ska bli kul med ett eget litet event. 

Henrik rapporterade att den största arbetsbelastningen på FVU fortfarande är ekonomin som de fortfarande försöker få koll på efter nollningen.
Utöver det har FVU hållt i märkespickniquen som gick bra och Henrik har haft budgetmöte. De har också bytt lampor. 

Mattias meddelade att InfU fortsätter sin verksamhet med att sprida information och upperätthålla sektionens teknik. 
Fotograferna har redigerat massa bilder från nollningen och äntligen har Gasquebilderna också blivit klara. Mingelbilderna kommer senare idag. 
Haft ytterligare teknikstrul i veckan. I onsdags så gick Hacke och därmed alla våra hemsidor ner under natten, vilket antaglien orsakades av elavbrott igen. Musikservern är fortfarande nere sen förra veckan men bör vara en enkel återgärd. Har väntat tills allt elarbete är över. 

Davida meddelade att KM nu är tillbaka på banan och redo för nytt gille nu på fredag.   

Jakob och ENU har haft näringslivsmöte för näringslivskontakterna samt planerat och marknadsfört NABC-workshop med VentureLab som arrangeras ikväll. ENU har också planerat in en lunchföreläsning med Assa Abloy nästa tisdag ihop med F och D. 
Utöver det har de marknadsfört åt Axis och planerat pubkväll med BorgWarner nästa vecka. 

Stephanie meddelade att hon fortfarande har mycket att göra från nollningen. NollU vill ha en liten paus för att komma tillbaka till studierna efter en hård arbetsbelastning. 
Stephanie är också lite stressad över Phaddertacket på grund av tillstånd och budget. 

Saga och NöjU har haft premiär med Sporta med E, och det var en väldigt bra uppslutning med nästan 20 personer. Saga meddelade även att Sångarstriden blir av tack vare Linnea, Elina och Morgan!
Utöver det har Saga lunchmöte på torsdag där agendan är massa planering inför hösten. På fredag är det invigning av Biljard och då ska bandet spela. 

Theo meddelade att Sexiga fick chansen förra terminen att jobba på VGs novishfest som var i fredags. Allt gick bra och Theo är stolt över sina mästare och jobbare. Det har varit minimalt med problem som faktiskt påverkat sittningar under året och Sexmästeriet har fått många komplimanger efter Gasque och nu i fredags. 
Utöver det har mästarna tänkt på saker att ha med i sina kravprofiler. 

Lina meddelade att hon ska på kickoff med SRX. Utöver det fortsätter verksamheten som vanligt. 

Axel meddelade att Valberedningens process rullar på. De träffades igår och nu kan de dra igång ordentligt för höstens jobb. 
Valberedningen skall prata om hur de ska jobba internt och hur man kan förbereda sig för vissa orosmoment. Axel ska komma med återkoppling till styrelsen framöver. 
Utöver det har Valberedningen 'Valinfo' om höstens valmöte imorgon under lunchen i E:A. 

\subp{C}{Ekonomisk rapport}{\info}

Henrik rapporterar att ekonomi går bra. Sektionen saknar några stora inbetalningar i form av fakturor som ej betalts, men det är på gång.  


\subp{D}{Kåren informerar}{\info}

Martin informerade att F1 Röj har temasläpp nu på onsdag den 10 oktober. Han informerade också att det snart är Höstfullmäktige där det finns massa poster att söka.


Ivar informerade att man likt föregående vecka kan hämta sitt mecenatkort i kårens expedition om man saknar det. 

%Martin - fullmäktitemöte


\subp{E}{Omvärldsrapport}{\info}

Mattias rapporterade att styrelsen har fått anmodan till Uppsala Studentförening Elektroteknik's 5-årsjubileum.

Henrik tog upp att styrelsen, när de besökte Chalmers i Göteborg, diskuterade möjligheten kring att ha ett event liknande en konferens tillsammans med Chalmers och KTH i framtiden.

\end{paragrafer}

\p{11}{Funktionärstacket}{\dis}

Erik från F1 Röj meddelar att projektgruppen F1 Röj är oroliga att deras event ska lida av att E-sektionen har sitt Funktionärstack samma helg som F1 Röj. 

Davida meddelade att styrelsen har pratat om att istället anordna Funktionärstacket den 23 november för att undvika datumkrock. 

Axel Voss poängterade att det är möjligt att lägga Funktionärstacket nästa år, om man skulle vilja det. 

Mötet diskuterade möjliga datum och idéer till Funktionärstacket. 

Edvard föreslog att man börjar leta nation så tidigt som möjligt då dessa blir uppbokade. 

Edvard \ypa lägga till 'Funktionärstacket' på Beslutsuppföljning till S23/19 med Edvard som ansvarig. 

\Mbaby


\p{12}{Äskning av pengar för inköp av nya ljusslingor}{\bes}

Davida presenterade handlingen.

Henrik meddelade att kostnaden istället kan läggas på en utskottsbudget. 

Davida \ypa stryka handlingen. 

\Mbaby

\p{13}{Äskning av pengar för inköp av förvaringslådor till KM}{\bes}

Davida presenterade handlingen. 

Davida \ypa stryka handlingen då det inte behöver belasta dispositionsfonden.

\Mbaby


\p{14}{Äskning av pengar för inköp av tidningsställ i LED}{\bes}
Emil presentade handlingen. 

Mötet diskuterade placering av tidningsstället. 

Henrik \ypa datumet för beslutsuppföljningen istället sätts till S25/19 så att man hinner kolla med Huset om lämplig placering. 

\Mbabay

\p{15}{Äskning av pengar för tårta till sektionen}{\bes}
Edvard presenterade handlingen. 

Mötet tyckte det var en bra idé och ser fram emot att äta tårta.

Mötet diskuterade hur serveringen av tårta skall gå till och övrig tårtlogistik. 

\Mbabay


\p{16}{Nästa styrelsemöte}{\bes}
\Mba nästa styrelsemöte ska äga rum 2019-09-16 12.10 i E:1123.

\p{17}{Beslutsuppföljning}{\bes}
%Edvard \ypa stryka ''Projektfunktionär: Vårbal'' från Beslutsuppföljning. Liknande projekt uppmuntras.
%\Mbaby
%Davida \ypa skjuta upp ''Inköp av draghandtag till cykelvagn'' till nästa styrelsemöte.
%\Mbaby

Emil redovisade handlingen 'Fysiska exemplar av HeHE'. HeHE har trycks i fina exemplar och det har gått bra bortsett från lite bokföringsfel av tryckkostnaderna. 

Henrik gav förslag på hur bokföringsfelet skall lösas.  

%Emil flyttar det som gått över till HeHes budget. 

%stryker från uppföljingen - edvard.

Edvard \ypa att stryka ''Fysiska exemplar av HeHE'' från Beslutsuppföljning. 

\Mbaby

Saga påminde styrelsen att lägga in bilder i styrelsedriven innan nästa torsdag angående 'Fotovägg i Diplomat'.


\p{18}{Övrigt}{\dis}

%Edvard yrkar på att välja in love som redaktör.  - flytta uppp

%bifall. 

Joakim och Daniel meddelade om miljömärkningen Sektionsgrodan som handlar om att sektionen ska vara mer hållbar och miljövänlig. 

Edvard meddelade att det är städvecka för CM och KM. Jakob tycker att städlappen på HK är tydlig och bra. 

\p{19}{Sammanfattning av mötet}{\info}

Styrelsen har fått anmodan till 'Uppsala Studentförening Elektroteknik' 5-årsjubileum.


\Mba stryka ''Äskning av pengar för inköp av nya ljusslingor'' samt ''Äskning av pengar för inköp av förvaringslådor till KM'' från handlingarna. 

\Mba köpa in ett tidningsställ för HeHE. 

\Mba köpa 3 tårtor för att fira att E-sek Events har nått 1000 medlemmar. 

\Fbs ''Fysiska exemplar av HeHE''.

Mötet har diskuterat Funktionärstacket. 

\p{20}{OFMA}{\bes}
{\mo} förklarade mötet avslutat kl. 12.52
\end{paragrafer}

%\newpage
\hidesignfoot
\begin{signatures}{3}
\signature{\mo}{Mötesordförande}
\signature{\ms}{Mötessekreterare}
\signature{\ji}{Justerare}
\end{signatures}
\end{document}
