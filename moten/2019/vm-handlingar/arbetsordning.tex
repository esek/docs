\documentclass[2019-vm-handlingar.tex]{subfiles}

\begin{document}
     
    \subsection*{Instruktion}

    Detta är en arbetsordning. Den beskriver hur mötet går till med regler som ska följas av deltagarna. Arbetsordningen förklarar vad respektive deltagare har för rättigheter och skyldigheter under mötet. Till skillnad från dagordningen som säger vad som tas upp på mötet, säger arbetsordningen hur mötet går till. Styrelsen lägger fram ett förslag på arbetsordning som mötet beslutar om på plats. Arbetsordningen får inte gå emot sektionens stadga.

    \subsection*{1. Yttranderätt, yrkanderätt och rösträtt}

    \begin{itemize}
    \item Yttranderätt och yrkanderätt tillfaller alla medlemmar i sektionen samt av mötets adjungerade personer.
    \item Rösträtt tillfaller samtliga ordinarie medlemmar i sektionen.
    \item Begäran om ordet sker genom anmälan direkt till mötesordförande.
    \\ \textit{Talarlista förs av denna, och talarlistan följer i den ordning som deltagarna begärt om ordet.}
    \end{itemize}


    \subsection*{2. Votering, omröstning och val}
    
    \begin{itemize}
    \item Omröstning sker i första hand genom acklamation, och skulle någon därefter begära det, genom sluten votering med hjälp av evote.esek.se.   
    \item Enkel majoritet gäller vid omröstning där inte annat föreskrivs i stadgan eller Reglementet. 
    \item En får inte rösta för någon annan än sig själv. En får inte rösta med fullmakt.
    \item Vid personval gäller följande:
        
        \begin{itemize}
        \item Vid lika antal röster avgörs beslutet genom lottning.
        \item Befinner man sig inte i möteslokalen kommer man inte ha möjlighet att komma in förens pågående funktionäspost är färdigbehandlad och mötet har tagit beslut.  
        \item Vid utfrågningen ställs samma frågor till samtliga kandidater så länge de anses relevanta.
        \item Om en post vakantsätts ges styrelsen mandat att vid ett senare tillfälle tillsätta den vakantsatta posten i de fall då Stadgan eller Reglementet inte föreskriver annorlunda.  
         \end{itemize}
    \item Votering vid personval:
    \begin{itemize}
        \item En kandidat, en ska tillsättas: 
        \\ Görs med acklamation om inte sluten votering begärs. Om acklamationen ej blir enhällig blir det automatiskt sluten votering.
        \item Mindre eller lika många kandidater än vad som ska tillsättas, flera ska tillsättas: 
        \\ Kandidaterna väljes en och en enligt ovan. Med acklamation kan mötet bestämma att välja alla i klump.
        \item Flera kandidater, en ska tillsättas:
        \\ Görs med sluten votering. Om någon kandidat erhåller majoriteten av rösterna är personen vald. I annat fall stryks den kandidaten som erhållit minsta röstetal. Efter det upprepas proceduren med de kvarvarande kandidaterna enligt passande fall.
        \newpage
        \item Flera kandidater än vad som ska tillsättas, flera ska tillsättas:
        \\ Görs med sluten votering. Varje röstberättigad person får maximalt lika många röster som det finns platser kvar att tillsättas. Om någon kandidat erhåller majoriteten av rösterna är personen vald, och valproceduren upprepas med de kvarvarande kandidaterna enligt passande fall. I annat fall stryks den kandidaten som erhållit minsta röstetal. Efter det upprepas proceduren med de kvarvarande kandidaterna enligt passande fall.
       \end{itemize}

    \end{itemize}

    \subsection*{3. Tidsbegränsningar}

    \begin{itemize}
    \item Vid val av styrelseposter:
    5 minuter anförande följt av 10 minuter utfrågning. 
    \item Vid val av vice utskottsordförande:
    3 minuter anförande följt av 5 minuter utfrågning.
    \item Vid val av övriga funktionärer:
    2 minuter anförande följt av 2 minuter utfrågning
    \item Diskussionstid bestäms av mötet.
    \end{itemize}

    \subsection*{4. Debattregler}
    \begin{itemize}
    \item Replik kan begäras om en i talarstolen blivit apostroferad (nämnd vid namn) eller om det tydligt framgår att en blivit omtalad i talarstolen. 
    \\ \textit{Det är mötesordförandens uppgift att bevilja en replikbegäran. Replikskiften kan högst ske i en omgång.}
    \item Ordningsfråga och sakupplysning kan begäras när som helst och går alltid före i talarlistan. 
    \\ \textit{Ordningsfråga är till exempel begäran om streck i debatten.}
    \item En ska hålla god ton i talarstolen och använda ett vårdat språk. 
    \item Vid plädering ska det i huvudsak pläderas till kandidatens fördel. Undantag kan vara sakupplysningar kring misskötsel. 
    \\ \textit{Pläderingen är den avslutande diskussionen där medlemmar argumenterar för sina ståndpunkter innan votering.}
    \end{itemize}

    Styrelsen föreslår
     \begin{attsatser}
        \att mötet antar ovanstående som arbetsordning för valmötet 2019
    \end{attsatser}

    \begin{signatures}{1}
        \emph{I Sektionens tjänst}
        \signature{Edvard Carlsson}{Ordförande}
    
    \end{signatures}


    \end{document}