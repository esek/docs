\documentclass[10pt]{article}
\usepackage[utf8]{inputenc}
\usepackage[swedish]{babel}

\def\mo{Edvard Carlsson}
\def\ms{Sonja Kenari}
\def\ji{Jonathan Benitez}
%\def\jii{}

\def\doctype{Protokoll} %ex. Kallelse, Handlingar, Protkoll
\def\mname{Styrelsemöte} %ex. styrelsemöte, Vårterminsmöte
\def\mnum{S07/19} %ex S02/16, E1/15, VT/13
\def\date{2019-03-06} %YYYY-MM-DD
\def\docauthor{\ms}

\usepackage{../e-mote}
\usepackage{../../../e-sek}

\begin{document}
\showsignfoot

\heading{{\doctype} för {\mname} {\mnum}}

%\naun{}{} %närvarane under
%\nati{} %närvarande till och med
%\nafr{} %närvarande från och med
\section*{Närvarande}
\subsection*{Styrelsen}
\begin{narvarolista}
\nv{Ordförande}{Edvard Carlsson}{E16}{}
\nv{Kontaktor}{Sonja Kenari}{E15}{}
\nv{Förvaltningschef}{Henrik Ramström}{E16}{}
\nv{Cafémästare}{Jonathan Benitez}{E17}{}
%\nv{Sexmästare}{Theo Nyman}{BME18}{}
\nv{Krögare}{Davida Åström}{BME17}{}
\nv{Entertainer}{Saga Åslund}{BME18}{}
\nv{SRE-ordförande}{Lina Samnegård}{BME16}{}
\nv{ENU-ordförande}{Jakob Pettersson}{E17}{}
%\nv{Øverphøs}{Stephanie Bol}{BME17}{}
\end{narvarolista}


\subsection*{Ständigt adjungerande}
\begin{narvarolista}
\nv{Vice Sexmästare}{Elina Yrlid}{E18}{}

%\nv{Fritidsledare}{}{}{}
%\nv{Fritidsledare}{}{}{}
%\nv{}{}{}{}
%\nv{Kårrepresentant}{Jacob Karlsson}{}{\nafr{3}}
%\nv{Valberedningens ordförande}{Elin Magnusson}{}{}
%\nv{Skattmästare}{Daniel Bakic}{E15}{}
%\nv{Vice Krögare}{Klara Indebetou}{BME17}{}
%\nv{Vice Krögare}{Hjalmar Tingberg}{BME16}{}
\nv{Kårrepresentant}{Philip Johansson}{}{}
\nv{Kårrepresentant}{Anna Qvil}{}{}
%\nv{Fullmäktigeledamot}{Magnus Lundh}{E15}{\nafr{12}}
%\nv{Chefredaktör}{Max Mauritsson}{BME16}{}
%\nv{Elektras Ordförande}{Elisabeth Pongratz}{}{}
%\nv{Inspektor}{Monica Almqvist}{}{}
%\nv{Valberedningens ordförande}{Axel Voss}{E15}{\nafr{11}}

\end{narvarolista}

\subsection*{Adjungerande}
\begin{narvarolista}
\nv{Sigillbevarare}{Matilda Horn}{BME18}{}
\nv{Co-phøs}{Tove Börjesson}{E17}{}
\end{narvarolista}


\section*{Protokoll}
\begin{paragrafer}
\p{1}{OFMÖ}{\bes}
Ordförande {\mo} förklarade mötet öppnat kl.12.14.

\p{2}{Val av mötesordförande}{\bes}
{\valavmo}

\p{3}{Val av mötessekreterare}{\bes}
{\valavms}

\p{4}{Val av justeringsperson}{\bes}
{\valavj}

\p{5}{Godkännande av tid och sätt}{\bes}
{\tosg}

\p{6}{Adjungeringar}{\bes}
%Adam Belfrage adjungerades.{}
Matilda Horn adjungerades. \\
Tove Börjesson adjungerades. 
%Förnamn Efternamn adjungerades
%\textit{Inga adjungeringar.}


\p{7}{Godkännande av dagordningen}{\bes}
%Theo \ypa lägga till sena handlingar till dagordningen.
Sonja \ypa lägga till punkten §14 Medaljutgivning Vårbalen.

\Mbaby
%Fredrik \ypa att lägga till \S18b ``Teknikfokus utnyttjande av LED-café''.

%Föredragningslistan godkändes med yrkandet.
%Föredragningslistan godkändes med samtliga yrkanden.

\p{8}{Föregående mötesprotokoll}{\bes}
\latillprot{S05/19}
%\textit{\ingaprot}

\p{9}{Fyllnadsval och entledigande av funktionärer}{\bes}
\begin{fyllnadsval} %"Inga fyllnadsval." fylls i automatiskt
%\fval{}{}


%\entl{Namn}{Post}
\end{fyllnadsval}

\p{10}{Rapporter}{}
\begin{paragrafer}
\subp{A}{Hur mår alla?}{\info}
Punkten protokollfördes ej.

\subp{B}{Utskottsrapporter}{\info}
CM sålde semlor, gick bra! Det saknas lite dioder till nästa LP så skriv upp er på listan som Jonathan pingar i E-sek Events!

FVU rensar lokaler. Vänd er till Henrik om det är något ni har av värde i t.ex Blå Dörren som inte ska slängas. FVU har också  löst oklara fakturer och ligger i fas med betalningarna. 

InfU går bra. Aktiviteten inom utskottet rullar på, lite problem med licenser men det ska nog gå och lösa denna veckan.

KM har gille på fredag! Gått igenom med Cøl om vad som förväntas av posten. Restaurangrapport har även varit aktuellt i veckan.

NollU beställer ouvvar och ska börja med intervjuer nu på fredag. Utbildningar har genomförts för FHÖBen.

ENU har haft kickoff och uppstartsmöten inför Lunch med en Ingenjör som kommer påbörjas v.20. Spons inför nollning är i rullning och kollegiemötet har gått bra. Jakob försöker även se över förbättningspotentialen för utskottet.

NöjU taggar ``På styret'' på fredag och DÖMD biljetter är kirrade! Övriga eventen rullar på som vanligt som bl.a Bowlingturnering denna veckan och Boxning nästa vecka. 

Elina informerar om att E6 planerar sittningar och ska beställa utskottshoodies.

SRE har fått tag i pluggphaddrar till nollningen. Posters för posterna som likabehandling och världsmästare är något som utskottet jobbar på att få fram.


\subp{C}{Ekonomisk rapport}{\info}
Det går bra med ekonomin. Vi fick in en stor betalning i veckan så nu ligger vi väldigt mycket plus.

\subp{D}{Kåren informerar}{\info}
Man kan söka poster på Kåren! 

\subp{E}{Utomlands rapport}{\info}
Skriv till Sonja om ni vill gå på KTH balen 27/4, vi har 2 platser. 
\end{paragrafer}

\p{11}{Äskning av pengar för inköp av funktionärsmedaljer}{\bes}
Matilda berättar att vi behöver beställa fler medaljer till lagret. 

Edvard informerar om att budgeten för inköp av medaljer är lite underlig och att det är något som har diskuterats tidigare.
Styrelsen diskuterar om att man borde höja budgeten inför nästa år så att ordentligare beställningar kan göras framöver.

Edvard yrkar på att lägga beslutsuppföljningen S09/19.

Mötet beslutade att bifalla yrkandet med tilläggsyrkandet.

\p{12}{Äskning av pengar för inköp av diverse utrustning för DrEamHackE}{\dis}
Saga informerar att denna äskning inte längre är aktuell då undertecknade hellre vill skicka in större motion till VTM. Dreamhacke kommer heller inte äga rum förrän i höst och är därför inte ett väldigt brådskande ärande. 

\p{13}{Kvällmöte inför vårterminsmötet}{\dis}
Edvard informerar om att vi kör uppstartsmöte inför VTM onsdag kl. 17 den 13/3 samt eftermiddagen söndagen den 24/3. \\
Möten med propositioner och motioner blir under kvällsmöten 27/3 och 28/3. \\
Kvällsmöte angående föredragningslistan blir söndagen den 31/3 samt 1/4. \\
Tisdagen 2/1 blir det möte angående maten inför VTM.

Edvard lägger in alla datum i kalendern.

\p{14}{Medaljutdelning på Vårbalen}{\dis}
Styrelsen diskuterar möjligheterna att dela ut bidragsmedalj  på Vårbalen. 


\p{14}{Nästa styrelsemöte}{\bes}
\Mba nästa styrelsemöte ska äga rum 2019-03-27 kl.12.10 i E:1123.

\p{15}{Beslutsuppföljning}{\bes}
\textit{Inga beslut att följa upp.}


\p{16}{Övrigt}{\dis}
Jonathan undrar vem som tömmer panten vid LED, vilket alla antar hamnar på CM.

Val är öppet men kommer på upp Facebook så fort Vårterminsmötet är utlyst.

Glöm inte ta bort era grejer bakom Blå Dörren.

Lite funderingar kring Vårbalen diskuteras. 
\p{17}{OFMA}{\bes}
{\mo} förklarade mötet avslutat kl. 12.57.
\end{paragrafer}

%\newpage
\hidesignfoot
\begin{signatures}{3}
\signature{\mo}{Mötesordförande}
\signature{\ms}{Mötessekreterare}
\signature{\ji}{Justerare}
\end{signatures}
\end{document}
