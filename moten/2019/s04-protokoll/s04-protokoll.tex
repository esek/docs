\documentclass[10pt]{article}
\usepackage[utf8]{inputenc}
\usepackage[swedish]{babel}

\def\mo{Edvard Carlsson}
\def\ms{Sonja Kenari}
\def\ji{Stephanie Bol}
%\def\jii{}

\def\doctype{Protokoll} %ex. Kallelse, Handlingar, Protkoll
\def\mname{Styrelsemöte} %ex. styrelsemöte, Vårterminsmöte
\def\mnum{S04/19} %ex S02/16, E1/15, VT/13
\def\date{2019-02-11} %YYYY-MM-DD
\def\docauthor{\ms}

\usepackage{../e-mote}
\usepackage{../../../e-sek}

\begin{document}
\showsignfoot

\heading{{\doctype} för {\mname} {\mnum}}

%\naun{}{} %närvarane under
%\nati{} %närvarande till och med
%\nafr{} %närvarande från och med
\section*{Närvarande}
\subsection*{Styrelsen}
\begin{narvarolista}
\nv{Ordförande}{Edvard Carlsson}{E16}{}
\nv{Kontaktor}{Sonja Kenari}{E15}{}
\nv{Förvaltningschef}{Henrik Ramström}{E16}{}
\nv{Cafémästare}{Jonathan Benitez}{E17}{}
\nv{Sexmästare}{Theo Nyman}{BME18}{}
\nv{Krögare}{Davida Åström}{BME17}{}
\nv{Entertainer}{Saga Åslund}{BME18}{}
\nv{SRE-ordförande}{Lina Samnegård}{BME16}{}
\nv{ENU-ordförande}{Jakob Pettersson}{E17}{}
\nv{Øverphøs}{Stephanie Bol}{BME17}{}
\end{narvarolista}


\subsection*{Ständigt adjungerande}
\begin{narvarolista}
%\nv{Sigillbevarare}{Matilda Horn}{BME18}{\nati{17}}
%\nv{}{}{}{}
%\nv{Kårrepresentant}{Jacob Karlsson}{}{\nafr{3}}
%\nv{Valberedningens ordförande}{Elin Magnusson}{}{}
%\nv{Skattmästare}{Daniel Bakic}{E15}{}
%\nv{Vice Krögare}{Klara Indebetou}{BME17}{}
%\nv{Vice Krögare}{Hjalmar Tingberg}{BME16}{}
%\nv{Kårrepresentant}{Philip Johansson}{}{}
\nv{Kårrepresentant}{Anna Qvil}{}{}
\nv{Chefredaktör}{Max Mauritsson}{BME16}{}
%\nv{Elektras Ordförande}{Elisabeth Pongratz}{}{}
%\nv{Inspektor}{Monica Almqvist}{}{}
%\nv{Valberedningens ordförande}{Axel Voss}{E15}{\nafr{11}}

\end{narvarolista}

%\begin{comment}
%\subsection*{Adjungerande}
%\begin{narvarolista}
%\nv{}{Alexander Wik}{BME17}{\nati{17}}
%\nv{Sigillbevarare}{Matilda Horn}{BME18}{}
%\nv{Sångförman}{Adam Belfrage}{BME17}{\nati{17}}
%\end{narvarolista}
%\end{comment}

\section*{Protokoll}
\begin{paragrafer}
\p{1}{OFMÖ}{\bes}
Ordförande {\mo} förklarade mötet öppnat kl.12.16.

\p{2}{Val av mötesordförande}{\bes}
{\valavmo}

\p{3}{Val av mötessekreterare}{\bes}
{\valavms}

\p{4}{Val av justeringsperson}{\bes}
{\valavj}

\p{5}{Godkännande av tid och sätt}{\bes}
{\tosg}

\p{6}{Adjungeringar}{\bes}
%Adam Belfrage adjungerades.{}
%Förnamn Efternamn adjungerades
\textit{Inga adjungeringar.}


\p{7}{Godkännande av dagordningen}{\bes}
Tid och sätt godkändes.
%Theo \ypa lägga till sena handlingar till dagordningen.


%Fredrik \ypa att lägga till \S18b ``Teknikfokus utnyttjande av LED-café''.

%Föredragningslistan godkändes med yrkandet.
%Föredragningslistan godkändes med samtliga yrkanden.

\p{8}{Föregående mötesprotokoll}{\bes}
\latillprot{S03/19}
%\ingaprot

\p{9}{Fyllnadsval och entledigande av funktionärer}{\bes}
\begin{fyllnadsval} %"Inga fyllnadsval." fylls i automatiskt
%\fval{Jonna Fahrman}{Skyddsombud med likabehandlingsansvar}


%\entl{Namn}{Post}
\end{fyllnadsval}

\p{10}{Rapporter}{}
\begin{paragrafer}
\subp{A}{Hur mår alla?}{\info}
Punkten protokollfördes ej.

\subp{B}{Utskottsrapporter}{\info}
Det går bra för InfU, mycket igång och som behöver fixas. 

NöjU ska hålla braziliansk kampssportsdans under ``Sporta på E''. Städa Billiard inför biljardturneringen står på schemat.

Sexmästaren har varit på kollegieskiphte. Gasquesittning nu på Tisdag för Teknikfokus som förhoppningsvis kommer gå bra. Theo hoppas att allt kommer gå bra.

KM planerar Gillen denna veckan och ser över PR möjligheter. Har också varit på skiphte och haft kul.

ENU har haft kontakt med Partykungen och håller en dialog. Håller möte angående planering inför FED pub. CV fotografering denna veckan.

FVU har haft kickoff, och bokfört så klart! Kontaktat V-sektionen angående gemensamt bokbål med gamla bokföringar.

CM har haft många dioder denna vecka så det är jättekul! Massa möten denna veckan och planering inför teprovning den 8/3! Kärleksmumbakning planeras inför alla hjärtansdag.

NollU har haft kollegieskiphte, känns bra efter all utbildning och information. Infomöte på torsdag den 14/2 för phaddrar. 

SRE har börjat cencurera CEQ. Har börjat planera en kickoff.



\subp{C}{Ekonomisk rapport}{\info}
Det ser bra ut.

\subp{D}{Kåren informerar}{\info}
Finns fortfarande lediga poster på Kåren. Finns massa kul grejer att välja mellan!

Påminnelse om styrelseutbildning den 19/2.
\end{paragrafer}

\p{11}{Äskning av pengar för att trycka upp fysiska exemplar av HeHE}{\bes}
Chefredaktör Max berättar att det hade varit kul med fysiska exemplar av HeHE. 

Tanken är att göra en upplaga i månaden men att välja 3 extra roliga utgåvor att printa som fysiska upplagor.

\Mbaby


\p{12}{Datum för Vårterminsmötet}{\bes}
Edvard \ypa vi preliminär bokar den 9e april för vårterminsmötet.

\Mbaby

\p{13}{Nästa styrelsemöte}{\bes}
\Mba nästa styrelsemöte ska äga rum 2019-02-20 kl.12.10 i E:1123.

\p{14}{Beslutsuppföljning}{\bes}
Theo \ypa stryka ``Inköp av mikrofon'' från beslutsuppföljningen.

\Mbaby

\p{15}{Övrigt}{\dis}
Styrelsen diskuterar när KPL ska planeras.

Henrik tar upp att vi borde kicka igång möte med alla funktionärer angående rättigheter och skyldigheter. Styrelsen diskuterar olika möjligheter.

Mötet diskuterar om eventuell arbetsglädje till E-tryckeriet som alltid ställer upp för sektionen. Edvard är ansvarig för att planen sätts i verk.

\p{16}{OFMA}{\bes}
{\mo} förklarade mötet avslutat kl. 12.44. 
\end{paragrafer}

%\newpage
\hidesignfoot
\begin{signatures}{3}
\signature{\mo}{Mötesordförande}
\signature{\ms}{Mötessekreterare}
\signature{\ji}{Justerare}
\end{signatures}
\end{document}
