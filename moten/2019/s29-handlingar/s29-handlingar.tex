\documentclass[10pt]{article}
    \usepackage[utf8]{inputenc}
    \usepackage[swedish]{babel}
    
    \def\doctype{Handlingar} %ex. Kallelse, Handlingar, Protkoll
    \def\mname{Styrelsemöte} %ex. styrelsemöte, Vårterminsmöte
    \def\mnum{S29/19} %ex S02/16, E1/15, VT/13
    \def\date{2019-12-09} %YYYY-MM-DD
    \def\docauthor{Edvard Carlsson}
    
    \usepackage{../e-mote}
    \usepackage{../../../e-sek}
    
    \begin{document}
    
    \heading{{\doctype} till {\mname} {\mnum}}
    

    \section*{Äskning av pengar för inköp av symaskin}
    
 	Sektionens symaskin fungerar mycket dåligt. Detta skapar problem för utskott som är i stort behov av just symaskiner. Exempelvis har NollU de senaste åren blivit tvungna att ta med maskiner hemifrån eller låna från familj och andra sektioner. Jag tycker att sektionen har ett tillräckligt stort behov av en fungerande symaskin och att nyttan den hade tillfört gör det värt att köpa in en ny. Förslagsvis den klassiska och nybörjarvänliga modellen “Husqvarna Viking” (\href{https://www.symaskinskungen.se/Symaskiner/HUSQVARNA-VIKING-SYMASKINER/Mekaniska-symaskiner/Symaskin-Husqvarna-viking-H-Class-E20.htm?sClickID=GoogleShopping-124&gclid=Cj0KCQjw9fntBRCGARIsAGjFq5EboAI093hS8rlKYf87E6Z-8H_8kmKeCPqz9YrwMDL9QLbpHydsQzcaAvYvEALw_wcB}{\textit{länk}}). 

    Jag yrkar
    \begin{attsatser}
        \att köpa in en symaskin, 
        \att avsätta en budget på \SI{3000}{kr} till ändamålet där kostnaden belastar utrustningsfonden, samt
        \att detta läggs på beslutsuppföljningen till S1/20 med undertecknad som ansvarig. 
    \end{attsatser}

    \begin{signatures}{1}
    \textit{\ist}
    \signature{Edvard Carlsson}{Ordförande}
    \end{signatures}



    \newpage

    \section*{Diskussion kring sammanslagning av Elektroteknik och Medicin och teknik på Chalmers}
    
 	Jag har fått följande meddelande med tillhörande frågor av Chalmers Studentkår som ber om input från oss angående uppstartandet av deras nya program:

    ''Jag skriver till er på uppdrag från Chalmers Studentkår. Nästa verksamhetsår kommer det nya grundutbildningsprogrammet Medicinteknik startas och ha sin fösta antagning till hösten 2020. Kårens roll i detta kommer vara att försäkra de nya studenternas möjlighet till sektionsengagemang. Den aktuella planen är att integrera det nya programmets studenter i Elektroteknologsektionen precis som ni och KTH har gjort med era liknande program. Vi söker nu input från er hur det har funkat både historiskt och i dagsläget med denna sammanslagning. Om du har möjlighet att svara på frågorna nedan hade vi i kårledningen och de som ska ansöka till det nya programmet varit evigt tacksamma.''

    \begin{dashlist}
        \item Har ni upplevt några problem med segregation eller liknande inom sektionen pga att era medlemmar läser två olika program?
        \item Har ni tillgång till sektions- och festlokaler där båda programmen får plats och känner sig välkomna?
        \item Upplever ni några svårigheter gällande kapacitet eller engagemang?
    \end{dashlist}

    \begin{signatures}{1}
    \textit{\ist}
    \signature{Edvard Carlsson}{Ordförande}
    \end{signatures}




   
    \end{document}
    