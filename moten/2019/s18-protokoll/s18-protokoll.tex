\documentclass[10pt]{article}
\usepackage[utf8]{inputenc}
\usepackage[swedish]{babel}

\def\mo{Edvard Carlsson}
\def\ms{Mattias Lundström}
\def\ji{Jakob Pettersson}
%\def\jii{}

\def\doctype{Protokoll} %ex. Kallelse, Handlingar, Protkoll
\def\mname{Styrelsemöte} %ex. styrelsemöte, Vårterminsmöte
\def\mnum{S18/19} %ex S02/16, E1/15, VT/13
\def\date{2019-09-09} %YYYY-MM-DD
\def\docauthor{\ms}

\usepackage{../e-mote}
\usepackage{../../../e-sek}

\begin{document}
\showsignfoot

\heading{{\doctype} för {\mname} {\mnum}}

%\naun{}{} %närvarane under
%\nati{} %närvarande till och med
%\nafr{} %närvarande från och med
\section*{Närvarande}
\subsection*{Styrelsen}
\begin{narvarolista}
\nv{Ordförande}{Edvard Carlsson}{E16}{}
\nv{Kontaktor}{Mattias Lundström}{E17}{}
\nv{Förvaltningschef}{Henrik Ramström}{E16}{}
\nv{Cafémästare}{Jonathan Benitez}{E17}{}
\nv{Sexmästare}{Theo Nyman}{BME18}{}
\nv{Krögare}{Davida Åström}{BME17}{}
\nv{Entertainer}{Saga Åslund}{BME18}{}
\nv{SRE-ordförande}{Lina Samnegård}{BME16}{}
\nv{ENU-ordförande}{Jakob Pettersson}{E17}{}
\nv{Øverphøs}{Stephanie Bol}{BME17}{}
\end{narvarolista}


\subsection*{Ständigt adjungerande}
\begin{narvarolista}
%\nv{}{}{}{}
%\nv{Skattmästare}{Daniel Bakic}{E15}{\nafr{10}}
%\nv{Vice Krögare}{Klara Indebetou}{BME17}{}
%\nv{Vice Krögare}{Hjalmar Tingberg}{BME16}{}
%\nv{Kårrepresentant}{Filip Johansson}{}{\nafr{10A}}
\nv{Kårrepresentant}{Martin Bergman}{}{}

%\nv{Valberedningens ordförande}{Axel Voss}{E15}{\nafr{10b}}
%\nv{Fullmäktigeledamot}{Magnus Lundh}{E15}{\nafr{12}}
%\nv{Chefredaktör}{Max Mauritsson}{BME16}{}
%\nv{Inspektor}{Monica Almqvist}{}{}


\end{narvarolista}

%\begin{comment}
\subsection*{Adjungerande}
\begin{narvarolista}
%\nv{post}{namn}{klass}{nati/nafr/tom}
%\nv{Likabehandlingsombud}{Jonna Fahrman}{BME17}{}
%\nv{Likabehandlingsombud}{Hanna Bengtsson}{BME18}{}
%\nv{Projekfunktionär}{Emma Hjörneby}{BME17}{}
%\nv{Macapär}{Filip Larsson}{E17}{}
%\nv{Kodhackare}{Vincent Palmer}{E18}{}
\end{narvarolista}
%\end{comment}

\section*{Protokoll}
\begin{paragrafer}
\p{1}{OFMÖ}{\bes}
Ordförande {\mo} förklarade mötet öppnat kl 12.10

\p{2}{Val av mötesordförande}{\bes}
{\valavmo}

\p{3}{Val av mötessekreterare}{\bes}
{\valavms}

\p{4}{Val av justeringsperson}{\bes}
{\valavj}

\p{5}{Godkännande av tid och sätt}{\bes}
{\tosg}

\p{6}{Adjungeringar}{\bes}
%Adam Belfrage adjungerades.{}
%Hanna Bengtsson adjungerades. \\
%Jonna Fahrman adjungerades.
%Vincent Palmer adjungerades.\\
%Filip Larsson adjungerades. 

%\textit{Inga adjungeringar.}


\p{7}{Godkännande av dagordningen}{\bes}

%Davida \ypa lägga till punkten ``Lophtet'' till dagordningen.\\
%Edvard \ypa lägga till punkten ``Ordensband'' til dagordningen.
%Fredrik \ypa att lägga till \S18b ``Teknikfokus utnyttjande av LED-café''.
%Jonathan \ypa ändra punkten §12 från att vara en beslutspunkt till diskussion. \\
%Föredragningslistan godkändes med yrkandet.
%Henrik \ypa lägga till punkten ``Faktura till F'' som §13.
%Jakob Pettersson \ypa tägga till punkten ''Øverphøs informerar'' som \S16.
Edvard \ypa lägga till punkten ''Äskning av pengar till inköp av mobil router'' från sena handlingar som \S14.

Edvard \ypa lägga till punkten ''Äskning av pengar för inköp av ljusutrustning'' från sena handlingar som \S15.

Mattias \ypa ändra punkten \S19 från att vara en diskussion till en informationspunkt. 

Föredragningslistan godkändes med yrkandet.

\p{8}{Föregående mötesprotokoll}{\bes}
\latillprotgodkand{S15/19 \& S16/19 \& S17/19}

%\textit{\ingaprot}

\p{9}{Fyllnadsval och entledigande av funktionärer}{\bes}
\begin{fyllnadsval} %"Inga fyllnadsval." fylls i automatiskt
%\fval{Moa Rönnlund}{Halvledare}
%\entl{Fanny Månefjord}{Husstyrelserepresentant från och med 30 juni}
\fval{Anton Jigsved}{Diod}
\fval{Hjalmar Tingberg}{Diod}
\fval{Andreas Bennström}{Diod}
\fval{Malin Heyden}{Diod}
\fval{Johanna Bengtsson}{Diod}
\fval{Amelie Bäck}{Diod}
\fval{Johannes Larsson}{Diod}
\fval{Sophia Calsson}{Diod}
\fval{Emma Hjörneby}{Diod}
\fval{Fabian Sondh}{Diod}
\fval{Filip Larsson}{Diod}
\fval{Johan Halldin}{Diod}
\fval{Cassandra Tran}{Halvledare}
\entl{Johan Vikstrand}{Källarmästare}
\entl{Hannes Björk}{Källarmästare}
\entl{Gabriela Medina}{Källarmästare}
\entl{Mia Cicovic}{Källarmästare}


\end{fyllnadsval}

\p{10}{Rapporter}{}
\begin{paragrafer}
\subp{A}{Hur mår alla?}{\info}
%Punkten protokollfördes ej.
Hälsan varierar hos styrelsen men humöret är trots allt bra.

\subp{B}{Utskottsrapporter}{\info}

Edvard meddelade att OK har presenterat sina nollningsoutfits och att temat är OKtoberfest. Han har även haft möte angående situationen med tilläggsmeriter och deltagit på ledarskapsutbildning.

Jonathan och CM fortsätter verksamheten som vanligt. Phaddrar och dioder har stått i LED. Jonathan även har tagit emot leveranser och rensat avlopp. Det har varit lite problem med access under morgonleveranser från Mormors bageri men den situationen är löst nu. Diodschemat för veckan ser bättre ut än det gjorde i fjol. 

Henrik och FVU har haft fullt upp med ekonomisk hjälp och bokföring. De har även lyckats få upp de nya mikrovågsugnarna vilket helt tycks ha eliminerat lunchkön. De har även hyrt ut våra grillar och planerat inför budgeteringen nästa år.  

Mattias har haft sitt första stormöte med hela InfU som bestod av statusuppdateringar från utskottets interna delar och en övergripande planering kring den fortsatta verksamheten. Utöver det är redaktionen igång och skriver artiklar, Fotograferna har lyckats fota varje event, Teknokraterna har riggat massa teknik och Picasso är redo för designbeställnignar. 

Davida och KM har haft planeringsmöte om kommande evenemang. Det är pubrunda på fredag och Regatta på söndag. De är i fas med planering och bokföring och är redo för en intensiv helg. 

Jakob har haft workshop med näringslivsgruppen där de gick igenom planering av nollningen och resten av hösten. Jakob har bokat mat, ordnat anmälan och marknadsföring av lunchföreläsning med BorgWarner. Alumniverksamheten har undersökt hur man på bästa sätt ska anordna inplanerade events. Utöver det har det fakturerats.  

Saga och NöjU, förutom den mer musikbegåvade delen, har haft en lugn vecka och vilat upp sig från det intensiva UtEDischot. Nästa event på schemat är MegaKulfEstivalen vilket alla är taggade på. Utöver det ska utskottet ha planeringsmöte. Saga meddelar också att stridsrop är oroliga att Sångarstriden inte skall bli av. 

Theo och Sexet har gått på sittning och haft en rolig vecka. De har planerat ett preliminärt tidsschemat för gasque och till helgen är det sittning med A och K. 

Lina och SRE ska ha sin första pluggkväll. Det är fler anmälda än räknat och jobbare har hoppat av, men de är tror att det skall gå bra ändå. Utöver det håller likabehandlingsombuden på med ett 'ensamhetsprojekt' tillsammans med D-sektionen och Lina har även planerat in CEQ-möten. 


\subp{C}{Ekonomisk rapport}{\info}
Henrik meddelar att ekonomin fortsatt går bra och att han har börjat titta på budgeteringen inför nästa år. Henrik har bett samtliga utskottsordföranden att se över sina internbudgetar och göra en lätt utvärdering av hur budgeteringen varit i år. 

Henrik påminde även att forsätta få in kvitton och underlag för bokföring så fort som möjligt. 

\subp{D}{Kåren informerar}{\info}

Kåren meddelade att kallelse till första fullmäktigemötet har kommit. Kåren informerade även att det fortfarande saknas projektgrupper till Sångarstriden och Nyårsbalen. Deadline för detta är den 30 september. 

\subp{E}{Omvärldsrapport}{\info}
Mattias informerade att vi tackat nej till inbjudan för KTHs nollegasque på grund av att ingen i styrelsen kan närvara. 

\end{paragrafer}

\p{11}{Øverphøs informerar}{\info}

Stephanie informerade om en incident under FlyING där en nolla fått ett astmaanfall och saknade mediciner. Situationen löste sig men Stephanie tog upp frågan om inköp av astmamediciner till sektionen. Det konstanterades även att vårt förstahjälpen-kit behöver uppdateras med diverse material. 
 
Biljettförsäljningen till FlyING var också problematiskt och Stephanie rekommenderar att Bonzai ska användas för detta nästs år. Alla bussarna fylldes ej, delvis på grund av strul med anmälningslänkar. Utöver det är Stephanie tacksam över styrelsens hjälp under paraden i Helsingborg. 

Stephanie informerade om NollEfredagen och att NollU återigen behöver hjälp av styrelsen med att ta sig till Sankt Hans i säker och kontrollerad form.

\p{12}{Biljard}{\dis}
Mötet diskuterade handlingen.

Henrik påminde mötet om vad som diskuterats på senaste styrelsemötet angående skötsel av våra lokaler. 

Davida påminde om de gemensamma städveckorna. 

Saga tyckte att sektionen borde ha en invigning av nya Biljard. Mötet tyckte det lät som en bra idé och att det motiverar liknande projekt i framtiden. 

\p{13}{Äskning av pengar för inköp av nya iZettle läsare}{\bes}
Henkrik presenterade handlingen.

Mötet diskuterade handlingen.

\Mbabay

\p{14}{Äskning av pengar till inköp av mobil router}{\bes}
Davida presenterade den sena handlingen.

Henrik ställde frågan om hur mobildata skall budgeteras. Mötet diskuterade detta. 

\Mbabay

\p{15}{Äskning av pengar för inköp av ljusutrustning}{\bes}

Edvard presenterade den sena handlingen.

Henrik föreslog att framföra handlingen på ett sektionsmöte istället. Jakob instämde och menade att ljusfrågan inte är tillräckligt akut. 

Mötet undrade hur lätt det är att rigga ljusanordningen. Edvard svarade att det är mycket enkelt och att teknokraterna tänkt på detta när de skapade inköpslistan. 

Davida undrade om det finns möjlighet att hyra detta av kåren. Hon tror inte vi regelbundet kommer rigga ljus på gillen.

Mattias tyckte idén med nytt ljus var bra och instämde att det hade lyft tillvaron men instämde också att handlingen är mer lämplig på ett sektionsmöte. 

Stephanie var positiv till nytt ljus och vill få bra ljus utomhus på enkelt sätt.

Henrik sa att sektionen redan har en rökmaskin. Oklart i vilket skick.  

Mötet diskuterade den sena handlingen. 

\Mbaay

\p{16}{Nästa styrelsemöte}{\bes}
\Mba nästa styrelsemöte ska äga rum 2019-09-16 kl 12.10 i E:3316.

\p{17}{Beslutsuppföljning}{\bes}
%Edvard \ypa stryka ''Projektfunktionär: Vårbal'' från Beslutsuppföljning. Liknande projekt uppmuntras.
%\Mbaby
%Davida \ypa skjuta upp ''Inköp av draghandtag till cykelvagn'' till nästa styrelsemöte.
%\Mbaby

Stephanie \ypa stryka ''Inköp av cheerdräkter'' från Beslutsuppföljningen.

\Mbaby

\p{18}{Övrigt}{\dis}

Saga meddelade att F-sektionen vill låna sektionens soundbox batteri en kort stund då deras är sönder. 

Davida påminde om den gemensamma städveckan och att styrelsen bör bestämma datum för nytt kvällsmöte.

Theo berättade att hans datorladdare försvunnit för två veckor sedan under ett sektionsevent och att denna kostnad kan tas från sektionens olycksfond. 

Jonathan känner det är styrelsen får för få anmodningar till sittningar. Theo svarade att det är vad det är. 

Mattias informerade att redaktionen vill ha information från samtliga utskott om vad som händer på sektionen efter nollningen till en artikel i HeHE.

\p{19}{Sammanfattning av mötet}{\info}
Funktionärer har fyllnadsvalts och entledigas.

Utskottens arbete och Ekonomin går frammåt. 

Edekvata är stökigt på många håll vilket ska återgärdas. 

Mötet beslutade att köpa in 2st nya iZettle läsare.

Mötet beslutade att köpa in en mobil router som ska förenkla dataanvändningen utanför husets lokaler, exempelvis vid försäljning. 

Mötet beslutade att avslå inköp av nytt festljus med motivationen att handlingen var mer passande på sektionsmöte.    

\p{20}{OFMA}{\bes}
{\mo} förklarade mötet avslutat kl. 13.04
\end{paragrafer}

%\newpage
\hidesignfoot
\begin{signatures}{3}
\signature{\mo}{Mötesordförande}
\signature{\ms}{Mötessekreterare}
\signature{\ji}{Justerare}
\end{signatures}
\end{document}
