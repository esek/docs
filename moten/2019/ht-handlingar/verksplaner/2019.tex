\documentclass[../_main/handlingar.tex]{subfiles}


\begin{document}
\verksplan{2019}
Syftet med verksamhetsplanen är att skapa en tydligare struktur och mer operativt långsiktigt arbete. Verksamhetsplanen innehåller övergripande mål för Sektionen och delmål för styrelsen och Sektionens utskott för nästkommande år. Verksamhetens mål skall redovisas på varje terminsmöte.

\emph{Budgeten och verksamhetsplanen ska komplettera varandra.}

\subsubsection*{E-sektionen}
\emph{E-sektionen ska verka för att:}
\begin{dashlist}
	\item Främja medlemmarnas studietid.
	\item Alla medlemmar trivs.
	\item Inkludera alla medlemmar.
	\item Sektionens verksamhet präglas av demokrati och transparens.
	\item Ha en ständigt framåtsträvande verksamhet.
	\item Varje funktionärspost har ett ansvar och kan göra skillnad.
	\item Hålla en god och ansvarsfull ekonomi.
	\item Anpassa Sektionens verksamhet efter medlemmarnas behov.
\end{dashlist}

\subsubsection*{Styrelsen}
Styrelsen ska arbeta för att göra medlemmarnas studietid så bra som möjligt. Mer specifikt innebär detta förutom att vara ansvariga för den dagliga verksamheten att arbeta med mer långsiktiga frågor som till exempel lokaler och ekonomi.

\emph{Delmål 2018:}
\begin{dashlist}
	\item Hjälpa CM med övergången till att driva caféet ideellt.
	\item Fortsätta jobba för att involvera Sektionen och dess medlemmar mer i arbetet med Teknikfokus.
	\item Sprida information om friheten man har som medlem att genomdriva egna projekt, t.ex. genom att äska från styrelsen eller sektionsmötet och att bli projektfunktionär.
	\item Se över och utvärdera Sektionens funktionärsposter.
	\item Under året se över Sektionens kostnader och intäkter.
	\item Att försöka ha mer utbyte av evenemang mellan sektionerna genom att delta aktivt i kollegierna och att marknadsföra våra event bättre utåt.
	\item Panta ännu mera \scalebox{0.4}{\recycle}.
\end{dashlist}
\newpage
\subsubsection*{Sexmästeriet}
Sexmästeriet ska under året anordna prisvärda sittningar och evenemang av hög kvalité för sektionens medlemmar. 

\emph{Delmål 2019:}
\begin{dashlist}
	\item Att arrangera evenemang även under andra delar om året än nollningen.
	\item Arbeta för att hålla fortsatt god ordning i Sexmästeriets förråd.
	\item Jobba för att minimera svinn i alkohollagret. 
	\item Jobba för en god alkoholkultur på sittningar.
	\item Ha så litet matsvinn som möjligt. 


\end{dashlist}

\subsubsection*{Cafémästeriet}
Cafémästeriets verksamhet håller i dagsläget på att ändras då caféet ska börja drivas ideellt. Utskottet bör därför jobba mot att caféet ska ha en fungerande ideell verksamhet samt fortsätta vara konkurrenskraftigt och bibehålla studentvänliga priser.

\emph{Delmål 2019:}
\begin{dashlist}
	\item Uppdatera arbetsbeskrivningar och regler vid behov. 
	\item Utvärdera försäljningspriserna för de olika varorna i sortimentet. 
	\item Minska arbetsbelastningen för Cafémästare, Vice och Inköps- och lagerchefer.
	\item Utvärdera posterna och dess beskrivningar.
	\item Jobba för att minska svinn både i LED och i Cafémästeriets förråd.
    \item Fortsätta med att kontinuerligt se över cafets ekonomi. 
    \item Fortsätta jobba för god hygien och arbetsmiljö i köket.
	%\item Fortsätta med att göra kvartalsbokslut för att hålla koll på ekonomin samt för att få en överblick över caféets kostnader.
\end{dashlist}

\subsubsection*{Näringslivsutskottet}
Näringslivsutskottet knyter samman sektionens medlemmar med näringslivet. Utskottet bör anordna en god blandning aktiviteter som på olika sätt främjar sektionens medlemmars chanser inför arbetslivet. Utskottet har också i uppgift att dra in pengar till sektionen via olika evenemang eller sponsring.

\emph{Delmål 2018:}
\begin{dashlist}
	\item Att jobba för att ha mer evenemang relaterade till medicinsk teknik.
	\item Bibehålla samarbetet med befintliga företag samt aktivt söka nya samarbeten med organisationer och företag som är relevanta för sektionens medlemmar.
	\item Se över prissättning och syfte med aktiviteter för att nå en god blandning av vinstdrivande och icke-vinstdrivande aktiviteter.
\end{dashlist}

\newpage

\subsubsection*{Förvaltningsutskottet}
%Sektionens medlemmar ska ha tillgång till fräscha uppehållslokaler ämnade både för studier och studiesociala aktiviteter. I nuläget prioriteras ekonomin alltid högre än Sektionens lokaler. Målet ska vara att se till att underhåll av Edekvata görs löpande utan att ekonomin prioriteras ned.

\textit{Förvaltningsutskottets huvuduppgifter är att förvalta sektionens ekonomi, lokaler och inventarier. Med detta så innefattas att bidra uppehållslokaler vilka skapar goda möjligheter för sektionens medlemmar att genomföra både sina studier och sina studiesociala aktiviteter. 
För att detta ska kunna skötas bättre genom hela året bör en större fokus på feedback mellan utskotten ske genom hela de kommande året. Detta så att vi får reda på vart de större problemen kommer ifrån och hur man enklast ska kunna åtgärda dem. 
En ordentligen inventering av våra lokaler bör också ske. Detta för att undersöka vad det är vi har för gammal som kan slängas bort vilket skulle möjliggöra för bättre hushållning av de lokalerna vi har tillgång till.
}

\textit{Vi bör även se över posten arkivarie som just nu är på väg att förfalla till en “kaffepost” och om posten anses av de kommande förvaltningschefer vara otillräckligt sysselsatt så bör alternativ till posten tilldelas.
Bokföringen och det ekonomiska arbetet är och förblir den delen som tar absolut störst del inom utskottet. Vidare effektivisering av detta bör därmed överses och implementeras genom de kommande åren.
Med detta så bör även ett effektivare sätt att lagerinventera ses över. Speciellt gällande alkoholförreåd där vår AHS just nu ofta innebär mer arbete än vad bara ett vanligt Excel-ark skulle innebära.
Utöver detta så är en stor problem med utskottet att det inte för närvarandet har några större PR-events och därmed faller utskottet ofta i glömska mellan mer synliga utskott såsom Nollningsutskottet och Sexmästeriet. Därmed så bör en tydligare fokus på dokumentering och uppvisning av utskottet skötas för att i framtiden även kunna locka yngre medlemmar att söka sig till Förvaltningsutskottet.
}

\emph{Delmål 2019:}
\begin{dashlist}
	\item Se över användandet av EKEA och Tyskland och försöka möjliggöra dessa lokaler för effektivare förvaring.
	\item Undersöka om posten Arkivarie är central för utskottet eller om man borde tänka om den posten till kommande år.
	\item Se över det ekonomiska arbetet och om man ser möjligheter till förbättring genomföra dem till nästa år.
	\item Ha flera ekonomiska utbildningar för relevanta poster för att bättre genom året kunna genomföra vårt ekonomiska arbete.
\end{dashlist}

\newpage

\subsubsection*{Informationsutskottet}
Informationsutskottet har till uppgift att ansvara för informationsspridningen på Sektionen, Sektionens tekniska utrustning och Sektionens alumniverksamhet. Utskottet är ett spritt uttskott som har funktionärer med många olika arbetsuppgifter. 2016 infördes en ny post - Vice Kontaktor - som infördes för att ha en person som jobbar med att knyta samman utskottet. Därför bör utskottet 2018 fortsätta att jobba med att komma igång med sin nya sammansättning för att få bättre sammanhållning i utskottet.

\emph{Delmål 2018:}
\begin{dashlist}
	\item Få igång Vice Kontaktorns samarbete med Kontaktorn och resten av utskottet.
	\item Arbeta för fortsatt god informationsspridning genom kontinuerlig utvärdering av informationskanalerna.
	\item Jobba för att hålla Sektionens datorsystem och hemsida uppdaterade.
	\item Utvärdera skicket på Sektionens tekniska utrustning för att se om något behöver bytas ut i förebyggande syfte.
\end{dashlist}

\subsubsection*{Källarmästeriet}
Källarmästeriet bör jobba för att gillena ska vara fortsatt attraktiva för teknologer med ett varierande utbud av mat, dryck och aktiviteter. Utskottet bör också fortsätta marknadsföra gillena till övriga Teknologkåren och delta i sammarbeten med andra sektioner.

\emph{Delmål 2019:}
\begin{dashlist}
	\item Försöka locka fler teknologer från hela LTH till att komma på gillena.
	\item Uppmuntra samarbete såväl inom Sektionen som med andra sektioner.
	\item Jobba för att fortsatt ha ett bra och konkurrenskraftigt pris på mat och dryck.
	\item Jobba för att hålla fortsatt god ordning i förråd samt alkohollager
	\item Jobba för att få så lite matsvinn som möjligt.
	\item Jobba för god alkoholkultur på Sektionens evenemang.
	\item Jobba för att få så lite svinn som möjligt i alkohollagret.
\end{dashlist}
\newpage
\subsubsection*{Nöjesutskottet}
Nöjesutskottet bör fortsatt arbeta för att bidra till gemenskap på sektionen, där alla ska känna sig välkomna och där det finns något för alla olika intressen som faller under Nöjesutskottets ram. Man bör fortsätta arbeta för att locka folk som inte annars deltar i sektionsevenemang. 

Utöver det ska NöjU arbeta för bra stämning på campus genom intersektionella evenemang. Sporta med E bör, tillsammans med F-sektionen och idrottsförmännen -19 och -20, utvärderas för att se om man kan dra ner på kostnaden för hallhyra och på så vis möjliggöra andra sportrelaterade aktiviteter till en lägre kostnad för sektionen.  


\emph{Delmål 2019:}
\begin{dashlist}
	\item Att utskottsaktiva regelbundet genomför evenemang för sektionen.
	\item Att utskottet aktivt arbetar för att det ska finnas event som tilltalar alla på sektionen, men även bli bättre på att lyssna på vad som efterfrågas på sektionen genom exempelvis enkäter eller undersökningar.
	\item Utnyttja möjligheten att genomföra evenemang genom att äska pengar från styrelsen och sektionsmötet.
	\item Hålla minst ett intersektionellt, utomhusligt, alkoholfritt evenemang.
	\item Utvärdera Sporta med E.
\end{dashlist}

\subsubsection*{Nolleutskottet}
NollUs mål är att alla nyantagna studenter på E-sektionen ska få ett så bra mottagande som möjligt. Detta ska göras genom att arrangera en mängd olika studiesociala aktiviteter där det ska finnas något för alla. Mottagningen ska även ge en positiv attityd till studier med hjälp av phaddrar och äldre studenter. 

\emph{Delmål 2019:}
\begin{dashlist}
	\item Arbeta för att integrera internationella studenter i sektionen. 
	\item Arbeta för att det ska fortsätta finnas en mångfald av aktiviteter.
	\item Arbeta för att få fram förslag och förbättringar kring nollningen från Sektionens medlemmar.
	\item Arbeta för att förmedla en positiv attityd till studierna.
	\item Utvärdera arbetsbördan på utskottets funktionärer.
	\item Arbeta för en sund alkoholkultur under nollningen och främja alkoholfria event.
	\item Arbeta för att främja en bra relation med andra sektioner.
\end{dashlist}

\subsubsection*{Studierådet}
Studierådet ska aktivt arbeta med studiebevakning, samt arbeta för att alla sektionsmedlemmar ska ha en god studiemiljö.Arbetet skall göras mer tillgängligt för sektionens medlemmar för att visa förändringar som genomförts. Utskottet ska jobba transparent och ska verka för att medlemmarna är medvetna  om sina möjligheter att påverka och förbättra sin utbildning.

\emph{Delmål 2019:}
\begin{dashlist}
	\item Arbeta för att synliggöra utskottets arbete och resultat till sektionens medlemmar.
	\item Anordna och utvärdera studierelaterade evenemang kontinuerligt under året
	\item Arbeta för att ha minst en representant från varje årskurs, inklusive årskurs fyra och fem.
	\item Arbeta för att öka svarsfrekvensen på CEQ-enkäterna.
\end{dashlist}

\newpage
\end{document}
