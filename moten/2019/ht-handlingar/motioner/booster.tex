\documentclass[../_main/handlingar.tex]{subfiles}

\begin{document}
\motion{Reglementesändring, införande av posten Booster}

Nolleutskottet består idag av ett Øverphøs, fem Cophøs och två Øvergudphaddrar och är därför ett av de minsta utskotten på sektionen. Med denna motion vill jag utöka Nolleutskottet med en grupp funktionärer under ØGP som kallas för Boosters.

Fördelningen av hierarkin inom Nolleutskottet är i nuläget skevt. Nolleutskottet är det enda utskottet inom sektionen med fler utskottsordföranden/vice utskottsordföranden än resterande medlemmar. Detta kan lätt leda till ett utanförskap då majoriteten av utskottet är Phøs med dess tillkomna förmåner och gemenskap med andra sektioners Fhøb. Detta är något som tidigare år har vittnat om. Därför anser jag att det är viktigt för ØGP att ha en egen ensemble när resterandet av utskottet medverkar på åligganden och representativt arbete ØGP ej får delta vid, till exempel fhøbevenemang samt diverse intersektionella möten.

Samma gäller även för arbetsbördan då utskottet skulle ha en större verkställande arbetsstyrka. Detta skulle likna andra utskotts uppbyggnad då till exempel Källarmästeriet har Källaremästare, Informationsutskottet Kodhackare, Förvaltningsutskottet Hustomtar och så vidare. Jag tror att 6-8 personer hade varit en lagom mängd för Nolleutskottet.

Detta innebär att inte endast två personer har ansvaret för att rigga på plats, ta fram banderoller, bära  samtidigt som de förväntas vara huvudansvariga för att hålla i tagg vid evenemang. 
I nuläget hjälper Phøset till med detta men då nollor närvarar försvinner den möjligheten. I fall de får en hel grupp under sig kan detta arbetet delats upp så alla parter hade fått en rimligare arbetsbörda. ØGP ska inte behöva missa föreläsningar varje gång det är ett nollningsevent efteråt.
Øvergudphaddrarna ska inte känna sig som ett Phøs utan mantel.

Införandet av posten Booster och därmed Boostergruppen hade medfört avlastning och fördelning av arbetet inom Nolleutskottet i form av på/-avriggning, försäljning och möjligheten att hålla i lekar och diverse events. Boostergruppen hade även säkrat de personer som behövts vid större evenemang och ersatt Nollehjelp. Nollehjelp är en väldigt viktig del av nollningen men då denna inte ens är en post, har detta resulterat i svårigheter att fylla passen. Det ger ökad stress och arbete inom utskottet som ständigt får leta bland vänner och bekanta. Som man ser på E-sektionens hemsida finns det massor av personer som vill bidra och ha mer ansvar under nollningen men som inte fått möjligheten till det. Införandet av posten hade öppnat upp möjligheten för fler på sektionen att medverka i nollningen och Nolleutskottet.

Med ovanstående i åtanke yrkar motionären

\begin{attsatser}

    \att i regelmentet under \S10:2:L under  Øvergudphadder lägga till \par

    \begin{dashlist}
        \item Ansvarig för Boostergruppen och kontakten mellan denna och Phøset.
    \end{dashlist}

    \att i regelmentet under \S10:2:L lägga till posten Booster (6-8*****) med följande underpunkter   
    \begin{dashlist}
        \item ****
        \item **** 
    \end{dashlist}

\end{attsatser}


\begin{signatures}{1}
    \textit{Signerat}
    \signature{Tove Börjeson}{Cophøs 2019}
\end{signatures}

\end{document}
