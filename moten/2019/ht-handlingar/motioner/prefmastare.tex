\documentclass[../_main/handlingar.tex]{subfiles}

\begin{document}
\motion{Ändring av antal Preferensmästare}

I nuläget har sektionen en preferensmästare som sköter alla olika allergier och specialkost inför och under sittningar. Det ligger ett stort ansvar på denna personen att alla som deltar på sexmästeriets evenemang får rätt kost. Det hade gynnat sektionen att ha två preferensmästare på flera punkter. 

Dels kan preferensmästarna dela upp arbetet och inte själva behöva utforma alla de alternativa kosterna. Om sektionen skulle välja att ha två preferensmästare skulle det också göra det möjligt för de med specialkost att få roligare och mer anpassade alternativ. Som ensam preferensmästare kan det vara jobbigt, dels tidsmässigt och ansvarsmässigt, att utforma väl anpassade menyer. Det som blir lidande när man jobbar ensam är att t.ex. laktosintoleranta får samma mat som veganer för att underlätta arbetsbördan för preferensmästaren. Är man istället två preferensmästare kan fler varianter på mat utformas och man har dessutom någon att bolla idéer med. 

Jämför man arbetsbördan med köksmästarna som utformar en meny, i och för sig för många fler personer, så är det rimligt att vara två personer som utformar fler alternativa menyer. Att dessutom vara två preferensmästare som tillsammans arbetar skapar mer jämställdhet likt de andra posterna i sexmästeriet som alla består av två personer. Det skulle innebära att det inte endast känns som en post med arbete och ansvar utan att det även känns roligare, vilket är en viktig del när man arbetar ideellt.


Därför yrkar vi
\begin{attsatser}

    \att i regelmentet under \S10:2:K under Preferensmästare ändra \par
    \begin{emptylist}
        \item Preferensmästare (1)
        \begin{dashlist}
            \item Ansvarar för planering av meny och inhandling av specialkost innan sittningen.
            \item Sköter tillagningen av specialkost och delegeringen av uppgifterna i köket under sittningen.
        \end{dashlist}
    \end{emptylist}

    till 

    \begin{emptylist}
        \item Preferensmästare \hl{(2)}
        \begin{dashlist}
            \item Ansvarar för planering av meny och inhandling av specialkost innan sittningen.
            \item Sköter tillagningen av specialkost och delegeringen av uppgifterna i köket under sittningen.
        \end{dashlist}
        \changenote
    \end{emptylist}

\end{attsatser}

\changenote

\begin{signatures}{2}
    \isekt
    \signature{Silke Kylberg}{Preferensmästare}
    \signature{\sexm}{Sexmästare}
\end{signatures}

\end{document}
