\documentclass[../_main/handlingar.tex]{subfiles}

\begin{document}
\motion{Införandet av Kröke}

Hej, 

Tillståndsmyndigheten har ju under senaste året bytt åsikt som jag byter kallingar. Ibland
får man dricka öl men inte äta mat och ibland måste man äta mat medan man dricker öl och
ibland så tar de ölen och lämnar en med en pizza värd 70 kr och en förstörd kväll. Otäckt
säger jag. Jag vet inte var jag ska ta mig till. Livet handlar väl om rutiner och om jag inte vet
när, var eller hur jag får kröka så känner jag att min värld är i gungning. Lokalerna vi har ekar
tomma, våra en gång så tomma spritskåp är fulla och vårt sexmästeri måste “utvärdera
kvällen” utomhus. Usch säger jag. Vem vet vi vad vi får göra? Vem har ett svar? Kan någon
hjälpa vår sektion ur denna knipa? Nej inte för tillfället. Väldigt få personer har koll på
någonting och detta någonting är så oklart att man ofta får svaret: “ Asså ni får ju dricka öl
där, men inte just idag eller imorgon, fast kanske ändå om ni verkligen vill, eller asså solen
har ju inte gått upp än så det är nog lugnt. Äh jag vet inte” Nu är det dags för ändring! Vi bör
skapa en post, eller en Kröke, som har koll på allt vi får och inte får göra. Hen ska inte
släppa tillstånd ur sin kikare. Man ska kunna skriva till Kröke, dag som natt, morgon som
kväll och säga “ Får jag kröka här?”. Kröke ska kunna svara med ett glasklart JA eller som
sämst “Inte idag”. Kröke säger aldrig nej. Därav namnet.


Därför yrkar jag på
\begin{attsatser}

    \att i regelmentet under \S10:2:P lägga till följande postbeskrivning: \par
    \begin{emptylist}
        \item Kröke (1)
        \begin{dashlist}
            \item Ansvarar för att hålla koll på “valfri” myndighet för att kunna besvara på frågor gällande denna
            \item Denne ska alltid vara kontaktbar, samt
            \item Ansvarar för att kunna svara på frågan : “Får jag kröka?”
        \end{dashlist}
    \end{emptylist}
\end{attsatser}


\begin{signatures}{1}
    \isekt
    \signature{Anonym}{Alldeles för nykter sektionsmedlem}
\end{signatures}

\end{document}
