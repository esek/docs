\documentclass[../_main/handlingar.tex]{subfiles}

\begin{document}
\motion{Budgetjustering för E-sektionens bidrag till Sångarstriden 2019}

Intresset för Sångarstriden har ökat de senaste två åren och vi har redan fler engagerade i år jämfört med förra året. I våras tillsattes en entusiastisk projektgrupp som satsar på ett större SåS i år. För att förverkliga storsatsningen vill vi att budgeten ska anpassas till antalet engagerade. 

Den nuvarande budgeten för Sångarstriden är 7000 kr som ska räcka till kostym, dekor, smink, verktyg, jobbarglädje och ett tackevenemang. Vi hoppas på att kunna tacka 60 personer i år och räknat med 50 kr per person önskar vi 3000 kr för att genomföra detta. De nuvarande 7000 kr kommer att läggas på själva framträdandet. 

Sångarstriden är ett evenemang som främjar musikalitet och kreativitet och är det enda av sitt slag på LTH. Evenemanget tillåter även sektionsmedlemmar att mötas över program och årskursgränser vilket främjar sektionsandan. Men eftersom det endast är stridsropen som räknas som funktionärer och får ta del av E-sektionens funktionärsförmåner tycker vi att det är viktigt att de engagerade tackas och omhändertas på annat vis. 


Vi yrkar därför på
\begin{attsatser}
    \att öka kostnaden för budgetposten Sångarstriden under NöjU med 3000 kr till 10 000 kr.
\end{attsatser}


\begin{signatures}{3}
    \isekt
    \signature{Elsa Lindhé}{Stridsrop 2019}
    \signature{Y Nhi Pham}{Stridsrop 2019}
    \signature{Emma Hjörneby}{Kostymansvarig Sångarstridens projektgrupp}
\end{signatures}

\end{document}
