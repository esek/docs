\documentclass[../_main/handlingar.tex]{subfiles}

\begin{document}
\motion{Avskaffandet av posten Karnevalsmalaj}

Vart fjärde år håller Lunds studenter gemensamt Sveriges största studentevenemang, Lundakarnevalen. 
Det senaste året var 2018, där 5000 studenter från hela Lunds universitet arbetade gemensamt inför samt under Lundakarnevalen. 
Sektionen hyllar Lundakarnevalen genom att läsårsmässigt välja in en Karnevalsmalaj inför Lundakarnevalen. 
Denna person har enligt reglementet uppgiften 
\begin{dashlist}
    \item att organisera Sektionens deltagande i och omkring Lundakarnevalen.
\end{dashlist}

Ett problem med detta är att Lundakarnevalen har en policy om att enskilda studentorganisationer såsom kårer eller nationer inte får representeras under Lundakarnevalen, för att visa att Lunds studentliv är enhetligt.
Det skapar en konflikt med Karnevalsmalajens arbetsbeskrivning, och begränsar alltså vad denna kan göra under sin mandatperiod. Det enda Karnevalsmalajen då i princip göra är att uppmuntra sektionens medlemmar att söka att bli karnevalister, och inte mycket mer.
Detta passar sig bättre att Entertainer/Vice Entertainer tar på sig rollen att uppmuntra medlemmar att söka, speciellt med tanke på att allt arbete i samband med Tandem försvinner året då Lundakarnevalen är, än att sektionen har en separat post för just detta.


Därför yrkar jag 
\begin{attsatser}

    \att i regelmentet under \S10:2:H under Funktionärerna i Nöjesutskottet, NöjU stryka \par
    \begin{emptylist}
        \item Karnevalsmalaj (1)
    \end{emptylist}

\end{attsatser}


\begin{signatures}{1}
    \isekt
    \signature{Filip Larsson}{Karnevalsmalaj 2017/2018}
\end{signatures}

\end{document}
