\documentclass[../_main/handlingar.tex]{subfiles}

\begin{document}
\motion{Införandet av funktionärsposten Banan}
Under det gågna året har Elektro Banana Band lagt ner mycket tid och själ för att sektionen ska ha grymma spelningar att lyssna på. Vi hade 2 spelningar under vårterminen, 4 st under hösterminen än så länge och ska medverka i sektionens bidrag till Sångarstriden. Det har varit varierande tillställningar från 30-60 minuters speltid med variation i utbudet av låtar. Det har varit otroligt kul och givande att få spela för sektionen men för att kunna göra det krävs det att man lägger ner mycket arbete. En del av arbetet radas upp nedan.
\begin{itemizedash}
    \item Spika en setlist
    \item Repa flera gånger med bandet, samt öva själv
    \item Rodda upp, spela, rodda ner
\end{itemizedash}

Detta tar mer tid än vad man kan tro. För att sätta det ur ett tydligare perspektiv kan vi ge ett exempel.
\subsection*{NollEGasque}
Under NollEGasquens eftersläpp spelade vi för gästerna. Detta var mycket uppskattat från alla som var på plats, oss själva inkluderat.
Inför denna spelning repade vi tillsammans 4 gånger och hade innan dess möte om vilka låtar som skulle spelas. För att kunna repa måste vi först och främst scouta efter en ledig lokal att vistas i, vilket oftast brukar vara ett godtyckligt klassrum eller en datorsal i E-huset. Därefter ska vi rodda upp all utrustning vilket tar ca 10-20 minuter. Sen repar vi tillsammans \textit{minst} en spelning, d.v.s 30 minuter om vi ska spela 30 minuter på en sittning. Oftast tar det längre tid än så. Dagen innan NollEgasquen repade vi i 4-5 timmar totalt med upp - och nedroddning.  

Samma dag som gasquen hölls var vi i Gasquesalen och roddade upp intrumenten och soundcheckade. Detta verkar inte så krävande vid första anblick, men det tar längre tid än vad man tror. Dels ska alla intrument fraktas från E-huset till Gasque och sen tar det tid att koppla upp och stämma av så att allt låter bra. Sen skulle instrumenten plockas av scenen igen. Det tog ca 4-5 timmar.
Efter sittningen gick vi dit för att rodda upp våra instrument igen för att göra en snabb soundcheck igen då samma mikrofoner och högtalare använts av Sångförmän, talare och gycklare under sittningen. Sen när salen var lagom fylld satte vi igång och spelade i drygt 50 minuter. Sen \textit{efter} att eftersläppet var slut skulle allt roddas ned och fraktas över till E-huset igen.

Som ni ser är det mycket tid som läggs ner för sådana tillställningar. Arbetet som läggs ner av husbandets medlemmar anser vi matchar arbetet som innehavare av många andra funktionärsposter på Sektionen lägger ner. Därför anser vi att bandmedlem borde göras till en officiell funktionärspost. 
Detta har varit på agendan under ett tidigare sektionsmöte men blev avslaget i brist på dåvarande engagemang från bandets sida, samt i fruktan om att det lätt skulle kunna bli en kaffepost. Nu däremot, finns intresse av att vara med i bandet bland nuvarande bandmedlemmar samt andra medlemmar på Sektionen. För att komma undan problemet med kaffepost, har vi även med ett ytterligare yrkande som sätter lite högre krav på bandet och som efterliknar de krav som posten Umphmeister har.

Vi yrkar på:
\begin{attsatser}
    \att under \S10:2:H i Reglemente efter Øverbanan lägga till Banan(e.a) med följande underpunkter
    \begin{itemizedash}
            \item Ansvarar för att spela livemusik enligt Øverbananens instruktioner.
    \end{itemizedash}  
    \att under \S10:2:H i Relgemente under Øverbananens beskrivning ersätta den andra listade punkten med följande:
    \begin{itemizedash}
            \item Ansvarar för husbandets medverkan i Sångarstriden samt på de tillställningar där livemusik är passande.
    \end{itemizedash}
\end{attsatser}

\begin{signatures}{5}
	\mvh
	\signature{William Sjödin}{Øverbanan}
	\signature{Daniel Bakic}{Bandmedlem}
    \signature{Oskar Magnusson}{Bandmedlem}
    \signature{Valter Möller}{Bandmedlem}
    \signature{Lukas Elmlund}{Bandmedlem}
\end{signatures}
\end{document}