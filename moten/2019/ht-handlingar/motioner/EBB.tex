\documentclass[../_main/handlingar.tex]{subfiles}

\begin{document}
\motion{Inköp av utrustning för Elektro Banana Band}

För ungefär ett år sedan återuppstod sektionens ökända husband (EBB) och har under årets gång underhållit sektionens medlemmar med grymma musikuppträdanden. Vi har spelat på flertalet sittningar, gillen samt andra event där ett band kan tänkas spela. Under tiden har vi i bandet stött på vissa svårigheter och begränsningar gällande sektionens utbud av utrustning och vi har fått lära oss att man behöver mer utrustning än vad man skulle kunna tro. För att öka kvalitén på framtida spelningar och framförallt underlätta för nuvarande och kommande bandmedlemmar lägger vi därför fram denna motion i hopp om att sektionen tycker det är en värd investering.

\textbf{Motivering för nedanstående förslag:} \newline
Just nu innehar sektionen endast två stycken bra (fungerande) sångmikrofoner, vilket ibland blir bristande. Dels om man skulle vilja ha flera sångare eller kunna ha mick till blåsinstrument vilket kan innebära mer dynamiska spelningar och öppnar möjligheten för fler sångare. Till dessa mikrofoner kommer man även behöva införskaffa två nya mickstativ. \newline
För att få bättre och jämnare ljudnivå mellan de olika instrumenten och därmed ökad ljudkvalité behöver man micka upp gitarr- och basförstärkare samt trummor. För att kunna göra det på ett bra och smidigt sätt behövs dynamiska mikrofoner som är ämnade för just det syftet samt tillbehörande låga mikrofonstativ. \newline
Vid spelningar inomhus så kan det ofta bli väldigt hög ljudnivå, framförallt från trumsetet. Vilket kan innebära risk för hörselskador hos både bandmedlemmar och publik. En väldigt bra lösning på detta är att sätta upp en omslutande skärm för trumsetet. Detta kommer för allas skull behövas vid spelningar i Edekvata. \newline
Det underlätta otroligt mycket för bandemedelemmarna om de faktiskt skulle kunna höra sig själva ordentligt när de spelar. Detta gäller framförallt trummisen då hen uttryckligen flertalet påpekat att hen inte hör resten av bandet när hen trummar. Andra medlemmar har även uppmärksammat svårigheter i att höra sig själva eller bandet som helhet, vilket försvårar spelandet. Problemet kan enkelt lösas med att införskaffa monitorer (högtalare) som riktas mot bandet så de hör sig själva.
Slutligen behövs ytterliggare kablage till den nya utrustningen i form av XLR kablar som brukar användas till att koppla upp mikrofoner men kan även användas till annat.

\textit{Förslagen vi lägger fram är framplockade med hjälp av en av sektionens teknokrater som själv sysslat med musik i flera år och som har god erfarenhet av ljudutrustnig inom och utanför sektionen. Priset på kvalitativ ljudutrustning stiger lätt ganska fort när man letar efter bra grejer. Vi har försökt hålla oss till en prisvärd budget där vi tänkt på att det inte är rimligt för sektionen att köpa in superdyr utrustning till en så pass liten och ``osäker'' verksamhet. Men samtidigt något som håller en så pass hög standard att de både är bra och håller flera år framöver.}

Vi lägger fram följande förslag:
\begin{dashlist}
    \item 2 st sångmikrofoner för \SI{420,68}{kr/st} \textit{(\href{https://www.thomann.de/se/the_tbone_mb85_beta.htm}{länk})}
    \item 2 st mikrofonstativ för \SI{241,07}{kr/st} \textit{(\href{https://www.thomann.de/se/km_27115.htm}{länk})}
    \item 2 st instrument mikrofoner för \SI{1056,43}{kr/st} \textit{(\href{https://www.thomann.de/se/shure_sm57_lc.htm}{länk})} \textit{alternativt} för \SI{345,17}{kr/st} \textit{(\href{https://www.thomann.de/se/the_tbone_mb75.htm}{länk})}
    \item 2 st låga mikrofonstativ för \SI{214,65}{kr/st} \textit{(\href{https://www.thomann.de/se/millenium_ms2006.htm}{länk})}
    \item 1 st Drum shield \SI{3117,33}{kr} \textit{(\href{https://www.thomann.de/se/the_t.akustik_ds4_4_drum_shield.htm}{länk})}
    \item 3 st aktiva monitorer för \SI{1564,06}{kr/st} \textit{(\href{https://www.thomann.de/se/behringer_f1220_eurolive.htm}{länk})}
    \item 3 st XLR kablar (6m) för \SI{85,21}{kr/st} \textit{(\href{https://www.thomann.de/se/pro_snake_tpm_6.htm}{länk})}
    \item 4 st XLR kablar (10m) för \SI{106,72}{kr/st} \textit{(\href{https://www.thomann.de/se/pro_snake_tpm_10.htm}{länk})}
\end{dashlist}

Intrument mikrofonerna går även att använda som sångmikrofoner vilket gör att de gynnar fler verksamheter än endast bandets. Anledningen till att vi lägger fram två olika förslag till mikrofonerna är för att det blir ganska dyrt med bra mikrofoner. Därför har vi letat fram ett billigare alternativ som fortfarande är dugliga till vår verksamhet. Men de som är dyrare är betydligt bättre.

Vi yrkar på:
\begin{attsatser}
    \att köpa in utrustning enligt ovanstående förslag,
    \att budget för detta inköp sätts till \SI{14200}{kr} (15\% extra som marginal),
	\att kostnaden belastar utrustningsfonden, samt
	\att detta läggs på beslutsuppföljningen till VTM/2020 med undertecknade som ansvariga.
\end{attsatser}


\textit{Om sektionsmötet anser att det billigare alternativet skulle gynna sektionen mer är vi villiga att jämka oss med det och lägger i så fall ett ändringsyrkande på:}
\begin{attsatser}
    \att budget för inköpet istället sätts till \SI{12600}{kr}
\end{attsatser}

\begin{signatures}{5}
	\mvh
	\signature{William Sjödin}{Øverbanan}
	\signature{Daniel Bakic}{Bandmedlem}
    \signature{Oskar Magnusson}{Bandmedlem}
    \signature{Valter Möller}{Bandmedlem}
    \signature{Lukas Elmlund}{Bandmedlem}
\end{signatures}

\end{document}
