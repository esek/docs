\documentclass[../_main/handlingar.tex]{subfiles}

\begin{document}
\verksplanuppf{HT 2019}

\subsubsection*{Styrelsen}


\subsubsection*{Källarmästeriet}

Sedan vårterminsmötet så har källarmästeriet fortsatt med marknadsföring till teknologer vid andra sektioner genom kårnytt men även genom att arrangera pubrunda med V och W och aktivt deltagande i kårens sexmästarkolegium. Detta har verkat haft god effekt och vi upplever att det är allt vanligare att se teknologer från andra sektioner på gillet. 

Detta har lett till gott samarbete med andra sektioner men även inom sektionen så har källarmästeriet verkat för gott samarbete genom att till exempel hålla i FED-pub och BorgWarner-pub med ENU, NöjU har hållit i många olika mycket uppskattade aktiviteter på gillena såsom karaoke, bandet har spelat samt biljardturnering, även bakom kulisserna så har många fina samarbeten med InfU och FvU skett under året för att med gemensamma krafter förvalta Edekvata och de system som källarmästeriets verksamhet nyttjar. 

Att källarmästeriets vinstkrav sänktes på VTM har bidragit till att källarmästeriet har kunnat leverera med och dryck till god kvalitet med till ett bra och konkurrenskraftigt pris. Även om sektionen under våren skakades av att Zlatopramen blev beställningsvara på systembolaget så kunde förrådet av teknologens favoritöl säkras. 

Inköpet av iZettle-scanners upplevs ha minskat antalet fel vid försäljning av alkohol vilket i sin tur verkar positivt för att minska svinnet i alkohollagret. Cølen har också gjort ett bra jobb med att regelbundet märka ut dryck som närmar sig bäst-före datum för att strategiskt kunna välja upp vad som tagits fram för försäljning. Svinn förväntas ändå uppstå på samma sätt som tidigare år men har regelbundet skrivits ut för att bättre kunna hanteras mot budget. 

\subsubsection*{Cafémästeriet}

Vi i CM har rullat på med vår verksamhet, men detta gör det även svårt för oss att kunna göra några större förbättringar i LED. Trots detta så har vi vågat ha stängt för att minska arbetsbelastningen på oss som är hel årare. Vilket vi bland annat gjorde sista veckan LP1, alltså hade stängt. 

Vi har även sett över vår ekonomi så gott det går, bland annat så har vi märkt att vi går riktigt mycket minus på chai te, detta var då självklart efter vi köpte en sista kartong chai te. Vi har även gjort en större städning utav CM under nollningen för att minska svinnet, och vår lagerändring från förra mötet har även hjälpt oss att ha bättre koll på CMs lager. Det var även väldigt uppskattat utav miljöförvaltningen. Det blev även stopp på ett av våra handfat, som vi fixade till, vilket vi anser är mycket pluspoäng. 

Sen har vi även haft god kontakt med Hustomtarna och fixat lite saker som höll på att falla ihop, som vårt skafferi och översta lådan vid kassalinjen, vilket har gjort det enklare för våra arbetare att jobba.

\subsubsection*{Informationsutskottet}

Informationsutskottet har fortsatt att jobba mot verksamhetsplanerna som sattes för 2019. Eftersom det genomfördes ett skifte av Utskottsordförande under sommaren var det viktigaste att få en ny bra sammanhållning och komma igång med en god informationsspridning direkt och fortsätta på vårens verksamheten som den tidigare Utskottsordförande påbörjat. Eftersom det är ett väldigt spritt utskott med många olika och framförallt självständiga arbetsuppgifter är sammanhållning något som hela tiden måste hållas igång och uppföljas. Just sammanhållning och att det finns en röd tråd genom hela utskottet är något som framförallt Vice Kontaktor kan fortsätta jobba med även i framtiden. 

Sektionens tekniska utrustning har utvärderats och våra webbsidor och mjukvara har fortsatt snurra på så gott det går trots den åldrande mjukvaran. Under nollningen var det många elavbrott som satte käppar i hjulet för sektionens teknik men det ordnades upp. Förutom att mailservern just nu håller på att bytas har sektionen även ett stort behov av att byta huvudserver och allmänt förnya sektionens mjukvara. Detta är något som får fortsätta vara en del av verksamhetsplanen inför nästa år.

\subsubsection*{Nolleutskottet}

\subsubsection*{Förvaltningsutskottet}

Om man skulle sammanfatta förra årets verksamhetsplanen för Förvaltningsutskottet så skulle den säga att genom året jobba för att lokalerna skulle prioriteras högre än vad de har gjort tidigare år. Med detta sagt så har vi i utskottet jobbat mer kontinuerligt under året med att Sektionens svåraste lokaler, Sicrit och EKEA, skulle skötas bättre än tidigare år. Detta mycket tack vare ett väldigt duktigt Vice förvaltningschef och hustomte-team som ska få många applåder för deras bravader. 

Vi fick även hjälp med Sicrit efter nollningen från flera uppdragsgrupper då dessa ansågs vara den största anledningen till att Sicrit inte sköttes genom den perioden. Detta anser jag var lyckat och rekommenderar att nästa års Förvaltningsutskott gör detsamma.

Vi hade utöver detta även två delmål att effektivisera sektionens bokföring och utbilda berörda funktionärer kring sektionens ekonomi och dess bokföring. Detta har jag försökt göra tillsammans genom att jag har jobbat på en bättre utbildningsplan till nästa års styrelse och klarare testamente till nästa års förvaltningschef. 

Jag tillsammans med Skattmästaren har även tittat på att använda oss mer utav projektfunktionen för att lättare kunna ge klar uppföljning till events och resultatenheter. Detta är något som i alla fall jag men förhoppningsvist också min skattmästare kommer fortsätta titta in på och ha med i min utbildning till nästa års styrelse.


\subsubsection*{Studierådet}

Studierådet ska verka för att vara ett synligt utskott. Detta försöker vi uppnå genom att sätta upp posters runt om i E-huset på några poster som vi anser behöver synas lite extra; likabehandlingsombud, skyddsombud samt världsmästare. Under nollningen har vi försökt visa upp utskottet för de nyantagna genom att vara med på utskottssafari, samt hållt i workshop och arrangerat fyra pluggkvällar. Under workshopen presenterades utskottets arbete samt vad de olika posterna inom utskottet innebär. 

Studierådet ska även försöka ha studentrepresentanter från alla årskurser. Vi har redan fått in två stycken representanter från de nyantagna, men letar fortfarande efter representanter för årskurs E1-ansvarig. För att ha medlemmar som läser specialiseringen försöker vi ha kvar de som tidigare har varit årskursansvariga i utskottet. Studierådet ska även verka för att höja svarsfrekvensen för CEQ-enkäter. Efter läsperiod 1 ska vi testa med en CEQ-utkottning då detta har gett bra resultat för svarsferkvensen på andra sektioner.


\subsubsection*{Sexmästeriet}

Sexmästeriet har under året satsat hårt på att anordna prisvärda sittningar för alla medlemmar. Detta blandat med sittningar som förväntas hållas av sexmästeriet, så som teknikfokusbanquetten, skiphtet och HTF-sittning, så har sexmästeriet haft fullt upp. Sexmäteriet tackade även nej till en del sittningar med utomstående föreningar/organisationer för att kunna fokusera mer på Sektionen.

Nollningen var en intensiv period för sexmästeriet och det tärde på mästarna och jobbarna. Det var en otroligt rolig period för sexmästeriet men av förklarliga skäl pausade sexmästeriet med sittningar resten av läsperioden. Däremot har sexmästeriet haft bra spridning på eventen under våren och kommer att fortsätta hålla i sittningar under LP2 för att inte endast ha anordnat sittningar under nollningen på hösten. 

I övrigt har sexmästeriet kontinuerligt jobbat systematiskt med underhåll av våra förråd. Köksmästarna har rensat kylarna efter sittningar innan mat hunnit bli gammal. Hovmästarna har rensat ur pump flertal gånger och sett till att ha ordning och reda där inne.

Sexmästaren och barmästarna har tillsammans jobbat hårt för att minska alkoholsvinn. AHS:en har inte alltid varit vän med oss men tillsammans med Krögaren har vi tillsammans löst de problem som uppstått. 

\subsubsection*{Nöjesutskottet}

Nöjesutskottet har hållit i evenemang regelbundet under våren ända fram till idag, i form av spelkvällar och Sporta med E. Utöver detta har vi haft större event såsom Dömd och Tandem, men även ett flertal event under nollningen.   

Att erbjuda något för alla på sektionen är en fortsatt utmaning, men i och med att DreamHackE skrevs in i reglementet har vi nu en tradition på ett event för folk som annars kanske inte lockas av sektionsevenemang. 

Att jobba med intersektionella evenemang kan även det förbättras. Planen för det enda intersektionella eventet under våren, tillsammans med A, D och F ställdes in pga dåligt väder. Under nollningen hölls event tillsammans med M och K. 

\subsubsection*{Näringslivsutskottet}


\newpage
\begin{signatures}{10}
    \mvh
    \signature{\ordf}{Ordförande 2019}
    \signature{\sekr}{Kontaktor 2019}
    \signature{\fvc}{Förvaltningschef 2019}
    \signature{\cafem}{Cafémästare 2019}
    \signature{\oph}{Øverphøs 2019}
    \signature{\sreordf}{SRE-Ordförande 2019}
    \signature{\enuordf}{Näringslivsutskottets Ordförande 2019}
    \signature{\sexm}{Sexmästare 2019}
    \signature{\krog}{Krögare 2019}
    \signature{\ent}{Entertainer 2019}
\end{signatures}

\end{document}
