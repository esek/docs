\documentclass[../_main/handlingar.tex]{subfiles}

\begin{document}
\motionssvar

Styrelsen håller med motionären om att ytterligare funktionärer i NollU hade gynnat utskottet. 
Då det de senaste åren varit ett oerhört stort intresset för nollningen, både som funktionärer inom NollU och som phaddrar, hade skapandet av denna post givit fler möjligheten att engagera sig under nollningen. Vi har även sett utvecklingen av ØGPs arbetsuppgifter där likt motionären beskriver det posten blivit mer och mer som ett phøs utan mantel. Med Booster-gruppen skulle ØGP distansera sig från phøsets planeringsarbete och istället arbeta med sina mer operativa åtaganden. 

Sammantaget ställer styrelsen sig positivt till motionen men har några invändningar. 

\begin{itemize}
    \item Fokusen för gruppen anser vi bör ligga i genomförande, därför ser vi det problematiskt att i postbeskrivinngen även skriva “planera”. All typ av genomförande av tillställningar innefattar givetvis någon form av planering men för tydlighet tycker vi detta bör strykas. 
    \item Även formuleringen “och resterande Nolleutskottet” ser vi som överflödig då det är underförstått att funktionären arbetar för sitt utskott och svarar till utskottschefen. 
    \item Vi vill även markera skillnaderna mellan posten och NöjUs funktion under nollningen och därför inte uttryckligen skriva “huvudsakliga ansvaret för tagg”.
    \item Att tillsätta åtta stycken nya funktionärer ser vi som en för drastisk utgångspunkt. Vi föreslår istället att sex stycken prövas och att antalet sedan utvärderas till kommande år. För att hålla postens arbetsbelastning lägre och dess funktion som motionären beskriver mer i stånd med KMs källarmästare ser vi även fördelen med att inte låta posten väljas på valmötet. Utan av styrelsen på rekommendation av NollU vid ett senare tillfälle.
    \item Slutligen är vi inte övertygade på namnet Boosters, istället vill vi föreslå ØsarE. Samtidigt vill vi vara tydliga med att en diskussion kring postnamn är ett farligt lätt sätt att fördriva mycket tid för sektionsmötet, därför önskar vi att hålla diskussionen kring detta så kort som möjligt. 
\end{itemize}

Därför ställer styrelsen följande motyrkande

\begin{attsatser}
    \att i reglementet under \S10:2:L under Övergudsphadder lägga till
    \begin{dashlist}
        \item Ansvarig för sektionens ØsarE och kontakten mellan dessa och Phøset. 
    \end{dashlist}

    \att i reglementet under \S10:2:L lägga till posten ØsarE (6) med följande underpunkter:
    \begin{dashlist}
        \item Bistå Övergudphaddrarna med att genomföra nollningsrelaterade evenemang.
        \item Bidra till en god stämning vid nollningsrelaterade evenemang.
        \item Väljs av styrelsen i samråd med NollU.
    \end{dashlist}
\end{attsatser}

\begin{signatures}{1}
    \ist
    \signature{\ordf}{Ordförande}
\end{signatures}

\end{document}
