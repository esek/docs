\documentclass[../_main/handlingar.tex]{subfiles}

\begin{document}
\utskottsrapport{Informationsutskottet}
\vspace{8px}
Informationsutskottet har fortsatt med sin verksamhet inom informationsspridning under hösten. 

Den nya redaktionen har kommit igång bra och har levererat utomordentligt fina upplagor av HeHE fyllda av härliga artiklar. Sedan tidigare år har man slutat använda maillistan, delvis på grund av GDPR och spamfilter, utan istället har HeHE trycks i fysiskt format i Edekvata och spridits på facebook. I skrivandes stund undersöks möjligheten att även sätta upp ett tidningsställ i samband med LED café för att så många som möjligt ska få chansen att kunna läsa Hent i Hus E. 

Under nollningen har fotograferna fångat många fina ögonblick på bild och vi har kunnat dokumentera rekordmånga evenemang. Sektionskameran som införskaffades förra året har använts flitigt och fungerar mycket bra. Nytt för i år är att varje fotograf nu har ett eget minneskort för att underlätta fotografering och redigering samt att vi nu har ett stativ om fotovägg skulle önskas.   

När det gäller grafik har Picasso gjort ett fantastiskt jobb och har levererat utifrån beställningar av olika slag, såsom posters, affischer och tygmärken. De har också gjort en bra insats i design av NollEguiden. Grafik till events har kunnat ses på våra anslagstavlor och våra infoskärmar runt om i E-huset. 

Sektionens tekniska utrustning har fått ta emot lite stryk med alla strömavbrott som drabbat E-huset under hösten. Den tekniska sidan av utskottet har jobbat för att få all vår hård- och mjukvara att fungera som det ska. När det gäller ljud och ljus så är sektionens utrustning i gott skick, och teknokrafterna har jobbat mycket med ljudriggning inför sittningar andra event under nollningen. Sektionens hemsidor fungerar och snurrar på som tidigare, men bakom kulisserna är sektionens mjukvara i sämre skick och behöver ses över i framtiden. Önskan likt tidigare år är att kunna genomföra ett byte av server. Tyvärr har det projektet inte kunnat inledas av DDG. En mer brådskande del är att sektionens mailserver behöver bytas ut på grund av säkerhetsrisker från gammal mjukvara. En projektgrupp för detta är tillsatt och parallellt ses möjligheten att hantera sektionens mailadresser med G Suite över. 



\begin{signatures}{1}
    \mvh
    \signature{Mattias Lundström}{Kontaktor}
\end{signatures}

\end{document}
