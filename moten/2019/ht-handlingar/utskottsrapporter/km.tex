\documentclass[../_main/handlingar.tex]{subfiles}

\begin{document}
\utskottsrapport{Källarmästeriet}
\vspace{8px}

Sedan vårterminsmötet har källarmästeriet regelbundet hållit i gillen och andra event av kläggig kraraktär, så som att sälja klägg på UtEDischot och Regattan. Event har hållits i samarbete med andra utskott på sektionen såsom företagspubbar med ENU och NöjU har hållit i uppskattade event och aktiviteter så som karaoke och biljardturnering. Källarmästeriet har också deltagit i genomförandet av en pubrunda med V- och W-sektionen vilket var ett uppskattat intersektionellt event under nollningen. Samtliga delar av utskottet har vid det här laget blivit mycket bekväma med sina uppgifter och har utan större problem hanterat allt ifrån elfel till kraschande iPads och kol som inte vill ta eld. Enligt tradition har även grillkol köpts in för att fylla samtliga av sektionens outnyttjade förvaringsplatser. Dessutom har Cølen hållit i en ölprovning, ett uppskattat komplement till källarmästeriets ordinarie verksamhet. 

Vid vårterminsmötet så sänktes vinstkravet för källarmästeriet vilket har nyttjats för att vara extra frikostiga med funktionärsvårdande inköp så som chips och dipp samt att alla utanför källarmästeriet som jobbat på gillen, till exempel fotografer, macapärer och Elektro Banana Band utan närmare eftertanke har kunnat få mat och jobbarglädje vilket verkar ha varit uppskattat då dessa personer också jobbar enligt vår uppfattning. 

Som huvudsakliga brukare av Edekvataköket har även inköp av såväl skedar som ljusslingor gjorts för att göra underlätta för källarmästarna att göra sitt jobb samt ytterligare förhöja upplevelsen av att gå på gille. Som ett utskott med mycket ansvar och kunskap kring sektionens alkoholtillstånd så har källarmästeriet varit behjälpliga i dessa frågor då andra utskott haft funderingar.  Jag har även i största möjliga mån deltagit i Teknologkårens Sexmästarkollegium där dessa typer av frågeställningar och mycket annat har diskuterats och samarbeten intersektionellt har tagit form. 


\begin{signatures}{1}
    \mvh
    \signature{Davida Åström}{Krögare}
\end{signatures}

\end{document}
