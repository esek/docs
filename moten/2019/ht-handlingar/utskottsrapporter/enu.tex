\documentclass[../_main/handlingar.tex]{subfiles}

\begin{document}
\utskottsrapport{Näringslivsutskottet}
\vspace{8px}


Sedan vårterminsmötet har utskottet arrangerat diverse företagsevenemang vid sidan av att näringslivskontakterna och utskottet sökt nya kontakter inom näringslivet. Vi arrangerade Lunch med Ingenjör som var uppskattat av både deltagare och företagsrepresentanter samt FED pub ihop med F- och D-sektionen. Det planerade eventet ihop med Knightec som var på tapeten i våras har blivit mycket uppskjutet men är inplanerat 18 november. Det ska då vara en workshop på deras kontor där E- och D-studenter ska bjudas in. 

Under nollningen har vi, i samarbete med phøset, arrangerat SVEP Robotic Challange, tre lunchföreläsningar samt haft Academic Work på besök i E-foajén. Efter nollningen höll vi i en workshop med VentureLab gällande NABC-metoden. Även en lunchföreläsning med D- och F-sektionen har utskottet genomfört samt en företagspub med BorgWarner med hjälp av KM! I och med puben med BorgWarner är huvudsponsoravtalet som slöts uppfyllt. Framöver har utskottet planerat in LinkedIn-föreläsning och CV-granskning med Academic Work och programmeringskväll på Altens kontor. Utskottet fortsätter söka kontakter med både nya och gamla företag.

Alumniverksamheten har genomfört en specialiseringskväll med SRE där alumner bjöds in för att sektionens studenter skulle få möjligheten att diskutera programmens olika specialiseringar. Alumniansvariga har även börjat på en insamling av statistik från alumner gällande vilken specialisering man har läst, var man fått jobb och vilken ingångslön man fått. I samband med detta ska alumniverksamheten marknadsföra alumnigrupperna på facebook och LinkedIn för att fånga in nyligen examinerade medlemmar och hålla grupperna aktiva. De har även planerat in en alumnipub och en nationssittning för 4or och 5or.

Teknikfokus arbetar på inför mässan och har börjat kontakta företag och kommit igång med planerandet. Projektgruppen valdes redan i slutet på vårterminen och gruppen satte igång med arbetet direkt vid höstterminens start.


\begin{signatures}{1}
    \mvh
    \signature{Jakob Pettersson}{ENU-ordförande}
\end{signatures}

\end{document}
