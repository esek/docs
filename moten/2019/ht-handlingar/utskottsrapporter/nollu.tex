\documentclass[../_main/handlingar.tex]{subfiles}

\begin{document}
\utskottsrapport{Nolleutskottet}
\vspace{8px}

I slutet av våren hade NollU med hjälp av E6 temasläpp tillsammans med Staben och A-phøs i Gasquesale vilket gick väldigt bra. Innan sommaren hölls även en phadderkickoff för att phaddrarna skulle lära känna varandra bättre och börja tagga igång inför nollningen. 

Under hösten har NollU hållit i En Nollning på Okänd Mark. Överlag har det gått superbra och nollorna har verkat väldigt nöjda med nollningens aktiviteter. Det var ovanligt stor uppslutning på nästan alla evenemang, stora som små, vilket var otroligt roligt. Bortsätt från de traditionella aktiviteterna arrangerade vi nya så som Mega Kul fEstivalen, ett alkoholfritt event tillsammans med M och K. Vi ordnade även en Uppdragssceremoni på torsdagen lv. 0 för att öka hypen kring uppdragen. För att inkludera Intisarna mer i nollningen ändrade vi i logistiken så de fick köpa ouvve och vara med på St Hans utan att eventet blev längre. Det uppskattades enormt av intisar och intisphaddrar. Välkommstgillet effektiviserades även så en timme sparades in. E-sektionen vann i år Regatta vilket var väldigt roligt. E D AW? efter WaDerloo fick ställas in efter ca 1h då det började regna och åska, dock hann de som var kvar köpa korv och dansa lite nolledanser så det var bra så länge det varade.

Efter nollningen har en utvärdering skickats ut till både nollor och phaddrar så att de har kunnat berätta vad de tyckte om nollningen och ge förslag på förändringar och förbättringar till nästa års nollning. Det som är kvar att göra för nollningsutskottet under resten av året är att sammanställa utvärderingarna samt jobba med testamente och överlämning så att nästa års nollningsutskott har så bra förutsättningar som möjligt när de kliver på. Tack för detta år!


\begin{signatures}{1}
    \mvh
    \signature{Stephanie Bol}{Øverphøs}
\end{signatures}

\end{document}
