\documentclass[../_main/handlingar.tex]{subfiles}

\begin{document}
\proposition{Namn på NollU-funktionärer}

Genom reglementet skrivs Phøs på flera olika sätt, vilket som är detta faktiskt “rätta” går inte längre att urskilja. Men vad vi vet säkert är att det endast är en stavning som används idag. Följande stavningar förekommer: Øverphøsare, Øverphøs, Co-phøsare, Co-phøs, Cophøs. 

Vi tycker att styrdokumentet ska vara konsekventa och spegla dagens verksamhet. 

Därför yrkar styrelsen

\begin{attsatser}
    \att i Reglementet \S10:2:L ändra namnet på funktionärsposten “Øverphøsare” till “Øverphøs” .

    \att i Reglementet \S10:2:L ändra namnet på funktionärsposten  “Co-phøsare” till “Cophøs” samt att ändra till samma namn i vederbörande beskrivning. 

    \att i Reglementet \S10:2:L ändra stavningen av “Övergudsphadder” till “Øvergudsphadder”


\end{attsatser}

\begin{signatures}{1}
    \ist
    \signature{\ordf}{Orförande}
\end{signatures}

\end{document}
