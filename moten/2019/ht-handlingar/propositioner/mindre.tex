\documentclass[../_main/handlingar.tex]{subfiles}

\begin{document}
\proposition{Reglementesändring, mindre uppdateringar}

Denna proposition är en gruppering av några mindre reglementesändringar. Där reglementet antingen är utdaterad, inte längre relevant eller inte speglar verksamheten, några föreslagna ändringar ämnar också att tydliggöra.

\subsubsection{Valberedningsledamöter} 

På Vårtterminsmötet 2019 beslutades det att ändra sammansättningen av valberedningen. 
Dock följde inte ändringen med till Reglementet som fortfarande säger att antalet ledamöter är två. 

\subsubsection{Utbildning av nyantagna} 

Tidigare år har det varit sektionerna som introducerar nollorna i LTHs datorsystem men nu görs det av skolan istället. 
Detta står fortfarande som ett åtagande för InfU i Reglementet. Detta tycker vi borde uppdateras, dock ser vi gärna att InfU spelar en roll när de nyantagna ska få tillgång till sektionens hemsida och datorsystem.

På dessa grunder yrkar styrelsen
\begin{attsatser}
    
    \att i reglementet \S10:2:N  under Valberedningsledamot ändra
        \begin{emptylist}
            \item Valberedningsledamot (exakt 2)
         \end{emptylist}
    
        till 
    
        \begin{emptylist}
            \item Valberedningsledamot \hl{(exakt 3)}
        \end{emptylist}
        \changenote

    \att i reglementet \S9:2:B Informationutskottet stryka det gulmarkerade 
        \begin{emptylist}
            \item Det åligger utskottet att
            \begin{dashlist}
                \item se till att Sektionens hemsida fungerar bra och har uppdaterad information.
                \item se till att Sektionens tekniska utrustning fungerar.
                \item skriva och publicera nollEguiden och HeHE.
                \item \hl{under nollningen utbilda nyantagna studenter i LTH:s och Sektionens datorsystem.}
            \end{dashlist}
        \end{emptylist}
    
        och istället ersätta med
    
        \begin{emptylist}
            \item Det åligger utskottet att
            \begin{dashlist}
                \item se till att Sektionens hemsida fungerar bra och har uppdaterad information.
                \item se till att Sektionens tekniska utrustning fungerar.
                \item skriva och publicera nollEguiden och HeHE.
                \item \hl{informera nyantagna studenter om Sektionens datorsystem.}
            \end{dashlist}
        \end{emptylist}
        \changenote
        
    

\end{attsatser}







\begin{signatures}{1}
    \ist
    \signature{\ordf}{Ordförande}

\end{signatures}

\end{document}
