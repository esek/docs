\documentclass[../_main/handlingar.tex]{subfiles}

\begin{document}
\proposition{Reglementesändring Nolleutskottet}

I dagsläget står det följande i reglementet under funktionsbeskrivningen av Co-phøs:

\begin{dashlist}
\item Ett Co-phøs ansvarar för den ekonomiska redovisningen av nollningen. 
\item Ett Co-phøs ansvarar för rekryteringen av phaddrarna.
\end{dashlist}

Att det ska finnas någon med ekonomiskt samt phadderansvar står redan under utskottsbeskrivningen av NollU. Därför anser styrelsen att NollU tillsammans ska ha friheten att bestämma vem i phøset som har dessa ansvar så länge någon av dem ansvarar över dessa områden. Øverphøset ska alltså ha möjlighet att till exempel ha ekonomiansvar om det skulle vara NollUs önskan. 

Därför yrkar styrelsen

\begin{attsatser}

    \att i Regelmentet \S10:2:L under Co-phøsare stryka det gulmarkerande

    \begin{emptylist}
        \item Co-phøsare (5)
        \begin{dashlist}
            \item Bistår Øverphøset i dennes arbete.
            \item Denna post är en vice till utskottsordföranden.
            \item \hl{Ett Co-phøs ansvarar för den ekonomiska redovisningen av nollningen.}
            \item \hl{Ett Co-phøs ansvarar för rekryteringen av phaddrarna.}
        \end{dashlist}
        \changenote
    \end{emptylist}


    \att i Reglementet \S9:2:H ändra 

    \begin{emptylist}
        \item {\large\textbf{Nolleutskottet, NollU}}

        \item Nolleutskottet har till uppgift att arrangera mottagandet av nyantagna studenter.
    
        \item Det åligger utskottet att

        \begin{dashlist}
            \item rekrytera, utbilda och organisera phaddrar till nollningen.
            \item arrangera nollningsaktiviteter som får de nyantagna att känna sig välkomna till Sektionen.
            \item i samråd med studievägledningen och Studierådet arrangera nollningsaktiviteter som främjarstudierna vid högskolan.
            \item inom utskottet utse en ekonomiansvarig.
        \end{dashlist}
    \end{emptylist}

    till 

    
    \begin{emptylist}
        \item {\large\textbf{Nolleutskottet, NollU}}
        \item Nolleutskottet har till uppgift att arrangera mottagandet av nyantagna studenter.
        
        \item Det åligger utskottet att

        \begin{dashlist}
            \item rekrytera, utbilda och organisera phaddrar till nollningen.
            \item arrangera nollningsaktiviteter som får de nyantagna att känna sig välkomna till Sektionen.
            \item i samråd med studievägledningen och Studierådet arrangera nollningsaktiviteter som främjarstudierna vid högskolan.
            \item inom utskottet utse \hl{ett phös till} ekonomiansvarig.
        \end{dashlist}
        \changenote
    \end{emptylist}

\end{attsatser}

\begin{signatures}{1}
    \ist
    \signature{\oph}{Øverphøs}
\end{signatures}

\end{document}
