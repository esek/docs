\documentclass[10pt]{article}
\usepackage[utf8]{inputenc}
\usepackage[swedish]{babel}
\def\mo{Niklas Gustafson}
\def\ms{Sonja Kenari}
\def\ji{Rasmus Sobel}
\def\jii{Sophia Carlsson}

\def\doctype{Protokoll} %ex. Kallelse, Handlingar, Protkoll
\def\mname{Vårterminsmöte} %ex. styrelsemöte, Vårterminsmöte
\def\mnum{VT/19} %ex S02/16, E1/15, VT/13
\def\date{2019-04-09} %YYYY-MM-DD
\def\docauthor{\ms}

\usepackage{../e-mote}
\usepackage{../../../e-sek}
\usepackage{../_sektion-handlingar/e-handlingar-sek}

\begin{document}
\showsignfoot

\heading{{\doctype} för {\mname} {\mnum}}

%\naun{}{} %närvarane under
%\nati{}{} %närvarande till och med
%\nafr{}{} %närvarande från och med
\section*{Närvarande}
\subsection*{Styrelsen}
\begin{narvarolista}
\nv{Ordförande}{Edvard Carlsson}{E16}{}
\nv{Kontaktor}{Sonja Kenari}{E15}{}
\nv{Krögare}{Davida Åström}{BME17}{}
\nv{Förvaltningschef}{Henrik Ramström}{E16}{}
\nv{ENU-ordförande}{Jakob Pettersson}{E17}{}
\nv{SRE-ordförande}{Lina Samnegård}{BME16}{}
\nv{Sexmästare}{Theo Nyman}{BME18}{\frun{14}{21}}
\nv{Entertainer}{Saga Åslund}{BME18}{\frun{14}{19}}
\nv{Cafémästare}{Jonathan Benitez}{E17}{}
\nv{Øverphøs}{Stephanie Bol}{BME17}{\frun{14}{19}}
\end{narvarolista}

\subsection*{Medlemmar}
\begin{narvarolista}
\nv{}{Sanna Nordberg}{BME16}{}
\nv{}{Anton Jigsved}{BME16}{}
\nv{}{Hjalmar Tingberg}{BME16}{}
\nv{}{Magnus Lundh}{E15}{\nati{20}{}}
\nv{}{Andreas Bennström}{BME16}{}
\nv{}{Vincent Palmer}{E18}{}
\nv{}{Filip Larsson}{E17}{}
\nv{}{Emil P. Lundh}{E17}{\nati{21}{}}
\nv{}{David Karlsson}{E17}{\nati{21}{}}
\nv{}{Fanny Månefjord}{BME16}{\nati{21}{}}
\nv{}{Adam Rosandell}{E17}{\nati{20}{}}
\nv{}{Måns Rosberg}{E17}{\nati{20}{}}
\nv{}{Elsa Lindhé}{BME17}{\nati{22}{}}
\nv{}{Evelina Morgan}{BME18}{\nati{21}{}}
\nv{}{Sofie Johannesson}{E17}{\nati{21}{}}
\nv{}{Hanna Bengtsson}{BME18}{\nati{21}{}}
\nv{}{Simon Mahdavi}{BME18}{\nati{21}{}}
\nv{}{Linnea Söderström}{BME18}{\nati{21}{}}
\nv{}{David Karlsson}{E18}{\nati{21}}
\nv{}{Malin Rudin}{E18}{\nati{21}}
\nv{}{Sophia Lennartsson}{E18}{\nati{21}}
\nv{}{Elina Yrlid}{E18}{}
\nv{}{Malin Heyden}{E16}{\nati{21H}}
\nv{}{Adam Belfrage}{BME17}{\nati{21K}}
\nv{}{Carl Rutholm}{E17}{}
\nv{}{Tove Börjesson}{E17}{}
\nv{}{Johan Vikstrand}{E17}{\nati{20}}
\nv{}{Moa Rönnlund}{E17}{}
\nv{}{Richard Byström}{E18}{\nati{21}}
\nv{}{Emma Hjörneby}{BME17}{\nati{20}}
\nv{}{Tina Tabandeh}{BME17}{}
\nv{}{Johan Halldin}{BME17}{}
\nv{}{Alfred Langerbeck}{E18}{}
\nv{}{Matilda Horn}{BME18}{}
\nv{}{Elin Johansson}{BME16}{\nati{21}}
\nv{}{Markus Kvist}{E17}{\nati{20}}
\nv{}{Amjad Belbisi}{E17}{\nati{20}}
\nv{}{Frida Pilcher}{E18}{\nati{21}}
\nv{}{Johannes Larsson}{E16}{}
\nv{}{Georgij Michaliutin}{E18}{}
\nv{}{Casper Schwerin}{BME18}{\nati{21}}
\nv{}{Hannes Björk}{E17}{}
\nv{}{Emil Eriksson}{E18}{}
\nv{}{Erica Elgcrona}{E18}{\nati{21}}
\nv{}{Love Sjelvgren}{E18}{}
\nv{}{Sophia Carlsson}{BME17}{}\\
\end{narvarolista}

\newpage

\begin{narvarolista}

\nv{}{Alexander Wik}{BME17}{\nati{21}}
\nv{}{Rasmus Sobel}{BME16}{}
\nv{}{William Sjödin}{E18}{\nati{21}}
\nv{}{Markus Rahne}{BME14}{\nati{21}}
\nv{}{Antonia Mundt-Petersen}{E17}{\naun{21A}{21K}}
\nv{}{Axel Voss}{E15}{\naun{1}{21}}
\nv{}{Max Mauritsson}{BME16}{\naun{4}{21}}
\nv{}{Linnea Sjödahl}{BME15}{\naun{9}{21I}}
\nv{}{Daniel Bakic}{E15}{\nafr{10}}
\nv{}{Oskar Branzell}{E18}{\naun{10}{21}}
\nv{}{Fredrik Berg}{E17}{\naun{14}{21}}
\nv{}{Adam Ekblom}{BME18}{\nati{21}}


%\nv{Post}{Namn}{Klass}{}
\end{narvarolista}

\subsection*{Ständigt adjungerande}
\begin{narvarolista}
\nv{Talman}{Niklas Gustafson}{E15}{}
%\nv{Revisor}{Erik Månsson}{E14}{}
%\nv{Revisor}{Johan Karlberg}{E14}{}
%\nv{Post}{Namn}{Klass}{}
\end{narvarolista}


\subsection*{Adjungerande}
\begin{narvarolista}
\nv{}{Filip Johansson}{Teknologkårens representant}{\nati{14}{A}}
%\nv{Post}{Namn}{Klass}{}
\end{narvarolista}


\newpage
\section*{Protokoll}
\begin{paragrafer}
\p{1}{TaFMÖ}{}
Talman {\mo} förklarade mötet öppnat kl.17.33.

\p{2}{Val av mötesordförande}{}
Talman {\mo} valdes.

\p{3}{Val av mötessekreterare}{}
Kontaktor {\ms} valdes.

\p{4}{Godkännande av tid och sätt}{}
Tid och sätt godkändes.

\p{5}{Val av två justeringspersoner}{}
\valavj

\p{6}{Adjungeringar}{}
%\ingaadj
Filip Johansson adjungerandes.

\p{7}{Godkännande av dagordningen}{}
%Föredragningslistan godkändes.

%Föredragningslistan godkändes med yrkandet.
%Föredragningslistan godkändes med samtliga yrkanden.

Förvaltningschef Henrik Ramström \ypa stryka \S18 ``Styrelsens förslag till disposition''.

Ordförande Edvard Carlsson \ypa byta ut \S21.F till sena handlingarna
 som punkt \S21.F istället.

Valberedningen \ypa behandla \S21.L före \S21.A-K.

Föredragningslistan godkändes med samtliga yrkanden.

\p{8}{Föregående sektionsmötesprotokoll}{}
\latillprot{HT/18 och VM/18}

\p{9}{Meddelanden}{}
%Ingen hade något att meddela.
Edvard Carlsson meddelade om att det skett sexuella trakasserier på sektionen.
Styrelsen ser allvarligt på detta och har försökt jobba fram en handlingsplan för att motverka dessa händelser på sektionen.
En enkät kommer att skickas ut som det skulle uppskattas om sektionsmedlemmar svarar på. Detta för vi ska kunna hålla en välkomnande stämning för alla på sektionen.

Henrik Ramström informerade om Fullmäktige på Teknologkåren angående vad de gör samt har gjort under det senaste halvåret.

Filip Johansson meddelade om att Teknologkåren fyller 35 i år och att det kommer vara en jubileumsvecka i Maj. Filip meddelade även att 
det går att nominera sina lärare till Teknologkårens pris. Det kommer ut en länk eller så kan man nominera genom att kontakta SRE. 

Sophia Carlsson uppmuntrade folk till att söka projektfunktionärer om man skulle vilja göra en egen bal kommande år.
Det har varit väldigt roligt att anordna det. Vill man göra ett eget projekt, behöver inte vara just en bal utan kan vara vad man vill skapa, kan man höra av sig till styrelsen.


\p{10}{Beslutsuppföljning}{}
Filip Larsson berättade om utrustningen som undertecknade köpte in till sektionen.

Filip Larsson \ypa stryka \emph{Uppgradera utrustning i Edekvatas kök} från beslutsuppföljningen. 

    Total kostnad uppgick till \SI{5908,50}{kr} och budget var \SI{6500}{kr}.

\Mbaby

Daniel Bakic var inte närvarande när uppföljningen av ``Inköp av mikrovågsugnar'' togs upp. Edvard Carlsson undrade om någon från 		föregående års styrelse ville lyfta beslutsuppföljningen. 

Magnus Lundh \ypa stryka \emph{Inköp av mikrovågsugnar} då motionen inte längre är aktuell från sektionens sida.

\Mbaby

Malin Heyden \ypa stryka \emph{Inköp av toppar till Vega} från beslutsuppföljningen. 

    Total kostnad uppgick till \SI{7052,16}{kr} och budget var \SI{8000}{kr}. 

\Mbaby

\p{11}{Utskottsrapporter}{}
Adam Belfrage \ypa hoppa över alla verksamhetsplaner, 
    uppföljningar samt utskottsrapporter under förutsättningen att alla läst handlingarna och att det inte finns någon 
        anledning till att Styrelsen 2019 eller 2018 läser innantill.

\Mbaby

Mötet fick istället möjligheten att ställa frågor till Styrelsen och Valberedningen.

\p{12}{Uppföljning av verksamhetsplan}{}
Mötet fick möjligheten att ställa frågor till Styrelsen och Valberedningen.


\p{13}{Ekonomisk rapport}{}
Sektionens Förvaltningschef Henrik Ramström gav en rapport för sektionens ekonomi.

Rasmus Sobel frågade hur mycket vi får tjäna och vad vi gör med pengarna.

Henrik Ramström svarar med att vi lägger in det överskott som vi får i fonder. 

%\textbf{\Mba lägga den ekonomiska rapporten till handlingarna.}

\p{14}{Val}{}
Valberedningen presenterade hur de har genomfört intervjuprocessen.

\textbf{\Mba välja Alexander Wik till Teknikfokusansvarig.}\par

\textbf{\Mba sätta Alumniansvarig BME till vakant}

%\end{paragrafer}
\p{15}{Verksamhetsberättelser för 2018}{}
Mötet fick möjligheten att ställa frågor om verksamhetsberättelsen 2018.

\p{16}{Bokslut}{}
Förvaltningschef Henrik Ramström \ypa bordlägga Bokslutet för 2018 till nästa sektionsmöte.

\Mbaby

Förvaltningschef Henrik Ramström \ypa korrigera bokslutet för 2017. I det nuvarande resultatet finns ett överskott på 245 000 kr berättar Henrik. 

\Mbaby

\p{17}{Revisionsberättelse för 2018}{}
Ordförande Edvard Carlsson informerade om att revisiorerna från 2018 har inte lämnat någon revisionsberättelse eftersom det inte finns något fullständigt resultat än.

Edvard Carlsson \ypa bordlägga punkten till nästa sektionsmöte.

\Mbaby

%\textbf{\Mba lägga revisionsberättelsen 2018 till handlingarna.}

\p{18}{Styrelsens förslag till resultatdisposition}{}
\textit{Punkten ströks från handlingarna.}

\p{19}{Uttag ur sektionens fonder sedan förra terminsmötet}{}
Förvaltningschef Henrik Ramström berättade om uttag ur sektionens fonder sedan förra terminsmötet.

\p{20}{Frågan om ansvarsfrihet för 2018}{}

  %  \begin{paragrafer}
  %      \subp{A}{Funktionärer}{}
  %      \textbf{\Mba finna funktionärerna 2018 ansvarsfria.}
  %      \subp{B}{Utskott}{}
  %      \textbf{\Mba finna utskotten 2018 ansvarsfria.}
  %      \subp{C}{Styrelse}{}
  %      \textbf{\Mba finna styrelsen 2018 ansvarsfria.}
  %      \subp{D}{Revisorer}{}
  %      \textbf{\Mba finna revisorerna 2018 ansvarsfria.}
  %      \subp{E}{Valberedning}{}
  %      \textbf{\Mba finna valberedningen 2018 ansvarsfria.}
  %  \end{paragrafer}


  Ordförande Edvard Carlsson \ypa bordlägga punkten till nästa sektionsmöte. Detta på grund av att revisorerna inte har 
    tillräckligt mycket att gå på eftersom bokslutet inte är färdigt.

\Mbaby

Edvard Carlsson \ypa 30 minuters matpaus kl.18.48. 

\Mbaby

Mötet återupptogs igen kl.19.23.

\p{21}{Behandling av motioner}{}
  \begin{paragrafer}
    \subp{L}{Ta bort NollU:s representant i valberedningen}{}
    Edvard Carlsson presenterade motionen.

    Edvard Carlsson presenterade styrelsens svar på motionen. 

    \textbf{\Mba bifalla motionen i sin helhet.}

    \subp{A}{Inköp av utrustning för Elektro Banana Band}{}
    William Sjödin och Daniel Bakic presenterade motionen.

    Saga Åslund presenterade styrelsens svar. 

    Adam Belfrage tyckte att PA ska rensas på allt gammalt om nytt ska köpas in.

    Emil P. Lundh informerade om att Teknokraterna kommer gå igenom PA så snart som i nästa vecka.

    Max Mauritsson undrade om man inte kollat på en modellerare istället som tar mycket mindre plats.

    William Sjödin svarade med att den specifierade basförstärkaren blivit rekommenderad från externa parter.
    
    \textbf{\Mba bifalla motionen i sin helhet.}

    \subp{B}{Renovering av Biljard}{}
    Adam Belfrage presenterade motionen.

    Henrik Ramström presenterade styrelsens svar.

    Mötet uttryckte sig positivt kring motionen.

    \textbf{\Mba bifalla motionen i sin helhet.}

    \subp{C}{Cafémästeriet}{}
    Emil Eriksson presenterade motionen.

    Jonathan Benitez presenterade styrelsens svar.

    Mötet diskuterade motionen.

    Talman Niklas Gustafson \ypa dra streck i debatten.

    \Mbaby

    \textbf{\Mba avslå motionen i sin helhet.}

    \subp{D}{Router och Switchar för DreamHackE}{}
    Vincent Palmer presenterade motionen.

    Edvard Carlsson presenterade styrelsens svar.

    \textbf{\Mba bifalla motionen i sin helhet.}

    \subp{E}{Budgetjustering för HeHE}{}
    Max Mauritsson presenterade motionen.

    Edvard Carlsson presenterade styrelsens svar.

    \textbf{\Mba bifalla motionen i sin helhet.}

    \subp{F}{Inköp av ny ljudteknik}{}
    Teknokraterna presenterade motionen.

    Adam Belfrage frågade hur användarvänligt allt är.

    Emil P. Lundh svarade att det är lättanvändligt.

    Markus Rahne frågade lite om vad som är fel med det analoga system som vi har nu.

    Jakob Pettersson undrade om det är allt eller inget som gäller i motionen.

    Emil P. Lundh svarade med att det som är mest akut är mixer med tillhörande tablet.

    Adam Belfrage \ypa höja budgeten till 19 000 för att lika bra kunna köpa in flera mikrofoner.

    Motionärerna jämkade sig med Adam Belfrages yrkande.

    \textbf{\Mba bifalla motionen med ändringsyrkandet.}

    
    \subp{G}{Tillägg i Policybeslut: Sektionens medaljer och dess utdelning}{}
    \textit{Motionären drog tillbaka sin motion.}

    Mötet ajournerades i 10 minuter.

    \subp{H}{Ändring av hur Sektionen väljer NollU-funktionärer}{}
    Henrik Ramström \ypa att diskussionen för de kommande två punkterna ska hållas till max. 20 min 
    per punkt.

    \Mbaby

    Edvard Carlsson presenterade motionen.

    Edvard Carlsson presenterade även styrelsens svar.

    Mötet diskuterade frågan.

    Edvard \ypa under 10:2:L under Øverphøsare stryka ``Ansvarar för rekryteringen av Co-phøs och Øvergudsphaddrar''.

    \textbf{\Mba bifalla motionen med tilläggsyrkandet.} 

    \subp{I}{Uppdatering av postbeskrivning för Øvergudphadder}{}
    Edvard Carlsson presenterade motionen. 

    Edvard Carlsson jämkade sig med styrelsens yrkande i styrelsens svar på motionen.

    Mötet beslutar att ta det på acklamation.

    \textbf{\Mba bifalla motionen med styrelsens ändringsyrkande.}

    \subp{J}{Utforma en funktionärspost för spektalek, Spexmästaren}{}
    Adam Ekblom presenterade motionen.

    Theo Nyman presenterade styrelsens svar. 

    Mötet diskuterade möjligheterna att lägga posten med spexansvar på andra ställen i utskotten som 
        till exempel under Stridsrop. 
    
    Talman Niklas Gustafson \ypa dra streck i debatten. 

    \textbf{\Mba avslå motionen i sin helhet. }

    \subp{K}{Stav till Entertainer}{}
    Adam Belfrage presenterade motionen.

    Mötet diskuterade motionen.

    Rasmus Sobel \ypa dra streck i debatten.

    \Mbaby

    \textbf{\Mba avslå motionen i sin helhet.}

    Daniel Bakic \ypa på att ajournera mötet i 10 min.

    Stephanie Bol \ypa ajournera mötet i 5 minuter.

    Daniel Bakic jämkade sig med yrkandet.

    \Mba bifalla yrkandet.

    Mötet ajournerades i 5 minuter.

\end{paragrafer}
\p{22}{Behandling av propositioner}{}
    \begin{paragrafer}
      \subp{A}{Budgetjustering för källarmästeriet och sexmästeriet}{}
      Krögare Davida Åström presenterade propostitionen.

      Sophia Carlsson undrade om det räcker med 5000 kr.

      Andreas Bennström undrade om man fortfarande kan subventionera NollEgasquen för nollor då.

      Davida Åström svarade att sexmästeriet går väldigt mycket över budget trots subventionering.

      \textbf{\Mba bifalla propositionen i sin helhet.}

      \subp{B}{Reglementesändring, postbeskrivning Fritidsledare}{}
      Entertainer Saga Åslund presenterade propositionen.

      \textbf{\Mba bifalla propostionen i sin helhet.}

      \subp{C}{Inköp av iZettle scanners}{}
      Krögare Davida Åström presenterade propositionen.

      Mötet diskuterade propositionen.

      \textbf{\Mba bifalla propositionen i sin helhet.}

      \subp{D}{Reglementesändring, mindre uppdateringar}{}
    Ordförande Edvard Carlsson presenterade propositionen. 
        
    \textbf{\Mba bifalla propositionen i sin helhet.}

    Sexmästare Theo Nyman \ypa riva upp beslutet för \S22 a) ``Budgetjustering för källarmästeriet och sexmästeriet''.

    \Mba bifalla yrkandet.

    Krögare Davida Åström \ypa ändra från 6000 till 20 000 i propositionen ``Budgetjustering för källarmästeriet och sexmästeriet''.
    
    \textbf{\Mba bifalla propositionen med ändringsyrkandet.}

      \subp{E}{Ge styrelsen rättigheter att genomföra redaktionella ändringar i styrdokumenten}{}
      Ordförande Edvard Carlsson presenterade propositionen.

      Mötet diskuterade frågan.

      \textbf{\Mba bifalla propositionen i sin helhet.}

      %Måste gå igenom nästa sektionsmöte med
      \subp{F}{Reglementesändring, införande av posten FilmarE}{}
      Kontaktor Sonja Kenari presenterade propositionen.

      \textbf{\Mba bifalla propositionen i sin helhet.}

      \subp{G}{Reglementesändring, Suppleanter till Øverphøset}{}
      Cafémästare Jonathan Benitez presenterade propositionen. 

      Mötet diskuterade propositionen.

      \textbf{\Mba bifalla propositionen i sin helhet.}

      %propositionen gick igenom detta möte.
      \subp{H}{Reglementesändring, postbeskrivning Projektgrupp Teknikfokus}{}
      Ordförande för näringslivsutskottet Jakob Pettersson presenterade propositionen.

        \textbf{\Mba bifalla propositionen i sin helhet.}

      \subp{I}{Inköp av cykelvagn}{}
      Krögare Davida Åström presenterade propositionen.

      \textbf{\Mba bifalla propositionen i sin helhet.}

      \subp{J}{Omstrukturering av budgeten för medaljer}{}
      Förvaltningschef Henrik Ramström presenterade propositionen.

      Henrik Ramström \ypa sänka budgeten till -1700 kr.

      \textbf{\Mba bifalla propositionen med ändringsyrkandet.}

      \subp{K}{Reglementesändring, alumniverksamheten till näringslivsutskottet}{}
      Ordförande för näringslivsutskottet Jakob Pettersson presenterade propositionen.

        \textbf{\Mba bifalla propositionen i sin helhet.}

      \subp{L}{Reglementesändring, postbeskrivningar i informationsutskottet}{}
      Kontaktor Sonja Kenari presenterade propositionen.

      \textbf{\Mba bifalla propositionen i sin helhet.}

      \subp{M}{Tygmärken för phaddergrupper}{}
       Øverphøs Stephanie Bol presenterade propostionen.

      Mötet diskuterade propostionen.

      \textbf{\Mba bifalla propositionen i sin helhet.}

      \subp{N}{Reglementesändring, postbeskrivningar i källarmästeriet och sexmästeriet}{}
      Sexmästare Theo Nyman presenterade propositionen.

      \textbf{\Mba bifalla propositionen i sin helhet.}

      \subp{O}{Inköp av kameratillbehör}{}
      Kontaktor Sonja Kenari presenterade propositionen.

      \textbf{\Mba bifalla propostionen i sin helhet.}

  \end{paragrafer}
\p{23}{Övrigt}{}
\p{24}{TaFMA}{}
Talman {\mo} förklarade mötet avslutat kl.22.51.

\end{paragrafer}

%\newpage
\hidesignfoot
\begin{signatures}{4}
\signature{\mo}{Mötesordförande}
\signature{\ms}{Mötessekreterare}
\signature{\ji}{Justerare}
\signature{\jii}{Justerare}
\end{signatures}
\end{document}
