\documentclass[10pt]{article}
\usepackage[utf8]{inputenc}
\usepackage[swedish]{babel}

\def\mo{Edvard Carlsson}
\def\ms{Mattias Lundström}
\def\ji{Theo Nyman}
%\def\jii{}

\def\doctype{Protokoll} %ex. Kallelse, Handlingar, Protkoll
\def\mname{Styrelsemöte} %ex. styrelsemöte, Vårterminsmöte
\def\mnum{S28/19} %ex S02/16, E1/15, VT/13
\def\date{2019-12-02} %YYYY-MM-DD
\def\docauthor{\ms}

\usepackage{../e-mote}
\usepackage{../../../e-sek}

\begin{document}
\showsignfoot

\heading{{\doctype} för {\mname} {\mnum}}

%\naun{}{} %närvarane under
%\nati{} %närvarande till och med
%\nafr{} %närvarande från och med
\section*{Närvarande}
\subsection*{Styrelsen}
\begin{narvarolista}
\nv{Ordförande}{Edvard Carlsson}{E16}{}
\nv{Kontaktor}{Mattias Lundström}{E17}{}
\nv{Förvaltningschef}{Henrik Ramström}{E16}{}
\nv{Cafémästare}{Jonathan Benitez}{E17}{\nafr{10C}} 
\nv{Sexmästare}{Theo Nyman}{BME18}{}
\nv{Krögare}{Davida Åström}{BME17}{}
\nv{Entertainer}{Saga Åslund}{BME18}{}
\nv{SRE-ordförande}{Lina Samnegård}{BME16}{}
\nv{ENU-ordförande}{Jakob Pettersson}{E17}{}
\nv{Øverphøs}{Stephanie Bol}{BME17}{}
\end{narvarolista}


\subsection*{Ständigt adjungerande}
\begin{narvarolista}
%\nv{}{}{}{}
%\nv{Skattmästare}{Daniel Bakic}{E15}{\nafr{10}}
%\nv{Vice Krögare}{Klara Indebetou}{BME17}{}
%\nv{Vice Krögare}{Hjalmar Tingberg}{BME16}{}
\nv{Kårrepresentant}{Ivar Vänglund}{}{}
\nv{Vice Förvaltningschef}{Rasmus Sobel}{BME16}{}
%\nv{Kårrepresentant}{Martin Bergman}{}{}
%\nv{Valberedningens ordförande}{Axel Voss}{E15}{\nafr{10b}}
%\nv{Fullmäktigeledamot}{Magnus Lundh}{E15}{\nafr{12}}
\nv{Chefredaktör}{Emil Eriksson}{E18}{}
%\nv{Inspektor}{Monica Almqvist}{}{}


\end{narvarolista}

%\begin{comment}
\subsection*{Adjungerande}
\begin{narvarolista}
%\nv{post}{namn}{klass}{nati/nafr/tom}
%\nv{Likabehandlingsombud}{Jonna Fahrman}{BME17}{}
%\nv{Likabehandlingsombud}{Hanna Bengtsson}{BME18}{}
%\nv{Projekfunktionär}{Emma Hjörneby}{BME17}{}
%\nv{Macapär}{Filip Larsson}{E17}{}
%\nv{Kodhackare}{Vincent Palmer}{E18}{}
\nv{Sexmästare Electus}{Anna Hollsten}{BME19}{}
\nv{Entertainer Electus}{Amir Ghanaatifard}{E19}{}
\nv{Øverphøs Electus}{Sophia Carlsson}{BME17}{}
\nv{Cafemästare Electus}{Frida Pilcher}{E18}{}
\nv{ENU-ordförande Electus}{Fredrik Berg}{E17}{}
\nv{Krögare Electus}{Love Sjelvgren}{E18}{}
\nv{SRE-ordförande Electus}{Hanna Bengtsson}{BME18}{}
\nv{Husstyrelserepresentant}{Joakim Magnusson Fredluend}{BME19}{}
\nv{Sigillbevarare}{Matilda Horn}{BME18}{\nati{14}}

\end{narvarolista}
%\end{comment}

\section*{Protokoll}
\begin{paragrafer}
\p{1}{OFMÖ}{\bes}
Ordförande {\mo} förklarade mötet öppnat kl 12.12

\p{2}{Val av mötesordförande}{\bes}
{\valavmo}

\p{3}{Val av mötessekreterare}{\bes}
{\valavms}

\p{4}{Val av justeringsperson}{\bes}
{\valavj}

\p{5}{Godkännande av tid och sätt}{\bes}
{\tosg}

\p{6}{Adjungeringar}{\bes}
%Adam Belfrage adjungerades.{}
%Hanna Bengtsson adjungerades. \\
%Jonna Fahrman adjungerades.
%Vincent Palmer adjungerades.\\
%Filip Larsson adjungerades. 

Anna Hollsten adjungerades. 

Amir Ghanaatifard adjungerades.

Sophia Carlsson adjungerades.

Fredrik Berg adjungerades.

Frida Pilcher adjungerades.

Rasmus Solel adjungerades.

Love Sjelvgren adjungerades.

Hanna Bengtsson adjungerades. 

Joakim Magnusson Fredluend adjungerades.

Matilda Horn adjungerades.
%\textit{Inga adjungeringar.}


\p{7}{Godkännande av dagordningen}{\bes}

%Davida \ypa lägga till punkten ``Lophtet'' till dagordningen.\\
%Edvard \ypa lägga till punkten ``Ordensband'' til dagordningen.
%Fredrik \ypa att lägga till \S18b ``Teknikfokus utnyttjande av LED-café''.
%Jonathan \ypa ändra punkten §12 från att vara en beslutspunkt till diskussion. \\
%Föredragningslistan godkändes med yrkandet.
%Henrik \ypa lägga till punkten ``Faktura till F'' som §13.
%Jakob Pettersson \ypa tägga till punkten ''Øverphøs informerar'' som \S16.

Föredragningslistan godkändes med yrkandet.

\p{8}{Föregående mötesprotokoll}{\bes}
\latillprotgodkand{S25/19}
%\textit{\ingaprot}

\p{9}{Fyllnadsval och entledigande av funktionärer}{\bes}
\begin{fyllnadsval} %"Inga fyllnadsval." fylls i automatiskt
%\fval{Moa Rönnlund}{Halvledare}
%\entl{Fanny Månefjord}{Husstyrelserepresentant från och med 30 juni}
\fval{Klara Wahldén}{Inköps och Lagerchef från och med 1 januari 2020}


\end{fyllnadsval}

\p{10}{Rapporter}{}
\begin{paragrafer}
\subp{A}{Hur mår alla?}{\info}
Mötet mår överlag bra. Edvard fryser för att han har börjat sin årliga fasta. 

%Punkten protokollfördes ej.

\subp{B}{Utskottsrapporter}{\info}

Jonathan och Cafémästeriet har myst hela veckan samt fixat glögg och varm choklad till sektionsmöte.
De har även sålt kaffe till Teknikfokus. Utöver det har allt rullat på.

Henrik och FVU har behandlat vinstfördelning sen nollningen, sett över kostnadsställe för sektionsmöte samt betalat fakturor. Utöver det har det allmänt varit ekonomiskt arbete. 

Mattias och skrivit protokoll och gjort klart mycket efterarbete av HTM och Valmötet. Utskottsmässigt fortsätter InfU som vanligt. Fotograf är spikad till Nobelsittningen och Julgillet. HeHE fortsätter och ska släppa ett numer till. Macapärerna gör klart det sista inför byte av mailtjänst till G Suite. 

Davida och KM har släppt anmälan till Julgillet och ordnat lite inför Glöggillet på fredag. Utöver det har Davida bokfört och hjälp KM20 att komma ingång. 

Stephanie och NollU har börjat med sin överlämning till nya NollU. Utöver det forsätter NollU med sina testamenten. 

Jakob och ENU har marknadsfört sitt sista event. De har haft möte med näringslivskontakter och gick igenom överlämning. I fredags var det lunchföreläsning med Academic Work som gick bra där de nya invalda fick jobba. Utöver det har Jakob jagat fakturor och imorgon ska ENU på nationspub. Till sist är det casekväll med ALTEN på onsdag.

Saga och NöjU har planerat NöjU-veckan och bestämt kickout för NöjU-19. Det blir middag på VGs. Imorgon är det spelkväll, på torsdag är det klättring med idrottsförmännen och på lördag är det Sångarstriden. 

Theo och Sexmästeriet har planerat sista sittningen för året.

Lina och SRE har censuretat klart alla CEQ-enkäter och nästa vecka är det dags för möten. Utöver det är det Speak Up-Days i foajen med Kåren och D-sektionen.
\subp{C}{Ekonomisk rapport}{\info}
 
Henrik väntar fortfarande på en hel del kvitton och han ska göra en sammanställning av vilka kvitton som inte kommit in sedan november. Bra om allt är inne innan jul. 

I övrigt ser ekonomin bra ut. 

\subp{D}{Kåren informerar}{\info}

Ivar meddelade att planering inför sångarstriden är i full gång och att man fortfarande kan köpa biljetter. Speak Up Days fortsätter också i Cornelis. 

Ivar meddelade också att studentkortet har begärt konkurs vilket berör främst nationerna. Ivar meddelade också att Mecenat intygat att deras ekonomi är god. 

\end{paragrafer}

\p{11}{Ansökan om att bli projektfunktionär}{\bes}
Henrik presenterade sin ansökan om att bli projektfunktionär i det pågående projektet 'Utredning om renovering av toaletterna'. 

Edvard tyckte det är kul om han inte är helt själv. 

\Mbaby 

\p{12}{Medaljutdelning}{\dis}

Mötet diskuterade medaljutdelning. 

Mötet beslutade att dela ut två bidragsmedaljer. 

\p{13}{Val av funktionärer}{\dis}
Saga presenterade diskussionunderlaget. Problemet sektionen har är att val av funktionärer tar väldigt lång tid. 

Mötet diskuterade för- och nackdelar av sektionens nuvarande valprocess och andra tillvägagångssätt så som att välja fler poster utanför ett valmöte. 

Diskussionen förs vidare till nästa år styrelse. 

\p{14}{Serveringstillstånd alla dagar i veckan}{\dis}

Theo framförde diskussionen om att ha ett allmänt tillstånd alla dagar i veckan. Det skulle innebära samma kostnad för hela veckan istället för bara fredag och lördag.

Theo menar att det kan öppna upp möjligheten för spontana pubbar utan att söka tillstånd varje gång.

Stephanie var bekymrad hur detta påverkar sektionens användning av FikaFika.

Edvard svarade att D-sektionen också har serveringstillstånd på onsdagar men att ingen vet exakt hur serveringstillstånd påverkar FikaFika. 

Henrik tyckte vi ska vara akstamma då vi kan straffas om FikaFika missköts. 

Saga undrade hur det skulle underlätta för Sexmästeriet och KM. 

Davida svarade att ansökan om serveringstillstånd är relativt enkelt. Theo instämde men poängterade att det skulle spara pengar. 


%Edvard - hårdare med alkohol i edekvata. Boka fika fika är ett problem. Davida svårt att vara privat när man bokar fika fika. Är ofta egentligen ett sektionsengagemang. 

%(Henrik - så som det ser ut nu så är det positivt att inte tillstånd är här hela tiden. )

\p{15}{Nästa styrelsemöte}{\bes}
\Mba nästa styrelsemöte ska äga rum 2019-12-09 12.10 i E:1123.

\p{16}{Beslutsuppföljning}{\bes}
%Edvard \ypa stryka ''Projektfunktionär: Vårbal'' från Beslutsuppföljning. Liknande projekt uppmuntras.
%\Mbaby
%Davida \ypa skjuta upp ''Inköp av draghandtag till cykelvagn'' till nästa styrelsemöte.
%\Mbaby
\textit{Inga beslut att följa upp.}
\p{17}{Övrigt}{\dis}

Saga undrade var tavlorna som sitter i Diplomat just nu skall flyttas. Edvard svarade att de kan slängas. 

Henrik meddelade att han hittat skit i Blå Dörren och hälften av allt är Niklas som åker skidor. 

Edvard informerade om att det är testamenteskrivarkväll.

Joakim rapporterade att Per-Henrik fortfarande väntar på svar från företag som tillverkat källsorteringsstation. 

\p{18}{Sammanfattning av mötet}{\info}
% \Fbs
% \Fbsup
Klara Wahldén valdes till Inköps och Lagerchef 2020. 

Henrik Ramström valdes till Projektfuntionär i projektet 'Utredning om renovering av toaletterna'.

Mötet diskuterade medaljutdelning, tillvägagångssätt vid val av funktionärer samt serveringstillstånd i sektionens lokaler. 


\p{19}{OFMA}{\bes}
{\mo} förklarade mötet avslutat kl. 13.03
\end{paragrafer}

%\newpage
\hidesignfoot
\begin{signatures}{3}
\signature{\mo}{Mötesordförande}
\signature{\ms}{Mötessekreterare}
\signature{\ji}{Justerare}
\end{signatures}
\end{document}
