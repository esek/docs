\documentclass[10pt]{article}
\usepackage[utf8]{inputenc}
\usepackage[swedish]{babel}

\def\mo{Edvard Carlsson}
\def\ms{Mattias Lundström}
\def\ji{-----}
%\def\jii{}

\def\doctype{Protokoll} %ex. Kallelse, Handlingar, Protkoll
\def\mname{Styrelsemöte} %ex. styrelsemöte, Vårterminsmöte
\def\mnum{S16/19} %ex S02/16, E1/15, VT/13
\def\date{2019-08-20} %YYYY-MM-DD
\def\docauthor{\ms}

\usepackage{../e-mote}
\usepackage{../../../e-sek}

\begin{document}
\showsignfoot

\heading{{\doctype} för {\mname} {\mnum}}

%\naun{}{} %närvarane under
%\nati{} %närvarande till och med
%\nafr{} %närvarande från och med
\section*{Närvarande}
\subsection*{Styrelsen}
\begin{narvarolista}
\nv{Ordförande}{Edvard Carlsson}{E16}{}
\nv{Kontaktor}{Mattias Lundström}{E17}{}
\nv{Förvaltningschef}{Henrik Ramström}{E16}{}
\nv{Cafémästare}{Jonathan Benitez}{E17}{}
\nv{Sexmästare}{Theo Nyman}{BME18}{}
\nv{Krögare}{Davida Åström}{BME17}{}
\nv{Entertainer}{Saga Åslund}{BME18}{}
\nv{SRE-ordförande}{Lina Samnegård}{BME16}{}
\nv{ENU-ordförande}{Jakob Pettersson}{E17}{}
\nv{Øverphøs}{Stephanie Bol}{BME17}{}
\end{narvarolista}


\subsection*{Ständigt adjungerande}
\begin{narvarolista}
%\nv{}{}{}{}
%\nv{Skattmästare}{Daniel Bakic}{E15}{\nafr{10}}
%\nv{Vice Krögare}{Klara Indebetou}{BME17}{}
%\nv{Vice Krögare}{Hjalmar Tingberg}{BME16}{}
%\nv{Kårrepresentant}{Filip Johansson}{}{\nafr{10A}}
%\nv{Kårrepresentant}{Anna Qvil}{}{}
%\nv{Valberedningens ordförande}{Axel Voss}{E15}{\nafr{10b}}
%\nv{Fullmäktigeledamot}{Magnus Lundh}{E15}{\nafr{12}}
%\nv{Chefredaktör}{Max Mauritsson}{BME16}{}
%\nv{Inspektor}{Monica Almqvist}{}{}


\end{narvarolista}

%\begin{comment}
\subsection*{Adjungerande}
\begin{narvarolista}
%\nv{post}{namn}{klass}{nati/nafr/tom}
%\nv{Likabehandlingsombud}{Jonna Fahrman}{BME17}{}
%\nv{Likabehandlingsombud}{Hanna Bengtsson}{BME18}{}
%\nv{Projekfunktionär}{Emma Hjörneby}{BME17}{}
%\nv{}{}{}{}
\end{narvarolista}
%\end{comment}

\section*{Protokoll}
\begin{paragrafer}
\p{1}{OFMÖ}{\bes}
Ordförande {\mo} förklarade mötet öppnat kl.12.11.

\p{2}{Val av mötesordförande}{\bes}
{\valavmo}

\p{3}{Val av mötessekreterare}{\bes}
{\valavms}

\p{4}{Val av justeringsperson}{\bes}
{\valavj}

\p{5}{Godkännande av tid och sätt}{\bes}
{\tosg}

\p{6}{Adjungeringar}{\bes}
%Adam Belfrage adjungerades.{}
%Hanna Bengtsson adjungerades. \\
%Jonna Fahrman adjungerades.


%\textit{Inga adjungeringar.}


\p{7}{Godkännande av dagordningen}{\bes}

%Davida \ypa lägga till punkten ``Lophtet'' till dagordningen.\\
%Edvard \ypa lägga till punkten ``Ordensband'' til dagordningen.
%Fredrik \ypa att lägga till \S18b ``Teknikfokus utnyttjande av LED-café''.
%Jonathan \ypa ändra punkten §12 från att vara en beslutspunkt till diskussion. \\
%Föredragningslistan godkändes med yrkandet.
%Henrik \ypa lägga till punkten ``Faktura till F'' som §13.

Föredragningslistan godkändes med yrkandet.

\p{8}{Föregående mötesprotokoll}{\bes}
\latillprot{S14/19 \& S15/19}
%\textit{\ingaprot}

\p{9}{Fyllnadsval och entledigande av funktionärer}{\bes}
\begin{fyllnadsval} %"Inga fyllnadsval." fylls i automatiskt
%\fval{Moa Rönnlund}{Halvledare}
%\entl{Fanny Månefjord}{Husstyrelserepresentant från och med 30 juni}
\end{fyllnadsval}

\p{10}{Rapporter}{}
\begin{paragrafer}
%\subp{A}{Hur mår alla?}{\info}
%Punkten protokollfördes ej.

\subp{A}{Utskottsrapporter}{\info}
%CM har köpt in lite nya saker till caféet. Förra veckan blev det även ett överköp i inköp av gurka men det löstes bra.

%FVU har fakturerat och bokfört. Utskottet har även haft lite uppföljning över hur det gått denna terminen samt vad som kan förbättras.

%KM börjar komma igång med sina nya iZettles. Terminens sista gille har planerats och hålls på fredag. Gilleplaneringen för nollningen är också igång.

%NollU har haft kontakt med SVEP och lokalfrågan löste sig. Uppdrags- och phadderkickoffen gick jättebra. Idag väntar möte med SVL och under veckan är det lunchmöte med resterande phaddergrupper. 


\subp{B}{Ekonomisk rapport}{\info}
%Det är inte så mycket som hände ekonomisk under omtentaveckorna så det mesta är under kontroll.

\subp{C}{Kåren informerar}{\info}
%Det är fullmäktigemöte på torsdag och på söndag är det EU-val.  

\subp{D}{Omvärldsrapport}{\info}
%Styrelsen var på V-sektionens jubileumsbal. 

\end{paragrafer}

\p{11}{Äskning av pengar för inköp av draghandtag till cykelvagn}{\bes}
Davida presenterade motionen.

\Mbaby 

\p{12}{Mer punker från kallelse..?}{\dis}

\p{14}{Nästa styrelsemöte}{\bes}
\Mba nästa styrelsemöte ska äga rum 2019-05-27 kl.12.10 i E:1123.

\p{15}{Beslutsuppföljning}{\bes}
Beslutsuppföljningen är tom.

\p{16}{Övrigt}{\dis}
%Lina har fått lite uppgifter angående flytten till Brunnshög då en viss del av undervisningen ska flyttas dit. Philip informerar om att fullmäktige har uttalat sig i frågan och att sektionerna representeras i fullmäktige också. Lina har blivit inbjuden till att delta på fler diskussionsmöten kring frågan. Det är redan beslutat att forskningsnivån av Nano kommer att flytta till Brunnshög, frågan är hur mycket av grundutbildningen som kommer flyttas upp dit. Vilket gör att det kan påverka BME framöver. Mötet diskuterade frågan.
%Edvard påminde om att fylla i tider för utvärderingsstunder för styrelsen samt om undersökningsenkäten för att kunna förbättra medlemskvalitéen på sektionen.

\p{15}{OFMA}{\bes}
{\mo} förklarade mötet avslutat kl. 
\end{paragrafer}

%\newpage
\hidesignfoot
\begin{signatures}{3}
\signature{\mo}{Mötesordförande}
\signature{\ms}{Mötessekreterare}
\signature{\ji}{Justerare}
\end{signatures}
\end{document}
