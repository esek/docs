\documentclass[10pt]{article}
\usepackage[utf8]{inputenc}
\usepackage[swedish]{babel}

\def\mo{Edvard Carlsson}
\def\ms{Mattias Lundström}
\def\ji{Jonathan Benitez}
%\def\jii{}

\def\doctype{Protokoll} %ex. Kallelse, Handlingar, Protkoll
\def\mname{Styrelsemöte} %ex. styrelsemöte, Vårterminsmöte
\def\mnum{S16/19} %ex S02/16, E1/15, VT/13
\def\date{2019-08-20} %YYYY-MM-DD
\def\docauthor{\ms}

\usepackage{../e-mote}
\usepackage{../../../e-sek}

\begin{document}
\showsignfoot

\heading{{\doctype} för {\mname} {\mnum}}

%\naun{}{} %närvarane under
%\nati{} %närvarande till och med
%\nafr{} %närvarande från och med
\section*{Närvarande}
\subsection*{Styrelsen}
\begin{narvarolista}
\nv{Ordförande}{Edvard Carlsson}{E16}{}
\nv{Kontaktor}{Mattias Lundström}{E17}{}
\nv{Förvaltningschef}{Henrik Ramström}{E16}{}
\nv{Cafémästare}{Jonathan Benitez}{E17}{}
\nv{Sexmästare}{Theo Nyman}{BME18}{}
\nv{Krögare}{Davida Åström}{BME17}{}
\nv{Entertainer}{Saga Åslund}{BME18}{}
%\nv{SRE-ordförande}{Lina Samnegård}{BME16}{}
\nv{ENU-ordförande}{Jakob Pettersson}{E17}{}
\nv{Øverphøs}{Stephanie Bol}{BME17}{}
\end{narvarolista}


\subsection*{Ständigt adjungerande}
\begin{narvarolista}
%\nv{}{}{}{}
%\nv{Skattmästare}{Daniel Bakic}{E15}{\nafr{10}}
%\nv{Vice Krögare}{Klara Indebetou}{BME17}{}
%\nv{Vice Krögare}{Hjalmar Tingberg}{BME16}{}
%\nv{Kårrepresentant}{Filip Johansson}{}{\nafr{10A}}
%\nv{Kårrepresentant}{Anna Qvil}{}{}
%\nv{Valberedningens ordförande}{Axel Voss}{E15}{\nafr{10b}}
%\nv{Fullmäktigeledamot}{Magnus Lundh}{E15}{\nafr{12}}
%\nv{Chefredaktör}{Max Mauritsson}{BME16}{}
%\nv{Inspektor}{Monica Almqvist}{}{}


\end{narvarolista}

%\begin{comment}
\subsection*{Adjungerande}
\begin{narvarolista}
%\nv{post}{namn}{klass}{nati/nafr/tom}
%\nv{Likabehandlingsombud}{Jonna Fahrman}{BME17}{}
%\nv{Likabehandlingsombud}{Hanna Bengtsson}{BME18}{}
%\nv{Projekfunktionär}{Emma Hjörneby}{BME17}{}
\nv{Macapär}{Filip Larsson}{E17}{}
\nv{Kodhackare}{Vincent Palmer}{E18}{}
\end{narvarolista}
%\end{comment}

\section*{Protokoll}
\begin{paragrafer}
\p{1}{OFMÖ}{\bes}
Ordförande {\mo} förklarade mötet öppnat kl.12.14.

\p{2}{Val av mötesordförande}{\bes}
{\valavmo}

\p{3}{Val av mötessekreterare}{\bes}
{\valavms}

\p{4}{Val av justeringsperson}{\bes}
{\valavj}

\p{5}{Godkännande av tid och sätt}{\bes}
{\tosg}

\p{6}{Adjungeringar}{\bes}
%Adam Belfrage adjungerades.{}
%Hanna Bengtsson adjungerades. \\
%Jonna Fahrman adjungerades.
Vincent Palmer adjungerades.\\
Filip Larsson adjungerades. 

%\textit{Inga adjungeringar.}


\p{7}{Godkännande av dagordningen}{\bes}

%Davida \ypa lägga till punkten ``Lophtet'' till dagordningen.\\
%Edvard \ypa lägga till punkten ``Ordensband'' til dagordningen.
%Fredrik \ypa att lägga till \S18b ``Teknikfokus utnyttjande av LED-café''.
%Jonathan \ypa ändra punkten §12 från att vara en beslutspunkt till diskussion. \\
%Föredragningslistan godkändes med yrkandet.
%Henrik \ypa lägga till punkten ``Faktura till F'' som §13.
Jakob Pettersson \ypa tägga till punkten ''Øverphøs informerar'' som \S16.

Föredragningslistan godkändes med yrkandet.

\p{8}{Föregående mötesprotokoll}{\bes}
%\latillprot{S14/19 \& S15/19}
\textit{\ingaprot}

\p{9}{Fyllnadsval och entledigande av funktionärer}{\bes}
\begin{fyllnadsval} %"Inga fyllnadsval." fylls i automatiskt
%\fval{Moa Rönnlund}{Halvledare}
%\entl{Fanny Månefjord}{Husstyrelserepresentant från och med 30 juni}
\entl{Magnus Lund}{Arkivarie}
\entl{Jennie Karlsson}{Näringslivskontakt}
\entl{Moa Rönnlund}{Hustomte}
\fval{Sonja Kenari}{Arkivarie}
\fval{Vincent Palmer}{Hustomte}

\end{fyllnadsval}

\p{10}{Rapporter}{}
\begin{paragrafer}
\subp{A}{Hur mår alla?}{\info}
%Punkten protokollfördes ej.
Alla närvarande mår bra och har haft en bra sommar. Henrik Ramström sålde nästan sin bil. 

\subp{A}{Utskottsrapporter}{\info}
Ordförande har haft ett välbehövt sommarlov men också förberett lite inför kommande terminen. 
Har deltagit på utbildning om hantering av sexuella trakasserier och sammanställt undersökningarna som gjorde i våras. 
Förberett inför HTM och börjat jobba på en valbroschyr. Även planerat inför styrelseåterträffen ikväll. 

CM har haft det lugnt under sommaren. Jonathan har gjort beställningar, pantat samt 
snackat med olika utskott angående hur det blir enklast för CM att ta emot beställningar.

En sommar kommit och en sommar har gått. Förvaltning har försökt att sköta sina uppgifter så bra de kunnat. Svängdörren är kortad så nu slår den inte i väggen längre tack vare Fabian “sågen” Sondh. 
Sicrit var städat en sista gång av hustomtarna inför att nollningen, men det verkar redan vara lite rörigt där igen. HK har fått en omstrukturering och en genomstädning av Henrik och Edvard. 
Till sist så har FVU även preppat med ekonomi inför nollningen, bokfört och fakturerat. 

Mattias har satt sig in i styrelsens arbete, uppdaterat sig med InfU och börjat planera inför kommande utskott- och styrelsearbete. Posten FilmarE har lagts till i Sektionens system. Teknokrater har fått information angående nollningsevent. Nollningssidan är färdig och live. Hemsidan har allmänt blivit uppdaterad och nya omslagsbilder har lagts till. Större utskottsmöte är inplanerat. 

KM har under sommaren förberett inför kommande evenemang. KM har också jagat jobbare, bokat utrustning, lokaler och skickat iväg uppdrag till Picasso. 
Utskottet har bra kontakt med NollU vilket skapar bra underlag för lyckade nollningsevenemang med KM. 
Cykelkärran är efter mycket om och men både här och betald, tjohej! 


ENU har under sommaren haft mailkontakt med företag. Bland annat AXIS för sponsring av mikrovågsugnar, Venturelabs för inplanering av en workshop i höst och Knightec för inplanering av studiebesök på deras kontor ihop med D-sektionen. 
I övrigt har det varit planering och förberedande arbete inför nollningen och hösten. 
Jakob har haft ett omfattande möte med Sophia där nollning och hösten diskuterades. Ett möte med alumniansvariga är inplanerat. 


Saga och NöjU märker att nollningen närmar sig och Sagas största stressfaktor är just nu UtEDischot. Saga har haft möte med projektgruppen och ska informera samtliga Phös och dela ut band. 
Saga ska även träffa vices och delta på möte om Mega-Kul-fEstival. 


Theo har under sommaren planerat sittningarna under nollningen. Bokat lokaler, kollat på ritningar till tillståndsansökningar och mer. 
Theo tycker det ska bli kul med sittningar.

\subp{B}{Ekonomisk rapport}{\info}
Sektionen har fortsatt god ekonomi. Ökar fortfarande i omsättning vilket är bra. 
Sektionens resurser har satsats på saker som sektionens medlemmar. Vår största kostnad är funktionärsvård.  
Övrig notering är broms på kickoffer under hösten om det ska finnas budget över till funktionärs- och utskottstack. 

\subp{C}{Kåren informerar}{\info}
Kåren är upptagen med Arrival Day för de nya internationella studenterna.\\
Stormöte med kåren imorgon 2019-08-21 kl 13.15 i E:A.

\subp{D}{Omvärldsrapport}{\info}
%Styrelsen var på V-sektionens jubileumsbal. 

Styrelsen har blivit anmodade till en bal i Norge av Sct. Omega Broderskab, se \S15

\end{paragrafer}

\p{11}{Utbildning i hantering av sexuella trakasserier}{\info}
Edvard informerade att utbildningen var givande. 

\p{12}{Nya regler kring rökning}{\info}
Davida informerade om de nya reglerna kring rökning som började gälla 1 juli. Mötet diskuterade de nya reglerna. Henrik tror att det finns risk att folk kommer vilja röka i köer, till exempel vid UtEDischot eller andra tillställningar. 
Øverphøs meddelar att phaddrar kommer informeras på phadderinfot. Huset ansvarar för att skyltar sätts ut runt om i E-huset.


\p{13}{Uppföljning av Styrelseutvärderingen}{\dis}
Mötet diskuterade Styrelseutvärderingen kort. Den var givande. 

\p{14}{Uppföljning av Påverkansenkäten}{\dis}

Mötet diskuterade Påverkansenkäten. Bra och givande enkätsvar men få. Mötet diskuterade om man i framtiden skall ha Påverkansenkäter kontinuerligt under året. \\
Saga poängterade att rutin med återgärdsplaner bör införas till framtida Påverkansenkäter. 


\p{15}{Sct. Omega Broderskabs Nollegasque samt anmodningar till vår Nollegasque}{\dis}

Styrelsen har blivit anmodade till Sct. Omega Broderskabs Nollegasque 11 september men ingen styrelsemedlem kunde delta det datumet. 

Edvard berättade att styrelsen har blivit anmodade till Chalmers 4-6 oktober. 

Mötet diskuterade anmodan till Sektionens egna Nollegasque. I år har Sektionen fler nyantagna och Theo har räknat på hur många som kan anmodas. Mötet diskuterade hur platserna skulle disponeras.

\p{16}{Äskning av pengar för inköp av nollningströjor}{\bes}

Stephanie vill köpa in fler nollningströjor då det har blivit överintag av nya studenter. 

Henrik informerade om att motionen bryter mot budgeteringen och att det istället kan lösas internt mellan ENU och NollU. Mötet diskuterade detta. 

Mötet beslutade att avslå alla yrkanden. 

\p{17}{Øverphøs informerar}{\dis} 
Stephanie informerade om Sankt Hans. Nytt för i år är att internationella studenter är med.
Stephanie och styrelsen hoppas att det inte regnar. Stephanie poängterade också att styrelsen ska vara med och tagga studenter under färd till Sankt Hans. 

Stephanie informerade om Uppropsdagen. Styrelsen är med och hämtar nyantagna studenter. 

Stephanie informerade om Utskottssafarit. 

Stephanie informerade om Grill och Lek. Styrelsen är med och grillar korv. Det är styrelsen bra på.  

Stephanie informarade om Stadsvandringen. Mötet diskuterade styrelsens aktivitet. Aerobics? Minibränboll? E-spark? 


\p{18}{Nästa styrelsemöte}{\bes}
\Mba nästa styrelsemöte ska äga rum 2019-09-02 12.10 i E:1123.

\p{19}{Beslutsuppföljning}{\bes}

Edvard \ypa stryka ''Projektfunktionär: Vårbal'' från Beslutsuppföljning. Liknande projekt uppmuntras.

\Mbaby

Davida \ypa skjuta upp ''Inköp av draghandtag till cykelvagn'' till nästa styrelsemöte.

\Mbaby

Stephanie \ypa stryka ''Inköp av cheerdräkter'' till nästa styrelsemöte. 

\Mbaby

Henrik \ypa stryka ''Inköp av förvaringslådor till arkivet'' från Beslutsuppföljningen. 

\Mbaby

\p{20}{Övrigt}{\dis}
Mötet diskuterade styrelsens spexidéer till framtida sittningar. 

Saga föreslog spex om Sorority Girls som möttes av blandade åsikter. Mer spexidéer var bland annat Hakona Matata och ett Abba medley.  

Henrik tog upp frågan om inköp av nya iZettles från kåren och möjlighet kring uthyrning. Finns tydligen bra deals! Mötet diskuterade frågan. 

Henrik tog upp frågan om inköp av ny skrivare. Mötet är ej eniga om skrivarens nuvarande skick men skannern konstaterades vara sönder. Edvard tycker Sektionen borde köpa ny skrivare. Mötet konstaterade att det finns lite erfarenhet av skrivare bland sektionen och att det kommer klinga hårt i Henriks budget.


\p{21}{OFMA}{\bes}
{\mo} förklarade mötet avslutat kl. 13.32 
\end{paragrafer}

%\newpage
\hidesignfoot
\begin{signatures}{3}
\signature{\mo}{Mötesordförande}
\signature{\ms}{Mötessekreterare}
\signature{\ji}{Justerare}
\end{signatures}
\end{document}
