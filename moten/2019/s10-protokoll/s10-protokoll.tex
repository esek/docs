\documentclass[10pt]{article}
\usepackage[utf8]{inputenc}
\usepackage[swedish]{babel}

\def\mo{Edvard Carlsson}
\def\ms{Sonja Kenari}
\def\ji{Henrik Ramström}
%\def\jii{}

\def\doctype{Protokoll} %ex. Kallelse, Handlingar, Protkoll
\def\mname{Styrelsemöte} %ex. styrelsemöte, Vårterminsmöte
\def\mnum{S10/19} %ex S02/16, E1/15, VT/13
\def\date{2019-04-08} %YYYY-MM-DD
\def\docauthor{\ms}

\usepackage{../e-mote}
\usepackage{../../../e-sek}

\begin{document}
\showsignfoot

\heading{{\doctype} för {\mname} {\mnum}}

%\naun{}{} %närvarane under
%\nati{} %närvarande till och med
%\nafr{} %närvarande från och med
\section*{Närvarande}
\subsection*{Styrelsen}
\begin{narvarolista}
\nv{Ordförande}{Edvard Carlsson}{E16}{}
\nv{Kontaktor}{Sonja Kenari}{E15}{}
\nv{Förvaltningschef}{Henrik Ramström}{E16}{}
\nv{Cafémästare}{Jonathan Benitez}{E17}{}
\nv{Sexmästare}{Theo Nyman}{BME18}{}
\nv{Krögare}{Davida Åström}{BME17}{}
\nv{Entertainer}{Saga Åslund}{BME18}{}
%\nv{SRE-ordförande}{Lina Samnegård}{BME16}{\nafr{9}}
\nv{ENU-ordförande}{Jakob Pettersson}{E17}{}
\nv{Øverphøs}{Stephanie Bol}{BME17}{}
\end{narvarolista}


\subsection*{Ständigt adjungerande}
\begin{narvarolista}
%\nv{Sigillbevarare}{Matilda Horn}{BME18}{\nati{17}}
%\nv{}{}{}{}
%\nv{Kårrepresentant}{Jacob Karlsson}{}{\nafr{3}}
%\nv{Valberedningens ordförande}{Elin Magnusson}{}{}
%\nv{Skattmästare}{Daniel Bakic}{E15}{}
%\nv{Vice Krögare}{Klara Indebetou}{BME17}{}
%\nv{Vice Krögare}{Hjalmar Tingberg}{BME16}{}
%\nv{Kårrepresentant}{Philip Johansson}{}{}
\nv{Kårrepresentant}{Anna Qvil}{}{}
\nv{Valberedningens ordförande}{Axel Voss}{E15}{\nafr{10b}}
%\nv{Fullmäktigeledamot}{Magnus Lundh}{E15}{\nafr{12}}
%\nv{Chefredaktör}{Max Mauritsson}{BME16}{}
%\nv{Elektras Ordförande}{Elisabeth Pongratz}{}{}
%\nv{Inspektor}{Monica Almqvist}{}{}
%\nv{Valberedningens ordförande}{Axel Voss}{E15}{\nafr{11}}

\end{narvarolista}

%\begin{comment}
\subsection*{Adjungerande}
\begin{narvarolista}
%\nv{post}{namn}{klass}{nati/nafr/tom}
\nv{Projektfunktionär}{Sophia Carlsson}{BME17}{}
\nv{Projekfunktionär}{Emma Hjörneby}{BME17}{}
%\nv{}{}{}{}
\end{narvarolista}
%\end{comment}

\section*{Protokoll}
\begin{paragrafer}
\p{1}{OFMÖ}{\bes}
Ordförande {\mo} förklarade mötet öppnat kl.12.11.

\p{2}{Val av mötesordförande}{\bes}
{\valavmo}

\p{3}{Val av mötessekreterare}{\bes}
{\valavms}

\p{4}{Val av justeringsperson}{\bes}
{\valavj}

\p{5}{Godkännande av tid och sätt}{\bes}
{\tosg}

\p{6}{Adjungeringar}{\bes}
%Adam Belfrage adjungerades.{}
Sophia Carlsson adjungerades.\\
Emma Hjörneby adjungerades.
%\textit{Inga adjungeringar.}


\p{7}{Godkännande av dagordningen}{\bes}
%Theo \ypa lägga till sena handlingar till dagordningen.
%Davida \ypa lägga till punkten ``Lophtet'' till dagordningen.\\
%Edvard \ypa lägga till punkten ``Ordensband'' til dagordningen.
%Fredrik \ypa att lägga till \S18b ``Teknikfokus utnyttjande av LED-café''.
Jonathan \ypa ändra punkten §12 från att vara en beslutspunkt till diskussion. \\
Föredragningslistan godkändes med yrkandet.
%Föredragningslistan godkändes med samtliga yrkanden.


\p{8}{Föregående mötesprotokoll}{\bes}
%\latillprot{S06/19 och S07/19}
\textit{\ingaprot}

\p{9}{Fyllnadsval och entledigande av funktionärer}{\bes}
\begin{fyllnadsval} %"Inga fyllnadsval." fylls i automatiskt
%\fval{Moa Rönnlund}{Halvledare}

%\entl{Namn}{Post}
\end{fyllnadsval}

\p{10}{Rapporter}{}
\begin{paragrafer}
\subp{A}{Hur mår alla?}{\info}
Punkten protokollfördes ej.

\subp{B}{Utskottsrapporter}{\info}
Det går bra för CM, fortfarande lite problem med att fylla diodschemat. Sista veckan LED är öppet nu innan det stängs för omtentaperioden.

FVU har städat igenom Sicrit. Utöver detta så har de så klart bokfört! En andra kickoff är även planerad den 18 maj.

InfU går det bra för. Sonja har haft lite möten och kollat igenom utvärderingarna för utskottet. Vissa poster har fått lite konkretare instruktioner. Uppskattas om sektionen kontaktar övriga funktionärer som de behöver hjälp av direkt istället för att gå via Kontaktorn. Bilder från veckans event håller på att redigeras. Utöver allt som händer så förbereds det även inför Vårterminsmötet.

KM har haft Gille! Det var livemusik och mexikanska klägg. På fredag är det gille utan Davida, spännande att se hur det kommer gå till. Kläggscannern ska även funka nu igen tack vare lite hjälp från InfU.

NollU har haft uppe uppdragsansökan ett tag till. Internationella grupperna har också utökats. Jättekul att phaddergrupperna kommit igång. Utöver arbetet så har Phøset varit på FHØB-helg.

ENU fortsätter söka företag för LMI och FED pub. Jakob har varit på säljutbildning på Teknologkåren med olika näringslivsansvariga. ENU har även börjat spika mer spons inför året.

NöjU har haft tandemsläpp och fyllt platserna. Kommer potentiellt finnas plats för några reserver med om andra sektioner inte sålt slut än. Var ett intresseant ``Sporta med E'' denna veckan. Planeras en eventuell äggjakt inför påsk, lite strul med att få ihop det tidsmässigt då det väntar sektionsmöte på tisdag samt DÖMD i helgen.

E6 höll i en HTF-sittning som gick bra. En sagoliksittning gick också bra. Temasläpp till nollningen är om exakt en månad. Theo har varit på kollegiemöte förra veckan. E6 lagar även maten för en alumnisittning på onsdag.

SRE har hållit Speak up-days. Psyksociala miljön i E-huset ska presenteras i veckan efter undersökningen som har gjorts där studenter från E- och D-sektionen blivit djupintervjuade. I övrigt har arbetet rullat på som vanligt.



\subp{C}{Ekonomisk rapport}{\info}
Inte mycket nytt utan det ser bra ut. Kommer en mer utförlig rapport på sektionsmötet imorgon.

\subp{D}{Kåren informerar}{\info}
Det är Innovation week denna veckan, utöver alla event kommer det vara en minimässa i Gasque på torsdag!

\end{paragrafer}

\p{11}{Äskning av pengar för inköp av vattendunkar}{\bes}
Edvard vill att vi ska köpa in vattendunkar eftersom vi alltid behöver låna av andra sektioner när det gäller.  \\
Var dessa dunkarna ska förvaras diskuteras. Styrelsen kommer till slutsatsen att det kommer finnas plats till 3 vattendunkar i antingen Buren eller CM.

\Mbaby

\p{12}{Ulla}{\dis}
Jonathan har hört från vår Mackbarsprinsessa Ulla som vill komma tillbaka i tjänst. Det är lite oklarheter kring hur det skulle gå till eftersom det inte är något som finns i verksamhetsplan eller budget. Jonathan och Henrik ska undersöka förfrågan så att styrelsen kan diskutera konkretare förslag.

\p{13}{Renovering av toaletterna}{\bes}
Lite diskussion angående på om vi ska anpassa toaletterna för alla. Extra utrymme för CM hade uppskattats samt underlättat för KM och CM om vi kan installera en tvättmaskin. Torkningsmöjligheter diskuteras. Ett ställ för handdukar att torka kan funka. Edvard ska ta det vidare till en ny diskussion med Akademiska Hus.

\p{14}{Nästa styrelsemöte}{\bes}
\Mba nästa styrelsemöte ska äga rum 2019-04-15 kl.12.10 i E:1123.

\p{15}{Beslutsuppföljning}{\bes}
Det har gått bra på balen. Balkommittéen är nöjda. Rekommenderar framtida sektionsmedlemmar att söka projektfunktionärer. Allt är inte bokfört än.
Edvard \ypa skjuta upp beslutsuppföljningen till S13.
\Mbaby
%\textit{Inga beslut att följa upp.}

\p{16}{Övrigt}{\dis}
Glöm inte att skriva upp om ni tar fika, CM behöver ha koll på lagret och det är svårt om folk inte skriver upp vad de tar! 

Från den 1 april 2019 är det lag på att alla inköp i offentlig sektor ska faktureras med e-faktura
    enligt en ny europeisk standard. Det innebär att alla leverantörer till offentlig sektor måste skicka e-fakturor enligt 
    den nya standarden, och att alla offentliga organisationer måste kunna ta emot dem. Osäker på hur du ska fakturera? Kontakta Henrik Ramström! 

Styrelsen är anmodade till V-sektionens 55-års jubileum den 18 maj. Det är även Teknologkårens jubileumsbal samma kväll men inte klart än om styrelsen anmodas till den. 

Axel tycker man borde sätta upp krokar runt om i Edekvata för att underlätta pyntning. 

\p{17}{OFMA}{\bes}
{\mo} förklarade mötet avslutat kl. 13.02.
\end{paragrafer}

%\newpage
\hidesignfoot
\begin{signatures}{3}
\signature{\mo}{Mötesordförande}
\signature{\ms}{Mötessekreterare}
\signature{\ji}{Justerare}
\end{signatures}
\end{document}
