\documentclass[10pt]{article}
\usepackage[utf8]{inputenc}
\usepackage[swedish]{babel}

\def\mo{Edvard Carlsson}
\def\ms{Saga Åslund}
\def\ji{Theo Nyman}
%\def\jii{}

\def\doctype{Protokoll} %ex. Kallelse, Handlingar, Protkoll
\def\mname{Styrelsemöte} %ex. styrelsemöte, Vårterminsmöte
\def\mnum{S09/19} %ex S02/16, E1/15, VT/13
\def\date{2019-04-01} %YYYY-MM-DD
\def\docauthor{\ms}

\usepackage{../e-mote}
\usepackage{../../../e-sek}

\begin{document}
\showsignfoot

\heading{{\doctype} för {\mname} {\mnum}}

%\naun{}{} %närvarane under
%\nati{} %närvarande till och med
%\nafr{} %närvarande från och med
\section*{Närvarande}
\subsection*{Styrelsen}
\begin{narvarolista}
\nv{Ordförande}{Edvard Carlsson}{E16}{}
%\nv{Kontaktor}{Sonja Kenari}{E15}{}
\nv{Förvaltningschef}{Henrik Ramström}{E16}{}
\nv{Cafémästare}{Jonathan Benitez}{E17}{}
\nv{Sexmästare}{Theo Nyman}{BME18}{}
\nv{Krögare}{Davida Åström}{BME17}{}
\nv{Entertainer}{Saga Åslund}{BME18}{}
\nv{SRE-ordförande}{Lina Samnegård}{BME16}{}
\nv{ENU-ordförande}{Jakob Pettersson}{E17}{}
\nv{Øverphøs}{Stephanie Bol}{BME17}{\nati{10}}
\end{narvarolista}


\subsection*{Ständigt adjungerande}
\begin{narvarolista}
%\nv{Sigillbevarare}{Matilda Horn}{BME18}{\nati{17}}
%\nv{}{}{}{}
%\nv{Kårrepresentant}{Jacob Karlsson}{}{\nafr{3}}
%\nv{Valberedningens ordförande}{Elin Magnusson}{}{}
\nv{Skattmästare}{Daniel Bakic}{E15}{}
%\nv{Vice Krögare}{Klara Indebetou}{BME17}{}
%\nv{Vice Krögare}{Hjalmar Tingberg}{BME16}{}
\nv{Kårrepresentant}{Philip Johansson}{}{}
\nv{Kårrepresentant}{Anna Qvil}{}{}
%\nv{Fullmäktigeledamot}{Magnus Lundh}{E15}{\nafr{12}}
%\nv{Chefredaktör}{Max Mauritsson}{BME16}{}
%\nv{Elektras Ordförande}{Elisabeth Pongratz}{}{}
%\nv{Inspektor}{Monica Almqvist}{}{}
%\nv{Valberedningens ordförande}{Axel Voss}{E15}{\nafr{11}}
\end{narvarolista}

%\begin{comment}
\subsection*{Adjungerande}
\begin{narvarolista}
%\nv{post}{namn}{klass}{nati/nafr/tom}
\nv{Sigillbevarare}{Matilda Horn}{BME18}{}
%\nv{}{}{}{}
\end{narvarolista}
%\end{comment}

\section*{Protokoll}
\begin{paragrafer}
\p{1}{OFMÖ}{\bes}
Ordförande {\mo} förklarade mötet öppnat kl.12.13.

\p{2}{Val av mötesordförande}{\bes}
{\valavmo}

\p{3}{Val av mötessekreterare}{\bes}
{\valavms}

\p{4}{Val av justeringsperson}{\bes}
{\valavj}

\p{5}{Godkännande av tid och sätt}{\bes}
{\tosg}

\p{6}{Adjungeringar}{\bes}
%Adam Belfrage adjungerades.{}
Matilda Horn adjungerades.
%\textit{Inga adjungeringar.}


\p{7}{Godkännande av dagordningen}{\bes}
%Theo \ypa lägga till sena handlingar till dagordningen.
Dagordningen godkändes.
%Fredrik \ypa att lägga till \S18b ``Teknikfokus utnyttjande av LED-café''.

%Föredragningslistan godkändes med yrkandet.
%Föredragningslistan godkändes med samtliga yrkanden.

\p{8}{Föregående mötesprotokoll}{\bes}
%\latillprot{S06/19 och S07/19}
\textit{\ingaprot}

\p{9}{Fyllnadsval och entledigande av funktionärer}{\bes}
\begin{fyllnadsval} %"Inga fyllnadsval." fylls i automatiskt
\fval{Amina Gojak}{Årskurs BME-3 ansvarig}
\fval{Malin Hjärtström}{Årskurs BME-3 ansvarig}
\fval{Emma Hjörneby}{Årskurs BME-2 ansvarig}
\fval{Nelly Ostréus}{Årskurs BME-2 ansvarig}
\fval{William Marnfeldt}{Årskurs E-2 ansvarig}
\fval{Måns Lindeberg}{Årskurs E-2 ansvarig}
\fval{Edvard Carlsson}{Årskurs E-3 ansvarig}
\fval{Henrik Ramström}{Årskurs E-3 ansvarig}
%\entl{Namn}{Post}
\end{fyllnadsval}

\p{10}{Rapporter}{}
\begin{paragrafer}
\subp{A}{Hur mår alla?}{\info}
Punkten protokollfördes ej.

\subp{B}{Utskottsrapporter}{\info}
NollU har fått schema från SVL där det har uppstått problem då det inte alls liknar tidigare år. Vidare fortsätter planering av temasläpp. 

FVU har planerat möten och förbereder sig inför Vårterminsmötet. Ekipering har sålt märken och ordensband. 

ENU har haft möte med näringslivskontakter och strukturerat upp hur posten ska fungera. 

NöjU har spikat datum och har tandemsläpp i veckan. 

E6 har ordnat balbrunch och Theo kollar på hoodies till sitt utskott och andra. På onsdag är det HTF-sittning och på lördag är det en Sagolik sittning.

CM saknar dioder denna läsperiod. Har börjat titta på kaffekort till små koppar. 

Idag är det Speak Up Days där SRE har representanter. Lina har varit på möten. 

Har haft gille med utökat tillstånd vilket gav en person nöjet att komma. Målet är att kunna ha gille på två av tre i trion och en väg dit är att ordna med checklistor. KM har haft möte och synkat schema med NollU. 


\subp{C}{Ekonomisk rapport}{\info}
Vi borde se över hur bokföringen sköts för vissa utskott. Det finns även en del fakturor som inte blivit betalda ännu.  

\subp{D}{Kåren informerar}{\info}
Speak Up Days - lyfta sina åsikter till kåren. 
I söndags var det valFM där alla heltidare utom en valdes och Kårstyrelsen förutom två. Utöver detta finns det fler lediga poster på kåren som hittas på hemsidan, www.tlth.se/lediga. 
\end{paragrafer}

\p{11}{Nästa styrelsemöte}{\bes}
\Mba nästa styrelsemöte ska äga rum 2019-04-08 kl.12.10 i E:1123.

\p{12}{Beslutsuppföljning}{\bes}
%\textit{Inga beslut att följa upp.}
Matilda yrkar på att stryka ``Inköp av funktionärsmedaljer'' ur beslutsuppföljningen.

\Mbaby


\p{13}{Övrigt}{\dis}
Mötet diskuterade aprilskämt.

\p{18}{OFMA}{\bes}
{\mo} förklarade mötet avslutat kl. 12.35.
\end{paragrafer}

%\newpage
\hidesignfoot
\begin{signatures}{3}
\signature{\mo}{Mötesordförande}
\signature{\ms}{Mötessekreterare}
\signature{\ji}{Justerare}
\end{signatures}
\end{document}
