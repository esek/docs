\documentclass[10pt]{article}
\usepackage[utf8]{inputenc}
\usepackage[swedish]{babel}

\def\mo{Edvard Carlsson}
\def\ms{Johannes Larsson}
\def\ji{Jakob Pettersson}
%\def\jii{}

\def\doctype{Protokoll} %ex. Kallelse, Handlingar, Protkoll
\def\mname{styrelsemöte} %ex. styrelsemöte, Vårterminsmöte
\def\mnum{S06/19} %ex S02/16, E1/15, VT/13
\def\date{2019-02-27} %YYYY-MM-DD
\def\docauthor{\ms}

\usepackage{../e-mote}
\usepackage{../../../e-sek}

\begin{document}
\showsignfoot

\heading{{\doctype} för {\mname} {\mnum}}

%\naun{}{} %närvarane under
%\nati{} %närvarande till och med
%\nafr{} %närvarande från och med
\section*{Närvarande}
\subsection*{Styrelsen}
\begin{narvarolista}
	\nv{Ordförande}{Edvard Carlsson}{E16}{}
	\nv{Förvaltningschef}{Henrik Ramström}{E16}{}
	\nv{Cafémästare}{Jonathan Benitez}{E17}{}
	\nv{Øverphøs}{Stephanie Bol}{BME17}{}
	\nv{SRE-ordförande}{Lina Samnegård}{BME16}{}
	\nv{ENU-ordförande}{Jakob Pettersson}{E17}{}
  \nv{Sexmästare}{Theo Nyman}{BME18}{}
	\nv{Krögare}{Davida Åström}{BME17}{}
	\nv{Entertainer}{Saga Åslund}{BME18}{}
\end{narvarolista}
\subsection*{Ständigt adjungerande}
\begin{narvarolista}
	\nv{Valberedningens ordförande}{Axel Voss}{E15}{}
	\nv{Vice Kontaktor}{Johannes Larsson}{E16}{}
  \nv{Vice Entertainer}{Simon Mahdavi}{BME18}{}	
  \nv{Kårrepresentant}{Anna Qvil}{}{}	
  \nv{Kårrepresentant}{Philip Johansson}{\nati{15}}{}	
\end{narvarolista}


\subsection*{Adjungerande}

\begin{narvarolista}
	\nv{Projektfunktionär}{Sophia Carlsson}{}{} 
	\nv{Projektfunktionär}{Emma Hjörneby}{}{}
	\nv{Fritidsledare}{Vincent Palmer}{}{}
\end{narvarolista}

\section*{Protokoll}
\begin{paragrafer}
	\p{1}{OFMÖ}{\bes}
		Ordförande {\mo} förklarade mötet öppnat 12:14.

	\p{2}{Val av mötesordförande}{\bes}
	{\valavmo}

	\p{3}{Val av mötessekreterare}{\bes}
	{\valavms}

	\p{4}{Val av justeringsperson}{\bes}
	{\valavj}

	\p{5}{Godkännande av tid och sätt}{\bes}
	{\tosg}

	\p{6}{Adjungeringar}{\bes}
	%Förnamn Efternamn adjungerades
\noindent
		Vincent Palmer adjungerades.

\noindent
		Emma Hjörneby adjungerades.

\noindent
		Sophia Carlsson adjungerades.


	\p{7}{Godkännande av dagordningen}{\bes}


	Dagordningen godkändes.
	%Föredragningslistan godkändes med yrkandet.

	\p{8}{Föregående mötesprotokoll}{\bes}
	%\latillprot{}
		%skjuts till nästa vecka
	\ingaprot

	\p{9}{Fyllnadsval och entledigande av funktionärer}{\bes}
	\begin{fyllnadsval} %"Inga fyllnadsval." fylls i automatiskt
		%\fval{namn}{post}
      
	\end{fyllnadsval}

	\p{10}{Rapporter}{}
	\begin{paragrafer}
		\subp{A}{Hur mår alla?}{\info}
		Punkten protokollfördes ej.

		\subp{B}{Utskottsrapporter}{\info}
		Punkten protokollfördes ej.

		\subp{C}{Ekonomisk rapport}{\info}
		Henkrik har varit borta men vi går nog för mycket plus, vilket är bra och dåligt. 
		Sektionen bör hitta saker att spendera pengar på.
		
		\subp{D}{Kåren informerar}{\info} 
		Kåren hälsar att det finns många poster att fylla; 
		valberedning, heltidare, styrelse, tandemgeneral med flera.
		Sök sök sök!

		\subp{E}{Omvärldsrapport}{\info}
		Halva styrelsen har varit i Norge, det var kul. De träffade Chalmers och KTH:s styrelse, bra kontakter.  
		Har fått inbjudan till KTH:s vårbal den 27 april. 


	\end{paragrafer}

	\p{11}{Äskning av pengar för inköp av diverse utrustning för DrEamHackE}{\bes}
	Vincent presenterade förslaget.

	Det är oklart när lanet kommer hållas, vårterminen är ganska fullbokad. 
  
	Henrik yrkade på återremmitering till nästa vecka.

	Mötet biföll Henriks yrkande.

	%voss när planerat lan
	%veckan efter balen
	%skjuta till vårterminen? +desktop brandvägg
	%får ej ha eget wifi

	%jakob var hittat utrustningen?

	%henrik tycker det är bra
	%bör gå på sektionsmöte men vi har gott om pengar att göra av med, inget problem
	
	%saga bra men varför på sektionsmötet

	%dispfonden begränsad men nog inget problem voss+henrik
	%voss sektionsmöte bättre

	%steph dreamhacke innan mötet
	%saga blir nog till hösten

	%lika bra att skjuta till sektionsmötet om lanet inte är innan edvard

	%saga vill kolla på fler datum innan beslut

	%behövs routern jakob

	%voss egentligen inte men bra för det blir lätt

	%jonathan skjut till nästa vecka när vi har haft tid att titta ordentligt

	%henke yrkar återremmitering. vårterminsmöte om det går annars styrlersemöte

	%jakob yrkar återremmitera en vecka

	%henriks yrkande bifalls

\p{12}{Äskning av pengar för vårbalen}{\bes}
Axel presenterade förslaget. 

Projektgruppen har mailat om sponsring för att hålla priset nere, men har inte fått svar än.
% steph föreslår spons från bergkvarabuss

Theo yrkade på att öka dekorposten till 6000 kr.
% jakob ändra till dekor+spex 6000
%voss påpekar kan justera på vtmöte

Motionen bifölls med Theos tilläggsyrkande.

	\p{13}{Nästa styrelsemöte}{\bes}
Mötet beslutade att nästa styrelsemöte ska äga rum 2019-03-06 kl. 12:10 i E:1123.

	\p{14}{Beslutsuppföljning}{\bes}
%Inga uppföljningar.

	\Ibfu
	\p{15}{Övrigt}{\dis}
Saltolåsen är konstiga ibland, var vaksamma på det. Henrik ska kolla upp det.

Henrik och FVU har städat och vill att alla ska se till att hålla det fint.



	\p{16}{OFMA}{\bes}
	{\mo} förklarade mötet avslutat 12:58.
\end{paragrafer}

%\newpage
\vspace*{\fill}
\hidesignfoot
\begin{signatures}{3}
	\signature{\mo}{Mötesordförande}
	\signature{\ms}{Mötessekreterare}
	\signature{\ji}{Justerare}
\end{signatures}
\end{document}
