\documentclass[../_main/handlingar.tex]{subfiles}

\begin{document}
\utskottsrapport{Nöjesutskottet}
\vspace{8px}

Sedan vårterminsmötet har NöjU hållit i flera event, stora som små. Ett lag var iväg på DÖMD, och dessutom hade E-sektionen flest deltagande av alla sektioner på LTH i Tandem! En mycket uppskattad visning av Eurovision Song Contest-finalen anordnades i maj. Flertalet spelkvällar hölls och givetvis gavs möjligheten att Sporta med E varje söndag! 

Tyvärr blev inte alla event av, ett sådant var FunDAE tillsammans med F, D och A som ställdes in på grund av dåligt väder. 

Under nollningen styrde NöjU i UtEDischot, spelkväll, NöjUs tur och Phaddergruppsolympiaden. De tre sistnämnda planerades av utskottet helt och hållet och alla gick jättebra och verkade vara uppskattade av nollor och phaddrar. Det gav också en bra möjlighet för utskottet att visa upp sig och diversiteten av saker man kan göra inom NöjU. 

UtEDischot gick inte som förväntat, eftersom vi hade slarvat med att läsa tillstånd och under kvällen strulade tekniken vilket skapade problem för artisten. Trots detta är vi glada att eventet blev av och kommer nu att försöka skapa de bästa möjliga förutsättningarna för  projektgruppen för UtEDischot 2020. 

Tillsammans med NollU anordnade vi MegaKulfEstivalen samt Way out East, där det förstnämnda var ett intersektionellt event med M- och K-sektionerna. Det var ett alkoholfritt, festivalinspirerat utomhusevent med lekar, som vi hoppas på ska bli tradition. 

Mest nyligen hölls DreamHackE och ölresan, två uppskattade event som förhoppningsvis lockade sektionsmedlemmar som inte annars deltar. 

Alla dessa event till trots är NöjU fortfarande inte riktigt klara, eftersom Sångarstriden går av stapeln lördagen den 7e december! 

\begin{signatures}{1}
    \mvh
    \signature{Saga Åslund}{Entertainer}
\end{signatures}

\end{document}
