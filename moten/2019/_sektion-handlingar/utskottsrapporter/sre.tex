\documentclass[../_main/handlingar.tex]{subfiles}

\begin{document}
\utskottsrapport{Studierådet}
\vspace{8px}

Studierådet sysslar ständigt med studiebevakning, och detta sker kontinuerligt genom granskning av CEQ-enkäter, uppföljning av studentärenden och utvärderingsmöten för kurser. Vi har representanter från alla årskurser för BME och för alla utom 1:an för E. Detta innebär att vi redan har fått in två stycken av de nyantagna som årskursansvariga vilket är roligt! Vi har även representanter från vissa specialiseringar, vilket ger oss en bättre inblick i specialiseringskurserna.

Under nollningen har SRE synts under utskottssafarit, hållt i SRE-workshop samt fyra pluggkvällar som alla har varit väldigt välbesökta. Vi hade många pluggphaddrar som hjälpte till med plugget, vilket var väldigt uppskattat. Under pluggkvällarna har vi dessutom haft möjligheten att lotta ut 6 stycken presentkort á 250 kronor på KFS. 

Studierådet håller även tät kontakt med programledningen, och har representanter i båda programledningarna. Där tas beslut gällande de två programmen, tex utformning av vissa kurser samt väljer in nya kurser. Utskottet har även kontakt med övriga studierådsordförande på kåren genom SRX. Även likabehandlingsombud, skyddsombud och världsmästare har kontakt med Teknologkåren, för att kunna diskutera vad som händer på kårnivå samt ge varandra tips och idéer. 


\begin{signatures}{1}
    \mvh
    \signature{Lina Samnegård}{SRE-Ordförande}
\end{signatures}

\end{document}
