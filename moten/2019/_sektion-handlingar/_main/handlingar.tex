\documentclass[10pt]{article}
\usepackage[utf8]{inputenc}
\usepackage[swedish]{babel}

\def\ordf{Erik Månsson}
\def\sekr{Johan Karlberg}

\def\doctype{Handlingar} %ex. Kallelse, Handlingar, Protkoll
\def\mname{Höstterminsmötet} %ex. styrelsemöte, vårterminsmöte
\def\mnum{HT/16} %ex S02/16, E01/15, VT/13
\def\date{2016-11-22} %YYYY-MM-DD
\def\docauthor{\sekr}

\def\mtime{17:15}
\def\place{E:B}

\usepackage{../../_sektion-handlingar/e-handlingar-sek}
\usepackage{../../e-mote}
\usepackage{../../../../e-sek}

\begin{document}

\firstpage{{\doctype} till {\mname} {\mnum}}{{\date} {\mtime} i {\place}}

\tableofcontents
\newpage

\subfile{../../_sektion-handlingar/_other/guide}
\newpage

\section{Dagordning}
\subsection{Tid och plats}
\tidplats

\subsection{Föredragningslista}
\begin{paralist}
    \pli{TaFMÖ}{}
    \pli{Val av mötesordförande}{}
    \pli{Val av mötessekreterare}{}
    \pli{Godkännande av tid och sätt}{}
    \pli{Val av två justeringspersoner}{}
    \pli{Adjungeringar}{}
    \pli{Godkännande av dagordningen}{}
    \pli{Föregående sektionsmötesprotokoll}{}
    \pli{Meddelanden}{}
    \pli{Beslutsuppföljning}{}

    \pli{Övrigt}{}
    \pli{TaFMA}{}
\end{paralist}

\begin{signatures}{2}
    \emph{I Sektionens tjänst}
    \signature{\ordf}{Ordförande}
    \signature{\sekr}{Kontaktor}
\end{signatures}

\section{Beslutsuppföljning}
\begin{busek}
    \beslutsek{VT/15}{Ett beslut}{Några personer}{...}{}
\end{busek}

\begin{supersection}{Beslutsuppföljning}{}
    %\subfile{../besuppf/example}
\end{supersection}

\begin{utskottsrapporter}
    \subfile{../utskottsrapporter/example}
\end{utskottsrapporter}

\begin{valforslags}
    \subfile{../valforslag/example}
\end{valforslags}

\begin{berattelser}
    \subfile{../berattelser/example}
\end{berattelser}

\begin{stadgeandringar}
    \subfile{../stadgeandringar/example}
\end{stadgeandringar}

\begin{motioner}
    \subfile{../motioner/example}
    \subfile{../motionssvar/example}
\end{motioner}

\begin{propositioner}
    \subfile{../propositioner/example}
\end{propositioner}

\end{document}
