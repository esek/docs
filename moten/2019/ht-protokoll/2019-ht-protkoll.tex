\documentclass[10pt]{article}
\usepackage[utf8]{inputenc}
\usepackage[swedish]{babel}
\def\mo{Pontus Landgren}
\def\ms{Mattias Landgren}
\def\ji{}
\def\jii{}

\def\doctype{Protokoll} %ex. Kallelse, Handlingar, Protkoll
\def\mname{Höstterminsmöte} %ex. styrelsemöte, Vårterminsmöte
\def\mnum{HT/19} %ex S02/16, E1/15, VT/13
\def\date{2019-11-12} %YYYY-MM-DD
\def\docauthor{\ms}

\usepackage{../e-mote}
\usepackage{../../../e-sek}

\begin{document}
\showsignfoot

\heading{{\doctype} för {\mname} {\mnum}}

%\naun{}{} %närvarane under
%\nati{}{} %närvarande till och med
%\nafr{}{} %närvarande från och med
\section*{Närvarande}
\subsection*{Styrelsen}
\begin{narvarolista}
    \nv{Ordförande}{Edvard Carlsson}{E16}{}
    \nv{Kontaktor}{Mattias Lundström}{E17}{}
    \nv{Förvaltningschef}{Henrik Ramström}{E16}{}
    \nv{Cafémästare}{Jonathan Benitez}{E17}{}
    \nv{Sexmästare}{Theo Nyman}{BME18}{}
    \nv{Krögare}{Davida Åström}{BME17}{}
    \nv{Entertainer}{Saga Åslund}{BME18}{}
    \nv{SRE-ordförande}{Lina Samnegård}{BME16}{}
    \nv{ENU-ordförande}{Jakob Pettersson}{E17}{}
    \nv{Øverphøs}{Stephanie Bol}{BME17}{}
\end{narvarolista}

\subsection*{Medlemmar}
\begin{narvarolista}
%\nv{Post}{Namn}{Klass}{}
\end{narvarolista}

\subsection*{Ständigt adjungerande}
\begin{narvarolista}
%\nv{Talman}{Johan Westerlund}{E11}{}
%\nv{Post}{Namn}{Klass}{}
\end{narvarolista}

\begin{comment}
\subsection*{Adjungerande}
\begin{narvarolista}
\nv{Kårrepresentant}{Namn}{Klass}{}
\end{narvarolista}
\end{comment}

\newpage
\section*{Protokoll}

\begin{paragrafer}
\p{1}{TaFMÖ}{}
Talman {\mo} förklarade mötet öppnat 17:.

\p{2}{Val avötesordförande}{}
Talman {\mo} valdes.

\p{3}{Val av mötessekreterare}{}
Kontaktor {\ms} valdes.

\p{4}{Godkännande av tid och sätt}{}
Tid och sätt godkändes.

\p{5}{Val av två justeringspersoner}{}
Rasmus Sobel och Filip Larsson valdes till justerare.

\valavj

\p{6}{Adjungeringar}{}
Ellen Belcher
Oskar styrelsemöte
Martin Bergman
Ivar Vänglund
%\ingaadj

\p{7}{Godkännande av dagordningen}{}
%Föredragningslistan godkändes.
%Fredrik \ypa att lägga till \S18b ``Teknikfokus utnyttjande av LED-café''.
%Föredragningslistan godkändes med yrkandet.
%Föredragningslistan godkändes med samtliga yrkanden.

Pontus Landgren \ypa behandla Revisionsberättelse E-sektionen 2018 från sena handlingar under \S12. 

Pontus Landgren \ypa behandla Diskussionspunkt: Sektionsrepresentativ klädsel från sena handlingar efter \S19 och att de resterade efter det. 

Pontus Landgren \ypa behandla Stadgaändring, Styrelsens sammansättnig och Informationsutskotet från sena handlingar efter \S22 C). 

%Framvaskade versionen av listan. 
\textbf{Mötet beslutade att godkänna föredragningslistan med samtliga yrkanden.}

\p{8}{Föregående sektionsmötesprotokoll}{}

\textbf{Mötet beslutade att lägga till protokollet för Extrainsatt Sektionsmöte E01/19 till handlingarna.}

\p{9}{Meddelanden}{}
Henrik Ramström, Oskar XXXXX och Ellen XXXX informerade om Fullmäktige och att man ska rösta i Fullmäktigevalet. 

Adam Belfrage meddelade att valberedningen är färdig med sitt arbete och att valberedningens förslag kommer ut på torsdag. 

Sophia Carlsson meddelade att det snart är F1 Röj och att de söker jobbare till eftersläppet. Mer information finns på deras hemsida. 

Ivar XXXX och Martin XXX presenterade sig som sektionens kårkontakt och berättade vad de gör i Kåren. 
De informerade om vad som hänt på kårnivå den senaste tiden. 
Inom utbildning har de jobbat emot beslutet att ta bort tillägsmeriter för engagerade. Inom nollningen har de gjort en del förändringar och de meddelade att man fortfarande kan söka Nollegeneral för både hösten vinternollningen.  
Inom näringsliv meddelade Ivar att de har valt en ny projektledare för Arkad och en ny med informationsansvar.
Martin meddelade också att kårens ekonomi går bra. Till slut fick sektionen komplimanger för dess engagemang inom årets Sångarstrid. 

Morgan XXXX meddelade att Sångarstriden söker jobbare.


\p{10}{Beslutsuppföljning}{}
Adam Belfrage presenterade beslutsuppföljningen av \emph{''Renovering av biljard''}.

\textbf{\Mba bifalla att-satserna i beslutsuppföljningen}.


Davida Åström presenterade beslutsuppföljningen av \emph{''Inköp av cykelvagn'}.

\textbf{\Mba bifalla att-satserna i beslutsuppföljningen}.

William presenterade beslutsuppföljningen av \emph{''Inköp av utrustning för Elektro Banana Band''}.

Saga Åslund tyckte att det var ett bra inköp. 

\textbf{\Mba bifalla att-satserna i beslutsuppföljningen}.

Davida Åström presenterade beslutsuppföljningen av \emph{''Inköp av iZettlescanners''}.

Saga Åslund tyckte att det var ett bra inköp. 

\textbf{\Mba bifalla att-satserna i beslutsuppföljningen}.

Edvard Carlsson presenterade beslutsuppföljningen av \emph{''Inköp av kameratillbehör''}.

Saga Åslund tyckte att det var ett bra inköp. 

\textbf{\Mba bifalla att-satserna i beslutsuppföljningen}.

Emil presenterade beslutsuppföljningen av \emph{''Inköp av ljudteknik''}.

Saga Åslund tyckte att det var ett bra inköp. 

\textbf{\Mba bifalla att-satserna i beslutsuppföljningen}.

Vincent Palmer presenterade beslutsuppföljningen av \emph{''Inköp av Router och switchar för DreamHackE''}.

Vincent meddelade att inköpet gick en bra bit under budget. 

\textbf{\Mba bifalla att-satserna i beslutsuppföljningen}.

\p{11}{Bokslut från 2018}{} %% Kanske vill fastställas

\textbf{\Mba lägga Bokslut från 2018 till handlingarna.}

\p{12}{Revisionsberättelse för 2018}{}
Pontus Landgren presenterade revisionsberättelsen för 2018. 

\textbf{\Mba lägga revisionsberättelsen 2018 till handlingarna.}

\p{13}{Resultatdisposition från 2018}{} % Beslut
Sektionens Förvaltningschef Henrik Ramström presenterade förslaget till resultatdisposition.

\textbf{\Mba godkänna resultatdispositionen.}

\p{14}{Frågan om ansvarsfrihet för 2018}{}

  \begin{paragrafer}
    \subp{A}{Funktionärer}{}

    \textbf{\Mba finna funktionärerna 2018 ansvarsfria.}

    \subp{B}{Utskott}{}

    \textbf{\Mba finna utskotten 2018 ansvarsfria.}
    
    \subp{C}{Styrelse}{}
    
    \textbf{\Mba finna styrelsen 2018 ansvarsfria.}
    
    \subp{D}{Revisorer}{}

    \textbf{\Mba finna revisorerna 2018 ansvarsfria.}

   
    \subp{E}{Valberedning}{}

    Pontus Landgren \ypa Edvard Carlsson finnes som mötesordförande under \S14:E då han själv är i jäv.

    \textbf{Mötet beslutade att bifalla yrkandet.}

    \textbf{\Mba finna valberedningen 2018 ansvarsfria.}

    Pontus Landgren återgick som mötesordförande. 


  \end{paragrafer}
\p{15}{Ekonomisk rapport}{}

Sektionens förvaltningschef Henrik Ramström gav en rapport för Sektionens ekonomi.

Sektiones ekonomi ser bra ut och vi har hög likviditet. Henrik meddelade att det finns ett litet fel i handlingarna men att sektionen har mer pengar än vad som står skrivet. 

Henrik förklarade för mötet att det är bra att sektionen har mycket pengar. Vi behöver en stor budget och fonder som täcker sektionens verksamhet och uppkommande renoveringar. 

Saga Åslund tyckte att Henrik har gjort ett bra jobb. 

\textbf{\Mba lägga den ekonomiska rapporten till handlingarna.}

\p{16}{Uttag ur Sektionens fonder sedan förra terminsmötet}{}
Sektionens Förvaltningschef Henrik Ramström berättade om uttagen ur Sektionens fonder sedan förra terminsmötet.

%En stor grej var Vårbalen som gick lite över budget. Vissa budgetposter var lite små. Första gången som olycksfonden användes. Theos laddare hade försvunnit. 
%Love undrade om Theo ens hade en Macbook. 

%\textbf{\Mba lägga till handlingarna.}

\p{17}{Resultatrapport}{}
Henrik presenterade resultatrapporten från 2019 och Henrik meddelade att Sektionen går med vinst. Framför allt Sexmästeriet och ENU. Inte Nolleutskottet. 

Saga Åslund startade en applåd till Henrik som fått springa fram och tillbaka till scenen. 

\textbf{\Mba lägga resultatrapporten till handlingarna.}

\p{18}{Utskottsrapporter}{}
Styrelsen och Valberedningen berättade om deras verksamhet under året.

Mötet fick möjligheten att ställa frågor.

\textit{Inga frågor ställdes.}

\textbf{\Mba lägga utskottsrapportera till handlingarna.}

\p{19}{Uppföljning av verksamhetsplan}{}

Mötet fick möjligheten att ställa frågor.

\textit{Inga frågor ställdes.}

\textbf{\Mba lägga uppföljningen av verksamhetsplanen till handlingarna.}

\p{20}{Diskussionspunkt: Sektionsrepresentativ klädsel}{} %%%%%%%%%%%%%% DISKUSSION

\textbf{Pontus Landgren \ypa ajounera mötet i 30 minuter.}

\textbf{Mötet beslutade att bifalla yrkandet.}

\textit{Mötet ajournerades 18:40 och återupptogs 19:19}

%Ivar XXXX \ypa Bryggare Bob fylls på med kaffe. 

Henrik Ramström och Matilda Horn presenterade diskussionspunkten om Sektionsrepresentativ klädsel. 

Saga Åslund och Jakob Pettersson gick på en catwalk med Toxic - Britney Spears som bakgrundsmusik och visade de olika alternativen för band till representationsklädsel.

Mötet diskuterade pris, antal färger och tjocklek på band. Mötet diskuterade även om det måste vara samma för tjejer och killar. 


Henrik Ramström gjorde en snabb undersökning på plats. 

Majoritet av mötet tyckte det var okej att det bandet blir dyrare. 

Majoritet av mötet tyckte att detaljen på bandet bör vara Krusidull-E istället för Hacke.

Majoritet av mötet tyckte att det var okej att tjejernas band är bredare. 

Majoritet av mötet tyckte att olika alterativ för tjejer var okej. Alternativen var band eller rosett.

Mötets åsikt angående att subventionera tjejernas band, som är dyrare, var blandad. 

\begin{comment}
Adam undrade vad priset lanande på. Henrik svarade att det landade på cirka 160kr. Henrik drog ner antalet färgband för att få ner priset.
Sophia undrade om tjejernas band får plats med medaljer. Sophia kår sa att det går bra. 

Tove - detalj pris vs detalj med Hacke eller bredare bandet.
THeo - kommentar ang vit väst. Det försvinner lite för killarna iom att vi har vit väst. 

Joakim - Ett till färgförslag. 
Filip tycker att det gamla bandet är snyggare. 5 färger är snyggt och cleant. Inga åsikter till tjejerna men förstår att det inte är lika bra. 
Edvard - tycker det gamla bandet är klart snyggare och fölkjet tlth standard -- naturligt att det blir dyrare men viktigare att det är enhetligt och att vi kan få samhörighet och poäng aatt ha likadana ordensband för tjejr och killar.
Stpeh - Najs om vi har samma. Men tycker att vi ska ha svart-vit-svart. För att det ser snyggare ut. -> Tjockare band är snyggare. Vill gärna att vi ocskå har en detalj. Tycker det gamla har varit kasst och tycker det ska bli kul att byta. Värt kostnadet. 

Matilda- smalare är bättre och kul att vi kan ändra. Rosett är inte lika fint. 

Saga lyfte frågan om rosett vs band + rosett. 

Love frågade om TLThs standard för ordensband och frackband. Pontus svarade att det breda vbandet är ordensband och den smala är frackkavaj.

Matilda tyckteen ide att sektionen ska subventionera tjejbandet eftersom det är dyrare men att det är vikigt att det är enhetligt. 
Sophia tyckte att det smalare för killar 

Ester tycker om det gamla för killarna. vill själv ha ett långt band men vill inte fästa det i sin klänning då det blir märken. Tror inte det ser fullt ut att någon har rosett och om någon har band.
Davida Stor vinst är att slippa sätta nålar i kläningen.
Filip - sammantattning. Konsensus är att vi tycker det gamla bandet på killar men samma på tjejer fast bredare. 
Sophia - Har inte data en tjockare för tjejer? Matilda svarade att de ha två olika tjocklekar. 
Adam - Går att fixa glattare band med samma mönster. Tycker inte att diskussionen ger något mer nu. 

Henrik - vill göra en live undersökning under mötet så att man får underhåll.
Majoritet - OK om det är dyrare
Lika dana band för killar och tjejer. Mixad åsikt. 
Glattare band för att ha en tyngd på. Få svar, mixat. 
Sektionsrepresenativt på. 
--- HACKE 
--- Krusidull E -> Majoritet. 

Bredare för tjejer än för killar - OK Majoritet

Olika alternativ för tjejer, roset och band -> Majoritet. 
Subventionera tjejers band - mixat men majoriet för att subvetionera. 
Samma utseende -> Majoirtet 
\end{comment}

\p{21}{Behandling av motioner}{}
    \begin{paragrafer}
      \subp{A}{Avskaffandet av posten Karnevalsmalaj}{}

        Filip Larsson presenterade motionen. 

        Filip Larsson berättade om sin tid som sektionens karnevalsmalaj.  

        %Emil Eriksson - undrade när den policyn från karnevalen kom till. Vissa undantag för de som spela instrumet och en musikorganistation. 

        Edvard Carlsson presenterade styrelsens svar på motionen. 

        \textbf{\Mba bifalla motionen i sin helhet.}


      \subp{B}{Införandet av Kröke}{}

      Adam Belfrage presenterade motionen. 

      Rasmus Sobel undrade om vi kan stava Kröke med stort e.

      Jonathan Benitez undrade om vi kan stava Kröke med danskt ö.

      Edvard presenterade styrelens svar och informeraede att Tillståndsmyndigheten har bra kundservice.

      Casper XXX frågade om man som funktionär blir ansvarsbefriad om Kröke givit klartecken att kröka. 

      Mötet diskuterade motionen.  

      \textbf{\Mba avslå motionen i dess helhet. }

      \subp{C}{Reglementesändring, Införandet av posten Booster}{}

      Tove XXXX presentade motionon. 

      Sophia Carlsson undrade vad Boosters ska göra under våren. Tove svarade med att de kan planera och hålla i event. 

      Filip Larsson undrade hur Boosters ska jämföras med Hjälpphadder och ØGP. Tove svarade att Nollehjälp inte varit så lockade då det inte är så inkluderande i nollningen. Tove förklarade att Booster kan jämföras med en ØGP som inte har huvudansvar men som kan hjälpa till att verkställa saker när inte Phøset kan vara på plats. 

      Casper XXX undrade om det är möjligt att vara Phadder samtidigt som Booster. Tove svarade att det är upp till nästa års phøs.
      


      Edvard Carlsson presenterade styrelsens svar på motionen. 
      
      Love XXX undrade hur många som är tillsatta på liknande poster på andra sektioner. 

      Richard Byström \ypa ändra antelet till (4)-(8) istället för (6)-(8), samt att namnet är FörstärkarE.

      Axel Voss \ypa att ändra antalet till (E.A), med motivering att det inte behöver vara ett specifikt antal.
      
      Jakob Petterson menar att det finns en poäng att det ska vara ett specifikt antal så att arbetsbelastning blir konsekvent och antalet inte dalar iväg. 

      Emma XXX frågade Tove om phøset i år hade någon tanke på att använda kontaktphaddrar till problemen som nämdes i motionen. Tove svarade att den här lösningen ansågs bättre då man får vara en del av NollU där all information finns. Om det inte fungerar kan man alltid revidera och göra på ett annat sätt till nästa år. 

      Stephanie Bol tyckte det är bra om antalet stannar vid 6st och man får utvärdera det vidare nästa är. Inte bra att göra utskottet för stort med en helt ny post. 

      
      %FIlip - YRKADE PÅ ATT .. KOLLA SLACK.
      %Sophia - ser ingen mening att man behöver lägga det i en till mandatperiod. Finns alltid mycket att göra i efterarbete. finns inge n anledning att begränsa det. 
      %Adam - känns dumt att ha in det 
      %Filip drar bort sitt yrkande. 
      %Saga - tycker inte man ska va phadder och Booster då det kan bli heirkiskt om 6 av 12 phaddergrupper, men att det inte heller är en diskussion vi ska ha nu. 

      Tove tyckte att styrelsens motionssvar var rimligt och jämkade sig med styrelsens förslag men stod fast vid Booster som namnförslag. 

      Emil Eriksson \ypa ändra namnförslaget FörstärkarE till PhørstärkarE.

      Mötesordförande Pontus Landgren informerade att motionen har tre förslag på namn: Booster, Øsare och PhørstärkarE.

      \textbf{Emil P. Lundh begärde sluten votering.} 

      \textbf{\Mba bifalla motionen med styrelens motyrkanden med namnet PhørstärkarE.}


      \subp{D}{Inköp av elbjörn till sektionen}{}

      Vincent Palmer presenterade motionen. 

      Theo Nyman yrkade på:
      \begin{attsatser}
        \att en krok köps in för att förvara kabeln på insidan av dörren till PA, samt
        \att höja budgeten till \SI{4750,00}{kr}.
      \end{attsatser} 
      
      Vincent Palmer jämkade sig med Theos yrkanden. 

      Edvard Carlsson presenterade styrelens motionssvar. 

      Adam Belfrage tycker det låter mysigt med en Elbjörn. 

      \textbf{\Mba bifalla motionen med Theos tillägsyrkanden.}

      \textbf{Morgan XXX \ypa ajounera mötet i 5 minuter.}
      
      \textbf{Mötet beslutade att bifalla yrkandet.}
      
      \textit{Mötet ajournerades 20:42 och återupptogs 20:47.}

      \subp{E}{Inköp av utrustning för Elektro Banana Band}{}

      Willaim Sjödin presenterade motionen. 

      Henrik undrade vart utrustningen ska få plats. William svarade att det finns plats i PA om man rensar ut gammalt. 

      Edvard presenterade styrelesens svar. 

      Daniel Bakic svarade att styrelsens förslag blir begränsat för bandet. 

      Mötet diskuterade alternativ på inköp.

      Pontus Landgren sammanfattade mötets blandade åsikter.

      \textbf{Pontus Langren yrkade på följande inköpsföslag samt att budgeten sätts till \SI{9500,00}{kr}.}

      \begin{dashlist}
        \item 2 st sångmikrofoner för \SI{420,68}{kr/st} 
        \item 2 st instrument mikrofoner för \SI{345,17}{kr/st} 
        \item 2 st mikrofonstativ för \SI{241,07}{kr/st} 
        \item 1 st Drum shield \SI{3117,33}{kr} 
        \item 2 st aktiva monitorer för \SI{1564,06}{kr/st}
        \item 2 st XLR kablar (10m) för \SI{106,72}{kr/st}
      \end{dashlist}

      \textbf{\Mba bifalla det framvaskade förslaget.}


      \textbf{Emil Eriksson yrkade på att behandla på 22 D) före 22 C).} 
      
      \textbf{\Mba bifalla yrkandet.}

      \subp{F}{Ändring av antal Preferensmästare}{}

      Theo Nyman presenterade motionen. 

      Edvard Carlsson presenterade styrelens svar. 

      \textbf{\Mba bifalla motionen i dess helhet.}

      \subp{G}{Budgetjustering för E-sektionens bidrag till Sångarstriden 2019}{}

      Elsa XXXX presenterade motionen. 

      Edvard Carlsson presenterade styrelsens svar på motionen. 
      
      \textbf{\Mba bifalla motionen i dess helhet.}

      \subp{H}{Införandet av funktionärspost Banan}{}

      William Sjödin presenterade motionen tillsammans med resten av motionärerna. 

      Edvard Carlsson presenterade styrelsens svar på motionen. 

      Pontus informerade om definitionen av en E.A post. 

      \textbf{\Mba biffa motionen i dess helhet.}

      \subp{I}{Inköp av ljudutrustning, subwoofers}{}

      Emil P. Lundh presenterade motionen. 

      Edvard Carlsson presenterade styrelsens svar på motionen. 

      Emil menade att subwoofers skulle ge en bättre och balancerad ljudbild och att det finns en poäng att slippa hyra från kåren varje gång sektionen vill spela bra ljud utomhus. 

      Vincet Palmer tyckte det kändes onödigt att göra ett inköp av subwoofers eftersom vi sällan går till kåren för att hyra. Det tyder på att behover inte är tillräckligt stort. 

      Henrik Ramström höll med och tyckte att sektionen saknar behov att bättre ljud just nu. 
 
      \textbf{\Mba avslå motionen i dess helhet. }

      \subp{J}{Äskning av pengar för inköp av ny tandemcykel}{}
      Casper XXXX presenterade motionen. 

      Rasmus Sobel undrade om vi ens hade en fungerande tandemcykel just nu. Casper svarade nej och informerade att NöjU har försökt att laga cyklarna utan att lyckas. 

      Love XXX undrade hur ofta reglerna för tandemstaffeten ändras. Casper svarade att det inte ändras alltför ofta och redogjorde lite av detta årets regler.

      Rasmus Sobel \ypa alternativt göra följande inköp, ''\textit{Trailgator}''. (\href{https://www.atredo.se/paahngscyklar/trail-gator---dragstaang?gclid=EAIaIQobChMIjv7j6MPl5QIVE853Ch0QYwiuEAQYBSABEgKoHvD_BwE}{\textit{länk}})
      
      Edvard Carlsson presenterade styrelsens svar på motionen.  

      \textbf{\Mba bifalla motionen i dess helhet.}

      \subp{K}{Utredning om renovering av toaletterna}{}

      Edvard Carlsson presenterade motionen. 

      Henrik Ramström presenterade styrelsens svar på motionen. 
      
      Daniel Bakic tyckte att idéen var väldigt bra och tycker också att toaletterna behöver renoveras.  
      
      Matilda Horn undrade var kassaskåpet som står i Ulla ska stå. Theo Nyman svarade att det kommer utredas av projektfunktionärer.

      \textbf{\Mba bifalla motionen i sin helhet.}

      \textbf{Pontus Landgren \ypa ajounera mötet i 10 minuter.}

      \textbf{Mötet beslutade att bifalla yrkandet.}
      
      \textit{Mötet ajournerades 21:50 och återupptogs 22:00}

    \end{paragrafer}

\p{22}{Behandling av propositioner}{}
    \begin{paragrafer}
      \subp{A}{Budgetförslag för 2020}{}
      Henrik presenterade till budgetförslaget för 2020. 

      Henriksändrings tillägs yrkande ******  FÅ FROM HENRIK. 
      Styrelsen jämkar sig. 
      
      

      Rasmus undrade vad Modulo 10 fonden är. Henrik förklarade kortfattat att 2001 satte en styrelsen in pengar för efter 10 år skulle det årets styrelse ha en liten sittning. 

      Bakic - Hur pass svårt har CM att nå upp till sina mål?
      Tove - Kommentar angående budgethöjningen från styrelsen internt. 

      Matilda - fråga om medaljbudget.
      Bakic - frågade om lincenser. 
      Emil undrade om arkivares budget. 


      Mötet antar det framvaskade förslaget. 


      \subp{B}{Verksamhetsplansförslag för 2020}{}
      Edvard presenterade verksmpan. för sektion och styrelse och mötet fick chans att svara på frågor.

      Utskottsordföranden presenterade förändringarna i verksamhetsplanen 2020 för respektive utksott osv....

      Mötet fick chans att ställa frågor. 

      Bakic yrkar på att stryka första raden I CMs beskrivnign... Kolla Slack. 

      Stephanie yrkar delmål.. Kolla slack. 

      Saga yrkar på ändirngar i NöjU banan. .. 



      Styrelsen jämkar sig. 

      Bifall..


      \subp{C}{Stadgaändring, Styrelsens sammansättning och Informationsutskottet}{}

      Rasmus - anledning till namnbyte. Mattias svarade.
      Filip - anlendeing till att man inte har 2 eller 3.  Mattias svarade. 
      Jonte - Hur blir det med övergång=? 
      Filip - kontaktor hamnar på vice ordförande/sekreterare. -> Detta namnet är bättre tycker filip. 
      Filip - Med tanke på att InfU inte bara har hand om teknik, en vice ordförande som kan hjälpa till i tekniken. Vice Kontaktor har varit en otrolig hjälp. 
      Emil - Älskar förslaget, inte riktigt hållbart. Tycker det är bättre att inte chefredaktören inte är vice.
      
      Vincent - Yrkar på att döpa om namnet på Kontaktor till Vice ordförande. Behålla namnet på kontaktor. 
      Bakic - Två vice. 

      Henrik - platt styrelse har vi nu. Det är bra. 

      
      Vi vill inte skapa en hirarki inom styrelsen. Bara få en till som kan ta en större arbetsbelastning inom. 
      Henrik tycker vi ska ta beslut om stadgaädringen nu. Reglementesändring kan ske på vårterminsmöte. 

      SLUTEN VOTERING - 37 isch. 
      Röstning om namn InfUencer och Informationschef
      Informationschef - Vann med majoritet. 

      Röstning Namnet Kontaktor på roll
      sekreretrarollen och infuchefsrollen. 
      Kontaktor och Viceordföranden. 

      Informationsordförande heter nu Informationschef.
      
      Röstning om namnet på sekreterarrollen i styrelsen. 
      Kontaktor.
      Vice Ordföranden. 

      Konaktor Vann. 

      Redaktionell ändring från Saga. Kolla Slack.

      Bifall för den framvaskande propositionen. Nästa del tas fram på vårterminsmötet. 

      \subp{D}{Ge styrelsen rättigheter att genomföra redaktionella ändringar i styrdokumenten}{}
        Edvard presenterade propositionen .


        Edvard yrkade att man ska ta upp ändringar på sektionsmöte. Kolla slack. 

        Bifall med edvards tillägsyrkande. 

      \subp{E}{Uppdatering av policy: Inbjudningar och anmodningar}{}

        Theo presenterade propositionen. 

        Rasmus undrade varför HeHE's chefredaktören ska anmodas. Theo svarade att tanken är bland annnat att chefredaktören ska rappoterande. 

        Tove tycker det är konstigt att cophös inte är inbjuden på Nollegasque. 

        Diskussion angående ordningen av inbjudningar till NollEgasque. 
        Emil kom med en sakupplysning om varför HeHEs chefredaktör historisk varit anmodad till Nollegasque och ständigt anmodad till styrelsemöte. 

        Filip - yrkar på att stryka HeHE's chefredaktör. . 
        Elina - yrkande på formulering av jobbare. Styrelsen jämför sig med förslagen. 
         
        Tove - Yrkande om HM konunge och cophös.
        Tove förklarade ännu mer varför cophös ska vara där. Henrik replik. 
        Filip undrade varför itne ØGP är med delmål. Tove tycker det ligger ett värde är där, att de ligger där. Men hon poängetar att phöset är där som represenation.
        Tove - Anledning för att inte hela styrelsen ska vara inbjudna -> Styrelsen har en budget för representations. 
        Henrik - Cophös blir också subventionerade.

        Jakob - Tycker inte man ska få sin tack genom att gå gratis på sittning. Vi har budget till funktoinärsvård och tack för det. 
        Saga - det finns en poäng med det Jakob säger. Vi är alla funktionärer på lika vilkor. 

        Går till beslut om tillägsyrkanden. 

        Yrkande om att styrka HM. Avslag. Kungen är kvar.
        Votering om Cophös ska flyttas upp till Inbjudnignar under NollEgasque. 38 personer närvarande. 
      Cophös är kvar under anmodade. 

      BIFALL. 

      \subp{F}{Införandet av policy: Hantering av ärenden som faller inom diskrimineringslagen}{}
      
      Edvard presenterade propositionen. 

      Daniel bakic tycker det är bra att vinner

      Bifall.


      \subp{G}{Införandet av riktlinje: Använding av G Suite}{}
      Mattias presenterade propositionen. 

      Bifall.

      \subp{H}{Inköp av sektionstält}{}

      Edvard presenterade sektionen. 

      Bifall.


      \subp{I}{Reglementesändring, uppdatering av postbeskrivningen för Halvledare}{}

      Jonathan presenterade propositionen. 

      Emil undrade om vem som är ansvarig om vi tar bort ansvaret för cafeet för en dag. Jonathan svarade att cafemästaren alltid är ytterst ansvarig.
      
      Diskussion angående vad som egentligen ingår i halvledarens uppgifter. 
      
      Bakic - ska man inte forma posten som man vill att den ska se ut, inte hur den ser ut idag.

      Bifall!

      \subp{J}{Reglementesändring, beskrivning av Nolleutskottet}{}
      Steph presenterade propositionen. 

      Bakic sa att tanken bakom förra alternativet var tydliggöra, så att ett cophös alltid är en phadderansvarig. 

      Vi är inne på dag två av mötet. Toalettdagen är över. Securitas sitter utanför och väntar. 

      Bakic yrkade på att ... phadder och ekonomi .. Kolla slack. 

      Bifall.

      Den framvaskade propositionen .. Bifalles. 

  

      \subp{K}{Reglementesändring, namn på NollU-funktionärer}{}

      Edvard presenterade propositionen. 

      Filip undarde varför vi tog bort bindestrecket i cophösare. Edvard sa att det är så vi använder det. 

      Bifall.

      \subp{L}{Reglementesändring, mindre uppdateringar}{}

      Edvard presenterade propostionen. 

      Bifall.
      
    \end{paragrafer}
\p{23}{Övrigt}{}
Saga tackade att så många var här. 

Bakic gå på fet röj
Matlida - arkad imorgon. 

Tack till os. 

\p{24}{TaFMA}{}
Talman {\mo} förklarade mötet avslutat 00:14.

\end{paragrafer}

% Mötet ajournerades xx:xx och återupptogs xx:xx.

%\p{}{Styrelsens förslag till resultatdisposition}{}
%Sektionens Förvaltningschef Anders Nilsson presenterade förslaget till resultatdisposition.
%
%\textbf{\Mba godkänna resultatdispositionen}

%Kontrapropositionsvotering

%\newpage
\hidesignfoot
\begin{signatures}{4}
\signature{\mo}{Mötesordförande}
\signature{\ms}{Mötessekreterare}
\signature{\ji}{Justerare}
\signature{\jii}{Justerare}
\end{signatures}
\end{document}
