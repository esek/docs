\documentclass[10pt]{article}
\usepackage[utf8]{inputenc}
\usepackage[swedish]{babel}
\def\mo{Pontus Landgren}
\def\ms{Mattias Lundström}
\def\ji{Rasmus Sobel}
\def\jii{Filip Larsson}

\def\doctype{Protokoll} %ex. Kallelse, Handlingar, Protkoll
\def\mname{Höstterminsmöte} %ex. styrelsemöte, Vårterminsmöte
\def\mnum{HT/19} %ex S02/16, E1/15, VT/13
\def\date{2019-11-12} %YYYY-MM-DD
\def\docauthor{\ms}

\usepackage{../e-mote}
\usepackage{../../../e-sek}

\begin{document}
\showsignfoot

\heading{{\doctype} för {\mname} {\mnum}}

%\naun{}{} %närvarane under
%\nati{}{} %närvarande till och med
%\nafr{}{} %närvarande från och med
%\frun{}{} %frånvarande under
\section*{Närvarande}
\subsection*{Styrelsen}
\begin{narvarolista}
    \nv{Ordförande}{Edvard Carlsson}{E16}{\frun{21D}{21E}}{}
    \nv{Kontaktor}{Mattias Lundström}{E17}{}{}
    \nv{Förvaltningschef}{Henrik Ramström}{E16}{}{}
    \nv{Cafémästare}{Jonathan Benitez}{E17}{}{}
    \nv{Sexmästare}{Theo Nyman}{BME18}{}{}
    \nv{Krögare}{Davida Åström}{BME17}{}{}
    \nv{Entertainer}{Saga Åslund}{BME18}{\frun{21A}{21B}}{}
    \nv{SRE-ordförande}{Lina Samnegård}{BME16}{}{}
    \nv{ENU-ordförande}{Jakob Pettersson}{E17}{\frun{21A}{21B}}{}
    \nv{Øverphøs}{Stephanie Bol}{BME17}{}{}
\end{narvarolista}

\subsection*{Medlemmar}
\begin{narvarolista}
%\nv{Post}{Namn}{Klass}{}
\nv{Antom Palmen}{E18}{\nati{21E}}{}
\nv{Richard Byström}{E18}{\nati{21E}}{}
\nv{Filip Larsson}{E17}{\frun{21A}{21B}, samt \S21E - \S21F}{}
\nv{Rasmus Sobel}{BME16}{}{}
\nv{Hanna Bengtsson}{BME18}{\nati{21J}}{}
\nv{Linnea Söderström}{BME18}{\nati{21K}}{}
\nv{Malin Svärdling}{BME18}{\nati{21K}}{}
\nv{Tove Börjeson}{E17}{}{}
\nv{Sophia Carlsson}{BME17}{\nati{21K}{}}{}
\nv{Emma Hjörneby}{BME17}{\nati{21K}}{}
\nv{Sophie G. Grahm}{BME14}{\nati{21J}{}}{}
\nv{Emil P. Lundh}{E17}{\frun{22E}{22F}}{}
\nv{Adam Belfrage}{BME17}{\nati{21E}}{}
\nv{León Martinez Saluena}{BME19}{}{}
\nv{Fabian Sondh}{E17}{\nati{20}}{}
\nv{Tom Andersson}{E17}{}{}
\nv{Viktor Andersson}{BME17}{}{}
\nv{Malin Heyden}{E16}{\nati{22C}}{}
\nv{Hugo Wikholm}{E19}{\nati{20}}{}
\nv{Johannes Larsson}{E16}{}{}
\nv{Jacob Rinderud}{E19}{}{}
\nv{Maria Pacheco}{E19}{\naun{18}{20}}{}
\nv{Ida Gusttafsson}{E18}{}{}
\nv{Tove Nimvik}{E19}{\naun{18}{20}}{}
\nv{Tor Hammarbäck}{E17}{\frun{21C}{21D}. \nati{21J}}{}
\nv{Andreas Bennström}{BME16}{\nati{20}}{}
\nv{Elsa Lindhé}{BME17}{\nati{21J}}{}
\nv{Jonna Fahrman}{BME17}{\nati{21J}}{}
\nv{Matilda Horn}{BME18}{}{}
\nv{Hilda Eliasson}{BME19}{\nati{18}}{}
\nv{Tilda Berglind}{BME19}{}{}
\nv{Ester Pörtfors}{BME18}{\nati{21K}}{}
\nv{Jakob Wisth}{E19}{}{}
\nv{Adam Lüning}{BME19}{\nati{21J}}{}
\nv{Morris Thånell}{BME19}{\nati{21J}}{}
\nv{Klara Almgren}{BME19}{\nati{20}}{}
\nv{David Karlsson}{E18}{\nati{21K}}{}
\nv{Oskar Branzell}{E18}{\nati{21D}}{}
\nv{Love Sjelvgren}{E18}{}{}
\nv{William Sjödin}{E18}{\nati{21J}}{}'

\end{narvarolista}

\newpage

\begin{narvarolista}
\nv{Simon Mahdavi}{BME18}{\nati{22E}}{}
\nv{Lukas Elmlund}{E18}{21J}{}
\nv{Axel Voss}{E15}{\naun{20}{22E}}{}
\nv{Andrea Cicovic}{E19}{\nati{22D}{}}{}
\nv{Philip Johansson}{E16}{\nati{22D}}{}
\nv{Pontus Landgren}{E14}{}{}
\nv{Morgan Bryer}{E18}{}{}
\nv{Vincent Palmer}{E18}{}{}
\nv{Alfred Langerbeck}{E18}{}{}
\nv{Frida Pilcher}{E18}{\nati{22I}}{}
\nv{Marcus Lindell}{BME18}{\frun{22G}{22H}}{}
\nv{Lukas Ekberg}{E19}{}{}
\nv{Johan Haltqvist}{E19}{}{}
\nv{Jonathan Do}{E19}{\nati{16}}{}
\nv{Amir Ghanaatifard}{E19}{}{}
\nv{Jacob Forsell}{E19}{}{}
\nv{Casper Schwerin}{BME18}{\nati{21K}}{}
\nv{Biborka Bihari}{BME18}{\nati{21K}}{}
\nv{Erik Wickström}{BME19}{}{}
\nv{Nora Öhlin}{BME19}{\nati{18}}{}
\nv{Jimmy Szentes}{E19}{}{}
\nv{Frida Holmvik}{BME19}{\nati{20}}{}
\nv{Cecilia Henningsson}{BME19}{\nati{18}}{}
\nv{Emil Bergman}{E19}{\nati{16}}{}
\nv{William Eriksson}{E19}{\nati{18}}{}
\nv{Oskar Siwerson}{E19}{\nati{20}}{}
\nv{Jack Johansson}{BME19}{\nati{21E}}{}
\nv{Petter Melander}{BME19}{\nati{21E}}{}
\nv{Minna Molin}{BME19}{\nati{20}}{}
\nv{Ebba Fritzell}{BME19}{\nati{20}}{}
\nv{Annie Mentzer}{BME19}{\nati{20}}{}
\nv{Albin Lidbäck}{E18}{\nati{22J}}{}
\nv{Erica Elgcrona}{E18}{\nati{20}}{}
\nv{Emil Eriksson}{E18}{}{}
\nv{Amanda Gustafsson}{BME17}{\nati{21J}}{}
\nv{Malin Rudin}{BME18}{\nati{21D}}{}
\nv{Adam Ekblom}{BME18}{}{}
\nv{Oskar Magnusson}{E18}{}{}
\nv{Valter Möller}{E18}{\nati{21J}}{}
\nv{Joakim Magnusson Fredlund}{BME19}{}{}
\nv{Elina Yrlid}{E18}{\frun{20}{21D}. \nati{22F}}{}
\nv{Daniel Bakic}{E15}{\nafr{9}{}}{}
\nv{Henrik Von Friesendorff}{E15}{\nati{18}{}}{}



\end{narvarolista}

\subsection*{Ständigt adjungerande}
\begin{narvarolista}
\nv{Styrelseledamot Kårstyrelsen}{Martin Bergman}{D14}{\nati{10}}
%\nv{Post}{Namn}{Klass}{}
\end{narvarolista}

\subsection*{Adjungerande}
\begin{narvarolista}
\nv{Kårrepresentant}{Ivar Vänglund}{F14}{\nati{22C}}
\nv{Fullmäktigeledamot}{Ellen Belcher}{W18}{}
\nv{Fullmäktigeledamot}{Oskar Ström}{D15}{}

\end{narvarolista}

\newpage
\section*{Protokoll}

\begin{paragrafer}
\p{1}{TaFMÖ}{}
{\mo} förklarade mötet öppnat 17:25.

\p{2}{Val av mötesordförande}{}
{\mo} valdes.

\p{3}{Val av mötessekreterare}{}
Kontaktor {\ms} valdes.

\p{4}{Godkännande av tid och sätt}{}
Tid och sätt godkändes.

\p{5}{Val av två justeringspersoner}{}

\valavj

\p{6}{Adjungeringar}{}
Ellen Belcher adjungerades.

Oskar Ström adjugerades.

Ivar Vänglund adjungerades. 
%\ingaadj

\p{7}{Godkännande av dagordningen}{}
%Föredragningslistan godkändes.
%Fredrik \ypa att lägga till \S18b ``Teknikfokus utnyttjande av LED-café''.
%Föredragningslistan godkändes med yrkandet.
%Föredragningslistan godkändes med samtliga yrkanden.

Pontus Landgren \ypa behandla Revisionsberättelse E-sektionen 2018 från sena handlingar under \S12. 

Pontus Landgren \ypa behandla Diskussionspunkt: Sektionsrepresentativ klädsel från sena handlingar efter \S19 och de resterade efter det. 

Pontus Landgren \ypa behandla Stadgaändring, Styrelsens sammansättnig och Informationsutskotet från sena handlingar efter \S22 C). 

%Framvaskade versionen av listan. 
\textbf{Mötet beslutade att godkänna föredragningslistan med samtliga yrkanden.}

\p{8}{Föregående sektionsmötesprotokoll}{}

\textbf{Mötet beslutade att lägga till protokollet för Extrainsatt Sektionsmöte E01/19 till handlingarna.}

\p{9}{Meddelanden}{}
Henrik Ramström, Oskar Ström och Ellen Belcher informerade om Fullmäktige och att man ska rösta i Fullmäktigevalet. 

Adam Belfrage meddelade att valberedningen är färdig med sitt arbete och att valberedningens förslag kommer ut på torsdag. 

Sophia Carlsson meddelade att det snart är F1 Röj och att de söker jobbare till eftersläppet. Mer information finns på deras hemsida. 

Ivar Vänglund och Martin Bergman presenterade sig som sektionens kårkontakt och berättade vad de gör. 

Ivar Vänglund och Martin Bergman informerade om vad som hänt på kårnivå den senaste tiden. 
%Inom utbildning har de jobbat emot beslutet att ta bort tillägsmeriter för engagerade. Inom nollningen har de gjort en del förändringar och de meddelade att man fortfarande kan söka Nollegeneral för både hösten vinternollningen.  
%Inom näringsliv meddelade Ivar att de har valt en ny projektledare för Arkad och en ny med informationsansvar.
%Martin meddelade också att kårens ekonomi går bra. Till slut fick sektionen komplimanger för dess engagemang inom årets Sångarstrid. 

Morgan Bryer meddelade att Sångarstriden söker jobbare.


\p{10}{Beslutsuppföljning}{}
Adam Belfrage presenterade beslutsuppföljningen av \emph{''Renovering av biljard''}.

\textbf{\Mba bifalla att-satserna i beslutsuppföljningen}.


Davida Åström presenterade beslutsuppföljningen av \emph{''Inköp av cykelvagn'}.

\textbf{\Mba bifalla att-satserna i beslutsuppföljningen}.

William presenterade beslutsuppföljningen av \emph{''Inköp av utrustning för Elektro Banana Band''}.

Saga Åslund tyckte att det var ett bra inköp. 

\textbf{\Mba bifalla att-satserna i beslutsuppföljningen}.

Davida Åström presenterade beslutsuppföljningen av \emph{''Inköp av iZettlescanners''}.

Saga Åslund tyckte att det var ett bra inköp. 

\textbf{\Mba bifalla att-satserna i beslutsuppföljningen}.

Edvard Carlsson presenterade beslutsuppföljningen av \emph{''Inköp av kameratillbehör''}.

Saga Åslund tyckte att det var ett bra inköp. 

\textbf{\Mba bifalla att-satserna i beslutsuppföljningen}.

Emil presenterade beslutsuppföljningen av \emph{''Inköp av ljudteknik''}.

Saga Åslund tyckte att det var ett bra inköp. 

\textbf{\Mba bifalla att-satserna i beslutsuppföljningen}.

Vincent Palmer presenterade beslutsuppföljningen av \emph{''Inköp av Router och switchar för DreamHackE''}.

%Vincent meddelade att inköpet gick en bra bit under budget. 

\textbf{\Mba bifalla att-satserna i beslutsuppföljningen}.

\p{11}{Bokslut från 2018}{} %% Kanske vill fastställas

\textbf{\Mba lägga Bokslut från 2018 till handlingarna.}

\p{12}{Revisionsberättelse för 2018}{}
Pontus Landgren presenterade revisionsberättelsen för 2018. 

\textbf{\Mba lägga revisionsberättelsen 2018 till handlingarna.}

\p{13}{Resultatdisposition från 2018}{} % Beslut
Sektionens Förvaltningschef Henrik Ramström presenterade förslaget till resultatdispositionen.

\textbf{\Mba godkänna resultatdispositionen.}

\p{14}{Frågan om ansvarsfrihet för 2018}{}

  \begin{paragrafer}
    \subp{A}{Funktionärer}{}

    \textbf{\Mba finna funktionärerna 2018 ansvarsfria.}

    \subp{B}{Utskott}{}

    \textbf{\Mba finna utskotten 2018 ansvarsfria.}
    
    \subp{C}{Styrelse}{}
    
    \textbf{\Mba finna styrelsen 2018 ansvarsfria.}
    
    \subp{D}{Revisorer}{}

    \textbf{\Mba finna revisorerna 2018 ansvarsfria.}

   
    \subp{E}{Valberedning}{}

    Pontus Landgren \ypa Edvard Carlsson finnes som mötesordförande under \S14:E då han själv är jävig.

    \textbf{Mötet beslutade att bifalla yrkandet.}

    \textbf{\Mba finna valberedningen 2018 ansvarsfria.}

    \textbf{Pontus Landgren återgick som mötesordförande.} 


  \end{paragrafer}
\p{15}{Ekonomisk rapport}{}

Sektionens förvaltningschef Henrik Ramström gav en rapport för Sektionens ekonomi.

Sektiones ekonomi ser bra ut och vi har hög likviditet. Henrik meddelade att det finns ett litet fel i handlingarna men att sektionen har mer pengar än vad som står skrivet. 

Henrik förklarade för mötet att sektionen behöver en stor budget och fonder som täcker sektionens verksamhet och uppkommande renoveringar. 

Saga Åslund tyckte att Henrik Ramström har gjort ett bra jobb. 

\textbf{\Mba lägga den ekonomiska rapporten till handlingarna.}

\p{16}{Uttag ur Sektionens fonder sedan förra terminsmötet}{}
Sektionens Förvaltningschef Henrik Ramström berättade om uttagen ur Sektionens fonder sedan förra terminsmötet.

%En stor grej var Vårbalen som gick lite över budget. Vissa budgetposter var lite små. Första gången som olycksfonden användes. Theos laddare hade försvunnit. 
%Love undrade om Theo ens hade en Macbook. 

%\textbf{\Mba lägga till handlingarna.}

\p{17}{Resultatrapport}{}
Sektionens förvaltningschef Henrik Ramström presenterade resultatrapporten från 2019 och han meddelade att sektionen går med vinst. Framförallt Sexmästeriet och ENU. 

Saga Åslund startade en applåd till Henrik Ramström som fått springa fram och tillbaka till podiet. 

\textbf{\Mba lägga resultatrapporten till handlingarna.}

\p{18}{Utskottsrapporter}{}
Styrelsen och Valberedningen berättade om deras verksamhet under året.

Mötet fick möjligheten att ställa frågor.

\textit{Inga frågor ställdes.}

\textbf{\Mba lägga utskottsrapportera till handlingarna.}

\p{19}{Uppföljning av verksamhetsplan}{}

Mötet fick möjligheten att ställa frågor.

\textit{Inga frågor ställdes.}

\textbf{\Mba lägga uppföljningen av verksamhetsplanen till handlingarna.}

\p{20}{Diskussionspunkt: Sektionsrepresentativ klädsel}{} %%%%%%%%%%%%%% DISKUSSION

\textbf{Pontus Landgren \ypa ajounera mötet i 30 minuter.}

\textbf{Mötet beslutade att bifalla yrkandet.}

\textit{Mötet ajournerades 18:40 och återupptogs 19:19}

%Ivar \ypa Bryggare Bob fylls på med kaffe. 

Henrik Ramström och Matilda Horn presenterade diskussionspunkten om Sektionsrepresentativ klädsel. 

Saga Åslund och Jakob Pettersson gick på en catwalk med Toxic - Britney Spears som bakgrundsmusik och visade de olika alternativen för band till representationsklädsel.

Mötet diskuterade pris, antal färger och tjocklek på band. Mötet diskuterade även om det måste vara samma för tjejer och killar. 


Henrik Ramström gjorde en snabb undersökning på plats. 

Majoritet av mötet tyckte det var okej att det bandet blir dyrare. 

Majoritet av mötet tyckte att detaljen på bandet bör vara Krusidull-E istället för Hacke.

Majoritet av mötet tyckte att det var okej att tjejernas band är bredare. 

Majoritet av mötet tyckte att olika alterativ för tjejer var okej. Alternativen var band eller rosett.

Mötets åsikt angående att subventionera tjejernas dyrare band var blandad. 

\begin{comment}
Adam undrade vad priset lanande på. Henrik svarade att det landade på cirka 160kr. Henrik drog ner antalet färgband för att få ner priset.
Sophia undrade om tjejernas band får plats med medaljer. Sophia kår sa att det går bra. 

Tove - detalj pris vs detalj med Hacke eller bredare bandet.
THeo - kommentar ang vit väst. Det försvinner lite för killarna iom att vi har vit väst. 

Joakim - Ett till färgförslag. 
Filip tycker att det gamla bandet är snyggare. 5 färger är snyggt och cleant. Inga åsikter till tjejerna men förstår att det inte är lika bra. 
Edvard - tycker det gamla bandet är klart snyggare och fölkjet tlth standard -- naturligt att det blir dyrare men viktigare att det är enhetligt och att vi kan få samhörighet och poäng aatt ha likadana ordensband för tjejr och killar.
Stpeh - Najs om vi har samma. Men tycker att vi ska ha svart-vit-svart. För att det ser snyggare ut. -> Tjockare band är snyggare. Vill gärna att vi ocskå har en detalj. Tycker det gamla har varit kasst och tycker det ska bli kul att byta. Värt kostnadet. 

Matilda- smalare är bättre och kul att vi kan ändra. Rosett är inte lika fint. 

Saga lyfte frågan om rosett vs band + rosett. 

Love frågade om TLThs standard för ordensband och frackband. Pontus svarade att det breda vbandet är ordensband och den smala är frackkavaj.

Matilda tyckteen ide att sektionen ska subventionera tjejbandet eftersom det är dyrare men att det är vikigt att det är enhetligt. 
Sophia tyckte att det smalare för killar 

Ester tycker om det gamla för killarna. vill själv ha ett långt band men vill inte fästa det i sin klänning då det blir märken. Tror inte det ser fullt ut att någon har rosett och om någon har band.
Davida Stor vinst är att slippa sätta nålar i kläningen.
Filip - sammantattning. Konsensus är att vi tycker det gamla bandet på killar men samma på tjejer fast bredare. 
Sophia - Har inte data en tjockare för tjejer? Matilda svarade att de ha två olika tjocklekar. 
Adam - Går att fixa glattare band med samma mönster. Tycker inte att diskussionen ger något mer nu. 

Henrik - vill göra en live undersökning under mötet så att man får underhåll.
Majoritet - OK om det är dyrare
Lika dana band för killar och tjejer. Mixad åsikt. 
Glattare band för att ha en tyngd på. Få svar, mixat. 
Sektionsrepresenativt på. 
--- HACKE 
--- Krusidull E -> Majoritet. 

Bredare för tjejer än för killar - OK Majoritet

Olika alternativ för tjejer, roset och band -> Majoritet. 
Subventionera tjejers band - mixat men majoriet för att subvetionera. 
Samma utseende -> Majoirtet 
\end{comment}

\p{21}{Behandling av motioner}{}
    \begin{paragrafer}
      \subp{A}{Avskaffandet av posten Karnevalsmalaj}{}

        Filip Larsson presenterade motionen. 

        Filip Larsson berättade om sin tid som sektionens karnevalsmalaj.  

        %Emil Eriksson - undrade när den policyn från karnevalen kom till. Vissa undantag för de som spela instrumet och en musikorganistation. 

        Edvard Carlsson presenterade styrelsens svar på motionen. 

        \textbf{\Mba bifalla motionen i sin helhet.}


      \subp{B}{Införandet av Kröke}{}

      Adam Belfrage presenterade motionen. 

      Rasmus Sobel undrade om vi kan stava Kröke med stort e.

      Jonathan Benitez undrade om vi kan stava Kröke med danskt ö.

      Edvard presenterade styrelens svar och informerade att Tillståndsmyndigheten har bra kundservice.

      Casper Schwerin frågade om man som funktionär blir ansvarsbefriad om Kröke givit klartecken att kröka. 

      Mötet diskuterade motionen.  

      \textbf{\Mba avslå motionen i dess helhet. }

      \subp{C}{Reglementesändring, Införandet av posten Booster}{}

      Tove Börjeson presentade motionon. 

      Sophia Carlsson undrade vad Boosters ska göra under våren. Tove svarade med att de kan planera och hålla i event. 

      Filip Larsson undrade hur Boosters ska jämföras med Hjälpphadder och ØGP. Tove svarade att Nollehjälp inte varit så lockade då de inte är så inkluderade i nollningen. Tove förklarade att Booster kan jämföras med en ØGP som inte har huvudansvar men som kan hjälpa till att verkställa saker när inte Phøset kan vara på plats. 

      Casper Schwerin undrade om det är möjligt att vara Phadder samtidigt som Booster. Tove svarade att det är upp till nästa års phøs.
      


      Edvard Carlsson presenterade styrelsens svar på motionen. 
      
      Love Sjelvgren undrade hur många som är tillsatta på liknande poster på andra sektioner. 

      Richard Byström \ypa ändra antelet till (4)-(8) istället för (6)-(8), samt att namnet är FörstärkarE.

      Axel Voss \ypa att ändra antalet till (E.A), med motivering att det inte behöver vara ett specifikt antal.
      
      Jakob Petterson menar att det finns en poäng att det ska vara ett specifikt antal så att arbetsbelastning blir konsekvent och antalet inte dalar iväg. 

      Emma Hjörneby frågade Tove om phøset i år hade någon tanke på att använda kontaktphaddrar till problemen som nämdes i motionen. Tove svarade att den här lösningen ansågs bättre då man får vara en del av NollU där all information finns. Om det inte fungerar kan man alltid revidera och göra på ett annat sätt till nästa år. 

      Stephanie Bol tyckte det är bra om antalet stannar vid 6st och man får utvärdera det vidare nästa är. Inte bra att göra utskottet för stort med en helt ny post. 

      
      %FIlip - YRKADE PÅ ATT .. KOLLA SLACK.
      %Sophia - ser ingen mening att man behöver lägga det i en till mandatperiod. Finns alltid mycket att göra i efterarbete. finns inge n anledning att begränsa det. 
      %Adam - känns dumt att ha in det 
      %Filip drar bort sitt yrkande. 
      %Saga - tycker inte man ska va phadder och Booster då det kan bli heirkiskt om 6 av 12 phaddergrupper, men att det inte heller är en diskussion vi ska ha nu. 

      Tove tyckte att styrelsens motionssvar var rimligt och jämkade sig med styrelsens förslag men stod fast vid Booster som namnförslag. 

      Emil Eriksson \ypa ändra namnförslaget FörstärkarE till PhørstärkarE.

      Mötesordförande Pontus Landgren informerade att motionen har tre förslag på namn: Booster, Øsare och PhørstärkarE.

      \textbf{Emil P. Lundh begärde sluten votering.} 

      \textbf{\Mba bifalla motionen med styrelens motyrkanden med namnet PhørstärkarE.}


      \subp{D}{Inköp av elbjörn till sektionen}{}

      Vincent Palmer presenterade motionen. 

      Theo Nyman \textbf{yrkade på}:
      \begin{attsatser}
        \att en krok köps in för att förvara kabeln på insidan av dörren till PA, samt
        \att höja budgeten till \SI{4750,00}{kr}.
      \end{attsatser} 
      
      Vincent Palmer \js med Theos yrkanden. 

      Edvard Carlsson presenterade styrelens motionssvar. 

      Adam Belfrage tycker det låter mysigt med en Elbjörn. 

      \textbf{\Mba bifalla motionen med Theos tillägsyrkanden.}

      \textbf{Morgan Bryer \ypa ajounera mötet i 5 minuter.}
      
      \textbf{Mötet beslutade att bifalla yrkandet.}
      
      \textit{Mötet ajournerades 20:42 och återupptogs 20:47.}

      \subp{E}{Inköp av utrustning för Elektro Banana Band}{}

      Willaim Sjödin presenterade motionen. 

      Henrik undrade vart utrustningen ska få plats. William svarade att det finns plats i PA om man rensar ut gammalt. 

      Edvard presenterade styrelesens svar. 

      Daniel Bakic svarade att styrelsens förslag blir begränsat för bandet. 

      Mötet diskuterade alternativ på inköp.

      Pontus Landgren sammanfattade mötets blandade åsikter.

      \textbf{Pontus Langren yrkade på följande inköpsförslag samt att budgeten sätts till \SI{9500,00}{kr}.}

      \begin{dashlist}
        \item 2 st sångmikrofoner för \SI{420,68}{kr/st} 
        \item 2 st instrument mikrofoner för \SI{345,17}{kr/st} 
        \item 2 st mikrofonstativ för \SI{241,07}{kr/st} 
        \item 1 st Drum shield \SI{3117,33}{kr} 
        \item 2 st aktiva monitorer för \SI{1564,06}{kr/st}
        \item 2 st XLR kablar (10m) för \SI{106,72}{kr/st}
      \end{dashlist}
      \textbf{\Mba bifalla Pontus Landgrens yrkande.}

      \textbf{\Mba bifalla det framvaskade förslaget.}


      \textbf{Emil Eriksson yrkade på att behandla på 22 D) före 22 C).} 
      
      \textbf{\Mba bifalla yrkandet.}

      \subp{F}{Ändring av antal Preferensmästare}{}

      Theo Nyman presenterade motionen. 

      Edvard Carlsson presenterade styrelens svar. 

      \textbf{\Mba bifalla motionen i dess helhet.}

      \subp{G}{Budgetjustering för E-sektionens bidrag till Sångarstriden 2019}{}

      Elsa Lindhé presenterade motionen. 

      Edvard Carlsson presenterade styrelsens svar på motionen. 
      
      \textbf{\Mba bifalla motionen i dess helhet.}

      \subp{H}{Införandet av funktionärspost Banan}{}

      William Sjödin presenterade motionen tillsammans med resten av motionärerna. 

      Edvard Carlsson presenterade styrelsens svar på motionen. 

      Pontus informerade om definitionen av en E.A post. 

      \textbf{\Mba bifalla motionen i dess helhet.}

      \subp{I}{Inköp av ljudutrustning, subwoofers}{}

      Emil P. Lundh presenterade motionen. 

      Edvard Carlsson presenterade styrelsens svar på motionen. 

      Emil menade att subwoofers skulle ge en bättre och balancerad ljudbild och att det finns en poäng att slippa hyra från kåren varje gång sektionen vill spela bra ljud utomhus. 

      Vincet Palmer tyckte det kändes onödigt att göra ett inköp av subwoofers eftersom vi sällan går till kåren för att hyra. Det tyder på att behover inte är tillräckligt stort. 

      Henrik Ramström höll med och tyckte att sektionen saknar behov att bättre ljud just nu. 
 
      \textbf{\Mba avslå motionen i dess helhet. }

      \subp{J}{Äskning av pengar för inköp av ny tandemcykel}{}
      Casper Schwerin presenterade motionen. 

      Rasmus Sobel undrade om vi ens hade en fungerande tandemcykel just nu. Casper svarade nej och informerade att NöjU har försökt att laga cyklarna utan att lyckas. 

      Love Sjelvgren undrade hur ofta reglerna för tandemstaffeten ändras. Casper svarade att det inte ändras alltför ofta och redogjorde lite av detta årets regler.

      Rasmus Sobel \textbf{yrkade på} 
      \begin{attsatser}
        \att istället göra följande inköp, ``\textit{Trailgator}''. (\href{https://www.atredo.se/paahngscyklar/trail-gator---dragstaang?gclid=EAIaIQobChMIjv7j6MPl5QIVE853Ch0QYwiuEAQYBSABEgKoHvD_BwE}{\textit{länk}})
      \end{attsatser}
      Edvard Carlsson presenterade styrelsens svar på motionen.  

      \textbf{\Mba avslå yrkandet.}

      \textbf{\Mba bifalla motionen i dess helhet.}

      \subp{K}{Utredning om renovering av toaletterna}{}

      Edvard Carlsson presenterade motionen. 

      Henrik Ramström presenterade styrelsens svar på motionen. 
      
      Daniel Bakic tyckte att idéen var väldigt bra och tycker också att toaletterna behöver renoveras.  
      
      Matilda Horn undrade var kassaskåpet som står i Ulla ska stå. Theo Nyman svarade att det kommer utredas av projektfunktionärer.

      \textbf{\Mba bifalla motionen i sin helhet.}

      \textbf{Pontus Landgren \ypa ajounera mötet i 10 minuter.}

      \textbf{Mötet beslutade att bifalla yrkandet.}
      
      \textit{Mötet ajournerades 21:50 och återupptogs 22:00}

    \end{paragrafer}

\p{22}{Behandling av propositioner}{}
    \begin{paragrafer}
      \subp{A}{Budgetförslag för 2020}{}
      Henrik Ramström presenterade propositionen. 
      
      Henrik Ramström \ypa göra följande budgetjusteringar:
      \begin{dashlist}
        \item Öka budgeten för Sångarstriden till en kostnad på \SI{10000}{kr}.
        \item Ändra till minustecken under Summa 2020 för SEK01.
        \item Budgetriktningen för SEK01, Funktionärsvård ersätts med ``All funktionärsvård skall belasta denna budgetpost exkluderat de utkott vars budgetbeskrivning beskriver detta. Alltså inkluderas kostnader som mötesfika, subventioneringen av mat och kaffe för funktionärer, samt kostnaden för utskottsaktiviteter med syfte att stärka utskottsgemenskapen. En del av denna budgetpost får med fördel användas för att bekosta arrangemang i syfte att avtacka funktionärerna''.
      \end{dashlist}

      Styrelsen \js med Henrik Ramströms tilläggsyrkande.      

      Rasmus Sobel undrade vad Modulo10fonden är. Henrik Ramström förklarade kortfattat att den skapades 2001 för att sittande styrelse skall bjuda in styrelsen 10 år tillbaka för att prata om sektionens utveckling. 

      %Daniel Bakic undrade hur pass svårt CM har att nå upp till sina ekonomiska mål. 
      %Bakic - Hur pass svårt har CM att nå upp till sina mål?

      Tove frågade om budgethöjning av STY01, Styrelsen internt. Henrik Ramström svarade att nuvarande budget inte täckte behovet om man ska inkludera KPL och arbetsmat till styrelsen inför sektionsmöten. 

      % Matilda - fråga om medaljbudget.
      Daniel Bakic frågade om vilka lincenser som innefattas i INFU01 - Lincenser. Mattias Lundström svarade att det främst gäller licenser till våra hemsidor och mjukvarulicenser till funktionärsposter som Picasso och Fotograf. 
      
      Filip Larsson undrade om ifall ett minustecken fallit bort under 'Ljud och Ljus'. Henrik Ramström svarade att tidigare har sektionen haft intäkter från uthyrning, men att de intäkterna inte är säkra under 2020. 

      Emil Eriksson undrade vad budgeten för Arkivare innefattar. Henrik Ramström svarade arkiveringsutrustning samt att det ska fungera som ett arbetsincentiv. 

      \textbf{\Mba anta det framvaskade förslaget.}


      \subp{B}{Verksamhetsplansförslag för 2020}{}

      Edvard Carlsson presenterade propositionen. 

      Utskottsordförande presenterade förändringarna från verksamhetsplanen för 2019 för respektive utskott.

      Mötet fick chans att ställa frågor till styrelsen. 

      Daniel Bakic \textbf{yrkade på}
      \begin{attsatser}
        \att första meningen i Cafémästeriets beskrivning stryks, samt
        \att ta bort ``därför'' i andra meningen.
      \end{attsatser}

      \textbf{\Mba bifalla yrkandet.}

      Saga Åslund \textbf{yrkade på} 
      \begin{attsatser}
        \att lägga till följande punkt under Delmål 2020 i Nöjesutskottets verksamhetsplan.
            \begin{dashlist}
                \item Integrera och utvärdera den nya posten Banan. 
            \end{dashlist}
      \end{attsatser}

      \textbf{\Mba bifalla yrkandet.}

      Stehpanie Bol \textbf{yrkade på} 
      \begin{attsatser}
        \att lägga till följande punkt under Delmål 2020 i Nolleutskottets verksamhetsplan.
            \begin{dashlist}
              \item Arbeta för att integrera och utvärdera posten PhørstärkarE i utskottet.
            \end{dashlist}
      \end{attsatser}

      \textbf{\Mba bifalla yrkandet.}


      \textbf{\Mba anta det framvaskade förslaget.}

      \subp{C}{Stadgaändring, Styrelsens sammansättning och Informationsutskottet}{}

      Edvard Carlsson och Mattias Lundström presenterade propositionen.

      Filip Larsson tyckte att namnet Vice Ordförande är mer representativt än Kontaktor för den nya styrelseposten.
      Filip tyckte också att det är bra med en Vice inom utskottet som exempelvis kan hjälpa till med Teknik och att Vice Kontaktor varit till otrolig hjälp det här året. 

      Mattias Lundström tyckte inte Vice Ordförande var passande eftersom arbetsuppgifterna främst kommer vara en utveckling av Kontaktorns nuvarande uppgifter inom styrelsen.
      Mattias svarade att anledningen till att man inte föreslagit införandet av en ny Vice post inom utskottet som exempelvis är ansvarig för teknik är att man vill få en bättre sammanhållning inom utskottet. 

      Henerik Ramström förklarade att styrelsen idag har en platt struktur och att han tycker det är väldigt bra. Henrik Ramström förklarade också att styrelsen inte vill skapa en styrelsehirarki utan bara inkludera en till i styrelsen som bara jobbar styrelserelaterat bredvid ordförande. 

      Emil Eriksson gillade förslaget och höll med om att situationen nu inte är hållbart men tyckte inte att chefredaktören bör vara Vice till utskottsordförande. 

      Mötet diskuterade namnförslag. 

      Vincent Palmer \textbf{yrkade på} följande förändring i propositionen
      \begin{attsatser}
        \att ändra namnet Kontaktor till Vice Ordförande, samt
        \att Informationschef ändrar namn till Kontaktor. 
      \end{attsatser}

      \textbf{\Mba avslå yrkandet.}

      %Daniel Bakic tyckte att man kunde ha två Vice inom utskottet. En för teknik och en för information. 

      Henrik Ramström tyckte vi ska ta beslut om stadgaändringen nu och menar att man kan diskutera detaljer angående Reglementesändring på vårterminsmötet.

      Saga Åström \textbf{yrkade på}
      \begin{attsatser}
        \att lägga till InfUencer som namnförslag på ordförande för Informationsutskottet.
      \end{attsatser}

      \textbf{\Mba avslå yrkandet.}
      
%     \textbf{Pontus Landgren begärde sluten votering.}

      \textbf{\Mba bifalla propositionen i dess helhet.}


      %Redaktionell ändring från Saga. Kolla Slack.

      %Bifall för den framvaskande propositionen. Nästa del tas fram på vårterminsmötet. 

      \subp{D}{Ge styrelsen rättigheter att genomföra redaktionella ändringar i styrdokumenten}{}
        Edvard Carlsson presenterade propositionen
        
        Edvard Carlsson \textbf{yrkade på}
        \begin{attsatser}
          \att under förutsättning att propositionen går igenom, även i Reglemente under 4:C Vårterminsmöte och 4:D Hösterminsmöte lägga till
              \begin{dashlist}
                \item e) Redaktionenlla ändringar av styrdokumenten sedan förra terminsmötet.
              \end{dashlist}
        \end{attsatser}


        \textbf{\Mba bifalla propositionen med Edvards tilläggsyrkande.}

      \subp{E}{Uppdatering av policy: Inbjudningar och anmodningar}{}

        Theo Nyman presenterade propositionen. 

        Rasmus Sobel undrade varför HeHE:s chefredaktör anmodas till Nollegasque. Emil Eriksson svarade att HeHE:s chefredaktör historiskt har varit anmodad till Nollegasquen för att kunna rapportera därifrån.

        %Tove tyckte det är konstigt att inte Cophøs är inbjudna till NollEgasque. Theo svarade att alla som är inbjudna är där för representation

        %Mötet diskuterade ordning

        %Diskussion angående ordningen av inbjudningar till NollEgasque. 
        %Emil kom med en sakupplysning om varför HeHEs chefredaktör historisk varit anmodad till Nollegasque och ständigt anmodad till styrelsemöte. 

        %Filip - yrkar på att stryka HeHE:s chefredaktör. . 
        %Elina - yrkande på formulering av jobbare. Styrelsen jämför sig med förslagen

        Filip Larsson \textbf{yrkade på}
        \begin{attsatser}
          \att under rubriken ``Nollegasque'' stryka
          \begin{dashlist} 
            \item ``HeHe:s chefredaktör''.
          \end{dashlist} 
        \end{attsatser}
         
        \textbf{\Mba bifalla yrkandet.}

        Elina Yrlid \textbf{yrkade på}
        \begin{attsatser}
          \att under rubriken ``Följande personer skall bjudas in'' ändra
            \begin{dashlist} 
              \item “Gäster som jobbar på sittningen” 
            \end{dashlist}

            till

            \begin{dashlist} 
             \item “Gäster som bistår i genomförande av och organisation under sittningen”.
            \end{dashlist}
        \end{attsatser}

        \textbf{\Mba bifalla yrkandet.}

        Tove Börjeson \textbf{yrkade på}
        \begin{attsatser}
          \att stryka HM Konungen Carl XVI Gustaf under rubriken ``Nollegasque'', samt
          \att flytta Cophøs till ``inbjudas'' under rubriken ``Nollegasque''.
        \end{attsatser}

        \textbf{\Mba avslå yrkandet.}

        Rasmus Sobel \textbf{yrkade på}
        \begin{attsatser}
          \att flytta Cophøs till ``inbjudas'' under rubriken ``Nollegasque''.
        \end{attsatser}

        \textbf{\Mba avslå yrkandet.}

        %Tove - Yrkande om HM konunge och cophös.
        %Tove förklarade ännu mer varför cophös ska vara där. Henrik replik. 
        %Filip undrade varför itne ØGP är med delmål. Tove tycker det ligger ett värde är där, att de ligger där. Men hon poängetar att phöset är där som represenation.
        %Tove - Anledning för att inte hela styrelsen ska vara inbjudna -> Styrelsen har en budget för representations. 
        %Henrik - Cophös blir också subventionerade.

        %Jakob - Tycker inte man ska få sin tack genom att gå gratis på sittning. Vi har budget till funktoinärsvård och tack för det. 
        %Saga - det finns en poäng med det Jakob säger. Vi är alla funktionärer på lika vilkor. 

        %Går till beslut om tillägsyrkanden. 

        %Yrkande om att styrka HM. Avslag. Kungen är kvar.
        %Votering om Cophös ska flyttas upp till Inbjudnignar under NollEgasque. 38 personer närvarande. 
        %Cophös är kvar under anmodade. 
        %BIFALL

        \textbf{\Mba bifalla det framvaskade förslaget.}

      \subp{F}{Införandet av policy: Hantering av ärenden som faller inom diskrimineringslagen}{}
      
      Edvard Carlsson presenterade propositionen. 

      \textbf{\Mba bifalla propositionen i dess helhet.}

      \subp{G}{Införandet av riktlinje: Använding av G Suite}{}
      Mattias Lundström presenterade propositionen. 

      \textbf{\Mba bifalla propositionen i dess helhet.}

      \subp{H}{Inköp av sektionstält}{}

      Edvard Carlsson presenterade propositionen. 

      \textbf{\Mba bifalla propositionen i dess helhet.}

      \subp{I}{Reglementesändring, uppdatering av postbeskrivningen för Halvledare}{}

      Jonathan Benitez presenterade propositionen. 

      Emil Eriksson undrade vem som är ansvarig för caféet för en dag. Jonathan svarade att Cafémästaren alltid är ytterst ansvarig.

      Mötet diskuterade kort vad som egentligen ingår i halvledarens arbetsuppgifter. 

      Daniel Bakic poängterade att man ska forma en post som man vill att den ska se ut, inte utifrån hur den ser ut idag. 

      \textbf{\Mba bifalla propositionen i dess helhet.}

      \subp{J}{Reglementesändring, beskrivning av Nolleutskottet}{}

      Stephanie Bol presenterade propositionen.

      Daniel Bakic \ypa lägga till
      \begin{dashlist}
        \item ``inom utskottet utse minst ett phøs till phadderansvarig'' under åligganden för utskottet. 
      \end{dashlist}
    
      \textbf{\Mba bifalla yrkandet}.

      \textbf{\Mba bifalla den framvaskade förslaget.}

      \subp{K}{Reglementesändring, namn på NollU-funktionärer}{}

      Edvard Carlsson presenterade propositionen. 


      Filip Larsson undarde varför vi tog bort bindestrecket i Cophøs. Edvard svarade att på senare tiden har sektionen alltid stavat det så här.

      \textbf{\Mba bifalla propositionen i dess helhet.}

      \subp{L}{Reglementesändring, mindre uppdateringar}{}

      Edvard presenterade propositionen. 

      \textbf{\Mba bifalla propositionen i dess helhet.}
      
    \end{paragrafer}
\p{23}{Övrigt}{}

Saga Åslund tackade för att så många närvarade på mötet. 

Daniel Bakic sa åt mötet att gå på F1 Röj!

Matilda Horn påminde mötet att det är Arkad imorgon.

\p{24}{TaFMA}{}
Mötesordförande {\mo} förklarade mötet avslutat 00:14.

\end{paragrafer}

% Mötet ajournerades xx:xx och återupptogs xx:xx.

%\p{}{Styrelsens förslag till resultatdisposition}{}
%Sektionens Förvaltningschef Anders Nilsson presenterade förslaget till resultatdisposition.
%
%\textbf{\Mba godkänna resultatdispositionen}

%Kontrapropositionsvotering

%\newpage
\hidesignfoot
\begin{signatures}{4}
\signature{\mo}{Mötesordförande}
\signature{\ms}{Mötessekreterare}
\signature{\ji}{Justerare}
\signature{\jii}{Justerare}
\end{signatures}
\end{document}
