\documentclass[10pt]{article}
\usepackage[utf8]{inputenc}
\usepackage[swedish]{babel}

\def\mo{Edvard Carlsson}
\def\ms{Sonja Kenari}
\def\ji{Theo Nyman}
%\def\jii{}

\def\doctype{Protokoll} %ex. Kallelse, Handlingar, Protkoll
\def\mname{Styrelsemöte} %ex. styrelsemöte, Vårterminsmöte
\def\mnum{S13/19} %ex S02/16, E1/15, VT/13
\def\date{2019-05-14} %YYYY-MM-DD
\def\docauthor{\ms}

\usepackage{../e-mote}
\usepackage{../../../e-sek}

\begin{document}
\showsignfoot

\heading{{\doctype} för {\mname} {\mnum}}

%\naun{}{} %närvarane under
%\nati{} %närvarande till och med
%\nafr{} %närvarande från och med
\section*{Närvarande}
\subsection*{Styrelsen}
\begin{narvarolista}
\nv{Ordförande}{Edvard Carlsson}{E16}{}
\nv{Kontaktor}{Sonja Kenari}{E15}{}
\nv{Förvaltningschef}{Henrik Ramström}{E16}{}
\nv{Cafémästare}{Jonathan Benitez}{E17}{}
\nv{Sexmästare}{Theo Nyman}{BME18}{}
\nv{Krögare}{Davida Åström}{BME17}{}
\nv{Entertainer}{Saga Åslund}{BME18}{\nafr{11}}
\nv{SRE-ordförande}{Lina Samnegård}{BME16}{}
\nv{ENU-ordförande}{Jakob Pettersson}{E17}{}
\nv{Øverphøs}{Stephanie Bol}{BME17}{\nati{10D}}
\end{narvarolista}


\subsection*{Ständigt adjungerande}
\begin{narvarolista}
%\nv{}{}{}{}
%\nv{Skattmästare}{Daniel Bakic}{E15}{\nafr{10}}
%\nv{Vice Krögare}{Klara Indebetou}{BME17}{}
%\nv{Vice Krögare}{Hjalmar Tingberg}{BME16}{}
\nv{Kårrepresentant}{Filip Johansson}{}{}
\nv{Kårrepresentant}{Anna Qvil}{}{}
%\nv{Valberedningens ordförande}{Axel Voss}{E15}{\nafr{10b}}
%\nv{Fullmäktigeledamot}{Magnus Lundh}{E15}{\nafr{12}}
%\nv{Chefredaktör}{Max Mauritsson}{BME16}{}
%\nv{Inspektor}{Monica Almqvist}{}{}


\end{narvarolista}

%\begin{comment}
\subsection*{Adjungerande}
\begin{narvarolista}
%\nv{post}{namn}{klass}{nati/nafr/tom}
\nv{Diod/Kodhackare}{Malin Heyden}{E16}{}
\nv{Projektfunktionär}{Sophia Carlsson}{BME17}{}
%\nv{Projekfunktionär}{Emma Hjörneby}{BME17}{}
%\nv{}{}{}{}
\end{narvarolista}
%\end{comment}

\section*{Protokoll}
\begin{paragrafer}
\p{1}{OFMÖ}{\bes}
Ordförande {\mo} förklarade mötet öppnat kl.12.10.

\p{2}{Val av mötesordförande}{\bes}
{\valavmo}

\p{3}{Val av mötessekreterare}{\bes}
{\valavms}

\p{4}{Val av justeringsperson}{\bes}
{\valavj}

\p{5}{Godkännande av tid och sätt}{\bes}
{\tosg}

\p{6}{Adjungeringar}{\bes}
%Adam Belfrage adjungerades.{}
Sophia Carlsson adjungerades. \\
Malin Heyden adjungerades.


%\textit{Inga adjungeringar.}


\p{7}{Godkännande av dagordningen}{\bes}

%Davida \ypa lägga till punkten ``Lophtet'' till dagordningen.\\
%Edvard \ypa lägga till punkten ``Ordensband'' til dagordningen.
%Fredrik \ypa att lägga till \S18b ``Teknikfokus utnyttjande av LED-café''.
%Jonathan \ypa ändra punkten §12 från att vara en beslutspunkt till diskussion. \\
%Föredragningslistan godkändes med yrkandet.


Dagordningen godkändes.


\p{8}{Föregående mötesprotokoll}{\bes}
%\latillprot{S12/19}
\textit{\ingaprot}

\p{9}{Fyllnadsval och entledigande av funktionärer}{\bes}
\begin{fyllnadsval} %"Inga fyllnadsval." fylls i automatiskt
%\fval{Moa Rönnlund}{Halvledare}
%\entl{Namn}{Post}
\end{fyllnadsval}

\p{10}{Rapporter}{}
\begin{paragrafer}
\subp{A}{Hur mår alla?}{\info}
Punkten protokollfördes ej.

\subp{B}{Utskottsrapporter}{\info}
CM har genomfört caféfesten. Jonathan har haft lite möten med potentiell kund som vill sköta en extern beställning inför Teknikåttan. Kaffebomberna har även fixats så att de inte läcker längre.

FVU har utöver vanliga verksamheten skrivit massa kontrakt och skött uthyrningar av sektionens utrustningar.

InfU har haft väldigt mycket på gång. Det är massa projekt som börjar rulla på. Redaktionen och Picasso kämpar på med NollEguiden. Macapärerna har fått igång kodhackarna och informationsmässigt drivs vi framåt.

KM hade FED-pub som det blev bra uppslutning på. iZettlescannersen är här och ska implementeras så fort som möjligt. KM har även planerat inför terminens sista gille. Davida har också varit med och sagt upp kontraktet med WEIQ då det krävde mer än det gav. 

NollU har haft temasläpp. Det är mycket att göra nu på grund av att temasläppet har prioriterats. I nuläget rullar luncher med phaddergrupper och uppdragsgrupper. På onsdag är det uppdragskickoff och på fredag är det phadderkickoff. Hela NollU ska även ha möten med utskott som håller event under nollningen samt hela styrelsen.

ENU har genomfört en FED-pub som gick väldigt bra, företagen som var med verkade nöjda. Alumni och SRE har haft möte inför specialiseringskvällen på torsdag.  LMI har dragit igång denna veckan och hittills verkar det ha gått väldigt bra. Utöver arbetet så har Teknikfokus haft intervjuer med sökande till projektgruppen.

E6 har hållit i temasläpp. Utskottets hoodies har även anlänt och är fina! Planeringen för EFING sittningen under nollningen har börjat och allmänt har Theo tillsammans med sin vice Elina gått igenom nollningsschemat.

SRE har tillsammans med ENU planerat inför specialiseringsmingel. Veckan är fullspäckad med CEQ-möten med årskursansvariga samt ett möte med vår nya SVL i eftermiddag.

\subp{C}{Ekonomisk rapport}{\info}
Henrik presenterade en mer utförlig ekonomisk rapport för utskotten specifikt. Utskottsordförandena fick möjlighet att ställa frågor till Henrik.

\subp{D}{Kåren informerar}{\info}
Hela kårstyrelsen är nu tillsatt! 

Denna vecka är det jubileumsvecka. Det saknas även fler marskalker till Jubileumsbalen som är på lördag. Ett anmälningsformulär finns på Teknologkårens Facebooksida. Som tack får man vara med på den magiska kvällen samt en kick-off och tackfest!

Det saknas fortfarande en heltidare för posten som Utbildningsansvarig för externa frågor.

I Cornelis är det tentafrukost 27-29 maj, det kommer även finnas studieplatser öppna i Cornelis så att man kan sitta är och plugga.

Missa inte heller att man kan förtidsrösta i EU-valet i Studiecentrum.

\subp{E}{Omvärldsrapport}{\info}
Styrelsen diskuterade present till V-sektionens 55-års jubileum.

Styrelsen har blivit anmodade till ytterligare en bal i Norge.

\end{paragrafer}

\p{11}{Äskning av pengar för specialiseringskväll}{\bes}
Jakob presenterade handlingen. 

Henrik tycker det är bra initiativ, däremot så tycker han att det borde belasta budgeten beroende på om det ska ligga på SRE eller ENU.

Mötet diskuterade frågan.

Jakob drog tillbaka handlingen. 

\p{12}{Nästa styrelsemöte}{\bes}
\Mba nästa styrelsemöte ska äga rum 2019-05-20 kl.12.10 i E:1123.

\p{13}{Beslutsuppföljning}{\bes}
Henrik \ypa att stryka ``Subvention vårbal'' från beslutsuppföljningen.

\Mbaby

\p{14}{Övrigt}{\dis}
Saga undrade om hur texterna i NollEguiden ska se ut. Styrelsen diskuterade lite alternativ. 

Väggmålning för väggen i E-huset diskuterades. Edvard ska lägga ut en förfrågan i E-sek Events om det finns någon som vill designa den.

Styrelsen diskuterade om att lägga ut en enkät som medlemmar kan fylla i för att se vad sektionen vill ha. 

Lina hälsade att Likabehandlingsombuden fått in massa svar i sin enkät och ska se om de kan sammanställa något till nästa styrelsemöte.
 
\p{15}{OFMA}{\bes}
{\mo} förklarade mötet avslutat kl. 12.51.
\end{paragrafer}

%\newpage
\hidesignfoot
\begin{signatures}{3}
\signature{\mo}{Mötesordförande}
\signature{\ms}{Mötessekreterare}
\signature{\ji}{Justerare}
\end{signatures}
\end{document}
