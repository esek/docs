\documentclass[10pt]{article}
\usepackage[utf8]{inputenc}
\usepackage[swedish]{babel}

\def\mo{Edvard Carlsson}
\def\ms{Mattias Lundström}
\def\ji{Saga Åslund}
%\def\jii{}

\def\doctype{Protokoll} %ex. Kallelse, Handlingar, Protkoll
\def\mname{Styrelsemöte} %ex. styrelsemöte, Vårterminsmöte
\def\mnum{S17/19} %ex S02/16, E1/15, VT/13
\def\date{2019-09-02} %YYYY-MM-DD
\def\docauthor{\ms}

\usepackage{../e-mote}
\usepackage{../../../e-sek}

\begin{document}
\showsignfoot

\heading{{\doctype} för {\mname} {\mnum}}

%\naun{}{} %närvarane under
%\nati{} %närvarande till och med
%\nafr{} %närvarande från och med
\section*{Närvarande}
\subsection*{Styrelsen}
\begin{narvarolista}
\nv{Ordförande}{Edvard Carlsson}{E16}{}
\nv{Kontaktor}{Mattias Lundström}{E17}{}
\nv{Förvaltningschef}{Henrik Ramström}{E16}{}
%nv{Cafémästare}{Jonathan Benitez}{E17}{}
\nv{Sexmästare}{Theo Nyman}{BME18}{}
\nv{Krögare}{Davida Åström}{BME17}{}
\nv{Entertainer}{Saga Åslund}{BME18}{}
\nv{SRE-ordförande}{Lina Samnegård}{BME16}{}
\nv{ENU-ordförande}{Jakob Pettersson}{E17}{}
\nv{Øverphøs}{Stephanie Bol}{BME17}{\naun{4}{15}}
\end{narvarolista}


\subsection*{Ständigt adjungerande}
\begin{narvarolista}
%\nv{}{}{}{}
%\nv{Skattmästare}{Daniel Bakic}{E15}{\nafr{10}}
%\nv{Vice Krögare}{Klara Indebetou}{BME17}{}
%\nv{Vice Krögare}{Hjalmar Tingberg}{BME16}{}
\nv{Kårrepresentant}{Ivar Vänglund}{}{\nafr{10A}}
\nv{Kårrepresentant}{Martin Bergman}{}{\nafr{10A}}
%\nv{Valberedningens ordförande}{Axel Voss}{E15}{\nafr{10b}}
%\nv{Fullmäktigeledamot}{Magnus Lundh}{E15}{\nafr{12}}
%\nv{Chefredaktör}{Max Mauritsson}{BME16}{}
%\nv{Inspektor}{Monica Almqvist}{}{}


\end{narvarolista}

%\begin{comment}
\subsection*{Adjungerande}
\begin{narvarolista}
%\nv{post}{namn}{klass}{nati/nafr/tom}
%\nv{Likabehandlingsombud}{Jonna Fahrman}{BME17}{}
%\nv{Likabehandlingsombud}{Hanna Bengtsson}{BME18}{}
%\nv{Projekfunktionär}{Emma Hjörneby}{BME17}{}
%\nv{Macapär}{Filip Larsson}{E17}{}
%\nv{Kodhackare}{Vincent Palmer}{E18}{}
\nv{Sigillbevarare}{Matilda Horn}{BME18}{}
\end{narvarolista}
%\end{comment}

\section*{Protokoll}
\begin{paragrafer}
\p{1}{OFMÖ}{\bes}
Ordförande {\mo} förklarade mötet öppnat kl 12.12.

\p{2}{Val av mötesordförande}{\bes}
{\valavmo}

\p{3}{Val av mötessekreterare}{\bes}
{\valavms}

\p{4}{Val av justeringsperson}{\bes}
{\valavj}


\p{5}{Godkännande av tid och sätt}{\bes}
{\tosg}

\p{6}{Adjungeringar}{\bes}
%Adam Belfrage adjungerades.{}
%Hanna Bengtsson adjungerades. \\
%Jonna Fahrman adjungerades.
%Vincent Palmer adjungerades.\\
%Filip Larsson adjungerades. 
Matilda Horn adjungerades.
%\textit{Inga adjungeringar.}


\p{7}{Godkännande av dagordningen}{\bes}

%Davida \ypa lägga till punkten ``Lophtet'' till dagordningen.\\
%Edvard \ypa lägga till punkten ``Ordensband'' til dagordningen.
%Fredrik \ypa att lägga till \S18b ``Teknikfokus utnyttjande av LED-café''.
%Jonathan \ypa ändra punkten §12 från att vara en beslutspunkt till diskussion. \\
%Föredragningslistan godkändes med yrkandet.
%Henrik \ypa lägga till punkten ``Faktura till F'' som §13.
Edvard \ypa lägga till punkten ''Äskning av pengar till inköp av ny skrivare'' från de sena handlingarna som \S15.

Föredragningslistan godkändes med yrkandet.

\p{8}{Föregående mötesprotokoll}{\bes}
%Föregånende möteprot lagt till handllingar
\latillprotgodkand{S14/19}

%\textit{\ingaprot}

\p{9}{Fyllnadsval och entledigande av funktionärer}{\bes}
\begin{fyllnadsval} %"Inga fyllnadsval." fylls i automatiskt
%\fval{Moa Rönnlund}{Halvledare}
%\entl{Fanny Månefjord}{Husstyrelserepresentant från och med 30 juni}
\fval{Theo Nyman}{Årskurs BME-2 ansvarig}
\fval{Linnea Söderström}{Årskurs BME-2 ansvarig}
\fval{Lina Tinnerberg}{Årskurs E-2 ansvarig}
\fval{Elina Yrlid}{Årskurs E-2 ansvarig}
\fval{Nelly Ostréus}{Årskurs BME-3 ansvarig}
\fval{Emma Hjörneby}{Årskurs BME-3 ansvarig}
\fval{William Marnfeldt}{Årskurs E-3 ansvarig}
\fval{Måns Lindeberg}{Årskurs E-3 ansvarig}




\end{fyllnadsval}

\p{10}{Rapporter}{}
\begin{paragrafer}
\subp{A}{Hur mår alla?}{\info}
Punkten protokollfördes ej.

\subp{B}{Utskottsrapporter}{\info}

Jotahan meddelade att LED-cafe har öppnat. Jonathan har tagit emot leveranser, ordnat bröd till ENU, lagat mat åt nollorna och försökt förenkla proccessen med IC rapporter från förra veckan. 
Det har även beställts läsk som kommer senare i veckan. 

Henrik och FVU har börjat nollningen med att sälja ouveraller och märken. Dessa har sålts till en viss förlust vilket tillsammans med att det såldes väldigt många så kan vi ligga lite under budgeteringen för tillfället. 
Henrik har även haft möte med utskottets ledningsgrupp om hur våren har gått. 
Utöver det har Henrik bokfört, fakturerat, skött utlåningar, hamrat och i övrigt skött ekonomin.

Mattias meddelade att InfU har fått en bra start på nollningen och att livestreamen under SVEP och Phuskföreläsningen var lyckad. Fotograferna har tagit fina bilder på alla våra större events och dessa har kunnat spridas på sektionens sociala medier. Redaktionens arbete är igång och datum är satt för första upplagan. 
Picasso är som alltid redo för design och Teknokraterna har varit i arbete med diverse teknik under nollningen. DDG ska förhoppningsvis snart komma igång med kodhackarkvällar. Datum för stormöte med utskottet är satt till 4/9. 
Utöver det har Mattias fått iväg anmodningar till våra vänsektioner till sektionens NollEgasque.

KM har hållit i Välkomstgille och hjälp till under UtEDischot. Davida meddelade att allt har gått toppen och att de största stressmomenten nu är över. KM meddelar också att det finns stora mängder kol över.

Jakob meddelade att han precis haft möte med alumniansvariga där det planerades evenemang under hösten. Mötet diskuterade även marknadsföring av alumnigrupper för att fånga in nyexade.
I onsdags ordnade ENU mat till SVEP. Jakob håller på att planera in workshop på Knightecs kontor där han inväntar återkoppling från D-sektionen för att kunna spika datum. Samma gäller för workshop med Venturelab. Utöver det har Jakob mailat och ikväll är det planeringsmöte med Sophia. Imorgon har ENU uppstartsmöte inför hösten.

Saga meddelade att första veckan av nollningen har passerat och därmed har NöjU överlevt UtEDischot. Av de Saga pratat med verkar besökarna haft kul och inte förstått allt som pågick bakom kulisserna. Saga är väldigt tacksam över att det fanns ordningsvakter på plats.
Utskottet har varit väldigt hjälpsam hela veckan och Saga tycker att sina Vices förtjänar en guldstjärna eftersom de har hjälpt till med bland annat Utskottssafarit när hon behövde fixa med UD. Umphmeisters gjorde succé på UD och hjälpte till att hålla stämningen uppe när tekniken började strula. Saga tycker det är väldigt kul att Umphmeisters och Husbandet börjar synas och integreras mer. Nu ska NöjU ladda om till MEK och andra events som kommer under nollningen. 

Theo meddelade att Sexet har hållit i fyra sittningar på åtta dagar och att det har varit megakul fast stressigt. Alla sittningar har gått bra men Sexet behöver förbättra planeringen inför kommande sittningar under nollningen. Behöver också förbättra kommunikationen med andra utskott för att ha bättre koll på vad som ska hända och när. 

Lina meddelade att SRE under tisdagen haft en workshop som Lina tyvärr missade. Lina har dock hört att hennes Vice och Tom som är SRE-ledamot löste det galant. Vid detta tillfälle fick flera från utskottet presentera sig, bland annat skyddsombud och likabehandlingsombud. 
Det blev inte så mycket diskussioner i grupperna som tidigare år vilket Lina tror kan bero på att workshopen hölls tidigare i år. En lärdom till nästa år är att inte ha den för tidigt på året. 
SRE har även börjat planera pluggkvällar och det ska även planeras in CEQ-möten. 


\subp{C}{Ekonomisk rapport}{\info}

Henrik informerade att sektionens ekonomi fortfarande ser bra ut. Ingen märkvärd skillnad sedan föregående möte. Utöver det påminde han om att det är viktigt att lämna in försäljningsrapporter och kvitton för bokföring så fort som möjligt. 

 

\subp{D}{Kåren informerar}{\info}
Kåren informerade att det är nollning. 

Ivar berättade kort om kommande kårevents och lunchföreläsningar.

Martin meddelade att det fortfarande inte finns projektgrupper för Sångarstriden eller Nyårsbalen. Det är viktigt att projekgrupper skapas så snart som möjligt så att eventen kan bli av.

\subp{E}{Omvärldsrapport}{\info}
Styrelsen har blivit inbjuden till \textit{Konglig Elektrosektionen och Sektionen för Medicinsk Teknik vid KTH}s n0llegasque den 14 september. 

Styrelsen har blivit inbjuden till \textit{Elektroteknologsektionen vid CTH}s höstbal och Kallefestivalen den 4-6 oktober.


\end{paragrafer}

\p{11}{Øverphøs informerar}{\info}

Stephanie berättade att NollU är tacksam för att styrelsen ställt upp på nollningsevent och att de gjort ett bra jobb. 
Ber att styrelsen och andra funktionärer ska tänka på att inte höja rösten eller skrika på nollor under pressade situationer när man exempelvis ber dem att gå undan då vissa kan ta illa upp.

Stephanie meddelade att hon återkommer med vad som förväntas av styrelsen under FlyING. 


\p{12}{Datum för höstens sektionsmöten}{\dis}
Mötet diskuterade datum för höstens sektionsmöten. 

Edvard föreslog den 12 november till höstterminsmöte och den 26 och 27 november för valmöte.  

\p{13}{Medaljutdelning på Gasque}{\dis}
Matilda informerade om medaljutdelning av Krusidull-E och Bidragsmedalj.

Mötet diskuterade utdelning av Krusidull-E och Bidragsmedalj.

\p{14}{Faktura till F}{\dis}
Mötet diskuterade idéer om hur fakturan till F-sektionen skall överlämnas. 

Mötet konstaterade att isblock är en bra och billig idé. Theo tycker att styrelsen ska elda lite på kanten för häftig effekt. 

Mötet diskuterade tillverkningsmetod av faktura-inuti-isblock. 

\p{15}{Äskning av pengar till inköp av ny skrivare}{\bes}
Henrik presenterade handlingen. 

Edvard undrade varför kostnaden inte kan budgeteras på vanlig budgetpost. 
Henrik svarade att kostnaden inte passar in på FVUs budgetsbeskrivning utan istället borde gå under inköp.

Jakob undrade om den har en bra scanner och hur dispositionsfonden ligger till. Henrik meddelade att scannern är bra enligt recensioner och att dispositionsfonden ser bra ut. 

\Mbabay

\p{16}{Nästa styrelsemöte}{\bes}
\Mba nästa styrelsemöte ska äga rum 2019-09-09 12.10 i en lämplig 33 sal.

\p{17}{Beslutsuppföljning}{\bes}
%Edvard \ypa stryka ''Projektfunktionär: Vårbal'' från Beslutsuppföljning. Liknande projekt uppmuntras.
%\Mbaby
Davida \ypa skjuta upp ''Inköp av draghandtag till cykelvagn'' till S19/19.

\Mbaby

\p{18}{Övrigt}{\dis}
Kåren informerade om kollegien som finns på kåren och påminde att dessa bör användas och utnyttjas mer. Ivar påminde att man ska komma ihåg att svara om man inte kan delta på kollegiemöten. 

Henrik informerade om sektionens lokaler bör skötas bättre av samtliga funktionärer. Det har varit väldigt stökigt senaste veckan och att nödutgången vid Biljard absolut inte får blockeras.
Saga påminde om alla borde bli bättre på att stänga dörren till BD och HK.  
Davida påminde om att man ska meddela när man tillfälligt ställer fram större saker så att ordningen och reda i lokalerna behålls.

Mötet diskuterade utlåning av sektionensmaterial till phaddergrupper. Risk att saker försvinner.

Edvard tog upp Projekt Hacke av Picasso och Mötet diskuterade hur detta ska användas på bästa sätt. Grafiken är väldigt välgjort och det är kul om den kommer till användning på ett bra sätt. 

Edvard informerade att tilläggsmeriter från Universitetet för engagerade studenter på Sektionen tas bort från och med nästa år. Detta gäller även Kåren.


\p{19}{OFMA}{\bes}
{\mo} förklarade mötet avslutat kl. 13.04
\end{paragrafer}

%\newpage
\hidesignfoot
\begin{signatures}{3}
\signature{\mo}{Mötesordförande}
\signature{\ms}{Mötessekreterare}
\signature{\ji}{Justerare}
\end{signatures}
\end{document}
