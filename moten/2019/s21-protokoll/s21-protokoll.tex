\documentclass[10pt]{article}
\usepackage[utf8]{inputenc}
\usepackage[swedish]{babel}

\def\mo{Edvard Carlsson}
\def\ms{Mattias Lundström}
\def\ji{Henrik Ramström}
%\def\jii{}

\def\doctype{Protokoll} %ex. Kallelse, Handlingar, Protkoll
\def\mname{Styrelsemöte} %ex. styrelsemöte, Vårterminsmöte
\def\mnum{S21/19} %ex S02/16, E1/15, VT/13
\def\date{2019-09-30} %YYYY-MM-DD
\def\docauthor{\ms}

\usepackage{../e-mote}
\usepackage{../../../e-sek}

\begin{document}
\showsignfoot

\heading{{\doctype} för {\mname} {\mnum}}

%\naun{}{} %närvarane under
%\nati{} %närvarande till och med
%\nafr{} %närvarande från och med
\section*{Närvarande}
\subsection*{Styrelsen}
\begin{narvarolista}
\nv{Ordförande}{Edvard Carlsson}{E16}{}
\nv{Kontaktor}{Mattias Lundström}{E17}{}
\nv{Förvaltningschef}{Henrik Ramström}{E16}{}
\nv{Cafémästare}{Jonathan Benitez}{E17}{}
\nv{Sexmästare}{Theo Nyman}{BME18}{}
\nv{Krögare}{Davida Åström}{BME17}{}
\nv{Entertainer}{Saga Åslund}{BME18}{}
\nv{SRE-ordförande}{Lina Samnegård}{BME16}{}
\nv{ENU-ordförande}{Jakob Pettersson}{E17}{}
\nv{Øverphøs}{Stephanie Bol}{BME17}{}
\end{narvarolista}


\subsection*{Ständigt adjungerande}
\begin{narvarolista}
%\nv{}{}{}{}
%\nv{Skattmästare}{Daniel Bakic}{E15}{\nafr{10}}
%\nv{Vice Krögare}{Klara Indebetou}{BME17}{}
%\nv{Vice Krögare}{Hjalmar Tingberg}{BME16}{}
\nv{Kårrepresentant}{Ivar Vänglund}{}{}
\nv{Kårrepresentant}{Martin Bergman}{}{}
%\nv{Valberedningens ordförande}{Axel Voss}{E15}{\nafr{10b}}
%\nv{Fullmäktigeledamot}{Magnus Lundh}{E15}{\nafr{12}}
%\nv{Chefredaktör}{Erik Eriksson}{--}{}
%\nv{Inspektor}{Monica Almqvist}{}{}


\end{narvarolista}

%\begin{comment}
\subsection*{Adjungerande}
\begin{narvarolista}
%\nv{post}{namn}{klass}{nati/nafr/tom}
%\nv{Likabehandlingsombud}{Jonna Fahrman}{BME17}{}
%\nv{Likabehandlingsombud}{Hanna Bengtsson}{BME18}{}
%\nv{Projekfunktionär}{Emma Hjörneby}{BME17}{}
%\nv{Macapär}{Filip Larsson}{E17}{}
%\nv{Kodhackare}{Vincent Palmer}{E18}{}
\end{narvarolista}
%\end{comment}

\section*{Protokoll}
\begin{paragrafer}
\p{1}{OFMÖ}{\bes}
Ordförande {\mo} förklarade mötet öppnat kl 12.15

\p{2}{Val av mötesordförande}{\bes}
{\valavmo}

\p{3}{Val av mötessekreterare}{\bes}
{\valavms}

\p{4}{Val av justeringsperson}{\bes}
{\valavj}

\p{5}{Godkännande av tid och sätt}{\bes}
{\tosg}

\p{6}{Adjungeringar}{\bes}
%Adam Belfrage adjungerades.{}
%Hanna Bengtsson adjungerades. \\
%Jonna Fahrman adjungerades.
%Vincent Palmer adjungerades.\\
%Filip Larsson adjungerades. 

\textit{Inga adjungeringar.}


\p{7}{Godkännande av dagordningen}{\bes}

%Davida \ypa lägga till punkten ``Lophtet'' till dagordningen.\\
%Edvard \ypa lägga till punkten ``Ordensband'' til dagordningen.
%Fredrik \ypa att lägga till \S18b ``Teknikfokus utnyttjande av LED-café''.
%Jonathan \ypa ändra punkten §12 från att vara en beslutspunkt till diskussion. \\
%Föredragningslistan godkändes med yrkandet.
%Henrik \ypa lägga till punkten ``Faktura till F'' som §13.
%Jakob Pettersson \ypa tägga till punkten ''Øverphøs informerar'' som \S16.

Föredragningslistan godkändes.% med yrkandet.

\p{8}{Föregående mötesprotokoll}{\bes}
%\latillprotgodkand{S14/19 \& S15/19}
\textit{\ingaprot}

\p{9}{Fyllnadsval och entledigande av funktionärer}{\bes}
\begin{fyllnadsval} %"Inga fyllnadsval." fylls i automatiskt
%\fval{Moa Rönnlund}{Halvledare}
%\entl{Fanny Månefjord}{Husstyrelserepresentant från och med 30 juni}


\end{fyllnadsval}

\p{10}{Rapporter}{}
\begin{paragrafer}
\subp{A}{Hur mår alla?}{\info}
%Punkten protokollfördes ej.


\subp{B}{Utskottsrapporter}{\info}

Edvard har skrivit E-sektiones versamhetsberättelse 18/19 till Kåren samt skrivit tal till NollEgasque och försökt få hit någon att laga mikrovågsugnen.

Jonathan och CM har kämpat vidare i CM samt haft möten. Trion behöver också prioritera sina studier så vid brist av 10-13 dioder kommer LED vara stängt. 
De har även städat caféet efter gasquehelgen och köpt in massa kanelbullar inför kanenbullens dag på fredag. 

Henrik och FVU har fortsatt sin verksamhet. Henrik har mestadels jobbat med det ekonomiska arbetet. Henrik har även kontaktat folk om att få tag i fakturor. 

Mattias meddelar att InfU har under veckan fått sina fina utskottshoddies.
Fotograferna har fotograferat under Gasque och de skall redigaras i veckan. Mattias har tänkt vattenmärka bilderna med sektions logga och hoppas att det blir snyggt. 
I övrigt har utksottsarbetet fortsatt som vanligt.  

Davida och KM meddelar att det inte har varit gille den här veckan. De har haft planeringsmöte och satt datum för resten av terminen och planerat de tre event som de har den här läsperioden. 

Stephanie och NollU är nöjda med nollningen och tycker att det var en fin avslutning. Nu är det dags för efterarbete i form av bland annat budget och phaddertack. 

Jakob och ENU meddelar att lunchföreläsningen med Ericsson gick bra och att de fyllde hela salen. Han har mailat, planerat events med Academic Work, planerat Workshop med VentureLabs, och diskuterat pris med Knightec och D-sektionen angående deras Workshop. 
Jakob har även spikat datum och form för kvällsevent med BorgWarner enligt avtal. Det blir den 17 oktober i form av en pub med KM. Till sist ska Jakob ha möte med Cellavision i eftermiddag.  

Saga och NöjU har hållt i NöjUs-tur vilket verkar ha varit lyckat. I torsdags var det spekväll och trots elavbrott gick de sista klockan 21.30. Nästa vecka är det fokus på DreamHackE och Sångarstriden.

Theo och Sexmästeriet har hållt i NollEgasque och det gick toppen. Det enda strulet var att laminatorn gick sönder. Theo är väldigt stolt över utskottet.

Lina och SRE har sista pluggkvällen ikväll. Likabehandlingsombud ska under veckan starta igång ett projekt med D-sektionen för alla ettor som anmält sig. I övrigt har arbetet fortsatt som vanligt. 

\subp{C}{Ekonomisk rapport}{\info}
Henrik meddelade att ekonomin ser bra ut likt föregående vecka. Han har skickat ut mötesbokningar inför budgeteringen nästa år.

\subp{D}{Kåren informerar}{\info}

Martin informerade att man kan hämta sitt Mecenatkort i kårens expidition om man saknar det.
Han berättade också att datum för styrelseutbildning är satt och att information om anmälan har skickats ut. 

Ivar informerade att ansökan att söka värd på Arkad stänger om två dagar. 
Han meddelade också att kåren har mösskifte i Kåraulan där det serveras kladdkaka och dryck med start kl 21 den 4 oktober. Han poängterade att kåren har stor fikabudget. 

\subp{E}{Omvärldsrapport}{\info}

Mattias rapporterade att helgens aktiviteter med våra vänsektioner har varit lyckad och uppskattad. Nästa vecka väntar Kallefestivalen på Elektroteknologsektionen vid Chalmers. 

\end{paragrafer}

\p{11}{Sektionens mailserver}{\info}

Mattias informerade om situationen med Exim som är sektionens mailserver. 
LDC har meddelat att den behöver uppdateras eller stängas ner på grund av att den inte längre anses säker. En uppdatering anses inte möjligt då huvudservern har för gammal mjukvara. 
Tills vidare har en temporär återgärd tillsats utav rekommendation från LDC, IT vid LU. Trots det är sektionens mailserver inte säker och det finns risk för dataintrång, vilket Macapärerna och datoransvariga vet om. 

Mattias har gjort ett projekt som leds av Macapärerna. De skall titta på hur sektionen kan få en säker och fungerande mail. Möjliga återgärder är exempelvis att ordna en ny mailserver eller att gå över till G-Suite.

\p{12}{Fotovägg i Diplomat}{\dis}

Saga tog upp diskussionen om fotovägg i Diplomat. Mötet diskuterade idéer om vad man kan göra. 

Mötet konstaterade att sektionen har många fina bilder som bör användas. Edvard och Davida poängterade att vi bör ha kvalitetsbilder och att bilderna bör vara jämt fördelade över året. 

Edvard \ypa lägga till 'Fotovägg i Diplomat' på Beslutsuppföljning till S26/19 med Saga som ansvarig. 

\Mbaby


\p{13}{Idéer till funktionärstacket}{\dis}

Henrik meddelade att sektionen har cirka 50 000 SEK kvar i budget till funktionärstack. 

Edvard tog upp exempel på vad som gjort tidigare och mötet diskuterade idéer. 

Edvard lyfte idéen om att åka till en stuga likt KPL. David undrad om folk är tillräckligt taggade.. Martin meddelade att D sektionen gör det och att det fungerar bra. 

Henrik meddelade att detta diskuterades förra året och att styrelsen då valde att hitta en annan lösning då de hade inneburit för hög arbetsbelastning på styrelsen. 

Saga poängterade att vi behöver boka datum nu om vi skall ha hjälp av en nation då dessa bokas upp snabbt. 

Mötet diskuterade lämpliga datum så att det ska undvika datumkrock med Sångarstriden och F1 Röj. 

%Edvard \ypa att lägga till 'Idéer till funktionärstacket' till Beslutsuppföljning med Edvard som ansvarig. 
Edvard föreslog att skjuta diskussionen till nästa styrelsemöte. 

\p{14}{Nästa styrelsemöte}{\bes}
\Mba nästa styrelsemöte ska äga rum 2019-10-07 12.10 i E:1123.

\p{15}{Beslutsuppföljning}{\bes}
%Edvard \ypa stryka ''Projektfunktionär: Vårbal'' från Beslutsuppföljning. Liknande projekt uppmuntras.
%\Mbaby
%Davida \ypa skjuta upp ''Inköp av draghandtag till cykelvagn'' till nästa styrelsemöte.
%\Mbaby
\textit{Inga beslut att följa upp}
%Inga beslut att följa upp. Sätt automatisk

\p{16}{Övrigt}{\dis}
Henrik lyfte frågan om fakturan till F sektionen.

Henke meddelade att det är skitigt i HK och ber styrelsen att hjälpas åt att städa. Styrelsen planerade in städkväll.

Theo meddelade att budgeten för Sexmästeriet är räddad då sektionen gjorde bra ifrån sig i baren under NollEgasque.

Davida tog upp invigningen av nya Biljard som ska ske nästa gille och frågade om idéer.

Stephanie planerar in datum för Phaddertack och ber styrelsen att vara noga med att skriva in alla events i vår gemensamma kalender. 

Martin meddelade att det antagligen inte blir en nyårsbal då en projektgupp saknas. 

\p{17}{Sammanfattning av mötet}{\info}
Mattias informerade om situationen med sektionens mailserver. 

Mötet diskuterade fotovägg i Diplomat. 'Fotovägg i Diplomat' lades på Beslutsuppföljningen till S26/19 med Saga som ansvarig. 

Mötet diskuterade idéer till funktionärstacket. 

\p{18}{OFMA}{\bes}
{\mo} förklarade mötet avslutat kl. 13.07
\end{paragrafer}

%\newpage
\hidesignfoot
\begin{signatures}{3}
\signature{\mo}{Mötesordförande}
\signature{\ms}{Mötessekreterare}
\signature{\ji}{Justerare}
\end{signatures}
\end{document}
