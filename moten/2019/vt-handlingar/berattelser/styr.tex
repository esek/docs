\documentclass[../_main/handlingar.tex]{subfiles}

\begin{document}
\berattelse{Styrelsen 2018}

I början av året låg fokus på att alla ska komma in i sina nya roller och lära känna
varandra. Sedan var det full rulle med kollegieskiphten, otroligt kul KPL och ett strålande funktionärsskiphte. Därefter var styrelsen iväg på styrelseutbildning hos Kåren och ekonomiutbildning hos Förvaltningschefen och dess företrädare. Verksamheten rullade på bra med lunchmöten varje vecka och kvällsmöten vid behov.

Inför Vårterminsmötet skrev styrelsen ihop en handfull propositioner som behandlade olika inköp samt hur sektionen väljer in några av sina funktionärer. Vi behandlade även många bra motioner som skickats in av sektionens medlemmar. Efter sektionsmötet låg stor fokus på dagligverksamheten.

Under sommaren tog vi en välförtjänt paus som sedan avslutades med en sittning med föregående års styrelse. Det fördes även ett par diskussioner gällande olika sorters inköp varpå en ny fasadskylts köptes in och monterades upp. Strax innan nollningen hölls möte med NollU för att stämma av vad som komma skall, mötet avslutades med trevligt häng tillsammans.

Det var bra samarbete mellan utskotten under nollningen och styrelsen hjälpte till här och var. Det var väldigt roligt att synas mycket och jag tror det lämnade ett väldigt gott intryck hos de nyankomna studenterna då många sedan sökte sig till alla möjliga funktionärsposter. Några incidenter inträffade, men det var ingenting som inte gick att hanterna. Under och efter nollningen var några av oss iväg på besök hos våra grannsektioner i Norge respektive på Chalmers, vilket var väldigt roligt.

Efter nollning och utomlundsresor togs en kort paus påföljt av arbete inför de kommande sektionsmötena. Tillsammans med Valberedningen anordnades ett mycket lyckat Expo. Söktrycket på sektionens poster har aldrig varit så högt som i år, vilket är otroligt kul! Inför Höstterminsmötet diskuterades framförallt sektionens Policys varpå många fick sig en updatering. Stort fokus låg även på budgetförslaget för nästkommande år som vi tror blev väldigt bra. På mötet fördes långa diskussioner kring hur sekionen väljer Phös och alla verkar vara överens om att det är ett återkommande problem.

Innan årets slut valde styrelsen in Cophös på ett extra stort styrelsemöte som bara behandlade den punkten. Vi tror att det var den absolut bästa med det systemet sektionen har idag. Under december månad hölls Musikhjälpen i Lund där sektionen hade många olika evenemang för att bidra till Radiohjälpen. VI hade en otroligt duktig projektgrupp för detta som blev invalda som projektfunktionärer. Därefter hölls ett riktigt lyckat funktionärstack som började på Utmaningarnas Hus i Malmö och som rundades av med sittning och eftersläpp på Göteborgs nation. Sist men inte minst avslutades styrelsens uppdraf med Kristna Pulka Lägret (KPL) med den nya styrelsen, där de fick lära sig att det inte går att driva med Snyggelsen 2018. 




\begin{signatures}{1}
    \mvh
    \signature{Daniel Bakic}{Ordförande 2018}
\end{signatures}

\end{document}
