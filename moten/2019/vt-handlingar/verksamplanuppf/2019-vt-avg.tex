\documentclass[../_main/handlingar.tex]{subfiles}

\begin{document}
\verksplanuppf{VT 2019 (avgående styrelsen)}

\subsubsection*{Styrelsen}

\subsubsection*{Informationsutskottet}

\subsubsection*{Källarmästeriet}

Under 2018 tog vi tillsammans med picasso fram en poster med reklam för gillen som vi satte i andra
hus. Vi upplevde att det kom lite fler besökare från andra sektioner efter det. Sent under året hölls
en pubrunda genom kåren och då jobbade vi tillsammans med några från K i Edekvata och det gick
bra. Pubrundan hade tema musikhjälpen och på detta gillet och några andra samarbetade vi med
projektgruppen för musikhjälpen som drog i massa bra aktiviteter för att samla in pengar.
Vi har fortsatt att köpa in svenskt kött till kläggen, och priset på ett klägg ligger fortfarande på
studentvänliga 35 kr, eller 15 för en funktionär.
Det har varit lite oklarheter runt alkoholförrådet under 2018, särskilt med alkoholhanteringsystemet.
Vi har gjort vårt absolut bästa för att få det rätt, men systemet bör antingen upprustas eller bytas ut.


\subsubsection*{Nolleutskottet}

I planeringen av nollningen lades fokus på att likt tidigare år ha en mångfald av olika aktiviteter under
nollningen så att de nyantagna skulle kunna känna att det fanns något för alla. Många av
nollningsaktiviteter var traditionsenliga men vi lade även till några nya event på nollningsschemat.
Både under våren och under hösten skickades formulär och utvärderingar ut till sektionens
medlemmar för att de skulle få möjlighet att påverka utformningen av nollningen. Under våren hölls
personliga möten med utskottets funktionärer för att utvärdera hur alla mådde och hur de kände för
arbetsbördan.
Vad gäller att integrera de internationella studenterna mer i Sektionen gjordes väldigt stora framsteg
under vår tid som nollningsutskott. Mycket tack vare att vi hade ett internationellt ansvarigt Cophøs.
Vi uppmuntrade kontinuerligt phaddrarna att förmedla en positiv attityd till studierna till nollorna
och flera pluggkvällar hölls under nollningen. Vi arbetade också för att informera mycket om
Sektionens utskott redan under nollningen. Ett utskottssafari hölls, flera av utskotten höll egna
evenemang och styrelsen var involverad i mycket.



\subsubsection*{Cafémästeriet}

När verksamhetsplanen för 2018 skrevs hade man ännu inte genomgått den stora förändringen från
att ha en anställd till att hela verksamheten drevs ideellt och flera av målen kändes därför inte
relevanta. Delmålen som handlar om att utvärdera belöningssystem och att göra kvartalsbokslut har
man därför inte arbetat mot. Posten Halvledare och att ibland stänga tidigare har delvis minskat
arbetsbelastningen för Cafémästare, Vice och Inköps- och lagerchefer men det krävs ytterligare
förbättring på den punkten. Arbetsbeskrivningarna har blivit tydligare och tillgängliga på engelska
vilket underlättar för utbytesstudenterna.
Försäljningspriser har utvärderats och genom att färre olika sorters mackor finns på menyn har
svinnet minskat. Några försök med nya snacks och andra varor testades men tidsbrist och hög
arbetsbelastning gjorde att de aldrig riktigt hanns med. Men priserna fortsatte vara låga och utbudet
konkurrenskraftigt.
Framförallt posten Halvledare som var ny till verksamhetsåret 2018 har utvärderats kontinuerligt och
ändrades lite från vår- till höstterminen vilket gav bra resultat. Ytterligare utvärdering och eventuellt
fler justeringar kan dock behövas i framtiden eftersom att det vissa läsperioder var så få Halvledare
att det var svårt att få en bild över hur det fungerade.

\subsubsection*{Förvaltningsutskottet}

Förvaltningsutskottet har under året arbetat på löpande med såväl den ekonomiska delen
som lokalunderhåll. Vice Förvaltningschefen har tillsammans med Hustomtarna arbetat
självständigt under årets gång och har dragit i många egna projekt de velat ta tag i. De
funktionärer som berörts av Sektionens ekonomi och bokföring har dels fått en
ekonomiutbildning i början av året, men även stöd under årets gång för att underlätta så
mycket som möjligt för alla inblandade.

\subsubsection*{Studierådet}

Studierådet har arbetat med att vara ett synligt utskott genom att bland annat ha posters i huset samt sätta upp mötesprotokoll. Vi har försökt förbättra pluggkvällarna under nollningen för att få mer studiero och det fanns många duktiga pluggphaddrar som kunde hjälpa till. Vi har fortsatt arbetet med att ständigt förbättra utbildningarna genom CEQ-möten.


\subsubsection*{Sexmästeriet}

Sexmästeriet har under 2018 arbetat aktivt för att arrangera prisvärda evenemang med god
kvalité. Vi tror (och hoppas) att ni tycker vi har lyckats med det. På grund av att nollningen är
den mest aktiva perioden för hela sektionen arrangerades flest evenemang under denna
tidsperiod. Men vi har även jobbat hårt för att arrangera evenemang på andra delar av året.
Pump har aldrig sett bättre ut och vi har jobbat för att minimera svinnet i alkohollagret.

\subsubsection*{Nöjesutskottet}

Nöjesutskottet har efter HT varit med och organiserat upp deltagandet i Sångarstriden samt
gjort en överlämning till det nya nöjesutskottet. Detta bidrar till att nästa års NöjU har bättre
koll på deras arbetsuppgifter och således också kunna följa verksamhetsplanen på ett bra
sätt. Deltagandet i SåS och det tillhörande SåStacket har ökat intresset för sångarstriden
vilket är väldigt bra och det är kul att vi för första gången på länge har två stridsrop som
verkade taggade på att bygga vidare på förra årets framgångar.

\subsubsection*{Näringslivsutskottet}

ENU har under året varit i kontakt med både nya och befintliga företagskontakter. Prissätningen sågs
över både inom sektionen och tillsammans med Näringslivskollegiet. ENU disskuterade vilka
evenemang som var nyttiga för studenterna och valde att lägga fokus på dessa, både vinstdrivande och
icke-vinstdrivande.

\newpage
\begin{signatures}{10}
    \mvh
    \signature{Daniel Bakic}{Ordförande 2018}
    \signature{Axel Voss}{Kontaktor 2018}
    \signature{Magnus Lundh}{Förvaltningschef 2018}
    \signature{Elin Johansson}{Cafémästare 2018}
    \signature{Andreas Bennström}{Øverphøs 2018}
    \signature{Fanny Månefjord}{SRE-ordförande 2018}
    \signature{Isabella Hansen}{Ordförande Näringslivsutskottet 2018}
    \signature{Alexander Wik}{Sexmästare 2018}
    \signature{Malin Heyden}{Krögare 2018}
    \signature{Adam Belfrage}{Entertainer 2018}
\end{signatures}

\end{document}
