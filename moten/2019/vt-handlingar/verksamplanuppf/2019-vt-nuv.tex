\documentclass[../_main/handlingar.tex]{subfiles}

\begin{document}
\verksplanuppf{VT 2019 (nuvarande styrelsen)}

\subsubsection*{Styrelsen}

\subsubsection*{Informationsutskottet}

\subsubsection*{Källarmästeriet}

Källarmästeriet har jobbat med att nå ut till fler teknologer genom affischer, facebookevenemang via sektionens sida och publicering i kårnytt och HeHe. Vi har under årets första läsperiod haft mycket god uppslutning och vi ser även att vi har haft en relativt stor del gäster från andra sektioner. VI ser att många verkar ha börjat se gillet som det naturliga stället att gå efter Fredmans vilket är jättekul! 

Samarbetet både inom och utanför sektionen har fungerat bra. Vi har haft gille för teknikfokus och NöjU har haft både biljardmatcher och På styret på gillen vilket har varit mycket uppskattat. Sammarbete med andra sektioner har skett via Sexmästarkollegiet där vi deltagit i en pubrunda i samband med ET-slasque. 

För att hålla konkurrenskraftigt pris på mat och dryck men samtidigt nå de budgetmål som satts för utskottet har ekonomin följts upp noggrant. Kalkyler av kostnader för klägg har utvärderats och vi har bestämt oss för att behålla prisnivåerna från föregående år och istället hantera det underskott som KM stått inför med aktivt arbete mot svinn, mycket PR och kontinuerlig ekonomisk uppföljning. Hittills går utskottets ekonomi bra men vi har ändå valt att lägga en motion för att ge oss mer ekonomiskt utrymme under resten av året.  

För att minimera det svinn som sker i alkohollagret har utbildning i alkoholhanteringssystemet hållits både inom utskottet och för sexmästeriet. Då vårt alholholhanteringssystem lämnar en del att önska när det kommer till användarvänlighet så har utredning av iZettles lagerhanteringssystem, vilket Data använder idag, inletts. Vi har sett över den utgångna alkohol som finns och kasserat denna vilket kommer leda till att året belastas med mycket svinn, som egentligen skulle belastat tidigare år. 

\subsubsection*{Nolleutskottet}

\subsubsection*{Cafémästeriet}

Vi har tittat igenom arbetsbeskrivningarna och jag tycker personligen att de just nu är rätt så bra nu så vi har lagt vår energi på andra saker i stället.

Vi i Cm har utvärderat priset på det mesta och det är svårt att konkurrera mer än vad vi gör just nu, i och med att vi har några fåtals öre vinst på det mesta. Vi har även gjort en större kalkyl på varor som vi producerar för att få en ungefärlig pris på detta.

Inköparna har gjort ett bra jobb för att minska deras arbetsbelastning. De har till exempel fixat ett schema så att varje inköpare får ta lika många pass. För oss Vice- och Cafémästare så har det inte varit lika enkelt, men vi letar efter halvledare så att det ska bli enklare för oss.

Vi har sorterat om lagret i LED så att allting ska bli enklare och lite mer logiskt. Samt så har detta gjort att det finns mer plats nu för nya saker och allting blir enklare att sortera. Vi har även börjat använda lådan till ryggsäckarna för att hålla det renare där. För LEDs förråd så har vi gjort flera men färre beställningar av läsk för att förbättra hygienen där, som förra året.

\subsubsection*{Förvaltningsutskottet}

\subsubsection*{Studierådet}

Studierådet ska verka för att vara ett synligt utskott. Detta görs genom att försöka synas mycket på nollningen, genom workshop i första läsveckan samt flera pluggkvällar. Vi tycker det är viktigt att nyantagna studenter blir medvetna om att utskottet finns och vad vi ska arbeta för. För att öka synligheten för vissa specifika poster som är viktigt att sektionens medlemmar vet om trycker vi posters. Studierådet ska även verka för att driva studierelaterade evenemang kontinuerligt under året, och vi har i nuläget planer på en specialiseringskväll tillsammans med ENU samt Latex-föreläsning. Vi ska även verka för att ha representanter från alla årskurser. I dagsläget har vi bara från årskurs 1-3, men inte för alla år och program i årskurs 4 och 5. 

\subsubsection*{Sexmästeriet}

\subsubsection*{Nöjesutskottet}

\subsubsection*{Näringslivsutskottet}

ENU har strukturerats om i år där näringslivskontakterna kontinuerligt ska söka kontakt med företag de finner av intresse. Detta ska förhoppningsvis leda till att vi söker kontakt med en bred grupp företag och således nå fler BME-företag och nya företag. Utskottet har kontaktat de företag som vi haft bäst kontakt med förra året för att bibehålla ett bra samarbete med dem. I början av året sågs sektionens priser över och korrigerades efter övriga sektioners priser och vad som ansågs rimligt att ändra på. Alumniverksamheten har integrerats i utskottet och har en övergripande plan över vad som man vill genomföra under året.

\newpage
\begin{signatures}{10}
    \mvh
   \signature{\ordf}{Ordförande 2019}
   \signature{\sekr}{Kontaktor 2019}
   \signature{\fvc}{Förvaltningschef 2019}
    \signature{\cafem}{Cafémästare 2019}
    \signature{\ophos}{Øverphøs 2019}
    \signature{\sreordf}{SRE-ordförande 2019}
    \signature{\enuordf}{ENU-ordförande 2019}
    \signature{\sexm}{Sexmästare 2019}
    \signature{\krog}{Krögare 2019}
    \signature{\ent}{Entertainer 2019}
\end{signatures}

\end{document}
