\documentclass[../_main/handlingar.tex]{subfiles}

\begin{document}
\proposition{Ge styrelsen rättigheter att genomföra redaktionella ändringar i styrdokumenten }

Inför vårterminsmötet har vi i styrelsen spenderat mycket tid bland styrdokumenten, och i våra undersökningar har vi upptäckt att de innehåller massvis av småfel. Det är versaler på fel ställen, stavfel, punkter mitt i meningar ord som hamnat fel i texten. Vi skulle vilja åtgärda dessa fel och vill därför ge styrelsen redaktionella rättigheter över våra styrdokument. Med redaktionella ändringar syftar vi på redigeringar gjorda för att rätta och förbättra språkliga samt typografiska fel där det krävs, viktigt är dock att understryka att dessa aldrig ska medföra en förändring av innehållet i styrdokumenten.

Rimligtvis är det inte med dessa fel styrdokumenten har antagits utan de har kommit till av misstag (till exempel HTF-ansvarig lades in som HFT-ansvarig). Om så är fallet skulle vi kunna tillrätta dem redan nu men eftersom det i många fall är svårt för oss att bevisa detta och då blir det även svårare att motivera ändringarna. På så sätt hade dessa rättigheter underlättat arbetsprocessen.
 
Något annat som hade förenklat arbetet med styrdokumenten var ifall de var versionshanterade, så att man alltid lätt kan gå tillbaka till tidigare versioner och följa vad till exempel styrelsen har ändrat. Därför skulle vi också vilja ha som krav att tidigare versioner av styrdokumenten finns tillgängliga på sektionen.

Med anledning ovan yrkar styrelsen
\begin{attsatser}
    \att under Kapitel 8 i Stadgarna lägga till

    \subsection*{§8:14 Redaktionella rättigheter}
    
    Styrelsen innehar rätten att genomföra redaktionella ändringar som rättar språkliga eller typografiska fel bland styrdokumenten i det fall då styrelsen är enhetlig i beslutet. 

    \att under  Kapitel 14 under §14:2 i Stadgarna lägga till 
    
    Styrelsen innehar rätten att genomföra redaktionella ändringar i Stadgarna i enlighet med §8:14.
    
    Efter genomförd ändring ska tidigare publicerade versioner av Stadgarna bevaras på sektionen så att spårbarhet finns.


    \att under  Kapitel 15 under § 15.4 i Stadgarna lägga till 
    
    Styrelsen innehar rätten att genomföra redaktionella ändringar i Reglementet i enlighet med §8:14.
    
    Efter genomförd ändring ska tidigare publicerade versioner av Reglementet bevaras på sektionen så att spårbarhet finns.
 

\end{attsatser}

\begin{signatures}{1}
    \ist
    \signature{\ordf}{Ordförande}
\end{signatures}

\end{document}
