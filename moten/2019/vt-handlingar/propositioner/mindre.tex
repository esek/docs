\documentclass[../_main/handlingar.tex]{subfiles}

\begin{document}
\proposition{Mindre reglementesändringar}

Denna proposition är en gruppering av några mindre reglementesändringar. Där reglementet antingen är utdaterad, inte längre relevant eller inte speglar verksamheten, några föreslagna ändringar ämnar också att tydliggöra.

\subsection{Studerandeskyddsombud}
    
    Den juridiska titeln är “studerandeskyddsombud” och för att underlätta funktionärernas arbete gentemot universitetet kan det vara bra att ha det uppdaterat. 

\subsection{HeHE}

Enligt reglementet är HeHE ett veckoblad, dock är det inte på detta sätt dess verksamhet idag bedrivs och har inte varit så på flera år faktiskt. Vi skulle därför vilja ändra detta och endast benämna det som HeHE i reglemente. På grund av att HeHE tidigare har varit en veckolig informationskälla har det används för att meddela om sektionsmöten, därför står det att inför ett sektionsmöte ska kallelsen skickas ut i två upplagor av HeHE samt att detta är ett av ordförandes åliggande. Detta tycker inte vi längre är relevant och anser därför att dessa rader bör strykas från reglementet. 

\subsection{Informera via E-post}

Inte nog med att kallelsen ska ut i HeHE så ska även alla utskottsordförande informera sina funktionärer om mötet via E-post. Att de ska informera tycker vi naturligtvis är viktigt men att det sker via E-post bör inte vara ett krav, därför vill vi ta bort detta krav.

\subsection{Terminsmöten}

I reglementet beskrivs det vilka punkter som ska tas upp vid terminsmöten, dessa är dock lite otydliga och missvisande. De föreslagna ändringar ämnar att tydliggöra bland detta. Till exempel är det inte uttryckt att styrelsen ska skriva utskottsrapporter.  

\newpage

På dessa grunder yrkar styrelsen
\begin{attsatser}
    
    \att i reglementet 10:2:M ändra 
        \begin{emptylist}
            \item Skyddsombud med ansvar för fysisk miljö (1)
         \end{emptylist}
    
        till 
    
        \begin{emptylist}
            \item Studerandeskyddsombud med ansvar för fysisk miljö (1)
        \end{emptylist}

    \att i reglementet 10:2:M ändra 
        \begin{emptylist}
            \item Skyddsombud med likabehandlingsansvar (2)
         \end{emptylist}
    
        till 
    
        \begin{emptylist}
            \item Studerandeskyddsombud med likabehandlingsansvar (2)
        \end{emptylist}
    
        
    \att i reglementet 10:2:F under Chefredaktör styrka det gulmarkerade
        \begin{emptylist}
            \item Chefredaktör (1)
            \begin{dashlist}
                \item Har det övergripande ansvaret för \hl{veckobladet} HeHE och vad därmed äga sammanhang. 
                \item Ansvarar tillsammans med Picasso, Redaktörer och NollU för produktion av nollEguiden
            \end{dashlist}
        \end{emptylist}
   

   \att i reglementet 10:2:F under Redaktörer styrka det gulmarkerade
        \begin{emptylist}
            \item Redaktörer (4) 
            \begin{dashlist}
                \item Bistår Chefredaktören i dennes arbete med \hl{veckobladet} HeHE. 
                \item Ansvarar tillsammans med Chefredaktör, Picasso och NollU för produktion av nollEguiden.
            \end{dashlist}
        \end{emptylist}


    \att i reglementet 4:E Kallelse styrka det gulmarkerade

    \hl{Kallelsen skall, förutom vad som nämns i stadgans §4:13, även införas i HeHE två gånger innan mötet hålls. För extra Terminsmöte gäller det en gång innan mötet hålls.} 

    Det åligger Sektionens Ordförande att muntligen informera studenter i årskurs 1 och 2, och som är ordinarie medlemmar i sektionen om Sektionsmötet i samband med en föreläsning. 

    Det åligger utskottscheferna eller motsvarande att via E-post i god tid informera utskottsfunktionärerna om Sektionsmötet.

 \newpage
    \att i reglementet 10:2.C Ordförande styrka det rödmarkerade

    \subsubsection{Ordförande}
    Det åligger Ordföranden
    \begin{attlist}
    \item representera Sektionen och föra dess talan,
    \item sammankalla handlingar till Sektionsmöte,
    \item tillsammans med Talmannen upprätta lämplig föredragningslista,
    \item \hl{införa kallelse i HeHE enligt reglementet,}
    \item muntligen informera studenter i årskurs 1 och 2, som är ordinarie medlemmar i sektionen om Sektionsmötet i samband med en föreläsning,
    \item tillsammans med Valberedningens Ordförande planera Expot och valmötet,
    \item sammankalla och upprätta lämpliga handlingar till Styrelsesammanträden,
    \item leda Styrelsesammanträdena och leda arbetet i Styrelsen,
    \item organisera Kurs på landet,
    \item kontinuerligt utbyta information med Inspektor,
    \item aktivt deltaga i TLTH:s Fullmäktiges möten och föra Sektionens talan,
    \item aktivt deltaga i TLTH:s Ordförandekollegie och utbyta information mellan Sektionerna,
    \item tillse att Sektionen är representerad vid TLTH:s kårbal och andra Sektioners högtidligheter,
    \item närvara vid Sektionens Nollegasque, skifte och Sektionsmöte,
    \item städa efter skiphtet, samt
    \item ansvara för att det hålls en omsitts för Ordförandeposten och en för posterna Talman, Sigillbevarare och Revisorer gemensamt.
    \end{attlist}

\newpage

\att i reglementet 4:C Vårterminsmöte ändra till följande

\subsection{Vårterminsmöte}
På Vårterminsmötet skall, förutom de som nämns i stadgans §4:10,
följande ärenden tas upp:
\begin{alphlist}
    \item beslutsuppföljning,
    \item utskottsrapporter (kort skriftlig redogörelse av utskottets och valberedningens verksamhet), samt
    \item uppföljning av verksamhetsplan (både nuvarande och avgående)
    \item ekonomisk rapport.
\end{alphlist}

till 

\subsection{Vårterminsmöte}
På Vårterminsmötet skall, förutom de som nämns i stadgans §4:10,
följande ärenden tas upp:
\begin{alphlist}
    \item beslutsuppföljning,
    \item utskottsrapporter (kort skriftlig redogörelse av utskottets\hl{, styrelsens } och valberedningens verksamhet), samt
    \item uppföljning av verksamhetsplan (både nuvarande och avgående \hl{styrelse})
    \item ekonomisk rapport.
\end{alphlist}
\changenote

\newpage

\att i reglementet 4:D Höstterminsmöte ändra till följande

\subsection{Höstterminsmöte}
På Höstterminsmötet skall, förutom de som nämns i stadgans §4:11,
följande ärenden tas upp:
\begin{alphlist}
    \item beslutsuppföljning,
    \item utskottsrapporter (kort skriftlig redogörelse av utskottets och valberedningens verksamhet),
   	\item uppföljning av verksamhetsplan
    \item ekonomisk rapport.
\end{alphlist}

till

\subsection{Höstterminsmöte}
På Höstterminsmötet skall, förutom de som nämns i stadgans §4:11,
följande ärenden tas upp:
\begin{alphlist}
    \item beslutsuppföljning,
    \item utskottsrapporter (kort skriftlig redogörelse av utskottets\hl{, styrelsens } och valberedningens verksamhet),
   	\item uppföljning av verksamhetsplan
    \item ekonomisk rapport.
\end{alphlist}
\changenote

\end{attsatser}







\begin{signatures}{1}
    \ist
    \signature{\ordf}{Ordförande}

\end{signatures}

\end{document}
