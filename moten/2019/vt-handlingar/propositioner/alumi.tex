\documentclass[../_main/handlingar.tex]{subfiles}

\begin{document}
\proposition{Reglementesändring, alumniverksamheten till näringslivsutskottet}

Eftersom Alumniverksamheten numera ligger under ENU istället för InfU anser vi att reglementet bör uppdateras därefter.

För att reglementet ska vara uppdaterat efter hur utskotten bedrivs idag yrkar styrelsen därför


\begin{attsatser}
    \att i reglemente, kapitel 9, §9:2:B stryka det gulmarkerade
    \subsubsection{Informationsutskottet, InfU}
            Informationsutskottet har till uppgift att ansvara för informationsspridningen på Sektionen, Sektionens tekniska utrustning\hl{ och Sektionens alumniverksamhet}.

    Det åligger utskottet att
    \begin{tightdashlist}
    \item se till att Sektionens hemsida fungerar bra och har uppdaterad information.
    \item se till att Sektionens tekniska utrustning fungerar.
    \item skriva och publicera nollEguiden och HeHE.
    \item under nollningen utbilda nyantagna studenter i LTH:s och Sektionens datorsystem.
    \item \hl{arrangera minst ett alumnievenemang under året.}
    \end{tightdashlist}

    

att i reglemente, kapitel 9, §9:2:C ändra


\subsubsection{Näringslivsutskottet, ENU}
    Näringslivsutskottet har till uppgift att vara kopplingen mellan näringslivet och Sektionens medlemmar.

    Det åligger utskottet att
    \begin{tightdashlist}
    \item anordna aktiviteter som främjar Sektionens medlemmar inför arbetslivet.
    \item vårda och utveckla samarbeten med företag.
    \item tillgodose Sektionens behov av sponsring.
    \item anordna en arbetsmarknadsmässa för Sektionens medlemmar.
    \item hålla ständig kontakt med övriga sektioners motsvarande utskott.
    \end{tightdashlist} 


    till

    \subsubsection{Näringslivsutskottet, ENU}
    Näringslivsutskottet har till uppgift att vara kopplingen mellan näringslivet och Sektionens medlemmar \hl{samt ansvara för Sektionens Alumniverksamhet.} 

    Det åligger utskottet att
    \begin{tightdashlist}
    \item anordna aktiviteter som främjar Sektionens medlemmar inför arbetslivet.
    \item vårda och utveckla samarbeten med företag.
    \item tillgodose Sektionens behov av sponsring.
    \item anordna en arbetsmarknadsmässa för Sektionens medlemmar.
    \item hålla ständig kontakt med övriga sektioners motsvarande utskott.
    \item \hl{arrangera minst ett alumnievenemang under året.}
    \end{tightdashlist} 

    \changenote



\end{attsatser}

\begin{signatures}{1}
    \ist
    \signature{\enuordf}{Näringslivsutskottets Ordförnade}
\end{signatures}

\end{document}
