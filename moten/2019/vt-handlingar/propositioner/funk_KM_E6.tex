\documentclass[../_main/handlingar.tex]{subfiles}

\begin{document}
\proposition{Reglementesändring, postbeskrivningar i källarmästeriet och sexmästeriet}

Efter införingen av barmästare i Sexmästeriet finns fortfarande allt ansvar vad gäller sektionens alkohol hos Cølarna även om de inte i praktiken köper in alkohol och prissätter åt sexmästeriet. Det är inte heller något som styrelsen anser att Cølarna bör göra utan istället bör reglementet ändras så att det inte råder delade meningar om arbetsfördelning i framtiden. Barmästarna ska alltså inte endast stå i baren och servera utan ska även ansvara för inköp och prissättning av alkohol som ska säljas på sittningar, precis som de gör i dagens läge.

Köksmästarnas och preferensmästarens ansvar i reglementet säger i dagsläget att de ska bistå Sexmästaren med att sätta meny och sköta matlagning under sittningar. Detta känns som en väldigt liten beskrivning motsvarande vad de faktiskt gör. Köksmästarna och preferensmästaren sätter meny i princip utan Sexmästaren och de sköter även inhandling innan sittningen. Styrelsen tycker därför att det är viktigt att ändra posternas ansvar i reglementet för att det ska vara tydligare för de som söker posten i framtiden vad det är man faktiskt gör. Sexmästaren finns självklart som hjälp och bollplank fortfarande men bör inte ta en aktiv del i planering av meny då denne redan har mycket att stå i.

\newpage 
Styrelsen yrkar 

\begin{attsatser}
  \att ändra i reglementet 10:2:I ändra under Cøl från:\par
  \begin{emptylist}
    \item Cøl (2)
      \begin{dashlist}
        \item sköta beställning, inköp och inventering av sektionens öl och sprit, 
        \item ansvarar för att prissättning för ovan nämnda drycker följer alkohollagen
      \end{dashlist}
    \end{emptylist}
    
    till 
    
    \begin{emptylist}
    \item Cøl (2)
      \begin{dashlist}
        \item sköta beställning, inköp och inventering av sektionens öl och sprit, \hl{exkluderat det som hanteras av barmästarna,} 
        \item ansvarar för att prissättning för ovan nämnda drycker följer alkohollagen
      \end{dashlist}
    \end{emptylist}
  \changenote


\att i reglementet 10:2:K lägga till under Barmästare följande punkter:
   
   
    

\begin{dashlist}
        \item sköter beställning, inköp och inventering av alkoholdryck för Sexmästeriet,
        \item ansvarar för att prissättning av ovan nämnda drycker följer alkohollagen.
      \end{dashlist}


\att i reglementet 10:2:K ändra under Köksmästare från:\par
  \begin{emptylist}
    \item Köksmästare (2) 
      \begin{dashlist}
        \item sköter matlagningen och delegeringen av uppgifterna i köket under sittningen. bistår sexmästaren vid planering av menyn 
      \end{dashlist}
    \end{emptylist}
    
    till 
    
    \begin{emptylist}
    \item Köksmästare (2)
      \begin{dashlist}
        \item \hl{Ansvarar för planering av meny och inhandling av mat innan sittningen.}
        \item sköter matlagningen och delegeringen av uppgifterna i köket under sittningen. bistår sexmästaren vid planering av menyn
      \end{dashlist}
    \end{emptylist}
  \changenote


    \att i reglementet 10:2:K ändra under Preferensmästare från:\par

    \begin{emptylist}
    \item Preferensmästare (1) 
      \begin{dashlist}
        \item sköter arbetet kring specialkost under sittningen. Bistår sexmästaren vid planering av menyn med avseende på specialkost.
      \end{dashlist}
    \end{emptylist}
    
    till 
    
    \begin{emptylist}
    \item Preferensmästare (1)
      \begin{dashlist}
        \item \hl{Ansvarar för planering av meny och inhandling av specialkost innan sittningen.}
        \item \hl{Sköter tillagningen av specialkost och delegeringen av uppgifterna i köket under sittningen.}
      \end{dashlist}
    \end{emptylist}
  \changenote



\end{attsatser}



\begin{signatures}{1}
    \ist
    \signature{\sexm}{Sexmästare}
\end{signatures}

\end{document}
