\documentclass[../_main/handlingar.tex]{subfiles}

\begin{document}
\proposition{Reglementesändring, postbeskrivning Projektgrupp Teknikfokus}

Då projektgruppen de två senaste åren (2017 och 2018) bestått av 4 funktionärer från E-sektionen men att det i reglementet står angivet att antalet funktionärer till posten “Projektgrupp Teknikfokus” är upp till 3 personer anser vi att detta borde uppdateras. Eftersom Teknikfokus som arbetsmarknadsmässa vuxit sig mycket större är det också rimligt att det i takt med detta behövs fler engagerade funktionärer till projektet. Detta är en speciell post på sektionen då man i posten egentligen arbetar för två sektioner vilket leder till att upplägget kan skilja mycket från år till år gällande mässans storlek samt storlek och konstellation av projektgruppen. För att låta Teknikfokus behålla sin frihet i att gemensamt med D-sektionen år till år bestämma storlek på mässa och projektgrupp anser vi det rimligt att denna post beskrivs som en e.a-post (erforderligt antal).

För att förhindra framtida brott mot reglemente samt att låta Teknikfokus välja sin projektgrupp utan onödig begränsning från reglementet yrkar styrelsen därför

\begin{attsatser}
  \att i reglementet ändra \S10:2:L från:\par
  \begin{emptylist}
    \item Projektgrupp Teknikfokus (3)
      \begin{dashlist}
        \item har till uppgift att bistå Teknikfokusansvarig i sitt arbete.
        \item har mandatperiod mellan 1 juli och 30 juni.
        \item Väljs av styrelsen inför varje läsår på rekommendation av ENU-ordförande och Teknikfokusansvarig. 
      \end{dashlist}
    \end{emptylist}
    
    till 
    
    \begin{emptylist}
    \item Projektgrupp Teknikfokus (\hl{e.a})
      \begin{dashlist}
        \item har till uppgift att bistå Teknikfokusansvarig i sitt arbete.
        \item har mandatperiod mellan 1 juli och 30 juni.
        \item Väljs av styrelsen inför varje läsår på rekommendation av ENU-ordförande och Teknikfokusansvarig. 
      \end{dashlist}
    \end{emptylist}
  \changenote
\end{attsatser}

\begin{signatures}{1}
    \ist
    \signature{\enuordf}{Näringslivsutskottets ordförande}
\end{signatures}

\end{document}
