
\documentclass[../_main/handlingar.tex]{subfiles}

\begin{document}
\motion{Ändring av hur Sektionen väljer NollU-funktionärer}
I nuläget väljs Øverphøs på valmötet medan Cophøs och ØGP väljs vid senare tillfällen.
Cophøsen väljs in av styrelsen på rekommendation av Øverphøset samt avgående
valberedning. ØGP väljs också in av styrelsen fast det ännu senare, efter en rekommendation
från Øverphøs och Cophøs.

Detta är inte ett rättvist tillvägagångssätt för medlemmar som vill engagera sig i större poster
på sektionen då inga andra poster tillsätts på detta vis. För sektionsmedlemmen som vill
engagera sig i NollU utöver posten som Øverphøs innebär detta att:

\begin{dashlist}
\item Sektionsmedlemmen ställer upp för andra poster på valmötet som denne kan falla tillbaka
på om medlemmen inte blir vald till önskad NollU-post.
\item Sektionsmedlemmen väljer att inte kandidera till poster den annars hade velat ha ifall den
inte blir vald till önskad NollU-post för att satsa helt på valprocessen efter valmötet.
\end{dashlist}

För andra utskott och sektionsmedlemmar innebär detta att sektionsmedlemmar som tänkt
söka NollU-funktionär och blir valda till någon post på valmötet ger en orättvis chans mot
övriga om sektionsmedlemmen senare blir vald till NollU och ber om att entledigas
från/bortprioriterar de övriga nya erhållna poster. Detta gör att posten inte fylls rättvist, ger
utskottsordförande och eventuellt sektionen extra arbete när posten ska fyllas på nytt.

För valberedningen innebär även detta extra arbete då det lätt kan bli att folk som vill söka
Cophøs väljer att ställa upp som Øverphøs i hopp om att detta ska förbättra deras chanser.
Detta leder då till att valberedningen kan behöva göra dubbla intervjuer med kandidater vars
ursprungliga mål var en annan NollU-post.

För NollU innebär detta att gruppen tillsätts väldigt sent vilket gör att arbetet blir försenat och
ger inget rum för teambuilding, något som är en viktigt startpunkt för en långtgående
projektgrupp som byggs upp av flera sektionsmedlemmar som inte nödvändigtvis arbetat med
varandra tidigare.

En ändring i den nuvarande processen som hade gynnat alla parter vore om Cophøs och ØGP
väljs som övriga poster under Valmötet. I sin helhet har detta fördelarna att:

\begin{dashlist}
    \item Valberedningen kan genomföra hela sitt arbete under en period under terminen, istället
    för att ha det uppdelat. Samt att mindre tid läggs på dubbla intervjuer.
    \item För sektionen blir det en rättvisare valprocess för både medlemmar och utskott.
    \item För NollU som utskott innebär detta att hela gruppen tillsätts vid ett tillfälle och kan börja
    jobba på gruppdynamik innan arbetet sätter igång samt att rekrytering av ØGP inte blir
    ett internt val för Phøset.
\end{dashlist}
\newpage
Med anledning som ovan yrkar motionärerna
\begin{attsatser}
\att i reglemente under \S10:2:L stryka markerade punkter:

\begin{emptylist}
    \item Co-phøsare (5)
        \begin{dashlist}
          \item Bistår Øverphøset i dennes arbete.
          \item Ett Co-phøs ansvarar för den ekonomiska redovisningen av nollningen.
          \item Ett Co-phøs ansvarar för rekryteringen av phaddrarna.
          \item \hl{Väljs av styrelsen på rekommendation av Øverphøset och avgående valberedning.}
        \end{dashlist}
    \item Övergudphadder (2)
        \begin{dashlist}
          \item Ansvarar tillsammans med ett Co-phøs för phadderverksamheten.
          \item \hl{Väljs av styrelsen på rekommendation av Øverphøset, Co-phøsen och avgående
          valberedning.}
        \end{dashlist}
\end{emptylist}
\changenote
\end{attsatser}
\begin{signatures}{2}
    \textit{Signerat}
    \signature{Edvard Carlsson}{Cophøs 2018}
    \signature{Sonja Kenari}{Cophøs 2018}
\end{signatures}

\end{document}
