\documentclass[../_main/handlingar.tex]{subfiles}

\begin{document}
\motion{Inköp av ljudteknik}

\subsection*{Bakgrund}
    Den nuvarande ljudutrustningen på sektionen är, enligt oss, undermålig. Detta i samband med evenemang såsom spelningar med husbandet, eftersläpp, SåS, eventuella framtida baler, bl.a. 
    
    Konkreta exempel på brister i nuvarande ljudsystemet är bland annat antalet ingångar, och kvaliteten på dessa. Vårt nuvarande mixerbord är begränsat till 2 mikrofoningångar och 4 linjeingångar. Det har inte heller några märkvärdiga ljudverktyg såsom kompressorer eller noise gates. 

    \subsection*{Förslag}
    Vi anser därför att stora delar av denna utrustning bör ersättas såsom mixerbord, delningsfilter, rackskåp och diverse tillbehör. 



    Mixerbordet vi ämnar att införskaffa har 16 XLR/Jack-ingångar, 8 fysiska utgångar samt styrning över ethernet/WiFi med väldigt goda virtuella kopplingsmöjligheter och effekter. 

    Mixerbordet är monterbart i rackskåp och saknar fysiska reglage. Därför vill vi även införskaffa en tablet vilket tillåter trådlöst mixande för ljudteknikern, vilket är extremt praktiskt. 

Därför yrkar vi
    \begin{attsatser}
       \att sektionen köper in ett mixerbord (till en kostnad av ~\SI{4500}{kr}),
       \att sektionen köper in en tablet (till en kostnad av ~\SI{6000}{kr}),
       \att sektionen köper in ett nytt delningsfilter (till en kostnad av ~\SI{3500}{kr}),
       \att sektionen köper in ett nytt rackskåp (till en kostnad av ~\SI{3500}{kr}),
       \att sektionen köper in kringutrustning till yrkanden ovan (till en kostnad av ~\SI{2500}{kr}),
       \att budgeten sätts till \SI{20000}{kr},
       \att kostnaderna belastar utrustningsfonden, samt, 
       \att detta läggs på beslutsuppföljningen till Höstterminsmötet 2019 med undertecknade som ansvariga.
    \end{attsatser}
    


\begin{signatures}{3}
        \textit{I  sektionens tjänst}
        \signature{David Karlsson}{Teknokrat 2019}
        \signature{Emil P. Lundh}{Teknokrat 2019}
        \signature{Moa Rönnlund}{Teknokrat 2019}
    \end{signatures}
\end{document}

