\documentclass[../_main/handlingar.tex]{subfiles}

\begin{document}
\section{Förslag på resultatdisposition av resultatet för 2018}

Sektionen har under verksamhetsåret 2018 fram till skrivandets stund (2019-03-31) fått ett resultat på \SI{126 174,94}{kr} i överskott. Vi borde lägga en större fokus på att använda upp dispositionsfonden! 

Det bör också nämnas att vi bör se över vårt fonderingssystem då vi för tillfället har flera fonder som mer eller mindre aldrig används.

Då större events har tagit plats och redan dragit väldigt mycket likvida tillgångar (t.ex  Vårbalen) och då vår period som innehåller de största ekonomiska utgifterna fortfarande är framför oss så skulle vi i styrelsen vilja ha ett tillägg på \SI{10 000}{kr} för att kunna fortsätta ha en mobilitet och för att kunna främja sektionens verksamhet kontinuerligt under året.

Då vi vill fortsätta delge vårt bidrag till sektionens 60års-jubileum så föreslås det också att vi fortsätter med att fördela in \SI{10 000}{kr} i jubileumsfonden.

Resterande resultat föreslås fortsätta följa vår ekonomiska plan och fonderas in i utrustningsfonden.

Till sist så vill vi också trycka på att vi borde göra en klarare långsiktig ekonomisk plan, om folk har tankar eller ideer om detta så är det bara att kontakta Henrik.

\subsubsection*{Förslag på resultatdisposition}
\begin{tabular}{l r}
    Jubileumsfonden & \SI{10000.00}{kr} \\
    Dispositionsfonden & \SI{10000.00}{kr} \\
    Utrustningsfonden & \SI{106 174,94}{kr} \\
    \hline
    \textbf{Summa} & \SI{126 174,94}{kr} \\
\end{tabular}

\begin{signatures}{2}
    \ist
    \signature{\fvc}{Förvaltningschef 2019}
    \signature{Magnus Lundh}{Förvaltningschef 2018}
\end{signatures}

\end{document}
