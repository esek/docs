\documentclass[../_main/handlingar.tex]{subfiles}

\begin{document}
\section{Ekonomisk rapport}

Den ekonomiska rapporten kan sammanfattas som att vi har gott om pengar. Under de senaste åren har vår tillgångar på framförallt banken exploderat och vi bör se över hur pengarna kan komma till bättre användning. Detta är troligtvist en följd av den högkonjuktur vi ligger i och att vi därmed har ett extremt stort intresse från företag vilket förblir vår största inkomstkälla. Efter detta så bedrivs vår verksamhet bra, däremot bör vi se över alkoholhanteringssystemet och ifall det finns några alternativ till det vi idag använder.


Utöver detta så har även dialoger fört tillsammans med TLTH:s generalsekreterare och hon rekommenderar oss att vi bör tänka igenom det som vi vill spendera pengar på lite mer. 
Därmed så rekommenderar jag oss att spendera långsiktigt, eller åtminstone att vi inte bör spendera pengar för att spendera pengarnas skull utan istället för att vi tror antingen att det kommer göra en skillnad eller för att det är något som kommer hjälpa våra medlemmar under en längre tid.

Jag har sett över budgeten och tills vidare ser det ut som att budgeten från förra året var bra strukturerad och att vi kommer hålla den med några mindre tweeks vilka står i propositionerna!

Vad gäller våra tillgångar (lagret exkluderat), så uppnår dem i skrivandets stund (2019-03-28) till \SI{1 163 771,06}{kr}. 
\begin{dashlist}
	\item Eget kapital, som är de pengar som behövs för daglig verksamhet.
	\item Olycksfonden ska användas vid reparationer och olycksfall.
	\item Dispositionsfonden är pengar som styrelsen kan använda för investeringar och för att förbättra sektionens verksamhet.
	\item Utrustningsfonden ska användas till större investeringar, projekt och inköp.
\end{dashlist}

Se bifogad balansrapport för mer information. 

\begin{signatures}{1}
	\mvh
	\signature{\fvc}{Förvaltningschef}
\end{signatures}

\end{document}
