\documentclass[10pt]{article}
\usepackage[utf8]{inputenc}
\usepackage[swedish]{babel}

\def\mo{Edvard Carlsson}
\def\ms{Mattias Lundström}
\def\ji{Lina Samnegård}
%\def\jii{}

\def\doctype{Protokoll} %ex. Kallelse, Handlingar, Protkoll
\def\mname{Styrelsemöte} %ex. styrelsemöte, Vårterminsmöte
\def\mnum{S19/19} %ex S02/16, E1/15, VT/13
\def\date{2019-11-25} %YYYY-MM-DD
\def\docauthor{\ms}

\usepackage{../e-mote}
\usepackage{../../../e-sek}

\begin{document}
\showsignfoot

\heading{{\doctype} för {\mname} {\mnum}}

%\naun{}{} %närvarane under
%\nati{} %närvarande till och med
%\nafr{} %närvarande från och med
\section*{Närvarande}
\subsection*{Styrelsen}
\begin{narvarolista}
\nv{Ordförande}{Edvard Carlsson}{E16}{}
\nv{Kontaktor}{Mattias Lundström}{E17}{}
\nv{Förvaltningschef}{Henrik Ramström}{E16}{}
\nv{Cafémästare}{Jonathan Benitez}{E17}{}
\nv{Sexmästare}{Theo Nyman}{BME18}{}
\nv{Krögare}{Davida Åström}{BME17}{}
%\nv{Entertainer}{Saga Åslund}{BME18}{}
\nv{SRE-ordförande}{Lina Samnegård}{BME16}{}
\nv{ENU-ordförande}{Jakob Pettersson}{E17}{}
\nv{Øverphøs}{Stephanie Bol}{BME17}{} sent !! Akom p11. 
\end{narvarolista}


\subsection*{Ständigt adjungerande}
\begin{narvarolista}
%\nv{}{}{}{}
%\nv{Skattmästare}{Daniel Bakic}{E15}{\nafr{10}}
%\nv{Vice Krögare}{Klara Indebetou}{BME17}{}
%\nv{Vice Krögare}{Hjalmar Tingberg}{BME16}{}
%\nv{Kårrepresentant}{Ivar Vänglund}{}{}
%\nv{Kårrepresentant}{Martin Bergman}{}{}
%\nv{Valberedningens ordförande}{Axel Voss}{E15}{\nafr{10b}}
%\nv{Fullmäktigeledamot}{Magnus Lundh}{E15}{\nafr{12}}
%\nv{Chefredaktör}{Erik Eriksson}{--}{}
%\nv{Inspektor}{Monica Almqvist}{}{}


\end{narvarolista}

%\begin{comment}
\subsection*{Adjungerande}
\begin{narvarolista}
%\nv{post}{namn}{klass}{nati/nafr/tom}
%\nv{Likabehandlingsombud}{Jonna Fahrman}{BME17}{}
%\nv{Likabehandlingsombud}{Hanna Bengtsson}{BME18}{}
%\nv{Projekfunktionär}{Emma Hjörneby}{BME17}{}
%\nv{Macapär}{Filip Larsson}{E17}{}
%\nv{Kodhackare}{Vincent Palmer}{E18}{}
\end{narvarolista}
%\end{comment}

\section*{Protokoll}
\begin{paragrafer}
\p{1}{OFMÖ}{\bes}
Ordförande {\mo} förklarade mötet öppnat kl 12.12

\p{2}{Val av mötesordförande}{\bes}
{\valavmo}

\p{3}{Val av mötessekreterare}{\bes}
{\valavms}

\p{4}{Val av justeringsperson}{\bes}
{\valavj}

\p{5}{Godkännande av tid och sätt}{\bes}
{\tosg}

\p{6}{Adjungeringar}{\bes}
%Adam Belfrage adjungerades.{}
%Hanna Bengtsson adjungerades. \\
%Jonna Fahrman adjungerades.
%Vincent Palmer adjungerades.\\
%Filip Larsson adjungerades. 

Emil(ständig), Joakim, albin, rasmus, casper
(Martin( ev p 10A) )
Ivar
%\textit{Inga adjungeringar.}


\p{7}{Godkännande av dagordningen}{\bes}

%Davida \ypa lägga till punkten ``Lophtet'' till dagordningen.\\
%Edvard \ypa lägga till punkten ``Ordensband'' til dagordningen.
%Fredrik \ypa att lägga till \S18b ``Teknikfokus utnyttjande av LED-café''.
%Jonathan \ypa ändra punkten §12 från att vara en beslutspunkt till diskussion. \\
%Föredragningslistan godkändes med yrkandet.
%Henrik \ypa lägga till punkten ``Faktura till F'' som §13.
%Jakob Pettersson \ypa tägga till punkten ''Øverphøs informerar'' som \S16.

Föredragningslistan godkändes med yrkandet.

\p{8}{Föregående mötesprotokoll}{\bes}
\latillprotgodkand{S24/19} %\& S15/19}
\textit{\ingaprot}

\p{9}{Fyllnadsval och entledigande av funktionärer}{\bes}
\begin{fyllnadsval} %"Inga fyllnadsval." fylls i automatiskt
%\fval{Moa Rönnlund}{Halvledare}
%\entl{Fanny Månefjord}{Husstyrelserepresentant från och med 30 juni}


\end{fyllnadsval}

\p{10}{Rapporter}{}
\begin{paragrafer}
\subp{A}{Hur mår alla?}{\info}

Mötets hälsa är blandad. Många är trötta efter funktionärstacket och veckan som väntar är lång. 

Steph - Lyckat med phaddertacket, speiellt med tanke på budgeten. De gjorde de bästa av situationen. Annars fortsätter NollU med testamentet. 
%Punkten protokollfördes ej.


\subp{B}{Utskottsrapporter}{\info}

Casper meddelade ..

\subp{C}{Ekonomisk rapport}{\info}
 
Henrik informerade att ekonomin ser bra ut.


\subp{D}{Kåren informerar}{\info}

Martin informerade att det är Speak Up Days, ett längre event angående utbildningspåverkan och studiemiljöevent. 
Martin informerade också att det är F1 Röj helg. Sista FM för året ligger i helsingborg...

Ivar -- Det finns mycket vatten att hämta i Kårhuset sedan Arkad. Tips inför Valmötet. 

\end{paragrafer}

\p{11}{Äskning av pengar till fika för testamentesskrivarkväll}{\bes}
Edvard presenterade äskningen. Förslag på datum är nästa måndag (datum??).

Edvard funderade på att använda våffeljärnet igen, vår fina present från KTH.

\Mbabay

\p{12}{Matlagning till Valmötet}{\dis}

Diskussion ang matlagning till Valmötet.

Maten måste hämtas på tisdag morgon. 

blbababl



\p{13}{Nästa styrelsemöte}{\bes}
\Mba nästa styrelsemöte ska äga rum 2019-12-02 12.10 i E:1123.

\p{14}{Beslutsuppföljning}{\bes}
%Edvard \ypa stryka ''Projektfunktionär: Vårbal'' från Beslutsuppföljning. Liknande projekt uppmuntras.
%\Mbaby
%Davida \ypa skjuta upp ''Inköp av draghandtag till cykelvagn'' till nästa styrelsemöte.
%\Mbaby

Casper \ypa stryka.. Allt har kommit och skall upp i vecan

\Mbaby

Edvard \ypa skjuta det två veckor till S29.. pga fakturor som kommer.. 

\Mbaby

\p{15}{Övrigt}{\dis}
Davida gav ut anmodningar till julgillet.

Theo -- undrade om fasta tillstånd alla dagar istället för bara fredag och lördag.. Diskussion vad det skulle innebära. 
Edvard - poäng att ha det alla dagar in veckan men vi får ta med oss det. Martin informerade att man kan ordna pub i  Cornelis då de har fasta tillstånd hela tidne. 


Theo - vi har massa läsk i BD och Pump. 

Theo - Vi har en under 20 som söker Sexmästare. Ska vi ha en gemensam åsikt(?) Står i kravprofilen men ej i stadgan. Mötet diskuterade detta kort. 
Jakob - det styrlsen kan göra är att informera valmötet men kanske inte bör ha en åsikt. Henrik undrade om VB ska skriva ihop något så att de kan informera -> bättre än att det kmr från styrelsen. Edvard ska prata med VB. 

Davida -- Påminna på VM att man pratar som sektionsmedlem. 

Joakim - ide med nya soppkärl inne i edekvata. Problemet är att vaktmästaren är dålig att höra av sig. Återkommer på nästa möte. 

Edvard informerade att SRE och NöjU har städvecka. 

\p{16}{Sammanfattning av mötet}{\info}
% \Fbs
% \Fbsup


\p{17}{OFMA}{\bes}
{\mo} förklarade mötet avslutat kl. 12.57
\end{paragrafer}

%\newpage
\hidesignfoot
\begin{signatures}{3}
\signature{\mo}{Mötesordförande}
\signature{\ms}{Mötessekreterare}
\signature{\ji}{Justerare}
\end{signatures}
\end{document}
