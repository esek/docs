\documentclass[10pt]{article}
\usepackage[utf8]{inputenc}
\usepackage[swedish]{babel}

\def\mo{Edvard Carlsson}
\def\ms{Mattias Lundström}
\def\ji{Lina Samnegård}
%\def\jii{}

\def\doctype{Protokoll} %ex. Kallelse, Handlingar, Protkoll
\def\mname{Styrelsemöte} %ex. styrelsemöte, Vårterminsmöte
\def\mnum{S27/19} %ex S02/16, E1/15, VT/13
\def\date{2019-11-25} %YYYY-MM-DD
\def\docauthor{\ms}

\usepackage{../e-mote}
\usepackage{../../../e-sek}

\begin{document}
\showsignfoot

\heading{{\doctype} för {\mname} {\mnum}}

%\naun{}{} %närvarane under
%\nati{} %närvarande till och med
%\nafr{} %närvarande från och med
\section*{Närvarande}
\subsection*{Styrelsen}
\begin{narvarolista}
\nv{Ordförande}{Edvard Carlsson}{E16}{}
\nv{Kontaktor}{Mattias Lundström}{E17}{}
\nv{Förvaltningschef}{Henrik Ramström}{E16}{}
\nv{Cafémästare}{Jonathan Benitez}{E17}{}
\nv{Sexmästare}{Theo Nyman}{BME18}{}
\nv{Krögare}{Davida Åström}{BME17}{}
%\nv{Entertainer}{Saga Åslund}{BME18}{}
\nv{SRE-ordförande}{Lina Samnegård}{BME16}{}
\nv{ENU-ordförande}{Jakob Pettersson}{E17}{}
\nv{Øverphøs}{Stephanie Bol}{BME17}{\nafr{11}} 
\end{narvarolista}


\subsection*{Ständigt adjungerande}
\begin{narvarolista}
%\nv{}{}{}{}
%\nv{Skattmästare}{Daniel Bakic}{E15}{\nafr{10}}
%\nv{Vice Krögare}{Klara Indebetou}{BME17}{}
%\nv{Vice Krögare}{Hjalmar Tingberg}{BME16}{}
\nv{Kårrepresentant}{Ivar Vänglund}{}{}
\nv{Kårrepresentant}{Martin Bergman}{}{}%\nafr{10A}
%\nv{Valberedningens ordförande}{Axel Voss}{E15}{\nafr{10b}}
%\nv{Fullmäktigeledamot}{Magnus Lundh}{E15}{\nafr{12}}
\nv{Chefredaktör}{Emil Eriksson}{E18}{}
\nv{Vice Förvaltningschef}{Rasmus Sobel}{BME16}{}
\nv{Vice Entertainer}{Casper Schwerin}{BME18}{}
%\nv{Inspektor}{Monica Almqvist}{}{}

\end{narvarolista}

%\begin{comment}
\subsection*{Adjungerande}
\begin{narvarolista}
%\nv{post}{namn}{klass}{nati/nafr/tom}
%\nv{Likabehandlingsombud}{Jonna Fahrman}{BME17}{}
%\nv{Likabehandlingsombud}{Hanna Bengtsson}{BME18}{}
%\nv{Projekfunktionär}{Emma Hjörneby}{BME17}{}
%\nv{Macapär}{Filip Larsson}{E17}{}
\nv{Husstyrelserepresentant}{Joakim Magnusson Fredluend}{BME19}{}
\nv{Picasso}{Albin Lidbäck}{E18}{}
\end{narvarolista}
%\end{comment}

\section*{Protokoll}
\begin{paragrafer}
\p{1}{OFMÖ}{\bes}
Ordförande {\mo} förklarade mötet öppnat kl 12.12

\p{2}{Val av mötesordförande}{\bes}
{\valavmo}

\p{3}{Val av mötessekreterare}{\bes}
{\valavms}

\p{4}{Val av justeringsperson}{\bes}
{\valavj}

\p{5}{Godkännande av tid och sätt}{\bes}
{\tosg}

\p{6}{Adjungeringar}{\bes}
%Adam Belfrage adjungerades.{}
%Hanna Bengtsson adjungerades. \\
%Jonna Fahrman adjungerades.
%Vincent Palmer adjungerades.\\
%Filip Larsson adjungerades. 
Joakim Magnusson Fredlund adjungerades. 

Albin Lidbäck adjungerades.

Rasmus Sobel adjungerades.

Casper Schwerin adjungerades.

%\textit{Inga adjungeringar.}

\p{7}{Godkännande av dagordningen}{\bes}

%Davida \ypa lägga till punkten ``Lophtet'' till dagordningen.\\
%Edvard \ypa lägga till punkten ``Ordensband'' til dagordningen.
%Fredrik \ypa att lägga till \S18b ``Teknikfokus utnyttjande av LED-café''.
%Jonathan \ypa ändra punkten §12 från att vara en beslutspunkt till diskussion. \\
%Föredragningslistan godkändes med yrkandet.
%Henrik \ypa lägga till punkten ``Faktura till F'' som §13.
%Jakob Pettersson \ypa tägga till punkten ''Øverphøs informerar'' som \S16.

Föredragningslistan godkändes med yrkandet.

\p{8}{Föregående mötesprotokoll}{\bes}
\latillprotgodkand{S24/19} %\& S15/19}
%\textit{\ingaprot}

\p{9}{Fyllnadsval och entledigande av funktionärer}{\bes}
\begin{fyllnadsval} %"Inga fyllnadsval." fylls i automatiskt
%\fval{Moa Rönnlund}{Halvledare}
%\entl{Fanny Månefjord}{Husstyrelserepresentant från och med 30 juni}


\end{fyllnadsval}

\p{10}{Rapporter}{}
\begin{paragrafer}
\subp{A}{Hur mår alla?}{\info}

%Mötets hälsa är blandad. Många är trötta efter funktionärstacket och veckan som väntar med valmötet är lång. 

Punkten protokollfördes ej.

\subp{B}{Utskottsrapporter}{\info}

Edvard meddelade att allt är redo inför Valmötet. Maten ska bara hämtas och tillagas. Utöver det gick Funktionärstacket i lördags bra och igår hade Edvard OK där det pratades överlämning.

Jonathan och CM har bakat lussekatter. De har också äntligen köpt glögg och på lördag är det julavslutning för CM. 
Jonathan har också haft CM möte och diskuterat överlämning. Jonathan och Antonia har även uppdaterat och allmänt fixat IC-rapportsmallen. Utöver det har Jonathan fortsatt med testamentet. 

Henrik och FVU har jobbat på som vanligt och planerat att städa HK och Blå dörren nu på söndag. Det ekonomiska arbete fortsätter som vanligt med fakturor och till sist har Henrik hjälpt Saga med bokföring. 

Mattias och InfU fortsätter som vanligt. Picasso har tagit emot grafikbeställningar och redaktionen har haft möte. 
Mattias har haft möte med Macapärerna om diverse IT-grejer som inte längre fungerar optimalt. Bland annat funkar inte 'Boka Fika Fika' modulen när man inte är inloggad och texten för HeHE's modul behöver uppdateras. Dessutom har de fått i uppgift att förbereda övergången till G Suite med kontoskapande, grupper och mailalias. 
Utöver det har Mattias skrivit klart lite mötesprotokoll som prioriterats ner efter HTM. 

Davida och KM har haft två gillen förra veckan. I tisdags var det alumnigille vilket var lyckat och kul. I fredags hade de även Gammal och Dryggille med KM-17 och KM-18. Det var väldigt uppskattat trots strul med AHS. Utöver det har Davida haft lunchmöte och planerat resten av terminen. Det blir Glöggille och Julgille.

Stephanie meddelade att Phaddertacket var lyckat, speciellt med tanke på den snåla budgeten. Stephanie är stolt och tyckte de gjorde det bästa av situationen. Utöver det fortsätter NollU med sina testamenten. 

Jakob och ENU har inlett veckan med inspirationskväll på Knighttec som var väldigt trevlig. ENU har även påbörjat marknadsföringen till casekväll på Alten och öppnat anmälan, bokat lokal samt beställt mat till lunchföreläsningen med Academic Work på fredag. I tisdags var det alumnipub arrangerat av alumniansvariga och det verkar ha varit uppskattat av pubdeltagarna. 

Casper meddelade att NöjU har haft spelkväll med god närvaro. Sångarstriden är även i full gång med rep, bygg och sång inför nästa helg. 
Entertainer-trion hade möte om NöjUs kick out och de har bestämt att äta middag på VGs. I veckan var det lunchmöte där NöjU-veckan planerades.

Theo och Sexmästeriet hade sittning i måndags. Theo meddelade att den var väldigt lyckad med lite annorlunda upplägg. Sista sittningen för året är spikad till den 10:e december där temat är Nobel. De har även spikat att Lunds Akademiska Vinsällskap kommer på besök. 

Lina och SRE har haft möte med hela utskottet. Lina har också haft kollegiemöte och haft en del studieärenden. Nu granskas CEQ-enkäter och utöver det har Lina har även skrivit ihop lite förbättringsområden inom utskotten. Det handlar främst hur posten för vice och ledamot kan nyttjas bättre.  

\subp{C}{Ekonomisk rapport}{\info}
 
Henrik informerade att ekonomin ser bra ut.

\subp{D}{Kåren informerar}{\info}

Martin informerade att det just nu är Speak Up Days. Det är ett längre event angående utbildningspåverkan och studiemiljö. 

Martin informerade också att det är F1 Röj helg. Sista fullmäktigemötet för året kommer hållas i Helsinborg. 

Ivar informerade att det finns mycket flaskvatten av varierande smak att hämta i Kårhuset sedan Arkad. Det skulle passa bra till sektionens valmöte. 

\end{paragrafer}

\p{11}{Äskning av pengar till fika för testamentesskrivarkväll}{\bes}
Edvard presenterade äskningen. Förslag på datum är måndag den 2:a december. 

Edvard funderade på att använda våffeljärnet som sektionen har fått från KTH. 

\Mbabay

\p{12}{Matlagning till Valmötet}{\dis}

Mötet diskuterade matlagning till Valmötet. Maten måste hämtas på tisdagmorgonen. 

Det bli en favorit i repris med Linas goda linsgryta. 

\p{13}{Nästa styrelsemöte}{\bes}
\Mba nästa styrelsemöte ska äga rum 2019-12-02 12.10 i E:1123.

\p{14}{Beslutsuppföljning}{\bes}
%Edvard \ypa stryka ''Projektfunktionär: Vårbal'' från Beslutsuppföljning. Liknande projekt uppmuntras.
%\Mbaby
%Davida \ypa skjuta upp ''Inköp av draghandtag till cykelvagn'' till nästa styrelsemöte.
%\Mbaby

Casper som representerar Saga under mötet \ypa stryka ''Fotovägg - Diplomat'' från Beslutsuppföljning. Allt som beställts har kommit och skall sättas upp senare i veckan. 

\Mbaby

Edvard \ypa skjuta upp ''Funktionärstacket'' två veckor till S29/19 på grund av fakturor som tar tid. 

\Mbaby

\p{15}{Övrigt}{\dis}
Davida gav ut anmodningar till styrelsen för Julgillet.  

Theo undrade om det är fördelaktigt att ha fast tillstånd alla dagar istället för bara fredag och lördag. Diskussionen bordlades till nästa styrelsemöte. 
%Edvard - poäng att ha det alla dagar in veckan men vi får ta med oss det. Martin informerade att man kan ordna pub i  Cornelis då de har fasta tillstånd hela tidne. 

Theo meddelade att vi har massa läsk i Blå dörren och i Punp. 

Theo undrade om mötets åsikter angående att en under 20 år sökes Sexmästare. Mötet diskuterade detta kort. Edvard tänker höra med Valberedningen. 

%Jakob - det styrlsen kan göra är att informera valmötet men kanske inte bör ha en åsikt. Henrik undrade om VB ska skriva ihop något så att de kan informera -> bättre än att det kmr från styrelsen. Edvard ska prata med VB. 

Davida påminde att man under Valmötet skall presentera sig och framföra sina åsikter som sektionsmedlem. 

Joakim har en idé med nya soppkärl inne i Edekvata men vaktmästaren är dålig på att höra av sig. Joakim återkommer på nästa möte. 

Edvard informerade att SRE och NöjU har städvecka. 

\p{16}{Sammanfattning av mötet}{\info}
% \Fbs
% \Fbsup
\Mba köpa in fika för en testamentesskrivarkväll. 

''Fotovägg - Diplomat'' \textbf{ströks} från Beslutsuppföljning.

''Funktionärstacket'' från Beslutsuppföljning \textbf{sköts upp} till S29/19.



\p{17}{OFMA}{\bes}
{\mo} förklarade mötet avslutat kl. 12.57
\end{paragrafer}

%\newpage
\hidesignfoot
\begin{signatures}{3}
\signature{\mo}{Mötesordförande}
\signature{\ms}{Mötessekreterare}
\signature{\ji}{Justerare}
\end{signatures}
\end{document}
