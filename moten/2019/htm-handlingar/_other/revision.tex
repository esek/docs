\documentclass[../_main/handlingar.tex]{subfiles}

\begin{document}
\section{Revisionsberättelse för E-sektionen 2018}

Undertecknade har, i egenskap av Sektionens revisorer för verksamhetsåret 2018, granskat räkenskaper och förvaltning för E-sektionen inom TLTH för verksamhetsåret 2018. Sektionsverksamheten har följts, styrelse- och sektionsmötesprotokoll har genomgåtts och bokföringen har noga granskats. Granskningen har utförts enligt god revisionssed. Vi vill efter fullgjord revision avge följande berättelse: 

Sektions- och styrelsemöten har under året hållits i enlighet med stadgarna. Detta innebär att möteskallelser gått ut på ett riktigt sätt i rätt tid, att de möten som ska hållas har hållits och att de punkter som ska behandlas på dessa möten har behandlats. Protokoll och handlingar från dessa möten har granskats och uppfyller de krav som Sektionen ställer på sina dokument. Protokollen är tydliga och för den som är något insatt i verksamheten lätta att förstå och följa. 

Sektionen har många styrdokument och utöver Sektionens egna dokument tillkommer även både teknologkårens styrdokument och inte minst den svenska lagen. Vid granskning av verksamheten kan konstateras att styrelsen, samt övriga funktionärer, gjort sitt bästa för att hålla sig uppdaterade med de regler som gäller och följa dessa. Under året har vi uppmärksammat mindre avsteg från Sektionens styrdokument. Detta är något som sker varje år och vi anser inte att någon av dessa avvikelser är i närheten av så allvarliga att de ska påverka styrelsens ansvarsfrihet. Vi vill även, med bakgrund av att tillsynen och kraven de senaste åren successivt trappats upp, skicka med att Sektionen och ansvariga funktionärer fortsatt behöver vara mycket noggranna när det kommer till hanteringen av alkohol. Det finns många lagar, föreskrifter och inte minst Sektionens egna styrdokument som gällande försäljning och hantering av alkoholhaltiga drycker. Dessa regleringar syftar till att säkerställa en säker alkoholservering samt en god dryckeskultur.

Sektionens bokföring är i god ordning. Inga oegentligheter i bokföringen har uppdagats i samband med revisionen. Värt att notera är dock att vissa resultat avviker kraftigt från budgeten. Särskilt gäller detta SEX01, som gått ungefär 45 000 mer plus än budget samt SEK01 där resultatet hamnar ungefär 25 000 under budget. Detta bör man ta med sig inför framtida budgetarbeten.

Vi kan slutligen, efter vår revision, med glädje konstatera att E-sektionen är en mycket välskött förening med välskött ekonomi och god kontroll gällande den löpande verksamheten.

Med ovanstående som bakgrund vill vi rekommendera sektionens medlemmar:
\begin{attsatser}
    \att bevilja funktionärer och styrelse ansvarsfrihet för 2018,
    \att godkänna resultatdispositionen av 2018-års resultat, samt,
    \att fastställa bosklutet för 2018.
\end{attsatser}

\begin{signatures}{2}
    \mvh
    \signature{Fredrik Peterson}{Revisor}
    \signature{Anders Nilsson}{Revisor}
\end{signatures}

\end{document}
