\documentclass[../_main/handlingar.tex]{subfiles}

\begin{document}
\section{Diskussionspunkt: Sektionsrepresentativ klädsel}

För att ta reda på mer om vad medlemmarna tycker om sektionsrepresentativ klädsel
lades en undersökning upp i E-sek events. Drygt 50 personer svarade och resultaten var
som sådana:

En knapp majoritet tycker att vi ska göra en ändring av ordensbandet eller lägga till ett
nytt.(55 \%) Av dessa var det en knapp majoritet som inte tycker att ordensbandet ska bli
bredare.(57,8 \%)

En knapp majoritet vill att det ska vara samma band för kvinnor såsom män. (53 \%)
De som inte vill ha samma band är det rätt lika mellan att ha ett annat band och att ha
kvar rosetten som enda alternativ. (30 svar)

Av de som skrev fritext tyckte många att om kvinnorna ska ha samma band - så ska det
bli bredare. Att ha ett bredare band följer också reglerna kring ordensband för kvinnor.
Några tyckte även att materialet ska ändras, till ett glattare material, såsom till exempel
F:s sektionsband. Några tycker också att det vanliga bandet till män är för tråkigt. De vill
ha ett mjukare/finare tyg. Majoriteten tycker att det skulle vara okej att ändra band och
att det blev dyrare, och nästan alla tycker det vore okej att ändra band för att göra det
snyggare.

Sammanfattningsvis är det relativt splittrat vad sektionen tycker och det krävs en
diskussion för att få fram hur vi ska göra.


\begin{signatures}{2}
    \mvh
    \signature{Matilda Horn}{Sigilbevare}
    \signature{Henrik Ramström}{Sigilbevare Emeritus}
\end{signatures}

\end{document}
