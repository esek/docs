\documentclass[../_main/handlingar.tex]{subfiles}

\begin{document}
\proposition{Stadgaändring, Styrelsens sammansättning och Informationsutskottet}

Efter att styrelsen kontinuerligt utvärderat årets arbete har vi kommit fram till ett omfattande förslag till förändring av sektionens organisatoriska struktur som skulle förbättra och gynna sektionens verksamhet både kortsiktigt i vardagliga processer samt på längre sikt genom möjliggörande och underlättande av strategiskt arbete. 

Sektionens struktur innebär just nu väsentligen följande svårigheter.

\subsubsection{1. Strategiskt arbete och Vice Ordförande} 

Förutom att styrelsen, som med sina roller som utskottsordföranden tillsammans med alla funktionärer driver sektionens dagliga och årliga verksamhet, har styrelsen också som uppgift att arbeta med mer långsiktiga frågor. Eftersom styrelsens ledamöter har utskott att leda och utveckla faller denna uppgift, när sammanhanget är sektionen som helhet, främst på sektionens Ordförande som enda styrelsepost utan eget utskott. Tyvärr motverkas detta då Ordförandeposten också innefattar stora mängder vardagligt operativt arbete. 

På sektionen finns en Vice Ordförande men inte i form av en funktionäspost. Istället utses en av ledamöterna i styrelsen internt till Vice Ordförande. Med enda uppgift från styrdokumenten att ställföreträda då Ordförande inte har möjlighet att fullfölja sitt uppdrag. I praktiken hade detta åtagande varit svårt att genomföra. Givetvis hade en utskottsordförande kunnat leda ett möte eller två, men att vara ställföreträdande över en längre period och samtidigt leda det egna utskottet hade inte varit hållbart. I fallet då Ordförande förblir närvarande har Vice Ordförande således inte mycket av en funktion, med undantag för sektionens representation vid Vice ordförandekollegiet på kåren. Där personen sitter med men i praktiken inte i egenskap av Vice Ordförande då dess obefintliga arbetsuppgifter skiljer sig markant från andra sektioners.

\subsubsection{2. Kontaktor och Informationsutskottet} 

Kontaktor har idag blivit en väldigt bred post. Förutom att agera ordförande och leda Informationsutskottet är arbetsuppgifterna enligt sektionens styrdokument i huvudsak styrelserelaterade. Man är sektionens sekreterare, ansvarig för externa relationer och kommunikationsansvarig, vilket är ansvarsområden som på andra sektioner består av upp till tre styrelseposter. 

Posten Vice Kontaktor infördes därför 2016 för att bistå Kontaktor med just utskottsarbete och ledning av utskottet vilket var ett nödvändigt steg i rätt riktning. Men att få detta att fungera bra i praktiken har varit svårt. En vicepost skall främst bistå utskottsordförande med vardagligt arbete och vara dess högra hand, inte självständigt leda utskottet. 

Informationsutskottet är också, tillsammans med Förvaltningsutskottet, sektionens bredaste utskott som sysslar med all typ av informationshantering. Utskottet skriver HeHE, dokumenterar event, sköter kommunikationskanaler, sysslar med PR- och marknadsföring och inte minst sköter all administration av våra hemsidor, mjukvara och underhåll av all sektionens teknik. Därför är det extra viktigt för en sån stor grupp med helt olika arbetsuppgifter att få en bra sammanhållning och att sektionen får en röd tråd genom all informationshantering. Detta underlättas stort med en aktiv utskottsordförande som helt kan fokusera på att leda och utveckla sektionens informationsverksamhet, och inte begränsas av sin stora roll inom styrelsen.

\newpage

\subsubsection{Styrelsens förslag} 

För att utveckla sektionen genom att möjliggöra mer långsiktigt strategiskt arbete från styrelsen och stärka Informationsutskottet föreslår styrelsen följande förändring.

\begin{itemize}
  \item Kontaktor flyttas till styrelsen där den tar rollen som sekreterare, ansvarig för externa relationer, kommunikationsansvarig samt axlar uppgiften som Vice Ordförande. Detta skapar ett slags presidium. Där Ordförande och Kontaktor sammankallar och förbereder möten samt säkerställer att styrelsen och sektionen även arbetar med långsiktiga strategiska frågor. 
  \item Vice Kontaktor ändras till att vara Ordförande för Informationsutskottet (Informationschef) som främst jobbar med att leda utskottet och förbättra sektionens informationsverksamhet. Samtidigt får Chefredaktören ny befogenhet som vice utskottsordförande och kan därmed vara ställföreträdare på styrelsemöten.
\end{itemize}

Eftersom detta är ett förslag på stadgaändring krävs det att det röstas igenom på två på varandra följande sektionsmöten för att träda i kraft. Vid eventuellt bifall kommer alltså även nästa års styrelse att få utvärdera förslaget och känna ifall det känns lämpligt även för deras styrelse.

Med ovanstående som anledning yrkar styrelsen
\begin{attsatser}
    
    \att i Stadgarna under §8:2 Sammansättning göra följande ändring:

    från

    Styrelsen består av
    \begin{alphlist}
    \item Ordförande,
    \item studierådsordförande, tillika utskottsordförande för Studierådet,
    \item sekreterare, tillika \hl{utskottsordförande för Informationsutskottet},
    \item kassör, tillika utskottsordförande för Förvaltningsutskottet, samt
    \item \hl{sex} ledamöter, tillika utskottsordföranden för Nolleutskottet,
        Nöjesutskottet, Cafémästeriet, Källarmästeriet, Sexmästeriet och
        Näringslivsutskottet.
    \end{alphlist}
    
    Benämning på ledamöterna i Styrelsen enligt c-e fastställs av Reglementet.
    
    \hl{Styrelsen utser inom sig en Vice Ordförande till att ställföreträda Ordföranden vid situationer då Ordföranden inte längre har möjligheten att fullfölja sitt uppdrag. Vice Ordförande har också till uppgift att agera styrelsemötesordförande då Ordföranden inte kan närvara.}
    
    till

    Styrelsen består av
    \begin{alphlist}
    \item Ordförande,
    \item studierådsordförande, tillika utskottsordförande för Studierådet,
    \item sekreterare \hl{och vice ordförande}, tillika \hl{Kontaktor},
    \item kassör, tillika utskottsordförande för Förvaltningsutskottet, samt
    \item \hl{sju} ledamöter, tillika utskottsordföranden för Nolleutskottet,
        Nöjesutskottet, Cafémästeriet, Källarmästeriet, Sexmästeriet\hl{, Informationsutskottet} och
        Näringslivsutskottet.
    \end{alphlist}
    
    Benämning på ledamöterna i Styrelsen enligt c-e fastställs av Reglementet.
    
    \changenote
    \newpage        

    \att under förutsättning att den föreslagna förändringen i attsatsen ovan går igenom i andra läsningen även göra följande ändringar i Reglementet:
    
    
    \begin{description}[font=$\bullet$~\normalfont\scshape\color{black!50!black}]
    \item under §8:A Sammansättning
   

    från

    Styrelsen består av
    \begin{alphlist}
    \item Ordföranden -- ansvarar för Styrelsearbetet,
    \item SRE-Ordföranden -- utskottschef för studierådet,
    \item Kontaktorn -- sekreterare\hl{, utskottschef för informationsutskottet},
    \item Förvaltningschefen -- kassör, utskottschef för förvaltningsutskottet,
    \item Øverphøs -- utskottschef för nolleutskottet,
    \item Entertainern -- utskottschef för nöjesutskottet,
    \item Cafémästaren -- utskottschef för cafémästeriet,
    \item Krögaren -- utskottschef för källarmästeriet,
    \item Sexmästaren -- utskottschef för sexmästeriet, \hl{samt}
    \item Ordförande för näringslivsutskottet, ENU
        -- utskottschef för näringslivsutskottet.
    \end{alphlist}

    till

    Styrelsen består av
    \begin{alphlist}
    \item Ordföranden -- ansvarar för Styrelsearbetet,
    \item SRE-Ordföranden -- utskottschef för studierådet,
    \item Kontaktorn -- sekreterare \hl{och vice ordförande},
    \item Förvaltningschefen -- kassör, utskottschef för förvaltningsutskottet,
    \item Øverphøs -- utskottschef för nolleutskottet,
    \item Entertainern -- utskottschef för nöjesutskottet,
    \item Cafémästaren -- utskottschef för cafémästeriet,
    \item Krögaren -- utskottschef för källarmästeriet,
    \item Sexmästaren -- utskottschef för sexmästeriet, 
    \item Ordförande för näringslivsutskottet, ENU -- utskottschef för näringslivsutskottet\hl{, samt}
    \item \hl{Informationschefen, utskottschef för informationsutskottet.}
    \end{alphlist}

    \newpage
    
    
    \item under §10:2:F Funktionärerna i Informationsutskottet, InfU 


    från 

    \hl{Kontaktor} (u)
        \begin{dashlist}
            \item \hl{är Sektionens sekreterare och har övergripande ansvar för Sektionens dokument och protokollföring av möten.}
            \item \hl{ansvarar för att Sektionens stydokument hålls aktuella.}
            \item \hl{ansvarar för att upprätta handlingar till Sektionsmötena.}
            \item har det övergripande ansvaret för Sektionens informationsspridning och PR-verksamhet.
            \item \hl{ansvarar för Sektionens kontakt med andra sektioner, såväl inom TLTH som på andra högskolor och universitet.}
            \item är ansvarig utgivare för HeHE.
        \end{dashlist}

    till

    \hl{Informationschef} (u)
        \begin{dashlist}
            \item har det övergripande ansvaret för Sektionens informationsspridning och PR-verksamhet.
            \item är ansvarig utgivare för HeHE.
        \end{dashlist}
    
  
    \item under §10:2:F Funktionärerna i Informationsutskottet, InfU 
 

    från

    Chefredaktör (1)
    		\begin{dashlist}
    		\item Har det övergripande ansvaret för HeHE och vad därmed äga sammanhang.
            \item Ansvarar tillsammans med Picasso, Redaktörer och NollU för produktion av nollEguiden.
    		\end{dashlist}

    till

    Chefredaktör (1)
    		\begin{dashlist}
            \item \hl{Denna post är en vice till utskottsordföranden.}
    		\item Har det övergripande ansvaret för HeHE och vad därmed äga sammanhang.
            \item Ansvarar tillsammans med Picasso, Redaktörer och NollU för produktion av nollEguiden.
    		\end{dashlist}
    
    
    \item  under §10:2:F Funktionärerna i Informationsutskottet, InfU  stryka  ``Vice Kontaktor (1)'' inklusive dennes efterföljande uppgifter
    \end{description}
    
    \changenote
    
    \newpage
   
    \att även detta under ovan nämnda förutsättnig i Reglementet under §10:2 Funktionärsbeskrivningar lägga till följande och därefter ordna efterföljande numrering
   
    
    10:2:D Kontaktor

    Det åligger Kontakorn

        \begin{attlist}
            \item som Vice Ordförande vara ställföreträdare till Ordföranden vid situationer då denne inte har möjligheten att fullfölja sitt uppdrag.
            \item arbeta tillsmammans med Ordförande både med vardagliga åtaganden samt säkerställadet av styrelsens och Sektionens kontinuitet och långsiktighet.
            \item vara Sektionens sekreterare och har övergripande ansvar för Sektionens dokument och protokollföring av möten. 
            \item ansvara för att Sektionens styrdokument hålls aktuella. 
            \item ansvara för att upprätta handlingar till Sektionsmötena.
            \item ansvara för Sektionens kontakt med andra sektioner, såväl inom TLTH som på andra högskolor och universitet.
            \item ansvara för att det hålls en omsitts för den egna posten.
        \end{attlist}

  
    
\end{attsatser}
 
 


\begin{signatures}{2}
    \ist
    \signature{\ordf}{Ordförande}
    \signature{\sekr}{Kontaktor}

\end{signatures}

\end{document}
