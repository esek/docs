\documentclass[../_main/handlingar.tex]{subfiles}

\begin{document}
\verksplanuppf{VT 2019}

\subsubsection*{Styrelsen}

Mycket av styrelsens arbete under 2019 har hittills handlat om att komma igång med verksamheten, mest aktivt arbete har lagts på det generella målet om medlemmarnas välmående istället för på delmålen. Vi vill att det ska vara kul att engagera sig och har uppmanat till mer teambuilding inom utskotten genom till exempel mer pengar till kickoffs. Styrelsen har även diskuterat medlemmarnas trygghet och säkerställandet av denna, vi strävar efter att alla ska känna sig välkomna på sektionen och kunna trivas på våra evenemang såväl som med vår dagliga verksamhet. 

Med detta sagt är inte delmålen orörda men ofta sker arbetet mot dessa mer indirekt. Gällande cafémästeriets nu helt ideella verksamhet försöker vi ha en tydlig dialog mellan utskottet och styrelsen för att få en bra bild av caféet och kunna lösa eventuella problem så smidigt som möjligt. Kostnader och intäkter kontrolleras kontinuerligt främst av förvaltningsutskottet, styrelsen har utvärderat budgetar och postbeskrivningar samt diskuterat möjliga förändringar. 

Som tidigare nämnt i utskottsrapporten har vi besökt flera andra lärosäten, vi tror och hoppas att kontakterna vi knutit där kommer leda till roliga och givande samarbeten i framtiden. Sen tänkte jag ta tillfället i akt och upplysa om att man väldigt gärna får söka till projektfunktionär om det finns något särskilt man skulle vilja driva på sektionen, man kan även äska pengar. Det var på detta sätt projektgruppen för vårbalen arbetade. Ansöka eller äska kan man göra på ett av våra veckoliga styrelsemöten. 


\subsubsection*{Informationsutskottet}

Informationsutskottet är ett väldigt spritt utskott. Det har varit en utmaning att få utskottet att komma igång på alla olika håll. Därför har det genomförts en utvärdering för de olika posterna i utskottet för att se hur vi kan ta nya tag framöver. 
Att kunna delegera mer till Vice Kontaktor, som är en relativt ny post, är en också en annan utmaning. Därför har det undersökts om det för framtida informationsutskott inte ska göras en ändring i struktur för hela utskottet. 
Alla våra informationskanaler har utvärderats och ordentliga PR riktlinjer är under bearbetning. Informationsdelen av utskottet har verkligen kommit till liv igen, vilket är jättekul! Teknokraterna har även jobbat på hårt med att utvärdera Sektionens tekniska utrustning och sett över potentiella förbättringar. 
Allmänt har InfU försökt jobba efter verksamhetsplanen. Men för att utskottet i framtiden ska vara hållbart utan för stor belastning inom vissa poster på utskottet så kommer det siktas på att strukturera upp olika delar inom utskottet samt se över överlämnings processen för posterna i utskottet.

\newpage

\subsubsection*{Källarmästeriet}

Källarmästeriet har jobbat med att nå ut till fler teknologer genom affischer, facebookevenemang via sektionens sida och publicering i kårnytt och HeHe. Vi har under årets första läsperiod haft mycket god uppslutning och vi ser även att vi har haft en relativt stor del gäster från andra sektioner. VI ser att många verkar ha börjat se gillet som det naturliga stället att gå efter Fredmans vilket är jättekul! 

Samarbetet både inom och utanför sektionen har fungerat bra. Vi har haft gille för teknikfokus och NöjU har haft både biljardmatcher och På styret på gillen vilket har varit mycket uppskattat. Sammarbete med andra sektioner har skett via Sexmästarkollegiet där vi deltagit i en pubrunda i samband med ET-slasque. 

För att hålla konkurrenskraftigt pris på mat och dryck men samtidigt nå de budgetmål som satts för utskottet har ekonomin följts upp noggrant. Kalkyler av kostnader för klägg har utvärderats och vi har bestämt oss för att behålla prisnivåerna från föregående år och istället hantera det underskott som KM stått inför med aktivt arbete mot svinn, mycket PR och kontinuerlig ekonomisk uppföljning. Hittills går utskottets ekonomi bra men vi har ändå valt att lägga en motion för att ge oss mer ekonomiskt utrymme under resten av året.  

För att minimera det svinn som sker i alkohollagret har utbildning i alkoholhanteringssystemet hållits både inom utskottet och för sexmästeriet. Då vårt alholholhanteringssystem lämnar en del att önska när det kommer till användarvänlighet så har utredning av iZettles lagerhanteringssystem, vilket Data använder idag, inletts. Vi har sett över den utgångna alkohol som finns och kasserat denna vilket kommer leda till att året belastas med mycket svinn, som egentligen skulle belastat tidigare år. 


\subsubsection*{Nolleutskottet}

Under året har NollU arbetat med mottagningsschemat för att de nyantagna studenterna på E-sektionen ska få ett så bra mottagande som möjligt. Det arbetas för att få in fler alkoholfria events för att säkerställa en mångfald av aktiviteter. I början av året skickades enkäter till förra årets kontak, intis och uppdragsphaddrar för att få reda på vad som var bra samt kan utvecklas. I år har världsmästarna visat intresse för samarbete vilket ska hjälpa till i processen för att integrera internationella ännu mer i Sektionen. Som tidigare år finns ett samarbete med SRE under nollningsveckorna för att förmedla en positiv attityd till studierna. Hur uppdragen presenteras för nollorna ska även utvärderas och ändras i år för att ge dem en bättre förståelse över innebörden och på så vis öka tagget.

I år har ØGP visat stor initiativ vilket har underlättat för resterande phøs arbetsbelastningsmässigt. För att underlätta arbetet för utskottets funktionärer ytterligare har även samarbeten med utskotten påbörjats. Ett exempel är InfU och Nolleguiden då det till stor del gjordes internt i NollU förra året. Inom den närmsta framtiden ska det utvärderas hur sektionens utskott ska kunna synas på bästa sätt under nollningen.

\newpage

\subsubsection*{Cafémästeriet}

Vi har tittat igenom arbetsbeskrivningarna och jag tycker personligen att de just nu är rätt så bra nu så vi har lagt vår energi på andra saker i stället.

Vi i Cm har utvärderat priset på det mesta och det är svårt att konkurrera mer än vad vi gör just nu, i och med att vi har några fåtals öre vinst på det mesta. Vi har även gjort en större kalkyl på varor som vi producerar för att få en ungefärlig pris på detta.

Inköparna har gjort ett bra jobb för att minska deras arbetsbelastning. De har till exempel fixat ett schema så att varje inköpare får ta lika många pass. För oss Vice- och Cafémästare så har det inte varit lika enkelt, men vi letar efter halvledare så att det ska bli enklare för oss.

Vi har sorterat om lagret i LED så att allting ska bli enklare och lite mer logiskt. Samt så har detta gjort att det finns mer plats nu för nya saker och allting blir enklare att sortera. Vi har även börjat använda lådan till ryggsäckarna för att hålla det renare där. För LEDs förråd så har vi gjort flera men färre beställningar av läsk för att förbättra hygienen där, som förra året.


\subsubsection*{Förvaltningsutskottet}

Vad gäller lokaler, hustomtar och vice förvaltningschef har det gått väldigt bra. Vi har framförallt jobbat med lokalerna “Sicrit” och “Blå dörren” samt försökt att rensa allmän skit i Edekvata. Utöver detta så har vi även beställt in en hel del utrustning som har känts nödvändig… typ hammare!
När det kommer till bokföringen har jag gått igenom den men har inte kommit på någon streamlining men kommer att göra en exempelpärm nu under våren till fler olika bokföringsordrar!
Vad gäller den sista punkten har en utbildning med utskottsordföranden genomförts samt så har flera mindre ekonomiska utbildningar med andra som också ligger i ekonomisk ansvar.


\subsubsection*{Studierådet}

Studierådet ska verka för att vara ett synligt utskott. Detta görs genom att försöka synas mycket på nollningen, genom workshop i första läsveckan samt flera pluggkvällar. Vi tycker det är viktigt att nyantagna studenter blir medvetna om att utskottet finns och vad vi ska arbeta för. För att öka synligheten för vissa specifika poster som är viktigt att sektionens medlemmar vet om trycker vi posters. Studierådet ska även verka för att driva studierelaterade evenemang kontinuerligt under året, och vi har i nuläget planer på en specialiseringskväll tillsammans med ENU samt Latex-föreläsning. Vi ska även verka för att ha representanter från alla årskurser. I dagsläget har vi bara från årskurs 1-3, men inte för alla år och program i årskurs 4 och 5. 

\newpage

\subsubsection*{Sexmästeriet}


Sexmästeriet har försökt planera in så många sektionsöppna evenemang som möjligt under våren. Vad kommer till prisvärdhet och kvalité har vi inte haft många evenemang där vi styrt budgetering men vi har det i åtanke inför kommande evenemang. Under planering av evenemang satsar vi på att inte gå för mycket plus, vi försöker räkna med alla intäkter och utgifter så att det blir så prisvärt som möjligt. 

Det har planerats in en hel del evenemang under våren, under februari och mars har det varit mycket skiphten och uppstart för utskottet. Vi ser fram emot att hålla sittningar och få ge många roliga stunder till sektionen.

Vi jobbar för att fortsatt hålla en god ordning i pump genom att slänga onödigheter och förvara saker på ett smidigt sätt.

Sexmästeriet har fått utbildning av KM om hur AHS:en fungerar och vad som är viktigt att tänka på. Vi jobbar för att vara noggranna och systematiska när vi rapporterar i AHS:en för att det inte ska bli fel. 



\subsubsection*{Nöjesutskottet}

Nöjesutskottets uppgift är att tillgodose sektionens medlemmar med nöjes-, fritids- och idrottsarrengemang, bland annat genom att funktionärerna i utskottet håller i regelbundna evenemang såsom spelkvällar och Sporta med E. Det hölls i en filmkväll tidigare i vår som kan komma bli mer regelbundna, men i så fall blir det nog till hösten. 

Att inkludera alla sektionens medlemmar genom olika sorters evenemang är väldigt viktigt och något vi alltid strävar efter att jobba aktivt för. Det har bland annat gjorts genom boxning med Sporta med E, drEamHacke som blir av i höst och för våra spexiga, sjungande eller dekorerande medlemmar har Stridsropen gjort ett jättelyft med Sångarstriden som man redan nu kan engagera sig i. 

Diskussion i AktU-kollegiet har förts om hur vi ska kunna genomföra fler intersektionella evenmang utöver nollningen. Då vi redan har event med framför allt D- och F-sektionen har en dialog påbörjats med andra sektioner. Om det blir genomförbart under denna fullspäckade vår återstår att se, men det är vår förhoppning. 

\subsubsection*{Näringslivsutskottet}

ENU har strukturerats om i år där näringslivskontakterna kontinuerligt ska söka kontakt med företag de finner av intresse. Detta ska förhoppningsvis leda till att vi söker kontakt med en bred grupp företag och således nå fler BME-företag och nya företag. Utskottet har kontaktat de företag som vi haft bäst kontakt med förra året för att bibehålla ett bra samarbete med dem. I början av året sågs sektionens priser över och korrigerades efter övriga sektioners priser och vad som ansågs rimligt att ändra på. Alumniverksamheten har integrerats i utskottet och har en övergripande plan över vad som man vill genomföra under året.


\newpage
\begin{signatures}{10}
    \mvh
    \signature{\ordf}{Ordförande 2019}
    \signature{Sonja Kenari}{Kontaktor 2019 VT}
    \signature{\fvc}{Förvaltningschef 2019}
    \signature{\cafem}{Cafémästare 2019}
    \signature{\oph}{Øverphøs 2019}
    \signature{\sreordf}{SRE-Ordförande 2019}
    \signature{\enuordf}{Näringslivsutskottets Ordförande 2019}
    \signature{\sexm}{Sexmästare 2019}
    \signature{\krog}{Krögare 2019}
    \signature{\ent}{Entertainer 2019}
\end{signatures}

\end{document}
