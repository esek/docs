\documentclass[../_main/handlingar.tex]{subfiles}

\begin{document}
\beslutsuppfoljning{Renovering av Biljard}

    Jag fick i uppdrag att renovera biljard under sommaren då det långsamt hade förfallit till ett ställe att förvara mer och mer skräp i. Under vårterminsmötet yrkade jag på:

\begin{attsatser}
    \att slänga ut allt skräp ur biljard,
    \att ersätta sofforna med tre barstolar av modell bianca och 6 barstolar av modell atlantic,
    \att byta ut det nuvarande biljardbordet mot ett av modell Biljardai Club,
    \att ersätta den nuvarande darttavlan med en av modell unicorn eclipse pro,
    \att byta ut den nuvarande ickefungerande TV-skärmen mot en valfri fungerande sådan,
    \att väggarna målas om i samma färger,
    \att budgeten sätts till \SI{35000}{kr},
    \att pengarna tas ifrån utrustningsfonden,
    \att renoveringen sker i sommar med undertecknad som ansvarig.
\end{attsatser} 

Renoveringen gick väl ganska långt ifrån planerat då jag hade problem med att företagen
som lovat leverans den 13/6 levererade istället den 14/6, 22/6 respektive 27/6. Men efter
mycket om och men lyckades vi till slut leverera ett fint biljard till våra medlemmar.

\begin{itemize}
  \item Darttavlan blev utbytt av en i kvalitet som ska hålla i åtminstone två år.
  \item TV.:n som inte fungerade är nu utbytt mot en något mindre sådan men dock fungerande. Den har kopplats till gillemode via med InfUs hjälp och man kan nu se när ens klägg är färdiga hela vägen från biljard.
  \item Sofforna, bordet och övrigt skräp slängdes ut och ersattes med tre bord och sex barstolar. Dock blev borden av en annan modell än de beskrivna ovan då företaget inte kunde lova mig en leveranstid innan den 25/8.
\item Det gamla biljardbordet skänktes bort och den nya installerades. Det blev till sist bra då en häftpistol införskaffades. Monteringen var annars en kamp med att försöka förstå den litauiska tolkningen.
\item Väggarna målades om i något andra färger då min nyansuppskattning uppenbarligen är bristfällig så den vinröda väggen blev något ljusare. Men det var något vi alla tyckte mestadels var en förbättring. I övrigt förblev den vita väggen vit och den svarta väggen svart.
\item Utöver renoveringen som var planlagd sattes även en griffertavla upp. Det gjordes för att man skulle kunna skriva resultat, kölistor eller bara kul budskap på den. Sedan målades även en tavla på en biljardspelande Hacke av tre taggade sektionsmedlemmar.
\item I det stora hela så blev renoveringen väldigt bra och jag hoppas att det dröjer tills nästa ska behövas. Möjligtvis att man skulle fundera över att byta ut det blå skåpet mot ett motsvarande.
\end{itemize}

Ekonomiskt slutade det hela (inklusive tackphest, jobbarmat och transportkostnader) på
\SI{35900}{kr} vilket är ca \SI{900}{kr} över budget. Jag hade glömt att räkna med jobbarmaten i budgeten
vilket gjorde att jag landade strax över budget, utöver det fick jag köpa nya rollers,
maskeringstejp och andra mindre ickeförutsedda utgifter som bidrog till
budgetöverstigningen. Med tanke på resultatet och den lilla budgetöverstigningen yrkar jag
på

\begin{attsatser}
    \att stryka \emph{Router och Switchar för DreamHackE} från beslutsuppföljningen
\end{attsatser}

\begin{signatures}{}
    \signature{Adam Belfrage}{}
\end{signatures}

\end{document}
