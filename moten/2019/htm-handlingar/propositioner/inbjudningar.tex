\documentclass[../_main/handlingar.tex]{subfiles}

\begin{document}
\proposition{Uppdatering av policy: Inbjudningar och anmodningar}

Under året har policyn angående inbjudningar och anmodningar varit svårtolkad och i vissa fall orimlig att följa. Till exempel så ska Inspektorn inbjudas till varenda slasque-sittning och phøsarE står med under både inbjudna och anmodade till NollEgasquen. För att underlätta inför nästkommande år och så att policyn faktiskt går att följa har vi skrivit om den en hel del.

De ändringar som har gjorts är framförallt följande tre:

För det första så har det ändrats så att Inspektor och hedersmedlemmar endast bör inbjudas/anmodas vid finare tillställningar. Exempel på detta är Teknikfokus-banquetten. Det har även förtydligats att personer som jobbar sittandes ska bjudas in. Under nollningsevenemang är det också förtydligat att ordinarie inbjudningar och anmodningar gäller. 

Den andra stora ändringen är inbjudningar och anmodningar till NollEgasquen. I den gamla versionen var det tolkningsfråga på vem som faktiskt skulle anmodas och bjudas in. Det finns också vissa riktlinjer hos kåren som inte stod med hos oss. Dessa är nu med för att göra det mer tydligt och utan överraskningar. Inbjudningarna har också stramats ner en del. Anledningen till att dra ner inbjudningarna och istället anmoda är för att det ska vara mer jämställt mellan funktionärer på sektionen. De som i den nya policyn är inbjudna, är det för att de representerar sektionen och till viss del arbetar sittningen genom att hålla tal eller liknande.

Till sist har det även lagts till en text som beskriver hur Sexmästaren ska förhålla sig till policyn. Om policyn känns orimlig inför ett visst evenemang ska sexmästaren i samråd med styrelsen avgöra huruvida policyn ska följas. Det står också med att sexmästaren bestämmer över vilka anmodningar och inbjudningar som innebär att man får ta med respektive. Anledning till detta är att det kan skilja sig från evenemang till evenemang hur många platser som finns och då även för vem det är rimligt att få ta med respektive. 

Med ovanstående motivering yrkar styrelsen på

\begin{attsatser}
    \att uppdatera \textit{Policybeslut: Inbjudningar och anmodningar} till den uppdaterade bifogade varianten.
\end{attsatser}

\begin{signatures}{1}
    \ist
    \signature{\sexm}{Sexmästare}
\end{signatures}

\newpage

{\Large \textbf{Policybeslut: Inbjudningar och anmodningar}}

{\large \textbf{1. Sittningar}}

Följande personer skall inbjudas
\begin{dashlist}
    \item Ordföranden
    \item Inspektorn (Vid finare tillställningar) 
    \item Gäster som jobbar på sittningen 
    \item Person som skall mottaga Krusidull-E
    
\end{dashlist}
Följande personer skall anmodas
\begin{dashlist}
    \item Styrelsemedlemmar
    \item Valberedningens ordförande
    \item Hedersmedlemmar (Vid finare tillställningar) 
\end{dashlist}

{\large \textbf{1.1 Sittningar under nollningen, exklusive Nollegasque}}

Utöver de vanliga inbjudningarna skall även följande personer inbjudas, 
\begin{dashlist}
    \item Phøs
\end{dashlist}
Utöver de vanliga anmodningarna skall även följande personer anmodas. 
\begin{dashlist}
    \item ØGP
\end{dashlist}

{\large \textbf{1.2 Nollegasque}}

Följande personer skall inbjudas
    \begin{dashlist}
        \item Ordföranden
        \item Øverphøs
        \item Inspektorn
        \item Hedersmedlemmar
        \item Person som skall mottaga Krusidull-E
        \item H.M. Konung Carl XVI Gustaf
    \end{dashlist}

Följande personer skall anmodas i den turordning som följer och i mån av plats

\begin{dashlist}
    \item Styrelsemedlemmar 
    \item Co-phøs och ØGP
    \item Nyantagna studenter
    \item Valberedningens ordförande
    \item HeHE:s chefredaktör
    \item Sektionens kontaktpersoner från Teknologkåren, samt heltidare och styrelseledamöter från E-sektionen
    \item Gamla sektionsordförande som fortfarande är studerande vid skolan eller deltog i förra årets arrangemang
    \item Representanter från LTH-externa vänsektioner
    \item Nollegeneral samt Nolleamiral
    \item Nollningsfunktionärer
    \item Övriga medlemmar vid E-sektionen
\end{dashlist}

{\large \textbf{1.3 Skiphtesgasque}}

Följande personer skall inbjudas
\begin{dashlist}
    \item Hedersmedlemmar
    \item Funktionärer, avgående och pågående
\end{dashlist}

{\large \textbf{2. Övrigt}}

\begin{dashlist}
    \item Person som skall mottaga medalj skall inbjudas alternativt anmodas
    \item Sigillbevararen inbjudes alternativt anmodas när denne skall dela ut medaljer
    \item Gamla sektionsordföranden som var med på föregående Svolder eller motsvarande arrangemang skall anmodas
    \item Övriga kan inbjudas/anmodas beroende på tillställning
\end{dashlist}

Denna policy ska följas i den mån Sexmästaren finner det lämpligt. Vid tillställningar då policyn kring anmodningar och inbjudningar inte är lämplig bör Sexmästaren vid ett styrelsemöte ta upp frågan och styrelsen besluta om att avvika från policyn. 
Sexmästaren beslutar även i vilken utsträckning som personer anmodas med respektive.

\vspace{20px}
{\large \textbf{Kommentar (ej del av policybeslut):}}
\begin{dashlist}
    \item Officiell representation inom LU betalas fullt ut av Sektionen
    (Styrelsebeslut 1993-05-05)
    \item Vad gäller icke specificerade inbjudningar och anmodningar så går
    frågan enligt turordningen
    \begin{numplist}
    \item Ordföranden
    \item Styrelsen
    \item Övriga medlemmar
    \end{numplist}
    \item Om flera personer på samma nivå är intresserade så delas inbjudan lika
    (Styrelsebeslut S3/93)
\end{dashlist}

\newpage

\end{document}
