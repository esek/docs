\documentclass[10pt]{article}
\usepackage[utf8]{inputenc}
\usepackage[swedish]{babel}

\def\doctype{Kallelse} %ex. Kallelse, Handlingar, Protkoll
\def\mname{styrelsemöte} %ex. styrelsemöte, Vårterminsmöte
\def\mnum{S11/19} %ex S02/16, E1/15, VT/13
\def\date{2019-04-15} %YYYY-MM-DD
\def\docauthor{Edvard Carlsson}

\def\mtime{12:10}
\def\place{E:1123}

\usepackage{../e-mote}
\usepackage{../../../e-sek}

\begin{document}

\heading{{\doctype} till {\mname} {\mnum}}

\section*{Tid och plats}
\tidplats

\section*{Föredragningslista}
\begin{paralist}
    \pli{OFMÖ}{\bes}
    \pli{Val av mötesordförande}{\bes}
    \pli{Val av mötessekreterare}{\bes}
    \pli{Val av justeringsperson}{\bes}
    \pli{Godkännande av tid och sätt}{\bes}
    \pli{Adjungeringar}{\bes}
    \pli{Godkännande av dagordningen}{\bes}
    \pli{Föregående mötesprotokoll}{\bes}
    \pli{Fyllnadsval och entledigande av funktionärer}{\bes}
    \pli{Rapporter}{}
    \begin{paralist}
        \pli{Hur mår alla?}{\info}
        \pli{Utskottsrapporter}{\info}
        \pli{Ekonomisk rapport}{\info}
        \pli{Kåren informerar}{\info}
    \end{paralist}
    \pli{Styrelsen i NollEguiden}{\dis}
    \pli{Enkät från likabehandlingsombud}{\dis}
    \pli{Do's and don'ts till HTM utifrån VTM}{\dis}
    \pli{Nästa styrelsemöte}{\bes}
    \pli{Beslutsuppföljning}{\bes}
    \pli{Övrigt}{\dis}
    \pli{OFMA}{\bes}
  \end{paralist}

\newpage

\section*{Beslutsuppföljning}

\emph{Beslutsuppföljningen är tom!}
\begin{signatures}{1}
    \emph{I styrelsens tjänst}
    \signature{Edvard Carlsson}{Ordförande}
\end{signatures}

\end{document}
