\documentclass[10pt]{article}
\usepackage[utf8]{inputenc}
\usepackage[swedish]{babel}

\def\mo{Edvard Carlsson}
\def\ms{Mattias Lundström}
\def\ji{Lina Samnegård}
%\def\jii{}

\def\doctype{Protokoll} %ex. Kallelse, Handlingar, Protkoll
\def\mname{Styrelsemöte} %ex. styrelsemöte, Vårterminsmöte
\def\mnum{S20/19} %ex S02/16, E1/15, VT/13
\def\date{2019-09-24} %YYYY-MM-DD
\def\docauthor{\ms}

\usepackage{../e-mote}
\usepackage{../../../e-sek}

\begin{document}
\showsignfoot

\heading{{\doctype} för {\mname} {\mnum}}

%\naun{}{} %närvarane under
%\nati{} %närvarande till och med
%\nafr{} %närvarande från och med
\section*{Närvarande}
\subsection*{Styrelsen}
\begin{narvarolista}
\nv{Ordförande}{Edvard Carlsson}{E16}{}
\nv{Kontaktor}{Mattias Lundström}{E17}{}
\nv{Förvaltningschef}{Henrik Ramström}{E16}{}
\nv{Cafémästare}{Jonathan Benitez}{E17}{}
\nv{Sexmästare}{Theo Nyman}{BME18}{\nafr{9}}
%\nv{Krögare}{Davida Åström}{BME17}{}
\nv{Entertainer}{Saga Åslund}{BME18}{}
\nv{SRE-ordförande}{Lina Samnegård}{BME16}{}
\nv{ENU-ordförande}{Jakob Pettersson}{E17}{}
\nv{Øverphøs}{Stephanie Bol}{BME17}{}
\end{narvarolista}


\subsection*{Ständigt adjungerande}
\begin{narvarolista}
%\nv{}{}{}{}
%\nv{Skattmästare}{Daniel Bakic}{E15}{\nafr{10}}
\nv{Vice Krögare}{Klara Indebetou}{BME17}{}
\nv{Vice Krögare}{Hjalmar Tingberg}{BME16}{}
\nv{Valberedningens ordförande}{Axel Voss}{E15}{\naun{9}{10A}}
%\nv{Kårrepresentant}{Ivar Vänglund}{}{}
\nv{Kårrepresentant}{Martin Bergman}{}{}
%\nv{Fullmäktigeledamot}{Magnus Lundh}{E15}{\nafr{12}}
%\nv{Chefredaktör}{Erik Eriksson}{--}{}
%\nv{Inspektor}{Monica Almqvist}{}{}

\end{narvarolista}

%\begin{comment}
\subsection*{Adjungerande}
\begin{narvarolista}
%\nv{post}{namn}{klass}{nati/nafr/tom}
%\nv{Likabehandlingsombud}{Jonna Fahrman}{BME17}{}
%\nv{Likabehandlingsombud}{Hanna Bengtsson}{BME18}{}
%\nv{Projekfunktionär}{Emma Hjörneby}{BME17}{}
%\nv{Macapär}{Filip Larsson}{E17}{}
\nv{Redaktör}{Adam Belfrage}{BME17}{}
\nv{Valberedningens sekreterare}{Sanna Nordberg}{BME16}{}

\end{narvarolista}
%\end{comment}

\section*{Protokoll}
\begin{paragrafer}
\p{1}{OFMÖ}{\bes}
Ordförande {\mo} förklarade mötet öppnat kl 12.10

\p{2}{Val av mötesordförande}{\bes}
{\valavmo}

\p{3}{Val av mötessekreterare}{\bes}
{\valavms}

\p{4}{Val av justeringsperson}{\bes}
{\valavj}

\p{5}{Godkännande av tid och sätt}{\bes}
{\tosg}

\p{6}{Adjungeringar}{\bes}
%Adam Belfrage adjungerades.{}
%Hanna Bengtsson adjungerades. \\
%Jonna Fahrman adjungerades.
 
Adam Belfrage adjungerades. \\
Sanna Nordberg adjungerades.

%\textit{Inga adjungeringar.}

\p{7}{Godkännande av dagordningen}{\bes}

%Davida \ypa lägga till punkten ``Lophtet'' till dagordningen.\\
%Edvard \ypa lägga till punkten ``Ordensband'' til dagordningen.
%Fredrik \ypa att lägga till \S18b ``Teknikfokus utnyttjande av LED-café''.
%Jonathan \ypa ändra punkten §12 från att vara en beslutspunkt till diskussion. \\
%Föredragningslistan godkändes med yrkandet.
%Henrik \ypa lägga till punkten ``Faktura till F'' som §13.
%Jakob Pettersson \ypa tägga till punkten ''Øverphøs informerar'' som \S16.

Lina yrkade på att lägga till ''Studiesocial enkät'' som \S12

Föredragningslistan godkändes med yrkandet.

\p{8}{Föregående mötesprotokoll}{\bes}
\latillprotgodkand{S19/19}
%\textit{\ingaprot}

\p{9}{Fyllnadsval och entledigande av funktionärer}{\bes}
\begin{fyllnadsval} %"Inga fyllnadsval." fylls i automatiskt
%\fval{Moa Rönnlund}{Halvledare}
%\entl{Fanny Månefjord}{Husstyrelserepresentant från och med 30 juni}
\fval{Matilda Horn}{Halvledare}

\fval{Alexander Lundqvist}{Diod}

Adam kandiderade till valberedningsledamot och fick möjlighet att presenterade sig själv. 

\fval{Adam Belfrage}{Valberedningsledamot}

\end{fyllnadsval}

\p{10}{Rapporter}{}
\begin{paragrafer}
\subp{A}{Hur mår alla?}{\info}
%Punkten protokollfördes ej.
Edvard har haft ont i ryggen och en anledning kan vara att hans hyrsäng från AFB är konstig. Edvard har därför börjat sova på soffan och nu känns det bätre. 

Adam förlorade i squashmatch igår vilket han tycker är trist.  

Utöver det mår mötet bra och många är taggade inför NollEgasque. 


\subp{B}{Utskottsrapporter}{\info}

CM har haft terminens första lunchmöte. Utöver det har CM sålt kaffe i mängder, både till studenter och organisationer. 

FVU fortsätter som vanligt. Henrik har letat kvitton och fortsatt med budgetering. Ekiperingsexperterna har börjat sälja phaddergruppsmärken men tog för lite betalt då det blev ett missförstånd med momsen. 
Henrik har också haft möte med Rasmus angående Sicrit och hur man kan lösa kaoset. 
Utöver det har FVU allmänt reparerat saker i lokalerna och betalat fakturor. 

InfU forsätter också som vanligt. Macapärerna har haft möte med DDG och redaktionen har släppt terminens första upplaga av HeHE. 
Mattias meddelade att det har varit endel teknikstrul de senaste veckorna, särskilt i samband med gillen. Elavbrott har gjort att en del av sektionens tekniska utrustning slutade fungera men det har felsökts och nu ska allt fungera som det ska. 
Utöver det har han haft mycket mailkontakt med våra vänsektioner som ska besökta sektionen inför NollEgasque. 

KM har haft nollningens sista gille och det gick bra trots att flera datorsystem gick ner innan öppning. 
Det löste sig tillslut och det blev en väldigt bra stäming när så många äldre Phøs kom. 
Den nya trådlösa routern kom också till god använding när eduroam gick ner. Utöver det har Davida bokfört och rättat tidigare bokföringsfel. 


NollU har gjort väldigt mycket den senaste veckan och är nu stressad inför Gasque. Utöver det meddelade Stephanie att hon är trött och vill gå in i dvala. 

ENU har haft lunchföreläsning med TetraPak och planerat nollningens sista lunchföreläsning med Ericsson. Academic Work har ställt upp med en monter i foajen. Invigning av microvågsugnarna med Axis är uppskjutet då de behöver beställa fler klisterlappar. 
Jakob har även haft möte med Alexander för att stämma av arbetet inför Teknikfokus. 

NöjU med idrottsförmännen har hållt i Phaddergruppsolympiaden och i veckan väntar två events, NöjUs tur samt Spelkväll i Edekvata. 

Sexmästeriet har hållt sittning som blev väldigt lyckad trots att ljusslingorna gick sönder. Utöver det har det planerats inför NollEgasquen och Theo ska även ha möte med Stephanie för att spika alla detaljer kring dagen. 

SRE har haft pluggkväll som gick bra trots strömavbrottet. Det är väldigt många som dyker upp vilket tyder på att det är uppskattat. Lina har även varit på SRX-möte där det diskuteras möjligheten till ett gemensamt tack för kursrepresentanter. 

\subp{C}{Ekonomisk rapport}{\info}
Henrik meddelade att ekonomin ser bra ut som vanligt men att arbetsbelastningen är stor då det kommer in vädligt många fakturor. Henrik bad också styrelsen se till att kvitton med tillhörande dokument lämnas in och görs korrekt. 

\subp{D}{Kåren informerar}{\info}

Kåren informerade att nomineringsperioden för höstfullmäktige har börjat. 

Kåren informade också att LU driver ett projekt som heter framtidsveckan mellan den 14 - 20 oktober och att det gärna stöttas av andra studentevenemang.

Till sist meddelade kåren att Mecenatkort kan hämtas i expeditionen för de som saknar. 

Edvard påminde att kåren har startat en namninsamling angående borttagning av tillägsmeriter. 

\subp{E}{Omvärldsrapport}{\info}
Mattias meddelade att alla gästande styrelser som kommer på fredag är taggade. 

Edvard meddelade att vi nu har fått 7 platser till Kallefestivalen som arrangeras av Elektroteknologsektionen vid Chalmers.

\end{paragrafer}

\p{11}{Øverphøs informerar}{\info}

Steph informerade kort om NollEgasquen och upplägget under dagen. 

\p{12}{Studiesocial enkät}{\info}

Lina informerade att Studierådet LTH ska göra en studiesocial enkät och att vi har möjlighet att lägga till 5 programspecifika frågor. 
Lina skall höra om skyddsombuden med likabehandlingsansvar har något att tillägga. 

Deadline för programspecifika frågor är på fredag och har man en idé kan man höra av sig till Lina. 

\p{13}{Nästa styrelsemöte}{\bes}
\Mba nästa styrelsemöte ska äga rum 2019-09-30 12.10 i E:1123.

\p{14}{Beslutsuppföljning}{\bes}
%Edvard \ypa stryka ''Projektfunktionär: Vårbal'' från Beslutsuppföljning. Liknande projekt uppmuntras.
%\Mbaby
%Davida \ypa skjuta upp ''Inköp av draghandtag till cykelvagn'' till nästa styrelsemöte.
%\Mbaby

Henrik \ypa stryka ''Inköp av ny skrivare'' från Beslutsuppföljning. Budget var satt till 6000 kr och kostnaden blev 6000 kr. 

Henrik \ypa stryka ''Inköp av nya izettle läsare''. Budget var satt till 1500 kr och kostnaden blev 1185 kr. 

\Mbabay

\p{15}{Övrigt}{\dis}

Saga undrade om det kommer en utvärdering av nollningen. Steph svarade att det kommer. Henrik påpekade också att vi bör ha en styrelseutvärdering.

Henrik meddelade att det fortfarande ligger saker vid nödutgången som inte får vara där. 

Henrik meddelade också att han kommer skriva i Slack angående budgetmöte med utskottsordföranden och att han försöker fixa med doodle-programmet. 

Edvard tog upp frågan om fakturan till F-sektionen som ska leveras i ett isblock. Mötet diskuterade vad som skulle göras och när. 

Edvard påminde också om valbroschyren till expot samt att man redan nu kan börja på tänka på sina kravprofiler. 
 
\p{16}{Sammanfattning av mötet}{\info}
Adam Belfrage kandiderade till Valberedningsledamot.

Adam Belfrage valdes som Valberedningsledamot. 

Matilda Horn valdes som Halvledare och Alexander Lundqvist som Diod. 

Studiesociala rådet LTH skall gå ut med en studiesocial enkät och sektionen har möjlighet att inkludera programspecifika frågor. 

Följande beslut ströks från Beslutsuppföljningen, ''Inköp av ny skrivare'' samt ''Inköp av nya izettle läsare''.

\p{17}{OFMA}{\bes}
{\mo} förklarade mötet avslutat kl. 12.48
\end{paragrafer}

%\newpage
\hidesignfoot
\begin{signatures}{3}
\signature{\mo}{Mötesordförande}
\signature{\ms}{Mötessekreterare}
\signature{\ji}{Justerare}
\end{signatures}
\end{document}
