\documentclass[10pt]{article}
\usepackage[utf8]{inputenc}
\usepackage[swedish]{babel}

\def\mo{Edvard Carlsson}
\def\ms{Mattias Lundström}
\def\ji{Jakob Pettersson}
%\def\jii{}

\def\doctype{Protokoll} %ex. Kallelse, Handlingar, Protkoll
\def\mname{Styrelsemöte} %ex. styrelsemöte, Vårterminsmöte
\def\mnum{S24/19} %ex S02/16, E1/15, VT/13
\def\date{2019-11-04} %YYYY-MM-DD
\def\docauthor{\ms}

\usepackage{../e-mote}
\usepackage{../../../e-sek}

\begin{document}
\showsignfoot

\heading{{\doctype} för {\mname} {\mnum}}

%\naun{}{} %närvarane under
%\nati{} %närvarande till och med
%\nafr{} %närvarande från och med
\section*{Närvarande}
\subsection*{Styrelsen}
\begin{narvarolista}
\nv{Ordförande}{Edvard Carlsson}{E16}{}
\nv{Kontaktor}{Mattias Lundström}{E17}{}
\nv{Förvaltningschef}{Henrik Ramström}{E16}{}
\nv{Cafémästare}{Jonathan Benitez}{E17}{}
\nv{Sexmästare}{Theo Nyman}{BME18}{}
\nv{Krögare}{Davida Åström}{BME17}{}
\nv{Entertainer}{Saga Åslund}{BME18}{}
%\nv{SRE-ordförande}{Lina Samnegård}{BME16}{}
\nv{ENU-ordförande}{Jakob Pettersson}{E17}{}
\nv{Øverphøs}{Stephanie Bol}{BME17}{}
\end{narvarolista}


\subsection*{Ständigt adjungerande}
\begin{narvarolista}
%\nv{}{}{}{}
%\nv{Skattmästare}{Daniel Bakic}{E15}{\nafr{10}}
%\nv{Vice Krögare}{Klara Indebetou}{BME17}{}
%\nv{Vice Krögare}{Hjalmar Tingberg}{BME16}{}
\nv{Kårrepresentant}{Ivar Vänglund}{}{}
%\nv{Kårrepresentant}{Martin Bergman}{}{}
%\nv{Valberedningens ordförande}{Axel Voss}{E15}{\nafr{10b}}
%\nv{Fullmäktigeledamot}{Magnus Lundh}{E15}{\nafr{12}}
%\nv{Chefredaktör}{Erik Eriksson}{--}{}
%\nv{Inspektor}{Monica Almqvist}{}{}


\end{narvarolista}

%\begin{comment}
\subsection*{Adjungerande}
\begin{narvarolista}
%\nv{post}{namn}{klass}{nati/nafr/tom}
%\nv{Likabehandlingsombud}{Jonna Fahrman}{BME17}{}
%\nv{Likabehandlingsombud}{Hanna Bengtsson}{BME18}{}
%\nv{Projekfunktionär}{Emma Hjörneby}{BME17}{}
%\nv{Macapär}{Filip Larsson}{E17}{}
%\nv{Kodhackare}{Vincent Palmer}{E18}{}
\nv{Øvergudsphadder}{Linnea Söderström}{BME18}{}
\nv{Husstyrelserepresentant}{Joakim Magnusson Fredluend}{BME19}{}
\end{narvarolista}
%\end{comment}

\section*{Protokoll}
\begin{paragrafer}
\p{1}{OFMÖ}{\bes}
Ordförande {\mo} förklarade mötet öppnat kl 12.12

\p{2}{Val av mötesordförande}{\bes}
{\valavmo}

\p{3}{Val av mötessekreterare}{\bes}
{\valavms}

\p{4}{Val av justeringsperson}{\bes}
{\valavj}

\p{5}{Godkännande av tid och sätt}{\bes}
{\tosg}

\p{6}{Adjungeringar}{\bes}
%Adam Belfrage adjungerades.{}
%Hanna Bengtsson adjungerades. \\
%Jonna Fahrman adjungerades.
Linnea Söderström adjungerades.\\
Joakim Magnusson Fredlund adjungeras. 

%\textit{Inga adjungeringar.}


\p{7}{Godkännande av dagordningen}{\bes}

%Davida \ypa lägga till punkten ``Lophtet'' till dagordningen.\\
%Edvard \ypa lägga till punkten ``Ordensband'' til dagordningen.
%Fredrik \ypa att lägga till \S18b ``Teknikfokus utnyttjande av LED-café''.
%Jonathan \ypa ändra punkten §12 från att vara en beslutspunkt till diskussion. \\
%Henrik \ypa lägga till punkten ``Faktura till F'' som §13.
%Jakob Pettersson \ypa tägga till punkten ''Øverphøs informerar'' som \S16.
%Theo yrkade på att ''Inköp av ny lamineringsmaskin'' ska läggas till på Beslutsuppföljning.


%Föredragningslistan godkändes med yrkandet.
Föredragningslistan godkändes.

\p{8}{Föregående mötesprotokoll}{\bes}
\latillprotgodkand{S22/19 \& S23/19}

%\textit{\ingaprot}

\p{9}{Fyllnadsval och entledigande av funktionärer}{\bes}
\begin{fyllnadsval} %"Inga fyllnadsval." fylls i automatiskt
%\fval{Moa Rönnlund}{Halvledare}
%\entl{Fanny Månefjord}{Husstyrelserepresentant från och med 30 juni}
\fval{Matilda Horn}{Halvledare}
\fval{Tove Nimvik}{Diod}
\fval{Amir Ghanaatifard}{Diod}
\fval{Jacob Forsell}{Diod}
\fval{Anton Jigsved}{Diod}
\fval{Hjalmar Tingberg}{Diod}
\fval{Linnea Sjödahl}{Diod}
\fval{León Martinez}{Diod}
\fval{Jimmy Szentes}{Diod}
\fval{Lina Samnegård}{Diod}
\fval{Amanda Nilsson}{Diod}
\fval{Robin Begtsson}{Diod}
\fval{Jonathan Nilsson}{Diod}
\fval{Jens Elfström}{Diod}
\fval{Malin Rudin}{Diod}
\fval{Erica Elgcrona}{Diod}
\fval{Klara Wahldén}{Diod}
\fval{Nora Öhlin}{Diod}
\fval{Tor Hammarbäck}{Diod}
\fval{Fabian Sondh}{Diod}
\fval{Richard Byström}{Diod}
\fval{Petter Melander}{Diod}
\fval{Klara Wahldén}{Halvledare}

\end{fyllnadsval}

\p{10}{Rapporter}{}
\begin{paragrafer}
\subp{A}{Hur mår alla?}{\info}
%Punkten protokollfördes ej.
Mötet mår bra. Stephanie har köpt ett snyggt bord.

Jonathan känner att han behöver komma in i rutiner igen. 

\subp{B}{Utskottsrapporter}{\info}
Edvard har planerat inför Funktionärstacket och Höstterminsmötet. 

Jonathan och CM har mest städat och sökt dioder. Det har gått bra.

Henrik och FVU har letat till orsak varför de tog bort registreringsdosan utanför Edekvata. Det är på grund av ett policy-beslut av LU. 
Cedervall rekommenderar bara att vänja oss.
Henrik har skrivit klart de sista delarna av ekonomin inför HTM och skött allmänt ekonomisk arbete. 

Mattias och informationsutskottet har fortsatt verksamheten som vanligt. Inget vidare har hänt utskottsmässigt sedan tentaperioden. Mer G Suite konton har förberetts tillsammans med Macapärer och Vice. Nästa upplaga av HeHE är också på gång!

Davida och Källarmästeriet har sedan förra mötet haft tre event. En Borgwarnerpub med ENU och två vanliga gillen. Allt har gått bra. De har även sålt biljetter till ölprovningen på fredag och planerat inför resten av terminen. 

Stephanie meddelade att Nolleutskottet mest har testamenten och utvärderingar framför sig. Utöver det har alla i utskottet haft många möten med folk som varit intresserade att söka deras post. 

Jakob och ENU har arrangerat pub med Borgwarner och KM. Allt gick bra och nu är avtalet med BorgWarner uppfyllt. ENU har också haft utskottsmöte och Alumniansvariga har arbetat med ett mailutskick gällande insamling av statistik från utexaminerade alumner samt arbetat med marknadsföring av Alumnipuben. 
Utöver det har Jakob länkat alumnigruppens facebooksida med E-sektionens facebook så att den i framtiden ska bli lättare att hitta. 

Saga och NöjU håller på för fullt att planera diverse event den kommande läsperioden. I helgen är det DreamhackE och intresset är stort. Ölresan har också släppt sin bindande anmälan som tog slut på cirka 4 minuter. Utöver det ordnar kåren en innebandyturnering på lördag. 

Theo och Sexmästeriet har planerat inför tävling i fest-veckan. De är taggade och det kommer förhoppningsvis gå bra. Utöver det har Sexmästeriet möte på onsdag där de bland annat skall städa i Pump. Som avslutning ska de gå på tack-sittning på VGs.


\subp{C}{Ekonomisk rapport}{\info}
Henrik meddelade att det inte är några stora förändringar i ekonomin. De har lite budgetering kvar och det finns en ny rapport om vilka kvitton som inte är inne i systemet.  

\subp{D}{Kåren informerar}{\info}
Ivar informerade att vi nu gått in i arkadveckorna. Det finns events hela veckan som man kan anmäla sig till. Arkadappen samt hemsidan är igång. 

På fredag är det ett kårevent, 'Najs Cool Bar', som är en slags ET slasque som görs av kåren i A huset. 

Ivar meddelade också att det är sista dagen att kandidera till Fullmäktige. 

\subp{E}{Omvärldsrapport}{\info}

Mattias meddelade att vi har tackat nej till Uppsalas inbjudan till deras 5-års jubileum på grund av att ingen av styrelsen kunde närvara. De är också taggade på att hälsa på i Lund i framtiden. 

%Edvard meddelade att Chalmers kommit med lite frågor angående deras uppstart av BME.  


\end{paragrafer}

\p{11}{Äskning av pengar för inköp av video-adapter}{\bes}

Mattias presenterade motionen. 

Mötet tyckte att det var rimligt att sektionen har en bra video-adapter. Tills vidare ska den förvaras i det lilla kassaskåpet tillsammans med övrig kamera- och datorutrustning. 

\textbf{Mötet beslutate att bifalla alla yrkanden från det första alternativet. } 

\p{12}{Funktionärstacket}{\dis}
 
Edvard meddelade förslaget om Curling i Landskrona. Det blir två omgångar med två timmar spel vardera. Efteråt blir det sittning på Göteborgs nation. Av budgetskäl får funktionärer bekosta sitt eget eftersläpp.

Mötet diskuterade lite planering och logistik kring funktionärstacket. 

\p{13}{Nästa styrelsemöte}{\bes}
\Mba nästa styrelsemöte ska äga rum 2019-11-11 17.12 i E:1123.

\p{14}{Beslutsuppföljning}{\bes}
%Edvard \ypa stryka ''Projektfunktionär: Vårbal'' från Beslutsuppföljning. Liknande projekt uppmuntras.
%\Mbaby
%Davida \ypa skjuta upp ''Inköp av draghandtag till cykelvagn'' till nästa styrelsemöte.
%\Mbaby

Theo \ypa skjuta upp ''Inköp av ny lamineringsmaskin' till nästa vecka. Bokföringen är ej klart.

\Mbaby

Henrik har godkänt bokföringen av tårta som köptes in för att fira E-sek events 1000 medlemmar. 
Edvard \ypa stryka ''Inköp av tårta till sektionen'' från Beslutsuppföljning. Edvard meddelade att tårtorna gick under budget. 

\Mbaby

Saga meddelade status på ''Fotovägg - diplomat''. Mötet diskuterade fotoväggens utseende. 

\p{15}{Övrigt}{\dis}

Edvard informerade att det är städvecka för NollU och E6. 

Joakim berättade om husstyrelsemöte som var fredagen innan tentaperioden. Det hände inget speciellt. Undersökning efter brandövningen visade att många inte visste var deras återsamlingsplats var. 
Henrik sa att det förut togs upp på funktionärsinformation i skolstartet. Mötet undrade om det är sektionens eller husets ansvar att informera om återsamlingsplats.

Mattias bad styrelsen att logga in på styrelsens tilldelade G Suite konton och byta lösenord. 

Edvard påminde alla att kolla igenom budgeten inför dagens kvällsmöte. 

\p{16}{Sammanfattning av mötet}{\info}
% \Fbs - Följande beslut \textbf{ströks} från Beslutsuppföljningen,
% \Fbsup Följande beslut från Beslutsuppföljningen \textbf{sköts upp}

%\Mba köpa in en video-adapter, HDMI till USB-C.

\Mba köpa in en video-adapter, HDMI till USB-C.

Dioder och Halvledare fyllnadsvaldes. 

''Inköp av tårta till sektionen'' \textbf{ströks} från Beslutsuppföljningen.

''Inköp av ny lamineringsmaskin'' från Beslutsuppföljningen \textbf{sköts upp} till nästa styrelsemöte. 

\p{17}{OFMA}{\bes}
{\mo} förklarade mötet avslutat kl. 12.59
\end{paragrafer}

%\newpage
\hidesignfoot
\begin{signatures}{3}
\signature{\mo}{Mötesordförande}
\signature{\ms}{Mötessekreterare}
\signature{\ji}{Justerare}
\end{signatures}
\end{document}
