\documentclass[10pt]{article}
    \usepackage[utf8]{inputenc}
    \usepackage[swedish]{babel}
    
    \def\doctype{Sena handlingar} %ex. Kallelse, Handlingar, Protkoll
    \def\mname{Styrelsemöte} %ex. styrelsemöte, Vårterminsmöte
    \def\mnum{S18/19} %ex S02/16, E1/15, VT/13
    \def\date{2019-09-09} %YYYY-MM-DD
    \def\docauthor{Edvard Carlsson}
    
    \usepackage{../e-mote}
    \usepackage{../../../e-sek}
    
    \begin{document}
    
    \heading{{\doctype} till {\mname} {\mnum}}
    
    \section*{Äskning av pengar till inköp av mobil router}
    
    För evenemang som arrangeras av sektionen på andra platser än i e-huset, tex gröngräset eller Lophtet,  och använder till exempel iZettles så finns det ett behov av tillgång till data utöver det som är tillgängligt via eduroam. Nu närmast gäller detta regattan. Detta behov har hittills tillgodosetts av att sektionsmedlemmar delat ifrån sin privata mobildata, något som inte känns schysst för medlemmar som redan ställer upp volontärt för sektionen. Detta har i några tidigare fall lösts med hyra från kåren men de svarar i nuläget att deras utrustning är bristfällig och att vi bör söka en egen lösning på problemet. En lösning på detta är att köpa in en mobil 4G-router till sektionen med kontantkort som kan laddas med data vid behov. Datakostnaden belastar i framtiden det specifika evenemanget medan routern finns tillgänglig för alla sektionens utskott och evenemang. Routern som föreslås är av en billig modell då den inte förväntas behöva användas i över 6 timmar på en laddning och med ett fåtal anslutningar, typiskt två iPads för att köra iZettle. 

    Jag yrkar därför


    \begin{attsatser}
        \att att köpa in en mobil router av märke Alcatel Link Zone med Hallon 20 GB startpaket (\href{https://www.kjell.com/se/produkter/mobilt/mobilt-bredband/router/alcatel-link-zone-med-hallon-20-gb-startpaket-p62100#ProductDetailedInformation}{\textit{länk}}),
        \att budget sätts till \SI{600}{kr},
        \att kostnaden belastar dispositionsfonden, samt
        \att detta läggs på beslutsuppföljningen till S19/19 med undertecknad som ansvarig. 
    \end{attsatser}

    \begin{signatures}{1}
    \textit{\ist}
    \signature{Davida Åström}{Krögare}
    \end{signatures}
    

    \newpage


     \section*{Äskning av pengar för inköp av ljusutrustning}
    
 	Efter vårterminsmötet lyckade inköp av ljudutrustning har jag insett att sektionens har alldeles för lite festgrejer. Vi anordnar många evenemang under året och inte minst nu under nollningen där ordentligt ljus hade kunna lyfta tillvaron. Därför tog jag kontakt med teknokraterna för att se vad de tyckte om idéen. De höll med om att sektionens utrustning var allt för begränsad och tog sedan fram följande förslag:


\begin{enumerate}
 \item Stairville DMX-Master 3 - FX Case Bundle(\href{https://www.thomann.de/se/stairville_dmx_master_3_fx_case_bundle.htm}{\textit{länk}})

 \item Stairville AF-150 DMX Fog Machine(\href{https://www.thomann.de/se/stairville_af_150_dmx_fog_machine.htm}{\textit{länk}})
 \item LED + Laser + Strobe Multi Effect Bar(\href{https://www.thomann.de/se/stairville_dmx_master_3_fx_case_bundle.htm}{\textit{länk}})
\end{enumerate}   

                          

Med denna motivering till valen: 

\textit{Tanken med denna inhandling var att införskaffa en grundläggande och expanderbar ljusrigg till E-sektionen. Den bör vara relativt portabel och fungera på evenemang såsom spelningar, klubb och diverse eftersläpp.} 

\textit{Med detta i åtanke har denna lista kompilerats som bör fylla de flesta krav en första inhandling kan tänkas ha. Den innehåller ljusbord, med case, rökmaskin, med vätska, kompatibel med både 3- och 5-pin DMX; FX-ljuskombo, med bl.a. laser, och LED-fixtur; och kablar till detta.}

\textit{All denna utrustning är dessutom kompatibel med existerande utrustning i Vega.}

Jag yrkar



    \begin{attsatser}
        \att köpa in utrustningen som beskriven ovan tillsammans med lite tillbehör så som kablar, fjärrkontroll och rökvätska(\href{https://www.thomann.de/se/wishlist_4u_05097b248996.html}{\textit{länk}}),
        \att budget sätts till \SI{9000}{kr},
        \att kostnaden belastar dispositionsfonden, samt
        \att detta läggs på beslutsuppföljningen till S20/19 med undertecknad som ansvarig. 
    \end{attsatser}

    \begin{signatures}{1}
    \textit{\ist}
    \signature{Edvard Carlsson}{Ordförande}
    \end{signatures}





   



    \end{document}
    
    