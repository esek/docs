\documentclass[10pt]{article}
\usepackage[utf8]{inputenc}
\usepackage[swedish]{babel}

\def\mo{Edvard Carlsson}
\def\ms{Mattias Lundström}
\def\ji{Theo Nyman}
%\def\jii{}

\def\doctype{Protokoll} %ex. Kallelse, Handlingar, Protkoll
\def\mname{Styrelsemöte} %ex. styrelsemöte, Vårterminsmöte
\def\mnum{S19/19} %ex S02/16, E1/15, VT/13
\def\date{2019-09-16} %YYYY-MM-DD
\def\docauthor{\ms}

\usepackage{../e-mote}
\usepackage{../../../e-sek}

\begin{document}
\showsignfoot

\heading{{\doctype} för {\mname} {\mnum}}

%\naun{}{} %närvarane under
%\nati{} %närvarande till och med
%\nafr{} %närvarande från och med
\section*{Närvarande}
\subsection*{Styrelsen}
\begin{narvarolista}
\nv{Ordförande}{Edvard Carlsson}{E16}{}
\nv{Kontaktor}{Mattias Lundström}{E17}{}
%\nv{Förvaltningschef}{Henrik Ramström}{E16}{}
\nv{Cafémästare}{Jonathan Benitez}{E17}{}
\nv{Sexmästare}{Theo Nyman}{BME18}{}
\nv{Krögare}{Davida Åström}{BME17}{\nafr{10A}}
\nv{Entertainer}{Saga Åslund}{BME18}{}
\nv{SRE-ordförande}{Lina Samnegård}{BME16}{}
\nv{ENU-ordförande}{Jakob Pettersson}{E17}{}
%\nv{Øverphøs}{Stephanie Bol}{BME17}{}
\end{narvarolista}


\subsection*{Ständigt adjungerande}
\begin{narvarolista}
%\nv{}{}{}{}
%\nv{Skattmästare}{Daniel Bakic}{E15}{\nafr{10}}
%\nv{Vice Krögare}{Klara Indebetou}{BME17}{}
%\nv{Vice Krögare}{Hjalmar Tingberg}{BME16}{}
\nv{Kårrepresentant}{Ivar Vänglund}{}{}
\nv{Kårrepresentant}{Martin Bergman}{}{}
%\nv{Valberedningens ordförande}{Axel Voss}{E15}{\nafr{10b}}
%\nv{Fullmäktigeledamot}{Magnus Lundh}{E15}{\nafr{12}}
%\nv{Chefredaktör}{Erik Eriksson}{--}{}
%\nv{Inspektor}{Monica Almqvist}{}{}
\nv{Vice Förvaltningschef}{Rasmus Sobel}{BME16}{}
\nv{Cophøs}{Tove Börjeson}{E17}{}



\end{narvarolista}

%\begin{comment}
\subsection*{Adjungerande}
\begin{narvarolista}
%\nv{post}{namn}{klass}{nati/nafr/tom}
%\nv{Cophøs}{Tove Börjeson}{E17}{}
%\nv{Likabehandlingsombud}{Jonna Fahrman}{BME17}{}
%\nv{Likabehandlingsombud}{Hanna Bengtsson}{BME18}{}
%\nv{Projekfunktionär}{Emma Hjörneby}{BME17}{}
%\nv{Macapär}{Filip Larsson}{E17}{}
%\nv{Kodhackare}{Vincent Palmer}{E18}{}
\end{narvarolista}
%\end{comment}

\section*{Protokoll}
\begin{paragrafer}
\p{1}{OFMÖ}{\bes}
Ordförande {\mo} förklarade mötet öppnat kl 12.12

\p{2}{Val av mötesordförande}{\bes}
{\valavmo}

\p{3}{Val av mötessekreterare}{\bes}
{\valavms}

\p{4}{Val av justeringsperson}{\bes}
{\valavj}

\p{5}{Godkännande av tid och sätt}{\bes}
{\tosg}

\p{6}{Adjungeringar}{\bes}
%Adam Belfrage adjungerades.{}
%Hanna Bengtsson adjungerades. \\
%Jonna Fahrman adjungerades.
%Vincent Palmer adjungerades.\\
%Filip Larsson adjungerades. 


\textit{Inga adjungeringar.}


\p{7}{Godkännande av dagordningen}{\bes}

%Davida \ypa lägga till punkten ``Lophtet'' till dagordningen.\\
%Edvard \ypa lägga till punkten ``Ordensband'' til dagordningen.
%Fredrik \ypa att lägga till \S18b ``Teknikfokus utnyttjande av LED-café''.
%Jonathan \ypa ändra punkten §12 från att vara en beslutspunkt till diskussion. \\
%Föredragningslistan godkändes med yrkandet.
%Henrik \ypa lägga till punkten ``Faktura till F'' som §13.
%Jakob Pettersson \ypa tägga till punkten ''Øverphøs informerar'' som \S16.

%Föredragningslistan godkändes med yrkandet.

Föredragningslistan godkändes.

\p{8}{Föregående mötesprotokoll}{\bes}
\latillprotgodkand{S18/19}
%\textit{\ingaprot}

\p{9}{Fyllnadsval och entledigande av funktionärer}{\bes}
\begin{fyllnadsval} %"Inga fyllnadsval." fylls i automatiskt
%\fval{Moa Rönnlund}{Halvledare}
%\entl{Fanny Månefjord}{Husstyrelserepresentant från och med 30 juni}
\entl{Malin Heyden}{Kodhackare från och med 16 september}

\end{fyllnadsval}

\p{10}{Rapporter}{}
\begin{paragrafer}
\subp{A}{Hur mår alla?}{\info}

%Punkten protokollfördes ej.

Alla mår bra och mötet är glada att sektionen vann regattan. 
Rasmus saknar rösten och Jakob hade en fartfylld helg.  




\subp{B}{Utskottsrapporter}{\info}
Jonathan och CM har fullt ös. 
Internationella studenter har hjälp till i caféet och Theo har hjälpt till genom att köpa in powerking. 
Alla pass från 8-13 är fulla vilket är bra. Antoina är snabb på IC rapporter. Utöver det har Jonathan även haft kvällsmöte med helårare samt bokat in Tour de CM. 


FVU har jobbat på som vanligt. Försökt göra en lättare städning av Sicrit. 
FVU har även lånat ut stolar till kåren för Nollehelgen och Henrik har varit på höstens första kollegiemöte. 
Edekvata har hyrts ut till V-sektionen för en sittning. Utöver det har det satts upp en ny skrivare i HK. 
\\ Theo kommenterade att vi behöver utforma ett bättre system tillsammans med D-sektionen när det gäller uthyrning av de gemensamma stolarna. Davida kommenterade också att det var glassplitter i edekvata efter V-sektionens sittning och att uthyrningen av våra lokaler måste fungera bättre. 


InfU har fortsatt sin verksamhet under veckan som vanligt med att fotografera, skriva artiklar till HeHE, designa, och rigga ljud. Macapärerna har uppstartmöte med DDG imorgonkväll som förhoppingsvis ska tagga till folk att fortsätta utvecklandet av vår mjukvara. Det finns massa saker som behöver göras och många förslag på projekt.
Utöver det är äntligen alla anmodningar och inbjudningar gjorda och utskickade. 


KM har haft en intensiv helg med pubrunda och regatta. Pubrundan var lyckad och det blev försäljningsrekord. Regattan var också lyckad och all mat tog slut. Nu väntar bokföring och planering av gille på fredag.


NollU förtsätter som planerat. Alla veckans event har varit uppskattade och relativt smärtfritt. NollU har även hört att regattan gick bra.  

ENU har planerat lunchföreläsningar och satt datum for workshop med VentureLabs. Planeringen av workshop med Knightec tillsammans med D-sektionen har också fortsatt under veckan. Utöver det skall AXIS komma till edekvata och inviga de nya mikrovågsugnarna.

NöjU har haft utskottsmöte om planering inför resten av nollningen och hösten. Idrottsförmännen planerar inför Phaddergruppsolympiaden som behöver hjälp att hypeas upp. NöjU höll även i MegaKulfEstivalen vilket blev lyckat. Bra med ett stort alkoholfritt event. Igår höll Saga i Regattaphesten och det verkade som folk hade roligt. 
Utöver det kan Saga meddela att en grupp på E-sektionen är taggade på att söka till projektgruppen för Sångarstriden. 

Sexmästeriet har hållt en finsittning med K- och A-sektionen vilket var en bra övning inför NollEgasquen. I veckan är det Vettiquette och på lördag är det Nollesittningen. 
Försäljningen har inte varit optimal och reservlistorna har varit lite problematiska när biljettförsäljning skett så nära inpå sittningarna. 

SRE anordnar pluggkväll ikväll vilket har varit uppskattat. Utöver det planeras CEQ-möten och imorgon ska Lina på kollegiemöte med kåren. 
\subp{C}{Ekonomisk rapport}{\info}

Rasmus informerade om sektionens ekonomi och meddelade att förra årets resultat finns för alla utskottsordförande att se i styrelsens Slack. 
Rasmus meddelade också att sektionen har bra med pengar på kontot och påminner alla som vanligt att se över och skicka in sina kvitton. 

\subp{D}{Kåren informerar}{\info}

Ivar påminde om anmodan för TLTH representanter till NollEgasquen. Kårens medlemssystem krånglar. Kåren funderar också på att rensa ut gamla mailkontakter för utskick av exempelvis Kårnytt. Detta för att slippa en ökad kostnad för kårens system. 

Ivar meddelade även att kåren har fortsatt rekrytering till de olika projekt inom aktivitetsutskottet och även Athena i näringslivsutskottet.

\subp{E}{Omvärldsrapport}{\info}
Mattias meddelade att SIK Aalto från Finland har svarat på anmodan till NollEGasquen och att de vill komma och besöka E-sektionen med två styrelserepresentanter.

\end{paragrafer}

\p{11}{Øverphøs informerar}{\info}

Tove informerade kort om nästa veckas nollningsaktiviteter. Nollningen har smått börjat lugna ner sig. 

Tove och NollU meddelade att det har uppmärksammats att det skett viss alkoholhets på nollningsevent vilket inte är okej. Sektionen skall ha en hälsosam alkoholkultur. 
Påminde styrelsen att vara en bra förebild i detta. 

\p{12}{Nästa styrelsemöte}{\bes}
\Mba nästa styrelsemöte ska äga rum 2019-09-24 12.10 i E:3316.

\p{13}{Beslutsuppföljning}{\bes}
%Edvard \ypa stryka ''Projektfunktionär: Vårbal'' från Beslutsuppföljning. Liknande projekt uppmuntras.
%\Mbaby
%Davida \ypa skjuta upp ''Inköp av draghandtag till cykelvagn'' till nästa styrelsemöte.
%\Mbaby

Davida \ypa stryka ''Inköp av draghandtag till cykelvagn'' från Beslutsuppföljning.

\Mbaby

Davida \ypa stryka ''Inköp av mobil router'' från Beslutsuppföljning.

\Mbaby

Rasmus \ypa skjuta upp ''Inköp av ny skrivare'' till nästa styrelsemöte S20/19.

\Mbaby 


\p{14}{Övrigt}{\dis}

Jakob påminde att styrelsen skall träffas imorgon och skriva spex.

Ivar undrade vem man ska kontakta angående ekonomin kring UtEDischot. Saga svarade att det faller på D-sektionen och deras skattmästare. 

Jonathan tycker att det har varit dåligt med information kring nollningsevents till sektionsmedlemmar som står utanför nollningen. 
Mötet diskuterade sektionens marknadsföring av nollningsevent.  

Lina påpekade att det snart är NollEgasque. Styrelsen behöver träffas för att gå igenom styrelsens instats. 

Theo berättade att användningen av de vita borden som delas med D-sektionen har missköts. Behöver sprida information om att de skall torkas av ordentligt efter varje användning. 

\p{15}{Sammanfattning av mötet}{\info}
Funktionärer har entledigats.

Ekonomin ser fortsatt god ut. 

Det har uppmärksammats alkoholhets på nollningsevenemang. Styrelsen skall jobba för att vara en bra förebild.

Följande beslut ströks från Beslutsuppföljningen, ''Inköp av draghandtag till cykelvagn'' samt ''Inköp av mobil router''.  

Följande beslut från Beslutsuppföljningen sköts upp till nästa styrelsemöte S20/19, ''Inköp av ny skrivare''.


\p{16}{OFMA}{\bes}
{\mo} förklarade mötet avslutat kl. 13.02
\end{paragrafer}

%\newpage
\hidesignfoot
\begin{signatures}{3}
\signature{\mo}{Mötesordförande}
\signature{\ms}{Mötessekreterare}
\signature{\ji}{Justerare}
\end{signatures}
\end{document}
