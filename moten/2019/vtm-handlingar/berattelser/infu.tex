\documentclass[../_main/handlingar.tex]{subfiles}

\begin{document}
\berattelse{Informationsutskottet 2018}
2018 har varit ett produktivt år för utskottet. DDG har haft en hel del projekt igång och macapären som varit ensam har därför haft mycket att göra. Bland annat så har DDG fixat till fotomojt, installerat nya spamfilter, utvecklat en grupprumsmodul och en sångboksapp. De större projekten från tidigare år lever fortfarande vidare, framförallt nollningshemsidan, gillemode, LED-rfid och den nya hemsidan. Vi valde in extra många kodhackare under första terminen men under hösten så entledigade jag många av dem. En stor del av dem hade inte varit så aktiva under våren och kände inte att de hade tid under hösten heller. 
Teknokraterna rivstartatade året med två stora fix-kvällar tillsammans med vice-förvaltningschefen där de såg de över vad som behövde lagas, planerade och la fram förslag på vad sektionen behöver köpa in för ny teknik. De har under året även skrivit dokumentation för vår utrustning och satt ihop ett fint testamente till sina efterträdare. Jag tycker utskottet ska fortsätta på det spåret, det är viktigt att saker dokumenteras! Topparna i Vega byttes även ut efter HT/18 och de två nya fungerar superbra.  

Trots den nya fina sektionskameran som köptes in efter VT/18 har det varit ganska svårt att aktivera våra fotografer. Det fotograferades relativt flitigt under nollningen men inte alls speciellt mycket under andra events under året. De har haft mycket annat för sig och sektionsarbetet har därför tyvärr blivit lite bortprioriterat på den fronten. Som tur är har vi fått hjälp av andra funktionärer när det krisat och under gasque fick vi hjälp av A-sektionen. 
Chefredaktören som blev invald efter vårteminsmötet kämpade på med HeHe även i slutspurten och lyckades otroligt bra med tidningen helt ensam. Jag tror att hans hjälteinsats är anledningen till att så många sökte till redaktionen detta året och jag hoppas att HeHe fortsätter publiceras eftersom tidningen verkligen är uppskattad.  
Picassosarna har gjort ett extremt bra jobb med många fina affischer och grafik till skärmarna.

I år har jag jobbat med att försöka få ett bra samarbete med de andra högskolorna i Sverige och övriga Norden. Jag tycker jag lyckades mycket bra på den fronten och vi har nästan veckolig kontakt med KTH och Chalmers men även med Trondheim och Aalto. Jag hoppas årets styrelse tar till vara på detta och försöker representera både sektionen och Lund när chansen väl ges.

\begin{signatures}{1}
    \mvh
    \signature{Axel Voss}{Kontaktor 2018}
\end{signatures}

\end{document}
