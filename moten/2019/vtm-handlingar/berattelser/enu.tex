\documentclass[../_main/handlingar.tex]{subfiles}

\begin{document}
\berattelse{Näringslivsutskottet 2018}

ENU började året med en CV-fotografering i sammarbete med sektionens fotografer. Utskottet drog även i ”Lunch med en ingenjör” både på våren och hösten, och intresset var högt från många studenter.  Det annordnades många lunchföreälsningar under året där vi hade caterad mat för att underlätta för ENU.
Teknikfokus drog in mycket pengar till sektionen och lockade både många företag och studenter. E-sektionen bidrog med en hel del funktionärer men fortfarande lite mindre än D-sektionen. Det är viktigt att engagemanget från båda sektionerna är lika stort för att mässan och samarbetet med D-sektionen ska fortsätta vara framgångsrikt.  
Under nollningen stod ENU för maten under SVEP robotic challenge och vi höll även i tre stycken lunchföreläsningar för de nyantagna studenterna. ENU har även fixat så att en av pluggkvällarna var sponsrade. 
ENU avslutade året med en CV-gransking tillsammans med Academic Work vilket var väldigt uppskattat bland studenterna. Alumniansvariga annordnade en pub för alumnerna tillsammans med KM och D-sektionen. Vi hade även två stycken lunchföreläsningar för att avsluta året.
”FED pub” har även annordnats två gånger under året i sammarbete med F-och D-sektionen. Dessa pubar har varit ett tillfälle för olika företag att träffa studenter under i en mer avslappnad atmosfär.   



\begin{signatures}{1}
    \mvh
    \signature{Isabella Hansen}{Näringslivsutskottets Ordförande 2018}
\end{signatures}

\end{document}
