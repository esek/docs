
\documentclass[../_main/handlingar.tex]{subfiles}

\begin{document}
\motion{Utforma en funktionärspost för spektakel, Spexmästaren}
Spektakel, från latinens Spectaculum( skådespel), eller i samtidens tunga, spex, är en form av amatörteater med en stark anknytning till studentikos. Spex uppstod under 1850-talet vid studentnationerna i Uppsala och var oftast parodier på kända grekiska draman eller dylika pjäser. Det gick inte lång tid förrän spex blev en allmän företeelse även i Lund, där även det första originalspexet Gerda iscensattes och som bringade tillkommelsen av akademiföreningens spexgrupp, Lundaspexarna, där Hans Hasse Alfredsson bland annat varit medlem. 
Varför bry sig då om detta? 
\\

Spex på E-sek

Spex är en essentiell del av studentlivet men är underutvecklat inom vår sektion. Det innebär inte att det ej förekommer då det är alltid personer som är måna att spexa på tillställningar. Dock är dessa spex, om än alltid lustiga och för en till tårar, ringa enahanda, då sång är oftast spexets kontenta. Om en skulle läsa vad betydelsen för ordet spex innefattar, ser en att spex dessutom innefattar skådespeleri. Vår älskade sektion har alltså endast skrapat på ytan av den potential som spex innefattar. Dock att ställa ett omåttligt krav på kvalité på personer som vill spexa är endast avskräckande, drygt och framförallt tidskonsumerande. Därför framförs förslaget att uftorma en ny funktionärspost preliminärt inom sexmästeriet, spexmästaren/arna.
\\

Vem är spexmästaren

Spexmästaren är en funktionärspost som förekommer på andra sektioner, exempelvis I-sektionen, där spexmästaren innefattar en av deras sexmästeris triumvirat. Spexmästarens/as ansvarsområde är tillsammans med sångförmännen att underhålla gäster under sittning och dylikt, men bär istället på ansvaret över spexen under sittningen. Spexmästaren/arna är också ansvarig att spexa tillsammans med en grupp av personer under tillställningen. förslagsvis väljs personer som arbetar med spexmästaren av spexmästaren via en form av audition där ett kort spex framförs och de personer som väljs sedan arbetar med spexmästaren resterande läsår, eftersom det är nödvändigt med en bra gruppdynamik som dessutom utvecklas så de bästa möjliga spexen kan framföras. Denna post räknas fortfarande som en "Sexig"-post, fast kan måhända kallas för "Spexig" istället. 
\\

Varför behövs spexmästaren?

Spexmästaren är en funktionärspost och flera som möjliggör en ny uttrycksform för engagemang inom sektionen för de som gillar att stå på scen och underhålla och känner att de kan tillföra med skådespeleri, finurligt skrivande och parodiska sånger.

Med anledning som ovan yrkar motionären
\begin{attsatser}
  \att i reglementet \S10:2:K lägga till:\par
  \begin{emptylist}
    \item Spexmästare (1)
      \begin{dashlist}
        \item Ansvarar för spex under sittningar.
        \item Spexar med sin grupp, beståendes av ett urval sexiga, under sittningar. 
      \end{dashlist}
    \end{emptylist}
    
    
    
    
\end{attsatser}




\begin{signatures}{1}
    \textit{Signerat}
    \signature{Adam Ekblom}{}

\end{signatures}

\end{document}
