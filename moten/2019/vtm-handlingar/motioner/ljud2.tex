\documentclass[../_main/handlingar.tex]{subfiles}
\usepackage{eurosym}


\begin{document}
\motion{Inköp av ljudteknik}

\subsection*{Bakgrund}
    Den nuvarande ljudutrustningen på sektionen är, enligt oss, undermålig. Detta i samband med evenemang såsom spelningar med husbandet, eftersläpp, SåS, eventuella framtida baler, bl.a. 
    Konkreta exempel på brister i nuvarande ljudsystemet är bland annat antalet ingångar, och kvaliteten på dessa. Vårt nuvarande mixerbord är begränsat till 2 mikrofoningångar och 4 linjeingångar. Det har inte heller några märkvärdiga ljudverktyg såsom kompressorer eller noise gates. Fler ingångar och ljudverktyg ger möjlighet till större uppsättningar och högre kvalitet på dessa. Till exempel kan hela sektionsbandet mickas upp, vilket objektivt ger en betydligt bättre ljudupplevelse. Konkreta exempel på ljudskillnader är tydligare sång, en bastrumma som känns och detaljerat ljud från resten av instrumenten så att allt hörs. 
    
    Vårt nuvarande delningsfilter är analogt och även begränsat till enbart frekvensfiltrering. Föreslagna delningsfiltret är digitalt och har funktioner som automatisk frekvensanpassning, samt fler utkanaler vilket tillåter expansion av stora PA (såsom till exempel fler subwoofers).
    
    Vårt nuvarande rack är både litet och otympligt. Det föreslagna har plats för nuvarande utrustning, ny utrustning, samt plats för expansion. Detta med hjul och stabilitet för förenklad förflyttning. 
   
   Ett alternativ till att införskaffa egen utrustning är att hyra av kåren. Där finns både fördelar och nackdelar, nedan finner vi några av dessa. 
   
   \subsubsection*{Sektionen hyr av kåren}
   Fördelar: 
    \begin{itemize}
        \item Mindre lagring
        \item Kortsiktigt lägre kostnad
    \end{itemize}

    Nackdelar: 
    \begin{itemize}
        \item Kräver ofta kontakt med kåren i god tid (>1 månad innan)
        \item Kåren är ofta otillgänglig på kort varsel under helgen
        \item Då kårens utrustning är uthyrd blir sektionen utan
        \item Kräver en hyresutgift
    \end{itemize}
   
   \subsubsection*{Sektionen skaffar egen utrustning}
    Fördelar: 
    \begin{itemize}
        \item Sektionen har garanterat god ljudutrustning till sina evenemang
        \item Minskar hyresutgifter
        \item Tillåter spontanevenemang
        \item Tillåter lättare helgevenemang
        \item Tillåter god ljudanläggning till rep inför bl.a. SåS
    \end{itemize}

    Nackdelar: 
    \begin{itemize}
        \item Kräver 0.6 kvadratmeters förvaringsutrymme
        \item Bär en investeringskostnad
    \end{itemize}

    \subsection*{Förslag}
    Vi anser därför att stora delar av denna utrustning bör ersättas såsom mixerbord, delningsfilter, rackskåp och diverse tillbehör. 

    Mixerbordet vi ämnar att införskaffa har 16 XLR/Jack-ingångar, 8 fysiska utgångar samt styrning över ethernet/WiFi med väldigt goda virtuella kopplingsmöjligheter och effekter. 

    Mixerbordet är monterbart i rackskåp och saknar fysiska reglage. Därför vill vi även införskaffa en tablet vilket tillåter trådlöst mixande för ljudteknikern, vilket är extremt praktiskt och man slipper då även dra en massa kablar över eventuella dansgolv. 
    
    Vi ämnar även införskaffa en spänningsstabilisator med inbyggd propp, vilket både minskar brus och skyddar vår utrustning mot eventuella strömfel. 
    
    Sammanfattningsvis skulle följande utrustning ge sektionen möjlighet till betydligt bättre ljud, med goda möjligheter för expansion i framtiden. 
    

    Utrustning: 
    \begin{itemize}
        \item Behringer X Air XR18 (398\euro)           \href{https://www.thomann.de/se/behringer_x_air_xr18.htm}{länk}
        \item t.racks DSP 206 (299\euro)                \href{https://www.thomann.de/se/the_t.racks_dsp_206.htm}{länk}
        \item Thon Rack 15U Profi 45 Wheels (319\euro) \href{https://www.thomann.de/se/thon_rack_15he_pro_live45_m_rollen.htm}{länk}
        \item Furman M-10x E (122\euro)                 \href{https://www.thomann.de/se/furman_m10x_e.htm}{länk}
        \item Samsung Galaxy Tab A6 (1890 SEK)          \href{https://www.netonnet.se/art/dator/surfplattor/samsung-galaxy-tab-a-10-1-2016-wifi-32gb-white/1002419.8899}{länk}
        \item Kringutrustning såsom kablage (~130\euro)
    \end{itemize}




    Därför yrkar vi 
    \begin{attsatser}
       \att sektionen köper in ett mixerbord (till en kostnad av 398\euro),
        \att sektionen köper in en tablet (till en kostnad av ca. 1890 SEK),
       \att sektionen köper in ett nytt delningsfilter (till en kostnad av 299\euro),
       \att sektionen köper in ett nytt rackskåp (till en kostnad av 319\euro),
       \att sektionen köper in kringutrustning till yrkanden ovan (till en kostnad av ca. 130\euro),
       \att budgeten sätts till 15 000 SEK,
       \att kostnaderna belastar utrustningsfonden, samt, 
       \att detta läggs på beslutsuppföljningen till Höstterminsmötet 2019 med undertecknade som ansvariga.

      \end{attsatser}




\begin{signatures}{3}
        \textit{I  sektionens tjänst}
        \signature{David Karlsson}{Teknokrat 2019}
        \signature{Emil P. Lundh}{Teknokrat 2019}
        \signature{Moa Rönnlund}{Teknokrat 2019}
    \end{signatures}
\end{document}

