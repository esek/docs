\documentclass[10pt]{article}
\usepackage[utf8]{inputenc}
\usepackage[swedish]{babel}

\def\mo{Edvard Carlsson}
\def\ms{Mattias Lundström}
\def\ji{Theo Nyman}
%\def\jii{}

\def\doctype{Protokoll} %ex. Kallelse, Handlingar, Protkoll
\def\mname{Styrelsemöte} %ex. styrelsemöte, Vårterminsmöte
\def\mnum{S25/19} %ex S02/16, E1/15, VT/13
\def\date{2019-11-11} %YYYY-MM-DD
\def\docauthor{\ms}

\usepackage{../e-mote}
\usepackage{../../../e-sek}

\begin{document}
\showsignfoot

\heading{{\doctype} för {\mname} {\mnum}}

%\naun{}{} %närvarane under
%\nati{} %närvarande till och med
%\nafr{} %närvarande från och med
\section*{Närvarande}
\subsection*{Styrelsen}
\begin{narvarolista}
\nv{Ordförande}{Edvard Carlsson}{E16}{}
\nv{Kontaktor}{Mattias Lundström}{E17}{}
\nv{Förvaltningschef}{Henrik Ramström}{E16}{}
\nv{Cafémästare}{Jonathan Benitez}{E17}{}
\nv{Sexmästare}{Theo Nyman}{BME18}{}
\nv{Krögare}{Davida Åström}{BME17}{}
\nv{Entertainer}{Saga Åslund}{BME18}{}
\nv{SRE-ordförande}{Lina Samnegård}{BME16}{}
\nv{ENU-ordförande}{Jakob Pettersson}{E17}{}
\nv{Øverphøs}{Stephanie Bol}{BME17}{}
\end{narvarolista}


\subsection*{Ständigt adjungerande}
\begin{narvarolista}
%\nv{}{}{}{}
%\nv{Skattmästare}{Daniel Bakic}{E15}{\nafr{10}}
%\nv{Vice Krögare}{Klara Indebetou}{BME17}{}
%\nv{Vice Krögare}{Hjalmar Tingberg}{BME16}{}
\nv{Kårrepresentant}{Ivar Vänglund}{}{}
%\nv{Kårrepresentant}{Martin Bergman}{}{}
%\nv{Valberedningens ordförande}{Axel Voss}{E15}{\nafr{10b}}
%\nv{Fullmäktigeledamot}{Magnus Lundh}{E15}{\nafr{12}}
\nv{Chefredaktör}{Erik Eriksson}{E17}{}
%\nv{Inspektor}{Monica Almqvist}{}{}



\end{narvarolista}

%\begin{comment}
\subsection*{Adjungerande}
\begin{narvarolista}
%\nv{post}{namn}{klass}{nati/nafr/tom}
%\nv{Likabehandlingsombud}{Jonna Fahrman}{BME17}{}
%\nv{Likabehandlingsombud}{Hanna Bengtsson}{BME18}{}
%\nv{Projekfunktionär}{Emma Hjörneby}{BME17}{}
%\nv{Macapär}{Filip Larsson}{E17}{}
%\nv{Kodhackare}{Vincent Palmer}{E18}{}
\nv{Sigillbevarare}{Matilda Horn}{BME18}{}
\end{narvarolista}
%\end{comment}

\section*{Protokoll}
\begin{paragrafer}
\p{1}{OFMÖ}{\bes}
Ordförande {\mo} förklarade mötet öppnat kl 17.13

\p{2}{Val av mötesordförande}{\bes}
{\valavmo}

\p{3}{Val av mötessekreterare}{\bes}
{\valavms}

\p{4}{Val av justeringsperson}{\bes}
{\valavj}

\p{5}{Godkännande av tid och sätt}{\bes}
{\tosg}

\p{6}{Adjungeringar}{\bes}
%Adam Belfrage adjungerades.{}
%Hanna Bengtsson adjungerades. \\
%Jonna Fahrman adjungerades.
%Vincent Palmer adjungerades.\\
%Filip Larsson adjungerades. 
Matilda Horn adjungerades. 

%\textit{Inga adjungeringar.}


\p{7}{Godkännande av dagordningen}{\bes}

%Davida \ypa lägga till punkten ``Lophtet'' till dagordningen.\\
%Edvard \ypa lägga till punkten ``Ordensband'' til dagordningen.
%Fredrik \ypa att lägga till \S18b ``Teknikfokus utnyttjande av LED-café''.
%Jonathan \ypa ändra punkten §12 från att vara en beslutspunkt till diskussion. \\
%Föredragningslistan godkändes med yrkandet.
%Henrik \ypa lägga till punkten ``Faktura till F'' som §13.
%Jakob Pettersson \ypa tägga till punkten ''Øverphøs informerar'' som \S16.

Föredragningslistan godkändes med yrkandet.

\p{8}{Föregående mötesprotokoll}{\bes}
%\latillprotgodkand{S14/19 \& S15/19}
\textit{\ingaprot}

\p{9}{Fyllnadsval och entledigande av funktionärer}{\bes}
\begin{fyllnadsval} %"Inga fyllnadsval." fylls i automatiskt
%\fval{Moa Rönnlund}{Halvledare}
%\entl{Fanny Månefjord}{Husstyrelserepresentant från och med 30 juni}
\fval{Y Nhi Pham}{Halvledare}
\fval{Jacob Forsell}{Årskurs E1-ansvarig}

\end{fyllnadsval}

\p{10}{Rapporter}{}
\begin{paragrafer}
\subp{A}{Hur mår alla?}{\info}
%Punkten protokollfördes ej.

Mötet mår överlag bra och mötet konstaterade att det doftar Arkad i huset. 

Ivar är mer taggad på vårt sektionsmöte än F-sektionens.

Theo är taggad på sittning igen. 

Emil mår fantastikt och han har äntligen fått en lägenhet i Lund.

Lina har ont i tån och Henrik är stressad. 

\subp{B}{Utskottsrapporter}{\info}

Edvard meddelade att styrelsen har gjort färdigt arbetet inför HT. Edvard är stolt över styrelsens arbete. 

Jonathan och CM har som vanligt kämpat på i Caféet. Denna vecka är det Arkad och CM kommer leverera kaffe. 

Henrik har presenterat budgeteringen och den har blivit godkänd av styrelsen. 
Henrik har också deltagit i Vice-kollegiet och släppt rapporter angående kvitton som är kvar och samt vart funktionärsbudgeten gått till. 
Inom FVU har nya soffor och bokbål diskuterats och som vanligt har det bokförts. 

Mattias har haft stormöte med InfU som bestod av statusuppdateringar, utvärdering av LP1, samt en övergripande planering av utskottets fortsatta verksamhet.

Davida och KM hade ölprovning i fredags och alla är supernöjda. Utöver det så löper planering och bokföring på som vanligt. 

Jakob och ENU har den senaste veckan markandsfört våra alumnigrupper. Som vanligt har Jakob mailat lite samt sålt marknadsföring till Sectra. 
Jakob har också marknadsfört evenemang med Knightec som är på måndag nästa vecka. Till sist har Jakob haft möte med sin Vice där de planerat resten av terminen. 

Saga och NöjU har hållt i DreamHackE och det var lyckat! Till helgen är det Ölresan och alla platser är fyllda. Sångarstriden har även börjar repetera ordentligt. 

Theo och Sexmästeriet har fortsatt planera hösten och de har spikat de sista datumet för sittning vilket kommer vara en finsittning i Edekvata. 
Utöver är planeringen av Tävling i Fest i full gång. 

Lina och SRE rullar på som vanligt. De har planerat CEQ-utlottning och de ska lotta ut presentkort på ICA. SRE ska även hålla en brunch som tack för årskursansvariga. 

\subp{C}{Ekonomisk rapport}{\info}

Henrik informerade att det ser ut som förra veckan. Återkommer med en utförlig rapport på höstterminsmötet. 
 
\subp{D}{Kåren informerar}{\info}
Ivar meddelade att det är Arkad och att det kommer vara påtagligt.

Ivar informerade att valet för FM har öppnat. I nuläget finns det 27 sökande till 27 platser. Rösta så sektionen får designa studentkortet!

Ivar informerade också att nya heltidare har valts men att Nollegeneral fortfarande söks.  

\subp{E}{Omvärldsrapport}{\info}

Mattias informerade att Uppsala Studentförening Elektroteknik har sitt 5 års jubileum på fredag. 

\end{paragrafer}

\p{11}{Funktionärsvård}{\info}
Henrik informerade att han har gjort en sammanställning av hur funktionärsvårdsbudgeten har använts.

Det är problematiskt hur vissa konstnader ska fördelas då vissa utkskott har funktionärsvård beskrivet i sin budgetpost. 

%Lösning är att skriva CM och Nollus funktionärsvård på respektive budgetposter, -> Som beskrivet. 
%Mötet diskuterade också vissa enskilda verifikat. 
%Diskussion om vissa enskilda verifikat.

\p{12}{Medaljutdelning på funktionärstacket}{\dis}

Edvard meddelade att vi har möjlighet att dela ut bidragsmedalj under funktionärstacket.

Jakob föreslog att man kan dela ut det under sektions Nobelsittning. 

Mötet diskuterade vid vilken tidpunkt medaljen kan delas ut. 

Arkad-Sanna kom in och ville ha rummet. Styrelsen hävdar starkt att de har bokat. 

Henrik meddelade att sektionen gav ett diplom till CM 2018, och att det är ett alternativ till att visa uppskattning till vårbalens projektgrupp.

Saga föreslog att vi kan ge blommor till vårbalens projektgrupp.  

Mötet bordlägger diskussionen till nästa styrelsemötet. 

\p{13}{Nästa styrelsemöte}{\bes}
\Mba nästa styrelsemöte ska äga rum 2019-11-19 12.10 i E:1123. 

\p{14}{Beslutsuppföljning}{\bes}
%Edvard \ypa stryka ''Projektfunktionär: Vårbal'' från Beslutsuppföljning. Liknande projekt uppmuntras.
%\Mbaby
%Davida \ypa skjuta upp ''Inköp av draghandtag till cykelvagn'' till nästa styrelsemöte.
%\Mbaby

Emil \ypa att stryka ''Inköp av tidningsställ i LED'' från Beslutsuppföljning. Han meddelade att inköp samt godkännande från huest är klart och att tidningstället nu finns i Blådörren.

\Mbaby

%Saga Åslund Fotovägg - Saga ska beställa ikväll. 

Theo \ypa stryka ''Inköp av ny lamineringsmaskin'' från Beslutsuppföljning. 

\Mbaby

\p{15}{Övrigt}{\dis}

Mötet diskuterade vad som ska göras i matväg inför sektionsmötet. 

Davida Åström meddelade att man inte ska glömma Instagram Takeover. 

\p{16}{Sammanfattning av mötet}{\info}
% \Fbs Följande beslut ströks från Beslutsuppföljningen, 
% \Fbsup
\Fbs ''Inköp av tidningsställ i LED'' samt ''Inköp av ny lamineringsmaskin''.

Y Nhi Pham valdes som Halvledare.

Jacob Forsell valdes som Årskurs E1-ansvarig.

\p{17}{OFMA}{\bes}
{\mo} förklarade mötet avslutat kl. 18.03
\end{paragrafer}

%\newpage
\hidesignfoot
\begin{signatures}{3}
\signature{\mo}{Mötesordförande}
\signature{\ms}{Mötessekreterare}
\signature{\ji}{Justerare}
\end{signatures}
\end{document}
