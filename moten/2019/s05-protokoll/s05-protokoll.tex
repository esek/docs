\documentclass[10pt]{article}
\usepackage[utf8]{inputenc}
\usepackage[swedish]{babel}

\def\mo{Edvard Carlsson}
\def\ms{Sonja Kenari}
\def\ji{Saga Åslund}
%\def\jii{}

\def\doctype{Protokoll} %ex. Kallelse, Handlingar, Protkoll
\def\mname{Styrelsemöte} %ex. styrelsemöte, Vårterminsmöte
\def\mnum{S05/19} %ex S02/16, E1/15, VT/13
\def\date{2019-02-20} %YYYY-MM-DD
\def\docauthor{\ms}

\usepackage{../e-mote}
\usepackage{../../../e-sek}

\begin{document}
\showsignfoot

\heading{{\doctype} för {\mname} {\mnum}}

%\naun{}{} %närvarane under
%\nati{} %närvarande till och med
%\nafr{} %närvarande från och med
\section*{Närvarande}
\subsection*{Styrelsen}
\begin{narvarolista}
\nv{Ordförande}{Edvard Carlsson}{E16}{}
\nv{Kontaktor}{Sonja Kenari}{E15}{}
%\nv{Förvaltningschef}{Henrik Ramström}{E16}{}
\nv{Cafémästare}{Jonathan Benitez}{E17}{}
\nv{Sexmästare}{Theo Nyman}{BME18}{}
\nv{Krögare}{Davida Åström}{BME17}{}
\nv{Entertainer}{Saga Åslund}{BME18}{}
\nv{SRE-ordförande}{Lina Samnegård}{BME16}{}
\nv{ENU-ordförande}{Jakob Pettersson}{E17}{}
\nv{Øverphøs}{Stephanie Bol}{BME17}{\nati{13}}
\end{narvarolista}


\subsection*{Ständigt adjungerande}
\begin{narvarolista}
%\nv{Sigillbevarare}{Matilda Horn}{BME18}{\nati{17}}
%\nv{}{}{}{}
%\nv{Kårrepresentant}{Jacob Karlsson}{}{\nafr{3}}
%\nv{Valberedningens ordförande}{Elin Magnusson}{}{}
%\nv{Skattmästare}{Daniel Bakic}{E15}{}
%\nv{Vice Krögare}{Klara Indebetou}{BME17}{}
%\nv{Vice Krögare}{Hjalmar Tingberg}{BME16}{}
\nv{Kårrepresentant}{Philip Johansson}{}{}
\nv{Kårrepresentant}{Anna Qvil}{}{}
\nv{Fullmäktigeledamot}{Magnus Lundh}{E15}{\nafr{12}}
%\nv{Chefredaktör}{Max Mauritsson}{BME16}{}
%\nv{Elektras Ordförande}{Elisabeth Pongratz}{}{}
%\nv{Inspektor}{Monica Almqvist}{}{}
%\nv{Valberedningens ordförande}{Axel Voss}{E15}{\nafr{11}}

\end{narvarolista}

%\begin{comment}
%\subsection*{Adjungerande}
%\begin{narvarolista}
%\nv{}{Alexander Wik}{BME17}{\nati{17}}
%\nv{Sigillbevarare}{Matilda Horn}{BME18}{}
%\nv{Sångförman}{Adam Belfrage}{BME17}{\nati{17}}
%\end{narvarolista}
%\end{comment}

\section*{Protokoll}
\begin{paragrafer}
\p{1}{OFMÖ}{\bes}
Ordförande {\mo} förklarade mötet öppnat kl.12.16.

\p{2}{Val av mötesordförande}{\bes}
{\valavmo}

\p{3}{Val av mötessekreterare}{\bes}
{\valavms}

\p{4}{Val av justeringsperson}{\bes}
{\valavj}

\p{5}{Godkännande av tid och sätt}{\bes}
{\tosg}

\p{6}{Adjungeringar}{\bes}
%Adam Belfrage adjungerades.{}
%Förnamn Efternamn adjungerades
\textit{Inga adjungeringar.}


\p{7}{Godkännande av dagordningen}{\bes}
%Theo \ypa lägga till sena handlingar till dagordningen.
Edvard \ypa lägga till §13 Kvällsmöte.

\Mbaby
%Fredrik \ypa att lägga till \S18b ``Teknikfokus utnyttjande av LED-café''.

%Föredragningslistan godkändes med yrkandet.
%Föredragningslistan godkändes med samtliga yrkanden.

\p{8}{Föregående mötesprotokoll}{\bes}
%\latillprot{S04/19}
\textit{\ingaprot}

\p{9}{Fyllnadsval och entledigande av funktionärer}{\bes}
\begin{fyllnadsval} %"Inga fyllnadsval." fylls i automatiskt
\fval{Axel Sandqvist}{Diod}


%\entl{Namn}{Post}
\end{fyllnadsval}

\p{10}{Rapporter}{}
\begin{paragrafer}
\subp{A}{Hur mår alla?}{\info}
Punkten protokollfördes ej.

\subp{B}{Utskottsrapporter}{\info}
Jonathan meddelar att det går bra för caféet och att kärleksmums var uppskattat. Semlor kommer finnas till fettisdagen! Cafeet gick bra under Teknikfokus med. 

Edvard berättar om FVUs verksamhet då Henrik inte var närvarande. Bokföring har gjorts som vanligt och saker som legat inne i Edekvata har till slut flyttats till EKEA då access funkar som vanligt igen. Hustomtarna har också fått koll på sina uppgifter för kommande tid.

InfU går det bra för. Planeringen och genomförandet av resan till Finland gick bättre än förväntat. Kick-offen är på fredag och ska förhoppningsvis gå bra. Sociala plattformar och hur sektionen syns utåt har utvärderats rejält och en förändring kommer påbörjas så snart som möjligt.

KM har haft sitt första gille och rensat sejdlar. Lite tråkiga nyheter för vad som hänt i huset under helgen utöver gillet som tyvärr kan hamna på KM. Gillet sålde mer än förväntat, men oväntade problem löstes snabbt på plats!

NollU har träffat sömmerska och haft Infomöte. Möten är planerade inför veckan och ØPK har haft möte angående Gasquesalen och E-sektionen får behålla sin lördag som NollEgasque efter lottning.

ENU har planerat kick-off och planerat en FED pub med de andra sektionerna, preliminärt datum är satt! Även lunch med en ingenjör är på angendan inför året.

NöjU är igång, en bowlingturnering bokad den 7e Mars. Sporta med E går bra och spelkvällarna rullar på. Planeringar inför UtEDischot är igång och UtEDischoansvarig kommer gå på ØPK möte och informera om eventet. 

Sexet har haft Teknikfokusbanketten och vilat efter den också. Väldigt nöjd med första sittningen i Gasque. Alla jobbare var väldigt duktiga! Annars har det rullat på som vanligt med administrativt arbete.

SRE håller på med CEQ-rapporter sen förra perioden. Her Tech Future har också fått tillräckligt med phaddrar för eventet vilket är otroligt kul!

\subp{C}{Ekonomisk rapport}{\info}
Ekonomin mår bra.

\subp{D}{Kåren informerar}{\info}
Kåren börjar ta fram verksamhetsplanen och budget för kommande år.
Mailsystemet är under underhållning. 
Posten inför valfullmäktige utlyses denna veckan! Vilka poster som finns syns på Kårens hemsida och valet stänger den 15/3.

\subp{E}{Utomlands rapport}{\info}
Edvard informerar om Finlandsresan. 
Vi har två platser på Teknologföreningens bal i Finland den 23/3 som vi jättegärna vill att 2 representanter åker till!
\end{paragrafer}

\p{11}{Hantering av social kanaler}{\dis}
Sonja informerar om att vi som sektion gått ifrån det ursprungliga modellen att bara ha en samlad kanal utåt som E-sektionen och att vi ska gå tillbaka till detta. Detta medför att utskott som har egna sidor kommer behöva ta ner detta och börja använda sig av E-sektionens officiella Facebook sida och hemsida. Bilder/media som lagts upp på andra sidor kommer hämtas hem, sparas och sen se till så att de kommer upp på E-sektionens Facebook. Undantag för detta är endast NöjUs ``Sporta med E''-grupp på Facebook samt Phøsets egna Instagram.

Varje utskott kommer behöva en utsedd PR-ansvarig eller bara utse utskottsordförande. Dessa kommer då få access till våra olika kanaler där det kommer kunna lägga upp event eller media på Instagram. 

Detta betyder att en del förändringar kommer behöva göras kring flera utskott och då gäller det att hålla kontakt med InfU för att tillsammans komma med andra lösningar.


\p{12}{Sexuella trakasserier - Funktionärsskiphtet}{\dis}
Styrelsen höll diskussion angående uppkomna händelser som förekommit under Funktionärsskiphtet. För att detta ska åtgärdas informerar Kåren om att är det problem kring denna situationen så finns hjälp från Kåren samt Kuratorer. 

Styrelsen kommer diskutera punkten vidare vid nästa styrelsemöte för att komma fram till en konkret lösning och ett bra underlag för resten av året samt kommande år.

\p{13}{Kvällmöte}{\dis}
Styrelsen beslutar att ha ett kvällsmöte tisdagen den 26/2 där välmående och KPL kan diskuteras. Mötet kommer ej protokollföras.

\p{14}{Nästa styrelsemöte}{\bes}
\Mba nästa styrelsemöte ska äga rum 2019-02-27 kl.12.10 i E:1123.

\p{15}{Beslutsuppföljning}{\bes}
\textit{Inga beslut att följa upp.}


\p{16}{Övrigt}{\dis}
Det har talats om en styrelsesittning i OK, förslagen är 5/4, 4/5 och 9/5. Det sista alternativet fungerar inte för styrelsen. 


Påminnelse från PH, det måste städas ordentligt i E-huset efter alla event, sen även över toaletterna. Alla ytterdörrar måste vara stängda och om någon branddörr i källaren har öppnats måste de stängas igen. 

Stadgar stadfästes informerar Magnus!

\p{17}{OFMA}{\bes}
{\mo} förklarade mötet avslutat kl. 13.06.
\end{paragrafer}

%\newpage
\hidesignfoot
\begin{signatures}{3}
\signature{\mo}{Mötesordförande}
\signature{\ms}{Mötessekreterare}
\signature{\ji}{Justerare}
\end{signatures}
\end{document}
