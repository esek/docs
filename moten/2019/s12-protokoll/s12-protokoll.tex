\documentclass[10pt]{article}
\usepackage[utf8]{inputenc}
\usepackage[swedish]{babel}

\def\mo{Edvard Carlsson}
\def\ms{Sonja Kenari}
\def\ji{Jakob Pettersson}
%\def\jii{}

\def\doctype{Protokoll} %ex. Kallelse, Handlingar, Protkoll
\def\mname{Styrelsemöte} %ex. styrelsemöte, Vårterminsmöte
\def\mnum{S12/19} %ex S02/16, E1/15, VT/13
\def\date{2019-05-06} %YYYY-MM-DD
\def\docauthor{\ms}

\usepackage{../e-mote}
\usepackage{../../../e-sek}

\begin{document}
\showsignfoot

\heading{{\doctype} för {\mname} {\mnum}}

%\naun{}{} %närvarane under
%\nati{} %närvarande till och med
%\nafr{} %närvarande från och med
\section*{Närvarande}
\subsection*{Styrelsen}
\begin{narvarolista}
\nv{Ordförande}{Edvard Carlsson}{E16}{}
\nv{Kontaktor}{Sonja Kenari}{E15}{}
\nv{Förvaltningschef}{Henrik Ramström}{E16}{}
\nv{Cafémästare}{Jonathan Benitez}{E17}{}
\nv{Sexmästare}{Theo Nyman}{BME18}{}
\nv{Krögare}{Davida Åström}{BME17}{}
\nv{Entertainer}{Saga Åslund}{BME18}{}
\nv{SRE-ordförande}{Lina Samnegård}{BME16}{}
\nv{ENU-ordförande}{Jakob Pettersson}{E17}{}
\nv{Øverphøs}{Stephanie Bol}{BME17}{}
\end{narvarolista}


\subsection*{Ständigt adjungerande}
\begin{narvarolista}
%\nv{Sigillbevarare}{Matilda Horn}{BME18}{\nati{17}}
%\nv{}{}{}{}
%\nv{Kårrepresentant}{Jacob Karlsson}{}{\nafr{3}}
%\nv{Valberedningens ordförande}{Elin Magnusson}{}{}
\nv{Skattmästare}{Daniel Bakic}{E15}{\nafr{10}}
%\nv{Vice Krögare}{Klara Indebetou}{BME17}{}
%\nv{Vice Krögare}{Hjalmar Tingberg}{BME16}{}
\nv{Kårrepresentant}{Filip Johansson}{}{}
\nv{Kårrepresentant}{Anna Qvil}{}{}
%\nv{Valberedningens ordförande}{Axel Voss}{E15}{\nafr{10b}}
%\nv{Fullmäktigeledamot}{Magnus Lundh}{E15}{\nafr{12}}
%\nv{Chefredaktör}{Max Mauritsson}{BME16}{}
%\nv{Elektras Ordförande}{Elisabeth Pongratz}{}{}
%\nv{Inspektor}{Monica Almqvist}{}{}
%\nv{Valberedningens ordförande}{Axel Voss}{E15}{\nafr{11}}

\end{narvarolista}

%\begin{comment}
\subsection*{Adjungerande}
\begin{narvarolista}
%\nv{post}{namn}{klass}{nati/nafr/tom}
\nv{Redaktör}{Elin Johansson}{BME16}{}
%\nv{Projektfunktionär}{Sophia Carlsson}{BME17}{}
%\nv{Projekfunktionär}{Emma Hjörneby}{BME17}{}
%\nv{}{}{}{}
\end{narvarolista}
%\end{comment}

\section*{Protokoll}
\begin{paragrafer}
\p{1}{OFMÖ}{\bes}
Ordförande {\mo} förklarade mötet öppnat kl.12.12.

\p{2}{Val av mötesordförande}{\bes}
{\valavmo}

\p{3}{Val av mötessekreterare}{\bes}
{\valavms}

\p{4}{Val av justeringsperson}{\bes}
{\valavj}

\p{5}{Godkännande av tid och sätt}{\bes}
{\tosg}

\p{6}{Adjungeringar}{\bes}
%Adam Belfrage adjungerades.{}
Elin Johansson adjungerades. \\


%\textit{Inga adjungeringar.}


\p{7}{Godkännande av dagordningen}{\bes}
%Theo \ypa lägga till sena handlingar till dagordningen.
Stephanie \ypa lägga till punkten ``Tavlor i Diplomat'' som §14. \\
Edvard \ypa lägga till punkten ``Väggmålning'' som §15. \\
%Davida \ypa lägga till punkten ``Lophtet'' till dagordningen.\\
%Edvard \ypa lägga till punkten ``Ordensband'' til dagordningen.
%Fredrik \ypa att lägga till \S18b ``Teknikfokus utnyttjande av LED-café''.
%Jonathan \ypa ändra punkten §12 från att vara en beslutspunkt till diskussion. \\
%Föredragningslistan godkändes med yrkandet.
Föredragningslistan godkändes med samtliga yrkanden.

%Dagordningen godkändes.


\p{8}{Föregående mötesprotokoll}{\bes}
\latillprot{S09/19 och S11/19}
%\textit{\ingaprot}

\p{9}{Fyllnadsval och entledigande av funktionärer}{\bes}
\begin{fyllnadsval} %"Inga fyllnadsval." fylls i automatiskt
%\fval{Moa Rönnlund}{Halvledare}
\fval{Jakob Pettersson}{Diod}
\fval{Adam Belfrage}{Diod}
\fval{Edvard Carlsson}{Diod}
\fval{Davida Åström}{Diod}
%\entl{Namn}{Post}
\end{fyllnadsval}

\p{10}{Rapporter}{}
\begin{paragrafer}
\subp{A}{Hur mår alla?}{\info}
Punkten protokollfördes ej.

\subp{B}{Utskottsrapporter}{\info}
CM behöver dioder under de sista veckorna nu innan sommaren.

FVU går det bra för.

KM håller i FED pub på tisdag och har därmed inget Gille på fredag.

InfU går det bra för.

NollU har filmat färdigt och har temasläpp på onsdag.

ENU har haft massa möten och skickat iväg kontrakt till företag. Nästa vecka är det Lunch med en Ingenjör som man kan anmäla sig till. Det saknas också lite hjälp till specialiseringsminglet.

NöjU taggar att det är tandemvecka. Framöver är det planeringar inför året som gäller.

E6 har planerat temasläppet tillsammans med A- och D-sex.

SREs verksamhet rullar på med möten och CEQ.

\subp{C}{Ekonomisk rapport}{\info}
Henrik rapporterar om att läget är bra och undrar om det är mer specifika saker som styrelsen vill ha reda på under punkten framöver. Styrelsen informerar om att man gärna har utskottsspecifika budgetuppdateringar.

\subp{D}{Kåren informerar}{\info}
Kåren söker fortfarande en Utbilningsansvarig för externa frågor som är en heltidarpost.

Imorgon är det Fullmäktigemöte på Kåren.

Nästa vecka är det 35-års jubileum vilket kommer bli roligt!
\subp{E}{Omvärldsrapport}{\info}
Delar av styrelsen berättade om sina upplevelser från Vårbalen på KTH.

\end{paragrafer}

\p{11}{Bonsai Campus}{\dis}
Styrelsen diskuterar om Bonsai Campus och hur det ska gå till framöver. Stephanie ska höra på sitt kollegiemöte om vad som bestäms och informera vidare. 


\p{12}{Ulla}{\dis}
Jonathan informerar om hur det står till om att Ulla vill komma tillbaka. Styrelsen diskuterade för- och nackdelar. Då caféet känner att verksamheten börjar gå bra sedan förändringen att sköta caféet helt ideellt så ledde diskussionen mot att behålla det som vi har det nu och inte gå tillbaka till att ha en anställd i LED.

\p{13}{Styrelsen i NollEguiden}{\dis}
Styrelsen beslutar att ses för att ta bilder till NollEguiden nästa måndag.

\p{14}{Tavlor i Diplomat}{\dis}
Styrelsen diskuterar vad man vill hänga upp på väggen i Diplomat. En fotovägg hade varit trevligt. Styrelsen ska tänka över vad de vill ha för bilder tills nästa möte.

\p{15}{Väggmålning}{\dis}
Edvard vill att vi ska göra en väggmålning i trappuppgången utanför Edekvata med Hacke på. Picasso ska få i uppgift att ta fram en bild och målning löser sig senare.

\p{16}{Nästa styrelsemöte}{\bes}
\Mba nästa styrelsemöte ska äga rum 2019-05-14 kl.12.10 i E:1123.

\p{17}{Beslutsuppföljning}{\bes}

Edvard \ypa stryka ``Inköp av vattendunkar'' från beslutsuppföljningen.
\Mbaby

\p{18}{Övrigt}{\dis}
Davida informerade angående alkoholtillstånd samt att hon och Theo ska på möte med Tillståndsenheten med sitt kollegie. Styrelsen fick möjligheten att bifoga frågor som de kan ta upp på mötet.

Theo frågade om marknadsföring. 
\p{19}{OFMA}{\bes}
{\mo} förklarade mötet avslutat kl. 12.54.
\end{paragrafer}

%\newpage
\hidesignfoot
\begin{signatures}{3}
\signature{\mo}{Mötesordförande}
\signature{\ms}{Mötessekreterare}
\signature{\ji}{Justerare}
\end{signatures}
\end{document}
