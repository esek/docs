\documentclass[10pt]{article}
\usepackage[utf8]{inputenc}
\usepackage[swedish]{babel}

\def\mo{Johan Westerlund}
\def\ms{Erik Månsson}
\def\ji{Olle Oswald}
\def\jii{David Uhler Brand}

\def\doctype{Protokoll} %ex. Kallelse, Handlingar, Protkoll
\def\mname{Vårterminsmöte} %ex. styrelsemöte, Vårterminsmöte
\def\mnum{VT/16} %ex S02/16, E1/15, VT/13
\def\date{2016-04-20} %YYYY-MM-DD
\def\docauthor{\ms}

\usepackage{../e-mote}
\usepackage{../../e-sek}

\begin{document}
\showsignfoot

\heading{{\doctype} för {\mname} {\mnum}}

%\naun{}{} %närvarane under
%\nati{}{} %närvarande till och med
%\nafr{}{} %närvarande från och med
\section*{Närvarande}
\subsection*{Styrelsen}
\begin{narvarolista}
    \nv{Ordförande}{Fredrik Peterson}{E14}{}
    \nv{Kontaktor}{Erik Månsson}{E14}{}
    \nv{Förvaltningschef}{Anders Nilsson}{E13}{Frånvarande under \S22:C) - \S23:A)}
    \nv{Cafémästare}{Stephanie Mirsky}{E13}{}
    \nv{Øverphøs}{Molly Rusk}{BME14}{\nati{22:D)}}
    \nv{SRE-ordförande}{Johan Persson}{E13}{}
    \nv{ENU-ordförande}{Johannes Koch}{E13}{}
    \nv{Sexmästare}{Martin Gemborn Nilsson}{E14}{}
    \nv{Krögare}{Malin Lindström}{BME14}{}
    \nv{Entertainer}{Dalia Khairallah}{E15}{Frånvarande under \S22:C) - \S23:A)}
\end{narvarolista}

\vspace{-0.25\baselineskip}
\subsection*{Ständigt adjungerande}
\begin{narvarolista}
    \nv{Talman}{Johan Westerlund}{E11}{}
    \nv{Revisor}{Hanna Nevalainen}{E12}{\nati{23:J)}}
    \nv{Inspektor}{Monica Almqvist}{}{}
\end{narvarolista}

\vspace{-0.25\baselineskip}
\subsection*{Medlemmar}
\begin{narvarolista}
    \nv{}{Anna Lagerquist}{BME15}{}
    \nv{}{Jennifer Hagman}{BME13}{\nati{22:D)}}
    \nv{}{Elin Magnusson}{BME12}{}
    \nv{}{Pontus Landgren}{E14}{}
    \nv{}{Johan Karlberg}{E14}{\nati{22:D)}}
    \nv{}{Josefine Sandström}{E14}{}
    \nv{}{Sophia Grimmeiss Grahm}{BME14}{}
    \nv{}{Viktor Persson}{E14}{\naun{9}{22:A)}}
    \nv{}{Christian Benson}{E14}{\naun{9}{22:A)}}
    \nv{}{Elin Branzell}{BME14}{\naun{9}{22:A)}}
    \nv{}{Linnea Hellholm}{BME14}{\naun{9}{22:A)}}
    \nv{}{Emil Nilén}{BME14}{\nati{21:A)}}
    \nv{}{Alexander Najafi}{E12}{\nati{22:D)}}
    \nv{}{Henrik Felding}{E12}{\nati{22:D)}}
    \nv{}{Sofia Karlén}{BME13}{\nati{22:D)}}
    \nv{}{Olle Oswald}{BME14}{}
    \nv{}{Andreas Sirenius}{E14}{}
    \nv{}{Daniel Johansson}{E14}{}
    \nv{}{David Uhler Brand}{E14}{}
    \nv{}{Niklas Karlsson}{E14}{\nafr{9}}
    \nv{}{Adam Waks}{E13}{\nati{22:C)}}
    \nv{}{Nicklas Norborg Persson}{E14}{\nafr{23:A)}}
    \nv{}{Markus Rahne}{BME15}{\nafr{23:A)}}
\end{narvarolista}

\newpage

\section*{Protokoll}
\begin{paragrafer}
\p{1}{TaFMÖ}{}
Talman {\mo} förklarade mötet öppnat 17:14.

\p{2}{Val av mötesordförande}{}
Talman {\mo} valdes.

\p{3}{Val av mötessekreterare}{}
Kontaktor {\ms} valdes.

\p{4}{Godkännande av tid och sätt}{}
Tid och sätt godkändes.

\p{5}{Val av två justeringspersoner}{}
Alexander Najafi nominerades men godtog inte nomineringen.

Olle Oswald nominerade sig själv.

David Uhler Brand nominerade sig själv.

\valavj

\p{6}{Adjungeringar}{}
\ingaadj

\p{7}{Godkännande av dagordningen}{}
Molly Rusk \ypa flytta punkt \S23:I) till \S23:0), d.v.s. till först av alla propositioner.

Föredragningslistan godkändes med yrkandet.

\p{8}{Föregående sektionsmötesprotokoll}{}
\latillprot{E01/16}

\p{9}{Meddelanden}{}
Molly Rusk sa att på fredag är det temasläpp.

Martin sa att den 4:e maj är det sittning.

\p{10}{Beslutsuppföljning}{}

Johan Westerlund \ypa stryka ``E-wiki'' från beslutsuppföljningen.

\textbf{\Mba bifalla yrkandet.}

Fredrik Peterson pratade om hur det går med ``Upprustning av E-husets yttre fasadskylt''. Skylten är offererad och på gång att beställas.

Johan Westerlund \ypa tilldela om ``Upprustning av E-husets yttre fasadskylt'' till Fredrik Peterson. %TODO

\textbf{\Mba bifalla yrkandet.}

\p{11}{Utskottsrapporter}{}
\emph{Ingen hade några kommentarer på utskottsrapporterna.}

\p{12}{Uppföljning av verksamhetsplan}{}
Fredrik Peterson berättade allmänt om uppföljandet av verksamhetsplanen.

\p{13}{Ekonomisk rapport}{}
Anders Nilsson gav en rapport för Sektionens ekonomi.

\p{14}{Val}{}
\begin{paragrafer}
    \subp{A}{Val av funktionärer}{}
    \textbf{\Mba vakantsätta posten Teknikfokusansvarig.}\par
    \textbf{Mötet valde Pontus Landgren (elt14pla) som Vice SRE-ordförande.}\par
    \textbf{\Mba vakantsätta posten Kinainriktningsansvarig.}\par
    \textbf{\Mba vakantsätta posten Specialiseringsansvarig.}\par
    \textbf{\Mba vakantsätta posten Studiekvällsansvarig.}

    \subp{B}{Val av inspektor}{}
    \textbf{Mötet valde Monica Almqvist som inspektor.}

\end{paragrafer}

\p{15}{Verksamhetsberättelser för 2015}{}
\emph{Ingen hade några kommentarer på utskottsrapporterna.}

\p{16}{Bokslut för 2015}{}
Henrik Felding, Förvaltningschef 2015, berättade om bokslutet 2015. Revisorernas kommentarer står i revisionsberättelsen.

Anders Nilsson Cafémästare 2015, kommenterade CM:s diff från budget och vilka åtgärder man tagit vid.

\p{17}{Revisionsberättelse för 2015}{}
Johan Westerlund läste upp revisionsberättelsen och revisorernas yrkanden. Anders Nilsson presenterade resultatdispositionen.

\textbf{\Mba bifalla att resultat- och balansräkning fastställes.}

\textbf{\Mba bifalla att styrelsens förslag till resultatdisposition tages.}

\textbf{\Mba bifalla att styrelsen för 2015 bevaras ansvarsfrihet.}

\p{18}{Styrelsens förslag till resultatdisposition}{}
\emph{Denna punkt behandlades tidigare under mötet.}

\p{19}{Uttag ur sektionens fonder sedan förra terminsmötet}{}
Anders Nilsson berättade om uttagen ur Sektionens fonder sedan förra terminsmötet.

\p{20}{Frågan om ansvarsfrihet för 2015}{}
    \begin{paragrafer}
        \subp{A}{Funktionärer}{}
        \textbf{\Mba finna funktionärerna 2015 ansvarsfria.}

        \subp{B}{Utskott}{}
        \textbf{\Mba finna utskotten 2015 ansvarsfria.}

        \subp{C}{Styrelse}{}
        \emph{Denna punkt behandlades tidigare under mötet.}

        \subp{D}{Revisorer}{}
        \textbf{\Mba finna revisorerna 2015 ansvarsfria.}

        \subp{E}{Valberedning}{}
        \textbf{\Mba finna valberedningen 2015 ansvarsfria.}

    \end{paragrafer}
\p{21}{Stadgeändringar i andra läsningen}{}
  \begin{paragrafer}
    \subp{A}{Ändring av antalet firmatecknare}{}
        Henrik Felding presenterade propositionen.

        \textbf{\Mba bifalla propositionen i sin helhet.}

        Molly \ypa flytta punkt \S23:0) till \S21.5, d.v.s. efter ``Stadgeändringar i andra läsningen.''.

        \textbf{\Mba bifalla yrkandet.}

    \subp{B}{Revidering av stadgar och reglemente ångående SRE}{}

    Sofia Karlén presenterade motionen.

    \textbf{\Mba bifalla motionen i sin helhet med samtliga yrkanden.}

\end{paragrafer}
\p{21.5}{Uppfräschning av gamla arkivet}{}
    Anders Nilsson presenterade propositionen.

    \textbf{\Mba bifalla propositionen i sin helhet.}

\p{22}{Behandling av motioner}{}
    \begin{paragrafer}
      \subp{A}{Välja Teknikfokusansvarig på Vårterminsmötet}{}
      Jennifer Hagman presenterade motionen.

      Fredrik Peterson presenterade styrelsens svar.

      \textbf{\Mba bifalla motionen i sin helhet.}

      \subp{B}{Inköp av backuplösning till Sektionens servrar}{}

      Daniel Johansson presenterade motionen.

      Fredrik Peterson presenterade styrelsens svar.

      \textbf{\Mba bifalla motionen i sin helhet.}

      \subp{C}{Motkandidering till styrelseposter}{}
      Elin Magnusson presenterade motionen.

      Fredrik Peterson presenterade styrelsens svar.

      \textbf{\Mba bifalla motionen i sin helhet.}

      Johan Westerlund \ypa ta matpaus efter \S22:D).

      \textbf{\Mba bifalla yrkandet.}

      \subp{D}{Utökning av E-sektionens alumniverksamhet}{}
      Elin Magnusson presenterade motionen.

      Fredrik Peterson presenterade styrelsens svar.

      \textbf{\Mba bifalla motionen i sin helhet.}

      Mötet ajournerades.

      Mötet återupptogs 18:30.

  \end{paragrafer}
\p{23}{Behandling av propositioner}{}
    \begin{paragrafer}
      \subp{A}{Införandet av riktlinjer och uppdatering av policybeslut}{}
      Erik Månsson presenterade propositionen.

      \textbf{\Mba bifalla propositionen i sin helhet.}

      \subp{B}{Förflyttandet av policys till riktlinjer}{}
        Erik Månsson presenterade propositionen.

        \textbf{\Mba bifalla propositionen i sin helhet.}

      \subp{C}{Uppdatering av Kontaktor- och Ordförandeposten samt införandet av en Vice Kontaktor}{}
        Erik Månsson presenterade propositionen.

        \textbf{\Mba bifalla propositionen i sin helhet.}

      \subp{D}{Ändringar i FvU:s struktur}{}
        Anders Nilsson presenterade propositionen.

        \textbf{\Mba bifalla propositionen i sin helhet.}

        Johan \ypa ta en snabb paus på 5 min.

        \textbf{\Mba bifalla yrkandet.}

        Mötet ajournerades.

        Mötet återupptogs 18:52.

      \subp{E}{Förflyttning av Teknokrater och Ekiperingsexperter}{}
        Erik Månsson presenterade propositionen.

        \textbf{\Mba bifalla propositionen i sin helhet.}

      \subp{F}{Uppdatering av mandatperiod för FpT-ansvarig}{}
          Martin Gemborn Nilsson presenterade propositionen.

          \textbf{\Mba bifalla propositionen i sin helhet.}

      \subp{G}{Inköp av ny spisfläkt till Edekvataköket}{}
          Anders Nilsson presenterade propositionen.

          \textbf{\Mba bifalla propositionen i sin helhet.}

      \subp{H}{Renovering och ombyggnad av HK och BD}{}
        Fredrik Peterson presenterade propositionen.

        \textbf{\Mba bifalla propositionen i sin helhet.}

      \subp{J}{Uppfräschning av Diplomat}{}
        Johannes Koch presenterade propositionen.

        Elin Magnusson tyckte att sofforna såg hårda ut. Johannes svarade att styrelsen har provat dem och de är sköna.

        \textbf{\Mba bifalla propositionen i sin helhet.}

  \end{paragrafer}
\p{24}{Övrigt}{}
Martin Gemborn Nilsson frågade hur det går för Ducks i slutspelet, varpå Johan Westerlund svarade med stor entusiasm.

David Uhler Brand vill tacka för maten.

\p{25}{TaFMA}{}
Talman {\mo} förklarade mötet avslutat 19:23.

\end{paragrafer}

%\newpage
\hidesignfoot
\begin{signatures}{4}
\signature{\mo}{Mötesordförande}
\signature{\ms}{Mötessekreterare}
\signature{\ji}{Justerare 1}
\signature{\jii}{Justerare 2}
\end{signatures}
\end{document}
