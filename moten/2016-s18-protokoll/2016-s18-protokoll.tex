\documentclass[10pt]{article}
\usepackage[utf8]{inputenc}
\usepackage[swedish]{babel}

\def\mo{Fredrik Peterson}
\def\ms{Erik Månsson}
\def\ji{Anders Nilsson}

\def\doctype{Protokoll} %ex. Kallelse, Handlingar, Protkoll
\def\mname{styrelsemöte} %ex. styrelsemöte, Vårterminsmöte
\def\mnum{S18/16} %ex S02/16, E1/15, VT/13
\def\date{2016-09-15} %YYYY-MM-DD
\def\docauthor{\ms}

\usepackage{../e-mote}
\usepackage{../../e-sek}

\begin{document}
\showsignfoot

\heading{{\doctype} för {\mname} {\mnum}}

%\naun{}{} %närvarane under
%\nati{} %närvarande till och med
%\nafr{} %närvarande från och med
\section*{Närvarande}
\subsection*{Styrelsen}
\begin{narvarolista}
\nv{Ordförande}{Fredrik Peterson}{E14}{}
\nv{Kontaktor}{Erik Månsson}{E14}{}
\nv{Förvaltningschef}{Anders Nilsson}{E13}{}
\nv{Cafémästare}{Stephanie Mirsky}{E13}{}
\nv{Øverphøs}{Molly Rusk}{BME14}{}
\nv{Sexmästare}{Martin Gemborn Nilsson}{E14}{}
\nv{Krögare}{Malin Lindström}{BME14}{\nati{15}}
\nv{Entertainer}{Dalia Khairallah}{E15}{}
\end{narvarolista}

\subsection*{Ständigt adjungerande}
\begin{narvarolista}
\nv{Kårordförande}{Linus Hammarlund}{}{}
\nv{Kårrepresentant}{Jacob Karlsson}{}{}
\end{narvarolista}

\section*{Protokoll}
\begin{paragrafer}
\p{1}{OFSÖ}{\bes}
Ordförande {\mo} förklarade mötet öppnat 12:14.

\p{2}{Val av mötesordförande}{\bes}
\valavmo

\p{3}{Val av mötessekreterare}{\bes}
\valavms

\p{4}{Tid och sätt}{\bes}
\tosg

\p{5}{Adjungeringar}{\bes}
\ingaadj

\p{6}{Val av justeringsperson}{\bes}
\valavj

\p{7}{Föredragningslistan}{\bes}
Föredragningslistan godkändes.

\p{8}{Föregående mötesprotokoll}{\bes}
\latillprot{S16/16}

\p{9}{Fyllnadsval/Entledigande av funktionärer}{\bes}
\begin{fyllnadsval} %"Inga fyllnadsval." fylls i automatiskt
\end{fyllnadsval}

\p{10}{Rapporter}{}
\begin{paragrafer}
\subp{A}{Check in}{\info}
Stephanie sa att LED-café går bra. De har fått in lite nollor som jobbar. Hon frågade om någon var i LED över helgen.

Martin har haft mycket att göra, men det har lugnat ner sig nu efter qasqueinbjudningarna har givits ut.

Dalia har hållt volleybollkväll som blev lyckad. Hon ska hålla Phadderolympiaden nästa vecka.

Molly sa att de går bra för NollU. Alla stora grejor med nollningen är nu klara.

Malin sa att det går bra för KM. De ska hålla sista nollningsgillet på fredag.

Anders har bokfört i mängder.

Erik mår bra och börjar bli lite mindre nollesjuk.

Fredrik har fixat lite småsaker och har dömt regattan som gick ``bra''.

\subp{B}{Kåren informerar}{\info}
Jacob sa att de har haft ett kort kårstyrelsemöte. Kåren har fortfarande problem efter serverkrashen. Det är FM nästa vecka.

\subp{C}{Ekonomi}{\info}
Anders sa att halvårsbokslutet börjar närma sig. Om någon har saker från i våras måste de lämnas in senast i veckan efter qasque för att kunna betalas ut.

\end{paragrafer}

\p{11}{Inköp av dammsugare}{\bes}
Fredrik sa att det är jättedyrt att köpa delar till våran gamla dammsugare, ungefär 750kr. Istället föreslår han att köpa in en ny dammsugare (Kärcher T 7/1 Classic) för 1200kr, eller eventuellt en bättre modell.

Mötet diskuterade och kom fram till att i alla fall köpa en ny dammsugare.

\emph{Punkten senarelades.}

\p{12}{Riktlinje för körersättning}{\bes}
Fredrik presenterade sitt förslag på en ny riktlinje som ersätter en gammal policy.

\textbf{\Mba att antaga den föreslagna riktlinjen \textit{Körersättningsriktlinje}.}

\p{13}{Nästa styrelsemöte}{\bes}
\Mba nästa styrelsemöte ska äga rum 2016-09-22 12:10 i E:1426.

\p{14}{Beslutsuppföljning}{\bes}
\Ibfu

\p{15}{Övrigt}{\dis}
Dalia berättade vad som händer på phadderolympiaden. Hon tycker att det hade varit trevligt om styrelsen var där och antingen var med och tävlade eller hejade på.

Stephanie sa att vi kommer få nya sponsorkoppar som vi ska använda i två veckor framåt.

Fredrik sa att vi behöver planera vårt spex till nolleqasquen. Styrelsen kom överrens om att göra detta på måndag.

Fredrik sa att vi bör planera in städdagar/städkvällar.

Dalia sa att den 8:e oktober är det paintball med alla andra sektioner förutom M och I.

Fredrik sa att man brukar dela ut bidragsmedaljer på qasquen, och uppmanade alla att fundera på vem som kan få en.

\p{16}{OFSA}{\bes}
{\mo} förklarade mötet avslutat 12:45.

\end{paragrafer}

%\newpage
\hidesignfoot
\begin{signatures}{3}
\signature{\mo}{Mötesordförande}
\signature{\ms}{Mötessekreterare}
\signature{\ji}{Justerare}
\end{signatures}
\end{document}
