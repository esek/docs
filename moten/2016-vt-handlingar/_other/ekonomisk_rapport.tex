\documentclass[../_main/handlingar.tex]{subfiles}

\begin{document}
\section{Ekonomisk rapport}

E-sektionens ekonomi mår bra och sektionen har en mycket god förmåga att betala alla sina kortfristiga skulder. Som sektion har vi i nuläget inga långfristiga skulder så som banklån eller liknande.

Sektionens tillgångar uppgår till \SI{837758.43}{kr} enligt bokfört underlag t.o.m. 12/4 2016 då lager exkluderas. En stor del av dessa tillgångar är öronmärkta och ligger i Sektionens fiktiva fonder. De pengarna som ligger i fonderna är avsedda för olika ändamål så som inköp av ny utrustning, reparationer eller som ersättning vid skador med mera.

Det bör tilläggas att det skulle vara mycket oklokt att spendera samtliga medel i fonderna  eftersom det skulle lämna Sektionen ekonomiskt sårbar om vi skulle drabbas av en större oförutsedd utgift.

På nästa sida ser ni en balansrapport för Sektionen där ni kan se aktuella tillgångar och skulder. Dagsaktuella värden på våra bakkonton respektive handkassan samt övriga tillgångar redovisas på mötet.

\begin{signatures}{1}
    \mvh
    \signature{Anders Nilsson}{Förvaltningschef}
\end{signatures}

\end{document}
