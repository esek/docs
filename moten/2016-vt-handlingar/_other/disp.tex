\documentclass[../_main/handlingar.tex]{subfiles}

\begin{document}

\section{Förslag på resultatdisposition av resultatet för 2015}
Sektionens överskott för 2015 är \SI{194441.64}{kr}. Styrelsen anser att det egna kapitalet måste ökas för att Sektionen inte ska drabbas av likviditetsproblem om stora uttag ur fonderna görs. På sikt bör Sektionens egna kapital ökas ännu mer för att undvika att tvingas ta lån ur fonderna för att klara den dagliga verksamheten. Med anledning av detta föreslår styrelsen att avsätta \SI{60000}{kr} till eget kapital. För att fortsätta bygga upp jubileumsfonden så kostnaden för kommande jubileum inte ska belasta endast ett år föreslår styrelsen att avsätta \SI{15000}{kr} till jubileumsfonden. Resterande överskott fördelas mellan Styrelsen dispositionsfond ($\sim50\%$) och utrustningsfonden ($\sim50\%$) samt en utjämning till olycksfonden för att fonden ska få ett jämnt belopp.

\subsubsection{Förslag på resultatdisposition}
\begin{tabular}{l r}
    Jubileumsfonden & \SI{15000.00}{kr} \\
    Olycksfonden & \SI{2989.10}{kr} \\
    Dispositionsfonden & \SI{60000.00}{kr} \\
    Eget kapital & \SI{60000.00}{kr} \\
    Utrustningsfonden & \SI{56452.54}{kr} \\
    \hline
    \textbf{Summa} & \SI{194441.64}{kr} \\
\end{tabular}

\begin{signatures}{2}
    \ist
    \signature{Anders Nilsson}{Förvaltningschef 2016}
    \signature{Henrik Felding}{Förvaltningschef 2015}
\end{signatures}

\end{document}
