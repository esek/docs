\documentclass[../_main/handlingar.tex]{subfiles}

\begin{document}
\verksplan{2016}
År 2015 infördes en verksamhetsplan för Sektionen.

Syftet med verksamhetsplanen är att skapa en tydligare struktur och mer operativt långsiktigt arbete. Verksamhetsplanen innehåller övergripande mål för Sektionen och delmål för styrelsen och sektionens utskott för nästkommande år. Verksamhetens mål skall redovisas på varje terminsmöte.

\emph{Budgeten och verksamhetsplanen ska komplettera varandra}

\subsubsection*{E-sektionen}
\emph{E-sektionen ska verka för att:}
\begin{dashlist}
    \item främja medlemmarnas studietid
    \item alla medlemmar trivs
    \item alla medlemmars åsikter görs hörda
    \item sektionens verksamhet präglas av demokrati och transparens
    \item ha en ständigt framåtsträvande verksamhet
    \item varje funktionärspost har ett ansvar och kan göra skillnad
    \item hålla en god och ansvarsfull ekonomi
    \item anpassa sektionens verksamhet efter medlemmarnas behov
    \item medlemmarna ska ha en god gemenskap
    \item inkludera alla medlemmar
\end{dashlist}

\subsubsection*{Styrelsen}
Sektionen har en stabil ekonomisk grund. Det som behöver arbetas med är framför allt att utvärdera den befintliga verksamheten och anpassa den efter den efterfrågan som finns från sektionens medlemmar. Styrelsen behöver arbeta för långsiktiga mål såsom sektionens lokaler, främst Edekvatas kök, som är i behov av renovering. Nuvarande och framtida samarbete med andra sektioner och kåren behöver utvärderas samt utvecklas.

\emph{Delmål 2016:}
\begin{dashlist}
    \item utvärdera styrelsens och sektionens struktur
    \item utvärdera Teknikfokus struktur och roll i sektionen
    \item ta fram en långsiktig plan för renovering av Edekvatas lokaler
    \item undersöka alternativa möjligheter för renhållning av sektionens lokaler
    \item synliggöra styrelsen samt styrelsens roll i verksamheten
    \item tydliggöra och förbättra samarbetet mellan utskottsordföranden
    \item arbeta för att alla funktionärer tackas på ett bra sätt
    \item arbeta för fler intersektionella samarbeten under året
    \item att genomföra en större funktionärsutbildning under våren
    \item förbättra överlämningen till nästkommande styrelse
    \item undersöka kontantfria betalningslösningar för sektionens verksamheter
    \item ta fram en ekonomisk vision som sträcker sig minst tre år framåt
\end{dashlist}

\subsubsection*{Förvaltningsutskottet}
Sektionens medlemmar ska ha tillgång till fräscha uppehållslokaler ämnade både för studier och studiesociala aktiviteter. I nuläget prioriteras ekonomin alltid högre än sektionens lokaler. Målet ska vara att se till att underhåll av Edekvata görs löpande utan att ekonomin prioriteras ned.

Sektionens medlemmar ska kunna ta del av de ekonomiska beslut som fattas av styrelsen på ett lätt sätt. Ekonomin ska skötas på ett sådant sätt att medlemsnyttan maximeras samt att planera sektionens ekonomi långsiktigt.

\emph{Delmål 2016:}
\begin{dashlist}
    \item ta fram en plan för löpande underhåll av Edekvata samt implementera ett system så att underhåll utförs kontinuerligt
\end{dashlist}

\subsubsection*{Studierådet}
Studierådet ska verka för att vara ett synligt utskott. Arbetet skall göras mer tillgängligt för sektionens medlemmar för att visa förändringar som har genomförts. Utskottet ska verka för att medlemmarna är medvetna om deras möjligheter att påverka och förbättra sin utbildning.

\emph{Delmål 2016:}
\begin{dashlist}
    \item arbeta för att synliggöra utskottets arbete och resultat till sektionens medlemmar
    \item anordna pluggkvällar kontinuerligt under hela året samt utvärdera deras struktur och syfte
    \item arbeta för att ha minst en representant från varje årskurs, inklusive årskurs fyra och fem
    \item arbeta för att öka svarsfrekvensen på CEQ-enkäterna
    \item arbeta för att bibehålla en god relation med Programledningarna och Studievägledningen
\end{dashlist}

\subsubsection*{Sexmästeriet}
Sexmästeriet har god kvalité på sina sittningar. I dagens läge anordnas de flesta sittningar under nollningen och därför behöver möjligheterna ses över ifall det går att anordna arrangemang för sektionens medlemmar under våren.

\emph{Delmål 2016:}
\begin{dashlist}
    \item undersöka möjligheterna att fördela mängden evenemang jämt under året
    \item arbeta för att förbättra ordningen i sexmästeriets förråd
    \item arbeta för att hålla de gemensamma arbetsområdena i ordning
\end{dashlist}

\subsubsection*{Cafémästeriet}
Cafémästeriet har en bra fungerande verksamhet. Arbeta för att bibehålla LED-cafés konkurrenskraft och studentvänliga priser.

\emph{Delmål 2016:}
\begin{dashlist}
    \item arbeta för en minskning av svinn i LED samt i Cafémästeriets förråd
    \item utvärdera mojternas funktion och deras behov
    \item utvärdera och utveckla LED:s sortiment
\end{dashlist}

\subsubsection*{Nöjesutskottet}
Det senaste året har utskottet arrangerat många nya aktiviteter. Därigenom har ansvaret ökat och ansvarsfördelningen inom utskottet behöver utvärderas. Utskottet har bra förutsättningar för att kunna förbättra sammanhållningen med andra sektioner genom aktiviteter.

\emph{Delmål 2016:}
\begin{dashlist}
    \item utveckla posternas roll samt beskrivning av dessa
    \item utvärdera utförandet av UtEDischot
    \item öka mångfalden av aktiviteter som anordnas av utskottet
\end{dashlist}

\subsubsection*{Nolleutskottet}
Sektionens nollning har en bra struktur där det senaste har varit att arbeta med ekonomin och föra samman E- och BME-programmen mer. På senare år har sektionen fått ta mer ansvar för de internationella studenterna. I och med detta behöver planeringen anpassa för att kunna integrera dem mer i sektionen.

\emph{Delmål 2016:}
\begin{dashlist}
    \item arbeta för att integrera internationella studenter i sektionen
    \item arbeta för en mångfald av aktiviteter
    \item arbeta för att få fram förslag och förbättringar kring nollningen från sektionens medlemmar
    \item ta fram en plan för renhållning av de sektionslokaler som används under nollningen
\end{dashlist}

\subsubsection*{Informationsutskottet}
Kontakten med andra organisationer har utökats och förbättrats. Genom att utbyta information om gemensamma situationer kan samarbetet utvecklas. Information som ska nå sektionsmedlemmar ska publiceras i alla informationskanaler.

\emph{Delmål 2016:}
\begin{dashlist}
    \item arbeta för att bibehålla relationen med andra organisationer
    \item uppdatera sektionens datorsystem och hemsida
    \item utveckla alumniverksamheten
    \item utvärdera sektionens informationskanaler, hur de bör användas och vidta åtgärder
\end{dashlist}

\subsubsection*{Näringslivsutskottet}
Utskottet har en stor mångfald av aktiviteter som har främjat sektionens medlemmar på arbetsmarknaden. Fördelningen av företagen som samarbetat med sektionen har speglat sektionens utbildningar mer.

\emph{Delmål 2016:}
\begin{dashlist}
    \item bibehålla samarbetet med befintliga företag samt aktivt söka nya samarbeten med organisationer och företag som är relevanta för sektionens medlemmar
    \item utvärdera om ett mentorsprogram kan vara relevant för studenterna på sektionen
    \item bibehålla samarbete med befintliga företag och aktivt söka nya samarbeten med organisationer och företag som är relevanta för sektionens medlemmar
    \item se över prissättning på de aktiviteter som erbjuds i förhållande till andra sektioners prissättning
    \item utvärdera hur evenemang ligger utspridda under året
\end{dashlist}

\subsubsection*{Källarmästeriet}
Utskottet fungerar mycket väl och är bra strukturerat. Källarmästeriet ska fortsätta vara konkurrenskraftiga gentemot det stora utbudet som finns i Lund. Utskottet ska se över möjligheterna att utöka marknadsföringen till den egna och andra sektioner.

\emph{Delmål 2016:}
\begin{dashlist}
    \item att marknadsföra arrangemang för sektionens medlemmar samt medlemmar i TLTH
    \item utvärdera ansvarsfördelningen samt arbetsbördan inom utskottet
\end{dashlist}

\newpage
\end{document}
