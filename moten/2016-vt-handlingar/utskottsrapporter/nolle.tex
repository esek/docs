\documentclass[../_main/handlingar.tex]{subfiles}

\begin{document}
\utskottsrapport{Nolleutskottet}

Sedan i januari har vi i Phøset gått ett antal heldagsutbildningar om bl.a. mottagande och värdegrund, team-building och ledarskap samt intervjuteknik. Vi har umgåtts en hel helg med andra phøsgrupper på TLTH för att lära oss hur nollning och engagemang ser ut på andra sektioner och hur man kan göra E-sektionens mottagning ännu bättre.

Vi har haft intervjuer med, och valt grupphaddrar, uppdragsphaddrar och internationella phaddrar. Vi har även jobbat tillsammans med studierådet för att i år ha pluggphaddrar under nollningen för att främja nollornas studier under mottagningen.

Andra utskott vi har jobbat med under våren är bl.a. NöjU för att planera aktiviteter och evenemang under nollningen, InfU för att göra nollningshemsidan samt nollEguiden, Näringslivsutskottet för att få spons till nollningen och även Sexmästeriet och KM. Vi är jättenöjda med hur mycket folk på sektionen vill hjälpa till och vara med i nollningen 2016 och jobbar för att ha en god relation med Sektionens alla utskott.

Vi jobbar just nu för fullt med att planera temasläppet nu på fredag vars eftersläpp hålls tillsammans med alla andra sektioner på TLTH. Vi har till detta även blivit klara med delar av våra kläder till nollningen.

Vi har jobbat mycket med att göra ändringar till de evenemang och uppdrag som tidigare år inte varit lika uppskattade som andra och har försökt att få med aktiviteter för alla. Bland annat har vi infört ett uppdrag som är mer inriktat till nollor med sportintresse och ett annat uppdrag för folk som tycker om gåtor.

Vi har även infört ett antal nya evenemang under nollningen med hjälp av andra utskott som vi tror kommer bli jätteroliga för alla som går på dem! Planeringen kommer att fortsätta för fullt under våren och sommaren och vi ser fram emot en fantastisk nollning!

\begin{signatures}{1}
    \mvh
    \signature{Molly Rusk}{Øverphøs}
\end{signatures}

\end{document}
