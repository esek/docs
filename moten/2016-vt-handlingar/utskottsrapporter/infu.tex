\documentclass[../_main/handlingar.tex]{subfiles}

\begin{document}
\utskottsrapport{Informationsutskottet}

HeHE-redaktionen har jobbat på som förr. Eftersom vi har en del som varit med förr har det varit lätt för dem att fortsätta med sitt arbete med HeHE. HeHE-redaktionen har kommit med ett förslag om en pappersupplaga, vilket vi förhoppningsvis kommer få se under året.

DDG har fått två engagerade Macapärer som alltid arbetar för att alla Kodhackare ska ha något att jobba på. Själva har de jobbat på att få sektionens backup-system i ordning, att fixa E-vote för andra sektioner, och att se över förslag från styrelsen. Några Kodhackare har sett till att få upp vår nya Nollningshemsida.

Picasson har hjälpt till lite här och där med att få upp snygga affischer och bilder på Ekoli. Ekiperingsexpertera har kommit in i deras nya post och jobbar nu på eventuella utskottstygmärken. My, vår gamla Lastgamla, har tagit på sig uppdraget att stå för alumniverksamheten i år igen. Vår fotograf har fotat styrelsens julkort samt hjälpt ENU med CV-fotografering som gick riktigt bra. Utskottet har tillsammans fått fram en arbetsgrupp som arbetar med NollEguiden tillsammans med Phøset.

Jag själv som Kontaktor har jobbat med mycket blandade grejer. På våra möten har jag försökt skriva bra utförliga protokoll och få ut dem i tid. Jag har också tagit fram nya mallar för sektionens alla dokument. Handlingarna till detta mötet är helt omgjorda och förhoppningsvis enklare att läsa. Tillsammans med resten av styrelsen har jag diskuterat fram en proposition om att införa en ``Vice Kontaktor'' för att någon ska kunna lägga mycket tid på utskottet i sig, samt några andra propositioner. Jag har också uppdaterat lite här och där på hemsidan, sett till att vi har WiFi för iZettle i Edekvata, och skapat en Instagram för Sektionen.

\begin{signatures}{1}
    \mvh
    \signature{Erik Månsson}{Kontaktor}
\end{signatures}

\end{document}
