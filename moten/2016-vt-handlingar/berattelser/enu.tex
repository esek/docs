\documentclass[../_main/handlingar.tex]{subfiles}

\begin{document}
\berattelse{Näringslivsutskottet 2015}

2015 har ENU haft tre arbetsgrupper; företagskontakterna som arbetar utåt mot företagen, PR som marknadsför utskottet och event samt eventgruppen som ser till att eventen blir av. Ordförande är ansvarig för PR och företagskontakter medan Vice Ordförande är ansvarig för event.

Eventgruppen skapade tidigt en kick-off för utskottet som blev väldigt lyckat med inslag av bland annat middag, skattjakt och lekar. De har även arbetat med lunchföreläsningar och att laga mat till dessa.

PR har jobbat för att göra marknadsföring inför bland annat CV-granskningen med Academic Work, affischer och reklam inför nollningsaktiviteter och även tagit fram en ENU-tröja för att sprida information om vårt arbete.

Företagskontakterna har kontaktat många företag för att sprida intresset från sektionens studenter och många har även kontaktat ENU för fortsatt samarbete.

Under Teknikfokus 2015 var det 23 medverkande företag, vilket är rekordmånga. Dessutom var det 2 medicintekniska företag närvarande, vilket aldrig tidigare skett heller. Väldigt kul tycker ENU!

Företagskontakterna skapade initiala kontakter med alla företag och några nya företag är intresserade av att inleda samarbete med oss studenter på E-sektionen.

Under nollningen blev ACTIC huvudsponsor i form av tryck på tröjor, lunchföreläsning samt besök i E-foajén. SVEP hade traditionsenligt SVEP robotic challange under första nollningsveckan, Tetra pak hade en lunchföreläsning, Unionen stod i foajén, Altran hade spel-gille, E.ON hade en pluggdag, ARM och Academic Work har stått i foajén. Min Doktor hade en riktat föreläsning till BME, SVEP höll i en branchkväll och E.ON bjöd på lunch och produktspons under instuderingsveckan.

Målet för ENU-budgeten 2015 nåddes med marginal och utskottet har arbetat mycket väl för att se till studenternas intresse.

\begin{signatures}{1}
    \mvh
    \signature{Lovisa Lundin}{ENU-ordförande 2015}
\end{signatures}

\end{document}
