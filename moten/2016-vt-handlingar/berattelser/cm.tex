\documentclass[../_main/handlingar.tex]{subfiles}

\begin{document}
\berattelse{Cafémästeriet 2015}

Under våren jobbade vi mycket med att utskottets funktionärer ska komma in i sina roller. De centrala delarna av utskottet hade lunchmöte varje tisdag för att samla utskottet och se vad som händer under veckan. Och vi jobbade med att få så många som mjöligt att jobba i caféet samt försökte utveckla och förnya vårt dagliga sortiment. Vi uppmärksammade även fettisdagen genom att serverade semlor (hela 200st), vilket var väldigt populärt

Under hösten fortsatt samma dagliga verksamhet men även lite annat. B.la. under nollningens första dag gjorde vi lunch till dåvarande nollor i form av pastasallader med styrelsens hjälp. Under nollningen fick även dåvarande nollorna fått testa på att jobba i LED och lära känna Ulla. Vi jobbade också med att ökat vår kontakt med våra leverantörer vilket har gjorde att vi fått lite bättre pris på delar av sortimentet. Efter nollningen fick även besök av miljö- och hälsovårdsmyndigheten där köket blev godkänt men dock var det lite papper och rutiner som de ville att vi reviderade. Under sommaren i samband med renoveringen slängdes 3 av de trasiga mojtarna medan flaskmojten skänktes till F, resterande mojtarna slängdes efter beslut från HT-mötet.

Något som bör nämnas är att CM för verksamhetsår 2015 slutade med ett dåligt resultat långt under förväntat. Orsaken till detta tros vara framförallt p.g.a. 3 faktorer, där den första är att Quixter la ner precis innan höstterminens början vilket beräknades orsaka stora intäktsreduceringar och där med vinst reducering. De andra orsakerna tros ligga i ökade kostnader från våra leverantörer som inte upptäckts men även differenser från lagerna som belastar CM. På grund av detta har vi jobbat med CM-16 för att hitta lösningar på problemet där bland att göra kvartalsrapporter efter var läsperiod för att få en bättre insyn i hur resultatet ligger till.

\begin{signatures}{1}
    \mvh
    \signature{Anders Nilsson}{Cafémästare 2015}
\end{signatures}

\end{document}
