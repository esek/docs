\documentclass[../_main/handlingar.tex]{subfiles}

\begin{document}
\verksplanuppf{VT 2016}

\subsubsection*{Informationsutskottet}
Informationsspridningen på Sektionen har fungerat bra och vi upplever att vi når ut till alla. Vi tittar just nu på att uppdatera datorerna i HK, fixa backupen och utvärdera vad vi ska göra med switchen i HK. Vi har en diskussion om hur vi ska utveckla Alumniverksamheten. En utvärdering av informationskanaler kommer senare under året.

Nu i början av året har Kontaktorn haft en del kontakt med andra sektioner och KTH, och siktar på att fortsätta kontakten med fler sektioner och skolor senare under året.

\subsubsection*{Källarmästeriet}
Vi har arbetat mycket med marknadsföring av gillen. Vi har fortsatt att sätta upp planscher i E-huset och information på TV-skärmarna. Vi har dessutom satt upp planscher i de andra husen, varit med i kårnytt och börjat göra evenemang på facebook, till de flesta arrangemang som anordnas. Det verkar ha gett god effekt då fler från andra sektioner har hittat till gillena. Dessutom jobbas det för att ha fler gillen med andra sektioner som indirekt marknadsför gillena.

Ingen speciell utvärdering av ansvarsfördelningen har gjort hittills, men kommer göras så småningom. Dock fungerar strukturen bra som den är nu.

\subsubsection*{Nolleutskottet}
Vi har under vårt arbete hittills jobbat mycket med att få en mångfald av aktiviteter. Detta har gjorts genom att införa nya samt ändra gamla uppdrag och evenemang som passar flera olika personer. För att öka spridningen på aktiviteter för nollorna har mycket av arbetet gjorts tillsammans med andra utskott vilket gett nya möjligheter i arbetet.

Vi har tagit även kontakt med Sektionens världsmästare för att jobba för att bättre integrera internationella studenter under nollningen och kommer ha våra internationella phaddrar till hjälp för detta.

Vi har via dels det gamla phøset men även andra på Sektionen fått respons från vad folk tyckt om tidigare års nollningar och tagit denna feed-back till oss för att göra nollningen 2016 så bra den bara kan bli.

En renhållningsplan för lokalerna under nollningen har ännu inte hunnits med men vi kommer att titta noggrannare på detta under våren innan nollningen börjar.

\subsubsection*{Cafémästeriet}
Cafémästeriet jobbar kontinuerligt för att minska svinn så att så lite som möjligt ska kastas och detta kommer fortgå. Likaså gäller utvecklingen av LED:s sortiment, det jobbas kontinuerligt med detta, eventuellt kommer vi åka till en så kallad inspirationsmässa från vår leverantör. Mojtarna har inte arbetats med då dessa inte längre finns kvar.

\subsubsection*{Förvaltningsutskottet}
Förvaltningsutskottet har arbetat med styrelsen för en kortsiktig plan för Edekvata men en direkt långsiktig plan saknas, istället har vi valt att lägga fram en proposition som föreslår hur lokalerna ska kunna jobbas med i längden.

\subsubsection*{Studierådet}
Studierådet fortsätter arbetet med att förbättra synligheten på Sektionen, med stor fokus på att visa upp oss under nollningen så att nya studenter varför vi finns och vad vi kan göra. Vi arbetar även med att ha fler pluggkvällar jämt utspridda under året och att under nollningen ändra på pluggkvällarnas struktur för att ge bättre studiero. Diskussioner om hur svarsfrekvensen på CEQ ska ökas hålls.

\subsubsection*{Sexmästeriet}
Att de de flesta sittningar är under nollningen beror mest på att Sexmästeriet har ganska mycket uppstartsjobb då det är ett stort utskott och att de personer som i störta utsträckning ska ska driva utskottet framåt har mycket att lära och ska på ett antal utbildningar innan det går smidigt och enkelt att hålla i evenemang. Många event som t.ex. Skipthet, Teknikfokus och Flickor på Teknis ligger även under våren vilket tar en del fokus från att hålla sittningar för Sektionen. Även det stora problemet med att hitta datum som passar bra har sannolikt en stor inverkan på saken i fråga.

Den 4/1 anordnade E6 tillsammans med 5 andra sexmästerier en stor sittning för sektions-medlemen. E6 har även för avsikt att under resterande delen av våren, om datumen tillåter, försöka fokusera lite mer på evenemang direkt riktade mot E-sektions-medlemen. Det blir förhoppningsvis en sittning 4/5 och även ett mindre samarbete med Nöju finns på ritbordet.

Att hålla rent i Pump är något som E6 bör fortsätta och förbättra sig på att göra. Mitt bland alla andra event är dock detta något som enklet glöms bort. Tyvärr är det lite ont om plats i Pump och det står endel saker på golvet, dock är den ruttna löken är i alla fall utkastad! I hyllorna är det dock ganska bra ordning, tacka E6-15 för det!

E6 har ännu inte hållt i något större event i Edekvata. Dock har vi använt köket endel och till största del städat bra efter oss. De gånger det inte har varit perfekt gjort har det i störta mån gjort i efterhand. En sak som E6 skulle kunna bli bättre på är att tömma diskstället.

\newpage
\begin{signatures}{10}
    \mvh
    \signature{Fredrik Peterson}{Ordförande}
    \signature{Erik Månsson}{Kontaktor}
    \signature{Anders Nilsson}{Förvaltningschef}
    \signature{Stephanie Mirsky}{Cafémästare}
    \signature{Molly Rusk}{Øverphøs}
    \signature{Johan Persson}{SRE-ordförande}
    \signature{Johannes Koch}{ENU-ordförande}
    \signature{Martin Gemborn Nilsson}{Sexmästare}
    \signature{Malin Lindström}{Krögare}
    \signature{Dalia Khairallah}{Entertainer}
\end{signatures}

\end{document}
