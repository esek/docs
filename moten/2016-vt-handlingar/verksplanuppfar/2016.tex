\documentclass[../_main/handlingar.tex]{subfiles}

\begin{document}
\verksplanuppf{2016}

\subsubsection*{Informationsutskottet}
Informationsspridningen på sektionen har fungerat bra och vi upplever att vi når ut till alla. Vi tittar just nu på att uppdatera datorerna i HK, fixa backupen och utvärdera vad vi ska göra med switchen i HK. Vi har en diskussion om hur vi ska utveckla Alumniverksamheten. En utvärdering av informationskanaler kommer senare under året.

Nu i början av året har Kontaktorn haft en del kontakt med andra sektioner och KTH, och siktar på att fortsätta kontakten med fler sektioner och skolor senare under året.

\subsubsection*{Källarmästeriet}
Vi har arbetat mycket med marknadsföring av gillen. Vi har fortsatt att sätta upp planscher i E-huset och information på TV-skärmarna. Vi har dessutom satt upp planscher i de andra husen, varit med i kårnytt och börjat göra evenemang på facebook, till de flesta arrangemang som anordnas. Det verkar ha gett god effekt då fler från andra sektioner har hittat till gillena. Dessutom jobbas det för att ha fler gillen med andra sektioner som indirekt marknadsför gillena.

Ingen speciell utvärdering av ansvarsfördelningen har gjort hittills, men kommer göras så småningom. Dock fungerar strukturen bra som den är nu.

\subsubsection*{Nolleutskottet}
Vi har under vårt arbete hittills jobbat mycket med att få en mångfald av aktiviteter. Detta har gjorts genom att införa nya samt ändra gamla uppdrag och evenemang som passar flera olika personer. För att öka spridningen på aktiviteter för nollorna har mycket av arbetet gjorts tillsammans med andra utskott vilket gett nya möjligheter i arbetet.

Vi har tagit även kontakt med sektionens världsmästare för att jobba för att bättre integrera internationella studenter under nollningen och kommer ha våra internationella phaddrar till hjälp för detta.

Vi har via dels det gamla phøset men även andra på sektionen fått respons från vad folk tyckt om tidigare års nollningar och tagit denna feed-back till oss för att göra nollningen 2016 så bra den bara kan bli.

En renhållningsplan för lokalerna under nollningen har ännu inte hunnits med men vi kommer att titta noggrannare på detta under våren innan nollningen börjar.

\newpage
\begin{signatures}{10}
    \mvh
    \signature{Fredrik Peterson}{Ordförande}
    \signature{Erik Månsson}{Kontaktor}
    \signature{Anders Nilsson}{Förvaltningschef}
    \signature{Stephanie Mirsky}{Cafémästare}
    \signature{Molly Rusk}{Øverphøs}
    \signature{Johan Persson}{SRE-ordförande}
    \signature{Johannes Koch}{ENU-ordförande}
    \signature{Martin Gemborn Nilsson}{Sexmästare}
    \signature{Malin Lindström}{Krögare}
    \signature{Dalia Khairallah}{Entertainer}
\end{signatures}

\end{document}
