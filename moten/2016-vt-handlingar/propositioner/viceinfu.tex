\documentclass[../_main/handlingar.tex]{subfiles}

\begin{document}
\proposition{Införandet av posten Vice InfU-ordförande}

Kontaktorn har just nu tre huvudsakliga ansvarsområden - att agera sekreterare, att vara ordförande för Informationsutskottet (InfU), och att sköta kontakten med andra sektioner och skolor. Nuvarande Kontaktorn (Erik Månsson) tillsammans med resten av styrelsen anser att detta leder till att Kontaktorn får fokusera på antingen styrelsearbetet eller InfU, och att det blir svårt att hinna med båda på sättet man kanske hade velat.

Många andra sektioner har delat upp posten i två, med en sekreterare som är administrativt ansvarig och en InfU-ordförande som är informationsansvarig. Detta är något som vi i framtiden funderar på att anamma, så nu föreslår vi att införa en \emph{Vice InfU-ordförande} som hjälper Kontaktorn i arbetet med InfU. Vi väljer att göra detta eftersom vi vill utvärdera en eventuell uppdelning av Kontaktorposten ska se ut.

Därför yrkar styrelsen på
\begin{attsatser}
    \att under \S10:2:F i reglementet under punkten om Kontaktorn lägga till:\par
    {\it
    Vice InfU-ordförande (1)
    \begin{itemize}[label={--}, topsep=0cm, noitemsep]
        \item bistår ordförande för Informationsutskottet i dennes arbete med utskottet.
        \item ansvarar för att hålla information på våra hemsidor aktuell.
        \item hjälper Macapärerna att leda Kodhackarna.
        \item hjälper Chefredaktören att leda HeHE-redaktionen.
    \end{itemize}
    }
    \att ändringen sker med omedelbar verkan.
    \att under pågående verksamhetsår (2016) låta styrelsen fyllnadsvälja posten.
    \att Kontaktorn ska ge sin utvärdering av posten på Höstterminsmötet 2016.
\end{attsatser}

\begin{signatures}{1}
    \ist
    \signature{Erik Månsson}{Kontaktor}
\end{signatures}

\end{document}
