\documentclass[../_main/handlingar.tex]{subfiles}

\begin{document}
\proposition{Införandet av riktlinjer och uppdatering av policybeslut}

Rutinen för vissa administrativa uppgifter styrelsen har skulle varit bra att föra in i Sektionens styrdokument, till exempel priser för uthyrning av lokaler eller hantering av handkassan. Dock blir det väldigt konkreta saker som i allmänhet är irrelevanta för sektionens medlemmar, för att skrivas i reglementet eller som en policy. Därför vill styrelsen införa så kallade ``Riktlinjer'' som beslutas av styrelsen.

Detta leder till att vi får bättre uppdaterade styrdokument, eftersom styrelsen lätt kan uppdatera riktlinjerna. Dessutom gör det att överlämningen till nästa styrelse blir enklare.

För att det ska bli enklare att bläddra bland styrdokumenten vill vi också flytta ut policies från reglementet till separata dokument. Detta gör ingen skillnad för dess betydelse, utan gör det bara enklare att administrera och läsa.

Därför yrkar styrelsen på
\begin{attsatser}
    \att i reglementet lägga till ett nytt kapitel:\par
    {\it
    \vspace*{-\baselineskip}
    \section*{Kapitel 17 - Riktlinjer}
    En riktlinje är ett konkret direktiv på hur ett utskott eller styrelsen ska arbeta och/eller agera i olika situationer. Riktlinjer får ej säga emot övriga styrdokument.

    Riktlinjer kan endast ändras eller tagas bort på styrelsemöte eller Sektionsmöte. Antagna riktlinjer införs automatiskt i listan nedan. Själva riktlinjen ska ligga på Sektionens hemsida tillgänglig för allmänheten.

    Vid ändringar av riktlinjer ska alla funktionärer samt berörda personer informeras. Den sittande styrelsen bör undvika att uppdatera riktlinjerna mer än en gång per år, och uppdateringen bör ske i slutet av året ifall inte speciella behov finns.

    Styrelsen har antagit följande riktlinjer:
    \begin{dashlist}
        \item {[...]}
    \end{dashlist}
    }
    \att skriva om Kapitel 16 i reglementet till:\par
    {\it
    \vspace*{-\baselineskip}
    \section*{Kapitel 16 - Policybeslut}
    Policybeslut kan endast antas, ändras eller tas bort på Sektionsmöte. Antagna policybeslut införs automatiskt i listan nedan. Själva policybeslutet ska ligga på Sektionens hemsida tillgänglig för allmänheten.

    Sektionen har antagit följande policybeslut:
    \begin{dashlist}
        \item {[...]}
    \end{dashlist}
    }
\end{attsatser}

\begin{signatures}{1}
    \ist
    \signature{Erik Månsson}{Kontaktor}
\end{signatures}

\end{document}
