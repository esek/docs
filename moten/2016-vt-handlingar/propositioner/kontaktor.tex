\documentclass[../_main/handlingar.tex]{subfiles}

\begin{document}
\proposition{Uppdatering av Kontaktor- och Ordförandeposten samt införandet av en Vice Kontaktor}
Kontaktorn har just nu tre huvudsakliga ansvarsområden - att agera sekreterare, att vara ordförande för Informationsutskottet (InfU), och att sköta kontakten med andra sektioner och skolor. Nuvarande Kontaktorn tillsammans med resten av styrelsen anser att detta leder till att Kontaktorn får fokusera på antingen styrelsearbetet eller InfU, och att det blir svårt att hinna med båda på sättet man kanske hade velat.

Många andra sektioner har delat upp posten i två, med en sekreterare som är administrativt ansvarig och en InfU-ordförande som är informationsansvarig. Detta är något som vi i framtiden funderar på att anamma, så nu föreslår vi att införa en \emph{Vice Kontaktor} som hjälper Kontaktorn i arbetet med InfU. Vi väljer att göra detta eftersom vi vill utvärdera en eventuell uppdelning av Kontaktorposten ska se ut.

Tillsammans med detta vill vi också uppdatera Kontaktorposten samt Ordförandeposten så att det passar med rutinerna vi haft i år. Det handlar mest om att Kontaktorns sekreteraransvar nu också innefattar att skriva sektionsmöteshandlingarna och också ha koll på våra styrdokument. Fram till detta mötet har det fungerat bra och vi har infört en ny sida på hemsidan.

Styrelsen yrkar på
\begin{attsatser}
    \att under \S10:2:F i reglementet under punkten om Kontaktorn lägga till:\par
    {\it
    Vice Kontaktor (1)
    \begin{tightdashlist}
        \item bistår ordförande för Informationsutskottet i dennes arbete med utskottet.
        \item hjälper till att hålla informationen på våra hemsidor aktuell.
    \end{tightdashlist}
    }
    \att under pågående verksamhetsår (2016) låta styrelsen fyllnadsvälja posten.

    \newpage

    \att under \S10:2:F i reglementet ändra punkten om Kontaktorn från:\par
    {\it
    Kontaktorn (u)
    \begin{tightdashlist}
        \item har det övergripande ansvaret för Sektionens informationsspridning och PR-verksamhet och vad därmed äga sammanhang,
        \item är chefsredaktör för tidningen Elskaren,
        \item ansvarar för sektionens kontakt med TLTH och övriga sektioner inom TLTH,
        \item ansvarar för sektionens kontakt med andra högskolor och universitet.
        \item ansvarar för sektionens elektroniska fotoarkiv.
    \end{tightdashlist}
    }
    till:\par
    {\it
    Kontaktor (u)
    \begin{tightdashlist}
        \item är Sektionens sekreterare och har övergripande ansvar för Sektionens dokument och protokollföring av möten.
        \item ansvarar för att Sektionens stydokument hålls aktuella.
        \item ansvarar för att upprätta handlingar till Sektionsmötena.
        \item har det övergripande ansvaret för Sektionens informationsspridning och PR-verksamhet.
        \item ansvarar för Sektionens kontakt med andra sektioner, såväl inom TLTH som på andra högskolor och universitet.
    \end{tightdashlist}
    }

    \newpage

    \att under \S10:2:C i reglementet ändra från:\par
    {\it
    Det åligger Ordföranden\par
    \textbf{att} representera Sektionen och föra dess talan,\\
    \textbf{att} sammankalla och upprätta handlingar till Sektionsmöte,\\
    \textbf{att} tillsammans med Talmannen upprätta lämplig föredragningslista,\\
    \textbf{att} införa kallelse i HeHE enligt reglementet,\\
    \textbf{att} muntligen informera studenter i årskurs 1 och 2, som är ordinarie medlemmar i sektionen om Sektionsmötet i samband med en föreläsning,\\
    \textbf{att} tillsammans med Valberedningens Ordförande planera Expot och valmötet,\\
    \textbf{att} sammankalla och upprätta lämpliga handlingar till Styrelsesammanträden,\\
    \textbf{att} leda Styrelsesammanträdena och leda arbetet i Styrelsen,\\
    \textbf{att} organisera Kurs på landet,\\
    \textbf{att} kontinuerligt utbyta information med Inspektor,\\
    \textbf{att} aktivt deltaga i TLTH:s Fullmäktiges möten och föra Sektionens talan,\\
    \textbf{att} aktivt deltaga i TLTH:s Ordförandekollegie och utbyta information mellan Sektionerna,\\
    \textbf{att} tillse att Sektionen är representerad vid TLTH:s kårbal och andra Sektioners högtidligheter,\\
    \textbf{att} närvara vid Sektionens Nollegasque, skifte och Sektionsmöte,\\
    \textbf{att} städa efter skiftet, samt\\
    \textbf{att} ansvara för att det hålls en omsitts för ordförandeposten och en för posterna talman, sigillbevarare och revisorer gemensamt.
    }

    till:\par
    {\it
    Det åligger Ordföranden\par
    \textbf{att} representera Sektionen och föra dess talan,\\
    \textbf{att} sammankalla handlingar till Sektionsmöte,\\
    \textbf{att} tillsammans med Talmannen upprätta lämplig föredragningslista,\\
    \textbf{att} införa kallelse i HeHE enligt reglementet,\\
    \textbf{att} muntligen informera studenter i årskurs 1 och 2, som är ordinarie medlemmar i sektionen om Sektionsmötet i samband med en föreläsning,\\
    \textbf{att} tillsammans med Valberedningens Ordförande planera Expot och valmötet,\\
    \textbf{att} sammankalla och upprätta lämpliga handlingar till Styrelsesammanträden,\\
    \textbf{att} leda Styrelsesammanträdena och leda arbetet i Styrelsen,\\
    \textbf{att} organisera Kurs på landet,\\
    \textbf{att} kontinuerligt utbyta information med Inspektor,\\
    \textbf{att} aktivt deltaga i TLTH:s Fullmäktiges möten och föra Sektionens talan,\\
    \textbf{att} aktivt deltaga i TLTH:s Ordförandekollegie och utbyta information mellan Sektionerna,\\
    \textbf{att} tillse att Sektionen är representerad vid TLTH:s kårbal och andra Sektioners högtidligheter,\\
    \textbf{att} närvara vid Sektionens Nollegasque, skifte och Sektionsmöte,\\
    \textbf{att} städa efter skiphtet, samt\\
    \textbf{att} ansvara för att det hålls en omsitts för Ordförandeposten och en för posterna Talman, Sigillbevarare och Revisorer gemensamt.
    }
\end{attsatser}

\begin{signatures}{2}
    \ist
    \signature{Erik Månsson}{Kontaktor}
    \signature{Fredrik Peterson}{Ordförande}
\end{signatures}

\end{document}
