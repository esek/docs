\documentclass[10pt]{article}
\usepackage[utf8]{inputenc}
\usepackage[swedish]{babel}

\def\ordf{Fredrik Peterson}
\def\sekr{Erik Månsson}
\def\talm{Johan Westerlund}

\def\doctype{Handlingar} %ex. Kallelse, Handlingar, Protkoll
\def\mname{Vårterminsmötet} %ex. styrelsemöte, vårterminsmöte
\def\mnum{VT/16} %ex S02/16, E01/15, VT/13
\def\date{2016-04-20} %YYYY-MM-DD
\def\docauthor{\sekr}

\def\time{17:15}
\def\place{E:B}

\usepackage{../../sektion-handlingar/e-handlingar-sek}
\usepackage{../../e-mote}
\usepackage{../../../e-sek}

\begin{document}

\firstpage

\tableofcontents
\newpage

\section{Hälsning från Kontaktorn} %TODO
Hej alla E-sektionsmedlemmar!

I år har jag tagit tag i att designa/skriva om handlingarna till Sektionsmötena. Jag hoppas att du som läsare har glädje av detta, och att de som skrivit motioner och dylikt har tyckt det varit smidigt. Nytt för i år är också att handlingarna innehåller en kort guide till Sektionsmötena, som jag hoppas ska göra det enklare för nya medlemmar att vara med på mötet.

Har du något förslag till förbättringar av detta dokument eller andra saker gällande förberedelser inför mötet? Hör då gärna till mig och berätta!

\href{mailto:erikm@esek.se}{\texttt{erikm@esek.se}}

\begin{signatures}{1}
    \mvh
    \signature{\sekr}{Kontaktor}
\end{signatures}
\newpage

\section{En kort guide till Sektionsmötena}
Sektionsmötena är E-sektionens högsta beslutande organ, det vill säga att det är här är alla de största och viktigaste besluten under året klubbas igenom. Det är också på Sektionsmötena alla förtroendevalda och funktionärer väljs. För att alla ska kunna delta i mötet på lika villkor följer här en ordlista på de vanligaste och viktigaste begreppen.

\begin{description}[style=multiline, leftmargin=45mm]
    \item[Acklamation]
    Det vanligaste sättet att ta beslut, där röstningen sker muntligt. Talmannen frågar om mötet vill \emph{bifalla} det liggande förslaget, och de som vill det svarar ``ja''. Därefter frågar densamma om någon är emot bifall, och de som är emot svarar ``ja''. Man svarar alltså aldrig ``nej''. Det är därefter upp till Talmannen att avgöra vilket alternativ som överväger. Om reslutatet verkar osäkert kan man begära \emph{votering} innan klubban fallit och beslutet fastställs.
    \item[Adjungera]
    Att tillfälligt låta någon utanför Sektionen stå som medlem. Personen kommer få yttra sig och yrka, men inte rösta.
    \item[Ajournera]
    Att avbryta mötet för att senare återuppta det, till exempel för en matpaus eller bensträckare.
    \item[Ansvarsfrihet]
    \item[Avslag]
    Att inte godkänna ett förslag.
    \item[Bifall]
    Att godkänna ett förslag.
    \item[Bokslut]
    \item[Bokslutsdisposition]
    \item[Bordläggning]
    \item[Justering av protokoll]
    \item[Justering av röstlängd]
    \item[Jäv]
    \item[Motion]
    Ett förslag från en eller flera sektionsmedlemmar. I regel brukar styrelsen ge ett svar på vad de tycker om förslaget.
    \item[Ordningsfråga]
    \item[Proposition]
    Ett förslag från styrelsen.
    \item[Replik]
    \item[Reservation]
    \item[Sakupplysning]
    \item[Streck i debatten]
    \item[Votering]
    \item[Yrkande]
    Ett formellt förslag till beslut. Alla yrkande måste lämnas skriftligen eller på mail (\href{mailto:yrka@esek.se}{\texttt{yrka@esek.se}}) till Talmannen.
\end{description}

\section{Dagordning}
\subsection{Tid och plats}
\tidplats

\subsection{Föredragningslista}
\begin{paragrafer}
    \p{1}{TaFMÖ}{}
    \p{2}{Val av mötesordförande}{}
    \p{3}{Val av mötessekreterare}{}
    \p{4}{Tid och sätt}{}
    \p{5}{Val av två justeringspersoner}{}
    \p{6}{Adjungeringar}{}
    \p{7}{Föredragningslistan}{}
    \p{8}{Föregående sektionsmötesprotokoll}{}
    \p{9}{Meddelanden}{}
    \p{10}{Beslutsuppföljning}{}
    \p{11}{Utskottsrapporter}{}
    \p{12}{Ekonomisk rapport}{}
    \p{13}{Val}{}
        \begin{paragrafer}
            \subp{A}{Val av valberedningens ordförande}{}
            \subp{B}{Val av funktionärer}{}
            \subp{C}{Val av inspektor}{}
        \end{paragrafer}
    \p{14}{Verksamhetsberättelser för 2015}{}
    \p{15}{Bokslut för 2015}{}
    \p{16}{Revisorernas berättelse för 2015}{}
    \p{17}{Frågan om ansvarsfrihet för 2015}{}
        \begin{paragrafer}
            \subp{A}{Funktionärer}{}
            \subp{B}{Utskott}{}
            \subp{C}{Styrelse}{}
            \subp{D}{Revisorer}{}
            \subp{E}{Valberedning}{}
        \end{paragrafer}
    \p{18}{Styrelsens förslag till resultatdisposition}{}
    \p{19}{Uttag ur sektionens fonder sedan förra terminsmötet}{}
    \p{20}{Stadgeändringar i andra läsningen}{}
    \p{21}{Behandling av motioner}{}
    \p{22}{Behandling av propositioner}{}
        \begin{paragrafer}
            \subp{A}{Förflyttning av Teknokrater och Ekiperingsexperter}{}
        \end{paragrafer}
    \p{23}{Övrigt}{}
    \p{24}{TaFMA}{}
\end{paragrafer}

\begin{signatures}{2}
    \ist
    \signature{\ordf}{Ordförande}
    \signature{\sekr}{Kontaktor}
\end{signatures}

\section{Beslutsuppföljning}
\begin{busek}
    \beslutsek{VT/15}{Ett beslut}{Några personer}{...}
\end{busek}

\begin{utskottsrapporter}
    %\subfile{../utskottsrapporter/...}
\end{utskottsrapporter}

\begin{valforslags}
    %\subfile{../valforslag/...}
\end{valforslags}

\begin{berattelser}
    %\subfile{../berattelser/...}
\end{berattelser}

\begin{stadgeandringar}
    %\subfile{../stadgeandringar/...}
\end{stadgeandringar}

\begin{motioner}
    %\subfile{../motioner/...}
\end{motioner}

\begin{propositioner}
    \subfile{../propositioner/teknoekip}
    \subfile{../propositioner/revposter}
    \subfile{../propositioner/viceinfu}
    \subfile{../propositioner/vicefvc}
\end{propositioner}

\end{document}
