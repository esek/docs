\documentclass[../_main/handlingar.tex]{subfiles}

\begin{document}

\section{En kort guide till Sektionsmötena}
Sektionsmötena är E-sektionens högsta beslutande organ, det vill säga att det är här är alla de största och viktigaste besluten under året klubbas igenom. Det är också på Sektionsmötena alla förtroendevalda och funktionärer väljs. För att alla ska kunna delta i mötet på lika villkor följer här en ordlista på de vanligaste och viktigaste begreppen.

\begin{description}[style=multiline, leftmargin=45mm]
    \item[Acklamation]
    Det vanligaste sättet att ta beslut, där röstningen sker muntligt. Talmannen frågar om mötet vill \emph{bifalla} det liggande förslaget, och de som vill det svarar ``ja''. Därefter frågar densamma om någon är emot bifall, och de som är emot svarar ``ja''. Man svarar alltså aldrig ``nej''. Det är därefter upp till Talmannen att avgöra vilket alternativ som överväger. Om reslutatet verkar osäkert kan man begära \emph{votering} innan klubban fallit och beslutet fastställs.
    \item[Adjungera]
    Att tillfälligt låta någon utanför Sektionen stå som medlem. Personen kommer få yttra sig och yrka, men inte rösta.
    \item[Ajournera]
    Att avbryta mötet för att senare återuppta det, till exempel för en matpaus eller bensträckare.
    \item[Ansvarsfrihet]
    \item[Avslag]
    Att inte godkänna ett förslag.
    \item[Bifall]
    Att godkänna ett förslag.
    \item[Bokslut]
    \item[Bokslutsdisposition]
    \item[Bordläggning]
    \item[Justering av protokoll]
    \item[Justering av röstlängd]
    \item[Jäv]
    \item[Motion]
    Ett förslag från en eller flera sektionsmedlemmar. I regel brukar styrelsen ge ett svar på vad de tycker om förslaget.
    \item[Ordningsfråga]
    \item[Proposition]
    Ett förslag från styrelsen.
    \item[Replik]
    \item[Reservation]
    \item[Sakupplysning]
    \item[Streck i debatten]
    \item[Votering]
    \item[Yrkande]
    Ett formellt förslag till beslut. Alla yrkande måste lämnas skriftligen eller på mail (\href{mailto:yrka@esek.se}{\texttt{yrka@esek.se}}) till Talmannen.
\end{description}

\end{document}
