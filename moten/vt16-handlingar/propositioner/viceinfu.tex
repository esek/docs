\documentclass[../main/handlingar.tex]{subfiles}

\begin{document}
\proposition{Införandet av posten Vice InfU-ordförande}

Kontaktorn har just nu två huvudsakliga ansvarsområden - att agera sekreterare och att vara ordförande för Informationsutskottet. Nuvarande Kontaktorn (Erik Månsson) tillsammans med resten av Styrelsen anser att detta leder till att Kontaktorn får fokusera på antingen styrelsearbetet eller InfU, och att det blir svårt att hinna med båda på sättet man kanske hade velat.

Många andra sektioner har delat upp posten i två - en sekreterare och en InfU-ordförande. Detta är något som vi i framtiden funderar på att anamma, så nu föreslår vi att införa en \emph{Vice InfU-ordförande} som hjälper Kontaktorn i arbetet med InfU. Vi väljer att göra detta eftersom
\begin{itemize}[label={--}, topsep=0cm, noitemsep]
    \item en ändring av stadgar kan i praktiken inte träda i kraft förens nästa verksamhetsår.
    \item en reglementesändring, som propositionen gäller, kan verkställas direkt.
    \item vi vill utvärdera hur arbetsuppgifterna bör fördelas mellan de eventuella framtida posterna.
\end{itemize}

Därför yrkar Styrelsen på
\begin{attsatser}
    \att under \S10:2:F i reglementet under punkten om Kontaktorn lägga till:\par
    {\it
    Vice InfU-ordförande (1)
    \begin{itemize}[label={--}, topsep=0cm, noitemsep]
        \item bistår ordförande för InfU i dennes arbete med utskottet.
        \item ansvarar för att hålla information på våra hemsidor aktuell.
        \item hjälper Macapärerna att leda DDG.
        \item hjälper Chefredaktören att leda HeHE-redaktionen.
    \end{itemize}
    }
    \att ändringen sker med omedelbar verkan.
    \att under pågående verksamhetsår (2016) låta Styrelsen fyllnadsvälja posten.
    \att Kontaktorn ska ge en utvärdering av posten på Höstterminsmötet 2016.
\end{attsatser}

\begin{signatures}{1}
    \ist
    \signature{Erik Månsson}{Kontaktor}
\end{signatures}

\end{document}
