\documentclass[10pt]{article}
\usepackage[utf8]{inputenc}
\usepackage[swedish]{babel}

\def\mo{Fredrik Peterson}
\def\ms{Erik Månsson}
\def\ji{Molly Rusk}

\def\doctype{Protokoll} %ex. Kallelse, Handlingar, Protkoll
\def\mname{styrelsemöte} %ex. styrelsemöte, Vårterminsmöte
\def\mnum{S20/16} %ex S02/16, E1/15, VT/13
\def\date{2016-10-06} %YYYY-MM-DD
\def\docauthor{\ms}

\usepackage{../e-mote}
\usepackage{../../e-sek}

\begin{document}
\showsignfoot

\heading{{\doctype} för {\mname} {\mnum}}

%\naun{}{} %närvarane under
%\nati{} %närvarande till och med
%\nafr{} %närvarande från och med
\section*{Närvarande}
\subsection*{Styrelsen}
\begin{narvarolista}
\nv{Ordförande}{Fredrik Peterson}{E14}{}
\nv{Kontaktor}{Erik Månsson}{E14}{}
\nv{Förvaltningschef}{Anders Nilsson}{E13}{}
\nv{Cafémästare}{Stephanie Mirsky}{E13}{}
\nv{Øverphøs}{Molly Rusk}{BME14}{}
\nv{SRE-ordförande}{Johan Persson}{E13}{}
\nv{ENU-ordförande}{Johannes Koch}{E13}{}
\nv{Sexmästare}{Martin Gemborn Nilsson}{E14}{}
\nv{Krögare}{Malin Lindström}{BME14}{}
\nv{Entertainer}{Dalia Khairallah}{E15}{\nafr{10A}}
\end{narvarolista}

\subsection*{Ständigt adjungerande}
\begin{narvarolista}
\nv{Talman}{Johan Westerlund}{E11}{\nati{16}}
\end{narvarolista}

\section*{Protokoll}
\begin{paragrafer}
\p{1}{OFSÖ}{\bes}
Ordförande {\mo} förklarade mötet öppnat 12:14.

\p{2}{Val av mötesordförande}{\bes}
\valavmo

\p{3}{Val av mötessekreterare}{\bes}
\valavms

\p{4}{Tid och sätt}{\bes}
\tosg

\p{5}{Adjungeringar}{\bes}
\ingaadj

\p{6}{Val av justeringsperson}{\bes}
\valavj

\p{7}{Föredragningslistan}{\bes}
Fredrik \ypa att stryka \S14 ``Inköp av dammsugare''.

Föredragningslistan godkändes med yrkandet.

\p{8}{Fyllnadsval/Entledigande av funktionärer}{\bes}
\begin{fyllnadsval} %"Inga fyllnadsval." fylls i automatiskt
\fval{Fanny Månefjord}{Årskurs BME-1 ansvarig}
\fval{Viktor Persson}{Valberedningsledamot}
\fval{Daniel Johansson}{Valberedningsledamot}
\end{fyllnadsval}

\p{9}{Föregående mötesprotokoll}{\bes}
\latillprot{S17/16}

\p{10}{Rapporter}{}
\begin{paragrafer}
\subp{A}{Check in}{\info}
Erik mår bra. HeHE går tyvärr dåligt, vilket Erik ska försöka ta tag i.

Malin mår bra. Malin, Ester och Dalia har varit på Chalmers och spridit reklam och glädje. KM ska hålla sin första pub efter nollningen på lördag.

Johannes sa att ENU har mer eller mindre arbetat klart för året. Budgeten är avklarad

Johan sa att SRE har hittat en etta som vill vara årskursledamot.

Molly sa att NollU är i princip färdiga med sitt arbete för året. De håller på att fixa utvärderingar.

Stephanie sa att LED går bra. Det är många ettor som är intrsserade av att jobba som dioder under nästa läsperiod. Just nu säljer de billig Coca Cola eftersom de håller på att gå ut.

Anders är det bra med. Han har inte gjort så mycket med den fortlöpande ekonomin utan jobbar med förra halvårets bokföring, vilket snart är färdigt.

Martin sa att nollingen är slut, och därför är E6:s jobb nästan slut. Han har lyckats rädda Anders ljud \& ljus-budget genom att hyra ut ljusslingor till Kåren.

Fredrik håller sig flytande, har börjat kolla på funktionärstack och expo.

Dalia sa att NöjU går det bra för. Det är paintball på lördag med pub på kvällen. Hon har börjat skriva ett ordentligt testamente för Entertainer-posten och andra postbeskrivningar.

\subp{B}{Kåren informerar}{\info}
\emph{Kåren lös med sin frånvaro.}

\subp{C}{Ekonomi}{\info}
Anders sa att det inte finns så mycket att diskutera om ekonomin.

\end{paragrafer}

\p{11}{Tour de Styrelse}{\dis}
Mötet diskuterade datum och alkoholmängd för Tour de Styrelse.

\p{12}{Funktionärstack}{\dis}
Mötet diskuterade funktionärstacket som preliminärt ska vara den 12:e november. Mötet kom fram till att försöka hålla en nationssittning.

Johan föreslog hockeytema...

\p{13}{Expo/Val}{\dis}
Fredrik har börjat prata med Elin Magnusson om expot. Han föreslog att lägga det den 1:a november.

\p{15}{Nästa styrelsemöte}{\bes}
\Mba nästa styrelsemöte ska äga rum 2016-10-13 12:10 i E:1426.

\p{16}{Beslutsuppföljning}{\bes}
\textbf{\Mba skjuta upp beslutsuppföljning av ``Inköp av stekbord'' till styrelsemöte 22.}

\Ibfu

\p{17}{Övrigt}{\dis}
Fredrik frågade om någon kunde hjälpa till att köra skrot till soptippen ikväll.

Molly frågade om det fanns några mikrovågsugnar kvar. Anders svarade att det tyvärr inte finns några kvar.

\p{18}{OFSA}{\bes}
{\mo} förklarade mötet avslutat 12:48.

\end{paragrafer}

%\newpage
\hidesignfoot
\begin{signatures}{3}
\signature{\mo}{Mötesordförande}
\signature{\ms}{Mötessekreterare}
\signature{\ji}{Justerare}
\end{signatures}
\end{document}
