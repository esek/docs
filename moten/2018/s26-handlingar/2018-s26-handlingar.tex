\documentclass[10pt]{article}
    \usepackage[utf8]{inputenc}
    \usepackage[swedish]{babel}
    
    \def\doctype{Sena handlingar} %ex. Kallelse, Handlingar, Protkoll
    \def\mname{Styrelsemöte} %ex. styrelsemöte, Vårterminsmöte
    \def\mnum{S26/18} %ex S02/16, E1/15, VT/13
    \def\date{2018-11-28} %YYYY-MM-DD
    \def\docauthor{Daniel Bakic}
    
    \usepackage{../e-mote}
    \usepackage{../../../e-sek}
    
    \begin{document}
    
    \heading{{\doctype} till {\mname} {\mnum}}
    
    \section*{Information om punkterna}
    
    \begin{paragrafer}
    
    \p{11}{Trumset}{\bes}
    Föregående möte lyftes frågan angående inköp av trumset. Mötet var oense och ville ha mer tid och mer research innan ett beslut skulle fattas och valde att bordlägga punkten tills nästa möte.
    
    \rnamnpost{Adam Belfrage}{Entertainer}

    \p{12}{Valprocess}{\dis}
    Jag vill diskutera igenom hur vi i styrelsen vill ta del av valet av Cophös. Tycker det är viktigt att vi tar större del av beslutet än vad som gjordes föregående år.  
    
    \rnamnpost{Daniel Bakic}{Ordförande}

    \p{13}{Styrelsetestamente}{\dis}
    Jag vill diskutera igenom vad vi vill lämna kvar till våra efterträdare som styrelse. Jag har börjat på ett litet utkast med punkter och rubriker som jag anser bör vara med i ett styrelsetestamente. Kolla gärna igenom och fyll i/radera utifrån hur ni känner. Så kan vi diskutera det på mötet också.
    
    \rnamnpost{Daniel Bakic}{Ordförande}
    \end{paragrafer}
    
    \begin{signatures}{1}
    \ist
    \signature{\docauthor}{Ordförande}
    \end{signatures}
    
    \end{document}