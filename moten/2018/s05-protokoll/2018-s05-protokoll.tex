\documentclass[10pt]{article}
\usepackage[utf8]{inputenc}
\usepackage[swedish]{babel}

\def\mo{Daniel Bakic}
\def\ms{Axel Voss}
\def\ji{Fanny Månefjord}
%\def\jii{}

\def\doctype{Protokoll} %ex. Kallelse, Handlingar, Protkoll
\def\mname{styrelsemöte} %ex. styrelsemöte, Vårterminsmöte
\def\mnum{S05/18} %ex S02/16, E1/15, VT/13
\def\date{2018-02-22} %YYYY-MM-DD
\def\docauthor{\ms}

\usepackage{../e-mote}
\usepackage{../../../e-sek}

\begin{document}
\showsignfoot

\heading{{\doctype} för {\mname} {\mnum}}

%\naun{}{} %närvarane under
%\nati{} %närvarande till och med
%\nafr{} %närvarande från och med
\section*{Närvarande}
\subsection*{Styrelsen}
\begin{narvarolista}
\nv{Ordförande}{Daniel Bakic}{E15}{}
\nv{Kontaktor}{Axel Voss}{E15}{}
\nv{Förvaltningschef}{Magnus Lundh}{E15}{}
\nv{Cafémästare}{Elin Johansson}{BME16}{}
\nv{Øverphøs}{Andreas Bennström}{BME16}{}
\nv{SRE-ordförande}{Fanny Månefjord}{BME16}{}
\nv{ENU-ordförande}{Isabella Hansen}{E16}{}
\nv{Sexmästare}{Alexander Wik}{BME17}{}
\nv{Krögare}{Malin Heyden}{E16}{}
%\nv{Entertainer}{Adam Belfrage}{BME17}{}
\end{narvarolista}
\subsection*{Ständigt adjungerande}


\begin{narvarolista}
%\nv{Inköps- och lagerchef}{Sofie Johannesson}{E17}{}
%\nv{Inköps- och lagerchef}{Fabian Sondh}{E17}{}
%\nv{Inköps- och lagerchef}{Albin Pålsson}{E17}{}
%\nv{Kårordförande}{Linus Hammarlund}{}{}
%\nv{Kårrepresentant}{Jacob Karlsson}{}{\nafr{3}}
%\nv{Kårrepresentant}{Agnes Sörliden}{}{}
%\nv{Valberedningens ordförande}{Elin Magnusson}{}{}
%\nv{Skattmästare}{Olle Oswald}{}{}
%\nv{Kårrepresentant}{Daniel Damberg}{}{}
%\nv{Kårrepresentant}{John Alvén}{}{}
%\nv{Nollegeneral}{Jakob Nilsson}{}{}
%\nv{Talman}{Erik Månsson}{E14}{}
%\nv{Elektras Ordförande}{Elisabeth Pongratz}{}{}
%\nv{Inspektor}{Monica Almqvist}{}{}
\nv{Sigillbevarare}{Henrik Ramström}{}{}
\nv{Vice Entertainer}{Emil Bergström}{}{}
\end{narvarolista}

\begin{comment}
\subsection*{Adjungerande}
\begin{narvarolista}
%\nv{Post}{Namn}{Klass}{}
\end{narvarolista}
\end{comment}

\section*{Protokoll}
\begin{paragrafer}
\p{1}{OFMÖ}{\bes}
Ordförande {\mo} förklarade mötet öppnat 12:12.

\p{2}{Val av mötesordförande}{\bes}
{\valavmo}

\p{3}{Val av mötessekreterare}{\bes}
{\valavms}

\p{4}{Val av justeringsperson}{\bes}
{\valavj}

\p{5}{Godkännande av tid och sätt}{\bes}
{\tosg}

\p{6}{Adjungeringar}{\bes}
%{\ingaadj}

Sophia Grimmeiss Grahm adjungerades.
%Förnamn Efternamn adjungerades


\p{7}{Godkännande av dagordningen}{\bes}
Dagordningen godkändes


%Föredragningslistan godkändes med yrkandet.
%Föredragningslistan godkändes med samtliga yrkanden.

\p{8}{Föregående mötesprotokoll}{\bes}
\latillprot{S03/18}
%\ingaprot

\p{9}{Fyllnadsval och entledigande av funktionärer}{\bes}
\begin{fyllnadsval} %"Inga fyllnadsval." fylls i automatiskt
%\fval{Namn}{Post}
\end{fyllnadsval}

\p{10}{Rapporter}{}
\begin{paragrafer}
\subp{A}{Hur mår alla?}{\info}
Punkten protokollfördes ej.

\subp{B}{Utskottsrapporter}{\info}
ENU håller i Cv-fotografering nästa vecka. De ska även försöka hålla i lunchföreläsning en gång i månaden.

Nöjesutskottet håller i sin bowlingturnering imorgon. De ska även beställa otroligt mycket märken.

Imorgon håller Malin och resten av km gille. På lördag har de kickoff.

Fanny har möten i veckan. Årskursansvariga förbereder sig inför CEQ-möten. Digitaliseringsprojekt har nu startat.

Elin och cafeét har haft en lugn vecka. Förra veckan slogs försäljningsrekordet och de hade kickoff i måndags.

Cophøset håller i intervjuer hela veckan. Temasläppet är spikat till den 4 maj.

Magnus har bokfört och lampan i blådörren är lagad.

InfU håller veckoliga kodhackarmöte. Fotomojt är lagad och bilderna är tänkt att visas på tvskärmen i edekvata.

Sexet har sittning ikväll som de har preppat till. SKiphtet gick bra.

Daniel har pratat med talman och valberedningen. Det preliminära datumet för vårterminsmötet är nu spikat.

\subp{C}{Ekonomisk rapport}{\info}
  Ekonomin ser bra ut hälsar Magnus.

\subp{D}{Kåren informerar}{\info}

\end{paragrafer}

\p{11}{Arbetskläder}{\dis}
  Km har köpt tröjor över sin budget.
  Det är lugnt hälsar Magnus.

  Tröjorna kommer användas flitigt och Malin kommer även se till att tröjorna återanvänds av Källarmästeriet nästa år.
\p{12}{Tankar och idéer på hur vi ska utveckla styrelsearbetet}{\dis}

  Fanny skulle gärna vilja ha omvärldsinformation på styrelsemötena. Då ska aktuell information om andra sektioner på LTH och andra lärosäten lyftas fram. Axel tar på sig att hålla i denna punkten.

  Mötet instämmer alla i att de behöver bli bättre på att följa upp diskutioner utanför möten. Mötet ska försöka ha mer beslutsuppföljningar även om det bara rör sig om fortsatt diskution.

  Styrelsen vill bjuda in Anders till ett möte med anledning av den personuppgiftslag som kommer träda i kraft inom kort.

  Sophia har bra utbildningar som hon vill att styrelsen går på. Exempelvis utbildning om ledarskap. Daniel tar på sig att kolla upp detta med kåren.


\p{13}{Nästa styrelsemöte}{\bes}
\Mba nästa styrelsemöte ska äga rum 2018-03-01 klockan 12:10 i E:1124.

\p{14}{Beslutsuppföljning}{\bes}

  Henrik \ypa stryka \emph{Kundvagnar} från beslutsuppföljningen.

  Mötet \ypa Alexander Wik ska undersöka behovet av kundvagnar.

\Mbabay

Beslutsuppföljning sattes till s06-2018

%\Ibfu


\p{15}{Övrigt}{\dis}

Axel höll kort info om InfU. Kåren har ett instagramkonto som de gärna ser att sektionen använder när vi har någonting kul att visa.

Daniel informerar om att vi fått brev från Pantamera där de klagar på att det slängs skräp i pantsäckarna.

\p{16}{OFMA}{\bes}
{\mo} förklarade mötet avslutat 12:54.
\end{paragrafer}

%\newpage
\hidesignfoot
\begin{signatures}{3}
\signature{\mo}{Mötesordförande}
\signature{\ms}{Mötessekreterare}
\signature{\ji}{Justerare}
\end{signatures}
\end{document}
