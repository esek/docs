    \documentclass[10pt]{article}
    \usepackage[utf8]{inputenc}
    \usepackage[swedish]{babel}
    \def\mo{Erik Månsson}
    \def\ms{Axel Voss}
    \def\ji{Henrik Ramström}
    \def\jii{Pontus Landgren}
    
    \def\doctype{Protokoll} %ex. Kallelse, Handlingar, Protkoll
    \def\mname{Vårterminsmöte} %ex. styrelsemöte, Vårterminsmöte
    \def\mnum{VT/18} %ex S02/16, E1/15, VT/13
    \def\date{2018-04-24} %YYYY-MM-DD
    \def\docauthor{\ms}
    
    \usepackage{../e-mote}
    \usepackage{../../../e-sek}
    
    \begin{document}
\showsignfoot

\heading{{\doctype} för {\mname} {\mnum}}

%\naun{}{} %närvarane under
%\frun{}{} %frånvarande under
%\nati{}{} %närvarande till och med
%\nafr{}{} %närvarande från och med
\section*{Närvarande}
\subsection*{Styrelsen}
\begin{narvarolista}
	\nv{Entertainer}{Adam Belfrage}{BME17}{}
	\nv{Sexmästare}{Alexander Wik}{BME17}{}
	\nv{Øverphøs}{Andreas Bennström}{BME16}{\frun{14}{21B}}
	\nv{Kontaktor}{Axel Voss}{E15}{}
	\nv{Ordförande}{Daniel Bakic}{E15}{}
	\nv{Cafémästare}{Elin Johansson}{BME16}{}
	\nv{SRE-ordförande}{Fanny Månefjord}{BME16}{}
	\nv{ENU-ordförande}{Isabella Hansen}{E16}{}
	\nv{Förvaltningschef}{Magnus Lundh}{E15}{\frun{14}{21B}}
	\nv{Krögare}{Malin Heyden}{E16}{}
\end{narvarolista}

\subsection*{Medlemmar}
\begin{narvarolista}
	\nv{}{Amanda Gustafsson}{BME17}{\nati{22}{A}}
	\nv{}{Amanda Nilsson}{BME16}{\nati{21}{A}}
	\nv{}{Anton Jigsved}{BME16}{}
	\nv{}{Björn F. Nimvik}{E15}{}
	\nv{}{Edvard Carlsson}{E16}{}
	\nv{}{Emil P. Lundh}{E17}{\nafr{14}{A}}
	\nv{}{Filip Winzell}{BME16}{\nati{22}{A}}
	\nv{}{Hannes Björk}{E17}{}
	\nv{}{Henrik Ramström}{E16}{}
	\nv{}{Henrik von Friesendorff}{BME17}{\nati{22}{A}}
	\nv{}{Jennie Karlsson}{E16}{\nati{21}{A}}
	\nv{}{Jessica Kågeman}{BME16}{\nati{21}{A}}
	\nv{}{Johan Karlberg}{E14}{}
	\nv{}{Johan Siwerson}{E17}{}
	\nv{}{Johan Wikstrand}{E17}{\nati{22}{A}}
	\nv{}{Johannes Larsson}{E16}{}
	\nv{}{Jonathan Benitez}{E17}{}
	\nv{}{Josefine Sandström}{E14}{}
	\nv{}{Kajsa Ekenberg}{BME17}{\nati{22}{A}}
	\nv{}{Lina Samnegård}{BME16}{\nati{21}{A}}
	\nv{}{Linnea Sjödahl}{BME15}{}
	\nv{}{Markus Rahne}{BME14}{}
	\nv{}{Max Mauritsson}{BME16}{\nati{22}{A}}
	\nv{}{Moa Rönnlund}{E17}{\nati{22}{A}}
	\nv{}{Pontus Landgren}{E14}{}
	\nv{}{Rasmus Sobel}{BME16}{}
	\nv{}{Sanna Nordberg}{BME16}{}
	\nv{}{Saga Juniwik}{E16}{}
	\nv{}{Sonja Kenari}{E16}{}
	\nv{}{Sophia Carlsson}{BME17}{}
	\nv{}{Sophia Grimmeiss Grahm}{BME14}{}
	\nv{}{Stephanie Bol}{BME17}{\nati{22}{A}}
	\nv{}{Tom Andersson}{E15}{}
	\nv{}{Tove Börjeson}{E17}{}
\end{narvarolista}

\subsection*{Ständigt adjungerande}
\begin{narvarolista}
	\nv{Revisor}{Anders Nilsson}{E13}{\nafr{21}{B}}
	\nv{Talman}{Erik Månsson}{E14}{}
	\nv{Revisor}{Fredrik Peterson}{E14}{}
	\nv{Inspektor}{Monica Almqvist}{----}{\naun{14}{14}}
\end{narvarolista}

\begin{comment}
\subsection*{Adjungerande}
\begin{narvarolista}
	\nv{Post}{Hannes Åström}{D-sektionen}{}
\end{narvarolista}
\end{comment}

\newpage
\section*{Protokoll}
\begin{paragrafer}
	\p{1}{TaFMÖ}{}
	Talman {\mo} förklarade mötet öppnat 17:20.

	\p{2}{Val av mötesordförande}{}
	Talman {\mo} valdes.

	\p{3}{Val av mötessekreterare}{}
	Kontaktor {\ms} valdes.

	\p{4}{Godkännande av tid och sätt}{}
	Tid och sätt godkändes.

	\p{5}{Val av två justeringspersoner}{}
	\valavj

	\p{6}{Adjungeringar}{}
	%\ingaadj
	Hannes Åström från D-sektionen.
	\p{7}{Godkännande av dagordningen}{}
	Erik Månsson \ypa att lägga till ``Sena Motioner'' under \S23.

	Erik Månsson \ypa välja posten inspektor när Monica Almqvist ankommer.

	%Föredragningslistan godkändes.
	%Föredragningslistan godkändes med yrkandet.
	Föredragningslistan godkändes med samtliga yrkanden.

	\p{8}{Föregående sektionsmötesprotokoll}{}
	\latillprot{VM/17}

	\p{9}{Meddelanden}{}
	Daniel Bakic vill att alla ska ta vara på att att man kan förändra och förbättra sektionen.

	Adam Belfrage \ypa Daniel Bakic ska ta av sig sina solglasögon.

	Mötet diskuterade frågan.

	Erik Månsson \ypa streck i debatten.

	\Mbabay

	Sophia Grimmeiss Grahm informerade om att kåren har sitt temasläpp den 3:e Maj.

	Andreas Bennström informerade om att E-phøs har temasläpp den 4:e Maj.

	Malin Heyden meddelade att det är tacogille den 27:e April.

	Markus Rahne meddelade att f1-röj har spikat sitt tema.

	Adam Belfrage meddelade att biljetter till Paintball kommer säljas i E-foajen.

	\p{10}{Beslutsuppföljning}{}

	Erik Månsson \ypa stryka \emph{Äskning av pengar
		till mentorsprogram} från Beslutsuppföljning.

	Total kostnad uppgick till 3080 kr och budget var på 15 000 kr.

	\Mbaby

	\p{11}{Utskottsrapporter}{}
	Möjligheten att ställa frågor till styrelsen och valberedningen gavs.

	\p{12}{Uppföljning av verksamhetsplan}{}
	Möjligheten att ställa frågor till styrelsen och valberedningen gavs.

	\p{13}{Ekonomisk rapport}{}
	Magnus Lundh presenterade den ekonomiska rapporten.

	Sektionen har bra ekonomi, våra tillgångar uppnår ungefär en miljon kronor.

	Johan Karlberg undrade om sektionen behöver spendera mer pengar.

	Magnus Lundh svarade att sektionen egentligen behöver spendera mer pengar.

	Erik Månsson undrade hur caféet går nu när sektionen inte längre betalar ut lön till någon fast anställd.

	Magnus Lundh svarade att han tyvärr inte har några exaka siffror på detta.

	Sonja Kenari undrade hur mycket vinst sektionen får gå med skattemässigt.

	Magnus Lundh svarade att vi egentligen inte får ha någon vinst men att vi placerar överskottet i interna fonder.


	\p{14}{Val av funktionärer}{}

	\textbf{\Mba välja Johan Wikstrand till Teknikfokusansvarig.}\par

	\textbf{\Mba välja Albin Nyström Eklund till Revisorsuppleant.}\par

	\textbf{\Mba välja Jonathan Benitez till Inköp och Lagerchef.}\par

	\textbf{\Mba välja Monica Almqvist till Inspektor.}\par


	\p{15}{Verksamhetsberättelser för 2017}{}
	%Styrelsen 2016 och Valberedningen 2016 gav sina verksamhetsberättelser för 2016.
	%\textbf{\Mba lägga verksamhetsberättelsen för 2017 till handlingarna.}

	\emph{Ingen hade några kommentarer på verksamhetsberättelserna}


	\p{16}{Bokslut för 2017}{}
	Sophia Grimmeiss Grahm prestenterade bokslutet för 2017.

	\p{17}{Revisionsberättelse för 2017}{}
	Erik Månsson presenterade revisionsberättelsen för 2017.

	\textbf{Mötet beslutade att bifalla revisorernas första yrkande.
	\emph{``Att resultat- och balansräkning fastställes.''}}

	\p{18}{Styrelsens förslag till resultatdisposition}{}
	Daniel Bakic presenterade förslaget till resultatdisposition.

	\textbf{Mötet beslutade att bifalla revisorernas andra yrkande.
	\emph{``Att styrelsens förslag till resultatsdisposition tages.''}}
	\newpage
	\p{19}{Uttag ur sektionens fonder sedan förra terminsmötet}{}
	Daniel Bakic berättade om uttagen ur sektionens fonder sedan förra terminsmötet.

	Sophia Grimmeiss Grahm undrade om mötet har koll på de olika fonderna.

	En förklaring på fonderna gavs och kan läsas \href{https://eee.esek.se/files/styrdokument/budgetar/budget-2018.pdf}{\textbf{här} (från sida 4)}.

	\p{20}{Frågan om ansvarsfrihet för 2017}{}
	\begin{paragrafer}
		\subp{A}{Funktionärer}{}
		\textbf{\Mba finna funktionärerna 2017 ansvarsfria.}
		\subp{B}{Utskott}{}
		\textbf{\Mba finna utskotten 2017 ansvarsfria.}
		\subp{C}{Styrelse}{}
		\textbf{\Mba finna styrelsen 2017 ansvarsfria.}
		\subp{D}{Revisorer}{}
		\textbf{\Mba finna revisorerna 2017 ansvarsfria.}
		\subp{E}{Valberedning}{}
		\textbf{\Mba finna valberedningen 2017 ansvarsfria.}


	\end{paragrafer}

	\p{21}{Behandling av motioner}{}
	\begin{paragrafer}
		\subp{A}{Införande av E-speleman}{}
		Henrik Ramström presenterade motionen.

		Emil P. Lundh undrade mer specifikt vilka arbetsområden som är tidskrävande.

		Henrik Ramström svarade att kontakt med företag angående sponsring har varit det som tagit mest tid.

		Malin Heyden undrar vad espelemannen ska göra när det inte är dreamhacke.

		Henrik Ramström svararde att denne kan hjälpa till under spelkvällar men att tanken är att det kommer vara dreamhacke oftare.

		Daniel Bakic presenterade styrelsens svar.

		\textbf{\Mba avslå motionen}

		\Mba ajournera mötet till klockan 19.05.

		\subp{B}{Införande av en Färgspecialist}{}

		Henrik Ramström presenterade motionen.

		Daniel Bakic prestenterade styrelsens svar.

		Mötet diskuterade frågan.

		Henrik Ramström \ypa streck i debatten.

		\Mbaby

		\textbf{\Mba avslå motionen och styrelsens svar i sin helhet.}

		\subp{C}{Inköp av ny sektionskamera}{}
		Daniel Bakic presenterade motionen.

		Daniel Bakic presenterade styrelsens svar på motionen.

		\textbf{\Mba bifalla motionen med styrelsens svar i sin helhet.}

		\subp{D}{Kvalitetskrav}{}
		Adam Belfrage presenterade motionen.

		Daniel Bakic presenterade styrelsens svar på motionen.

		\textbf{\Mba avslå motionen i sin helhet.}

		\subp{E}{Uppdaterad utrustning för linbanan}{}
		Magnus Lundh presenterade Motionen.

		Daniel bakic presenterade styrelsens svar.

		\textbf{\Mba bifalla motionen i sin helhet.}

		\subp{F}{Internationell Nolleguide}{}

		Edvard Carlsson presenterade motionen.

		Daniel Bakic presenterade styrelsens svar.

		Mötet diskuterade frågan.

		\textbf{\Mba bifalla motionen utan styrelsens tilläggsyrkande.}

		\Mba ajournera mötet till 20.05.

		\subp{G}{Inköp av Sexmästarsabel}{}

		Markus Rahne presenterade motionen.

		Sexmästare Alexander Wik vill inte ha en sabel.

		Daniel Bakic presenterade styrelsens svar på motionen.

		Rahne \ypa stryka \emph{``och sittande Sexmästare''} från sitt sista yrkande eftersom denne inte stödjer motionen, samt att i sitt första yrkande ändra modell till m/1893.

		Tom Andersson lyfte fram att den planerade budgeten klarar av att täcka även en sabel av modell 1893.

		Andreas Bennström \ypa streck i debatten.

		\Mbaby.

		Mötet diskuterade frågade.

		\textbf{\Mba avslå motionen i sin helhelt.}

		\subp{H}{Utskottsmärken}{}

		Henrik Ramström presenterade motionen.

		Fanny Månefjord undrade vad utskottsordförande skulle göra.

		Henrik Ramström svarade att det är utskottsordförande som fattar beslut om vilka märken som kommer användas och ansvarar för att dela ut dessa till sina respektive utskott.

		Daniel Bakic presenterade styrelsens svar på motionen.

		Henrik Ramström jämkade sig med styrelsens yrkande.

		Linnea Sjödahl undrade om man får märke även om man inte är med i utskottet längre.

		Henrik Ramström menar att märkena kommer ta slut väldigt snabbt då.

		Sanna Nordberg undrade om gamla funktionärer i alla fall kan få möjlighet att köpa dessa.

		Henrik Ramström ska kolla vidare på det.

		Pontus Landgren \ypa lägga till ``Uppföljning Utskottsmärken'' till nästa sektionsmöte med ekiperingsexperterna som ansvariga.

		\textbf{Mötet beslutade att bifalla motionen med styrelsens ändringsyrkande och Pontus tilläggsyrkande}
	\end{paragrafer}

	\p{22}{Behandling av propositioner}{}
	\begin{paragrafer}
		\subp{A}{Inköp av umph-box}{}

		Daniel Bakic presenterade propositionen.

		\textbf{\Mba bifalla propositionen i sin helhet.}
		\subp{B}{Införskaffande av nya mikrovågsugnar}{}
		Daniel Bakic presenterade propositionen.

		\textbf{\Mba bifalla propositionen i sin helhet.}
		\subp{C}{Ändring av hur sektionen väljer Phøs}{}
		Andreas Bennström presenterade propositionen.

		Mötet diskuterade frågan.

		Pontus Landgren menar att argumentet med minskad arbetsbelastning på valberedning inte stämmer. Detta eftersom det fortfarande är lika många intervjuer från valberedningens sida.

		Fredrik Peterson sa att propositionen borde läggas på hösterminsmötet istället. Detta eftersom man då hinner testa den aktuella processen två gånger innan den röstas igenom på mötet och ändras.

		Henrik Ramström yrkande på streck i debatten.

		\Mbaby

		Fredrik Peterson \ypa återremittera propositionen till HT18

		\textbf{\Mba återremittera propositionen HT18.}

		\Mba ajournera mötet till klockan 21.35.

		Mötet återupptogs klockan 21:37.

		\subp{D}{Uppdatering av funktionärers skyldigheter i reglemente}{}
		Elin Johansson presenterade propositionen.

		\textbf{\Mba bifalla propositionen i sin helhet.}
		\subp{E}{Införande av UtEDischoansvarig}{}
		Adam Belfrage presenterade propositionen.

		Adam Belfrage \ypa stryka de tre sista att-satserna.

		Pontus Landgren \ypa justera reglementestillägget i propositionen från \emph{ ``...och eventuell UtEDischoansvarig.'' } till \emph{ ``...och UtEDischoansvarig.'' }.

		\textbf{\Mba bifalla propositionen tillsammans med Pontus Landgrens yrkande och Adam Belfrages ändringsyrkande.}
	\end{paragrafer}

	\p{23}{Behandling av sena motioner}{}
	\begin{paragrafer}
		\subp{A}{Borttagandet av redaktionella organ från stadgan}{}

		Erik Månsson och Sophia Grimmeiss Grahm presenterade motionen.

		\textbf{\Mba bifalla motionen i sin helhet.}

		\subp{B}{Vice-poster som styrelsens suppleanter}{}
		Erik Månsson och Sophia Grimmeiss Grahm presenterade motionen.

		\textbf{\Mba bifalla motionen i sin helhet.}

		\subp{C}{Upprustning av bar i Edekvata}{}
		Markus Rahne presenterade motionen.

		Henrik Ramström undrade om det inte är smidigare och billigare med rostfritt stål.

		Markus Rahne svarade att han redan har undersökt saken och anser att det är jobbigare att montera och framförallt dyrare.

		\textbf{\Mba bifalla motionen i sin helhet.}
		\subp{D}{Inköp av ny tappanordning}{}
		Markus Rahne presenterade motionen.

		Markus Rahne \ypa istället köpa in en tappanordning med kontakt av typ L40 med en budget på 7500kr.

		\textbf{\Mba bifall motionen med Markus Rahnes ändringsyrkande.}
	\end{paragrafer}
	\p{24}{Övrigt}{}
	\textbf{\Mba välja Hannes Björk till posten Chefredaktör.}

	Björn \ypa justera budgeten för ``PHOS01 NollU allmänt'' med \SI{-12000}{kronor} till \SI{-32000}{kronor}. \emph{En förtydling av yrkandet kan ses under bilagor.} 

	Anders \ypa en notis läggs till i protokollet under punkten övrigt att en sen handling inkommit och behandlats av sektionsmötet.
	Handlingen behandlar en budgetjustering av PHOS01 NollU allmänt och att den sena handlingen lades till övriga sena handlingar under mötets gång.

	\textbf{\Mbabay}

	Björn F. Nimvik \ypa sänka priser för Gasquebiljetter. \emph{En förtydling av yrkandet kan ses under bilagor.} 

	Anders Nilsson ifrågasatte varför detta yrkande lyfts först nu. När en motion borde ha skickats in innan mötet så att alla sektionenes medlemmar fått chans att läsa igenom den.

	Björn F. Nimvik svararde att han inte var upplyst om att budgetrevideringar kunde genomföras under Vårterminsmöte.

	Sophia Grimmeiss Grahm menar att eftersom budgetkravet på E6 har sänkts finns det ingen anledning till att yrkandet ska gå igenom.

	Anders Nilsson menar att detta kan lyftas hos styrelsen och att man i så fall tar pengar från dispositionsfonden vid behov.

	Johan Karlberg svararade att man med kravsänkningen på sexmästeriets budget och kombination med äskning från styrelsen vid behov har stor marginal att räkna med.

	Björn F. Nimvik la ner sitt yrkande. 

	\p{25}{TaFMA}{}
	Talman {\mo} förklarade mötet avslutat 22.52.

\end{paragrafer}
\hidesignfoot
\begin{signatures}{4}
	\signature{\mo}{Mötesordförande}
	\signature{\ms}{Mötessekreterare}
	\signature{\ji}{Justerare}
	\signature{\jii}{Justerare}
\end{signatures}
\newpage
\section*{Bilagor}
\thispagestyle{empty}
%\includepdf[pages=-,pagecommand={\section{Bilagor} \thispagestyle{empty}}, fitpaper=true]{../vt-handlingar/motioner/nollubudget.pdf}
%\includepdf[pages=-]{../vt-handlingar/motioner/nollubudget.pdf}
\includegraphics[scale=0.8]{../vt-handlingar/motioner/nollubudget.pdf}
\newpage
\thispagestyle{empty}
\includegraphics[scale=0.8]{../vt-handlingar/motioner/gasque.pdf}
%\includepdf[pages=-]{../vt-handlingar/motioner/gasque.pdf}
\end{document}