\documentclass[10pt]{article}
\usepackage[utf8]{inputenc}
\usepackage[swedish]{babel}
\def\mo{Erik Månsson}
\def\ms{Axel Voss}
\def\ji{xxx}
\def\jii{xxx}

\def\doctype{Protokoll} %ex. Kallelse, Handlingar, Protkoll
\def\mname{Vårterminsmöte} %ex. styrelsemöte, Vårterminsmöte
\def\mnum{VT/18} %ex S02/16, E1/15, VT/13
\def\date{2018-04-24} %YYYY-MM-DD
\def\docauthor{\ms}

\usepackage{../e-mote}
\usepackage{../../../e-sek}

\begin{document}
\showsignfoot

\heading{{\doctype} för {\mname} {\mnum}}

%\naun{}{} %närvarane under
%\nati{}{} %närvarande till och med
%\nafr{}{} %närvarande från och med
\section*{Närvarande}
\subsection*{Styrelsen}
\begin{narvarolista}
    \nv{Ordförande}{Daniel Bakic}{E15}{}
	\nv{Kontaktor}{Axel Voss}{E15}{}
	\nv{Förvaltningschef}{Magnus Lundh}{E15}{}
	\nv{Cafémästare}{Elin Johansson}{BME16}{}
	\nv{Øverphøs}{Andreas Bennström}{BME16}{}
	\nv{SRE-ordförande}{Fanny Månefjord}{BME16}{}
	\nv{ENU-ordförande}{Isabella Hansen}{E16}{}
	\nv{Sexmästare}{Alexander Wik}{BME17}{}
	\nv{Krögare}{Malin Heyden}{E16}{}
	\nv{Entertainer}{Adam Belfrage}{BME17}{}
\end{narvarolista}

\subsection*{Medlemmar}
\begin{narvarolista}
\end{narvarolista}

\subsection*{Ständigt adjungerande}
\begin{narvarolista}
\nv{Talman}{Erik Månsson}{E14}{}
%\nv{Post}{Namn}{Klass}{}
\end{narvarolista}

\begin{comment}
\subsection*{Adjungerande}
\begin{narvarolista}
%\nv{Post}{Namn}{Klass}{}
\end{narvarolista}
\end{comment}

\newpage
\section*{Protokoll}
\begin{paragrafer}
\p{1}{TaFMÖ}{}
Talman {\mo} förklarade mötet öppnat 17:xx.

\p{2}{Val av mötesordförande}{}
Talman {\mo} valdes.

\p{3}{Val av mötessekreterare}{}
Kontaktor {\ms} valdes.

\p{4}{Godkännande av tid och sätt}{}
Tid och sätt godkändes.

\p{5}{Val av två justeringspersoner}{}
\valavj

\p{6}{Adjungeringar}{}
\ingaadj

\p{7}{Godkännande av dagordningen}{}
XX XX \ypa att lägga till ``xx xx xx'' under \S XX.

%Föredragningslistan godkändes.
%Föredragningslistan godkändes med yrkandet.
%Föredragningslistan godkändes med samtliga yrkanden.

\p{8}{Föregående sektionsmötesprotokoll}{}
\latillprot{VM/18}

\p{9}{Meddelanden}{}
Ingen hade något att meddela.

\p{10}{Beslutsuppföljning}{}

\p{11}{Utskottsrapporter}{}
Möjligheten att ställa frågor till styrelsen och valberedningen gavs.

\p{12}{Uppföljning av verksamhetsplan}{}
Möjligheten att ställa frågor till styrelsen och valberedningen gavs.

\p{13}{Ekonomisk rapport}{}

%\textbf{\Mba lägga den ekonomiska rapporten till handlingarna.}

\p{14}{Val}{}
\begin{paragrafer}
    \subp{A}{Val av funktionärer}{}

%Mötet vakantsatte allt i klump.

\textbf{\Mba vakantsätta posten Teknikfokusansvarig.}\par

\textbf{\Mba vakantsätta posten Redaktör.}\par

\textbf{\Mba vakantsätta posten Teknokrat.}\par

\textbf{\Mba vakantsätta posten Fritidsledare.}\par

\textbf{\Mba vakantsätta posten Karnevalsmalaj.}\par

\textbf{\Mba vakantsätta posten Stridsrop.}\par

\textbf{\Mba vakantsätta posten Umph-meister.}\par

\textbf{\Mba vakantsätta posten Revisorsuppleant.}\par
\end{paragrafer}
\end{paragrafer}
	\begin{paragrafer} \item[] %quick and dirty
	\begin{paragrafer}
    \subp{B}{Val av hedersmedlemmar}{}
\end{paragrafer}

\p{15}{Verksamhetsberättelser för 2017}{}
%Styrelsen 2016 och Valberedningen 2016 gav sina verksamhetsberättelser för 2016.
\textbf{\Mba lägga verksamhetsberättelsen för 2016 till handlingarna. Beslutet togs på \S20}

\p{16}{Bokslut för 2017}{}
Anders Nilsson prestenterade bokslutet för 2016.
%\textbf{\Mba lägga bokslutet 2016 till handlingarna.}

\p{17}{Revisionsberättelse för 2017}{}
Jesper Ek presenterade revisionsberättelsen för 2016.
%\textbf{\Mba lägga revisionsberättelsen 2016 till handlingarna.}
Jesper Ek tog tillbaka sitt yrkande om
\begin{attsatser}
    \att resultat- och balansräkning fastställes,
    \att styrelsens förslag till resultatsdisposition tages, samt
    \att styrelsen för 2016 beviljas ansvarsfrihet.
\end{attsatser}

\p{18}{Styrelsens förslag till resultatdisposition}{}
Anders Nilsson presenterade förslaget till resultatdisposition.

Jesper Ek yrkade på
\begin{attsatser}
    \att resultat- och balansräkning fastställes, samt
    \att styrelsens förslag till resultatsdisposition tages.
\end{attsatser}

\Mbaby

\textbf{\Mba godkänna resultatdispositionen}

\p{19}{Uttag ur sektionens fonder sedan förra terminsmötet}{}
Erik Månsson berättade om uttagen ur sektionens fonder sedan förra terminsmötet.

\p{20}{Frågan om ansvarsfrihet för 2016}{}
    \begin{paragrafer}
        \subp{A}{Funktionärer}{}
        \textbf{\Mba finna funktionärerna 2016 ansvarsfria.}
        \subp{B}{Utskott}{}
        \textbf{\Mba finna utskotten 2016 ansvarsfria.}
        \subp{C}{Styrelse}{}
        \textbf{\Mba finna styrelsen 2016 ansvarsfria.}
        \subp{D}{Revisorer}{}
        \textbf{\Mba finna revisorerna 2016 ansvarsfria.}
        \subp{E}{Valberedning}{}
        \textbf{\Mba finna valberedningen 2016 ansvarsfria.}

        Erik Månsson \ypa ta 45 minuters matpaus.

        Fredrik Peterson \ypa ta 30 minuters matpaus.

        Rasmus Sobel \ypa ta 35 minuters matpaus.

        Josefine Sandström \ypa ta 40 minuters matpaus.

        Fredrik Peterson jämkade sig med Rasmus Sobel.

        \ji  \ypa ta 37,5 minuters matpaus.

        Erik Månsson begär sluten votering.

        Erik Månsson \ypa sträck i debatten.

        \Mbaby.

        Mötet bestlutade att ajournera mötet i 35 minuter.

        Mötet återupptogs 18:56.
    \end{paragrafer}

    \p{21}{Behandling av motioner}{}
        \begin{paragrafer}
          \subp{A}{Make E-lektro banana band great again}{}


          \subp{B}{BLED-café}{}

          \subp{C}{Förbättrad förvaring för sektionens lager}{}

          \Mba bifalla motionen i sin helhet.

          \subp{D}{Ansvarig för uppdatering av examenstavlor}{}

          \Mba bifalla motionen i sin helhet.

          \subp{E}{Representationsklädsel åt Inspektorn}{}

          \Mba bifalla motionen i sin helhet.
      \end{paragrafer}


      \p{22}{Behandling av propositioner}{}
          \begin{paragrafer}
            \subp{A}{Öppna kravprofiler för valberedning}{}
            Erik Månsson presenterade propositionen.

            \Mba bifalla propositionen i sin helhet.

            Pontus Landgren \ypa på 5 minuter paus efter propositionen.

            Mötet ajounerades i 5 minuter.

            Mötet återupptogs 19:57
            \subp{B}{Flytta policybeslut ``Närvaro vid Sektionsmöte'' till reglementet och uppdatera valmetoden}{}
            Erik Månsson presenterade propositionen.

            \Mba bifalla propositionen i sin helhet.
            \subp{C}{Äskning av pengar till mentorsprogram}{}
            Erik Månsson presenterade propositionen med bakgrund.

            \Mba bifalla propositionen i sin helhet.
            \subp{D}{Uppgradering av ljudsystem i Edekvata}{}
            Pontus Landgren presenterade propositionen.

            \Mba bifalla propositionen i sin helhet.
            \subp{E}{Inköp av PA-toppar}{}
            Pontus Landgren presenterade propositionen.

            \Mba bifalla propositionen i sin helhet.
            \subp{F}{Införande av projektfunktionärer}{}
            Erik Månsson presenterade propositionen.

            Erik Månsson yrkar på att ändra den sista punkten i att-satsen till:
            ``Har en mandatperiod som bestäms vid valtillfället, och är maximalt ett år lång.''

            Styrelsen jämkade sig med Erik Månssons yrkande.

            \Mba bifalla propositionen i sin helhet med ändringsyrkandet.
            \subp{G}{Borttagning av giltig terminsräkning i reglementet}{}
            Erik Månsson presenterade propositionen sittandes.

            \Mba bifalla propositionen i sin helhet.
            \subp{H}{Ändring av hur Sektionen väljer Phøs}{}
            Mötet ajounerades i 5 minuter.

            Niklas Gustafson presenterade propositionen.

            \Mba bifalla propositionen i sin helhet.
        \end{paragrafer}

\p{23}{Övrigt}{}
Anders Nilsson informerade om jubileumsveckan som just nu är igång.
\begin{itemize}
  \item På torsdag är det målning av cylindern, ätas korv och spelas musik.
  \item På fredagen är det gille, det är asnice.
  \item På lördag är det bal med eftersläpp.
\end{itemize}

\p{24}{TaFMA}{}
Talman {\mo} förklarade mötet avslutat 20:59.

\end{paragrafer}

%\newpage
\hidesignfoot
\begin{signatures}{4}
\signature{\mo}{Mötesordförande}
\signature{\ms}{Mötessekreterare}
\signature{\ji}{Justerare}
\signature{\jii}{Justerare}
\end{signatures}
\end{document}
