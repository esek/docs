\documentclass[../_main/handlingar.tex]{subfiles}

\begin{document}
\utskottsrapport{Cafémästeriet}
Under våren har i utskottet arbetat mot att alla i utskottet ska känna sig bra med sina funktionärsposter. Det centrala teamet (Cafémästare, Vice Cafémästare och Inköps- och lagerchefer) har tillsammans med Ulla haft möten nästan varje vecka för att hålla sig uppdaterade om vad som komma skall. Vi har även ställt fram ett pantkärl utanför LED och ser till att det skötseln kring den hanteras, vilket har gått väldigt bra än så länge. Önskemål om matavfallsinsamling har tagits upp och ska ses över med PH.

Utskottet har även samarbetat med andra utskott och även utomstående med olika projekt. Exempelvis har vi sålt kaffe till folk som vill promota grejer i exempelvis E-foajén. Flickor på Teknis beställde 80st pastasallader från LED till en lunchföreläsning. Teknikfokus hyrde LED (exklusive köket) under morgonen då dem skulle ha frukost vilket fungerade bra för båda parterna. Vi sålde massvis med goda semlor under fettisdagen. Jag och SRE-ordförande höll i en av stationerna under phadderutbildningen där vi informerade om hur viktigt LED är för sektionen och tryckte på att phaddrar bör komma in och hjälpa till innan nollningen då det under nollningen är deras plikt att ta över för deras nollor om dem inte kan. Vi har även diskuterat lite med Peter (f.d vaktmästare) om att huset eventuellt ska köpa mackor från LED till sina möten.

För övrigt har verksamheten flutit på ytterst bra och alla funktionärer (och Ulla) verkar trivas bra med sina roller i caféet.

\begin{signatures}{1}
    \mvh
    \signature{Elin Johansson}{Cafémästare}
\end{signatures}

\end{document}
