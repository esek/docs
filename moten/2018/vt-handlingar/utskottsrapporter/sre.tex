\documentclass[../_main/handlingar.tex]{subfiles}

\begin{document}
\utskottsrapport{Studierådet}
Studierådets främsta uppgift är studiebevakning. Utskottet har granskat och censurerat CEQ:enkä- ter och årskursrepresentanter har gått på CEQ-möten med kursansvariga för att diskutera och utvärdera gångna kurser. Mötenas främsta mål är att hitta förbättringsmöjligheter. Vi har även nominerat studenter till programledningar och institutionsstyrelser och teknologkåren har tillsatt alla våra nominerade studenter. Intresset för utskottet och för att få vara med i programledningar har varit stort vilket är roligt!

Utskottet har haft ett likabehandlingsevangemang med filmvisning av Sameblod på samernas nationaldag där vi även bjöd på mat. Likabehandlingsombuden har haft en välmåendevecka tillsammans med Likabehandlingskollegiet på kåren. Världsmästarna håller i event med världsmästarkollegiet.

Tillsammans med NollU har undertecknad diskuterat pluggkvällar under nollningen samt tillsatt pluggphaddrar. Vi har förhoppningar att pluggkvällarna ska bli mer studiefokuserade under nästa nollning.

Jag har deltagit i SRX-kollegiet på kåren med personer med motsvarande poster från andra sektioner. Vi har diskuterat matematikinstitutionens arbete med Endimensionell analys och utbytt idéer och tankar för hur man bäst kan bedriva studiebevakning. Här får jag även en inblick i vad kåren och andra studieråd på LTH arbetar med.

Slutligen har utskottet haft en mysig kick-off och vi har idéer på fler saker som ska hända under året!
\begin{signatures}{1}
    \mvh
    \signature{Fanny Månefjord}{SRE-ordförande}
\end{signatures}

\end{document}
