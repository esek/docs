\documentclass[../_main/handlingar.tex]{subfiles}

\begin{document}
\utskottsrapport{Studierådet}
SRE har granskat och censurerat de CEQ-enkäter som studenterna på E- och BME-programmet har fyllt i. Därefter har representanter från studierådet deltagit i möten med kursansvarig och programledning för att utvärdera och försöka ta fram eventuella förbättringar inför framtida kursomgångar.

Utskottet har nominerat studeranderepresentanter i de programledningar och institutionsstyrelser som E-sektionen representerar. Dessa representanter väljs sedan av Teknologkårens styrelse.

Undertecknad har deltagit i kollegiet för studierådsordföranden, SRX, och utbytt studierelaterad information mellan kåren och alla sektioner. I SRX delas tips och trix mellan sektionerna samtidigt som man enklare kan koordinera problem som sträcker sig mellan flera sektioner.

Under hösten samt i början av våren anordnade SRE CEQ-tävlingar med syftet att försöka öka svarsfrekvensen på enkäterna. Utfallet av dessa testomgångar gav ingen tillfredsställande ökning varvid beslutet togs att inte genomföra tävlingen efter läsperiod 3.

SRE i samarbete med NollU har valt pluggphaddrar inför nollningen. I SRE ska även arbetet med SRE-workshopen under nollningen ta sin början. Under våren kommer även Speak up days, ett evenemang i samarbete med kåren, anordnas. Utskottet kommer även fortsatt anordna pluggkvällar och utvärdera upplägget och nyttan av pluggkvällarna.
\begin{signatures}{1}
    \mvh
    \signature{Fanny Månefjord}{SRE-ordförande}
\end{signatures}

\end{document}
