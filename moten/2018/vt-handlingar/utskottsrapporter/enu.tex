\documentclass[../_main/handlingar.tex]{subfiles}

\begin{document}
\utskottsrapport{Näringslivsutskottet}
Under våren har ENU främst försökt hitta olika nya och intressanta företag som skulle vilja komma till sektionen och hålla i någon form av event. I år har lite ändringar gjorts angående vilka företagsrepresentanter som kommer till sektionens event. Detta är mest för att det ska komma ingenjörer istället för bara HR-personer, vilket gör att studenterna får ut mer av att gå på ett event.

Det första eventet som hölls av ENU i samarbete med sektionens fotografer var en CV-fotografering. “Lunch med en ingenjör” kommer att utföras för tredje året i rad och detta året kommer förhoppningsvis fler BME-företag delta. ENU har även hållit i en lunchföreläsning och det är många spännande projekt på gång.

Teknikfokus hölls i februari och gick väldigt bra, det var ca 35 betalande företag där och de var alla väldigt nöjda med dagen. E-sektionen var lite underrepresenterade när det gällde funktionärsdeltagandet och detta kommer förhoppningsvis att öka till nästa år.

Ett nytt projekt som precis har startats upp är “en FED pub” som är ett samarbete med F- och D-sektionen. Tanken är att skapa ett unikt tillfälle för studenter och företag att träffa varandra och knyta värdefulla kontakter. Detta koncept kommer att vara tre gånger där varje sektion håller i en pub var, F-sektionen har redan haft sin och det blev väldigt lyckat.
\begin{signatures}{1}
    \mvh
    \signature{Isabella Hansen}{Ordförande Näringslivsutskottet}
\end{signatures}

\end{document}
