\documentclass[../_main/handlingar.tex]{subfiles}

\begin{document}
\utskottsrapport{Näringslivsutskottet}
I början av året delades näringslivsutskottet upp i flera små ”eventgrupper” där varje grupp hade hand om ett eget event från kontakt med företag till marknadsföring och slutligen till själva utförandet. Idén var att de som har hand om eventet vet bäst hur det ska marknadsföras och utföras etc. I nuläget har vissa grupper, vars event redan ägt rum, delats upp och mer generella mejl skickas ut till företag som bland annat inkluderar intresse för att synas under nollningen. Utskottet har också haft flera lunchmöten där medlemmarna ger statusrapporter om hur arbetet går och om några oklarheter har uppkommit.

Året började med en CV-fotografering för sektionens medlemmar.  Förra året lanserades eventet ”lunch med en ingenjör” vilket ENU utförde även i år. Nytt var att företagen fick betala per ingenjör vilket fungerade bra men resulterade delvis till att det blev färre BME-företag. Utskottet har även hållit i en lunchföreläsning, ett kvällsföredrag och annonserat på sektionens facebooksida och hemsida samt haft en mycket lyckad kick-off!

Teknikfokus gick mycket bra i år med ca 30 betalande företag varav två BME-företag. Förhoppningen är att antalet BME-företag fortsätter att växa nästa år. Funktionärsdeltagandet var ungefär lika stort från både E- och D-sektionen vilket är jättebra och förhoppningsvis fortsätter det så!
\begin{signatures}{1}
    \mvh
    \signature{Isabella Hansen}{Ordförande Näringslivsutskottet}
\end{signatures}

\end{document}
