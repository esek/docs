\documentclass[../_main/handlingar.tex]{subfiles}

\begin{document}
\proposition{Uppdatering av reglemente}
I dagsläget står det i reglementet under \S10:2:A \textbf{Allmänna funktionärsåtaganden}:\par

Det åligger en funktionär
\begin{attlist}
    \item utföra de sysslor som faller inom funktionärsbeskrivningens ram,
    \item ägna sig åt den verksamhet som faller inom utskottsbeskrivningens ram,
    \item framföra idéer till arrangemang och förbättringar till Styrelsen, samt
    \item i mån av tid bistå med assistans i LED-café när detta behövs.
\end{attlist}

Den sista att-satsen kan vara svår för funktionärer i övriga utskott att följa om det inte organiseras av sina utskottsordföranden. Tanken är att styrelsen lägger upp en plan med olika veckor för olika utskott, som passar bra för utskottets egen verksamhet. Frågan är aktuell eftersom att den centrala gruppen i Cafémästeriet i dagsläget står utan skydd om för få personer vill vara Dioder eller Halvledare under en läsperiod. Detta kan leda till stress, lägre kvalitet på mackor och sallader och förkortade öppettider vilket i sin tur ökar svinnet och minskar intäkterna. LED café tillhör sektionen och är en stor inkomstkälla som gynnar alla medlemmar.

För att Cafémästeriet inte ska stå utan skydd yrkar styrelsen på
\begin{attsatser}
    \att i reglementet \S9:1:A lägga till: \par
    \subsection*{9:1:A Skyldigheter}
    Det åligger utskotten
    \begin{attlist}
        \item i första hand ägna sig åt den verksamhet som faller innanför
        utskottsbeskrivningens ramar,
        \item arrangera och genomföra skiftet,
        \item under Expot presentera utskottets verksamhet och dess
        funktionärsposter,
        \item i övrigt verka för en levande Sektion och för Sektionens bästa,
        \item lämna uppdaterad information ämnad för hemsidan till Kodhackare, samt
        \item varje år ordna minst en omsitts för tidigare och nuvarande
        funktionärer.
        \item \hl{bistå med assistans i LED-café under de läsveckor som av styrelsen tilldelats utskottet.}
    \end{attlist}
    \changenote
\end{attsatser}

\begin{signatures}{1}
    \ist
\signature{Elin Johansson}{Cafémästare}
\end{signatures}

\end{document}