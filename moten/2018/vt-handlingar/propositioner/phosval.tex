\documentclass[../_main/handlingar.tex]{subfiles}

\begin{document}
\proposition{Ändring av när sektionen väljer Øverphøs}

Under förra vårterminsmötet gjordes en förändring av hur sektionen väljer Co-phøs. Denna förändring slog väl ut, då det var fler personer än normalt som valde att söka till både Øverphøs och Co-phøs. Det som dock är problematiskt med det nya systemet för val av Co-phøs är att dessa väljs efter valmötet. Att vara Co-phøs är nämligen tidskrävande, och det är därför mycket svårt att inneha en annan post på sektionen samtidigt. Detta innebär att om en person vill söka till Co-phøs kommer denna person inte att kunna söka någon annan post under valmötet. Om personen i fråga sedan inte skulle bli vald till Co-phøs har hen inte någon möjlighet till att söka en annan större post på sektionen till det kommande verksamhetsåret. Detta anser styrelsen vara problematiskt, eftersom det kan innebära att sektionen går miste om drivna och engagerade personer som inte får möjlighet att söka till andra poster. 

Andra aspekter som kan tas i beaktelse är att det lätt kan leda till stress för Valberedningen och Øverphøs electus då Co-phøsen väljs efter valmötet. Valmötet brukar hållas sent på hösten och det kan därför bli kort om tid att hinna välja Co-phøs innan juluppehållet. Eftersom personerna som väljs in i phøset kanske inte heller känner varandra sedan tidigare är det bra om phøset är komplett tidigare under hösten. De kan på så sätt hinna lära känna varandra och bygga en bra gruppkemi innan det tunga arbetet drar igång under vårterminen. 

Styrelsen ser att en lösning på problemet är att Øverphøset väljs tidigare under hösten, innan valmötet äger rum. Styrelsen ser två olika alternativ till hur detta skulle kunna gå till.
\begin{enumerate}
\item Att Øverphøset väljs på ett separat sektionsmöte som hålls innan höstterminsmötet.
eller
\item Att Øverphøset väljs på höstterminsmötet och att detta hålls i god tid innan valmötet.
\end{enumerate}

Problemet som ses med alternativ 1 är att det kan bli svårt att få sektionens medlemmar att gå på fyra olika sektionsmöten varje år. Detta kan innebära att deltagandet på vissa möten minskar, vilket innebär att en mindre andel av sektionens medlemmar är med och röstar i viktiga beslut. Styrelsen anser därför att alternativ 2 kommer göra störst nytta för sektionen. 
\newpage
Styrelsens förslag är alltså att Øverphøset väljs på höstterminsmötet, vilket ska hållas i god tid innan valmötet. Co-phøsen kan kandidera samtidigt som Øverphøset och detta leder till att intervjuerna kan påbörjas direkt efter höstterminsmötet. Detta kommer leda till att Co-phøsen kommer att väljas innan valmötet äger rum. Därmed kommer de personer som inte väljs till Co-phøs få möjlighet till att söka andra poster under valmötet. Samtidigt kommer också phøset att bli komplett tidigare under hösten. Phøset kommer då att ha tid att lära känna varandra bra redan under hösten. 

Med anledning som ovan yrkar styrelsen på 

\begin{attsatser}
    \att i stadgan ändra \S4:11 till:
    \subsection*{\S4:11 Höstterminsmötet}
    Vid Höstterminsmötet skall följande ärenden tas upp:
    \begin{alphlist}
        \item 	budget inför kommande verksamhetsår,
        \item 	resultatrapport från första halvan av verksamhetsåret,
        \item 	revision av utskottsbeskrivningar,
        \item 	motioner inlämnade i rätt tid enl. §4:14,
        \item   \hl{val av Øverphøs, samt}
        \item 	de punkter som föreskrivs i Reglementet.
    \end{alphlist}
    \changenote
    \att i stadgan ändra \S4:12 till:
    \subsection*{\S4:12 Valmötet}
    Vid Valmötet skall endast följande ärenden tas upp:
    \begin{alphlist}
        \item val av Styrelse\hl{, exluderat Øverphøs,} och funktionärer enligt Reglemente,
        \item val av Valberedning,
        \item val av Revisorer,
        \item val av representanter till TLTH:s utskott enligt gällande Stadgar och
        Reglemente för TLTH, samt
        \item övriga valärenden enligt Reglementet.
    \end{alphlist}
    \changenote
\end{attsatser}

\begin{signatures}{1}
    \ist
    \signature{\ordf}{Ordförande}
\end{signatures}

\end{document}
