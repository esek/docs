\documentclass[../_main/handlingar.tex]{subfiles}

\begin{document}
\proposition{UtEDischoansvarig i NöjU}


Som Entertainer finns det ganska mycket att hålla koll på. Framförallt om det är mycket personen i fråga vill driva och många event personen vill genomföra. Om Entertainern förväntas ta hälften av ansvaret för UtEDischot ligger det väldigt mycket på Entertainerns axlar. Tar inte Entertainern hälften av ansvaret är det inte heller schysst mot D-sektionens UtEDischoansvarig som då får dra ett tyngre lass än vad denne kanske ställt in sig på. 

Därför är det inte mer än rimligt att E-sektionen skapar en ny funktionärspost med namnet UtEDischoansvarig vars ansvar är att planera och genomföra UtEDischot. Med det sagt förväntas Entertainern fortfarande vara delaktig i planeringen till viss del, hjälpa till med något om den ansvarige behöver det samt övervaka processen så att allt går rätt till. Entertainern är alltså fortfarande ytterst ansvarig för att UtEDischot blir av. Finns det ingen UtEDischoansvarig att tillgå får då entertainern kliva in och ta över det ansvaret. 

Styrelsen yrkar därför på: 

\begin{attsatser}
    \att i reglementet under \S10:2H ändra postbeskrivningen för Entertainer till:
    \subsubsection*{10:2:H Funktionärerna i Nöjesutskottet, NöjU}

    \begin{emptylist}
    \item Entertainer (u)
        \begin{dashlist}
          \item Har det övergripande ansvaret för Sektionens kultur-, nöjes- och fritidsaktiviteter.
          \item Ansvarar för Sektionens instrument.
          \item Ansvarar för planering och genomförandet av UtEDischot tillsammans med D-sektionen \hl{och eventuell UtEDischoansvarig}.
          \item Ansvarar för utkvittering av access till biljard- och pingisskåpet.
        \end{dashlist}
        \end{emptylist}
        \changenote
    \att i reglementet under \S10:2H lägga till UtEDischoansvarig (1) med följande underpunkt,
    \begin{itemizedash}
        \item Ansvarar för planeringen och genomförandet av UtEDischot tillsammans med \\D-sektionen och Entertainern.
    \end{itemizedash}
    \att ändringen träder i kraft direkt,
    \att posten fyllnadsväljs under nästa styrelsemöte för posten i år, samt
    \att posten för nästkommande år valbereds och väljs på valmötet. 

\end{attsatser}

\begin{signatures}{1}
    \ist
    \signature{\ordf}{Ordförande}
\end{signatures}

\end{document}
