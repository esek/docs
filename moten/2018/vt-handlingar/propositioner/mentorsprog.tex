\documentclass[../_main/handlingar.tex]{subfiles}

\begin{document}
\proposition{Äskning av pengar till mentorsprogram}

Styrelsen har startat upp ett mentorsprogram till sektionen som ska köras för första gången under nollningen 2017. Mentorsprogrammet beskrivs i ett dokument som man kan hitta på hemsidan under ``Information'' och sedan ``Om E-sektionen'', eller via följande länk:

\href{https://eee.esek.se/files/styrdokument/misc/mentorsprogram-2017.pdf}{\texttt{https://eee.esek.se/files/styrdokument/misc/mentorsprogram-2017.pdf}}

Tanken med programmet är som följer, taget ur dokumentet:

\textit{``Syftet med E-sektionens mentorsprogram är först och främst \textbf{att främja den personliga kontakten mellan adepten, en nyantagen student, och mentorn, en äldre student}.
Tanken med en sådan kontakt är att adepten ska kunna bolla idéer och prata om sin studiesituation öppet med någon som har mycket erfarenhet.
Målet är att det ska hjälpa adepten att bättre och snabbare lära sig planera sin tid och lägga upp sina studier.
I slutändan är visionen att det ska hjälpa fler nyantagna studenter till att klara sig bättre i början och stanna kvar på sitt program.
Enkelt sagt - vi vill att fler stannar kvar hos oss och klarar sina studier.''}

Eftersom ingen budget finns för mentorsprogrammet måste vi äska pengar för att kunna starta upp det. Den totala budgeten är satt till \SI{15000}{kr}, men programledningen kommer med stor sannolikhet sponsra en betydande del av detta. Dock vet vi inte ännu hur mycket, så vi vill äska hela budgeten för säkerhets skull.

Därför yrkar styrelsen på

\begin{attsatser}
    \att avsätta \SI{15000}{kr} till mentorsprogrammet 2017,
    \att kostnaderna belastar utrustningsfonden, samt
    \att redovisning av projektet läggs på beslutsuppföljningen till HT/17 med Erik Månsson och Pontus Landgren som ansvariga.
\end{attsatser}

\begin{signatures}{2}
    \ist
    \signature{\ordf}{Ordförande}
    \signature{\sreordf}{SRE-ordförande}
\end{signatures}

\end{document}
