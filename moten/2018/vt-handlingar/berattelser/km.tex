\documentclass[../_main/handlingar.tex]{subfiles}

\begin{document}
\berattelse{Källarmästeriet}
Under 2017 så höll Källarmästeriet gille i princip varje fredag och avslutade med det traditionella julgillet med 110 sittande. Resultatmässigt gick verksamheten väldigt bra och KM höll ett oerhört lyckat jubileumsgille i sammanknytning med E-sektionens 55-årsjubileum, både sett ur det slutliga resultatet men även besökarmässigt. Utöver de reguljära fredagarna har KM även fixat mat till UtEDischot, regattan och anordnade E-sektionens andra officiella ölprovning. Välkomstgillet under nollning stakades upp och planerades tillsammans med NollU och resultatet blev att ett tidigare kaosevent gick förvånansvärt smidigt.

Under året har det lagts mycket fokus på att förbättra utbudet i form av mat och cocktails, där Vice Krögare tog det på sig som sitt heliga uppdrag att rusta upp kvalitén på maten, till mångas belåtenhet. En ny cocktailmeny instiftades till första gillet och bibehölls under året som gick.

Personalmässigt var året lite gungigt då vi var tvungna att byta Vice Krögare efter halva året och tog in en Vice Krögare från källarmästarna som fick sitt eldprov under nollningen. På grund av personalskiftet och upplärningsperioden blev arbetet tyngre för dom redan sittande krögarna och arbetsfördelningen under hösten blev skev som resultat, men trots detta hölls kontinuerliga gillen till stor framgång, dock till lite väl hög personlig belastning. Sett ur hela utskottets perspektiv så hölls den traditionella kick-offen i form av Kapsyltömmarnatten och under hösten hölls en middag för utskottet samt den från 2016 nystartade traditionen med jobbartack tillsammans med nya och gamla KM emellan i samband med matlagning till julgillet. Försök att engagera utskottet genom att planera gillen och övriga event gjordes men med lågt intresse från utskottet.

Rent utrustningsmässigt så sparkade KM igång nytt liv i barens tappanordning och började sälja fatöl för första gången på flera år. Resultat var lyckat och KM17 rekommenderar att en riktig anordning köps in för att ersätta den gamla. Försök gjordes även att restaurera klämgrillen KM16 lyckades erhålla. Insidan rengjordes och elektroniken ersattes men högtemperaturisolering saknas för att den ska kunna sättas i bruk.

I sin helhet var 2017 ett bra år för Källarmästeriet med välbesökta gillen, härligt utskottsgäng, lagom mängd kaos på gillena och förbättringar inom mat- och spritutbudet som hela sektionen kunde uppskatta. Överlämningen mellan ’17 och ’18 genomfördes under julgillet med strålande resultat och ansvaret överlämnades med stor tillit.

\begin{signatures}{1}
    \mvh
    \signature{Markus Rahne}{Krögare 2017}
\end{signatures}

\end{document}
