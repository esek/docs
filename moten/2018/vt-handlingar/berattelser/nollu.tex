\documentclass[../_main/handlingar.tex]{subfiles}

\begin{document}
\berattelse{Nolleutskottet}
Nollningsuskottet har haft ett händelserikt år under 2017. Under våren var det mycket möten, både inom NollU men även med externa parter som kåren, andra sektioner och självfallet inom E-sektionen. Som vanligt valdes phaddrar för att ta emot de nya och vi kunde inte vara nöjdare med deras insats. 
Andra nämnvärda arrangemang under våren var phadderutbildningen och temasläppet. Vi valde att presentera temat i samband av en sittning tillsammans med D-sektionen, vilket var mycket uppskattat. 
Vi såg till att vara i princip klara med allt planerande innan alla åkte hem över sommaren för att slippa jobba under en välbehövd paus från nollningsarbetet. Väl tillbaka efter sommaren började vi genomföra allt praktiskt som vi inte gjort/kunnat göra innan sommaren, som tillexempel måla om väggen i edekvata. 
Några event som funnits tidigare togs bort (t.ex. F.I.S.S, Philm och Øverphøs0lympiaden) och ersattes med några nya (t.ex. Nollesöndagen och ØverphøsStafetten).  Nollesöndagen fick många komplimanger och bestod av ett avslappnat söndagshäng på gröngräset. Under dagen spelades det bland annat såpfotboll och en öl eller två konsumerades, det var även första dagen som phøset släppte på stoneface. 
Arbetsbelastningen var som vanligt hög på nollningsutskottets medlemmar men vi har haft det otroligt roligt ihop. Testamentet redigerades och skickades vidare till nästkommande NollU.

\begin{signatures}{1}
    \mvh
    \signature{Niklas Gustafson}{Øverphøs 2017}
\end{signatures}

\end{document}
