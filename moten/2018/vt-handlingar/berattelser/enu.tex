\documentclass[../_main/handlingar.tex]{subfiles}

\begin{document}
\berattelse{Näringslivsutskottet}
ENU startade verksamhetsåret med en CV fotografering och senare en kick-off. Det
anordnades även en lunchföreläsning, en branschkväll och en pub. I slutet av våren kom Axis
och invigde de nya mikrovågsugnarna genom att bjuda på mikromat och
sätta upp stickers.

Under nollningen stod ENU för maten under SVEP robotic challenge och vi höll i två
lunchföreläsningar. Till lunchföreläsningarna hade vi caterad mat vilket underlättade för
ENU. Efter nollningen anordnades det även två monterevent i foajén där företagen köpte
kaffe från LED.

Lunch med Ingenjör anordnades både på våren och hösten. För att locka fler studenter
anordnade ENU olika tävlingar, varav att lotta ut en biljett till julgillet var den bättre av
dem. Trots tävlingar var intresset inte så högt som vi hade hoppats på. De som faktiskt
medverkade på Lunch med Ingenjör verkade dock nöjda.

Teknikfokus fortsatte att dra in mycket pengar till sektionen samt lockade många företag och besökare. E-sektionens deltagande var bra men fortfarande lite mindre än D-
sektionens vilket måste förbättras för att mässan och samarbetet med D ska fortsätta vara framgångsrikt.

Under årets gång har ENU även gjort mycket marknadsföring åt företag via Facebook,
affischer och TV-skärmarna.
\begin{signatures}{1}
    \mvh
    \signature{Josefine Sandström}{Näringslivsutskottets ordförande 2017}
\end{signatures}

\end{document}
