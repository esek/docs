\documentclass[../_main/handlingar.tex]{subfiles}

\begin{document}
\berattelse{Informationsutskottet}

Under året jobbade HeHE-redaktionen på bra med att släppa vårt nyhetsbrev. Vår picasso hjälpte till att göra många fina affischer. Vår fotograf var på ett antal evenemang och fotograferade väldigt fina bilder åt oss som vi publicerat på facebooksidan. Tillsammans med Ph\o set släpptes en fin nollEguide! Ekiperingsexpertera, som på den tiden tillhörde InfU, jobbade på bra med försäljningen.

DDG jobbade mycket med hemsidan och fixat med bl.a. alkoholhanteringssystemet, biljardaccesssystemet och backup-systemet. De fixade också E-vote till flera systersektioner som alla är mycket nöjda med hur det fungerar. Några Kodhackare såg till att få upp vår nya Nollningshemsida. Under nollningen hjälpte Macapärerna och ett gäng Kodhackare till med att hålla datorstugan vilket gick bra. Till det nästkommande året (alltså i år) blev det bra rekrytering till DDG, som fortsätter jobba på flitigt.

Tyvärr saknade sektionen någon Lastgammal/Alumniansvarig under större delen av året och har därför inte haft någon alumniverksamhet. Däremot, sedan propositionen om utökandet av alumniverksamheten gick igenom HT/16, har engagemanget för alumniverksamheten ökat.

Jag själv som Kontaktor har jobbat med mycket blandade grejer. På våra möten har jag försökt skriva bra utförliga protokoll och få ut dem i tid. Jag har också tagit fram nya mallar för sektionens alla dokument. Handlingarna till sektionsmötena är helt omgjorda och förhoppningsvis enklare att läsa. Tillsammans med resten av styrelsen lade jag fram propositionen som att införa posten ``Vice Kontaktor'', vilket gick igenom. Jag har också uppdaterat lite här och där på hemsidan, sett till att vi har WiFi för iZettle i Edekvata, och skapat en Instagram för Sektionen. Vår Instagram har i skrivande stund $\sim240$ följare, så det känns lyckat! Utöver allt dedär har jag mest jobbat med att hålla alla andra informationskanaler uppdaterade.

Viktigt under året var ju också inbjudningarna till Nolleqasquen, där stavning och korrekturläsning var extra viktigt eftersom de skickades till så många fina gäster. Jag tillsammans med resten av styrelsen såg till att välkomna våra vänstyrelser från Chalmers och KTH hit, vilket var väldigt trevligt och bådade gott för framtida år tillsammans.

\begin{signatures}{1}
    \mvh
    \signature{Johan Karlberg}{Kontaktor 2017}
\end{signatures}

\end{document}
