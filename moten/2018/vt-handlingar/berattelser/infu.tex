\documentclass[../_main/handlingar.tex]{subfiles}

\begin{document}
\berattelse{Informationsutskottet}
Under året har samtliga i utskottet bidragit till Sektionens verksamhet. Vi har fått ut HeHE, fina affischer och fina bilder som bland annat har publicerats på facebook. Tillsammans med NollU släpptes även en nollEguide och en Nollningshemsida.

DDG har fungerat bra under året, mycket tack vare väldigt engagerade Macapärer. DDG till exempel varit drivande i Nollningshemsidan och andra projekt för att försöka öka intresset för DDG på Sektionen. Underhåll av befintlig utrustning gjordes kontinuerligt under året, bland annat byttes några datorer ut mot Rassbery Pi:s. Något som diskuterats under året är behovet av att uppgradera Hacke. Till skillnad från föregående år så anlitade SVL studenter och en representant från Universitetets centrala DDG för att hålla i datorstugan. Det bedöms inte nödvändigt men skall ha fungerat bra.

Intresset för alumniverksamheten känns inte överväldigande på Sektionen men under jubileet var det ett flertal alumner som kom på återbesök. Även på gemensamma evenemang med andra sektioner kom det alumner på återbesök till Lund. Jag tror och hoppas att intresset för alumniverksamheten från Sektionens sida kommer att öka i och med att fler BME-studenter tar examen.

Som Kontaktor har jag huvudsakligen fokuserat på Sektionens dagliga verksamhet samt att utföra mitt styrelseuppdrag så bra som möjligt. Tack grunden som lades förra året, alla mallar till sektionens dokument, så har arbetet med protokoll och handlingar gått väldigt bra. För att öka transparensen gentemot Sektionens medlemmar har jag försökt att använda Instagram för att förmedla vad som händer på styrelsens möten.

Jag har pratat med min Vice Kontaktor om dennes post. Vår bedömning är att det är en post som tål att experimenteras med för att hitta hur posten kan tillföra så mycket som möjligt till utskottet.

\begin{signatures}{1}
    \mvh
    \signature{Johan Karlberg}{Kontaktor 2017}
\end{signatures}

\end{document}
