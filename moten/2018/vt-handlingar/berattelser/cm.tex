\documentclass[../_main/handlingar.tex]{subfiles}

\begin{document}
\berattelse{Cafémästeriet}
Under 2017 har Cafémästeriet haft möten varje vecka för att samla utskottet och prata om vad som har hänt under veckan samt annat av relevans. Vi satte i början av året upp ett pantkärl utanför LED och skötseln av denna har legat på CM vilket har gått bra. Alla i utskottet har jobbat hårt för att allt i LED ska fungera så bra som möjligt och för att vidareutveckla verksamheten.

Under våren låg största fokus i att alla i utskottet ska komma in i sina nya roller. Vi har även samarbetat med andra utskott och externer som exempelvis Teknikfokus som fick hyra LED under morgonen. Vi gjorde även sallader till dåvarande Flickor på Teknis inför en av deras lunchföreläsningar, vilket var lyckat. Vi sålde en hel del semlor under fettisdagen, mycket gott och mycket uppskattat. LED var även delaktig i phadderutbildningen för alla phaddrar inför nollningen.

Hösten började med att vi städade och monterade upp nya hyllor i vårt läsklager. Det är nu mycket fräschare och finns mer utrymme. Sedan gjorde vi, tillsammans med utomstående hjälp, pastasallader till alla nollor, phaddrar, styrelsen, phøset och programledningen till måndag lv. 0. Samma dag presenterades LED och dess verksamhet för alla nollor och senare under dagen på Camp St. Hans höll Cafémästaren i en av stationerna. Under nollningen fick nollorna testa på att jobba i LED enligt specifikt schema. Detta gav goda resultat och många av nollorna har valt att engagera sig inom utskottet.

Första läsperioden var väldigt tung för utskottet, framförallt för Cafémästare, Vice Cafémästare och Inköpscheferna, då Ulla var sjukskriven hela perioden. Ingen var beredd på att Ulla inte skulle vara där och det krävdes stor omväxling för att saker och ting skulle fungera. All möda gav oss i utskottet bredare perspektiv om vad som fungerar och inte fungerar och vi har under resterande delen av hösten jobbat mycket för att caféet ska rulla på utan heltidsanställd inför 2018. Bland annat skrevs en proposition om ny post i caféet, Halvledare, inför HT/17. Tydligare instruktioner om vad de olika posterna i utskottet innebär gjordes och instruktioner för hur den dagliga verksamheten ska skötas förtydligades. 

Andra saker som gjordes under hösten var att införskaffa eftermiddagsdioder, slopa kontanthantering och införskaffa stämpelkort till kaffe. Avtal kring sponsorer gjordes i samband med näringslivsutskottet. Miljö- och hälsovårdsmyndigheten var på besök för att säkerställa att vi följer de regler som finns kring exempelvis mathantering. Vi fick fåtal klagomål och ändrade utifrån det hur vi gör med kylvaror och vi strukturerade om vårt lager som är mycket bättre nu än tidigare. Annars har vi haft fortsatt gott samarbete med andra utskott och andra utanför sektionen som exempelvis köpt kaffe. 

Resultatet för CM under verksamhetsåret 2017 var bättre än förväntat, resultatet var ungefär dubbelt så stort jämfört med vad vi budgeterat. Detta beror på icke utbetalda löner i och med Ullas sjukskrivning. Om vi räknar med att vi betalat ut lön som vanligt hade vi gått lite under vårt önskade resultat. Detta beror högst troligen på brist av arbetare, speciellt under hösten. Nästa års budget är lite oklar då vi inte riktigt vet vad en rimlig uppskattning är då caféet ska drivas helt ideellt. Det är viktigt att se över detta under året så att rimligare budget kan sättas till året därpå.

\begin{signatures}{1}
    \mvh
    \signature{Daniel Bakic}{Cafemästare 2017}
\end{signatures}

\end{document}
