\documentclass[../_main/handlingar.tex]{subfiles}

\begin{document}
\berattelse{Styrelsen}
I vanlig ordning sparkade styrelsen igång året med kollegieskiphten, en riktigt lärorik KPL, och ett gött funktionärsskiphte. Efter dessa bravader var styrelsen på styrelseutbildning på Kåren och på ekonomiutbildning med Förvaltningschefen. Sammanlagt en mycket bra start på året, vilket gjorde att styrelsemötena kunde rulla på som vanligt med möte nästan varje vecka och kvällsmöten vid behov.

Till vårterminsmötet lade styrelsen fram en handfull propositioner där största fokus låg på hur vi väljer funktionärer. Styrelsen tog också tag i att uppgradera våra ljudsystem. Efter Vårterminsmötet låg styrelsens fokus mest på dagligverksamheten, vilket vi har ansett vara väldigt viktigt under året. Direkt efter vårterminsmötet arrangerades också ett mycket lyckat 55-årsjubileum!

Då Pontus Landgren avgick som SRE-ordförande för att bli heltidare på Kåren fick vi på ganska kort varsel välja in en ny. Lyckligtvis gick det smidigt och ersättaren Edvard Carlsson kom snabbt in i sin roll i styrelsen!

Dessvärre fick nollningen en ganska tuff start för styrelsen och CM då Ulla tyvärr inte kunde vara med oss på grund av personliga skäl. Mycket tid gick till att stötta och hjälpa CM att driva caféet helt ideellt. Eftersom Ulla skulle gå i pension vid årsskiftet var styrelsen inställd på att caféet skulle drivas ideellt i framtiden, men det var inte något som vi var beredda på redan till höstterminen.

Resten av nollningen gick bra med ett fint samarbete mellan utskotten. Själva styrelsen har hjälpt till lite här och där när det behövts, inte minst på första dagen på St. Hans vilket var väldigt trevligt! Några enstaka incidenter har inträffat men det är ingenting som man inte kunnat reda ut. Nollningen rundades sedan av med en fantastisk NollEqasque.

Efter nollningen tog styrelsen en liten paus för att sedan raskt gå vidare till att börja jobba med sektionsmötena. Innan inläsningsveckan drog igång höll styrelsen i det årliga expot som drog väldigt mycket folk! Till HT/17 var styrelsens största fråga hur caféet skulle drivas i framtiden och hur budgeten skulle byggas upp för att ``kompensera'' för vår förväntade ökade vinst i CM.

Efter sektionsmötena anordnade styrelsen ett funktionärstack innehållandes Laserdome och sittning på Wermlands Nation. Liksom förra året var detta arrangemang mycket uppskattat av Sektionens funktionärer! Sist men inte minst avslutades styrelsens uppdrag med en Kurs På Landet med nya styrelsen, där de fick lära sig att man inte kan driva med Styrelsen 2017.

\begin{signatures}{1}
    \mvh
    \signature{Erik Månsson}{Ordförande 2017}
\end{signatures}

\end{document}
