\documentclass[../_main/handlingar.tex]{subfiles}

\begin{document}
\verksplanuppf{VT 2017 (avgående styrelsen)}

\subsubsection*{Styrelsen}
Sedan höstterminsmötet är det främst två punkter från verksamhetsplanen som styrelsen arbetetat med. För det första togs en ekonomisk plan fram, denna ska fungera som en riktlinje för hur pengar ska fonderas i framtiden och ungefär vilket resultat Sektionen bör sikta på. Denna riktlinje kommer att vara i behov av revideringar, troligen årligen. Under slutet av året arrangerades även ett funktionärstack som bestod av laserdome efterföljt av sittning och eftersläpp på Blekingska, detta verkade vara mycket uppskattat hos funktionärerna och vi hoppas att detta tack kan bli årligen återkommande. Ett problem med funktionärstack i allmänhet är att det är svårt att veta hur mycket pengar som finns kvar av budgeten för detta då flera saker vanligtvis bokförs sent på året, till exempel kostnader för kaffe, för att att bättra kunna utnyttja budgeten skulle en mer kontinuerlig bokföring underlätta.

\subsubsection*{Informationsutskottet}
Informationsutskottet har under 2016 lyckats bra med att jobba mot delmålen. Relationen med våra vänsektioner är bibehållen - KTH och Chalmers var och besökte oss under nolleqasquen, och bjöd i gengäld oss tillbaka på deras tillställningar. DDG har jobbat för att Sektionens datorsystem och hemsida hållts uppdaterad. Sektionens informationskanaler är något som ständigt utvärderades - varje gång någon har velat nå ut till Sektionen har vi varit noggranna med att försöka nå ut på bästa sätt till så många som möjligt.

Alumniverksamheten låg på is under året. På den ljusa sidan gick vår proposition om att utöka alumniverksamheten igenom på VT/16. Det gav ett positivt resultat i form av fler sektionsmedlemmar som var intresserade av att jobba med alumniverksamheten 2017.

\subsubsection*{Källarmästeriet}
Vi har arbetat mycket med marknadsföring av gillen. Vi har fortsatt att sätta upp planscher i E-huset och information på TV-skärmarna. Vi har dessutom satt upp planscher i de andra husen, varit med i kårnytt varje vecka och börjat göra evenemang på facebook, till de flesta arrangemang som anordnas. Det verkar ha gett god effekt då fler från andra sektioner har hittat till gillena. Dessutom jobbas det för att ha fler gillen med andra sektioner som indirekt marknadsför gillena.

Det har jobbats med att involvera alla i utskottet mer genom att låta källarmästarna vara med och bestämma lite mer genom idéer från workshop. Dessutom har vice krögare haft stort ansvar och cølen har fått mycket fria händer inom sitt område. Det har varit en väldigt tydlig fördelning av arbetsuppgifter inom krögartrion och denna struktur har fungerat väldigt bra.

\subsubsection*{Nolleutskottet}
När nollningen 2016 planerades fokuserade utskottet mycket på att få in en större mångfald av aktiviteter vilket också visade sig under genomförandet. Genom att i sedvanlig ordning genomföra en utvärdering efter nollningen, hoppas utskottet på att kunna skicka vidare förslag och synpunkter till nästa års NollU. Någon plan för renhållning av Sektionens lokaler togs aldrig fram och självklart var det lite rörigt under nollningen men efteråt städades det och även andra utskott, till exempel sexet och KM bidrog till att hålla så god ordning som möjligt under nollningen.
\newpage
\subsubsection*{Cafémästeriet}
Under 2016 har det kontinuerligt jobbats för att minska mängden svinn i LED, detta bland annat genom att ha förbättrat hur det är organiserat i förvaringsutrymmena då det på så vis lättare går att se huruvida något hinner gå ut.

Mojterna avskaffades inför verksamhetsår 2016 varpå dessa inte brukats alls.
Att utveckla och utvärdera sortimentet är något som gjordes konstant, dels genom förslag, men också genom att delta i en matmässa som vår leverantör anordnade.

\subsubsection*{Förvaltningsutskottet}
Under verksamhetsåret jobbade FVU framför allt med att förbereda ett nytt upplägg av funktionärer inom utskottet för att kunna jobba mer långsiktigt med lokalerna samtidigt som det kortsiktiga arbetet inte skulle påverkas. Vi jobbade även med styrelsen för att ta fram en långsiktig ekonomisk plan för Sektionen.

\subsubsection*{Studierådet}
Studierådet har fortsatt arbetet med att förbättra synligheten inom Sektionen. Vi har även arbetat med att ha fler pluggkvällar jämnt utspridda under året samt under nollningen ändrat på pluggkvällarnas struktur för att ge bättre studiero. SRE har infört ett lotteri för den som fyller i alla sina CEQ-enkäter, detta för att öka svarsfrekvensen.

\subsubsection*{Sexmästeriet}
Evenemangen har under året spridits ut bra. Efter uppstarten i början av årets så hölls tre (fyra om Skiphtet inkluderas) sittningar innan sommaren som Sektionens medlemmar kunde gå på. Under nollningen var det tätt med event och i läsperiod fyra hölls det ytterligare två evenemang som Sektionens medlemmar kunde ta del av. Överlag har bra ordning i Sexmästeriets förråd hållits då flera städningar gjorts. Under nollningen var det inte helt perfekt men det kan förklaras med att det var mycket folk i rörelse i förrådet samt att all äppeljuice förvarades där. I för utskotten gemensamma arbetsutrymmen har, även där, överlag bra ordning eftersträvas. De gånger ordningen varit under kritik har det åtgärdats.

\subsubsection*{Nöjesutskottet}
Vi har jobbat i utskottet för att förbättra postbeskrivningar men det är ännu inte helt perfekt. Jobbar fortfarande på detta. Testamente har skrivits för Utedishot så nästa år kommer det att vara enkelt och smidigt för de som tar över det ärofyllda uppdraget att arrangera festen. Vi arbetar även med att försöka få en bättre arbetsfördelning mellan alla i utskottet.


\subsubsection*{Näringslivsutskottet}
ENU har under året vårdat kontakterna med befintliga företag, samt utökat samarbetet med nya företag. Prissättning och blandning av vinstdrivande och icke-vinstdrivande event har diskuterats och utvärderats under hela året. Efterträdare på utskottsordförandeposten har aktivt deltagit i detta vilket lämpar sig bra inför kommande år. Uppdaterad hemsida är något ENU16 inte åtog sig.

\newpage
\begin{signatures}{10}
    \mvh
    \signature{Fredrik Peterson}{Ordförande 2016}
    \signature{Erik Månsson}{Kontaktor 2016}
    \signature{Anders Nilsson}{Förvaltningschef 2016}
    \signature{Stephanie Mirsky}{Cafémästare 2016}
    \signature{Molly Rusk}{Øverphøs 2016}
    \signature{Johan Persson}{SRE-ordförande 2016}
    \signature{Johannes Koch}{ENU-ordförande 2016}
    \signature{Martin Gemborn Nilsson}{Sexmästare 2016}
    \signature{Malin Lindström}{Krögare 2016}
    \signature{Dalia Khairallah}{Entertainer 2016}
\end{signatures}

\end{document}
