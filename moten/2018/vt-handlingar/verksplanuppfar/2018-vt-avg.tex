\documentclass[../_main/handlingar.tex]{subfiles}

\begin{document}
\verksplanuppf{VT 2018 (avgående styrelsen)}

\subsubsection*{Styrelsen}
Sedan HT/17 har Styrelsens fokus till största del legat på överlämning där caféets framtid har legat i fokus. I samråd med CM klubbades en proposition igenom på HT/17 angående införende av dagsansvariga i caféet, vilket verkar ha bidragit till en stabil grund i caféets verksamhet efter den blev ideell. Styrelsen har också jobbat för att engagera fler medlemmar i Teknikfokus. En viss förbättring har skett, men detta fortsätter vara en fråga för nästkommande styrelse(r) att jobba med. Vidare biföll styrelsens budgetförslag, vilket innebär stora satsningar för att ge tillbaka så mycket som möjligt till den gemene sektionsmedlemmen.

\subsubsection*{Informationsutskottet}

\subsubsection*{Källarmästeriet}
Gillena under 2017 har varit välbesökta men fortsätter framförallt att vara en pub av E:are för E:are. Den lilla skaran extra som besöker gillena är folk som E-sektionens medlemmar lyckas dra med för ett enstaka besök. Marknadsföring via Facebook och affischering i E-huset har pågått kontinuerligt och en allmän affisch har skapats men affischering i de övriga husen på LTH blev inte av.

Under 2017 uteblev tyvärr samarbete sektionerna emellan på grund av bristande intresse från krögartrion och brist på datum inom Sexkollegiet.

Priset på klägg höjdes med 5 kronor under 2017 för att försäkra att utrymme för experiment fanns samt att enbart svenskt kött serverades. Då priset på klägg stagnerat kring 30 kr under flera år var prishöjningen motiverad. Dryck har funnits i ett brett spektrum och gäster har haft möjlighet att köpa allt från billig uppfriskande öl till dyrare, mer komplex dryck för de som känt sig manade till det. När det kommer till svinn så har KM jobbat aktivt för att minska diffen men då spritförrådet delas av två utskott och rigorös utbildning varierar mellan funktionärer av båda utskotten så är det i princip omöjligt att förhindra att svinn uppstår.

\subsubsection*{Nolleutskottet}

\subsubsection*{Cafémästeriet}
Under hela året har vi jobbat för att caféet ska vara fortsatt konkurrenskraftigt och bibehålla studentvänliga priser. Sedan höstterminsmötet är det framförallt ett av delmålen vi jobbat med och det är att utvärdera posterna och dess beskrivningar. I samband med personalkrisen i höstas samt vetskapen om att Ulla skulle gå i pension till våren har vi jobbat hårt för att strukturera om verksamheten så att den ska vara ideellt genomförbar. Många av de dåvarande posterna har uppdaterats och nya poster med tillhörande beskrivningar har införts.

Utöver detta har vi jobbat för att minska svinn i både LED och förrådet. För att uppnå detta har vi börjat använda datummärkning på alla varor samt strukturerat upp våra lagerutrymmen så att allt är mer lättillgängligt.

Att utöka caféets öppettider i dagsläget omöjligt då vi knappt kan hålla de vanliga öppettiderna. Detta kommer vara ett ännu större problem när verksamheten nästa verksamhetsår ska drivas helt ideellt och kommer behövas utvärderas ytterligare.

\subsubsection*{Förvaltningsutskottet}

\subsubsection*{Studierådet}
Studierådet har arbetat med överlämningen till årets utskott, så som att lära ut hur vi censurerar CEQ-rapporter och hur arbetsprocessen med dessa går till. Utöver detta utfördes det kontinuerliga arbetet med CEQ-behandling och möten.

\subsubsection*{Sexmästeriet}
Sexmästeriet har under hösten som gick hållit i ett antal event och sittningar och har vid varje tillfälle strävat efter (och lyckats, tycker vi) att hålla hög kvalité på varje event. Av naturliga skäl skedde flest evenemang under nollningen, men en sittning blev även av efteråt. Utskottet hade planerat att hålla i ett afternoon tea-event också, men det blev tyvärr inställt då intresset var lågt. Vi har ansträngt oss för att hålla ordning i Pump och då det blivit stökigt har vi tagit tag i det och städat upp. Vi har också jobbat för att minimera svinn i alkohollagret.

\subsubsection*{Nöjesutskottet}
Under året har utskottet arbetat för att sprida arbetsbelastningen inom utskottet bättre. Ett sätt att göra det på var att uppdatera postbeskrivningarna så att alla visste vad sin post innebar vilket genomfördes. I samband med uppdateringen av postbeskrivningarna togs även posten ”speleman” bort. Deras tidigare ansvarsområde ansvarar nu fritidsledarna för istället.

Det gjordes ytterligare en utveckling inom utskottet. Vice entertainer började att arbeta närmre entertainern och bistå denne i sitt arbete. En förändring som gjordes var att ge vice entertainer en vice-kavaj. Det ledde till att vice entertainer syntes mer under evenemang och automatiskt tog ett större ansvar. Det var en uppskattad förändring som vi valt att hålla kvar vid kommande år.

\subsubsection*{Näringslivsutskottet}
Under året var ENU i kontakt med både befintliga och nya företag. Prissättningen sågs över både i början av våren och efter sommaren. Olika aktiviteter diskuterades och utvärderades under mötena. Den nya företagshemsidan kom tyvärr inte upp.

\newpage
\begin{signatures}{10}
    \mvh
    \signature{Erik Månsson}{Ordförande 2017}
    \signature{Johan Karlberg}{Kontaktor 2017}
    \signature{Sophia Grimmeiss Grahm}{Förvaltningschef 2017}
    \signature{Daniel Bakic}{Cafémästare 2017}
    \signature{Niklas Gustafson}{Øverphøs 2017}
    \signature{Edvard Carlsson}{SRE-ordförande HT 2017}
    \signature{Josefine Sandström}{ENU-ordförande 2017}
    \signature{Linnea Sjödahl}{Sexmästare 2017}
    \signature{Markus Rahne}{Krögare 2017}
    \signature{Albin Nyström Eklund}{Entertainer 2017}
\end{signatures}

\end{document}
