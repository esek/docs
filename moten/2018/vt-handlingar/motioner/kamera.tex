\documentclass[../_main/handlingar.tex]{subfiles}

\begin{document}
\motion{Inköp av en ny sektionskamera}

Den nuvarande kamerautrustningen som finns på sektionen klarar inte av miljöer med dåligt ljus eller med kort skärpedjup för bra porträtt, till exempel på den årliga CV-fotograferingen som sektionen håller i. 

Jag tycker att de sektionsmedlemmar som har ett intresse för fotografering ska få möjligheten att fotografera med bra och kapabel utrustning. Man ska inte behöva använda egen bra utrustning som är värd tusentals kronor för att kunna fotografera i mer utmanande miljöer.

Jag har tittat på olika kameror och rekommenderar en Canon 6D i fint skick (går att köpa begagnat på blocket) där priset pendlar mellan 6500-7500 kronor. Detta är en fullformatskamera med funktioner som är lämpade för miljöer med dåligt ljus.
Jag hade tänkt att man kan köpa till två objektiv, ett Canon 85 f/1.8 (kostar runt 2000 kr) för porträtt och sport, och ett allround-objektiv 24-105 f/4L som kostar runt 5000 kr. Ett till objektiv som jag rekommenderar är Canons 70-200 f/4L som kostar runt 5000 kr också. Totalt skulle all utrustning hamna på cirka 20 000. Detta inkluderar då 3-4 objektiv och ett fullformatshus.

Därför yrkar jag på

\begin{attsatser}
    \att sektionen ska köpa in en ny kamera och 3-4 objektiv till en total kostnad av \SI{20000}{kr}, 
    \att kostnaden belastar utrustningefonden, samt,
    \att detta läggs på beslutsuppföljningen till HT/18 med undertecknad som ansvarig.
\end{attsatser}

\begin{signatures}{1}
    \mvh
    \signature{Eltayeb Bayomi}{Fotograf 2018}
\end{signatures}

\end{document}
