\documentclass[../_main/handlingar.tex]{subfiles}

\begin{document}
\motion{Internationell Nolleguide}

Nolleguiden är ett fantastiskt verktyg för alla nya i Lund. Den är fylld med givande och rolig information om Lund, E-sektionen, studentlivet och såklart själva nollningen.
 
Idag finns det ingen ordentlig guide för de nya internationella studenterna, de får ett välkomstbrev avsett för alla internationella på LTH från teknologkåren med en kortare text om E-sektionen. Denna är inte alls lika välkomnade som vår nolleguide och framförallt inte lika informerande. Med en internationell nolleguide anser jag att intresset för nollning såväl som sektionen i allmänhet hade ökat bland de internationella studenterna. Dessutom hade det avlastat och hjälpt phaddrarna i deras redan tunga arbete. 
 
Med ovanstående i åtanke yrkar jag på  


\begin{attsatser}
    \att avsätta 4000 kr till en nolleguide för de internationella studenterna, 
    \att kostnaden belastar utrustningefonden, samt,
    \att detta läggs på beslutsuppföljningen till HT/18 med undertecknad som ansvarig.
\end{attsatser}

\begin{signatures}{1}
    \mvh
    \signature{Edvard Carlsson}{Cophøs 2018}
\end{signatures}

\end{document}
