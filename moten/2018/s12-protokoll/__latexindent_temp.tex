\documentclass[10pt]{article}
\usepackage[utf8]{inputenc}
\usepackage[swedish]{babel}
    
\def\mo{Daniel Bakic}
\def\ms{Axel Voss}
\def\ji{Adam Belfrage}
    %\def\jii{}
    
    \def\doctype{Protokoll} %ex. Kallelse, Handlingar, Protkoll
    \def\mname{styrelsemöte} %ex. styrelsemöte, Vårterminsmöte
    \def\mnum{S12/18} %ex S02/16, E1/15, VT/13
    \def\date{2018-05-09} %YYYY-MM-DD
    \def\docauthor{\ms}
    
    \usepackage{../e-mote}
    \usepackage{../../../e-sek}
    
    \begin{document}
    \showsignfoot
    
    \heading{{\doctype} för {\mname} {\mnum}}
    
    %\naun{}{} %närvarane under
    %\nati{} %närvarande till och med
    %\nafr{} %närvarande från och med
    \section*{Närvarande}
    \subsection*{Styrelsen}
    \begin{narvarolista}
        \nv{Ordförande}{Daniel Bakic}{E15}{}
        \nv{Kontaktor}{Axel Voss}{E15}{}
        \nv{Förvaltningschef}{Magnus Lundh}{E15}{}
        \nv{Cafémästare}{Elin Johansson}{BME16}{}
        \nv{Øverphøs}{Andreas Bennström}{BME16}{}
        \nv{SRE-ordförande}{Fanny Månefjord}{BME16}{}
        \nv{ENU-ordförande}{Isabella Hansen}{E16}{}
        \nv{Sexmästare}{Alexander Wik}{BME17}{}
        \nv{Krögare}{Malin Heyden}{E16}{}
        \nv{Entertainer}{Adam Belfrage}{BME17}{}
    \end{narvarolista}
    \subsection*{Ständigt adjungerande}
    
    
    \begin{narvarolista}
        %\nv{Inköps- och lagerchef}{Sofie Johannesson}{E17}{}
        %\nv{Inköps- och lagerchef}{Fabian Sondh}{E17}{}
        %\nv{Inköps- och lagerchef}{Albin Pålsson}{E17}{}
        %\nv{Kårordförande}{Linus Hammarlund}{}{}
        %\nv{Kårrepresentant}{Jacob Karlsson}{}{\nafr{3}}
        %\nv{Kårrepresentant}{Agnes Sörliden}{}{}
        %\nv{Valberedningens ordförande}{Pontus Landgren}{}{}
        %\nv{Skattmästare}{Olle Oswald}{}{}
        %\nv{Kårrepresentant}{Daniel Damberg}{}{}
        %\nv{Kårrepresentant}{John Alvén}{}{}
        \nv{Nollegeneral}{Jakob Nilsson}{}{}
        %\nv{Skyddsombud}{Axel Sandqvist}{E17}{}
        %\nv{Saga}{}{}{}
        %\nv{Max}{}{}{}
    
        %\nv{Talman}{Erik Månsson}{E14}{}
        %\nv{Elektras Ordförande}{Elisabeth Pongratz}{}{}
        %\nv{Inspektor}{Monica Almqvist}{}{}
        \nv{Sigillbevarare}{Henrik Ramström}{}{}
        %\nv{Vice Entertainer}{Emil Bergström}{}{}
    \end{narvarolista}
    
    \begin{comment}
    \subsection*{Adjungerande}
    \begin{narvarolista}
        %\nv{Post}{Namn}{Klass}{}
    \end{narvarolista}
    \end{comment}
    
    \section*{Protokoll}
    \begin{paragrafer}
        \p{1}{OFMÖ}{\bes}
        Ordförande {\mo} förklarade mötet öppnat 12:15.
        
        \p{2}{Val av mötesordförande}{\bes}
        {\valavmo}
        
        \p{3}{Val av mötessekreterare}{\bes}
        {\valavms}
        
        \p{4}{Val av justeringsperson}{\bes}
        {\valavj}
        
        \p{5}{Godkännande av tid och sätt}{\bes}
        {\tosg}
        
        \p{6}{Adjungeringar}{\bes}
        {\ingaadj}
        %Förnamn Efternamn adjungerades
        Johan Sievert lindeskog
        Axel Sondh
        
        \p{7}{Godkännande av dagordningen}{\bes}
        
        Adam \ypa lägga till ``Utedischo'' efter äska \S12
        Andy \ypa lägga till `unionen' efter \S15
        %Föredragningslistan godkändes med yrkandet.
        Föredragningslistan godkändes med samtliga yrkanden.
        
        \p{8}{Föregående mötesprotokoll}{\bes}
        %\latillprot{}
        \ingaprot
        
        \p{9}{Fyllnadsval och entledigande av funktionärer}{\bes}
        \begin{fyllnadsval} %"Inga fyllnadsval." fylls i automatiskt
        \end{fyllnadsval}
        
        \p{10}{Rapporter}{}
        \begin{paragrafer}
            \subp{A}{Hur mår alla?}{\info}
            Punkten protokollfördes ej.
            
            \subp{B}{Utskottsrapporter}{\info}
            Malin har inte gjort så mycket.
            
            Magnus har bokfört, hustomtarna har städat sicrit.
    
            Nöju har planerat lite inför nollningen. De hade spelkväll igår och det gick bra. 
    
            Alexander och sexet hade temasläpp i fredags. Denna veckan har de planerat mer inför nollningen.
    
            Isabella har börjat skriva avtal med academic work, de ska ha två lunchföreläsningar. De har börjat planera ``lunch med en ingenjör''.
    
            CM och Elin har sålt mycket glass och läsk. De hade café-festen igår, det gick bra.
    
            Fanny har börjat planera pluggkvällar inför nollningen. Ingrid vill att SRE ska närvara under mottagningen.
    
            Andy och phøset har haft temasläpp. Ikväll så är det phadderkickoff.
    
            Daniel har varit på ok-möte. Det finns ingen sås-general så om någon är intresserad av detta är det bara att söka. Daniel påminner om att han vill ha utvärdering av styrelsearbetet.
    
            \subp{C}{Ekonomisk rapport}{\info}
            Ekonomin ser väldigt bra ut enligt Magnus!
            
            \subp{D}{Kåren informerar}{\info}
            
            Sista fm är idag. Eventuellt kommer nya bord och ny takbelysning inhandlas till gasque. 
    
            \subp{E}{Omvärldsrapport}{\info}
            
            Denna vecka finns det ingenting nytt att rapportera.
            
            
        \end{paragrafer}
        
        \p{11}{Äska pengar till skärmar}{\dis}
        Axel presenterade 
    
        \Mbaby
    
        \p{12}{Äska pengar till Ölresa}{\dis}
    
        Adam presenterade motionen
    
        mötet diskuterade frågan
    
        Adam yrkade på att skjuta fram yrkandet till efter sommaren
    
        \p{13}{Utedischo}{\dis}
        
        Adam lyfte att det är relativt få från esektionen som är anmälda till att jobba under Utedischo. Han undrar om det är någon från styrelsen som vill jobba under eventet.
    
        \p{14}{Utskottsmärken}{\dis}
      
        Henrik lyfte frågan om Utskottsmärken.
    
        Mötet dis frågan.
    
        Han vill att alla fortsätterfundera på märken till sina utskott. 
    
        \p{15}{Unionen}{\dis}
        
        Andreas tog upp att de under nollningen eventuell vill att unionen ska få ha en station där de bjuder på mat under ett event tillsammans med I.
        
        Mötet diskuterade frågan 
    
        \p{12}{Nästa styrelsemöte}{\bes}
        \Mba nästa styrelsemöte ska äga rum nästa onsdag 2018-05-17 klockan 12:10 i E:1124.
          
        
        \p{14}{Beslutsuppföljning}{\bes}
        
        Axel \ypa stryka ``XxX'' från beslutsuppföljningen. 
    
        \Mbaby 
    
        Malin \ypa skjuta upp ``Kaffekokare''
    
        \Mbaby 
        
        Axel \ypa stryka ``Biljardaccess'' från beslutsuppföljningen.
    
        \Mbaby 
        
        Adam \ypa stryka ``Rensa skåpet i biljard'' från beslutsuppföljningen.
    
        \Mbaby 
        
        Adam \ypa stryka ``Se över EKEA'' från beslutsuppföljningen.
        
        \Mbaby 
        
        Magnus \ypa skjuta upp ``Inköp av soundboks'' S14.
        
        \Mbaby 	
        %\Ibfu
        \p{15}{Övrigt}{\dis}
        
        Elin tog upp att de med access till LED måste vara otroligt noga med att låsa.
        
        Adam vill att styrelsen ska ställa upp i på styret. 
    
    
        \p{16}{OFMA}{\bes}
        {\mo} förklarade mötet avslutat 13:00.
    \end{paragrafer}
    
    %\newpage
    \hidesignfoot
    \begin{signatures}{3}
        \signature{\mo}{Mötesordförande}
        \signature{\ms}{Mötessekreterare}
        \signature{\ji}{Justerare}
    \end{signatures}
    \end{document}
    