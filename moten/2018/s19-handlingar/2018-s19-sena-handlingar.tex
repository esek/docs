\documentclass[10pt]{article}
    \usepackage[utf8]{inputenc}
    \usepackage[swedish]{babel}
    
    \def\doctype{Sena handlingar} %ex. Kallelse, Handlingar, Protkoll
    \def\mname{styrelsemöte} %ex. styrelsemöte, Vårterminsmöte
    \def\mnum{S19/18} %ex S02/16, E1/15, VT/13
    \def\date{2018-09-27} %YYYY-MM-DD
    \def\docauthor{Daniel Bakic}
    
    \usepackage{../e-mote}
    \usepackage{../../../e-sek}
    
    \begin{document}
    
    \heading{{\doctype} till {\mname} {\mnum}}
    
    \section*{Information om punkterna}
    
    \begin{paragrafer}
    
    \p{12}{Expo}{\dis}
    Skulle gärna vilja diskutera upplägget kring Expot så att vi har en plan.

     \rnamnpost{Daniel Bakic}{Ordförande}

    \p{13}{Funktionärstack}{\dis}
    Jag skulle vilja lyfta frågan om hur vi vill göra med funktionärstacket i år. Tidigare år har det spelats Laserdome i Malmö under dagen och sedan har det bjudits på nationssittning + eftersläpp på kvällen. Det här året har vi en större budget till tacket och kan göra något riktigt sjukt (om vi vill). Fundera gärna på vad man kan göra och framförallt när det borde hållas.

        \rnamnpost{Daniel Bakic}{Ordförande}

    \p{14}{Samarbete med WEIQ}{\dis}
     I tisdags (18-09-25) hade jag, Magnus och Malin möte med CMO från företaget WEIQ som utvecklar orderhanteringssystem. De är ute efter så kallade VIP-kunder som skulle vilja testa deras produkt och hjälpa de komma igång på marknaden. Som VIP-kund får man använda produkten gratis mot att man ger ständig feedback över hur produkten fungerar på marknaden samt updatera om vad som är bra eller mindre bra med den. Om vi väljer att bli en av deras VIP-kunder är tanken att vi ska börja använda deras orderhanteringssystem på gillena. Systemet går ut på att man i en applikation ska beställa produkter i baren och sen när beställningen är klar ska man få en push-notis, detta skulle då reducera fysiskt köande i Edekvata.\newline
     Jag, Magnus och Malin verkade positivt inställda till att testa systemet och vill gärna höra era synpunkter kring det. Bifogar WEIQ:s presentation som bilaga så ni kan läsa på lite mer om systemet.
     
     \rnamnpost{Daniel Bakic}{Ordförande}
    \end{paragrafer}
    
    \begin{signatures}{1}
    \ist
    \signature{\docauthor}{Ordförande}
    \end{signatures}
    
    \end{document}
    