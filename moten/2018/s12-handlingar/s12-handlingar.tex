\documentclass[10pt]{article}
    \usepackage[utf8]{inputenc}
    \usepackage[swedish]{babel}
    
    \def\doctype{Handlingar} %ex. Kallelse, Handlingar, Protkoll
    \def\mname{styrelsemöte} %ex. styrelsemöte, Vårterminsmöte
    \def\mnum{S12/18} %ex S02/16, E1/15, VT/13
    \def\date{2017-02-09} %YYYY-MM-DD
    \def\docauthor{Axel Voss}
    
    \usepackage{../e-mote}
    \usepackage{../../../e-sek}
    
    \begin{document}
    
    \heading{{\doctype} till {\mname} {\mnum}}
    
    \section*{Äskning av pengar till extra skärm i Edekvata}
    
    \subsection*{Bakgrund}
    Under året har KM använt sig av det nya kläggbiljettsystem och det verkar tillsynes vara populärt. Att kunna se vilka klägg som är färdiga utan att behöva gå fram till bardisken minskar både köbildning framme vid baren samt spring i lokalen. Detta bidrar till en trevligare stämning på våra mysiga gillen. 
    
    Det har dock uppstått viss problematik då man under året har haft ett fåtal gillen där man använt den stora tvskärmen till att visa annat än biljettsystemet, exempelvis under teknikfokuspuben eller OS-gillet. 
    Jag anser därför att det är rimligt att sektionen köper in en till skärm av samma modell som de två som redan hänger i baren. Denna skärm kommer framförallt användas till att visa biljettsystemet på kvällar efter klockan 17.00 men kan såklart komma att användas till annat också. Under dagtid behöver denna skärm inte vara avstängd utan kan förslagsvis visa avgående bussar från hållplatsen utanför kårhuset. 

    En extra skärm gör att biljettsystemet alltid kommer kunna användas fullt ut, även om den stora skärmen visar annat. Skärmen kommer även göra att de som sitter mot baren slipper vända sig om för att se vilka klägg som är färdiga.
    \subsection*{Implementation}
    Skärmen kommer monteras på samma sätt som de två som redan hänger ovanför baren. Skillnaden är att denna skärm kommer placeras på sidan mot diplomat och inte på sidan mot CM. Skärmen kommer placeras långt in mot hörnet och riktas snett utåt så att det ska vara lätt att se den från alla bord i edekvata. Raspberryn kommer ligga i taket och vara kopplad med en HDMI-sladd till skärmen. 

    \subsection*{Förslag}
    
    \begin{attsatser}
        \att köpa in 1st 27”-skärm av samma modell som de i baren med tillhörande VESA-fästen (1990-2490kr/skärm),
        \att köpa in 1st Raspberry Pi 3 Kit, d.v.s. Raspberry Pi, micro-sd, case och laddare (749-817kr/kit),
        \att köpa in 1st väggfästen (199-379kr/st),
        \att budgeten sätts till 4000kr och belastar dispositionsfonden, samt 
        \att detta läggs på beslutsuppföljning till S13/18 med undertecknad som ansvarig. 
    \end{attsatser}

    \begin{signatures}{1}
    \ist
    \signature{\docauthor}{Kontaktor}
    \end{signatures}
    
    \end{document}
    