\documentclass[10pt]{article}
    \usepackage[utf8]{inputenc}
    \usepackage[swedish]{babel}
    
    \def\doctype{Handlingar} %ex. Kallelse, Handlingar, Protkoll
    \def\mname{styrelsemöte} %ex. styrelsemöte, Vårterminsmöte
    \def\mnum{S18/18} %ex S02/16, E1/15, VT/13
    \def\date{2018-09-21} %YYYY-MM-DD
    \def\docauthor{Daniel Bakic}
    
    \usepackage{../e-mote}
    \usepackage{../../../e-sek}
    
    \begin{document}
    
    \heading{{\doctype} till {\mname} {\mnum}}
    
    \section*{Information om punkterna}
    
    \begin{paragrafer}
    
    \p{12}{Projektorer}{\dis}
     På husstyrelsemötet i fredags diskuterades behovet av projektorer i salar som är mindre till storleken. Husstyrelsen funderar på att installera fler projektorer i E-husets salar men behöver oss studenters och undervisares hjälp till att klargöra vad det finns för behov. Olika aspekter att ta hänsyn till kan vara ifall det också finns behov av tavlor i rummen, ifall man ska kunna ha projektorn igång och samtidigt kunna skriva på tavlan, o.s.v. Vill att alla tänker till och sen kan vi diskutera vad vi tycker samt komma fram till hur vi bör sprida detta vidare för mer indata.  
    
     \rnamnpost{Daniel Bakic}{Ordförande}

    \p{13}{Sophantering}{\dis}
        På senare tid har soporna sköts sämre än vad de borde och vi har fått klagomål från huset. Oftast gäller det att sopsäckar fyllda med sopor lämnats kvar på ställen de inte ska vara istället för att slängas. Detta leder till att problem med råttor uppstår samt mer (onödigt) jobb för husets vaktmästare. 
        
        \rnamnpost{Daniel Bakic}{Ordförande}

    \end{paragrafer}
    
    \begin{signatures}{1}
    \ist
    \signature{\docauthor}{Ordförande}
    \end{signatures}
    
    \end{document}
    