\documentclass[10pt]{article}
\usepackage[utf8]{inputenc}
\usepackage[swedish]{babel}

\def\mo{Daniel Bakic}
\def\ms{Axel Voss}
\def\ji{Magnus Lundh}
%\def\jii{}

\def\doctype{Protokoll} %ex. Kallelse, Handlingar, Protkoll
\def\mname{styrelsemöte} %ex. styrelsemöte, Vårterminsmöte
\def\mnum{S18/18} %ex S02/16, E1/15, VT/13
\def\date{2018-09-21} %YYYY-MM-DD
\def\docauthor{\ms}

\usepackage{../e-mote}
\usepackage{../../../e-sek}

\begin{document}
\showsignfoot

\heading{{\doctype} för {\mname} {\mnum}}

%\naun{}{} %närvarane under
%\nati{} %närvarande till och med
%\nafr{} %närvarande från och med
\section*{Närvarande}
\subsection*{Styrelsen}
\begin{narvarolista}
    \nv{Ordförande}{Daniel Bakic}{E15}{}
    \nv{Kontaktor}{Axel Voss}{E15}{}
    \nv{Förvaltningschef}{Magnus Lundh}{E15}{}
    \nv{Cafémästare}{Elin Johansson}{BME16}{}
    %\nv{Øverphøs}{Andreas Bennström}{BME16}{}
    \nv{SRE-ordförande}{Fanny Månefjord}{BME16}{}
    \nv{ENU-ordförande}{Isabella Hansen}{E16}{}
    \nv{Sexmästare}{Alexander Wik}{BME17}{}
    \nv{Krögare}{Malin Heyden}{E16}{}
    \nv{Entertainer}{Adam Belfrage}{BME17}{}
\end{narvarolista}
\subsection*{Ständigt adjungerande}
        
\begin{narvarolista}
    \nv{Kårrepresentant}{Philip Johansson}{}{}
\end{narvarolista}
    
\begin{comment}
\subsection*{Adjungerande}
\begin{narvarolista}
    %\nv{Post}{Namn}{Klass}{}
\end{narvarolista}
\end{comment}
    
\section*{Protokoll}
\begin{paragrafer}
    \p{1}{OFMÖ}{\bes}
    Ordförande {\mo} förklarade mötet öppnat 12:15.
    
    \p{2}{Val av mötesordförande}{\bes}
    {\valavmo}

    \p{3}{Val av mötessekreterare}{\bes}
    {\valavms}

    \p{4}{Val av justeringsperson}{\bes}
    {\valavj}

    \p{5}{Godkännande av tid och sätt}{\bes}
    {\tosg}

    \p{6}{Adjungeringar}{\bes}
    {\ingaadj}
    %Förnamn Efternamn adjungerades

    \p{7}{Godkännande av dagordningen}{\bes}

    Mangus Lundh \ypa lägga till \S14 ``Faktura till F-Sektionen''

    Föredragningslistan godkändes med yrkandet.

    %Föredragningslistan godkändes med samtliga yrkanden.

    \p{8}{Föregående mötesprotokoll}{\bes}
    %\latillprot{}
    \ingaprot

    \p{9}{Fyllnadsval och entledigande av funktionärer}{\bes}
    \begin{fyllnadsval} %"Inga fyllnadsval." fylls i automatiskt
        \fval{David Karlsson}{Diod}
        \fval{Saga Junivik}{Diod}
        \fval{Tom Andersen}{Källarmästare}
        \fval{Theo Nyman}{Årskurs BME-1 ansvarig}
    \end{fyllnadsval}

    \p{10}{Rapporter}{}
    \begin{paragrafer}
        \subp{A}{Hur mår alla?}{\info}
        Punkten protokollfördes ej.

        \subp{B}{Utskottsrapporter}{\info}
        Idrottsförmännen hade phaddergruppsolympiad, det va lyckat! Nöju har gjort en krisplan om vädret inte håller för agent00E.    

        E6 hade karnevalssittning tillsammans med K och A. Sittningen gick superbra. De håller fortfarande på för fullt med gasque.

        SRE har haft en till pluggkväll. Det var lyckat. Nästa pluggkväll är nu på måndag.

        ENU hade ett event tillsammans med Academic work och SRE i samband med Pluggkvällen.

        Axel har fixat anmodningar och inbjudningar. Samt skrivit protokoll. Ikväll har InfU sin episka kickoff.

        Malin har Neongille ikväll. De kommer servera tequila sunrise.

        LED café går bra. Det är phaddrar som jobbar i led denna veckan och de har varit duktiga!

        Magnus har bokfört mycket och betalat utlägg.

        Daniel har planerat inför höstterminsmöte och valmötet. Han ska även prata med Pontus angående utskotts-Expo.
        Husstyrelsen kommer ersätta ventilationsknapparna med sensorer.

        \subp{C}{Ekonomisk rapport}{\info}

        Ekonomin ser bra ut enligt Magnus!

        \subp{D}{Kåren informerar}{\info}
        Det har kommit massa poster man kan söka! Det finns på både hemsidan och kårens facebook. 

        \subp{E}{Omvärldsrapport}{\info}
        Axel informerade om anmodningar till gasque för KTH och Chalmers. De är supertaggade på att komma till oss.
    
        %Denna vecka finns det ingenting nytt att rapportera.
    \end{paragrafer}
        
    \p{11}{Olämliga sånger}{\dis}

    Fanny ville diskutera olämpliga sånger.

    Mötet diskuterade frågan.

    \p{12}{Projektorer}{\dis}
    
    Daniel informerade om projektorer.

    Mötet diskuterade frågan. 

    Axel tycker att man kan använda bordsplacerade projektorer istället. Då behöver man inte montera projektorer i varje rum.

    Fanny ska ta upp det på ett studierådsmöte.

    \p{13}{Sophantering}{\dis}

    PH har haft möte med några i styrelsen angående sophantering. Det har inte skötts bra och måste tas tag i. 
    Berörda utskott pratar med sina jobbare. 

    \p{14}{Faktura till F-sektionen}{\dis}
    Axel kom med förslaget att trycka fakturan på en flagga.

    Daniel \ypa ``Flagga till F-sektionen'' läggs till Beslutsuppföljning för nästa styrelsemöte.

    \Mbaby

    \p{15}{Nästa styrelsemöte}{\bes}

    \Mba nästa styrelsemöte ska äga rum nästa onsdag 2018-09-27 klockan 12:10 i E:1124.
    
    \p{16}{Beslutsuppföljning}{\bes}
    
    Adam \ypa att stryka ``Riktlinjer för städning av FikaFika''.

    \Mbaby 	

    Daniel \ypa skjuta upp ``Design styrelsemärke''.

    \Mbaby  
    %\Ibfu

    \p{17}{Övrigt}{\dis}
    Elin tog upp att vi borde anmäla till kåren att vi inte kan vara där under styrelseutbildningen eftersom vi är bjudna till Chalmers. 

    Daniel vill hålla expo den 2018-10-11.

    \p{18}{OFMA}{\bes}
    {\mo} förklarade mötet avslutat 13:04.
\end{paragrafer}
    
%\newpage
\hidesignfoot
\begin{signatures}{3}
    \signature{\mo}{Mötesordförande}
    \signature{\ms}{Mötessekreterare}
    \signature{\ji}{Justerare}
\end{signatures}
\end{document}
    