\documentclass[10pt]{article}
    \usepackage[utf8]{inputenc}
    \usepackage[swedish]{babel}
    
    \def\doctype{Sena handlingar} %ex. Kallelse, Handlingar, Protkoll
    \def\mname{styrelsemöte} %ex. styrelsemöte, Vårterminsmöte
    \def\mnum{S20/18} %ex S02/16, E1/15, VT/13
    \def\date{2018-10-04} %YYYY-MM-DD
    \def\docauthor{Daniel Bakic}
    
    \usepackage{../e-mote}
    \usepackage{../../../e-sek}
    
    \begin{document}
    
    \heading{{\doctype} till {\mname} {\mnum}}
    
    \section*{Information om punkterna}
    
    \begin{paragrafer}
    
    \p{12}{Flagga till F}{\bes}
    Föregående möten har denna punkt varit på beslutsuppföljningen. Tanken var då att samtliga ledamöter ska ha kollat runt på olika alternativ. Det har kollats runt en del på olika alternativ men i mån av tid har inget riktigt beslut tagits. Nu är det dags att ta ett ordentligt beslut.

     \rnamnpost{Daniel Bakic}{Ordförande}

    \p{13}{UtEDischo-tackphest}{\dis}
    Lophtet är sedan lång tid tillbaka bokat för en tackphest för de funktionärer som arbetade under UtEDischot. Dock har tillståndsenheten på senare tid valt att göra en hårdare tolkning av alkohollagen vilket resulterat i att det i princip är omöjligt för oss att hålla i tillställningar av den typen utan att ha serveringstillstånd. Tackphesten är inplanerad den 13:e oktober. Hur tar vi ställning till detta?  

        \rnamnpost{Adam Belfrage}{Entertainer}

    \end{paragrafer}
    
    \begin{signatures}{1}
    \ist
    \signature{\docauthor}{Ordförande}
    \end{signatures}
    
    \end{document}