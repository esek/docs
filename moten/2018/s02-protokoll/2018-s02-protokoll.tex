\documentclass[10pt]{article}
\usepackage[utf8]{inputenc}
\usepackage[swedish]{babel}

\def\mo{Daniel Bakic}
\def\ms{Axel Voss}
\def\ji{Fanny Månefjord}
%\def\jii{}

\def\doctype{Protokoll} %ex. Kallelse, Handlingar, Protkoll
\def\mname{styrelsemöte} %ex. styrelsemöte, Vårterminsmöte
\def\mnum{S02/18} %ex S02/16, E1/15, VT/13
\def\date{2018-02-01} %YYYY-MM-DD
\def\docauthor{\ms}

\usepackage{../e-mote}
\usepackage{../../../e-sek}

\begin{document}
\showsignfoot

\heading{{\doctype} för {\mname} {\mnum}}

%\naun{}{} %närvarane under
%\nati{} %närvarande till och med
%\nafr{} %närvarande från och med
\section*{Närvarande}
\subsection*{Styrelsen}
\begin{narvarolista}
\nv{Ordförande}{Daniel Bakic}{E15}{}
\nv{Kontaktor}{Axel Voss}{E15}{}
\nv{Förvaltningschef}{Magnus Lundh}{E15}{}
\nv{Cafémästare}{Elin Johansson}{BME16}{\nati{19}}
\nv{Øverphøs}{Andreas Bennström}{BME16}{\nati{19}}
\nv{SRE-ordförande}{Fanny Månefjord}{BME16}{\nati{19}}
\nv{ENU-ordförande}{Isabella Hansen}{E16}{}
\nv{Sexmästare}{Alexander Wik}{BME17}{}
\nv{Krögare}{Malin Heyden}{E16}{}
\nv{Entertainer}{Adam Belfrage}{BME17}{}
\end{narvarolista}
\subsection*{Ständigt adjungerande}


\begin{narvarolista}
%\nv{Inköps- och lagerchef}{Sofie Johannesson}{E17}{}
%\nv{Inköps- och lagerchef}{Fabian Sondh}{E17}{}
%\nv{Inköps- och lagerchef}{Albin Pålsson}{E17}{}
\nv{Kårrepresentant}{Agnes Sörliden}{}{}
%\nv{Kårordförande}{Linus Hammarlund}{}{}
%\nv{Kårrepresentant}{Jacob Karlsson}{}{\nafr{3}}
%\nv{Valberedningens ordförande}{Elin Magnusson}{}{}
%\nv{Skattmästare}{Olle Oswald}{}{}
%\nv{Kårrepresentant}{Daniel Damberg}{}{}
%\nv{Kårrepresentant}{John Alvén}{}{}
\nv{Nollegeneral}{Jakob Nilsson}{}{}
%\nv{Talman}{Erik Månsson}{E14}{}
%\nv{Elektras Ordförande}{Elisabeth Pongratz}{}{}
%\nv{Inspektor}{Monica Almqvist}{}{}
\nv{Sigillbevarare}{Henrik Ramström}{}{}
\end{narvarolista}

\begin{comment}
\subsection*{Adjungerande}
\begin{narvarolista}
%\nv{Post}{Namn}{Klass}{}
\end{narvarolista}
\end{comment}

\section*{Protokoll}
\begin{paragrafer}
\p{1}{OFMÖ}{\bes}
Ordförande {\mo} förklarade mötet öppnat 12:10.

\p{2}{Val av mötesordförande}{\bes}
{\valavmo}

\p{3}{Val av mötessekreterare}{\bes}
{\valavms}

\p{4}{Val av justeringsperson}{\bes}
{\valavj}

\p{5}{Godkännande av tid och sätt}{\bes}
{\tosg}

\p{6}{Adjungeringar}{\bes}
{\ingaadj}
%Förnamn Efternamn adjungerades


\p{7}{Godkännande av dagordningen}{\bes}
Dagordningen godkändes
%Fredrik \ypa att lägga till \S18b ``Teknikfokus utnyttjande av LED-café''.

%Föredragningslistan godkändes med yrkandet.
%Föredragningslistan godkändes med samtliga yrkanden.

\p{8}{Föregående mötesprotokoll}{\bes}
\latillprot{S30/17 och S01/18}
%\ingaprot

\p{9}{Fyllnadsval och entledigande av funktionärer}{\bes}
\begin{fyllnadsval} %"Inga fyllnadsval." fylls i automatiskt
%\fval{Namn}{Post}
%\entl{Namn}{Post}
\fval{Jonas Thurborg}{Diod}
\fval{Ivar Söderberg}{Diod}
\fval{Gabriela Medina}{Halvledare}
\fval{Jonathan Benitez}{Picasso}
\entl{Eltayeb Bayomi}{Diod}
\entl{Gabriela Medina}{Diod}
\end{fyllnadsval}

\p{10}{Rapporter}{}
\begin{paragrafer}
\subp{A}{Hur mår alla?}{\info}
Punkten protokollfördes ej.

\subp{B}{Utskottsrapporter}{\info}
Henrik (Sigilbevarare) berättar att det är väldigt få som har hämtat sina medaljer.

Det dagliga arbetet har rullat på bra i CM, både dioder och halvledare börjar bli varma i kläderna. Några dioder har tillkommit och en del har försvunnit men CM har numera halvledare för alla dagar åtminstone. Inköparna jobbar på att hitta en lagom mängd att beställa och ska nästa vecka få lära sig göra IC-rapporter. Sist men inte minst så har 175 semlebullar beställts till fettisdagen om två veckor, woho!

Förvaltningsutskottet börjar röra på sig. Magnus har hållit lite grundläggande information om hur sektionens ekonomi fungerar och hur en utskottsordförande ska bokföra sina kostnader och intäkter. Magnus har hjälpt skattmästaren igång med sitt arbete och bokfört de första åtta verifikaten!

KM har planerat en tysklandsresa. De har också spånat lite på teman som ska spikas snart. KM tänkte bland annat boka electro banana band för något gille. De håller också på att planera kapsyltömmarnatten och Malin har bokat fikafika men inte fått något svar ännu.

Arbetet har börjat rulla igång rejält nu för NollU. I veckan var de på ledarskapsutbildning - det kändes givande och lärorikt. Många datum har börjat bestämmas inför våren. I helgen ska ÖPK iväg för planering, och samtidigt ska resten av FHÖBen gå på nationssittning - första gången de festar tillsammans. Tysklandsresa på fredag. NollU ska sy text på sina ouvvar nästa vecka. Phadderinfo måndagen den 12e februari. De har gjort om phaddersystemet en del inför i år. Skillnaderna är att man som kontaktphadder kommer att få ett större ansvar för att förmedla information till gruppen. Denna post kommer man att få skriva att man söker redan i ansökan till phadder. Nytt är också att vi inför extraphadder, där framförallt mästarna i sexet kommer att ses mer som extraphaddrar, eftersom de inte kan vara med lika mycket som andra phaddrar under nollningen.

ENU hade sitt första möte förra veckan och det bestämdes att det inte blir mer förrän efter Teknikfokus. De gick dock igenom lite om vad de vill göra under året.

Denna veckan har varit en produktiv vecka för NöjU. Biljardturneringen är planerad och drar igång den 9/2. Bowlingturneringen är snart full och NöjU beräknar att det kommer gå med minimal förlust för att det ska bli så billigt som möjligt för sektionens medlemmar. KarnEvalshypEn är planerad i mindre drag och de kommer göra en storskalig planering på fredag. De har bestämt special för nästa spelkväll vilket kommer att vara dartspecial. Mest för att Adam ska få lite eld i baken för att spika upp tavlan. Initialt datum satt till 24/3-25/3 för DrEamHackE och det planeras i full gång av Henrik och Jesper i samarbete med InfU, som kommer att hjälpa till med den tekniska biten. Vi har pratat om budget och sponsorer med mera. TandEm till Helsingborg, Paintball, Agent 00E, Glassförsäljning och DÖMD är event som kommer bli av LP2. E-spark ska bli av under läsveckorna och kommer bli en skön paus för alla i ungefär en timme. UtEdischot går framåt. Picasso har blivit informerad om temat och har börjat planera märken och affischer. Vi har tillsatt en till baransvarig, Mia Cicovic, som tilsammans med Henrik Ramström kommer att börja rekrytera jobbare i februari-mars och vara ansvarig för varsin bar. Adam kommer att stå som serveringsansvarig medan Saga och Hanna kommer att vara alkoholansvariga. Fortfarande lite oklart hur NöjU ska dela upp ansvarsrollerna. Pedellen har haft möte med sektionernas UtEDischoansvariga.

InfU jobbar på. Axel har planerat möte med vice kontaktor och macapären. Det första kodhackarmötet är planerat till torsdag.

Köket har kommit fram till ett menyförslag för teknikfokus och jobbar på ett förslag till Skiphtet. E6 har utvärderat Athena-fikan samt haft möte med Athena angående deras sittning den 22/2. Dessutom är alla jobbarscheman för vårens första sittningar klara och publicerade så att alla sexiga vet var och när de jobbar. Alla mästare i sexmästeriet har fått klara instruktioner kring vad som förväntas av dem både innan, under och efter sittningar.

SRE har haft sitt första möte och arbetet har kommit igång. CEQ-enkäter ska censureras av årskursansvariga och ledningen. Likabehandligsombuden ska ha en filmkväll 6/2 på samernas nationaldag och ska visa Sameblod. Vi ska göra posters för att uppmärksamma likabehandlingsombuden, världsmästarna och skyddsombudet. Fotograf är kontaktad och efter det ska Picasso göra posters åt oss. Planering för kick-off har också startat, vi tänker förmodligen gå på ett pub-quiz!

Daniel meddelar att lophtet är bokat till skiphte, även fikafika och glasverandan är bokade!

\subp{C}{Ekonomisk rapport}{\info}

Magnus hälsar att ekonomin ser bra ut.

\subp{D}{Kåren informerar}{\info}
Kåren informerar om att projektledare för arkad är vald. Andreas sitter kvar från förra året för en kortare överlämning men man ska vända sig till Christian som är projekledare detta året om man har funderingar.
\end{paragrafer}

\p{11}{Profilkläder}{\dis}
Mötet diskuterade profilkläder till utskotten.

\Mba vid nästa möte redogöra för hur intresset angående profilkläder ser ut hos de enskilda utskotten.

\p{12}{Biljettförsäljning}{\dis}

Adam undrade hur betalning vid försäljning av biljetter går till. Magnus lovade att visa och hjälpa till.

\p{13}{Skiphte}{\dis}
Mötet diskuterade Skiphte.

\p{14}{Serveringsansvariga}{\dis}
Mötet diskuterade frågan.

Alexander och Malin presenterade de personer de ansåg lämpade till uppgiften.

Magnus tog på sig att lämna dessa personers namn och personnummer till tillståndenheten.
\p{15}{Testamenten}{\dis}

Daniel vill att alla ska dela sina testamenten i driven.

\p{16}{Nivå på spex}{\dis}

Punkten protokollfördes ej.

\p{17}{Nästa styrelsemöte}{\bes}
\Mba nästa styrelsemöte ska äga rum nästa vecka 12:10 i E:1124.

\p{18}{Beslutsuppföljning}{\bes}

\Ibfu
%\Mbaby

\p{19}{Övrigt}{\dis}
Andreas undrade vilka som vill söka phaddrar och gav information om att styrelsen eventuellt kan vara extraphaddrar under detta året.

\p{20}{OFMA}{\bes}
{\mo} förklarade mötet avslutat 12:56.
\end{paragrafer}

%\newpage
\hidesignfoot
\begin{signatures}{3}
\signature{\mo}{Mötesordförande}
\signature{\ms}{Mötessekreterare}
\signature{\ji}{Justerare}
\end{signatures}
\end{document}
