\documentclass[10pt]{article}
\usepackage[utf8]{inputenc}
\usepackage[swedish]{babel}

\def\mo{Erik Månsson}
\def\ms{Axel Voss}
\def\ji{Johan Karlberg}
\def\jii{Rasmus Sobel}

\def\doctype{Protokoll} %ex. Kallelse, Handlingar, Protkoll
\def\mname{Höstterminsmötet} %ex. styrelsemöte, 
\def\mnum{HT/18} %ex S02/16, E1/15, VT/13
\def\date{2018-11-22} %YYYY-MM-DD
\def\docauthor{\ms}

\usepackage{../e-mote}
\usepackage{../../../e-sek}

\begin{document}
\showsignfoot

\heading{{\doctype} för {\mname} {\mnum}}

%\naun{}{} %närvarane under
%\nati{} %närvarande till och med
%\nafr{} %närvarande från och med
\section*{Närvarande}
\subsection*{Styrelsen}
\begin{narvarolista}
	\nv{Entertainer}{Adam Belfrage}{BME17}{}
	\nv{Sexmästare}{Alexander Wik}{BME17}{}
	\nv{Øverphøs}{Andreas Bennström}{BME16}{}
	\nv{Kontaktor}{Axel Voss}{E15}{}
	\nv{Ordförande}{Daniel Bakic}{E15}{}
	\nv{Cafémästare}{Elin Johansson}{BME16}{\nati{16G}}
	\nv{SRE-ordförande}{Fanny Månefjord}{BME16}{}
	\nv{ENU-ordförande}{Isabella Hansen}{E16}{}
	\nv{Förvaltningschef}{Magnus Lundh}{E15}{}
	\nv{Krögare}{Malin Heyden}{E16}{}
\end{narvarolista}

\subsection*{Ständigt adjungerande}
\begin{narvarolista}
    \nv{Talman}{Erik Månsson}{E14}{}
\end{narvarolista}

\subsection*{Medlemmar}
\begin{narvarolista}
	\nv{}{Adam Rosandell}{E17}{}
	\nv{}{Agnes Wallén}{E18}{}
	\nv{}{Amanda Gustafsson}{BME17}{\nati{16I}}
	\nv{}{Amanda Nilsson}{BME16}{\nati{16H}}
	\nv{}{Amanda Zarkout}{E17}{\nati{16I}}
	\nv{}{Anton Jigsved}{BME16}{\nati{17A}}
	\nv{}{Antonia Mundt-Petersen}{E18}{\nati{16I}}
	\nv{}{Carl Rutholm}{E17}{}
	\nv{}{Casper Schwerin}{BME18}{\nati{17A}}
	\nv{}{David Karlsson}{E17}{}
	\nv{}{David Karlsson}{E18}{\nati{16I}}
	\nv{}{David Uhler Brand}{E14}{\nati{16}}
	\nv{}{Davida Åström}{BME17}{}
	\nv{}{Edvard Carlsson}{E16}{}
	\nv{}{Elina Yrlid}{E18}{}
	\nv{}{Emil Eriksson}{E18}{}
	\nv{}{Emil P. Lundh}{E17}{\nafr{16}}
	\nv{}{Emma Hjörneby}{BME17}{}
	\nv{}{Ester Pörtfors}{BME18}{\nati{16I}}
	\nv{}{Evelina Morgan}{E18}{\nati{16I}}
	\nv{}{Fabian Sondh}{E17}{}
	\nv{}{Filip Larsson}{E17}{}
	\nv{}{Filip Winzell}{BME16}{}
	\nv{}{Freja Sahlin}{BME17}{}
	\nv{}{Frida Pilcher}{E18}{}
	\nv{}{Georgij Michaliutin}{E18}{}
	\nv{}{Hannes Björk}{E17}{}
	\nv{}{Henrik Ramström}{E16}{}
	\nv{}{Henrik von Friesendorff}{BME17}{\nafr{16I}}
	\nv{}{Hjalmar Tingberg}{BME16}{\nati{16H}}
	\nv{}{Ida Gustafsson}{E18}{\nati{16}}
	\nv{}{Jakob Pettersson}{E17}{}
	\nv{}{Jasmina Trinh}{E18}{\nati{16I}}
	\nv{}{Johan Halldin}{BME17}{}
	\nv{}{Johan Karlberg}{E14}{}
	\nv{}{Johan Siwerson}{E17}{}
	\nv{}{Johannes Larsson}{E16}{}
	\nv{}{Jonathan Benitez}{E17}{}
	\nv{}{Kajsa Ekenberg}{BME17}{\nati{17E}}
	\nv{}{Klara Indebetou}{BME17}{}
	\nv{}{Lina Samnegård}{BME16}{\nati{16H}}
	\nv{}{Linnea Söderström}{BME18}{\nati{16I}}
	\nv{}{Love Sjelvgren}{E18}{}
	\nv{}{Ludvig Lifting}{E17}{\nati{17A}}
	\end{narvarolista}
	\newpage
	\begin{narvarolista}
	\nv{}{Malin Rudin}{E18}{}
	\nv{}{Marcus Lindell}{BME18}{}
	\nv{}{Matilda Horn}{BME18}{}
	\nv{}{Mattias Lundström}{E17}{\nati{16H}}
	\nv{}{Max Mauritsson}{BME16}{\naun{16I}{16H}}
	\nv{}{Melina Alnasser}{BME17}{\nati{17A}}
	\nv{}{Moa Rönnlund}{E17}{}
	\nv{}{Oscar Nilsson}{E18}{\nati{16I}}
	\nv{}{Philip Johansson}{E16}{}
	\nv{}{Pontus Landgren}{E14}{\nafr{16K}}
	\nv{}{Rasmus Sobel}{BME16}{}
	\nv{}{Saga Juniwik}{E16}{}
	\nv{}{Sofie Johannesson}{E17}{\nati{16H}}
	\nv{}{Sonja Kenari}{E15}{}
	\nv{}{Sophia Carlsson}{BME17}{}
	\nv{}{Sophia Grimmeiss Grahm}{BME14}{\nafr{16}}
	\nv{}{Stephanie Bol}{BME17}{}
	\nv{}{Tina Tabandeh}{BME17}{}
	\nv{}{Tom Andersson}{E17}{}
	\nv{}{Tor Hammarbäck}{E17}{\nafr{17A}}
	\nv{}{Tove Börjeson}{E17}{}
	\nv{}{Viktor Björkman}{E15}{\nafr{17A}}
	\nv{}{Vincent Palmer}{E18}{}
	\nv{}{William Goldberg}{E17}{}
\end{narvarolista}

\begin{comment}
	\subsection*{Adjungerande}
	\begin{narvarolista}
	\end{narvarolista}
\end{comment}
\newpage
\section*{Protokoll}

\begin{paragrafer}
\p{1}{TaFMÖ}{}

Talman {\mo} förklarade mötet öppnat 17:22.

\emph{Vid mötets början var 72 personer närvarande.}

\p{2}{Val av mötesordförande}{}
Talman {\mo} valdes.

\p{3}{Val av mötessekreterare}{}
Kontaktor {\ms} valdes.

\p{4}{Godkännande av tid och sätt}{}
Tid och sätt godkändes.

\p{5}{Val av två justeringspersoner}{}

Johan Karlberg och Rasmus Sobel valdes till justerare.
\p{6}{Adjungeringar}{}
\emph{Inga adjungeringar.}

\p{7}{Godkännande av dagordningen}{}

Erik Månsson \ypa behandla \S16J före \S16H och \S16I.

Erik Månsson \ypa behandla \S16H och \S16L efter \S16I - \S16M.

Erik Månsson \ypa behandla \S17A före \S16H och \S16L

\textbf{Mötet beslutade att godkänna föredragningslistan med samtliga yrkanden.}

\p{8}{Föregående sektionsmötesprotokoll}{}
\textbf{Mötet beslutade att lägga till protokollet för VT/18 till handlingarna.}

\p{9}{Meddelanden}{}

David Uhler Brand meddelade att man ska rösta till Fullmäktigevalet. Det är viktigt att man röstar både så man får tårta och så att man får vara med att bestämma motiv på teknologkortet.

Det är Julgille den 15e december, biljetterna släpps den 28e november klockan 18:18 hälsar Malin.

Magnus Lundh meddelade att f1röj är fett, det är ett fett evenemang med mycket fest. 

Rasmus Sobel informerade om att man röstar till fullmäktigevalet på fmval.tlth.se. Gå in och rösta.

Fanny Månefjord informerade om att är det en föreläsning om AI på tisdag. Anmälan kommer ut snart i esek events.

Axel Voss meddelade att möteshandlingar till valmötet nu ligger ute på hemsidan.

\newpage
\p{10}{Beslutsuppföljning}{}

Filip Larsson presenterade beslutsuppföljningen av \emph{``Veckoliga studiekvällar med tilltugg''}. 

Filip Larsson informerade om att projektet varit tidskrävande än vad han och William förväntat sig och att de inte fortsatte mer med projektet efter julen.

\textbf{\Mba stryka ``Veckoliga studiekvällar med tilltugg'' från beslutsuppföljningen}.

Magnus Lundh presenterade beslutsuppföljningen av \emph{``Uppdaterad utrustning för linbanan''}.

Magnus Lundh informerade om att de gick lite över budget men att linbanan blev otroligt bra. Magnus \ypa hela den extra kostnaden belastar utrustningsfonden.

\textbf{\Mba stryka ``Uppdaterad utrustning för linbanan'' från beslutsuppföljningen med Magnus tilläggsyrkande}.

Daniel Bakic presenterade beslutsuppföljningen av \emph{``Inköp av en ny sektionskamera''}.

Daniel informerade om att inköpet gick lite över budgeten men sektionen fick en mycket bättre kamera än tänkt, vilket förhoppningsvis kommer leda till finare och fler bilder. Daniel \ypa hela den extra konstnaden belastar utrustningsfonden.

\textbf{\Mba stryka ``Inköp av en ny sektionskamera'' från beslutsuppföljningen med Daniels tilläggsyrkande}.

Henrik Ramström presenterade beslutsuppföljningen av \emph{``Utskottstack i form av märken''}.

\textbf{\Mba stryka ``Utskottstack i form av märken'' från beslutsuppföljningen}.

Magnus Lundh presenterade beslutsuppföljningen av \emph{``Upprustning av bar i Edekvata''}.

Magnus informerade om att projektet gick lite över budget men att baren har blivit jättefint. Magnus \ypa den extra kostnaden går på utrustningsfonden.

Henrik Ramström undrade vad Sexmästeriet tycker om baren.

``Den är asbra'' enligt sexmästare Alexander Wik.

\textbf{\Mba stryka ``Upprustning av bar i Edekvata'' från beslutsuppföljningen med Magnus tilläggsyrkande}.

Magnus Lundh presenterade beslutsuppföljningen av \emph{``Inköp av ny tappanordning''}.

\textbf{\Mba stryka ``Inköp av ny tappanordning'' från beslutsuppföljningen}.

Daniel Bakic presenterade beslutsuppföljningen av \emph{``Inköp Umphbox''}.

Daniel informerade om att vi numera hörs mest på hela LTH igen.

\textbf{\Mba stryka ``Inköp Umphbox'' från beslutsuppföljningen}.

Daniel Bakic presenterade beslutsuppföljningen av \emph{``Mikrovågsugnar''}.

Vi har haft elektriker som har kollat på det men eftersom det är en byggfråga så är det Akademiska hus som bestämmer. Akademiska hus är långsamma på att svara så sektionen kan inte göra mer i nuläget.

Daniel \ypa skjuta upp beslutet till VT/19 med detta som underlag.

\textbf{\Mba skjuta upp ``Mikrovågsugnar'' till VT/19}.

Edvard Carlsson presenterade beslutsuppföljningen av \emph{``Internationell  NollEguide''}.

NollEguiderna var mycket omtyckta av de internationella nollorna och Edvard hoppas traditionen fortsätter även nästa år. 

\emph{Tyvärr får man inte posta guider till Chile.}

\textbf{\Mba stryka ``Internationell  NollEguide'' från beslutsuppföljningen}.

\p{11}{Utskottsrapporter}{}

\emph{Inga frågor ställdes till styrelsen.}

\p{12}{Uppföljning av verksamhetsplan}{}

\emph{Inga frågor ställdes till styrelsen.}

\p{13}{Ekonomisk rapport}{}

Magnus Lundh informerade om Sektionens ekonomi. 

Magnus meddelade att vi går plus varje år och att vi placerar överskottet i interna fonder. 

Just nu har vi ungefär en miljon kronor på kontot och Magnus är inte otrolig för att vi just nu går minus.

Rasmus Sobel poängterade att belopp och summor inte stämmer i handlingarna. 

Det är ingen fara enligt Magnus.

Henrik Ramström undrade var planen är för att inte gå så inte mycket plus. 

Genom att spendera mer pengar svarade Magnus.

\p{14}{Uttag ur Sektionens fonder sedan förra terminsmötet}{}

Erik informerade om vad de olika tabellerna betyder. 

\emph{Det är uttag som inte täcks av budgetar.}

Magnus Lundh rapporterade om uttagen ur Sektionens fonder sedan förra terminsmötet.

\p{15}{Resultatrapport från första halvan av verksamhetsåret}{}

Erik Månsson undrade hur mycket minus vi va förra verksamhetsåret. 

Ungefär -55 000 svarade Magnus Lund. Vi är -200 000 just nu.

\p{16}{Behandling av motioner}{}
    \begin{paragrafer}

		\subp{A}{Borttagandet av redaktionella organ från stadgan}{}
		Erik Månsson förklarade att en stadgaändring måste röstas igenom på två efterkommande sektionsmöten.

		Erik Månsson presenterade Motionen. 

		Magnus Lundh undrade varför det inte står någonstans att det är styrelsen som bestämmer över informationskanaler. 

		Henrik Ramström undrade om vi ska ta bort HeHe-redaktionen.

		Mötet svarade bestämt \textbf{Nej} på Henriks fråga.

		\textbf{Mötet beslutade att bifalla motionen.}

		\subp{B}{Vice-poster som styrelsens suppleanter}{}
		
		Erik Månsson presenterade motionen.

		Filip Larsson undrade vad som händer om skattmästare och vice förvaltningschef båda två vill vara suppleanter till förvaltningschefen. 

		Erik Månsson förklarade att det är utskottsordförandes beslut.

		Henrik Ramström undrade varför inte cophøs står med i motionen.
		
		Det valdes att inte behandlas av denna motionen enligt Erik.

		\textbf{Mötet beslutade att bifalla motionen.}

		\subp{C}{StraffBong}{}

		Jonathan Benitez är en godtydlig sektionemedlem. Han presenterade därför motionen.

		Carl Rutholm undrade om det är en offentlig bongning.

		Det kan sektionen rösta fram svarade Jonathan.
		
		William Goldberg undrade hur bongningen ska redovisas. 

		Henrik Ramström undrade om det är kontaktorn som ska bonga.

		Jonathan svarade att kontaktorn inte är på alla möten.

		Filip Larsson undrade om den alkoholdrycken ska vara en tuborg. 

		Det är rimligt enligt Erik Månsson.

		Adam undrade om bongningen ska ske på ølhäfvargillet. 

		Love Sjelvgren undrade om styrelsen ska bonga ikapp alla försenade protokoll.

		William Goldberg undrade om vi inte ska höja alkoholhalten för varje läsdag som protokoll är sent.
		
		Magnus Lundh \ypa dra streck i debatten.

		\textbf{\Mba bifalla Magnus yrkande.}

		Daniel Bakic presenterade styrelsens svar. 

		David Uhler Brand undrade vilken budget konstnaden ska läggas på. 

		Henrik Ramström tycker att styrelsen gör ett superbra jobb. 

		\textbf{Mötet beslutade att avslå motionen och styrelsens svar.}

		\newpage

		\subp{D}{Hotell Brödraskapet}{}

		\emph{På grund av skrivkramp förkortast hädanefter ``Brödraskapet BME'' med ``BB''}

		BB presenterade motionen. 

		Adam Belfrage undrade om BB har tänkt på killarna i BME17. 

		BB svarade att de bestämmer och kan ge förtur.

		Henrik Ramström undrade om BB har tänkt på de personer som är äldre, De vill kanske också vila.

		BB svarade att de bestämmer och kan ge förtur.

		Adam Belfrage undrade om det går att muta BB. 

		BB svarar att de är öppna för mutor. 

		David Uhler Brand undrade om BB har ett kostnadsunderlag för skylten.

		Nej svarar BB, de va för trötta för att kolla upp vad en skylt kostar. 

		Tom Andersson undrade om han kan boka nästa måndag ponerat att motionen går igenom.
		
		BB svarade att de bestämmer och kan ge förtur.
		
		Daniel Bakic presenterade styrelsens svar.

		Henrik undrade om styrelsen ska sköta bokning även när de gått av. 

		Axel Voss svarade att han redan har fixat en modul till hemsidan.

		Filip Larsson undrade om namnet verkligen är representativt.

		Jakob Pettersson undrade om det är Andreas Bennström som styr eftersom han går vinnande oavsett.

		\textbf{Mötet beslutade att avslå motionen och styrelsens svar.}

		\subp{E}{Sektionsgrodan}{}

		Daniel Bakic presenterade motionen. 

		Rasmus Sobel undrade om posten skulle väljas direkt nu på valmötet i år ponerat att motionen går igenom.

		Erik Månsson svarade att det beror på hur mötet röstar.

		Tom Andersson undrade vad det kan bli för krav på sektionen. Kan sektionsgrodan-ansvariga stänga ner exempelvis event. 

		William undrade vilka sektionen som är miljömärkta idag.

		Daniel Bakic presenterade styrelsens svar på motionen. 

		\textbf{Mötet beslutade att avslå motionen.}

		\newpage
		
		\subp{F}{Uppgradera utrustning i Edekvatas kök}{}

		Filip Larsson presenterade motionen. 

		David Uhler Brand undrade om motionärerna har haft kontakt med vaktmästaren i huset.

		Nej svarade Filip. 

		David undrade varför de vill ha tunna skärbrädor istället för tjocka.

		Det är på grund av plats. Det behövs fler skärbrädor än vad som finns nu svarade Filip.

		Rasmus Sobel undrar hur många skärbrädor det finns just nu.

		Ungefär 8 stycken enligt Filip.

		William Goldberg undrade hur man kom fram till budgeten för utredningen.

		Filip har avrundat uppåt, men kollat vad hantverkare kostar i timmen
		
		Daniel Bakic presenterade styrelsens svar på motionen.

		Johan Karlberg undrade vad styrelsen menar med ``mer samtidigt''.

		Daniel svarade exempelvis ny ugn eller stekbord.

		David upplyser om att det är huset som sköter elen. Styrelsen borde därför inte behöva lägga pengar på en utredning. 

		Filip jämkade sig med styrelsens förslag.

		Henrik Ramström informerade om att detta är en perfekt grej att söka projektfunktionär till.

		\textbf{\Mba bifalla yrkandet med styrelsens svar.}

		\subp{G}{Kuddar i Diplomat}{}

		\emph{Ingen lyfte motionen.}

		\newpage

		\subp{I}{Uppdatering av postbeskrivning för Övergudphadder}{}

		Edvard Carlsson presenterade motionen.

		Henrik Ramström undrade om det verkligen är en förtydling av posten.

		Edvard tycker beskrivningen är tydligt. 

		Adam Belfrage undrade om ØGP inte tagit på sig mer ansvar än vad de borde de senaste åren. 

		Daniel Bakic presenterade styrelsens svar på motionen.

		Hannes Björk undrade om man inte ska ändra postbeskrivningar för alla poster som hjälper till i så fall.

		Sonja Kenari undrade om Andreas var delaktig när styrelsen diskuterade motionen.

		Andreas Bennström svarade att han inte varit delaktig i diskussionen.
		 
		Axel Voss tycker att man kanske ska införa ett extra cophøs om NollU behöver så mycket avlastning. 

		Sophia Carlsson tycker att det är bra att øgp gör så mycket som de gör eftersom phøset inte kan gå runt i vanliga kläder under nollningen.

		Magnus Lundh tycker inte att det är så snällt ge otydliga arbetsuppgifter. Man kliver ändå på en post och man har då rätt att veta vad man ger sig in på.

		Johan Karlberg tycker inte att detta är någon direkt förändring utan en reflektion över hur det ser ut idag.

		Rasmus Sobel ansera att phøset kan inte göra allt under nollningen, och att ta in fler cophøs löser inte det problemet. 

		William Goldberg tycker att phadderverksamheten omfattar mycket av det øgp gör idag. 

		Filip Larsson tycker att phøset har tolkningsföreträde.

		Axel tycker verkligen inte någon har tolkningsföreträde över mötet.

		Adam Belfrage tycker att ``bistå'' är en dålig formulering. Den beskrivningen som står nu, skyddar øgp från för mycket arbetsbelastning.

		Sonja tycker att bistå är en bra beskrivning för man kan komma fram till arbetsuppgifter internt tillsammans med phøset. Øgp får fortfarande säga vad de tycker.

		Emil Eriksson tycker att man ska införa en ny post om man ska utöka arbetsbelastningen.

		Tom Andersson tycker det är krångligt att vi ska ha flera roller. Det är bättre med en dynamisk roll som kan anpassa sig. 

		Adam Belfrage tycker det är konstigt att vi jämför en post som bara väljs av cophøs med en som väljs av hela mötet.

		Jakob Pettersson tycker att det bra att uppdatera postbeskrivningen så det speglar de faktiska arbetet som øgp gör. 

		\textbf{Mötet beslutade att bifalla motionen med enkel majoritet.}

		\emph{Motionen måste därför lyftas på nästa Terminsmöte.}

		\newpage

		\textbf{Erik Månnson \ypa ajournera mötet i 40 minuter.}
		
		\textbf{Mötet beslutade att bifalla yrkandet.}

		\emph{Mötet ajournerades 19:10 och återupptogs 20:00.}

		\subp{J}{Ta bort NollU:s representant i valberedningen}{}
			Pontus Landgren presenterade motionen.

			Henrik Ramström håller inte med eftersom det finns risk att ett helt kompisgäng blir valt. Han tycker också att man inte bör minska antalet sökande till valberedningen. 

			Adam Belfrage anser att det är valmötet och därigenom sektionen som väljer de som sitter i valberedningen, ett kompisgäng kan därför undgås.

			Elin Johansson tycker inte det är speciellt demokratiskt att ett utskott ska ha en garanterad plats i valberedningen. 

			Daniel Bakic yrkande på att i stadgan under \S7.2 göra följande ändring utöver Pontus förslag:

			\emph{Valberedningen består av:
				\begin{alphlist}
				\item Ordförande,
				\item en sekreterare,
				\item två (2) ledamöter,
				\item en (1) representant ur Nolleutskottet,  samt
				\item en (1) av Valberedningen utsedd representant bland de nyinskrivna.
			\end{alphlist}}
			till:

			\emph{Valberedningen består av:
			\begin{alphlist}
				\item Ordförande,
				\item en sekreterare,
				\item \hl{tre (3) ledamöter,}
				\item en (1) representant ur Nolleutskottet,  samt
				\item en (1) av Valberedningen utsedd representant bland de nyinskrivna.
			\end{alphlist}}
			\changenote

			Pontus jämkade sig med Daniels yrkandet.

			\textbf{Mötet beslutade att bifall det framvaskade förslaget.}

		\newpage
		\subp{K}{Suppleant för Øverphøs}{}

		Jonathan Benitez presenterade motionen.

		William Goldberg undrade om suppleanten ska vara ett cophøs. 
		
		Ja svarade Jonathan.
		
		Henrik Ramström undrade om suppleanten ska ta över Øverphøsets arbete med nollningen om denne inte kan fullfölja sitt uppdrag. 
		
		Jonathan har gjort lite research och kan inte hitta någonting i stadga eller reglemente som kan styrka detta.

		Daniel Bakic presenterade styrelsens svar. 

		Adam Belfrage \ypa återremitera motionen eftersom den är otydligt formulerad samt att styrelsen har tolkat sitt svar på motionen fel.

		Jonathan jämkade sig med yrkandet.

		\textbf{Mötet beslutade att avslå motionen.}

		\subp{M}{Wiki}{}

		\emph{Ingen lyfte motionen.}

		%\newpage

		\subp{L}{Ändringar för val av Øverphøs och Co-phøs}{}

		Jonathan Benitez presenterade motionen. 

		Adam Belfrage ifrågasatte tidsplaneringen eftersom man utgått efter hur HT och VM legat detta året. 

		Daniel Bakic presenterade styrelsens förslag på motionen. 

		\emph{Denna fråga diskuterades tillsammans med punkt \S16H}

		\subp{H}{Ändring av hur Sektionen väljer NollU-funktionärer}{}
		Sonja Kenari presenterade motionen.

		Sonja \ypa beslutet börjar gälla nästa år. 
		
		Daniel undrade om motionärerna vill att cophøsen väljs och valbereds individuellt, det finns inte någonting som säger det specifikt i motionen.

		Det vill motionärerna.

		Daniel Bakic presenterade styrelsens svar på motionen. 

		William Goldberg \ypa ajournera mötet i 5 minuter.
		
		\textbf{Mötet beslutade att bifalla yrkandet.}

		\emph{Mötet ajournerades 22:25 och återupptogs 22:35.}

		\textbf{Mötet beslutade att bifalla Edvard och Sonjas motion med enkel majoritet och därigenom avslå Jonathans motion \S16L. }
	
		\emph{Motionen måste därför lyftas på nästa Terminsmöte.}		

		\newpage
	\end{paragrafer}
\subp{17}{Behandling av propositioner}{}
    \begin{paragrafer}

		\subp{A}{Budgetförslag för 2019}{}
		
		Daniel Bakic presenterade Propositionen.

		\textbf{Mötet beslutade att bifalla propositionen i sin helhet.}

		\subp{B}{Verksamhetsplansförslag för 2019}{}
		
		Daniel Bakic presenterade Propositionen.
		
		\textbf{Mötet beslutade att bifalla propositionen i sin helhet.}

		Erik Månsson \ypa ajournera mötet i 5 minuter.

		\textbf{Mötet beslutade att bifalla yrkandet.}

		\emph{Mötet ajournerades 20:50 och återupptogs 20:55.}

		\subp{C}{Flytta HTF-ansvariga till SRE}{}
		
		Fanny Månefjord presenterade propositionen.

		\textbf{Mötet beslutade att bifalla propositionen i sin helhet.}

		\subp{D}{Uppdatering av policy för nycklar och access}{}
		Elin Johansson presenterade propositionen.

		\textbf{Mötet beslutade att bifalla propositionen i sin helhet.}


		\subp{E}{Uppdatering av policy Principer för deltagande i Sektionsaktiviteter}{}
		Daniel Bakic presenterade proposition. 

		\textbf{Mötet beslutade att bifalla propositionen i sin helhet.}

		\newpage
		\subp{F}{Uppdatering av policy för Inbjudningar och anmodningar}{}
		
		Daniel Bakic presenterade propositionen.
		
		Sophia Grimmeiss Grahm \ypa lägga till  
		\begin{tightdashlist}
			\item ``Representant från TLTH:s styrelse'', samt 
			\item ``Nollegeneralen''
		\end{tightdashlist}

		under personer som ska anmodas till Nollegasquen i policyn för Inbjudningar och anmodningar.

		\textbf{\Mba bifalla yrkandet.}

		Malin Heyden \ypa lägga till 
		\begin{tightdashlist}
			\item ``De nyantagna studenterna vid E-sektionen''
		\end{tightdashlist}
		
		under personer som ska anmodas till Nollegasquen i policyn för Inbjudningar och anmodningar.

		\textbf{\Mba bifalla yrkandet.}

		Filip Larson \ypa ta bort 

		\begin{tightdashlist}
			\item ``Valberedningens ordförande''
		\end{tightdashlist}
		
		från personer som skall anmodas till "Sittingar", exklusive "Nollegasque".

		\textbf{\Mba avslå yrkandet.}

		\textbf{\Mba bifalla det framvaskade förslaget.}

		\subp{G}{Inköp av toppar till Vega}{}

		Malin Heyden presenterade propositionen.

		\textbf{Mötet beslutade att bifalla propositionen i sin helhet.}

    \end{paragrafer}

\p{18}{Övrigt}{}

\emph{Ingen hade någonting att ta upp på punkten.}
\p{19}{TaFMA}{}
Talman {\mo} förklarade mötet avslutat 23:48.

\emph{Vid mötets slut var 52 personer närvarande.}
 
\end{paragrafer}

%\newpage
\hidesignfoot
\begin{signatures}{4}
\signature{\mo}{Mötesordförande}
\signature{\ms}{Mötessekreterare}
\signature{\ji}{Justerare}
\signature{\jii}{Justerare}
\end{signatures}
\end{document}
