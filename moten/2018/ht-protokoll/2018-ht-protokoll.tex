\documentclass[10pt]{article}
\usepackage[utf8]{inputenc}
\usepackage[swedish]{babel}

\def\mo{Erik Månsson}
\def\ms{Axel Voss}
\def\ji{xxx}
\def\jii{xxx}

\def\doctype{Protokoll} %ex. Kallelse, Handlingar, Protkoll
\def\mname{Höstterminsmötet} %ex. styrelsemöte, 
\def\mnum{HT/18} %ex S02/16, E1/15, VT/13
\def\date{2018-11-22} %YYYY-MM-DD
\def\docauthor{\ms}

\usepackage{../e-mote}
\usepackage{../../../e-sek}

\begin{document}
\showsignfoot

\heading{{\doctype} för {\mname} {\mnum}}

%\naun{}{} %närvarane under
%\nati{} %närvarande till och med
%\nafr{} %närvarande från och med
\section*{Närvarande}
\subsection*{Styrelsen}
\begin{narvarolista}
\end{narvarolista}

\subsection*{Ständigt adjungerande}
\begin{narvarolista}
    \nv{Talman}{Erik Månsson}{E14}{}
\end{narvarolista}

\subsection*{Medlemmar}
\begin{narvarolista}
\end{narvarolista}

\subsection*{Adjungerande}
\begin{narvarolista}
\end{narvarolista}
\newpage
\section*{Protokoll}

\begin{paragrafer}
\p{1}{TaFMÖ}{}

Talman {\mo} förklarade mötet öppnat 17:22.

\emph{Vid mötets början var XX personer närvarande.}

\p{2}{Val av mötesordförande}{}
Talman {\mo} valdes.

\p{3}{Val av mötessekreterare}{}
Kontaktor {\ms} valdes.

\p{4}{Godkännande av tid och sätt}{}
Tid och sätt godkändes.

\p{5}{Val av två justeringspersoner}{}


\textbf{Johan Karlberg och Rasmus Sobel}
\p{6}{Adjungeringar}{}
inga adjungeringar.

\p{7}{Godkännande av dagordningen}{}

Erik Månsson \ypa ``Flytta motion J före 16 HL''

\Mba
%Viktor Persson \ypa lägga till \S17.5 ``Behandling av sen motion: Utökning av antal sökande till posten Co-phøsare''.

\textbf{Mötet beslutade att godkänna föredragningslistan.}

\p{8}{Föregående sektionsmötesprotokoll}{}
\textbf{Mötet beslutade att lägga till protokollet för VT/18 till handlingarna.}

\p{9}{Meddelanden}{}

Brand Meddelade att man ska rösta till Fullmäktigevalet. Det är viktigt att man röstar så man får tårta och vara med att bestämma.

Det är Julgille , biljetterna släpps den xx klockan 18:18.

Magnus meddelar fett röj, det är ett fett evenemang. Med mycket fest. 

Rasmus informerade om att man röstar på fmval.tlth.se. Gå in och rösta.

På tisdag är det en föreläsning om AI. Anmälan kommer ut snart på esek.

Axel meddelade att möteshandlingar till VT ligger ute på hemsidan.

\p{10}{Beslutsuppföljning}{}

FIlip Larsson informerade om de tog mycket tid och att de inte fortsatte mer med projektet efter julen.

\Mba

Magnus Lundh informerade om att de gick lite över budget men att linbanan blev otrligt bra. Magnus \ypa att hela kostnaden tas utrustningsfonden.

\Mba alla magnus yrkanden.

Daniel representerade Eltayebs motion. Daniel sa att vi gick lite över budgeten med Daniel \ypa hela kostnaden går på utrustningsfonden.

\Mba

Henrik presenterade alla märken och yrka på att styka utskottstack från Beslutsuppföljning. 

\Mba

Magnus presenterade Upprustning av Bar i edekvata. Det gick lite över budget med baren har blivit jättefint. Ypa kostnaden går på utrustningsfonden
Henrik undrade vad Sexmästeriet tycker om baren.
Den är asbra enligt Alex.

\Mba

Magnus presenterade beslut... Tappanordning.

\Mba

Daniel Inköp umphbox. Vi hörs mest på hela LTH. Vi gick under budget. 

\Mba 

Daniel informerade om att vi skulle köpa in fler microvågsugnar. Vi har haft elektriker som har kollat på det men eftersom det är en LU-byggfråga så är det Akademiska hus som bestämmer. De är långsamma på att svara.
Yrka på att skjuta upp till VT/19.

\Mba

Edvard presenterade Internationell nolleguide. 

Man får inte posta till Chile tydligen

\Mba

\p{11}{Utskottsrapporter}{}

Erik säger att alla har läst dem

\Mba 

\p{12}{Uppföljning av verksamhetsplan}{}

\Mba

\p{13}{Ekonomisk rapport}{}

Magnus informerade om Sektionens ekonomi. Vi går plus varje år som vi placerar i interna fonder. VI har ungefär en miljon på kontot. 
Magnus är inte orolig över att vi går minus.

Rasmus Sobel poängterade att belopp och summor inte stämmer. Det är ingen fara. 

Henrik UNdrade var planen är för att inte gå så inte mycket plus. 

Spendera mer pengar!!

Axel har gömt biobiljetter i handlingarna.



\p{14}{Uttag ur Sektionens fonder sedan förra terminsmötet}{}

	Erik informerade om vad de olika tabellerna betyder. Det är uttag som inte täcks av budgetar.

\p{15}{Resultatrapport från första halvan av verksamhetsåret}{}

Erik undrade hur mycket minus vi va förra verksamhetsåret. Ca -55 tusen sa Magnus. -200 tusen just nu.

\p{16}{Behandling av motioner}{}
    \begin{paragrafer}

		\subp{A}{Borttagandet av redaktionella organ från stadgan}{}
		Erik förklarade att en stadgaändring måste röstas igenom på två efterkommande sekitonemöten.

		Erik presenterade Motionen. 

		Magnus undrade varför det inte står någonstans att det är styrelsen som bestämmer över informationskanaler. 

		Henrik undrade om vi ska ta bort HeHe-redaktionen.

		Svar NEJ

		Det finns biobiljetter i stadgan.

		Mötet beslutade att bifalla motionen.

		\subp{B}{Vice-poster som styrelsens suppleanter}{}
		
		Erik Månsson presenterade motionen.

		Filip Larsson undrade om skattmästare och vice båda vill vara Suppleant. 

		Erik Månsson förklarade att det är utskottsordförandets beslut.

		Henrik undrade varför inte cophös står med i motionen. 

		\subp{C}{StraffBong}{}

		Jonathan benitez är en godtydlig sektionemedlem. 

		Han presenterade motionen.

		Hannes undrade om det skulle.

		Karl undrade om det är en offentlig bongning.

		Det kan sektionen rösta fram
		
		William undrar hur bongen ska redovisas. 

		Henrik undrade om det är kontaktorn som ska bonga??

		Jonathan säger att kontaktorn inte är på alla möten.

		Filip undrar om alkoholdrycken ska vara tuborg. 

		Det är rimligt enligt Erik.

		Adam undrar om bongningen ska ske på ölhäfvar gillet. 

		Love ska styrelsen bonga ikapp alla försenade protokoll.

		William undrar om vi inte ska höja alkoholhalten för varje läsdag som protokoll är sent.
		
		Magnus yrka på streck i debatten.

		Daniel presenterade styrelsens svar. 

		Brand undrade vilken budget det ska dras från. 

		Henrik tycker att styrelsen gör ett superbra jobb. 

		Mötet beslutade att avslå motionen och styrelsens svar. 

		\subp{D}{Hotell Brödraskapet}{}

		Brödrarskapet BME presenterade motionen. 

		Adam undrar om de har tänkt på killarna i BME17. 

		BME säger att de bestämmer och kan ge förtur.

		Henrik undrar om de har tänkt på de personer som är äldre. De vill kanske också vila.

		BME säger att de bestämmer och kan ge förtur.

		Adam undrar om det går att muta. 

		BME är öppna för mutor. 

		Brand undrar om de har ett kostnadsunderlag för SKylten.

		Nej svarar BME, de va för trötta för att kolla upp. 

		Tom undrar om han kan boka nästa måndag. 

		Daniel presenterade styrelsens svar.

		Henrik undrade om styrelsen ska sköta bokning även när de gått ut. 

		Filip Larsson undrar om namnet verkligen är representativt.  

		Love undrade vad straffet är.

		Det är smisk som är straffet.

		Jacob undrar om det är Andy som styr eftersom han går vinnande oavsett.

		Erik gick igenom lite mötesformalia. 

		Mötet beslutade att avslå båda motionerna. 

		\subp{E}{Sektionsgrodan}{}

		Daniel presenterade motionen. 

		Rasmus undrade om posten skulle väljas direkt nu på valmötet iår.

		Erik - Det beror på.

		Tom undrade vad det kan bli för krav på sektionen. Kan sektionsgrodan-ansvariga stänga ner event. 

		William undrade vilka sektionen som är miljömärkta idag.

		Daniel presenterade styrelsens svar. 

		MÖtet beslutade att avslå motionen.

		\subp{F}{Uppgradera utrustning i Edekvatas kök}{}

		Filip Larsson presenterade motionen. 

		Brand undrar om de har haft kontakt med vaktmästaren i huset.

		Nej svara Filip. 

		Brand undrar varför de vill ha tunna skärbrädor istället för tjocka.

		Det är på grund av plats. Det behövs fler skärbrädor än vad som finns nu.

		Rasmus undrar hur många skärbrädor det finns.

		Filip svara ungefär 8 stycken.

		William undrade hur man kom fram till budgeten för utredningen.

		Filip har avrundat uppåt. Men kollat vad hantverkare kostar i timmen
		
		Daniel presenterade styrelsens svar på motionen. 

		Johan undrade vad vi menar med "mer samtidigt"

		Daniel svara typ ugn exempelvis.

		Brand upplyser om att det är huset som sköter elen. Vi borde därför inte behöva lägga pengar på en utredning. 

		Filip jämkar sig med styrelsens förslag.

		Henrik informerade om att detta är en perfekt grej att söka projektfunktionär till.

		Mötet biföll yrkandet med styrelsens svar. 

		\subp{G}{Kuddar i Diplomat}{}

		Ingen lyfte motionen.		

		\subp{I}{Uppdatering av postbeskrivning för Övergudphadder}{}


		Edvard presenterade motionen.

		Henrik undrar om det verkligen är en förtydling av posten.

		Edvard tycker det är tydligt. 

		Adam undrar om øgps inte tagit på sig mer än vad de borde göra. 

		Daniel presenterade styrelsens svar på motionen.

		Hannes undrar om man inte ska ändra postbeskrivningar för alla poster som hjälper till.

		Sonja undrade om Andy va med när vi diskuterade motionen. Nej det va han inte.

		Axel tycker att man kanske ska införa ett extra cophös om de behöver avlastning. 

		Sophia tycker att det är bra att øgp gör lite eftersom phøset inte kan gå runt i tex vanliga kläder.

		Magnus tycker inte att det är så schysst att inte ge tydliga arbetsuppfiger. man kliver ändå på en post och man måste då veta vad man ger sig in på.

		Johan karlberg tycker att detta inte är en direkt förändring utan en reflektion över hur det ser ut idag.

		Rasmus. Phøset kan inte göra allt under nollningen, att ta in fler phøs löser inte det problemet. 

		William tycker att phadderverksamheten omfattar mycket det øgp gör. 

		Filip tycker att FIlip har tolkningsföreträde. 

		Adam tycker att bistå är en dålig formulering. Den beskrivningen som står nu, skyddar øgp från för mycket arbetsbelastning.

		Sonja tycker att bistå är en bra beskrivning för man kan komma fram till arbetsuppgofter tillsammans med phöset. øgp får fortfarande säga vad de tycker.

		Henrik - FRÅGA VAD HENRIK tykcker

		Emil tycker att man ska införa en ny post om man ska utöka arbetsbelastning.

		Tom tycker det är krångligt att vi ska ha flera roller. Det är bättre med en dynamisk roll som kan anpassa sig. 

		Adam tycker det är konstigt att vi jämför en post som bara väljs av cophös med en som väljs av hela mötet.

		Jakob tycker att det bra att uppdatera postbeskrivningen så det speglar de faktiska arbetet som øgp gör. 

		Mötet beslutade att bifalla motionen med enkel majoritet. 

		mat 19.10

		mötet återupptogs 20.00

		\subp{H}{Ändring av hur sektionen väljer NollU-funktionärer}{}
		
		
		\subp{J}{Ta bort NollU:s representant i valberedningen}{}
			Pontus presenterade motionen.

			Henrik håller inte med eftersom det finns risk att ett helt kompisgäng blir valt. Han tycker också att man inte bör minska antalet sökande till valberedningen. 

			Adam anser att det är valmötet som väljer de som sitter i valberedningen, ett kompisgäng kan därför undgå.

			Elin tycker inte det är speciellt demokratiskt att ett utskott ska ha en garanterad plats i valberedningen. 

			Daniel yrkar på att i stadgan kapitel 7, 7.2 ersätta "c) två (2) ledamöter" till "c) tre (3) ledamöter"
		
			Pontus jämkade sig med yrkandet.

			Mötet beslutade att bifall det framvaskade förslaget. 

		\subp{K}{Suppleant för Øverphøs}{}

		Benitez presenterade motionen.

		William undrar om suppleanten ska vara ett cophøs. Ja svarar Benitez.
		
		Henrik undrar om suppleanten ska ta över Øverphøsets arbete med nollningen om denne inte kan fullfölja sitt jobb. Benitez har gjort lite research och kan inte hitta någontin i stadga/reglemente angående detta.

		Daniel presenterade styrelsens svar. 

		Adam \ypa återremitera

		\subp{L}{Ändringar för val av Øverphøs och Co-phøs}{}
		\subp{M}{Wiki}{}

		Motionen togs inte upp.

		\subp{N}{Borttagandet av redaktionella organ från stadgan}{}
		
		\subp{O}{Ändring av val för Øverphøs och Co-phøs}{}
		
		Benitez presenterade motionen. 

		Adam ifrågasatte tidsplaneringen eftersom man utgått efter hur mötena legat iår. 

		Daniel presenterade styrelsens förslag på motionen. 

		\subp{P}{Ändring av hur Sektionen väljer NollU-funktionärer}{}
		Sonja presenterade motionen.

		Sonja \ypa beslutet börjar gälla nästa år. 
		
		Daniel undrar om motionärerna vill att cophøsen väljs och valbereds individuellt. För de har inte någonting som säger det specifikt i motionen.

		Det tycker Motionärerna.

		William undrar när phøset valdes in, KOLLA I protokoll. 

		Daniel presenterade styrelsens svar på motionen. 

		William \ypa 5 minuters paus. 

		Mötet pausat 5 min 22.25. 

		Mötet återupptogs 22.35.

		Tom tycker att Øverphøset ska vara med och valbereda. 

		Johan tycker att rollbeskrivning till phøset skulle kunna leda till att folk inte söker i grupp. 

		Sophia Carlsson tycker att det är lätt att sektionen kan tappa engagerade medlemmar eftersom de inte har chans att söka till andra.

		Adam \ypa streck i debatten. 

		\Mba bifall yrkandet. 

		Edvard tycker inte att systemet fungerar. Han valde in sig själv som cophøs. 

		Mötet beslutade att bifalla yrkandet med enkel majoritet. 


	\end{paragrafer}
\subp{17}{Behandling av propositioner}{}
    \begin{paragrafer}

		\subp{A}{Budgetförslag för 2019}{}
		
		Daniel presenterade Propositionen.

		Henrik \ypa nollbudgetera posten medaljer ska strykas. FRÅGA om motivering.

		Magnus menar att medaljerna ska lagerföras och man skriver ut de medaljerna man säljer årsvis.

		Henrik drog tillbaka sitt yrkande. 

		Mötet biföll yrkandet.

		Isabella \ypa att ändra lägga till en saknad budgetpost. Allumniansvarig. 

		Mötet beslutade att bifalla yrkande.

		\subp{B}{Verksamhetsplansförslag för 2019}{}
		
		Daniel presenterade Propositionen.
		
		kissepaus 20:50 

		mötet återupptogs 20:55

		\subp{C}{Flytta HTF-ansvariga till SRE}{}
		
		Fanny presenterade propositionen.

		Mötet biföll propositionen.

		\subp{D}{Uppdatering av policy för nycklar och access}{}
			Elin presenterade

			\Mba bifalla yrkande.

		\subp{E}{Uppdatering av policy Principer för deltagande i Sektionsaktiviteter}{}
		Daniel presenterade proposition. 

		\Mba bifalla yrkandet.

		\subp{F}{Uppdatering av policy för Inbjudningar och anmodningar}{}
		
		Daniel presenterade motionen.

		
		Sophia Yrkar på att lägga till "Representant från TLTH:s styrelse" och "Nollegeneralen" under personer som ska anmodas till Nollegasquen i policyn om Inbjudningar och anmodningar
		
		\Mba bifalla yrkandet

		Malin \ypa lägga till "de nyantagna studenterna" under följanse personer skall anmodas under rubriken nollegasque

		\Mba bifalla yrkandet

		Filip Yrkar på att ta bort "Valberedningens ordförande" från personer som skall anmodas till "Sittingar", exklusive "Nollegasque"

		\Mba avslå yrkandet.

		Mötet biföll det framvaskade förslaget. 

		\subp{G}{Inköp av toppar till Vega}{}

		Malin presenterade propositionen.

		Mötet beslutade att bifalla propositionen.

    \end{paragrafer}

\p{18}{Övrigt}{}

\p{19}{TaFMA}{}
Talman {\mo} förklarade mötet avslutat 23:48.

\emph{Vid mötets slut var 52 personer närvarande.} <- Detta stämmer
 
\end{paragrafer}

%\newpage
\hidesignfoot
\begin{signatures}{4}
\signature{\mo}{Mötesordförande}
\signature{\ms}{Mötessekreterare}
\signature{\ji}{Justerare}
\signature{\jii}{Justerare}
\end{signatures}
\end{document}
