\documentclass[../_main/handlingar.tex]{subfiles}

\begin{document}
\motionssvar
\textit{Vi har haft flera långvariga diskussioner angående motionen men står dessvärre inte riktigt enade om vilket beslut som ska fattas. Vi kommer nedan presentera majoritetens åsikter i frågan i hopp om att det kan vara underlag för givande diskussioner på sektionsmötet. Men kommer \textbf{inte} lägga fram några yrkanden.}

Många av våra argument förespråkar att det bör vara upp till sektionsmötet att välja Cophøs, vilket även motionärerna har yrkat på. Men risken finns att det, precis som tidigare år, blir att folk söker i grupp och att någon därmed kan ställas ensam mot en fullständig grupp. Detta är i våra ögon inte rätt och skulle man låta sektionsmötet bestämma är det viktigt att man förespråkar att man inte bör söka i grupp. Det var detta problem man försökte lösa med propositionen från VT/17 och det motionärerna är ute efter nu är att stryka de ändringar som kom till då och återgå till hur det var tidigare.

Vi ser även andra problem med systemet som finns nu. Den absolut största nackdelen är att det bara är valberedning tillsammans med Øverphøs som får ta del att valet av Cophøs. Därmed har resten av sektionens medlemmar ingenting att säga till om. Vilket även går emot sektionens verksamhetsplan som säger att “Sektionens verksamhet präglas av demokrati och transparens”. Även om problemet med att man söker i grupp kan komma till att kvarstå så anser vi att det inte är lika illa som att inte låta sektionsmötet ta del av beslutet.

Inför höstterminsmötet har vi lyckats luska fram fyra olika utfall över hur motionen kan röstas igenom:
\begin{enumerate}
    \item Motionen bifalles i sin helhet med 2/3 majoritet och ändringarna i reglementet börjar gälla direkt till årets valmöte.

    Även om vi tycker att alla poster bör väljas på valmötet av sektionen, inte av valberedningen tillsammans med Øverphøset, kan det bli problematiskt då denna ändring kommer alltför tätt inpå valmötet för att kunna göra valprocessen rättvis. Dels för den som blir vald till Øverphøs, men framför allt för de som tänkt söka Cophøs. De flesta Øverphøskandidater har inte känt till denna motion förrän handlingarna inför HT kom ut och har därmed inte vetat att valprocessen kan komma att ändras. Vilket kan göra att de är inställda på något helt annat än hur det kan komma att bli.

    Till de som söker Cophøs anser vi inte att det är rättvist att rösta igenom denna motion nu då de helt plötsligt ska valberedas och sedan ska kandidera på valmötet. Om det är fler än fem kandidater kommer inte alla kunna bli nominerade av valberedningen. Detta hade i vanliga fall varit okej då valberedningens förslag ska anslås åtta läsdagar innan valet. Då hade alla kandidater haft tid på sig att planera sitt framförande och kanske lägga upp sin kandidatur på ett annat sätt. Som det kommer bli i detta fall kommer valberedningen lägga fram sina förslag endast ett par dagar innan om de ska hinna intervjua alla vilket inte är rättvist. 
    \item Motionen bifalles i sin helhet med majoritet och VT avgör om detta blir en ändring.

    Får i princip samma utfall som att återremittera motionen, bara att man inte kan garantera huruvida detta kommer ske eller inte då man inte vet hur folk kommer rösta. Det blir ingen direkt ändring, men man hinner se hur det gick med årets valprocess och kan ha det som underlag när motionen tas upp igen.
    \newpage
    \item Motionen återremitteras.

    Man hade kunnat välja att återremittera motionen till VT. Då skulle man kunna undvika denna turbulens som kommer skapas och man får även mer underlag från årets valprocess och har därmed inte bara ett val som grund till denna ändring. Vilket var en av anledningarna till att vår proposition från VT återremitterades. Dock är nackdelen med detta att man då måste använda sig av samma valprocess som förra året, som uppenbarligen fått kritik. Sektionsmötet får återigen ingenting att säga till om angående valet av Cophøs.
    \item Motionen avslås i sin helhet.

    Detta är egentligen inte bra, då sektionen sägs stå för demokrati och transparens. Det blir tydligt att detta inte är fallet när en grupp på ca 6 personer själva väljer ett helt utskott. Man kan argumentera för att då valberedning och Øverphøs är förtroendevalda av sektionen samt att de inom valberedningen röstar om sina val så blir även deras process demokratisk. Dock inte i samma utsträckning som om sektionsmötet får rösta om valet.
\end{enumerate}

Motionen behandlar även valet av ØGP. Då även detta är en post med större ansvar är det rimligt att valet av posten tas upp på ett sektionsmöte. I detta fall är valet av ØGP ännu mer odemokratiskt då det bara har varit phøset som beslutat om denna posten förut.

Värt att påpeka är att även fast det är antingen valberedning + Øverphøs eller Phøset som skött valprocessen för posterna som tagits upp, så är det styrelsen som faktiskt beslutar om personerna i fråga ska väljas in eller inte. Även det är problematiskt då styrelsen tidigare inte fått höra mer om själva valprocessen utan blint litat på att den sköts på ett bra sätt av de som varit ansvariga. Ska man fortsätta på det viset bör Styrelsen få mer insikt i valprocesserna för de olika posterna.

Sammanfattningsvis kan man säga att Cophøsen och ØGP väljs in på valmötet har följande fördelar:
\begin{dashlist}
    \item Alla funktionärer prioriteras lika, eftersom alla väljs samtidigt.
    \item Istället för en sluten valberedningsprocess och ett val som i stort sett tas av valberedningen och överphøs, eller phös i samband med val av ØGP, får alla medlemmar vara med och rösta, vilket ger mer transparens i valet.
    \item Man kan då söka både Øverphøs, Cophøs, och andra poster om man så önskar.
    \item Eftersom styrelsen väljs först, har Øverphøs en chans att ta del av Cophøsens presentationer samt tala för de kandidater den tycker är mest lämpade under valet av Cophøs och får därför viss påverkan på valet.
    \item Precis som alla andra ansvarsposter på sektionen blir det mötet som väljer in Cophøs.
\end{dashlist}

\begin{signatures}{1}
	\ist
	\signature{Daniel Bakic}{Ordförande}
\end{signatures}

\end{document}