\documentclass[../_main/handlingar.tex]{subfiles}

\begin{document}
\motionssvar
\textit{Svaret nedan hålls kort och koncist då vi behandlat frågan mer utförligt i en annan motion}

Vi håller med om att man eventuellt kan tappa många taggade personer då man delar upp valet och lägger valet av Co-phøs efter valmötet. Dock kommer det alltid bli så att man måste prioritera något före det andra oavsett hur man lägger upp valet. Nackdelen med hur systemet är just nu är att man låser folk som vill söka Co-phøs till att endast kunna söka den posten då valet sker efter valmötet. Vilket vi anser är fel.

Vad gäller argumenten som togs upp angående nomineringsperioden anser vi att de är bristande då mötena inte sker vid samma tidpunkt varje år. Dock är det upp till styrelsen att schemalägga mötena efter vad som passar för dem och resten av sektionen.

Däremot anser vi inte att man ska prioritera hela Phöset framför resten av sektionens funktionärer. En lösning hade kunnat vara att välja in hela styrelsen på höstterminsmötet och därefter ta valet av Co-phøs först på agendan inför valmötet. Det är dock inte det som motionären yrkar på.

Styrelsen yrkar därför på

\begin{attsatser}
    \att avslå motionen i sin helhet.
\end{attsatser}


\begin{signatures}{1}
	\ist
	\signature{Daniel Bakic}{Ordförande}
\end{signatures}

\end{document}