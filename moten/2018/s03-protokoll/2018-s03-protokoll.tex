\documentclass[10pt]{article}
\usepackage[utf8]{inputenc}
\usepackage[swedish]{babel}

\def\mo{Daniel Bakic}
\def\ms{Axel Voss}
\def\ji{Elin Johansson}
%\def\jii{}

\def\doctype{Protokoll} %ex. Kallelse, Handlingar, Protkoll
\def\mname{styrelsemöte} %ex. styrelsemöte, Vårterminsmöte
\def\mnum{S03/18} %ex S02/16, E1/15, VT/13
\def\date{2018-02-08} %YYYY-MM-DD
\def\docauthor{\ms}

\usepackage{../e-mote}
\usepackage{../../../e-sek}

\begin{document}
\showsignfoot

\heading{{\doctype} för {\mname} {\mnum}}

%\naun{}{} %närvarane under
%\nati{} %närvarande till och med
%\nafr{} %närvarande från och med
\section*{Närvarande}
\subsection*{Styrelsen}
\begin{narvarolista}
\nv{Ordförande}{Daniel Bakic}{E15}{}
\nv{Kontaktor}{Axel Voss}{E15}{}
\nv{Förvaltningschef}{Magnus Lundh}{E15}{\nafr{10b}}
\nv{Cafémästare}{Elin Johansson}{BME16}{}
\nv{Øverphøs}{Andreas Bennström}{BME16}{}
\nv{SRE-ordförande}{Fanny Månefjord}{BME16}{}
\nv{ENU-ordförande}{Isabella Hansen}{E16}{}
\nv{Sexmästare}{Alexander Wik}{BME17}{}
\nv{Krögare}{Malin Heyden}{E16}{}
\nv{Entertainer}{Adam Belfrage}{BME17}{}
\end{narvarolista}
\subsection*{Ständigt adjungerande}


\begin{narvarolista}
%\nv{Inköps- och lagerchef}{Sofie Johannesson}{E17}{}
%\nv{Inköps- och lagerchef}{Fabian Sondh}{E17}{}
%\nv{Inköps- och lagerchef}{Albin Pålsson}{E17}{}
%\nv{Kårordförande}{Linus Hammarlund}{}{}
%\nv{Kårrepresentant}{Jacob Karlsson}{}{\nafr{3}}
\nv{Kårrepresentant}{Tim Djärf}{}{}
%\nv{Kårrepresentant}{Agnes Sörliden}{}{}
%\nv{Valberedningens ordförande}{Elin Magnusson}{}{}
%\nv{Skattmästare}{Olle Oswald}{}{}
%\nv{Kårrepresentant}{Daniel Damberg}{}{}
%\nv{Kårrepresentant}{John Alvén}{}{}
%\nv{Nollegeneral}{Jakob Nilsson}{}{}
%\nv{Talman}{Erik Månsson}{E14}{}
%\nv{Elektras Ordförande}{Elisabeth Pongratz}{}{}
%\nv{Inspektor}{Monica Almqvist}{}{}
\nv{Sigillbevarare}{Henrik Ramström}{}{}
\end{narvarolista}

\begin{comment}
\subsection*{Adjungerande}
\begin{narvarolista}
%\nv{Post}{Namn}{Klass}{}
\end{narvarolista}
\end{comment}

\section*{Protokoll}
\begin{paragrafer}
\p{1}{OFMÖ}{\bes}
Ordförande {\mo} förklarade mötet öppnat 12:12.

\p{2}{Val av mötesordförande}{\bes}
{\valavmo}

\p{3}{Val av mötessekreterare}{\bes}
{\valavms}

\p{4}{Val av justeringsperson}{\bes}
{\valavj}

\p{5}{Godkännande av tid och sätt}{\bes}
{\tosg}

\p{6}{Adjungeringar}{\bes}
{\ingaadj}
%Förnamn Efternamn adjungerades


\p{7}{Godkännande av dagordningen}{\bes}
%Dagordningen godkändes
Daniel \ypa att lägga till \S14 ``Möte med Mats och PH''.

Malin \ypa att lägga till \S15 ``Sjunga för Elin''.

%Föredragningslistan godkändes med yrkandet.
Föredragningslistan godkändes med samtliga yrkanden.

\p{8}{Föregående mötesprotokoll}{\bes}
\latillprot{S02/18}
%\ingaprot

\p{9}{Fyllnadsval och entledigande av funktionärer}{\bes}
\begin{fyllnadsval} %"Inga fyllnadsval." fylls i automatiskt
%\fval{Namn}{Post}
\fval{Oscar Uggla}{Kodhackare}
\entl{Ivar Söderberg}{Diod}
\end{fyllnadsval}

\p{10}{Rapporter}{}
\begin{paragrafer}
\subp{A}{Hur mår alla?}{\info}
Punkten protokollfördes ej.

\subp{B}{Utskottsrapporter}{\info}

  Elin hälsar att caféet går framåt. Inköparna har fått utbildning i att göra IC-rapporter. De jobbar på att hitta balans i inköpen och testa sig fram för att se hur mycket sallader de kan göra per dag. De kommer även sälja semlor på fettisdagen!

  Malin har varit i tyskland. KM håller sitt första gillet imorgon och de är taggade. Nästa vecka ska de planera färdigt sin kick-off.

  Andreas och resten av NollU har sytt sina ouvvar. Deras planeringsarbetet går framåt och under måndagen kommer de hålla i phadderinformation. De kommer även försöka ha möten med alla utskott denna veckan.

  InfU ska hålla första kodhackarmötet ikväll. Johannes har installerat nytt spamfilter som förhoppningsvis fungerar bätter än det förra.

  ENU har har börjat kolla på datum till kickoffen. Isabella har blivit kontaktad av lite företag. All fokus har legat på Teknikfokus, efter detta kommer ENU köra igång på riktigt.

  Adam har fixat och donat i veckan. NöjU hade en spelkväll igår och biljardturneringen börjar på fredag. Adam hade möte med Aktukollegiet där de kom fram till att det endast finns en vecka där ett större intersektionellt event kan genomföras. Planeringen av UtEDischot går framåt.

  Alexander har bokat edekvata till de sittningar E6 kommer ha i februari. Alexander och Phøset har börjat planera temasläpp och hovet har inventerat pump.

  SRE fortsätter censurera CEQ:rapporter och arbetsrapporter börjar rulla in. I tisdags hade de en filmkväll som blev mycket lyckad. De har även haft möte med NollU för att diskutera pluggphaddrar och pluggkvällar.

\subp{C}{Ekonomisk rapport}{\info}
  Magnus hälsar att ekonomin mår bra.
\subp{D}{Kåren informerar}{\info}
  Kåren har börjat arbeta med GDPR. Kåren har även startat upp en verksamhetsgrupp samt en budgetgrupp.
\end{paragrafer}

\p{11}{Kameraövervakning}{\dis}
  Mötet diskuterade frågan.

  Mötet är för kameraövervakning men emot att hålla ytterdörrar in till E-Huset ständigt låsta under dagtid. Detta eftersom man anser att en person som verkligen vill komma in i huset kommer lyckas ändå.
\p{12}{Kundvagnar}{\dis}

  Mötet diskuterade frågan.

  Elin \ypa Henrik och resten av E6-2017 ska ta tillbaka kundvagnar och att vi behåller två stycken totalt.

  \Mbaby

  Beslutsuppföljning sattes till s05-2018
\p{13}{Profilkläder}{\bes}

  \Mba i veckan börja kolla upp hemsidor att beställa tröjor från. Samt att fortsätta eventuell diskussion om detta i Slacken.

\p{14}{Möte med Mats och PH}{\dis}

  Daniel vill att alla i styrelsen ska hänga med och träffa Mats och Ph. Datum till detta kommer även det att diskuteras i Slacken.

\p{15}{Sjunga för Elin}{\dis}

  Styrelsen sjöng för Elin <3.

\p{16}{Nästa styrelsemöte}{\bes}
\Mba nästa styrelsemöte ska äga rum 2018-02-15 klockan 12:10 i E:1124.

\p{18}{Beslutsuppföljning}{\bes}

\Ibfu
%\Mbaby

\p{19}{Övrigt}{\dis}

  Elin informerade om Livsmedelsutbilning. Hon vill att de i sexmästeriet och källarmästeriet ska kolla upp vilka som skulle kunna tänka sig gå utbildningen.

  Adam, pratade om biljardaccess. Axel lovade att hjälpa till med att ge access till de som är anmälda till turneringen.

\p{20}{OFMA}{\bes}
{\mo} förklarade mötet avslutat 12:58.
\end{paragrafer}

%\newpage
\hidesignfoot
\begin{signatures}{3}
\signature{\mo}{Mötesordförande}
\signature{\ms}{Mötessekreterare}
\signature{\ji}{Justerare}
\end{signatures}
\end{document}
