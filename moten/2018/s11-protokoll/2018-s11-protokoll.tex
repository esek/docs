\documentclass[10pt]{article}
\usepackage[utf8]{inputenc}
\usepackage[swedish]{babel}

\def\mo{Daniel Bakic}
\def\ms{Axel Voss}
\def\ji{Malin Heyden}
%\def\jii{}

\def\doctype{Protokoll} %ex. Kallelse, Handlingar, Protkoll
\def\mname{styrelsemöte} %ex. styrelsemöte, Vårterminsmöte
\def\mnum{S11/18} %ex S02/16, E1/15, VT/13
\def\date{2018-05-02} %YYYY-MM-DD
\def\docauthor{\ms}

\usepackage{../e-mote}
\usepackage{../../../e-sek}

\begin{document}
\showsignfoot

\heading{{\doctype} för {\mname} {\mnum}}

%\naun{}{} %närvarane under
%\nati{} %närvarande till och med
%\nafr{} %närvarande från och med
\section*{Närvarande}
\subsection*{Styrelsen}
\begin{narvarolista}
	\nv{Ordförande}{Daniel Bakic}{E15}{}
	\nv{Kontaktor}{Axel Voss}{E15}{}
	\nv{Förvaltningschef}{Magnus Lundh}{E15}{}
	\nv{Cafémästare}{Elin Johansson}{BME16}{}
	\nv{Øverphøs}{Andreas Bennström}{BME16}{}
	\nv{SRE-ordförande}{Fanny Månefjord}{BME16}{}
	\nv{ENU-ordförande}{Isabella Hansen}{E16}{}
	\nv{Sexmästare}{Alexander Wik}{BME17}{}
	\nv{Krögare}{Malin Heyden}{E16}{}
	\nv{Entertainer}{Adam Belfrage}{BME17}{}
\end{narvarolista}
\subsection*{Ständigt adjungerande}


\begin{narvarolista}
	%\nv{Inköps- och lagerchef}{Sofie Johannesson}{E17}{}
	%\nv{Inköps- och lagerchef}{Fabian Sondh}{E17}{}
	%\nv{Inköps- och lagerchef}{Albin Pålsson}{E17}{}
	%\nv{Kårordförande}{Linus Hammarlund}{}{}
	%\nv{Kårrepresentant}{Jacob Karlsson}{}{\nafr{3}}
	%\nv{Kårrepresentant}{Agnes Sörliden}{}{}
	%\nv{Valberedningens ordförande}{Pontus Landgren}{}{}
	%\nv{Skattmästare}{Olle Oswald}{}{}
	%\nv{Kårrepresentant}{Daniel Damberg}{}{}
	%\nv{Kårrepresentant}{John Alvén}{}{}
	\nv{Nollegeneral}{Jakob Nilsson}{}{}
	\nv{Skyddsombud}{Axel Sandqvist}{E17}{}
	\nv{Vice Kontaktor}{Saga Juniwik}{E16}{}
	\nv{Vice Cafémästare}{Max Mauritsson}{BME16}{}

	%\nv{Talman}{Erik Månsson}{E14}{}
	%\nv{Elektras Ordförande}{Elisabeth Pongratz}{}{}
	%\nv{Inspektor}{Monica Almqvist}{}{}
	%\nv{Sigillbevarare}{Henrik Ramström}{}{}
	%\nv{Vice Entertainer}{Emil Bergström}{}{}
\end{narvarolista}

\begin{comment}
\subsection*{Adjungerande}
\begin{narvarolista}
	%\nv{Post}{Namn}{Klass}{}
\end{narvarolista}
\end{comment}

\section*{Protokoll}
\begin{paragrafer}
	\p{1}{OFMÖ}{\bes}
	Ordförande {\mo} förklarade mötet öppnat 12:10.
	
	\p{2}{Val av mötesordförande}{\bes}
	{\valavmo}
	
	\p{3}{Val av mötessekreterare}{\bes}
	{\valavms}
	
	\p{4}{Val av justeringsperson}{\bes}
	{\valavj}
	
	\p{5}{Godkännande av tid och sätt}{\bes}
	{\tosg}
	
	\p{6}{Adjungeringar}{\bes}
	{\ingaadj}
	%Förnamn Efternamn adjungerades
	
	
	\p{7}{Godkännande av dagordningen}{\bes}
	Axel \ypa lägga till punkten \S15 Äska pengar till tv.
	
	Föredragningslistan godkändes med yrkandet.
	%Föredragningslistan godkändes med samtliga yrkanden.
	
	\p{8}{Föregående mötesprotokoll}{\bes}
	\latillprot{S08/18 och S09/18}
	%\ingaprot
	
	\p{9}{Fyllnadsval och entledigande av funktionärer}{\bes}
	\begin{fyllnadsval} %"Inga fyllnadsval." fylls i automatiskt
	\end{fyllnadsval}
	
	\p{10}{Rapporter}{}
	\begin{paragrafer}
		\subp{A}{Hur mår alla?}{\info}
		Punkten protokollfördes ej.
		
		\subp{B}{Utskottsrapporter}{\info}
		
		Axel har fixat med protokoll.

		Elin har börjat sälja gurkstavar med dip. De har tackfest men inte planerat något.

		Fanny har kollat schema och srx har delat ut pris till en lärare på maskin.

		Malin behövde inte jobba senaste gille utan hon lämnade detta till vice. Det va skönt med en ledig helg. Malin ska göra en utvärdering till att jobbare, om de behöver ändra något inför nollningen. 

		Phøset har filmat klart filmen,temasläpp på fredag.

		Nöju fick tyvärr ställa in paintballen. De ska försöka få ihop en spelkväll nästa vecka. 

		Alexander har planerat temasläpp och etslask samt sin egna sittning den andra juni. 

		Isabella har planerat in lunchföreläsningar, bland annat med tetrapak.

		Dörren till köket är lagad, lampor är bytta och sicrit ska städas hälsar magnus. 

		Daniel har vait med några av styrelsen i stockholm och hälsat på KTH på deras vårbal. Esektionen gav KTHarna mycket fint målade tavlor.

		\subp{C}{Ekonomisk rapport}{\info}
		Ekonomin ser väldigt bra ut enligt Magnus!
		
		\subp{D}{Kåren informerar}{\info}
		
		Det är fullmäktigemöte nr 5 nästa vecka. Nästa vecka på tisdag är det f1röj-pub de kommer även testa sin nya app för matbeställning. 
		
		\subp{E}{Omvärldsrapport}{\info}
		
		Denna vecka finns det ingenting nytt att rapportera.
		
		
	\end{paragrafer}
	
	\p{11}{Umphbox}{\dis}
		
		Axel och Magnus \ypa ``Inköp soundboks'' till beslutsuppföljningen S13/18 med Axel Voss och Magnus Lundh som ansvariga.  

		\Mbaby

	\p{12}{Utskottsmärken}{\dis}

		Henrik presenterade förslag på märken.

		Mötet diskuterade frågan.

		Punkten kommer tas upp på nästa möte. Utskotten fundera lite på hur de vill att deras märken ska se ut.

	\p{13}{Färg}{\dis}
	
		Daniel \ypa ``Rensa det blå skåpet'' läggs till beslutsuppföljningen S12/18 med Adam Belfrage som ansvarig.

	\p{14}{EKEA}{\dis}
  
		Adam informerar om att det är otroligt stökigt i EKEA. Han vill att styrelsen ska se över sina hyllor, vad som ska kastas och vad som ska behållas. 

		Fanny \ypa ``Alla ska se över sina hyllor'' till S12/18.

		\Mbaby

	\p{15}{Äska pengar TVskärm}{\dis}
	
		Axel informerade om...

	\p{12}{Nästa styrelsemöte}{\bes}
	\Mba nästa styrelsemöte ska äga rum nästa onsdag 2018-05-09 klockan 12:10 i E:1124.
	  
	
	\p{14}{Beslutsuppföljning}{\bes}
	
	Axel \ypa stryka ``XxX'' från beslutsuppföljningen. 

	\Mbaby 

	Malin \ypa skjuta upp 

	Axel \ypa skjuta upp
	
	%\Ibfu
	
	
	\p{15}{Övrigt}{\dis}
	
	Fanny lyfte att vi ska säga till våra funktionärer att det inte får slänga skräp i pantpåsarna. 

	Andy informerade om att arkivet kan användas som möteslokal och att phøset städar undan där ifall någon behöver ha möte.


	\p{16}{OFMA}{\bes}
	{\mo} förklarade mötet avslutat 13:02.
\end{paragrafer}

%\newpage
\hidesignfoot
\begin{signatures}{3}
	\signature{\mo}{Mötesordförande}
	\signature{\ms}{Mötessekreterare}
	\signature{\ji}{Justerare}
\end{signatures}
\end{document}
