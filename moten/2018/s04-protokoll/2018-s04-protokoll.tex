\documentclass[10pt]{article}
\usepackage[utf8]{inputenc}
\usepackage[swedish]{babel}

\def\mo{Daniel Bakic}
\def\ms{Axel Voss}
\def\ji{Malin Heyden}
%\def\jii{}

\def\doctype{Protokoll} %ex. Kallelse, Handlingar, Protkoll
\def\mname{styrelsemöte} %ex. styrelsemöte, Vårterminsmöte
\def\mnum{S04/18} %ex S02/16, E1/15, VT/13
\def\date{2018-02-15} %YYYY-MM-DD
\def\docauthor{\ms}

\usepackage{../e-mote}
\usepackage{../../../e-sek}

\begin{document}
\showsignfoot

\heading{{\doctype} för {\mname} {\mnum}}

%\naun{}{} %närvarane under
%\nati{} %närvarande till och med
%\nafr{} %närvarande från och med
\section*{Närvarande}
\subsection*{Styrelsen}
\begin{narvarolista}
\nv{Ordförande}{Daniel Bakic}{E15}{}
\nv{Kontaktor}{Axel Voss}{E15}{}
\nv{Förvaltningschef}{Magnus Lundh}{E15}{}
\nv{Cafémästare}{Elin Johansson}{BME16}{}
\nv{Øverphøs}{Andreas Bennström}{BME16}{}
%\nv{SRE-ordförande}{Fanny Månefjord}{BME16}{}
\nv{ENU-ordförande}{Isabella Hansen}{E16}{}
\nv{Sexmästare}{Alexander Wik}{BME17}{}
\nv{Krögare}{Malin Heyden}{E16}{}
\nv{Entertainer}{Adam Belfrage}{BME17}{}
\end{narvarolista}
\subsection*{Ständigt adjungerande}


\begin{narvarolista}
%\nv{Inköps- och lagerchef}{Sofie Johannesson}{E17}{}
%\nv{Inköps- och lagerchef}{Fabian Sondh}{E17}{}
%\nv{Inköps- och lagerchef}{Albin Pålsson}{E17}{}
%\nv{Kårordförande}{Linus Hammarlund}{}{}
%\nv{Kårrepresentant}{Jacob Karlsson}{}{\nafr{3}}
\nv{Kårrepresentant}{Agnes Sörliden}{}{}
%\nv{Valberedningens ordförande}{Elin Magnusson}{}{}
%\nv{Skattmästare}{Olle Oswald}{}{}
%\nv{Kårrepresentant}{Daniel Damberg}{}{}
%\nv{Kårrepresentant}{John Alvén}{}{}
\nv{Nollegeneral}{Jakob Nilsson}{}{}
%\nv{Talman}{Erik Månsson}{E14}{}
%\nv{Elektras Ordförande}{Elisabeth Pongratz}{}{}
%\nv{Inspektor}{Monica Almqvist}{}{}
\nv{Sigillbevarare}{Henrik Ramström}{}{}
\end{narvarolista}

\begin{comment}
\subsection*{Adjungerande}
\begin{narvarolista}
%\nv{Post}{Namn}{Klass}{}
\end{narvarolista}
\end{comment}

\section*{Protokoll}
\begin{paragrafer}
\p{1}{OFMÖ}{\bes}
Ordförande {\mo} förklarade mötet öppnat 12:12.

\p{2}{Val av mötesordförande}{\bes}
{\valavmo}

\p{3}{Val av mötessekreterare}{\bes}
{\valavms}

\p{4}{Val av justeringsperson}{\bes}
{\valavj}

\p{5}{Godkännande av tid och sätt}{\bes}
{\tosg}

\p{6}{Adjungeringar}{\bes}
{\ingaadj}
%Förnamn Efternamn adjungerades


\p{7}{Godkännande av dagordningen}{\bes}
%Dagordningen godkändes
Axel \ypa att lägga till \S9 ``Extra val Picasso''.

Föredragningslistan godkändes med yrkandet.
%Föredragningslistan godkändes med samtliga yrkanden.

\p{8}{Föregående mötesprotokoll}{\bes}
%\latillprot{S03/18}
\ingaprot

\p{9}{Extra val Picasso}{\dis}
Mötet diskuterade frågan.


\p{10}{Fyllnadsval och entledigande av funktionärer}{\bes}
\begin{fyllnadsval} %"Inga fyllnadsval." fylls i automatiskt
%\fval{Namn}{Post}
\end{fyllnadsval}

\p{11}{Rapporter}{}
\begin{paragrafer}
\subp{A}{Hur mår alla?}{\info}
Punkten protokollfördes ej.

\subp{B}{Utskottsrapporter}{\info}

Elin hälsar att LED hade sin bästa försäljningsdag för detta året. Det är lite brist på jobbare och de har därför tvingats öppna lite senare, de hoppas att kaffesugna personer kan få upp ögonen för att hjälp behövs.

KM har haft sitt första gille. Det var lyckat och de sålde för ca 10000kr. Stephanie har hämtat nya tröjor som kommer delas ut i veckan. Nästa lördag är det dags för vår kickoff.

I söndags hade Phøset utbildning/workshop för FHØBen. Det togs upp bland annat hur man handskas med förväntningar samt gruppdynamik och det var uppskattat. I måndags höll de i phadderinformation och öppnade ansökan.

Nöjesutskottet har sökt biljetter till DÖMD, de har även beställt tält till UtEDischot. De har börjat ta betalt för bowlingturneringen och nästa gång ska de göra anmälan bindande så man inte går minus på avhopp. Bastukväll och nästa spelkväll är bokad och övriga events har diskuterats. NöjU har också bestämt sig för att byta motto till Effektivitet Energi Engagemang.

Sexmästeriet hade teknikfokus i onsdags, det va kul. På fredag är det skiphte och på torsdag nästa vecka har Athena sittning.

Studierådet har denna veckan planerat in CEQ-möten och Fanny ska gå på SRX-möte. SRE ska ha möte nästa vecka för att diskutera digitalisering.

\subp{C}{Ekonomisk rapport}{\info}
  Ekonomin ser bra ut enligt Magnus.

\subp{D}{Kåren informerar}{\info}
Kåren informerar om att priserna för att hyra saker från kåren har gått upp. Detta beror på prisbasbeloppet.
\end{paragrafer}

\p{12}{Vårterminsmöte}{\dis}

Preliminärt datum satt för Vårterminsmöte tisdag den 24e april.

\p{13}{Replokal för Elektro banana band}{\dis}
Mötet diskuterade frågan.

De frågor som lyftes fram var framförallt hur stort intresset över bandet är samt om det kommer leva kvar.

\p{14}{Nästa styrelsemöte}{\bes}
\Mba nästa styrelsemöte ska äga rum 2018-02-22 klockan 12:10 i E:1124.

\p{15}{Beslutsuppföljning}{\bes}

\Ibfu
%\Mbaby

\p{16}{Övrigt}{\dis}
Adam skulle vilja att sektionen köper in en GoPro. Denna skulle kunna användas flitigt menar han.

Henrik har städat i Sicrit. Han vill att information ska nå ut till utskotten att man inte ska dumpa grejer där.

Daniel informerade om sektionskalendern, använd den säger han!
\p{17}{OFMA}{\bes}
{\mo} förklarade mötet avslutat 12:49.
\end{paragrafer}

%\newpage
\hidesignfoot
\begin{signatures}{3}
\signature{\mo}{Mötesordförande}
\signature{\ms}{Mötessekreterare}
\signature{\ji}{Justerare}
\end{signatures}
\end{document}
