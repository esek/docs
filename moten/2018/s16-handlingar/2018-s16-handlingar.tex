\documentclass[10pt]{article}
    \usepackage[utf8]{inputenc}
    \usepackage[swedish]{babel}
    
    \def\doctype{Handlingar} %ex. Kallelse, Handlingar, Protkoll
    \def\mname{styrelsemöte} %ex. styrelsemöte, Vårterminsmöte
    \def\mnum{S16/18} %ex S02/16, E1/15, VT/13
    \def\date{2018-09-07} %YYYY-MM-DD
    \def\docauthor{Daniel Bakic}
    
    \usepackage{../e-mote}
    \usepackage{../../../e-sek}
    
    \begin{document}
    
    \heading{{\doctype} till {\mname} {\mnum}}
    
    \section*{Information om punkterna}
    
    \begin{paragrafer}
    
    \p{11}{Uppföljning GDPR}{\dis}
     Den nya lagen om personuppgifter har sedan slutet av maj trätt i kraft. Axel har jobbat med detta under sommaren och jag vill att vi diskuterar ämnet så att alla är med på vad som gäller.
 
     \rnamnpost{Daniel Bakic}{Ordförande}

    \p{12}{Diodbrist}{\dis}
        Nu till LP1 ser det väldigt glest ut med jobbare i LED vilket är tråkigt och har lett till att det hållts stängt under LV1. Skulle vilja att vi diskuterar situationen för att komma fram till en ekonomisk och arbetsbelastningsmässig lösning.
 
        \rnamnpost{Elin Johansson}{Cafémästare}

    \p{13}{Per capsulam S14, S15}{\bes}
    Under sommaruppehållet har det hållts i två styrelsemöten per capsulam via mejlkontakt, dessa bör läggas till officiellt till möteshandlingarna. Under S14 tog jag personligen upp punkten ``Äskning av pengar till inköp av ny fasadskylt'' där mina yrkanden bifölls i sin helhet. Under S15 diskuterades ``Inköp av tvättmaskin till sektionen'' där vi kom fram till att det är opassande att ha tvättmaskin i köket pga utrymmesbrist samt att frågan bör diskuteras mer. Jag har valt att bordlägga punkten till ett senare möte. 
 
    \rnamnpost{Daniel Bakic}{Ordförande}

    \p{14}{Anmodningar E-Phøs}{\bes}
     Under tidigare år har E-Phøs gått gratis på sittningar under nollningen. Dock står det inget i sektionens policy om att det ska vara på så vis. Jag anser att frågan bör diskuteras och att eventuella ändringar av policys görs utefter vad mötet kommer fram till.

     \rnamnpost{Daniel Bakic}{Ordförande}
    \end{paragrafer}
    
    \begin{signatures}{1}
    \ist
    \signature{\docauthor}{Ordförande}
    \end{signatures}
    
    \end{document}
    