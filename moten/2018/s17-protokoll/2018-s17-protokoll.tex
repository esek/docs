\documentclass[10pt]{article}
\usepackage[utf8]{inputenc}
\usepackage[swedish]{babel}

\def\mo{Daniel Bakic}
\def\ms{Axel Voss}
\def\ji{Alexander Wik}
%\def\jii{}
    
\def\doctype{Protokoll} %ex. Kallelse, Handlingar, Protkoll
\def\mname{styrelsemöte} %ex. styrelsemöte, Vårterminsmöte
\def\mnum{S17/18} %ex S02/16, E1/15, VT/13
\def\date{2018-09-14} %YYYY-MM-DD
\def\docauthor{\ms}
    
\usepackage{../e-mote}
\usepackage{../../../e-sek}
    
\begin{document}
\showsignfoot
    
\heading{{\doctype} för {\mname} {\mnum}}
    
%\naun{}{} %närvarane under
%\nati{} %närvarande till och med
%\nafr{} %närvarande från och med
\section*{Närvarande}
\subsection*{Styrelsen}
\begin{narvarolista}
    \nv{Ordförande}{Daniel Bakic}{E15}{}
    \nv{Kontaktor}{Axel Voss}{E15}{}
    \nv{Förvaltningschef}{Magnus Lundh}{E15}{}
    \nv{Cafémästare}{Elin Johansson}{BME16}{}
    \nv{Øverphøs}{Andreas Bennström}{BME16}{}
    \nv{SRE-ordförande}{Fanny Månefjord}{BME16}{}
    \nv{ENU-ordförande}{Isabella Hansen}{E16}{}
    \nv{Sexmästare}{Alexander Wik}{BME17}{}
    \nv{Krögare}{Malin Heyden}{E16}{}
    \nv{Entertainer}{Adam Belfrage}{BME17}{}
\end{narvarolista}
\subsection*{Ständigt adjungerande}
    
    
\begin{narvarolista}
    %\nv{Inköps- och lagerchef}{Sofie Johannesson}{E17}{}
    %\nv{Inköps- och lagerchef}{Fabian Sondh}{E17}{}
    %\nv{Inköps- och lagerchef}{Albin Pålsson}{E17}{}
    %\nv{Kårordförande}{Linus Hammarlund}{}{}
    %\nv{Kårrepresentant}{Jacob Karlsson}{}{\nafr{3}}
    %\nv{Kårrepresentant}{Agnes Sörliden}{}{}
    %\nv{Valberedningens ordförande}{Pontus Landgren}{}{}
    %\nv{Skattmästare}{Olle Oswald}{}{}
    %\nv{Kårrepresentant}{Daniel Damberg}{}{}
    \nv{Kårrepresentant}{Philip Johansson}{}{}
    %\nv{Nollegeneral}{Jakob Nilsson}{}{}
    %\nv{Skyddsombud}{Axel Sandqvist}{E17}{}
    %\nv{Saga}{}{}{}
    %\nv{Max}{}{}{}
    %\nv{Talman}{Erik Månsson}{E14}{}
    %\nv{Elektras Ordförande}{Elisabeth Pongratz}{}{}
    %\nv{Inspektor}{Monica Almqvist}{}{}
    \nv{Sigillbevarare}{Henrik Ramström}{E16}{}
    %\nv{Vice Entertainer}{Emil Bergström}{}{}
\end{narvarolista}
    
\begin{comment}
\subsection*{Adjungerande}
\begin{narvarolista}
    %\nv{Post}{Namn}{Klass}{}
\end{narvarolista}
\end{comment}
    
\section*{Protokoll}
\begin{paragrafer}
    \p{1}{OFMÖ}{\bes}
    Ordförande {\mo} förklarade mötet öppnat 12:15.
    
    \p{2}{Val av mötesordförande}{\bes}
    {\valavmo}
    
    \p{3}{Val av mötessekreterare}{\bes}
    {\valavms}
    
    \p{4}{Val av justeringsperson}{\bes}
    {\valavj}
    
    \p{5}{Godkännande av tid och sätt}{\bes}
    {\tosg}
    
    \p{6}{Adjungeringar}{\bes}
    {\ingaadj}
    %Förnamn Efternamn adjungerades

    \p{7}{Godkännande av dagordningen}{\bes}
    
    Adam \ypa lägga till punkten \S14 ``FikaFika''.

    Andreas \ypa lägga till punkten \S15 ``Medaljer''.
    %Föredragningslistan godkändes med yrkandet.

    Föredragningslistan godkändes med samtliga yrkanden.

    \p{8}{Föregående mötesprotokoll}{\bes}
    \latillprot{11/18, 12/18 och 13/18}
    %\ingaprot

    \p{9}{Fyllnadsval och entledigande av funktionärer}{\bes}
    \begin{fyllnadsval} %"Inga fyllnadsval." fylls i automatiskt
    \fval{Fanny Månefjord}{Diod}
    \fval{Lina Samnegård}{Diod}
    \fval{Jessica Kågeman}{Diod}
    \fval{Martin Ollén}{Diod}
    \fval{Klara Eriksson}{Diod}
    \fval{Hannes Björk}{Diod}
    \fval{Alicia Lindmark}{Diod}
    \fval{Johan Halldin}{Diod}
    \fval{Henrik Von Friesendorff}{Diod}
    \fval{Elin Andersson}{Diod}
    \fval{Antonia Mundt-Petersen}{Diod}
    \end{fyllnadsval}

    \p{10}{Rapporter}{}
    \begin{paragrafer}
    \subp{A}{Hur mår alla?}{\info}
    Punkten protokollfördes ej.

    \subp{B}{Utskottsrapporter}{\info}
    Axel har planerat inför den mytomspunna infu-kickoffen.

    Elin har hållit i utbildning, LED öppnar på måndag, de kommer hålla öppet 8-13 tillsvidare.

    Sexet höll i sittning i onsdags. De har sittning på lördag också tillsammans med A och K.

    Nollningen går bra, nollorna är taggade. Andreas vill festa! VI VANN FLYING!!!

    SRE lagade mat till 120 personer i måndags. Nästa vecka får bara nollorna mat eftersom SRE inte klarar av att göra till alla!

    NöjU höll i FightEn det gick bra.

    Magnus kämpar på med Bokslutet. Rahne håller koll på verktygen i sicrit.

    ENU hade lunchföreläsning med tetrapak i tisdags. De verkade mycket nöjda!

    Malin och KM har planerat inför gille, både kvällens och nästa veckas NEon-Gille. 

    Daniel hade OK-möte. Kåren behöver hjälp till sin valberedning. Även K behöver hjälp till sin valberedning. Gunnar har mailat angående alkoholpolicy.

    \subp{C}{Ekonomisk rapport}{\info}

    Ekonomin ser bra ut enligt Magnus!

    \subp{D}{Kåren informerar}{\info}

    Kåren behöver hjälp till sin valberedning. På måndag är det fullmäktigemöte nr 6.

    \subp{E}{Omvärldsrapport}{\info}
            
    Axel informerade om Norgeresan och Sanctus Omega Broderskap.        
            
    \end{paragrafer}

    \p{11}{Höstterminsmöte}{\dis}

    Daniel lyfte frågan angående datum för höstterminsmöte.

    Preliminära datum är satta den 22 och den 29 november. 

    \p{12}{Musikhjälpen}{\dis}

    Daniel informerade om Musikhjälpen.

    \p{13}{Kravprofiler}{\dis}

    Daniel informerade om Kravprofiler. Han vill att alla ska se över kravprofilerna till sina respektive poster.

    Daniel \ypa ``Utkast kravprofiler'' läggs till beslutsuppföljning s19 med samtliga ledarmöter som ansvariga. 
    
    \Mbaby

    \p{14}{FikaFika}{\dis}

    Magnus tycker att det ofta är dåligt städat i FikaFika. Vi måste göra det tydligare att det är ett krav på städning och lägga upp riktlinjer på hemsidan.

    Adam \ypa ``Riktlinjer för städning av FikaFika'' läggs till beslutsuppföljning s18 med honom själv som ansvarig.

    \Mbaby

    \p{15}{Medaljer}{\dis}

    Henrik informerade om nomineringar för medaljer och undrade om styrelsen hade några invändningar på nomineringsförslagen.
    
    \Mdf

    \p{16}{Nästa styrelsemöte}{\bes}

    \Mba nästa styrelsemöte ska äga rum på onsdag 2018-09-21 klockan 12:10 i E:1124.

    \p{17}{Beslutsuppföljning}{\bes}

    Daniel \ypa stryka ``Uppdatering av policy för Inbjudningar och anmodningar'' från beslutsuppföljning och ta upp detta som en proposition till Ht/18 istället.

    \Mbaby 	
    %\Ibfu
    \p{18}{Övrigt}{\dis}

    Henrik Ramström presenterade nya utskottsmärken. Han vill att alla utskottschefer ska bestämma sig snarast så att märkena kan beställas.  

    Mötet diskuterade ett styrelsemärke. 

    Adam \ypaltbus{Design styrelsemärke}{s18/18}{samtliga ledarmöter}
    
    \Mbaby
    \p{19}{OFMA}{\bes}
    {\mo} förklarade mötet avslutat 13:11.
\end{paragrafer}
    
%\newpage
\hidesignfoot
\begin{signatures}{3}
    \signature{\mo}{Mötesordförande}
    \signature{\ms}{Mötessekreterare}
    \signature{\ji}{Justerare}
\end{signatures}
\end{document}
    