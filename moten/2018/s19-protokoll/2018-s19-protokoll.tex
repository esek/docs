\documentclass[10pt]{article}
\usepackage[utf8]{inputenc}
\usepackage[swedish]{babel}

\def\mo{Daniel Bakic}
\def\ms{Magnus Lundh}
\def\ji{Isabella Hansen}
%\def\jii{}

\def\doctype{Protokoll} %ex. Kallelse, Handlingar, Protkoll
\def\mname{styrelsemöte} %ex. styrelsemöte, Vårterminsmöte
\def\mnum{S19/18} %ex S02/16, E1/15, VT/13
\def\date{2018-09-27} %YYYY-MM-DD
\def\docauthor{\ms}

\usepackage{../e-mote}
\usepackage{../../../e-sek}

\begin{document}
\showsignfoot

\heading{{\doctype} för {\mname} {\mnum}}

%\naun{}{} %närvarane under
%\nati{} %närvarande till och med
%\nafr{} %närvarande från och med
\section*{Närvarande}
\subsection*{Styrelsen}
\begin{narvarolista}
	\nv{Ordförande}{Daniel Bakic}{E15}{}
	%\nv{Kontaktor}{Axel Voss}{E15}{}
	\nv{Förvaltningschef}{Magnus Lundh}{E15}{}
	\nv{Cafémästare}{Elin Johansson}{BME16}{}
	\nv{Øverphøs}{Andreas Bennström}{BME16}{}
	\nv{SRE-ordförande}{Fanny Månefjord}{BME16}{\nati{14}}
	\nv{ENU-ordförande}{Isabella Hansen}{E16}{}
	\nv{Sexmästare}{Alexander Wik}{BME17}{}
	\nv{Krögare}{Malin Heyden}{E16}{}
	\nv{Entertainer}{Adam Belfrage}{BME17}{}
\end{narvarolista}
\subsection*{Ständigt adjungerande}


\begin{narvarolista}
	%\nv{Inköps- och lagerchef}{Sofie Johannesson}{E17}{}
	%\nv{Inköps- och lagerchef}{Fabian Sondh}{E17}{}
	%\nv{Inköps- och lagerchef}{Albin Pålsson}{E17}{}
	%\nv{Kårordförande}{Linus Hammarlund}{}{}
	%\nv{Kårrepresentant}{Jacob Karlsson}{}{\nafr{3}}
	\nv{Kårrepresentant}{Hanna Järpedal}{}{}
	\nv{Kårrepresentant}{Philip Johansson}{}{}
	%\nv{Valberedningens ordförande}{Pontus Landgren}{}{}
	%\nv{Skattmästare}{Olle Oswald}{}{}
	%\nv{Kårrepresentant}{Daniel Damberg}{}{}
	%\nv{Kårrepresentant}{John Alvén}{}{}
	%\nv{Nollegeneral}{Jakob Nilsson}{}{}
	%\nv{Skyddsombud}{Axel Sandqvist}{E17}{}
	%\nv{Saga}{}{}{}
	%\nv{Max}{}{}{}
	%\nv{Talman}{Erik Månsson}{E14}{}
	%\nv{Elektras Ordförande}{Elisabeth Pongratz}{}{}
	%\nv{Inspektor}{Monica Almqvist}{}{}
	\nv{Sigillbevarare}{Henrik Ramström}{E16}{}
	%\nv{Vice Entertainer}{Emil Bergström}{}{}
\end{narvarolista}

\begin{comment}
\subsection*{Adjungerande}
\begin{narvarolista}
	%\nv{Teknokrat}{Oscar Uggla}{E15}{}
	%\nv{Post}{Namn}{Klass}{}
\end{narvarolista}
\end{comment}

\section*{Protokoll}
\begin{paragrafer}
	\p{1}{OFMÖ}{\bes}
	Ordförande {\mo} förklarade mötet öppnat 12:18.

	\p{2}{Val av mötesordförande}{\bes}
	{\valavmo}

	\p{3}{Val av mötessekreterare}{\bes}
	{\valavms}

	\p{4}{Val av justeringsperson}{\bes}
	{\valavj}

	\p{5}{Godkännande av tid och sätt}{\bes}
	{\tosg}

	\p{6}{Adjungeringar}{\bes}
	{\ingaadj}
	%Förnamn Efternamn adjungerades

	\p{7}{Godkännande av dagordningen}{\bes}
	Daniel \ypa lägga till sena handlingar till dagordningen.

	Henrik Ramström \ypa lägga till \S14 ``Rengöring av fikafika''.

	%Dagordningen godkändes.
	%Föredragningslistan godkändes med yrkandet.
	Föredragningslistan godkändes med samtliga yrkanden.

	\p{8}{Föregående mötesprotokoll}{\bes}
	%\latillprot{}
	\ingaprot

	\p{9}{Fyllnadsval och entledigande av funktionärer}{\bes}
	\begin{fyllnadsval} %"Inga fyllnadsval." fylls i automatiskt
		%\fval{namn}{post}
		\entl{Lisa Linárd Pedersen}{SRE-ledamot}
	\end{fyllnadsval}

	\p{10}{Rapporter}{}
	\begin{paragrafer}
		\subp{A}{Hur mår alla?}{\info}
		Punkten protokollfördes ej.

		\subp{B}{Utskottsrapporter}{\info}
		Punkten protokollfördes ej.

		\subp{C}{Ekonomisk rapport}{\info}
		Ekonomin ser kanonfin ut - Pengar rullar in som det ska, det går bra nu!
		\subp{D}{Kåren informerar}{\info}
		Balmästare saknas till nyårsbalen och är öppen att söka för den som är intresserad.
    Priset på Rulle har blivit  dyrare. Kostnaden är nu 4 kr/km och minsta debitering 120 kr.
    Kåren jobbar med en remiss på alkohol och droger. Kårens Nollningsutskott har omstrukturerats.
    Från och med nästa år kommer Nollegeneralen inte vara arvoderad och har delats upp i två poster.
    Nästa helg är det Styrelseutbildning. Det kommer en extern föreläsare som brukar vara uppskattad.

		\subp{E}{Omvärldsrapport}{\info}

		Denna vecka finns det ingenting nytt att rapportera.

	\end{paragrafer}

	\p{11}{Expo}{\dis}
	Mötet diskuterade vad som skulle göras på expot.


	\p{12}{Funktionärstack}{\dis}
	Punkten diskuterades och några förslag om vad man kan göra kom upp.

	\p{13}{Samarbete med WEIQ}{\bes}
	Daniel, Magnus och Malin hade möte med WEIQ som letade VIP-kunder för att testa deras app och koncept.
	De var positiva och trodde att detta kan vara sågot som kan gynna sektionen.
	Mötet belsutade att testa deras lösning på ett gille senare i år.

	\p{14}{Rengöring av fikafika}{\dis}
	De nya riktlinjerna för städningen av fikafika ansågs vara omständiga och att det inte alltid behövdes städas så noga.
	Punkten diskuterades och det påpekades att detta som sagt bara var riktlinjer.


	\p{15}{Nästa styrelsemöte}{\bes}
	\Mba nästa styrelsemöte ska äga rum nästa torsdag 2018-10-04 klockan 12:10 i E:1124.

	\p{16}{Beslutsuppföljning}{\bes}

	Daniel \ypa skjuta upp ``Styrelsemärke'' till nästa styrelsemöte.

	\Mbaby

	Daniel \ypa stryka ``Flagga till F'' från beslutsuppföljningen.

	\Mbaby


	%\Ibfu
	\p{17}{Övrigt}{\dis}

	Sektionens Gopro är försvunnen. Den tros vara i PA som ska finkammas för att den ska komma fram!

	Henrik behöver ha in design på utskottsmärken som beslutades på VT18.

	\p{18}{OFMA}{\bes}
	{\mo} förklarade mötet avslutat 13:03.
\end{paragrafer}

%\newpage
\hidesignfoot
\begin{signatures}{3}
	\signature{\mo}{Mötesordförande}
	\signature{\ms}{Mötessekreterare}
	\signature{\ji}{Justerare}
\end{signatures}
\end{document}
