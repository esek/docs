\documentclass[10pt]{article}
\usepackage[utf8]{inputenc}
\usepackage[swedish]{babel}

\def\mo{Daniel Bakic}
\def\ms{Axel Voss}
\def\ji{Magnus Lundh}
%\def\jii{}

\def\doctype{Protokoll} %ex. Kallelse, Handlingar, Protkoll
\def\mname{styrelsemöte} %ex. styrelsemöte, Vårterminsmöte
\def\mnum{S13/18} %ex S02/16, E1/15, VT/13
\def\date{2018-05-17} %YYYY-MM-DD
\def\docauthor{\ms}

\usepackage{../e-mote}
\usepackage{../../../e-sek}

\begin{document}
\showsignfoot

\heading{{\doctype} för {\mname} {\mnum}}

%\naun{}{} %närvarane under
%\nati{} %närvarande till och med
%\nafr{} %närvarande från och med
\section*{Närvarande}
\subsection*{Styrelsen}
\begin{narvarolista}
	\nv{Ordförande}{Daniel Bakic}{E15}{}
	\nv{Kontaktor}{Axel Voss}{E15}{}
	\nv{Förvaltningschef}{Magnus Lundh}{E15}{}
	\nv{Cafémästare}{Elin Johansson}{BME16}{}
	\nv{Øverphøs}{Andreas Bennström}{BME16}{}
	%\nv{SRE-ordförande}{Fanny Månefjord}{BME16}{}
	\nv{ENU-ordförande}{Isabella Hansen}{E16}{}
	\nv{Sexmästare}{Alexander Wik}{BME17}{}
	\nv{Krögare}{Malin Heyden}{E16}{}
	\nv{Entertainer}{Adam Belfrage}{BME17}{}
\end{narvarolista}
\subsection*{Ständigt adjungerande}


\begin{narvarolista}
	%\nv{Inköps- och lagerchef}{Sofie Johannesson}{E17}{}
	%\nv{Inköps- och lagerchef}{Fabian Sondh}{E17}{}
	%\nv{Inköps- och lagerchef}{Albin Pålsson}{E17}{}
	%\nv{Kårordförande}{Linus Hammarlund}{}{}
	%\nv{Kårrepresentant}{Jacob Karlsson}{}{\nafr{3}}
	%\nv{Kårrepresentant}{Agnes Sörliden}{}{}
	%\nv{Valberedningens ordförande}{Pontus Landgren}{}{}
	%\nv{Skattmästare}{Olle Oswald}{}{}
	%\nv{Kårrepresentant}{Daniel Damberg}{}{}
	%\nv{Kårrepresentant}{John Alvén}{}{}
	\nv{Nollegeneral}{Jakob Nilsson}{}{}
	%\nv{Skyddsombud}{Axel Sandqvist}{E17}{}
	%\nv{Saga}{}{}{}
	%\nv{Max}{}{}{}
	\nv{Talman}{Erik Månsson}{E14}{}
	%\nv{Elektras Ordförande}{Elisabeth Pongratz}{}{}
	%\nv{Inspektor}{Monica Almqvist}{}{}
	%\nv{Sigillbevarare}{Henrik Ramström}{}{}
	%\nv{Vice Entertainer}{Emil Bergström}{}{}
\end{narvarolista}

\begin{comment}
\subsection*{Adjungerande}
\begin{narvarolista}
	%\nv{Post}{Namn}{Klass}{}
\end{narvarolista}
\end{comment}

\section*{Protokoll}
\begin{paragrafer}
	\p{1}{OFMÖ}{\bes}
	Ordförande {\mo} förklarade mötet öppnat 12:13.
		
	\p{2}{Val av mötesordförande}{\bes}
	{\valavmo}
		
	\p{3}{Val av mötessekreterare}{\bes}
	{\valavms}
		
	\p{4}{Val av justeringsperson}{\bes}
	{\valavj}
		
	\p{5}{Godkännande av tid och sätt}{\bes}
	{\tosg}
		
	\p{6}{Adjungeringar}{\bes}
	{\ingaadj}
	%Förnamn Efternamn adjungerades
		
	\p{7}{Godkännande av dagordningen}{\bes}
	Adam \ypa lägga till punkten \S12 ``Esek events''.
	
	Andreas \ypa lägga till punkten \S13 ``Grillkväll''.
	%Dagordningen godkändes.
	%Föredragningslistan godkändes med yrkandet.
	
	Föredragningslistan godkändes med samtliga yrkanden.
		
	\p{8}{Föregående mötesprotokoll}{\bes}
	%\latillprot{}
	\ingaprot
		
	\p{9}{Fyllnadsval och entledigande av funktionärer}{\bes}
	\begin{fyllnadsval} %"Inga fyllnadsval." fylls i automatiskt
	\end{fyllnadsval}
		
	\p{10}{Rapporter}{}
	\begin{paragrafer}
		\subp{A}{Hur mår alla?}{\info}
		Punkten protokollfördes ej.
				
		\subp{B}{Utskottsrapporter}{\info}
		Axel har fixat Protokoll. InFu har inte gjort så mycket.

		Magnus har bokfört.

		Elin har planerat städning, caféet stänger idag.

		Nöju har planerat inför nollningen. De hade märkespåtagning i veckan, det blev lyckat.

		Enu har inte gjort så mycket, de hade möte tillsammans med Björn från phøset med unionen inför event under nollningen.

		Nollu har haft möte med utskotten och de ska stämma av så alla vet vad som ska göras inför nollningen. 

		Alexander och Davida har planerat budgetar. Sexet har haft möte med K och A och planerat intersektionell sittning.

		Malin har planerat inför sitt sista gille. Det blir kul, kom dit.

		Daniel har skrivt nytt kontrakt till nästa års teknikfokus.
		\subp{C}{Ekonomisk rapport}{\info}
		Ekonomin ser väldigt bra ut enligt Magnus!
				
		\subp{D}{Kåren informerar}{\info}
				
		Det saknas fortfarande en person till heltidarposten studiesocialt-ansvar på kåren. Det kommer inhandlas nya bord till gasque. GDPR-policyn kommer publiceras inom kort.  

		\subp{E}{Omvärldsrapport}{\info}
				
		Denna vecka finns det ingenting nytt att rapportera.
						
	\end{paragrafer}
		
	\p{11}{Äska pengar till lådbilsdäck}{\dis}
	
		Andreas vill att sektionen ska köpa in nya däck till lådbilen.

		Andreas \ypa köpa in nya däck för en budget på 500kr, att kostnaden belastar dispositionsfonden, samt att detta läggs till beslutsuppföljningen s15 med honom själv som ansvarig. 
	
		\Mbaby

	\p{12}{Esek-events}{\dis}

	Adam tycker inte att informationsspridning i esek-events fungerar så bra, speciellt inte för att locka till evenemang.
	Han undrar om mötet har förslag på vilka förbättringar som kan göras. 

	Mötet diskuterade frågan.

	Axel lovar att fixa en mall till skärmarna där man kan se vad som kommer hända i veckan.
	
	\p{13}{Grillkväll}{\dis}
	
	Andreas lyfte frågan om pengar till Grillkväll. 
	
	Mötet diskuterade frågan.

	Mötet tycker det är bäst om nollorna själva betalar för grillkvällen. De kommer behöva anmäla sig till eventet sen så köper sektionen in mat.

	\p{14}{Nästa styrelsemöte}{\bes}
	\Mba nästa styrelsemöte ska äga rum nästa torsdag 2018-08-30 klockan 12:10 i E:1124.
		  
	\p{14}{Beslutsuppföljning}{\bes}
		
	Malin \ypa stryka ``Kaffekokare'' från beslutsuppföljningen.
	
	\Mbaby 
		
	Axel \ypa skjuta upp ``Inköp av skärm''.
	
	\Mbaby 
	
	Axel \ypa skjuta upp ``Inköp av soundboks''.
	
	\Mbaby
	
		
	%\Ibfu
	\p{15}{Övrigt}{\dis}
		
	Axel vill att sena handlingar ska komma ut inför möte om punkter ska läggas till på dagordningen. Då får alla bättre diskussionsunderlag inför mötet. 

	Axel fixar doodle inför sommarfoto.
	
	\p{16}{OFMA}{\bes}
	{\mo} förklarade mötet avslutat 12:51.
\end{paragrafer}

%\newpage
\hidesignfoot
\begin{signatures}{3}
	\signature{\mo}{Mötesordförande}
	\signature{\ms}{Mötessekreterare}
	\signature{\ji}{Justerare}
\end{signatures}
\end{document}
