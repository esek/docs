\documentclass[10pt]{article}
\usepackage[utf8]{inputenc}
\usepackage[swedish]{babel}

\def\mo{Daniel Bakic}
\def\ms{Axel Voss}
\def\ji{Elin Johansson}
%\def\jii{}

\def\doctype{Protokoll} %ex. Kallelse, Handlingar, Protkoll
\def\mname{styrelsemöte} %ex. styrelsemöte, Vårterminsmöte
\def\mnum{S22/18} %ex S02/16, E1/15, VT/13
\def\date{2018-10-19} %YYYY-MM-DD
\def\docauthor{\ms}

\usepackage{../e-mote}
\usepackage{../../../e-sek}

\begin{document}
\showsignfoot

\heading{{\doctype} för {\mname} {\mnum}}

%\naun{}{} %närvarane under
%\nati{} %närvarande till och med
%\nafr{} %närvarande från och med
\section*{Närvarande}
\subsection*{Styrelsen}
\begin{narvarolista}
	\nv{Ordförande}{Daniel Bakic}{E15}{}
	\nv{Kontaktor}{Axel Voss}{E15}{}
	\nv{Förvaltningschef}{Magnus Lundh}{E15}{}
	\nv{Cafémästare}{Elin Johansson}{BME16}{}
	\nv{Øverphøs}{Andreas Bennström}{BME16}{}
	%\nv{SRE-ordförande}{Fanny Månefjord}{BME16}{}
	\nv{ENU-ordförande}{Isabella Hansen}{E16}{}
	\nv{Sexmästare}{Alexander Wik}{BME17}{}
	%\nv{Krögare}{Malin Heyden}{E16}{}
	\nv{Entertainer}{Adam Belfrage}{BME17}{}
\end{narvarolista}
\subsection*{Ständigt adjungerande}


\begin{narvarolista}
	%\nv{Vice Krögare}{Stephanie Bol}{BME17}{}
	%\nv{Stridsrop}{Tove Börjesson}{E17}{}
	%\nv{Inköps- och lagerchef}{Sofie Johannesson}{E17}{}
	%\nv{Inköps- och lagerchef}{Fabian Sondh}{E17}{}
	%\nv{Inköps- och lagerchef}{Albin Pålsson}{E17}{}
	%\nv{Kårordförande}{Linus Hammarlund}{}{}
	%\nv{Kårrepresentant}{Jacob Karlsson}{}{\nafr{3}}
	%\nv{Kårrepresentant}{Hanna Järpedal}{}{}
	\nv{Kårrepresentant}{Philip Johansson}{\nati{17}}{}
	%\nv{Valberedningens ordförande}{Pontus Landgren}{}{}
	%\nv{Skattmästare}{Olle Oswald}{}{}
	%\nv{Kårrepresentant}{Daniel Damberg}{}{}
	%\nv{Kårrepresentant}{John Alvén}{}{}
	%\nv{Nollegeneral}{Jakob Nilsson}{}{}
	%\nv{Skyddsombud}{Axel Sandqvist}{E17}{}
	%\nv{Saga}{}{}{}
	%\nv{Max}{}{}{}
	%\nv{Talman}{Erik Månsson}{E14}{}
	%\nv{Elektras Ordförande}{Elisabeth Pongratz}{}{}
	%\nv{Inspektor}{Monica Almqvist}{}{}
	\nv{Sigillbevarare}{Henrik Ramström}{E16}{}
	%\nv{Vice Entertainer}{Emil Bergström}{}{}
\end{narvarolista}

%\begin{comment}
\subsection*{Adjungerande}
\begin{narvarolista}
	\nv{Teknokrat}{Oscar Uggla}{E15}{}
	\nv{Näringslivskontakt}{Filip Larsson}{E17}{}
	\nv{Teknikfokusansvarig}{Johan Vikstrand}{E17}{}
	%\nv{Post}{Namn}{Klass}{}
\end{narvarolista}
%\end{comment}

\section*{Protokoll}
\begin{paragrafer}
	\p{1}{OFMÖ}{\bes}
	Ordförande {\mo} förklarade mötet öppnat 12:15.

	\p{2}{Val av mötesordförande}{\bes}
	{\valavmo}

	\p{3}{Val av mötessekreterare}{\bes}
	{\valavms}

	\p{4}{Val av justeringsperson}{\bes}
	{\valavj}

	\p{5}{Godkännande av tid och sätt}{\bes}
	{\tosg}

	\p{6}{Adjungeringar}{\bes}
	%{\ingaadj}
	%Förnamn Efternamn adjungerades
	Filip Larsson och Johan Vikstrand adjungerades.

	\p{7}{Godkännande av dagordningen}{\bes}
	
	Axel \ypa lägga till \S15 ``Photoshop till Picasso''.

	%Dagordningen godkändes.
	Föredragningslistan godkändes med yrkandet.
	%Föredragningslistan godkändes med samtliga yrkanden.

	\p{8}{Föregående mötesprotokoll}{\bes}
	%\latillprot{}
	\ingaprot

	\p{9}{Fyllnadsval och entledigande av funktionärer}{\bes}
	\begin{fyllnadsval} %"Inga fyllnadsval." fylls i automatiskt
		%\fval{namn}{post}
		%\entl{Lisa Linárd Pedersen}{SRE-ledamot}
	\end{fyllnadsval}

	\p{10}{Rapporter}{}
	\begin{paragrafer}
		\subp{A}{Hur mår alla?}{\info}
		Punkten protokollfördes ej.

		\subp{B}{Utskottsrapporter}{\info}

		Adam har planerat ølresa. De planerar killergame tillsammans med data.

		Alexander har planerat en sittning i samband med musikhjälpen.

		ENU hade cv-granskning med academic work.

		Johan informerar om att det går bra för teknikfokus. De har låst in en tredjedel av alla företag. De ligger i väldigt bra fas. 

		NollU har inte gjort så mycket. De har skickat ut en intresseanmälan för att kunna se var de kan hålla tacket någonstans. De kommer också börja utvärdera nollningen. 

		Elin har varit på kollegiemöte. De diskuterade mycket om vad mer man kan göra med kollegiet. De kommer hålla stängt nästa vecka. 

		Magnus har bokfört, räknat pengar och skickat fakturor.

		Daniel hade lunchmöte med ledningen. Igår hade han även möte med kollegiet, de pratade om musikhjälpen. Redan idag kan man börja skänka pengar. 

		\subp{C}{Ekonomisk rapport}{\info}

		Ekonomin ser bra ut!

		\subp{D}{Kåren informerar}{\info}

		Det var fullmäktigemöte i tisdags. FM klubbade sina åsiktet angående CSV. Det går att kandidera och nominera till fm. Nästa vecka står kåren och delar ut tentafrukost. Det behövs även jobbare till nyårsbalen. 

		\subp{E}{Omvärldsrapport}{\info}

		Denna vecka finns det ingenting nytt att rapportera.


	\end{paragrafer}

	\p{11}{Utomstående Förtjänstmedalj}{\bes}
	
	Axel lyfte att han gärna vill ge Elin Vilhelmsson ``E-sektionens Utomstående Förtjänstmedalj'' som tack för att hon ställde upp helt ideellt under hela dagen och kvällen under gasque. 

	Sigillbevarare Henrik Ramström är också positiv till utdelandet av medaljen.
	
	Daniel \ypa ge en medalj till Elin Vilhelmsson.

	\Mbaby

	\p{12}{Äskning av pengar till Ölresa}{\bes}

	Adam presenterade anledningen till äskningen. 

	Mötet diskuterade frågan grundligt. 

	Det som lyftes var bland annat att det inte är ett event för alla sektionens medlemmar eftersom det bara är en del som har möjlighet att delta samt att det är ett event som inte alla är intresserade av.

	Mötet var eniga om att pengarna också kan gå på Nöjus budget.

	Mötet beslutade att avslå Adams yrkande.

	\p{13}{Funktionärstack}{\dis}

	Elin lyfte fråga angående Funktionärstack. Hon undrade var vi ska vara och kom med förslaget att hyra en stuga någonstans.

	Mötet diskuterade frågan. 

	Philip Johansson informerade om att det finns en stuga som Lunds Universistets studentkår äger, denna kan man hyra relativt billigt. 
	
	Axel tyckte att vi borde undersöka hur många som faktiskt är intresserade av att delta på funktionärstacket.

	Axel \ypaltbu{Intressekoll}{s23}{honom själv}

	\Mbaby

	\p{14}{Informationsspridning}{\dis}

	Daniel vill gärna att informationen som går ut på våra informationskanaler är på engelska också. Vi har många internationella studenter som är intresserade av att engagera sig mer i sektionen och de vill jättegärna delta på event också. 

	Daniel tycker att vi ska utvärdera bonsaicampus betalsystem. 

	Axel vill gärna att vi använder hemsidan mer nu när vi har ett effektivt sätt att skicka ut notiser. 
	
	Adam \ypaltbu{Undersöka bonsaicampus betalsystem}{S23}{Magnus Lundh}

	\Mbaby

	\p{15}{Photoshop}{\dis}

	Axel undrade hur mötet ställer sig till att subventionera photoshoplicenser till de som är Picasso.
	De har fått ta pengar ur sina egna fickor under hela året för ett ändamål som bara är till för sektionen.
	Sektionen har bland annat köpt in en ny kamera för att fotograferna inte ska behöva använda egna och Axel tycker att detta går i linje med tidigare fattade beslut om stöd till funktionärer.
	
	\Mdf

	Mötet va positiva till att stå för betalningen av Photoshoplicenser.

	Axel ska återkomma med ordentliga handlingar efter att ha pratat med nuvarande picassos. 

	\p{15}{Nästa styrelsemöte}{\bes}

	\Mba nästa styrelsemöte ska äga rum fredag 2018-11-09 klockan 12:10 i E:1124.

	\p{16}{Beslutsuppföljning}{\bes}

	Elin \ypa stryka ``Inköp pantkärl''

	Magnus \ypa stryka ``Inköp flagga''.

	Daniel \ypa skjuta upp ``Inköp styrelsemärke''.

	\Mbabay

	%\Ibfu
	\p{17}{Övrigt}{\dis}

	Johan Vikstrand undrade hur det ser ut angående teamdrive till teknikfokus. Axel svarade att sektionen har en teamdrive som de gärna får använda.

	\p{18}{OFMA}{\bes}
	
	{\mo} förklarade mötet avslutat 13:10.

\end{paragrafer}

%\newpage
\hidesignfoot
\begin{signatures}{3}
	\signature{\mo}{Mötesordförande}
	\signature{\ms}{Mötessekreterare}
	\signature{\ji}{Justerare}
\end{signatures}
\end{document}
