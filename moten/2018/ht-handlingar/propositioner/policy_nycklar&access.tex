\documentclass[../_main/handlingar.tex]{subfiles}

\begin{document}
\proposition{Uppdatering av policy för nycklar och access}
Vi i styrelsen har kollat över sektionens nyckelpolicy och märkt att den är extremt otydlig och utdaterad. Därför vill vi införa en ny policy för nycklar och även inkludera policy för dörr-access i denna policy som ersätter den gamla nyckelpolicyn.

Därför yrkar styrelsen på

\begin{attsatser}
    \att uppdatera policybeslutet \emph{Nyckelpolicy för E-sektionens funktionärer} till det bifogade förslaget.
\end{attsatser}

\begin{signatures}{1}
    \ist
    \signature{Elin Johansson}{Cafémästare}
\end{signatures}

\newpage
\section*{Policybeslut: Nyckel- och accesspolicy för E-sektionens funktionärer}
\subsubsection*{Allmänt}
Funktionären bör ha tillgång till de nycklar och den access som krävs för att sköta posten. Sektionen är dock inte skyldig att tilldela en funktionär nyckel eller access, och det är således ingen rättighet för funktionären.

Sektionsstyrelse kan dock av kostnadsskäl, allmän tjuvaktighet eller annan anledning dra ner på den generella nyckel- eller accesstilldelningen.

Beslut om att inte tilldela funktionär nyckel eller access fattas av Styrelsen. Tilldelning av nycklar och access sköts normalt av Förvaltningschef eller Sektionsordförande. Om oenighet uppstår mellan dem och funktionären bestämmer Styrelsen.

Denna policy gäller för alla Sektionens nycklar förutom den gemensamma nyckeln till skåpen i LED Café. Den gemensamma nyckeln skall vara tydligt markerad och ansvaret för den åligger Cafémästaren. Cafémästaren är dock inte ersättningsskyldig om nyckeln skulle försvinna.

\begin{enumerate}[label=\S\arabic*.]
    \item Om funktionär lämnar in avsägelse, studerar utomlands, eller av annan anledning inte förmodas kunna fullfölja sina uppgifter, skall nycklarna utan dröjsmål lämnas tillbaka och access ska tas bort.
    \item För att få en nyckel utkvitterad ska funktionären acceptera och skriva under avtalet för nyckelutlämning samt betala angiven deposition.
    \item Funktionär är skyldig att omedelbart efter mandatperiodens slut inlämna utkvitterade nycklar. Om nycklarna inte lämnas in ska en avgift betalas enligt avtalet för nyckelutlämning.
    \item Funktionären ansvarar för sina nycklar och sin access, och kan bli ersättningsskyldig för såväl skador som stölder som orsakas av misskötsel från funktionärens sida. Detta kan gälla försummelse att låsa Sektionens lokaler eller utlåning av nycklar/LU-kort till obehörig.
    \item Förlust av nyckel skall omedelbart anmälas till Sektionsordförande. Sektionens åtgärder i sådant fall bestäms av Styrelsen, och i brådskande fall av ordföranden.
    \item Funktionär som tilldelats nyckelbricka, är skyldig att alltid låta denna sitta tillsammans med utkvitterade nycklar.
    \item Vid misstanke om oegentligheter, misskötande av förtroendeuppdrag eller dylikt, äger Sektionsordförande rätt att omedelbart och tillfälligt dra in funktionärs nyckel eller access. Beslut om permanent indragning tas av styrelsen.
\end{enumerate}
\end{document}
