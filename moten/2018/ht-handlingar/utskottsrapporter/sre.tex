\documentclass[../_main/handlingar.tex]{subfiles}

\begin{document}
\utskottsrapport{Studierådet}

Studierådet och dess medlemmar sysslar främst med studiebevakning. Detta sker kontinuerligt under året genom granskning av CEQ-enkäter, uppföljningar av studentärenden och utvärderingsmöten för kurser. Vi har representanter från både E och BME i årskurserna 1-3 vilket innebär att vi redan har fått in två nya årskursansvariga bland de nyantagna studenterna vilket är roligt! Vi har även studenter från årskurserna 4-5 med i utskottet och det ger oss en bättre inblick i master-kurser. 

Under nollningen har SRE varit synliga på events som utskottssafarin, SRE-workshopen och dessutom har vi haft fyra välbesökta pluggkvällar. Vi är nöjda med utformningen av pluggkvällarna i år eftersom det har blivit mer fokus på studier. Detta mycket tack vare att vi bokade egna salar för pluggandet och att vi hade duktiga pluggphaddrar som hjälpte till. Vi har också bjudit på rejäla måltider för att folk ska orka sitta kvar länge. ENU fixade så att en pluggkväll blev sponsrad av ett företag. Efter nollningen har vi arrangerat en utbyteskväll där studenter samlades för att hjälpas åt med ansökan för utbytesstudier. 

SRE har studentrepresentanter i programledningarna för E och BME samt några  institutionsstyrelser. Här tas det beslut som gäller programmen, t.ex. när nya kurser ska väljas in. Utskottet håller tät kontakt med Teknologkåren via utskotten SRX, LiBe, VMX och SkyX. I SRX träffas alla studierådsorförandena från alla sektioner och diskuterar vad som händer på kårnivå, i respektive studieråd, samt andra tips och idéer. 

\begin{signatures}{1}
    \mvh
    \signature{Fanny Månefjord}{SRE-Ordförande}
\end{signatures}

\end{document}
