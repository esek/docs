\documentclass[../_main/handlingar.tex]{subfiles}

\begin{document}
\utskottsrapport{Nolleutskottet}
I slutet av våren hade NollU först temasläpp tillsammans med Staben och A-phøs i Gasquesalen. Det gick jättebra. Innan sommaren hölls också en phadderkickoff för att phaddrarna skulle lära känna varandra bättre och börja tagga igång inför nollningen.

Under hösten har NollU hållit i En Lysande Nollning. Överlag har det gått super och nollorna har verkat väldigt nöjda med nollningens aktiviteter. Det var stor uppslutning på nästan alla evenemang vilket var otroligt roligt. Det fanns några nya event på schemat men också många traditionella aktiviteter. För tredje året i rad vann E-sektionen FlyING, vilket innebär att hattricket blev komplett. Vi vann även WaDerloo. Tyvärr var inte alltid vädret på vår sida i alla lägen. Detta innebar bland annat att aktiviteterna på St Hans fick avbrytas halvvägs, att grillkvällen med styrelsen istället blev filmkväll, att märkesmålningen för andra året i rad fick flyttas och att DE-lekarna blev inställda. Förutom dessa fadäser så har alla evenemang gått mer eller mindre som planerat.

Efter nollningen har en utvärdering skickats ut till både nollor och phaddrar så att de har kunnat berätta vad de tyckte om nollningen och ge förslag på förändringar och förbättringar till nästa års nollning. Det som är kvar att göra för nollningsutskottet under resten av året är att sammanställa utvärderingarna samt jobba med testamente och överlämning så att nästa års nollningsutskott har så bra förutsättningar som möjligt när de kliver på.

\begin{signatures}{1}
    \mvh
    \signature{Andreas Bennström}{Øverphøs}
\end{signatures}

\end{document}
