\documentclass[../_main/handlingar.tex]{subfiles}

\begin{document}
\utskottsrapport{Cafémästeriet}
Efter vårterminsmötet fick CM en ny lager- och inköpschef, Jonathan Benitez. Han bidrog inte bara med minskad arbetsbelastning för inköparna utan också med ny energi och positiv anda i utskottet. Läsperiod fyra var mycket tuff; det var svårt att få tag i jobbare på grund av Karnevalen och det var väldigt lite folk i skolan som ville handla. Men läsperioden avslutades med en lyckad Caféfest tillsammans med F, D och K och vid bokslutet var hela terminens försäljning väl inom budget. 
Även i början av höstterminen ekade jobbarschemat tomt så vi tog beslutet att hålla stängt läsperiodens första två veckor. Avsaknaden av kaffe och mycket promoting gjorde att schemat snabbt fylldes på och försäljningen gick i toppen direkt när vi öppnade. Upplägget för att integrera phaddergrupperna i LED fungerade väldigt bra. Det verkar som att majoriteten av de phaddrar och nollor som jobbade uppskattade det och många av dem är dioder nu. 
Utöver att försöka få verksamheten att rulla på har vi sålt kaffe till andra utskott och företag. Eftersom arbetsbelastningen redan varit väldigt hög har vi varit tvungna att tacka nej till några erbjudanden om att sälja mackor till kvälls- och lunchevent. 
Utbudet i LED har varit ganska intakt. Under den extremt varma försommaren sålde vi stora mängder glass samt grönsaksstavar med hummus.  Under några tillfällen i höst har vi testat att sälja yoghurt med müsli och olika sorters wraps. Alla nya saker har varit uppskattade hos kunderna och om vi hade haft mer tid skulle vi gärna gjort det oftare.
\begin{signatures}{1}
    \mvh
    \signature{Elin Johansson}{Cafémästare}
\end{signatures}

\end{document}
