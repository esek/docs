\documentclass[../_main/handlingar.tex]{subfiles}

\begin{document}
\utskottsrapport{Näringslivsutskottet}
I slutet av våren hade vi ``Lunch med en ingenjör'' som var väldigt lyckat och både studenter och företag var nöjda. ENU planerade mycket inför hösten och vi tog kontakt med så många företag som möjligt. Vi fick tidigt ett fullspäckat schema med lunchföreläsningar till nollningen vilket var kul. Vi samarbetade också med phøset för att få in sponsorer till nollningen där vi bland annat fick vatten från TetraPak precis som föregående år.
Vi höll i SVEP och tre stycken lunchföreläsningar under nollningen. Vi hade även fixat så att en av pluggkvällarna var sponsrad. Allt detta gick väldigt bra och alla företagen var nöjda.
Efter nollningen höll vi tillsammans med Academic Work en CV-granskning där studenterna fick 15 minuter med en person från företaget. Detta var väldigt uppskattat både av studenterna och företaget.
Det har även varit en alumnipub som var lyckad. KM höll i denna tillsammans med våra alumniansvariga som gjorde ett väldigt bra jobb. Det kom många alumner och vi i ENU hoppas på att vi kan fortsätta ha fler evenemang med alumnerna då dessa kan vara väldigt bra kontakter för studenterna.

Vi ska ha en lunchföreläsning med BorgWarner som förhoppningsvis blir jättebra. Detta är ett företag som vi inte har haft innan och det är kul att få in sådana företag.
Tillsammans med F- och D-sektionen har vi planerat en ny FED-pub som D-sek ska hålla i den 5e december. Vi hoppas att denna gång få in fler företag än sist men att det ska komma lika många studenter som förra gången.
ENU har även en lunchföreläsning inplanerad med Academic work.

\begin{signatures}{1}
	\mvh
	\signature{Isabella Hansen}{ENU-ordförande}
\end{signatures}

\end{document}
