\documentclass[../_main/handlingar.tex]{subfiles}

\begin{document}
\utskottsrapport{Informationsutskottet}
Chefredaktören som blev invald efter vårteminsmötet har kämpat på med HeHe ensam och det har varit kvalité framför kvantitet på tidningen i år. På grund av GDPR samt att utskicken ofta verkar hamna i spamkorgen har vi inte använt oss av maillistorna, istället har en kopia av tidningen lagts upp i vår facebook grupp. Hur man ska gå tillväga nästa år måste utredas vidare.

Picassosarna har fortsatt att göra ett extremt bra jobb med många fina affischer. De har även försökt göra bilder till skärmarna i samma veva som de fått en plansch-beställning så att även dessa visar aktuell information.

Det köptes in en riktig fin sektionskamera innan nollningen men trots detta har det inte fotograferats så mycket av våra egna fotografer som jag hade velat. De har haft mycket annat för sig och sektionsarbetet har därför tyvärr blivit lite bortprioriterat på den fronten. Som tur är har vi fått hjälp av andra funktionärer när det krisat och under gasque fick vi hjälp av A-sektionen.

DDG fortsätter att träffas hyfsat regelbundet, jag har entledigat en del kodhackare som inte varit aktiva och som inte känner att de har tid att göra mer nu de sista månaderna heller. 
Jag har tillsammans med Macapären fortsatt att underhålla våra hemsidor och även börjat planera för en serveruppgradering. Det är ett jobb som kommer ta mycket tid men som jag hoppas kunna fortsätta på när jag gått av.

Teknokraterna har sett över vår tekniska utrustning och börjat med att föra en riktig dokumentation av vad vi har för utrustning samt hur saker och ting fungerar. Det är viktigt vid överlämningen och det är något som jag tycker att även DDG ska förbättra.
Tyvärr gick två passiva toppar i Vega sönder under hösten och var bortom all räddning, en proposition har lagts fram inför HT/18. Annars mår vår teknik bra.  

I år har jag jobbat med att försöka få ett bra sammarbete med de andra högskolorna i Sverige och övriga Norden. Jag tycker jag lyckades mycket bra på den fronten och vi har nästan veckolig kontakt med KTH och Chalmers men även med Trondheim och Aalto. Jag hoppas att nästa års styrelse tar till vara på detta och försöker representera både sektionen och Lund när chansen väl ges. 

\begin{signatures}{1}
    \mvh
    \signature{\sekr}{Kontaktor}
\end{signatures}

\end{document}
