\documentclass[../_main/handlingar.tex]{subfiles}

\begin{document}
\section{Ekonomisk rapport}
Sektionens ekonomi mår bra och vi har en god förmåga att klara av vår dagliga verksamhet och betala de kortfristiga skulder vi har. Några långfristiga skulder såsom banklån eller liknande har vi inte i nuläget.
I skrivande stund (2018-11-12) uppgår våra tillgångar, lager exkluderat, till \SI{944 591,49}{kr}. Eftersom E-sektionen är en icke vinstdrivande organisation placeras eventuella överskott in i fiktiva fonder, öronmärkta till olika ändamål inom vår verksamhet.
\begin{dashlist}
    \item Eget kapital, som är de pengar som behövs för daglig verksamhet.
    \item Olycksfonden ska användas vid reparationer och olycksfall.
    \item Dispositionsfonden är pengar som styrelsen kan använda för investeringar och för att förbättra sektionens verksamhet.
    \item Utrustningsfonden ska användas till större investeringar, projekt och inköp.
\end{dashlist}

Se bifogad balansrapport för mer information. Det finns däremot ett behov av renoveringar och investeringar som kommer behövas de närmaste åren och därför bör vi fortsatt budgetera överskott.

\begin{signatures}{1}
    \mvh
    \signature{Magnus Lundh}{Förvaltningschef}
\end{signatures}

\end{document}
