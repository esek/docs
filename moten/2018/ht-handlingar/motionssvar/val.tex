\documentclass[../_main/handlingar.tex]{subfiles}

\begin{document}
\motionssvar
Vi i styrelsen anser att den bästa lösningen är att samtliga funktionärer i nolleutskottet väljs på valmötet. Detta tillvägagångsätt har flera fördelar som vi ser det:

\begin{dashlist}
	\item Alla funktionärer prioriteras lika, eftersom alla väljs samtidigt
	\item Istället för en sluten valberedningsprocess och ett val som i stort sett tas av valberedningen och överphös får alla medlemmar vara med och rösta, vilket även ger mer transparens i valet.
	\item Man kan då söka både överphös, cophös, och andra poster om man så önskar
	\item Eftersom styrelsen väljs först, har överphös en chans att tala för de kandidater den tycker är mest lämpade under valet av cophös och får därför viss påverkan på valet.
	\item Precis som alla andra ansvarsposter på sektionen är det mötet som väljer in cophös
\end{dashlist}

Styrelsen yrkar därför på

\begin{attsatser}
    \att avslå motionen i sin helhet.
\end{attsatser}


\begin{signatures}{1}
	\ist
	\signature{Daniel Bakic}{Ordförande}
\end{signatures}

\end{document}