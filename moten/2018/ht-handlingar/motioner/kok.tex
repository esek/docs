
\documentclass[../_main/handlingar.tex]{subfiles}

\begin{document}
\motion{Uppgradera utrustning i Edekvatas kök}
Efter årets alla sittningar har allt fler brister i Edekvatas kök uppmärksammats av
Köksmästarna 2018 samt Preferensmästaren 2018. Det kan vara alltifrån att spisen inte
längre går att använda för att proppen går konstant, brist på effektiva verktyg som
underlättar arbetet och andra saker som gör att arbetet inför, under samt efter sittningar inte
flyter på lika effektivt som det skulle kunna göra.
Mycket av utrustningen som underlättar E6 arbete skulle vara möjlig att låna av andra
sektioner här på LTH, men under nollningen då många sektioner har sittningar samtidigt är
det svårt att låna deras utrustning, vilket är problematiskt då det är denna period under året
som E6 som mest behöver använda nämnda utrustning.

Därför yrkar köket 2018 på:

\begin{attsatser}
\att tillsätta en utredning av elen till spisen som görs av en facklig hantverkare,
\att köpa storköksstavmixer av modell RCSM-SET1 med tillhörande utrustning av företaget
Royal catering,
\att öpa reservdelar till matberedaren,
\att öpa ett stort durkslag / sil,
\att öpa in fler tunna skärbrädor i plast för att ersätta nuvarande, tjocka skärbrädor,
\att rensa bort saker som anses endast ta plats i köksförvaringen utan att användas
regelbundet,
\att avsätta 17500 kr till projektet i sin helhet, varav 11500 av totalen går till utredningen och
6000 går till utrustningen,
\att kostnaderna belastar Utrustningsfonden,
\att detta läggs på beslutsuppföljningen till VT/19 med författarna som huvudansvariga.
\end{attsatser}
\begin{signatures}{3}
    Signerat
    \signature{Fredrik Berg}{Köksmästare 2018}
    \signature{Filip Larsson}{Köksmästare 2018}
    \signature{Y Nhi Pham}{Preferensmästare 2018}

\end{signatures}

\end{document}
