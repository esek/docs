\documentclass[../_main/handlingar.tex]{subfiles}

\begin{document}
\motion{Vice-poster som styrelsens suppleanter}

\def\postarvice{Denna post är en vice till utskottsordföranden.}
\newcommand{\laggatillpostarvice}[2]{
\att i reglementet, under #1, under ``#2'', lägga till
\begin{dashlist}
    \item \postarvice
\end{dashlist}
}

I samband med införandet av vice-kavajerna på HT/17 uppkom det diskussioner om vilka poster som egentligen är vice till styrelsen och vad det innebär att vara vice. I en del utskott är detta väldigt tydligt, till exempel i Källarmästeriet. Vice Krögare kan hålla i egna gillen och på så vis ersätta eller komplettera Krögaren. Förvaltningsutskottet och Nollningsutskottet är exempel på utskott där strukturen är mindre tydlig. Till Förvaltningschefen finns det två möjliga vice: Skattmästaren och Vice Förvaltningschef. Vi anser att båda dessa är naturliga vice men med olika ansvarsområden. I Nollningsutskottet väljs Vice Øverphøs internt och kan i dagsläget inte kopplas till en specifik post. En möjlighet skulle kunna vara att införa en post Vice Øverphøs, men vi väljer att inte behandla det i denna motion.

I dagsläget är glappet mellan styrelsen och deras vice stort. Av erfarenhet vet vi att som vice är det väldigt svårt att få insikt inte bara i styrelsens verksamhet, men även i de övriga utskottens, sektionernas samt Kåren centrala verksamhet. Med våra föreslagna ändringar skulle vice på ett naturligt sätt komma närmare styrelsen och få bättre insikt i deras arbete. Detta skulle leda till att fler, utöver styrelsen, får insikt i sektionens kärnverksamhet. Detta möjliggör även för alla vice utskottsordförande att ta mer ansvar och få bättre förutsättningar för att avlasta och komplettera sina respektive utskottsordförande. Med anledning av detta vill vi nu lyfta frågan med ett förslag på hur detta kan genomföras.

\newpage

Vi yrkar på
\begin{attsatser}
    \att i stadgan lägga till paragraf \S8:4 Suppleanter, före nuvarande paragraf \S8:4 Mandattid, med följande text:\par

    De poster som i reglementet uttryckligen beskrivs som en \emph{vice} är suppleanter till sina respektive utskottsordförande i styrelsen. En suppleant kan, efter respektive utskottsordförandes skriftliga godkännande till Ordföranden, ställföreträda utskottsordföranden vid ett Styrelsemöte med yttrande-, yrkande- och rösträtt.

    För att en suppleant ska få ställföreträda sin respektive utskottsordförande måste denne uppfylla kraven beskrivna under \S8:3.

    \att i stagdan under \S8:6 Beslut, ändra

    Styrelsen är beslutsmässig om minst sex (6) ledamöter är närvarande.

    till

    Styrelsen är beslutsmässig om minst sex (6) ordinarie ledamöter är närvarande.

    \att i stadgan under \S8:2 Sammansättning, ändra

    Styrelsen utser inom sig en vice ordförande.

    till

    Styrelsen utser inom sig en Vice Ordförande till att ställföreträda Ordföranden vid situationer då Ordföranden inte längre har möjligheten att fullfölja sitt uppdrag. Vice Ordförande har också till uppgift att agera styrelsemötesordförande då Ordföranden inte kan närvara.

    \att i stadgan under \S8:8 Ständigt adjungerade, lägga till

    a) Styrelsens suppleanter enligt \S8:3

    före

    a) Revisorerna,

    \att i stadgan under \S8:9 Kallelse, lägga till

    b) Styrelsens suppleanter enligt \S8:3

    före

    b) Revisorerna,

    \att i stadgan under \S8:11 Skyldigheter, lägga till

    \emph{att} hålla styrelsens suppleanter välinformerade om styrelsens och Sektionens verksamhet,

    före

    \emph{att} även i övrigt på alla sätt verka för sektionens bästa.

    \newpage
    \laggatillpostarvice{10:2:E}{Vice Förvaltningschef}
    \laggatillpostarvice{10:2:E}{Skattmästaren}
    \laggatillpostarvice{10:2:F}{Vice Kontaktor}
    \laggatillpostarvice{10:2:G}{Vice Näringslivsutskottsordförande}
    \laggatillpostarvice{10:2:H}{Vice Entertainer}
    \laggatillpostarvice{10:2:I}{Vice Krögare}
    \laggatillpostarvice{10:2:J}{Vice Cafémästare}
    \laggatillpostarvice{10:2:K}{Vice Sexmästare}
    \laggatillpostarvice{10:2:M}{Vice SRE-Ordförande}
    \att de föreslagna ändringarna av reglementet tas i samband med den andra läsningen av de föreslagna ändringarna av stadgan.
\end{attsatser}

\begin{signatures}{2}
    \mvh
    \signature{Erik Månsson}{Ordförande 2017}
    \signature{Sophia Grimmeiss Grahm}{Förvaltningschef 2017}
\end{signatures}

\end{document}
