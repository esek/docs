
\documentclass[../_main/handlingar.tex]{subfiles}

\begin{document}
\motion{Suppleant för Øverphøs}
I samband med motionen för införandet av ”Vice-poster som styrelsens suppleanter” så kommer
varje utskott förutom nollningsutskottet ha en suppleant om mötet bifaller motionen. Eftersom
Øverphøs är en tidskrävande post så anser jag att en suppleant till Øverphøset kan vara till
sektionens och styrelsens nytta. Då det inte finns någon klar Vice- Øverphøs post och att i phøset så
delas arbetsbördan i olika delansvar så finns det två möjliga sätt att implementera en suppleant utan
för många revideringar i stadgan/reglementet.

\begin{enumerate}
\item Att Co-phøs posten anses vara en Vice-post och därmed så kommer varje invald Co-phøs
vara en suppleant till Øverphøset i Styrelsemötena
\item Att en eller två Co-phøs väljs av Styrelsen på rekommendation av Øverphøset till posten
\textit{Suppleant för Øverphøset} som medför att dem är vice till utskottsordförande.
\end{enumerate}

För att förtydliga så kommer detta endast ge komplikationer för vilka av posterna som borde få vice-
kavajer, men eftersom det inte finns några specifika regler för vilka som ska få dessa förutom en
”norm” att det är en vice så anser jag även att Co-phøs/Suppleant för Øverphøset inte är en
kavajpost. För motionens skull så yrkar jag på att det ska vara den andra implementationen med få
grundargument, alltså att införa en ny post.

Därför yrkar jag på

\begin{attsatser}
\att i reglementet under \S10:2:L, lägga till:\par
\begin{emptylist}
    \item Suppleant för Øverphøset (2)
      \begin{dashlist}
        \item Denna post är en vice till utskottsordföranden.
        \item Väljs av styrelsen på rekommendation av Øverphøset och Co-phøsen.
      \end{dashlist}
    \end{emptylist}
\end{attsatser}
\begin{signatures}{1}
    Signerat
    \signature{Jonathan Benitez}{Inköps- och Lagerchef}
\end{signatures}

\end{document}
