
\documentclass[../_main/handlingar.tex]{subfiles}

\begin{document}
\motion{Sektionsgrodan}
Hej
Jag är inte helt insatt i vilka beslut som tas av årsmötet och vilka som styrelsen tar (tex skapandet av funktionärsposter) men jag tänkte ändå motionera (det är nyttigt har jag hört) för att sektionen arbeta för att tilldelas den fantastiska märkningen Sektionsgrodan, som Lund sustainable engineers ger ut.

I korthet är det en miljömärkning som går ut på att vi ska ha ett systematiskt miljö och hållbarhetsarbete, uppnå 40 poäng (vad som get poäng och hur mycket vet jag inte än), och tillsätta en kontaktperson (en sektionsgroda)  som är ser till att ta fram en miljöpolicy tillsammans med sektionens medlemmar och har kontakten med LSE för att rapportera hur arbetet går. 

I-sektionen har fått denna märkning och vi vill ju inte vara sämre än dem! 

Jag kom på detta lite sent så jag vet inte så mycket om det, men följande information har jag stulit från LSE's hemsida: 

\section*{Sektionsgrodan}
Sektionsgrodan är en miljömärkning för sektionerna på LTH. Projektet har haft utgångspunkt i Hållbart Universitets lyckade koncept Nationsgrodan, en miljömärkning för Lunds studentnationer som funnits sedan 2009. Syftet med Sektionsgrodan är att öka miljömedvetenheten bland studenter samt minska sektionernas miljöpåverkan. En miljömärkning kan ses som ett verktyg för att underlätta och systematisera miljöarbete, där ett lyckat arbete synliggörs och belönas med märkningen. Även utanför Lund finns ett intresse för denna märkning, vilket ger detta projekt än större spridning i Sverige. Mer information och detaljer kring hur du ska göra för att bli märkt med Sektionsgrodan hittar du här. (lundsustainableengineers.se/sektionsgrodan
\textit{Projektledare: Maja Lindblad}

\subsection*{5.1 Miljömärkning av Sektionen}
Det finns fördelar både för er som sektion och som enskilda studenter att förknippas med en miljömärkning. Som sektion går det att spara på resurser, både ekonomiskt och materiellt. Utöver detta visar en miljömärkning för samarbetspartners, sponsorer och framtida studenter att sektionen är driven inom hållbarhet. Att redan som student komma in i “tänket” kring miljöledningssystem och förbättringsåtgärder är även värdefullt inför det framtida arbetslivet.

\subsection*{5.2 Strukturella krav för Sektionsgrodan 2017}
För att märkas med miljömärkningen Sektionsgrodan under 2017 ska sektionen:

\begin{itemize}
    \item Ha en miljöpolicy som uppdateras kontinuerligt.
    \item Sprida information om sin miljöpolicy, samt rutinerna som gäller för miljöarbetet, till dem som berörs av det.
    \item Utse en kontaktperson som ansvarar för sektionens kontakt med Sektionsgrodan.
    \item Planera, dokumentera och följa upp sitt miljöarbete i en miljöhandlingsplan.
    \item Sätta upp minst 1 miljömål.
    \item Uppfylla ett antal kriterier som motsvarar minst 40 poäng.    
\end{itemize}

\subsection{5.3 Komma igång}
Här följer en lista på hur ni steg för steg påbörjar arbetet med Sektionsgrodan.
\begin{itemize}
\item Se till att er sektion har möjlighet att uppfylla de strukturella krav som krävs för att erhålla märkningen. Ni kan hitta dessa krav under Strukturella krav eller i kriteriedokumentet “Kriterier för 2017”    
\item Läs igenom kriteriedokumentet. Skicka runt det till styrelsen och andra aktiva medlemmar i sektionen för att samla åsikter om vilka kriterier er sektion bör fokusera på. För att erhålla märkningen måste er sektion uppfylla kriterier som motsvarar minst 40 poäng (gäller för 2017).
\item Utse en kontaktperson (eller miljösamordnare) som ansvarar för kontakten mellan sektionen och Sektionsgrodans administration.
\item Skriv en miljöhandlingsplan som beskriver de kriterier ni planerar att uppfylla samt era miljömål. Det som ska vara med är:
\item Vilka kriterier/mål ni vill uppfylla
\item Hur ni planerar att uppfylla dem
\item Vem som är ansvarig
\item Hur det har gått
\item Se till att er sektion har en miljöpolicy som har godkänts av ett sektionsmöte. Passa gärna på att i den uppfylla några av de kriterier som berör policydokumentet! Ett förslag på hur en miljöpolicy kan se ut finns länkat.
\item När ni kan visa för Sektionsgrodans administration att ni uppfyller alla strukturella krav samt kriterier som motsvarar minst 40 poäng (för 2017), så kommer ni att få certifieringen i ett kalenderår. I praktiken innebär det att ni skickar in er miljöhandlingsplan och ett ifyllt kriteriedokument till oss. När vi godkänt dokumenten är ni fria att skryta om er miljömärkning på det sätt som ni anser lämpligt! Notera att rutin- och slumpmässiga kontroller kommer att utföras, samt att vi förbehåller oss rätten att ändra poängkrav och kriterier på årsbasis. Hurra!
\end{itemize}

Därför yrkar jag på
\begin{attsatser}
\att införa en post som agerar kontaktperson för Sektionsgrodan.
\end{attsatser}
\begin{signatures}{1}
    \mvh
    \signature{Hannes Byden}{Kodhackare 2018}
\end{signatures}

\end{document}
