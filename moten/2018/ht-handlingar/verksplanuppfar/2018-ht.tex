\documentclass[../_main/handlingar.tex]{subfiles}

\begin{document}
\verksplanuppf{HT 2018}

\subsubsection*{Styrelsen}

\subsubsection*{Informationsutskottet}

Informationsutskottet har under året lagt mycket fokus på att utvärdera sektionens informationskanaler. Vi har under året även börjat använda oss av appen Bonsai för att kunna nå ut till medlemmar med push-notiser. 
Hurvida vi kommer använda denna app i fortsättningen måste utredas vidare men jag och resten av styrelsen är positiva eftersom både kåren och andra sektionen finns på samma plattform.

Hemsidan används inte så mycket som jag hade velat men facebook och instagram går fortfarande varmt.
En plan för hur en ny mobilanpassad hemsida ska utvecklas har också lagts fram, tanken är att det kommer ske i samband med en serveruppgradering.

Vice kontaktor posten måste utvärderas vidare, främst behövs lite tydligare arbetsuppgifter tas fram. Jag tycker fortfarande att posten är nödvändig och det är skönt med avlastning samt att kunna ha någon att bolla idéer med.

En långsiktig plan för uppgradering av vår tekniska utrustning har även tagits fram. 

\subsubsection*{Källarmästeriet}
Källarmästeriet har under det gångna året jobbat hårt för att teknologkårens medlemmar ska ha trevliga gillen att gå på i princip varje fredag. Vi har utvecklat en reklamposter som satts upp i övriga hus på LTH för att försöka locka fler teknologer från andra sektioner till gillena. Detta har delvis lyckats. Det kommer folk från andra sektioner hyfsat ofta, men inte jättemånga. 
Km ska vara med och ordna en pubrunda med resten av sexmästarkollegiet för alla teknologer. Detta bidrar till att locka fler till Edekvata samtidigt och är även ett tydligt exempel på samarbete med andra sektioner. Vi har även haft samarbete inom E-sektionen i form av alumnipub, och kick-off för andra utskott.
Priset på mat och dryck är kontinuerligt extremt lågt. Alkoholen säljs billigt men lagligt. Priset på mat bör utvärderas för att se om det behöver höjas. Även om det höjs är 40 kr, eller 15-20 för funktionärer är fortfarande mycket konkurrenskraftigt.

KM har jobbat för att minska spritdiffen genom att noga se till att våra arbetare vet hur man skriver ut och in alkohol ur lagret samt se till att det alltid är låst så att ingen obehörig kommer in. fel på ahs?

\subsubsection*{Nolleutskottet}
Nollningsutskottet har under året planerat och genomfört en nollning för de nyantagna studenterna. Vi har hela tiden jobbat för att det ska finnas en mångfald av olika aktiviteter, så att det ska finnas något som passar alla. Detta tycker vi att vi har lyckats med, då det fanns en stor variation i de olika aktiviteter som hölls under nollningen. 

Vi har jobbat med att integrera de internationella studenterna mer i sektionen, vilket vi tycker att vi lyckats väl med. Några av sakerna som gjorts i år är att vi inom phøset har haft en som varit internationellt ansvarig. Vi gjorde en internationell nollEguide så att även de internationella studenterna skulle kunna få den information som finns däri. Vi skapade två internationella phaddergrupper, med många duktiga phaddrar i båda grupperna. Vi lät också två internationella studenter hålla ett tal på nollEgasquen. Dessa förändringar tycker vi har gjort stor skillnad, då det var ett stort deltagande hos de internationella studenter på många av nollningens aktiviteter.

Under våren lade vi ut ett formulär där sektionens medlemmar fick skicka in förslag på saker de ville förändra med nollningen. Efter nollningen har en utvärdering skickats ut till både nollor och phaddrar för att få reda på hur nollningen upplevdes samt få in förslag på hur nollningen kan förändras och förbättras. 

Vi tycker också att vi fick ut mycket information om sektionens utskott under nollningen. Styrelsen var involverad och visade sig vid många tillfällen. Många utskott höll i olika sorters evenemang och det hölls ett utskottssafari där alla utskott fick berätta om vad de gör på sektionen.

NollU har många gånger sagt till phaddrar hur viktigt det var att de skulle förmedla en positiv inställning till studierna till de nya studenterna. Att det är viktigt att plugga från början och vi har uppmuntrat dem att plugga tillsammans i phaddergrupperna. Under nollningen hölls också fyra pluggkvällar där nollorna fick mat och på plats fanns det pluggphaddrar för att hjälpa till.

Under nollningen har arbetsbelastningen varit hög på samtliga av utskottets funktionärer. Inga direkta utvärderingar har dock gjorts för att utvärdera arbetsbördan under hösten. Detta kommer dock att göras i samband med att testamentet till nästa års phøs uppdateras.

\subsubsection*{Cafémästeriet}
Cafémästeriets översiktliga mål, att fortsätta vara konkurrenskraftigt och bibehålla studentvänliga priser, anser jag har uppnåtts då försäljningen är hög och priserna låga. Svinnet i LED har ibland blivit högt, framförallt innan uppehåll för tentaperioderna och under våren då det knappt var några läsveckor som hade fem dagar. Svinnet i CM är på bättre nivå, mindre läsk har blivit för gammal tack vare tydlig datummärkning och mindre beställningar åt gången. När det gäller chokladbollar i CM så hinner de knappt landa på hyllan innan de är borta. 
Utvärdering av caféets sortiment har skett kontinuerligt. De populäraste sakerna har vi gjort mer av och vi har även testat en del nya tillskott såsom yoghurt och wraps. Några prisjusteringar har gjorts på grund av höjda inköpspriser.
Kvartalsbokslut har inte gjorts under 2018 men en aktiv dialog mellan mig och inköparna gjorde att vi kände oss säkra på att vi sålde enligt plan under våren. Halvårsbokslutet visade att vi låg bra till och efter det har försäljningen ökat ytterligare. 
Även posterna har utvärderats kontinuerligt. Jag anser att Vice Cafémästare har fått en tydligare roll i år och arbetsfördelningen mellan mig som Cafémästare och dem har varit bra. Halvledare visade sig att bli en väldigt tuff post för de som inte hade dioder på sin dag, så vi ändrade om den posten till en lite lägre belöning men också mindre ansvar. Det nya arbetssättet för  Halvledarna kommer att testas och utvärderas under läsperiod 2. Det krävs även att få några läsperioder med halvledare innan det går att säga om arbetsbelastningen för Cafémästare, Vice och Inköps- och lagerchefer har minskats. För tillfället har det varit väldigt hög arbetsbelastning.
\subsubsection*{Förvaltningsutskottet}
Enligt sektionens verksamhetsplan för 2018 nämns det att sektionens medlemmar ska ha tillgång till fräscha uppehållslokaler ämnade både för studier och studiesociala aktiviteter. Det nämns även att ekonomin ofta prioriteras högre än Sektionens lokaler, något som till viss del tyvärr även skett i år. Dock har vice förvaltningschef varit till stor hjälp genom att ansvara för uthyrningen av lokaler och inventarier samt för skötsel av dessa tillsammans med hustomtarna, vilket lett till att underhållet har skötts på ett bra sätt utan att ekonomin behövts prioriterats ned. 

Utöver detta har berörda funktionärer fått fortsatt utbildning samt det stöd de behöver för att underlätta de ekonomiska delarna så mycket som möjligt. 
\subsubsection*{Studierådet}
SRE ska arbeta för att vara ett synligt utskott för sektionens medlemmar. Det har vi försökt uppnå på en rad olika sätt. Vi skriver minnesanteckningar om studierådsmöten och sätter upp på sektionens anslagstavla och vi har satt upp posters i E-huset på likabehandlingsombud, skyddsombud och världsmästare. För att visa upp utskottet för de nyantagna studenterna har vi haft en workshop och arrangerat fyra pluggkvällar. Likabehandlingsombuden och studirådsordförande har även talat med och berättat om sitt arbete för de nya studenterna. 

Studierådet ska försöka ha studentrepresentanter från alla årskurser från både E och BME. Vi har redan fått in två nya årskursansvariga bland årskurs ett. För att ha utskottsmedlemmar som läser sin master försöker vi ha kvar de som tidigare har varit årskursansvariga i utskottet. SRE jobbar ständigt med CEQ-enkäter och vi försöker påminna studenter om att fylla i enkäterna. Eftersom CEQ-tävlingar inte har gett en större ökning i svarsfrekvens har vi valt att inte lägga pengar på det i år. 
\subsubsection*{Sexmästeriet}

\subsubsection*{Nöjesutskottet}
Nöjesutskottet har sen vårterminen fortsatt att jobba med att följa verksamhetsplanen på ett bra sätt genom att regelbundet arbeta med: 

Regelbundna event som utförs varannan vecka som exempelvis spelkvällar. Spelkvällarna i sig har också i sig fortsatt öka aktiviteten vid biljardbordet och darttavlan även efter att vårens biljardturnering tagit slut. Det har varit stor uppslutning på de senaste spelkvällarna på grund av bra marknadsföringen under nollningen vilket är väldigt kul.

Intersektionellt samarbete som till exempel vårt killergame vi ska hålla tillsammans med D-sektionen nu i november. Värt att notera är att det finns ännu större möjligheter att utföra intersektionella samarbeten, dels genom att utnyttja kollegiet bättre och dels genom att ta direkt kontakt med aktivitetssamordnare på andra sektioner. Det hade varit bra att försöka vidga vyerna till sektioner som exempelvis V, W, K eller I som vi inte har särskilt mycket samarbete med för att stärka gemenskapen på campus. 

Att marknadsföra SåS. Sångarstriden är något vi vidare försöker arbeta med och försöker att aktivt marknadsföra för att kunna få till ett värdigt deltagande. Vidare borde det utvärderas om inte denna bör komma igång då SåS just nu har hamnat lite i skymundan av andra händelser. Jag anser att det bör kunna marknadsföras tidigare så att man redan efter tentorna kan börja dra ihop olika grupper och öva. 

Det som varit fortsatt svårt är att äska pengar till större event. Vi försökte att göra detta till Ölresan i år för att subventionera biljetterna men det gick inte riktigt då det var svårt att argumentera för allmännyttan för sektionen och varför inte bara budgeten borde täcka en subventionering av biljetter. Istället borde man kanske äska pengar för ett något som ger allmännytta för sektionen oberoende av deltagande. 

\subsubsection*{Näringslivsutskottet}
Näringslivsutskottet har haft många olika evenemang men främst lunchföreläsningar. Vi har haft både vinstdrivande och icke vinstdrivande event vilket har varit vårt mål genom året. 
Målet att ha mer evenemang med BME-företag har gått sådär då det är väldigt svårt att få kontakt med dessa företag. De var dock en hel del BME-företag med på Lunch med en Ingenjör vilket var väldigt kul och det har även varit många som vill sprida sin marknadsföring på vår FAcebook-sida.
Det har varit största del E-företag som har varit på de evenemang vi har anordnat. ENU försöker dock fortfarande få in mer BME-företag till våra evenemang. Teknikfokus har bland annat stort fokus på just detta.
Vi har skrivit kontrakt med vissa företag som vi har haft kontakt med innan för att ha ett långvarigt samarbete med dessa vilket vi tror har varit väldigt bra. De skrivna kontrakten gör det också lättare att verkligen veta vad det är som gäller för båda parter. 

\newpage
\begin{signatures}{10}
    \mvh
    \signature{Erik Månsson}{Ordförande 2017}
    \signature{Johan Karlberg}{Kontaktor 2017}
    \signature{Sophia Grimmeiss Grahm}{Förvaltningschef 2017}
    \signature{Daniel Bakic}{Cafémästare 2017}
    \signature{Niklas Gustafson}{Øverphøs 2017}
    \signature{Edvard Carlsson}{SRE-ordförande HT 2017}
    \signature{Josefine Sandström}{ENU-ordförande 2017}
    \signature{Linnea Sjödahl}{Sexmästare 2017}
    \signature{Markus Rahne}{Krögare 2017}
    \signature{Albin Nyström Eklund}{Entertainer 2017}
\end{signatures}

\end{document}
