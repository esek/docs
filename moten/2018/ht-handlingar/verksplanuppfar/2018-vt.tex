\documentclass[../_main/handlingar.tex]{subfiles}

\begin{document}
\verksplanuppf{VT 2018}

\subsubsection*{Styrelsen}
Vi jobbar ständigt för att sektionen ska förbättras och dess medlemmar trivas på bästa möjliga sätt. Som hjälp har vi delmål att följa och försöker följa dessa så gått det går, men tänker oftast inte på de specifika målen utan snarare på helheten i “att göra medlemmarnas studietid så bra som möjligt”.

Vi har haft en del diskussioner kring caféets verksamhet i och med de stora förändringarna och jobbar för att kunna driva verksamheten som vanligt utan att CM ska få allt för hög arbetsbelastning. Vi har framförallt tryckt på att stänga caféet när det inte finns nog med hjälp, vilket förhoppningsvis öppnar upp ögonen för folk.

Intresset för att engagera sig i Teknikfokus går i vågor. Detta året var det mindre engagemang från E-sektionen än förra året och vi har tyvärr behövt jaga funktionärer. Dock verkar det vara en långsiktig ökning i engagemang och vi bör fortsätta jobba åt det hållet. 

Vi försöker sprida information om att man som gemene medlem har friheten att genomdriva egna projekt genom att bland annat göra inlägg i E-sektionens facebook grupp. Detta är något vi vill trycka mer på.

Vi ser ständigt över sektionens funktionärsposter och jobbar för att inga poster ska vara för påfrestande samt att inga poster ska vara kaffeposter. Vi har även sett över strukturen i de olika utskotten och kan på så sätt se vilka poster som behövs/inte behövs och huruvida arbetsbelastningen ska delegeras.

Sektionernas kostnader och intäkter ses ständigt över av Förvaltningsutskottet och även inom de andra utskotten. Många har gjort en internbudget för sitt utskott och ser till att denna följs.

Alla i styrelsen deltar aktivt i sina kollegien och sprider på sina möten information om relevanta event och liknande som är till för fler än bara sektionens medlemmar. Dessutom har vi infört en ny informationspunkt till våra möten där Kontaktorn rapporterar vad som händer i vår omvärld så vi kan ta del av andras event också.

Panten har skötts lite dåligt och vi har fått klagomål om att Pantamera hittat skräp bland vår pant. Nu har det satts upp lappar på varje kärl som förtydligar att skräp ej är tillåtet att slängas i dessa. Förhoppningsvis kommer folk följa detta och även panta mer.

\subsubsection*{Informationsutskottet}
Informationsutskottet jobbar kontinuerligt mot att nå sina delmål. DDG med macapären i spetsen och teknokraterna jobbar ständigt med att förbättra och förnya allt tekniskt på sektionen. Det största och kanske viktigaste projektet vi har framför oss är att uppgradera vår server. Detta har legat på agendan hos InfU under en lång tid men måste verkligen genomföras snarast. 

Informationsspridningen tycker jag fungerar mycket bra, jag ser gärna att vi fortsätter att använda facebook då vi genom denna plattform får extremt stor och riktad spridning men jag tycker även att vi ska lägga upp nyheter på hemsidan. Instagram har också använts flitigt under terminen. 

Avslutningsvis vill jag tillägga att även om jag tycker att InfU fungerar bra just nu så känner jag att det finns mycket som kan bli bättre. Bland annat är det svårt att få ihop utskottet eftersom alla jobbar med så olika saker, fotograferna kontra kodhackarna exempelvis. Jag tror dock att detta är någonting som viceposten hade kunnat hjälpa till med. Denne kan understödja utvärderingen av det interna medan kontaktorn lägger mer fokus på styrelsearbetet och den externa kontakten. 

\subsubsection*{Källarmästeriet}
Källarmästeriet har hittills under året hållit i 6 gillen som alla har gått bra. Vi i krögartrion har tillsammans med picasso tagit fram en reklamposter som ska sättas upp i de andra husen här på LTH. På så vis hoppas vi locka fler teknologer från andra sektioner till våra fantastiska gillen.
Vi har inte haft något samarbete planerat med andra sektioner än men det kan ske i framtiden! Förhoppningsvis blir det en pubrunda i höst tillsammans med de andra sektionernas sexmästare. Det blev ingen nu i vår eftersom inte tillräckligt många ville vara med. 

Under våren har vi samarbetat med Nöju genom biljardturneringen vars matcher spelas under gillen och vi hoppas på fler framtida samarbeten inom sektionen.
 
Vi har än så länge hållit samma priser som förra året vilket jag tycker är väldigt billigt, särskilt om man är funktionär.
 
Vi gör vårt bästa för att ha så lite svinn som möjligt i alkohollagret genom att visa jobbarna tydligt hur man gör med in- och utskrivning av alkohol i systemet.


Allt som allt tycker jag att vi följer verksamhetsplanen på ett tillfredsställande sätt hittills och det ska vi fortsätta med.

\newpage
\subsubsection*{Nolleutskottet}
Nollningsutskottet har under våren arbetat med att planera inför nollningen. Än så länge har ett preliminärt schema inför nollningen satts ihop. Vi har till stor del följt förra årets nollningsschema för att bevara en stor variation i aktiviteter, så att det ska kunna finnas något som passar alla. I det preliminära schemat finns evenemang som är till för att sektionens utskott ska få visa upp sig själva och sin verksamhet. Vi ska försöka få in ännu fler sådana evenemang i det slutgiltiga schemat. Under nollningen kommer SRE att hålla studiekvällar för att uppmuntra nollorna till att studera redan från början. Till dessa pluggkvällar har det rekryterats pluggphaddrar som ska finnas till hands och hjälpa nollorna. 

Vi har i år ett Co-phøs som är internationellt ansvarig, som har som huvudansvar att jobba med att få den internationella nollningen så bra som möjligt. Det har också varit ett högt söktryck till internationell grupphadder i år, vilket har lett till att vi beslutat att göra två internationella phaddergrupper. Vi hoppas att detta ska leda till att de internationella studenterna kan integreras ännu mer i sektionen samtidigt som vi ska fortsätta jobba vidare på detta.

Øverphøset har under våren hållit individuella samtal med alla funktionärer i utskottet. Detta för att diskutera hur alla mår samt utvärdera arbetsbördan på varje enskild individ. Dessa samtal hölls också för att utvärdera NollUs arbete och eventuella förbättringar vi kan göra i detta. Likt förra året är vi sex personer i phøset, vilket innebär att vi kan fördela ut arbetet på fler personer och därmed försöka få en mer jämn arbetsfördelning. 

För att få in förslag på hur nollningen kan förändras och förbättras har ett Google formulär delats via facebook. I detta kan alla medlemmar inom sektionen som vill fylla i sina förslag kring nollningen.

\subsubsection*{Cafémästeriet}
Cafémästeriet har fortsatt arbeta för att minska svinnet i LED och förrådet av chokladbollar och läsk i CM. Genom att gå ifrån att dioderna själva improviserar vilka pålägg det ska vara på ciabattorna till att ha en mer kontinuerlig meny, har vi dragit ned på antalet artiklar i kylen och har på så sätt bättre koll. Läskbeställningarna har gjorts lite oftare och då med mindre order vilket har minskat både svinn och risken att vissa sorter tar slut. Även för chokladbollar har svinnet minskat då vi nu beställer dem från Martin\&Servera tillsammans med övriga varor. 

Caféts sortiment har inte hunnit utvärderas så mycket mer än att vi infört stor kaffe och sänkt priset på kaffe och frukostmackorna. Postbeskrivningar och utvärdering av posterna har inte heller kunnat göras ännu då hela organisationen är under utveckling och alla håller på att komma in i sina roller. Även planerna på att utöka caféts öppettider ligger på is eftersom att vi för tillfället knappt har tillräckligt med funktionärer för att hålla öppet till klockan 15. 

\subsubsection*{Förvaltningsutskottet}
Början av året har bestått mycket av att se till att alla funktionärer som berörs av sektionens ekonomi kan sköta detta på ett så smidigt och smärtfritt sätt som möjligt.  
En ekonomiutbildning hölls i januari där de berörda funktionärerna blev informerade om vad som gäller rörande sektionens ekonomi.
Utöver detta har förvaltningsutskottet hittills börjat arbeta lite smått för att hålla våra lokaler i gott skick med hjälp av hustomtarna och vice förvaltningschefen i spetsen. De har haft ett uppstartsmöte och satt upp en plan för vad som vill åstadkommas under verksamhetsåret 2018 för att se till att sektionens medlemmar har tillgång till så fräscha uppehållslokaler som möjligt, ämnade både för studier och studiesociala aktiviteter.

\subsubsection*{Studierådet}
Studierådet ska verka för att vara ett synligt utskott, detta försöker vi arbeta mot genom att trycka upp affischer med de poster vi tycker att det är extra viktigt att sektionens medlemmar vet om. Vi vill vara aktiva under nollningen för att nya studenter ska veta om att utskottet finns och detta mål ska vi fortsätta arbeta mot. Vi ska även verka med att ha kontinuerliga pluggkvällar men eftersom sektionen har två projektfunktionärer som driver ett pluggkvällsprojekt under våren har SRE inte haft några egna pluggkvällar än. Vi ska tillsammans med projektfunktionärerna utvärdera pluggkvällarna. Utskottet vill ha representanter från alla årskurser, i nuläget har vi representanter från 1-3 från båda programmen men inte alla från de högre årskurserna.  

Studierådet ska verka för att vara ett synligt utskott. Arbetet skall göras mer tillgängligt för sektionens medlemmar för att visa förändringar som har genomförts. Utskottet ska verka för att medlemmarna är medvetna om deras möjligheter att påverka och förbättra sin utbildning.
Delmål:
\begin{dashlist}
    \item Arbeta för att synliggöra utskottets arbete och resultat till sektionens medlemmar. 
    \item Anordna pluggkvällar kontinuerligt under hela året samt utvärdera deras struktur och syfte.
    \item Arbeta för att ha minst en representant från varje årskurs, inklusive årskurs fyra och fem.
    \item Arbeta för att öka svarsfrekvensen på CEQ-enkäterna.
\end{dashlist}

\subsubsection*{Sexmästeriet}
Sexmästeriet har under våren  försökt arangera evenmang som följer sektionens verksamhetsplan. Vi har organiserat Skiphtet och Teknikfokus som är två sittningar som E-sektionens Sexmästeri förväntas hålla i. Dessutom har Sexmästeriet anordnat två evenemang för Athena. Detta har gynnat E-sektionens medlemmar eftersom många har kunnat gå på prisvärda sittningar med hög kvalité. De sittningar som många av E-sektionens medlemmar varit exkluderade ifrån har inom sexmästeriet bidragit till en ökad prestanda inför kommande sittningar. 

Dessvärre har Sexmästeriet hittills misslyckats med att hålla i en sittning som inkluderar alla E-sektionens medlemmar. Men under LP 4 kommer vi hålla i flera evenemag för alla sektionens medlemmar samt fortsätta arrangera sittningar som Sexmästeriet förväntas anordna.

Under början av året rensade Hovmästarna Pump och har därefter bibehållit en god ordning i förrådet. Vice Sexmästare har med hjälp av Köksmästarna bibehållit en god ordning i CM där Sexmästeriet förvarar kylvaror, i kylskåpen och frysen då.

\subsubsection*{Nöjesutskottet}
Nöjesutskottet har under den tid som genomfört ett par saker som vi tror har bidragit till att följa verksamhetsplanen på ett bra sätt.
\begin{dashlist}
    \item Dels har vi ökat mängden regelbundna event och ansvaret har fördelats till viss del inom utskottet. Själva fördelningen är något som hade kunnat göras bättre.
    \item Vi har startat en biljardturnering vilket har bidragit till en ökad aktivitet kring biljardbordet. Själva aktiviteten i Biljard i sig har också ökat i och med uppspikandet av darttavlan. Den ökade aktiviteten har dock ökat slitaget på bordet och pilarna så vi förväntar oss en ökad kostnad för uppehåll av biljard och dart.
    \item De intersektionella evenemangen har uteblivit hittills då det är mycket som kommit i vägen i år. Däremot är de på gång snart och E-sektionen har varit drivande för Paintballen och vi är faktiskt ansvariga för den själva vilket vi är väldigt stolta över. Sen har vi också varit engagerade i Valborgsfestivalen som kommer ske den 25/4. 
\end{dashlist}

Det vi inte har varit bra på är att äska pengar från styrelsen för större projekt, men det är kanske något vi ska ha i åtanke inför Ölresan vi planerar att ha till hösten så sektionens medlemmar kan få dricka riktigt god öl i valfria mängder. 

Sångarstriden har vi inte riktigt tänkt på i nuläget då det för tillfället är ganska långt bort i tidslinjen. 

\subsubsection*{Näringslivsutskottet}
Utskottet har arrangerat olika evenemang, både vinstdrivande och icke vinstdrivande. De flesta företagskontakterna från förra året har kontaktats. Det är svårt att få tag på BME-företag, dock har ENU en lista med olika företag från BME-alumner och SVL som har kontaktats. Förhoppningsvis kommer några av dessa vara intresserade av event med sektionen. Det nya eventet “en FED pub” kommer förmodligen öka chanserna för att nya företag att få upp ögonen för sektionen.
Tillsammans med Näringslivskollegiet har prissättningen setts över för att se till att alla sektioner har någorlunda liknande priser.

\newpage
\begin{signatures}{10}
    \mvh
    \signature{Daniel Bakic}{Ordförande 2018}
    \signature{Axel Voss}{Kontaktor 2018}
    \signature{Magnus Lundh}{Förvaltningschef 2018}
    \signature{Elin Johansson}{Cafémästare 2018}
    \signature{Andreas Bennström}{Øverphøs 2018}
    \signature{Fanny Månefjord}{SRE-ordförande 2018}
    \signature{Isabella Hansen}{Ordförande Näringslivsutskottet 2018}
    \signature{Alexander Wik}{Sexmästare 2018}
    \signature{Malin Heyden}{Krögare 2018}
    \signature{Adam Belfrage}{Entertainer 2018}
\end{signatures}

\end{document}
