\documentclass[10pt]{article}
    \usepackage[utf8]{inputenc}
    \usepackage[swedish]{babel}
    
    \def\doctype{Kallelse} %ex. Kallelse, Handlingar, Protkoll
    \def\mname{styrelsemöte} %ex. styrelsemöte, Vårterminsmöte
    \def\mnum{S24/18} %ex S02/16, E1/15, VT/13
    \def\date{2018-11-16} %YYYY-MM-DD
    \def\docauthor{Daniel Bakic}
    
    \def\mtime{12:10}
    \def\place{E:1124}
    
    \usepackage{../e-mote}
    \usepackage{../../../e-sek}
    
    \begin{document}
    
    \heading{{\doctype} till {\mname} {\mnum}}
    
    \section*{Tid och plats}
    \tidplats
    
    \section*{Föredragningslista}
    \begin{paralist}
        \pli{OFMÖ}{\bes}
        \pli{Val av mötesordförande}{\bes}
        \pli{Val av mötessekreterare}{\bes}
        \pli{Val av justeringsperson}{\bes}
        \pli{Godkännande av tid och sätt}{\bes}
        \pli{Adjungeringar}{\bes}
        \pli{Godkännande av dagordningen}{\bes}
        \pli{Föregående mötesprotokoll}{\bes}
        \pli{Fyllnadsval och entledigande av funktionärer}{\bes}
        \pli{Rapporter}{}
        \begin{paralist}
            \pli{Hur mår alla?}{\info}
            \pli{Utskottsrapporter}{\info}
            \pli{Ekonomisk rapport}{\info}
            \pli{Kåren informerar}{\info}
            \pli{Omvärldsrapport}{\info}
        \end{paralist}
    
        \pli{Äska pengar till Musikhjälpen}{\bes}
        \pli{Nästa styrelsemöte}{\bes}
        \pli{Beslutsuppföljning}{\bes}
        \pli{Övrigt}{\dis}
        \pli{OFMA}{\bes}
      \end{paralist}
    
    \newpage
    
    \section*{Beslutsuppföljning}
   % \emph{Beslutsuppföljningen är tom!}
    \begin{beslutsuppf}
    \beslut{S20/18}{Inköp styrelsemärke}{}{Daniel Bakic}{S24/18}
    \beslut{S23/18}{Bokning funktionärstack}{}{Daniel Bakic, Malin Heyden, Axel Voss}{S24/18}
    \beslut{S22/18}{Undersökning av Bonsai Campus betalsystem}{}{Magnus Lundh}{S25/18}
    \end{beslutsuppf}
    
    \begin{signatures}{1}
        \emph{I styrelsens tjänst}
        \signature{Daniel Bakic}{Ordförande}
    \end{signatures}    
    
    \end{document}