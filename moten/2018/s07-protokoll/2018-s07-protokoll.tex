\documentclass[10pt]{article}
\usepackage[utf8]{inputenc}
\usepackage[swedish]{babel}

\def\mo{Daniel Bakic}
\def\ms{Axel Voss}
\def\ji{Fanny Månefjord}
%\def\jii{}

\def\doctype{Protokoll} %ex. Kallelse, Handlingar, Protkoll
\def\mname{styrelsemöte} %ex. styrelsemöte, Vårterminsmöte
\def\mnum{S07/18} %ex S02/16, E1/15, VT/13
\def\date{2018-03-21} %YYYY-MM-DD
\def\docauthor{\ms}

\usepackage{../e-mote}
\usepackage{../../../e-sek}

\begin{document}
\showsignfoot

\heading{{\doctype} för {\mname} {\mnum}}

%\naun{}{} %närvarane under
%\nati{} %närvarande till och med
%\nafr{} %närvarande från och med
\section*{Närvarande}
\subsection*{Styrelsen}
\begin{narvarolista}
	\nv{Ordförande}{Daniel Bakic}{E15}{}
	\nv{Kontaktor}{Axel Voss}{E15}{}
	\nv{Förvaltningschef}{Magnus Lundh}{E15}{}
	\nv{Cafémästare}{Elin Johansson}{BME16}{}
	\nv{Øverphøs}{Andreas Bennström}{BME16}{}
	\nv{SRE-ordförande}{Fanny Månefjord}{BME16}{}
	%\nv{ENU-ordförande}{Isabella Hansen}{E16}{}
	\nv{Sexmästare}{Alexander Wik}{BME17}{\nafr{10B}}
	\nv{Krögare}{Malin Heyden}{E16}{}
	\nv{Entertainer}{Adam Belfrage}{BME17}{}
\end{narvarolista}
\subsection*{Ständigt adjungerande}


\begin{narvarolista}
	%\nv{Inköps- och lagerchef}{Sofie Johannesson}{E17}{}
	%\nv{Inköps- och lagerchef}{Fabian Sondh}{E17}{}
	%\nv{Inköps- och lagerchef}{Albin Pålsson}{E17}{}
	\nv{Kårrepresentant}{Agnes Sörliden}{}{}
	%\nv{Valberedningens ordförande}{Pontus Landgren}{}{}
	\nv{Nollegeneral}{Jakob Nilsson}{}{}
	%\nv{Talman}{Erik Månsson}{E14}{}
	%\nv{Inspektor}{Monica Almqvist}{}{}
	%\nv{Sigillbevarare}{Henrik Ramström}{}{}
	%\nv{Vice Entertainer}{Emil Bergström}{}{}
	\nv{Likabehandlingsombud}{Tina Tabandeh}{}{}
	
\end{narvarolista}

\begin{comment}
\subsection*{Adjungerande}
\begin{narvarolista}
	%\nv{Post}{Namn}{Klass}{}
\end{narvarolista}
\end{comment}

\section*{Protokoll}
\begin{paragrafer}
	\p{1}{OFMÖ}{\bes}
	Ordförande {\mo} förklarade mötet öppnat 12:15.
	
	\p{2}{Val av mötesordförande}{\bes}
	{\valavmo}
	
	\p{3}{Val av mötessekreterare}{\bes}
	{\valavms}
	
	\p{4}{Val av justeringsperson}{\bes}
	{\valavj}
	
	\p{5}{Godkännande av tid och sätt}{\bes}
	{\tosg}
	
	\p{6}{Adjungeringar}{\bes}
	%{\ingaadj}
	%Förnamn Efternamn adjungerades
	
	
	\p{7}{Godkännande av dagordningen}{\bes}
	Dagordningen godkändes.
	
	
	%Föredragningslistan godkändes med yrkandet.
	%Föredragningslistan godkändes med samtliga yrkanden.
	
	\p{8}{Föregående mötesprotokoll}{\bes}
	%\latillprot{S03/18}
	\ingaprot
	
	\p{9}{Fyllnadsval och entledigande av funktionärer}{\bes}
	\begin{fyllnadsval} %"Inga fyllnadsval." fylls i automatiskt
		%\fval{Henrik Stålbom}{Näringslivskontakt}
		
		\fval{Paulina Sager}{Diod}
		\fval{Anton Jigsved}{Diod}
		\fval{Hjalmar Tingberg}{Diod}
		\fval{Hannes Björk}{Diod}
		\fval{Jonathan Benitez}{Diod}
		\fval{Mohammad Bijani}{Diod}
		\fval{Felix Långberg}{Diod}
		\fval{Viktor Drakfeldt}{Diod}
		\fval{Jessica Kågeman}{Diod}
		\fval{Fanny Månefjord}{Diod}
		\fval{Daniel Bakic}{Diod}
		\fval{Srinivasan Muthukrishnan}{Diod}
		\fval{Andreas Gustafsson}{Diod}
		\fval{Ibrahim Sakini}{Diod}
		\fval{Y Nhi Pham}{Diod}
		\fval{Tom Andersen}{Diod}
		\fval{Isa Clementsson}{Halvledare}
		\fval{Gabriela Medina}{Halvledare}
		\fval{Lina Samnegård}{Halvledare}
		
	\end{fyllnadsval}
	
	\p{10}{Rapporter}{}
	\begin{paragrafer}
		\subp{A}{Hur mår alla?}{\info}
		Punkten protokollfördes ej.
		
		\subp{B}{Utskottsrapporter}{\info}
		Malin hade gille i lördags, det va inte så mycket folk eftersom det var karnevelj samtidigt.
		
		Fanny har varit på SRx-möte. Det har varit massor med CEQ-möten och pluggphaddrar har valts inför nollningen.
		
		Nöju har haft en lugn vecka. Mycket fokus har legat på DÖMD. Dreamhacke blir av till helgen.
		
		E6 har haft lite interna fester. De har börjat planera inför temasläpp och en egen sittning.
		
		Andreas och resten av phøoset har haft ett break under tentaveckorna. I princip alla phaddrar har blivit valda, de letar efter ett fåtal till uppdragsphaddrar. De har rekryterat väldigt många internationella phaddrar i år, så de har tänkt ha två internationella phaddergrupper.
		
		Elin har svårt att hitta folk som kan jobba på eftermiddagen. En ny och billigare leverantör för delikatobollar har hittats! Caféet har även köpt en dammsugare och en mixer.
		
		InfU har configat nätverk och switchar inför DreamhackE.
		
		Magnus har bokfört, han har även sökt alkoholtillstånd till påskgille.
		
		Daniel har tänkt mycket på vårterminsmöte. Han tycker allt känns bra.
		
		\subp{C}{Ekonomisk rapport}{\info}
		Ekonomin ser jättebra ut!
		
		\subp{D}{Kåren informerar}{\info}
		
		Kåren hade fullmäktigemöte i veckan. De klubbad igenom en del policys.
		På söndag har de nästa fullmäktigemöte, det kommer vara ett valmöte..
		
		De kommer även hålla en styrelseutbildning, nu på måndag.
		
	\end{paragrafer}
	
	\p{11}{Välmåendevecka}{\info}
	
	Likabehandlingsombudsutskotten på sektionerna kommer hålla en välmåendevecka. De behöver 3000 kr från varje sektion.
	
	Mötet var enade om att detta gärna är någonting sektionen bidrar till. 
	\p{12}{Nästa styrelsemöte}{\bes}
	\Mba nästa styrelsemöte ska äga rum på onsdag den 2018-03-28 klockan 12:10 i E:1124.
	
	\p{14}{Beslutsuppföljning}{\bes}
	
	\Ibfu
	
	
	\p{15}{Övrigt}{\dis}
	
	Mötet diskuterade remiss för GDPR-policy. Alla styrelsemedlemmar ska läsa igenom remissen inför nästa lunchmöte.
	
	K och I har gasque samma dag. Eftersom det är dyrare att ha gasque på bryggan lades förslaget att allas övriga sektioners biljetter kostar 10kr mer för att betala mellanskillnaden fram. Mötet såg positivt på detta.
	  
	Adam vill gärna köpa in en GoPro och kollar upp detta till nästa möte.
	
	Elin vill att alla ska läsa propositioner inför kvällsmötet. Så blir det tidseffektivare på själva kvällsmötet.
	
	\p{16}{OFMA}{\bes}
	{\mo} förklarade mötet avslutat 12:47.
\end{paragrafer}

%\newpage
\hidesignfoot
\begin{signatures}{3}
	\signature{\mo}{Mötesordförande}
	\signature{\ms}{Mötessekreterare}
	\signature{\ji}{Justerare}
\end{signatures}
\end{document}
