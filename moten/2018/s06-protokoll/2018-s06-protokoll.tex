\documentclass[10pt]{article}
\usepackage[utf8]{inputenc}
\usepackage[swedish]{babel}

\def\mo{Daniel Bakic}
\def\ms{Axel Voss}
\def\ji{Isabella Hansen}
%\def\jii{}

\def\doctype{Protokoll} %ex. Kallelse, Handlingar, Protkoll
\def\mname{styrelsemöte} %ex. styrelsemöte, Vårterminsmöte
\def\mnum{S06/18} %ex S02/16, E1/15, VT/13
\def\date{2018-03-01} %YYYY-MM-DD
\def\docauthor{\ms}

\usepackage{../e-mote}
\usepackage{../../../e-sek}

\begin{document}
\showsignfoot

\heading{{\doctype} för {\mname} {\mnum}}

%\naun{}{} %närvarane under
%\nati{} %närvarande till och med
%\nafr{} %närvarande från och med
\section*{Närvarande}
\subsection*{Styrelsen}
\begin{narvarolista}
\nv{Ordförande}{Daniel Bakic}{E15}{}
\nv{Kontaktor}{Axel Voss}{E15}{}
\nv{Förvaltningschef}{Magnus Lundh}{E15}{}
\nv{Cafémästare}{Elin Johansson}{BME16}{}
\nv{Øverphøs}{Andreas Bennström}{BME16}{}
\nv{SRE-ordförande}{Fanny Månefjord}{BME16}{}
\nv{ENU-ordförande}{Isabella Hansen}{E16}{}
\nv{Sexmästare}{Alexander Wik}{BME17}{}
\nv{Krögare}{Malin Heyden}{E16}{}
\nv{Entertainer}{Adam Belfrage}{BME17}{}
\end{narvarolista}
\subsection*{Ständigt adjungerande}


\begin{narvarolista}
%\nv{Inköps- och lagerchef}{Sofie Johannesson}{E17}{}
%\nv{Inköps- och lagerchef}{Fabian Sondh}{E17}{}
%\nv{Inköps- och lagerchef}{Albin Pålsson}{E17}{}
%\nv{Kårordförande}{Linus Hammarlund}{}{}
%\nv{Kårrepresentant}{Jacob Karlsson}{}{\nafr{3}}
\nv{Kårrepresentant}{Agnes Sörliden}{}{}
\nv{Valberedningens ordförande}{Pontus Landgren}{}{}
%\nv{Skattmästare}{Olle Oswald}{}{}
%\nv{Kårrepresentant}{Daniel Damberg}{}{}
%\nv{Kårrepresentant}{John Alvén}{}{}
%\nv{Nollegeneral}{Jakob Nilsson}{}{}
%\nv{Talman}{Erik Månsson}{E14}{}
%\nv{Elektras Ordförande}{Elisabeth Pongratz}{}{}
%\nv{Inspektor}{Monica Almqvist}{}{}
\nv{Sigillbevarare}{Henrik Ramström}{}{}
%\nv{Vice Entertainer}{Emil Bergström}{}{}
\end{narvarolista}

\begin{comment}
\subsection*{Adjungerande}
\begin{narvarolista}
%\nv{Post}{Namn}{Klass}{}
\end{narvarolista}
\end{comment}

\section*{Protokoll}
\begin{paragrafer}
\p{1}{OFMÖ}{\bes}
Ordförande {\mo} förklarade mötet öppnat 12:13.

\p{2}{Val av mötesordförande}{\bes}
{\valavmo}

\p{3}{Val av mötessekreterare}{\bes}
{\valavms}

\p{4}{Val av justeringsperson}{\bes}
{\valavj}

\p{5}{Godkännande av tid och sätt}{\bes}
{\tosg}

\p{6}{Adjungeringar}{\bes}
%{\ingaadj}
%Förnamn Efternamn adjungerades


\p{7}{Godkännande av dagordningen}{\bes}
Dagordningen godkändes.


%Föredragningslistan godkändes med yrkandet.
%Föredragningslistan godkändes med samtliga yrkanden.

\p{8}{Föregående mötesprotokoll}{\bes}
%\latillprot{S03/18}
\ingaprot

\p{9}{Fyllnadsval och entledigande av funktionärer}{\bes}
\begin{fyllnadsval} %"Inga fyllnadsval." fylls i automatiskt
\fval{Henrik Stålbom}{Näringslivskontakt}
\end{fyllnadsval}

\p{10}{Rapporter}{}
\begin{paragrafer}
\subp{A}{Hur mår alla?}{\info}
Punkten protokollfördes ej.

\subp{B}{Utskottsrapporter}{\info}
Led rullar på som vanligt. De stänger imorgon inför tentavecka och har trappat ner på lagret. Inköparna fixar läskbeställning.

I SRE fortsätter årskursansvariga med CEQ-möten.

Enu har haft en lyckad Cv-fotografering. Lunchföreläsning fixad, de kommer eventuellt hålla en till med Sveriges ingenjörer.

Phøset hälsar att alla phadderintervjuer nu är klara. Grupper ska börja väljas.

E6 har börjat utvärdera sina sittningar. De har skickat ut enkäter till alla som jobbar.

Nöju har haft en lugn vecka. Bastukvällen gick bra men det va ganska dålig uppslutning.

FVU har inte gjort så mycket, Magnus har betalat räkningar.

Infu har hackat kod och gjort affischer.

KM har gille imorgon de hade kickoff i helgen och det var lyckat.

Daniel har fokuserat mycket på skolan i veckan och har därför inte hunnit göra så mycket mer än det vanliga.

\subp{C}{Ekonomisk rapport}{\info}
  Ekonomin ser jättebra ut!

\subp{D}{Kåren informerar}{\info}

  My kommer hålla en styrelseutbildning i läsvecka 1, LP2. De håller även i tentafrukost nästa vecka. Det går också att nominera till heltidarposter nu, valet är den 25e mars.
  Lycka till med tentorna hälsar kåren!

\end{paragrafer}

\p{11}{Kravprofiler}{\dis}
  Under förra året infördes kravprofiler. Det ska finnas tydligare postbeskrivningar, det vill säga vilken personlighetstyp som valberedningen eftersöker.

  Inspektor och teknikfokusansvarig är poster som valbereds inför Vt-mötet och därför ska det inför mötet finnas utförligare beskrivningar till dessa poster.

  Mötet diskuterade frågan och beslutade att det är valberedningens uppgift att lägga fram ett förslag och föra diskussion med respektive styrelsepost.

  \p{12}{SVS}{\dis}

  Mötet diskuterade huruvida sektionen tar ställnig till de olika scenarion som lyfts fram i frågan om Lunds universitets etablering på Science Village Scandinavia.

  Dialog ska försöka hållas med äldre studenter. Studievägledningen kan användas för att nå ut till de studenter som redan är klara.

  \p{13}{Vårterminsmöte}{\dis}

  I driven finns information om vad styrelsen behöver göra innan vårterminsmötet. Styrelsen kommer behöva hålla kvällsmöten vilka kommer diskuteras vidare utanför mötet.
  En tidsplan behöver läggas fram, just nu siktar styrelsen på att påbörja förberedelserna efter tentorna.

  Styrelsen bör fundera på motioner och propositioner och lägga in dessa i drive.

\p{13}{Nästa styrelsemöte}{\bes}
\Mba nästa styrelsemöte ska äga rum 2018-03-22 klockan 12:10 i E:1124.

\p{14}{Beslutsuppföljning}{\bes}

  Alexander \ypa stryka \emph{Kundvagnar} från beslutsuppföljningen.
\Mbaby
%\Ibfu


\p{15}{Övrigt}{\dis}

  Daniel informerar om städning och fester i huset. Städningen måste vara noggrannare!

  Adam informerar om en campuskalender.

  Henrik slängde ut frågan om tygmärken till de funktionärer som engagerar sig. Det kan vara en extra morot samt ett sätt att få folk att prata om de olika posterna. Tanken är då att man har olika märken för olika utskott och poster.

\p{16}{OFMA}{\bes}
{\mo} förklarade mötet avslutat 13:06.
\end{paragrafer}

%\newpage
\hidesignfoot
\begin{signatures}{3}
\signature{\mo}{Mötesordförande}
\signature{\ms}{Mötessekreterare}
\signature{\ji}{Justerare}
\end{signatures}
\end{document}
