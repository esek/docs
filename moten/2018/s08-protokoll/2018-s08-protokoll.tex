\documentclass[10pt]{article}
    \usepackage[utf8]{inputenc}
    \usepackage[swedish]{babel}
    
    \def\mo{Daniel Bakic}
    \def\ms{Axel Voss}
    \def\ji{Alexander Wik}
    %\def\jii{}
    
    \def\doctype{Protokoll} %ex. Kallelse, Handlingar, Protkoll
    \def\mname{styrelsemöte} %ex. styrelsemöte, Vårterminsmöte
    \def\mnum{S08/18} %ex S02/16, E1/15, VT/13
    \def\date{2018-03-28} %YYYY-MM-DD
    \def\docauthor{\ms}
    
    \usepackage{../e-mote}
    \usepackage{../../../e-sek}
    
\begin{document}
\showsignfoot

\heading{{\doctype} för {\mname} {\mnum}}

%\naun{}{} %närvarane under
%\nati{} %närvarande till och med
%\nafr{} %närvarande från och med
\section*{Närvarande}
\subsection*{Styrelsen}
\begin{narvarolista}
	\nv{Ordförande}{Daniel Bakic}{E15}{}
	\nv{Kontaktor}{Axel Voss}{E15}{}
	%\nv{Förvaltningschef}{Magnus Lundh}{E15}{}
	\nv{Cafémästare}{Elin Johansson}{BME16}{}
	\nv{Øverphøs}{Andreas Bennström}{BME16}{}
	\nv{SRE-ordförande}{Fanny Månefjord}{BME16}{}
	\nv{ENU-ordförande}{Isabella Hansen}{E16}{}
	\nv{Sexmästare}{Alexander Wik}{BME17}{}
	\nv{Krögare}{Malin Heyden}{E16}{\nafr{7}}
	\nv{Entertainer}{Adam Belfrage}{BME17}{}
\end{narvarolista}
\subsection*{Ständigt adjungerande}


\begin{narvarolista}
	%\nv{Inköps- och lagerchef}{Sofie Johannesson}{E17}{}
	%\nv{Inköps- och lagerchef}{Fabian Sondh}{E17}{}
	%\nv{Inköps- och lagerchef}{Albin Pålsson}{E17}{}
	%\nv{Kårordförande}{Linus Hammarlund}{}{}
	%\nv{Kårrepresentant}{Jacob Karlsson}{}{\nafr{3}}
	\nv{Kårrepresentant}{Agnes Sörliden}{}{}
	%\nv{Valberedningens ordförande}{Pontus Landgren}{}{}
	%\nv{Skattmästare}{Olle Oswald}{}{}
	%\nv{Kårrepresentant}{Daniel Damberg}{}{}
	%\nv{Kårrepresentant}{John Alvén}{}{}
	%\nv{Nollegeneral}{Jakob Nilsson}{}{}
	%\nv{Talman}{Erik Månsson}{E14}{}
	%\nv{Elektras Ordförande}{Elisabeth Pongratz}{}{}
	%\nv{Inspektor}{Monica Almqvist}{}{}
	\nv{Sigillbevarare}{Henrik Ramström}{E16}{}
	%\nv{Vice Entertainer}{Emil Bergström}{}{}
\end{narvarolista}

\begin{comment}
\subsection*{Adjungerande}
\begin{narvarolista}
	%\nv{Post}{Namn}{Klass}{}
\end{narvarolista}
\end{comment}

\section*{Protokoll}
\begin{paragrafer}
	\p{1}{OFMÖ}{\bes}
	Ordförande {\mo} förklarade mötet öppnat 12:10.

	\p{2}{Val av mötesordförande}{\bes}
	{\valavmo}

	\p{3}{Val av mötessekreterare}{\bes}
	{\valavms}

	\p{4}{Val av justeringsperson}{\bes}
	{\valavj}

	\p{5}{Godkännande av tid och sätt}{\bes}
	{\tosg}

	\p{6}{Adjungeringar}{\bes}
	{\ingaadj}
	%Förnamn Efternamn adjungerades


	\p{7}{Godkännande av dagordningen}{\bes}
	Malin \ypa lägga till punkten \S14 kaffekokare.

	Föredragningslistan godkändes med yrkandet.
	%Föredragningslistan godkändes med samtliga yrkanden.

	\p{8}{Föregående mötesprotokoll}{\bes}
	%\latillprot{S03/18}
	\ingaprot

	\p{9}{Fyllnadsval och entledigande av funktionärer}{\bes}
	\begin{fyllnadsval} %"Inga fyllnadsval." fylls i automatiskt
		\fval{Axel Sneitz-Björkman}{Umph-Meister}
	\end{fyllnadsval}

	\p{10}{Rapporter}{}
	\begin{paragrafer}
		\subp{A}{Hur mår alla?}{\info}
		Punkten protokollfördes ej.

		\subp{B}{Utskottsrapporter}{\info}

		Phøset hade phadderutbildning i lördags. Andreas har haft individuella möten med alla i NollU.

		NöjU har haft DreamHackE i helgen, det var mycket lyckat. De håller på att planera event till nollningen och andra mer närliggande evenemang som paintball och valborgsfestivalen.

		Alex har planerat temasläpp och varit på kollegiemöte. Han kommer även ha möte med Andreas angående nollning.

		Malin har haft gille och KM har fixat i förrådet. Rahne och Malin har funderat lite på förbättringar till Edekvata.

		Elin har varit på kollegiemöte och de planerar caféphest.

		Isabella hade lunchföreläsning och det gick bra. ENU kommer hålla en branchkväll med SVEP den 17e April.

		Fanny har varit på SRX möte, nominering för pedagogisktpris har öppnat. Fanny har haft möte med programlednigen för BME, de deltar gärna under nollningen.

		Axel har hjälpt till under DreamHackE. Infu rullar på som vanligt. Kodhackarna hackar, teknokraterna lagar och picassosen fixar med affischer.

		Daniel har fått in många motioner. Han hade kollegiemöte där en hel del saker togs upp. Han går in djupare på det senare.


		\subp{C}{Ekonomisk rapport}{\info}
		Ekonomin ser bra ut enligt Adam!

		\emph{Eftersom förvaltningschef Magnus Lundh var frånvarande under mötet ska detta tas med en nypa salt...}

		\subp{D}{Kåren informerar}{\info}

		Kåren hade valförmäktige i söndags och de valde in nya poster.

		\subp{E}{Omvärldsrapport}{\info}

		Denna vecka finns det ingenting nytt att rapportera.


	\end{paragrafer}


	\p{11}{GDPR}{\dis}

	Denna punkten protokollfördes ej

	\p{12}{GoPro}{\bes}

	Daniel \ypa köpa in en GoPro för \SI{5000}{kr} och att detta belastar dispositionsfonden.

	\Mbaby
	\p{13}{Umphboxar}{\dis}

	\Mba lägga fram en proposition under kvällsmötet.

	\p{14}{Kaffekokare}{\dis}

	Malin tycker att vi ska kasta de kaffekokare som inte fungerar. Hon tar på sig att kasta de som inte fungerar och kolla upp vad nya hade kostat.

	Malin \ypa lägga till detta på beslutsuppföljning s09 med henne själv som undertecknad.

	\Mbaby
	\p{12}{Nästa styrelsemöte}{\bes}
	\Mba nästa styrelsemöte ska äga rum på onsdag 2018-04-18 klockan 12:10 i E:1124.

	\p{14}{Beslutsuppföljning}{\bes}

	\Ibfu


	\p{15}{Övrigt}{\dis}

	Fanny lyfte frågan angående fotografi och styrelsekort.

	Elin informerar om att det finns många platser till livsmedelseutbildning på kåren, om det finns fler funktionärer som vill gå.

	Andy vill gärna att phøset ska få pengar till middag under de kvällar de har möte fram till nollningen. Beräknad 4000 för 10 veckor. Malin föreslog sallader istället. Elin beräknade att detta då kommer kosta ungefär \SI{2000}{kr}.

	\p{16}{OFMA}{\bes}
	{\mo} förklarade mötet avslutat 13:02.
\end{paragrafer}

%\newpage
\hidesignfoot
\begin{signatures}{3}
	\signature{\mo}{Mötesordförande}
	\signature{\ms}{Mötessekreterare}
	\signature{\ji}{Justerare}
\end{signatures}
\end{document}