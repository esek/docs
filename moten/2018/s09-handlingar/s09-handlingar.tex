\documentclass[10pt]{article}
    \usepackage[utf8]{inputenc}
    \usepackage[swedish]{babel}
    
    \def\doctype{Handlingar} %ex. Kallelse, Handlingar, Protkoll
    \def\mname{styrelsemöte} %ex. styrelsemöte, Vårterminsmöte
    \def\mnum{S09/18} %ex S02/16, E1/15, VT/13
    \def\date{2018-04-18} %YYYY-MM-DD
    \def\docauthor{Daniel Bakic}
    
    \usepackage{../e-mote}
    \usepackage{../../../e-sek}
    
    \begin{document}
    
    \heading{{\doctype} till {\mname} {\mnum}}
    
    \section*{Information om punkterna}
    
    \begin{paragrafer}
    
    \p{12}{GDPR}{\dis}
     Efter TLTH:s workshop angående GDPR vill jag lyfta denna fråga. Vi har fått mycket information och det är mycket vi behöver göra för att följa lagen som träder i kraft i maj. Alla ombeds att läsa igenom anteckningarna och medföljande länkt som finns i driven så alla har en stadig grund att diskutera utifrån.
    
     \rnamnpost{Elin Johansson}{Cafémästare}

    \p{13}{Lyfta grejer hos LTH:s ledning och dylikt}{\dis}
     Vi har i OK diskuterat lite granna kring vilka personer vi skulle vilja få våra åsikter hörda hos. Jag vill att ni tänker igenom vilka dessa personer kan vara och vad ni vill lyfta. Det kan vara vem som helst, någon från TLTH, någon inom LTH:s ledning, vår rektor eller dylikt.
    
     \rnamnpost{Daniel Bakic}{Ordförande}

    \end{paragrafer}
    
    \begin{signatures}{1}
    \ist
    \signature{\docauthor}{Ordförande}
    \end{signatures}
    
    \end{document}
    