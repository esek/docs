\documentclass[10pt]{article}
\usepackage[utf8]{inputenc}
\usepackage[swedish]{babel}

\def\mo{Daniel Bakic}
\def\ms{Magnus Lundh}
\def\ji{Adam Belfrage}
%\def\jii{}

\def\doctype{Protokoll} %ex. Kallelse, Handlingar, Protkoll
\def\mname{styrelsemöte} %ex. styrelsemöte, Vårterminsmöte
\def\mnum{S21/18} %ex S02/16, E1/15, VT/13
\def\date{2018-10-12} %YYYY-MM-DD
\def\docauthor{\ms}

\usepackage{../e-mote}
\usepackage{../../../e-sek}

\begin{document}
\showsignfoot

\heading{{\doctype} för {\mname} {\mnum}}

%\naun{}{} %närvarane under
%\nati{} %närvarande till och med
%\nafr{} %närvarande från och med
\section*{Närvarande}
\subsection*{Styrelsen}
\begin{narvarolista}
	\nv{Ordförande}{Daniel Bakic}{E15}{}
	%\nv{Kontaktor}{Axel Voss}{E15}{}
	\nv{Förvaltningschef}{Magnus Lundh}{E15}{}
	\nv{Cafémästare}{Elin Johansson}{BME16}{}
	\nv{Øverphøs}{Andreas Bennström}{BME16}{}
	\nv{SRE-ordförande}{Fanny Månefjord}{BME16}{}
	\nv{ENU-ordförande}{Isabella Hansen}{E16}{}
	\nv{Sexmästare}{Alexander Wik}{BME17}{}
	\nv{Krögare}{Malin Heyden}{E16}{}
	\nv{Entertainer}{Adam Belfrage}{BME17}{}
\end{narvarolista}
\subsection*{Ständigt adjungerande}


\begin{narvarolista}
	\nv{Vice Krögare}{Stephanie Bol}{BME17}{}
	%\nv{Stridsrop}{Tove Börjesson}{E17}{}
	%\nv{Inköps- och lagerchef}{Sofie Johannesson}{E17}{}
	%\nv{Inköps- och lagerchef}{Fabian Sondh}{E17}{}
	%\nv{Inköps- och lagerchef}{Albin Pålsson}{E17}{}
	%\nv{Kårordförande}{Linus Hammarlund}{}{}
	%\nv{Kårrepresentant}{Jacob Karlsson}{}{\nafr{3}}
	%\nv{Kårrepresentant}{Hanna Järpedal}{}{}
	\nv{Kårrepresentant}{Philip Johansson}{}{}
	%\nv{Valberedningens ordförande}{Pontus Landgren}{}{}
	%\nv{Skattmästare}{Olle Oswald}{}{}
	%\nv{Kårrepresentant}{Daniel Damberg}{}{}
	%\nv{Kårrepresentant}{John Alvén}{}{}
	%\nv{Nollegeneral}{Jakob Nilsson}{}{}
	%\nv{Skyddsombud}{Axel Sandqvist}{E17}{}
	%\nv{Saga}{}{}{}
	%\nv{Max}{}{}{}
	%\nv{Talman}{Erik Månsson}{E14}{}
	%\nv{Elektras Ordförande}{Elisabeth Pongratz}{}{}
	%\nv{Inspektor}{Monica Almqvist}{}{}
	\nv{Sigillbevarare}{Henrik Ramström}{E16}{}
	%\nv{Vice Entertainer}{Emil Bergström}{}{}
\end{narvarolista}

%\begin{comment}
\subsection*{Adjungerande}
\begin{narvarolista}
	%\nv{Teknokrat}{Oscar Uggla}{E15}{}
	\nv{Stridsrop}{Tove Börjesson}{E15}{}
	%\nv{Post}{Namn}{Klass}{}
\end{narvarolista}
%\end{comment}

\section*{Protokoll}
\begin{paragrafer}
	\p{1}{OFMÖ}{\bes}
	Ordförande {\mo} förklarade mötet öppnat 12:17.

	\p{2}{Val av mötesordförande}{\bes}
	{\valavmo}

	\p{3}{Val av mötessekreterare}{\bes}
	{\valavms}

	\p{4}{Val av justeringsperson}{\bes}
	{\valavj}

	\p{5}{Godkännande av tid och sätt}{\bes}
	{\tosg}

	\p{6}{Adjungeringar}{\bes}
	%{\ingaadj}
	%Förnamn Efternamn adjungerades
	Tove Börjesson adjungerades.

	\p{7}{Godkännande av dagordningen}{\bes}
	Fanny \ypa lägga till \S11 ``Present till Chalmers''.
	
	Elin \ypa lägga till \S12 ``Sjunga för Fanny''.
	
	Stephanie \ypa lägga till \S13 ``Musikhjälpen''.

	%Dagordningen godkändes.
	%Föredragningslistan godkändes med yrkandet.
	Föredragningslistan godkändes med samtliga yrkanden.

	\p{8}{Föregående mötesprotokoll}{\bes}
	\latillprot{19/18 och 20/18}
	%\ingaprot

	\p{9}{Fyllnadsval och entledigande av funktionärer}{\bes}
	\begin{fyllnadsval} %"Inga fyllnadsval." fylls i automatiskt
		%\fval{namn}{post}
		%\entl{Lisa Linárd Pedersen}{SRE-ledamot}
	\end{fyllnadsval}

	\p{10}{Rapporter}{}
	\begin{paragrafer}
		\subp{A}{Hur mår alla?}{\info}
		Punkten protokollfördes ej.

		\subp{B}{Utskottsrapporter}{\info}

		Magnus har bokfört mycket samt betalat fakturor och utlägg.
		
		Nöju har haft spelkväll. Det var lyckat och bra uppslutning.
		
		CM har haft caféet öppet. De lyckades få många dioder till nästa läsperiod på expot.
		
		EnU har haft möte med utskottet. På måndag är de cv-granskning med Academic Work. De fick många anmälningar på Expot.
		
		NollU har inte gjort så mycket. Planerat lite inför Phaddertack.
		
		KM har planerat mycket inför bättre förr-Gille. Kommit på massa roliga grejer. Stay Tuned!
		
		Fanny och Lina har skickat fakturor för pluggkvällar. Nästa måndag är det utbyteskväll där man kan mingla och ställa frågor.
		
		Alex fortsatt planera sittningar i LP 2 och börjat tänka på tema. Även bokfört lite från nollningens evenemang.
		
		Daniel har haft fokus på att planera expot som skedde igår. Han har även hämtat posten samt skrivit in
		alla kravprofiler i \LaTeX. Det ser sexigt ut.

		\subp{C}{Ekonomisk rapport}{\info}
		Ekonomin ser kanonfin ut - Pengar rullar in som det ska, det går bra nu!
		\subp{D}{Kåren informerar}{\info}
		På tisdag är det Höstfullmäktige. Handlingarna har kommit ut. Kåren tackar för en bra nollning. Kårexpeditionen var stängd i onsdags.

		\subp{E}{Omvärldsrapport}{\info}

		%Denna vecka finns det ingenting nytt att rapportera.

		Märkestävling och musikhjälpen.

	\end{paragrafer}

	\p{11}{Present till Chalmers}{\dis}
	Mötet diskuterade vad som skulle ges bort i present till Chalmers som blivit bortglömt när styrelsen åkte och hälsade på.
	
	Fanny \ypaltbu{Present}{22}{henne själv}

	\Mbaby

	\p{12}{Sjunga för Fanny}{\dis}
	Mötet sjöng för Fanny som fyllde år. Då Fanny inte är från skåne framfördes ett fyrfaldigt leve.

	\p{13}{Musikhjälpen}{\dis}
	Den 10-16 december är det Musikhjälpen.
	Stephanie Bol var på möte om musikhjälpen i tisdags och fick massor av idéer om hur sektionen kan medverka genom att dra in pengar.

	Mötet diskuterade massor idéer om vad man kunde göra och beslutade att ha ett uppstartsmöte nästa vecka för att brainstorma idéer om vad som kan göras.


	\p{14}{Nästa styrelsemöte}{\bes}
	\Mba nästa styrelsemöte ska äga rum nästa fredag 2018-10-19 klockan 12:10 i E:1124.

	\p{15}{Beslutsuppföljning}{\bes}

	Elin \ypa stryka ``Inköp av pantkärl''.

	\Mbaby

	``Flagga till F'' samt ``Styrelsemärke'' diskuterades.


	%\Ibfu
	\p{16}{Övrigt}{\dis}

	Inget övrigt att ta upp.

	\p{17}{OFMA}{\bes}
	{\mo} förklarade mötet avslutat 12:56
\end{paragrafer}

%\newpage
\hidesignfoot
\begin{signatures}{3}
	\signature{\mo}{Mötesordförande}
	\signature{\ms}{Mötessekreterare}
	\signature{\ji}{Justerare}
\end{signatures}
\end{document}
