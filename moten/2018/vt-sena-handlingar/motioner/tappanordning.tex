\documentclass[../_main/handlingar.tex]{subfiles}

\begin{document}
\motion{Inköp av ny tappanordning}

Under 2017 så utfördes provkörningar av kökets gamla tappanordning för att undersöka dess funktion samt intresset av att servera fatöl på gillen, utöver sin nuvarande funktion som vattenkran. Försäljning gick bra och var uppskattad men för att fortsatt servering ska vara ett gångbart alternativ måste ny utrustning införskaffas då den existerande är mycket gammal med udda delar och mycket arbete för att sätta upp. Som alternativ finns portabla tappanordningar som ställs i baren och kopplas till fat. Då anläggningen är portabel ges även möjligheten att servera dryck från fat på andra platser och tillfällen, förslagsvis lophtet. Maskinerna finns med antingen med 1 eller 2 kranar.

Därför yrkar jag
\begin{attsatser}
    \att köpa in en en Lindr Overcounter Kontakt 40/K med \textbf{2 tappkranar} tillsammans med kringutrustning och budget sätts till 9000 kr,
    \att inköpet belastar utrustningsfonden, samt
    \att beslutsuppföljningen läggs till HT/18 där undertecknad står som ansvarig.
\end{attsatser}

\textit{eller}

\begin{attsatser}
    \att köpa in en en Lindr Overcounter Kontakt 25/K med \textbf{1 tappkran} kringutrustning och budget sätts till 6000 kr
    \att inköpet belastar utrustningsfonden, samt
    \att beslutsuppföljningen läggs till HT/18 där undertecknad står som ansvarig.
\end{attsatser}

\begin{signatures}{1}
    \mvh
    \signature{Markus Rahne}{Vice förvaltningschef}
    \end{signatures}

\end{document}
