\documentclass[10pt]{article}
\usepackage[utf8]{inputenc}
\usepackage[swedish]{babel}

\def\poltitle{Policy för utlåning och hantering av arbetskläder}
\def\antagen{2016-11-22}
%\def\uppdaterad{}

\usepackage{./e-policy}
\usepackage{../e-styrdok}
\usepackage{../../e-sek}

\begin{document}
    \section*{\doctitle}

    Arbetskläder ska i största möjliga mån finnas tillgängliga för utlåning till medlemmar av de utskott som i sin huvudsakliga verksamhet representerar Sektionen utåt och/eller arrangerar evenemang där alkohol mat servera. Detta ses som en service till personer/organisationer/företag som på olika sätt betalar pengar till Sektion. Att kunna göra valet att inte lägga ut egna pengar på eller riskera att förstöra sina egna kläder, är något som den som ger av sitt engagemang till Sektionen bör ha möjlighet till.

    \begin{dashlist}
        \item Följande gäller för kläder som av Sektionen lånas ut till funktionärer:
        \item Kläderna används bara under de tider som personen i fråga jobbar för det berörda utskottet.
        \item Personen i fråga, som lånat kläderna, står själv för tvätt av dessa.
        \item Eventuell kostnad för kläder som tappas bort eller förstörs i sammanhang som inte har med Sektionens verksamhet att göra står personen som lånat kläderna för.
        \item Utskottets ordförande ansvarar för hantering och utlåning av kläderna.
        \item Om tillräckligt med kläder finns och det i övrigt anses lämpligt är det ok att låna ut ett plagg över längre tid, till exempel en mandatperiod. Övriga punkter i policyn gäller fortsatt.
    \end{dashlist}
\end{document}
