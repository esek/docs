\documentclass[10pt]{article}
\usepackage[utf8]{inputenc}
\usepackage[swedish]{babel}

\def\poltitle{Principer för deltagande i sektionsaktiviteter}
\def\antagen{1984-12-03}
%\def\uppdaterad{}

\usepackage{./e-policy}
\usepackage{../e-styrdok}
\usepackage{../../e-sek}

\begin{document}
\section*{\doctitle}

Alla Sektionsarrangemang bör vara öppna för deltagare från andra Sektioner.
Det är viktigt att denna princip tillämpas generellt, i annat fall riskerar man att Sektionerna alltmer motarbetar varandra.

Detta kan i värsta fall leda till en viss fientlighet mellan de olika Sektionernas medlemmar. Denna fientlighetskänsla tar sig då i uttryck på flera sätt, man kan få höra kommentarer som ``Vad har du här att göra?'' vid besök på andra Sektionsfik. I samband med nollningen finns de största möjligheterna att påverka situationen i någon riktning. Det är alltså då det är som mest viktigt att skapa en gemensam ``vi-känsla'' för hela teknis.

De principer som bör tillämpas kan sammanfattas i två punkter:

\begin{enumerate}
\item Alla Sektionsarrangemang ska vara öppna för alla och utan prioritet för den egna Sektionens medlemmar
\item Sektionsstyrelsen kan bevilja undantag från regel 1 i fall där det är av särskild vikt att Sektionen representeras av egna medlemmar. Att undantag enligt detta gjorts ska då ovillkorligen meddelas i samband med anmälningslistor o dyl.
\end{enumerate}

\end{document}
