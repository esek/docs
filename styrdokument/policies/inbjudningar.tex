\documentclass[10pt]{article}
\usepackage[utf8]{inputenc}
\usepackage[swedish]{babel}

\def\poltitle{Inbjudningar och anmodningar}
\def\antagen{1986-04-22}
\def\uppdaterad{2019-11-12} 

\usepackage{./e-policy}
\usepackage{../e-styrdok}
\usepackage{../../e-sek}

\begin{document}
\section*{\doctitle}

\subsection*{1. Sittningar}
Följande personer skall inbjudas
\begin{dashlist}
    \item Ordföranden
    \item Inspektorn (Vid finare tillställningar) 
    \item Gäster som bistår i genomförande av och organisation under sittningen
    \item Person som skall mottaga Krusidull-E
    
\end{dashlist}
Följande personer skall anmodas
\begin{dashlist}
    \item Styrelsemedlemmar
    \item Valberedningens ordförande
    \item Hedersmedlemmar (Vid finare tillställningar) 
\end{dashlist}

\subsubsection*{1.1 Sittningar under nollningen, exklusive Nollegasque}
Utöver de vanliga inbjudningarna skall även följande personer inbjudas, 
\begin{dashlist}
    \item Phøs
\end{dashlist}
Utöver de vanliga anmodningarna skall även följande personer anmodas. 
\begin{dashlist}
    \item ØGP
\end{dashlist}

\subsubsection*{1.2 Nollegasque}
Följande personer skall inbjudas
    \begin{dashlist}
        \item Ordföranden
        \item Øverphøs
        \item Inspektorn
        \item Hedersmedlemmar
        \item Person som skall mottaga Krusidull-E
        \item H.M. Konung Carl XVI Gustaf
    \end{dashlist}

Följande personer skall anmodas i den turordning som följer och i mån av plats
\begin{dashlist}
    \item Styrelsemedlemmar 
    \item Co-phøs och ØGP
    \item Nyantagna studenter
    \item Valberedningens ordförande
 %   \item HeHE:s chefredaktör %Ströks HTM19
    \item Sektionens kontaktpersoner från Teknologkåren, samt heltidare och styrelseledamöter från E-sektionen
    \item Gamla sektionsordförande som fortfarande är studerande vid skolan eller deltog i förra årets arrangemang
    \item Representanter från LTH-externa vänsektioner
    \item Nollegeneral samt Nolleamiral
    \item Nollningsfunktionärer
    \item Övriga medlemmar vid E-sektionen
\end{dashlist}

\subsubsection*{1.3 Skiphtesgasque}
Följande personer skall inbjudas
\begin{dashlist}
    \item Hedersmedlemmar
    \item Funktionärer, avgående och pågående
\end{dashlist}

\subsection*{2. Övrigt}

\begin{dashlist}
    \item Person som skall mottaga medalj skall inbjudas alternativt anmodas
    \item Sigillbevararen inbjudes alternativt anmodas när denne skall dela ut medaljer
    \item Gamla sektionsordföranden som var med på föregående Svolder eller motsvarande arrangemang skall anmodas
    \item Övriga kan inbjudas/anmodas beroende på tillställning
\end{dashlist}

Denna policy ska följas i den mån Sexmästaren finner det lämpligt. Vid tillställningar då policyn kring anmodningar och inbjudningar inte är lämplig bör Sexmästaren vid ett styrelsemöte ta upp frågan och styrelsen besluta om att avvika från policyn. 
Sexmästaren beslutar även i vilken utsträckning som personer anmodas med respektive.

\vspace{20px}
\subsection*{Kommentar (ej del av policybeslut):}
\begin{dashlist}
    \item Officiell representation inom LU betalas fullt ut av Sektionen
    (Styrelsebeslut 1993-05-05)
    \item Vad gäller icke specificerade inbjudningar och anmodningar så går
    frågan enligt turordningen
    \begin{numplist}
    \item Ordföranden
    \item Styrelsen
    \item Övriga medlemmar
    \end{numplist}
    \item Om flera personer på samma nivå är intresserade så delas inbjudan lika
    (Styrelsebeslut S3/93)
\end{dashlist}


\end{document}
