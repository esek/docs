\documentclass[10pt]{article}
\usepackage[utf8]{inputenc}
\usepackage[swedish]{babel}

\def\poltitle{Nyckelpolicy för E-sektionens funktionärer}
\def\antagen{1996-10-15}
\def\uppdaterad{2004-05-06}

\usepackage{./e-policy}
\usepackage{../e-styrdok}
\usepackage{../../e-sek}

\begin{document}
\section*{\doctitle}

\subsubsection*{Allmänt}

Funktionären bör ha tillgång till de nycklar som krävs för att sköta posten. Sektionen är inte skyldig att tilldela funktionär nyckel, och det är således inget rättighet för funktionären.

Målsättningen är att alla funktionärer inom E-Sektionen skall få tilldelat nycklar så att de kan komma åt telefon, dator, kopiator mm. Sektionsstyrelse kan dock av kostnadsskäl, allmänn tjuvaktighet eller annan anledning dra ner på den generella nyckeltilldelningen.

Beslut om att inte tilldela funktionär nyckel, beroende på funktionärs person, fattas av Styrelsen. Tilldelning av nycklar sköts normalt av Sektionsordförande. Om oenighet föreligger mellan funktionär och ordförande bestämmer Styrelsen. Funktionär äger rätt att få frågan upptagen på Terminsmöte.

\begin{enumerate}[label=\S\arabic*.]
  \item Om funktionär lämnar in avsägelse, studerar utomlands, eller av annan anledning inte förmodas kunna fullfölja sina uppgifter, skall nycklarna utan dröjsmål inlämnas.
  \item Funktionär är skyldig att omedelbart efter mandatperiods utgång, inlämna utkvitterade nycklar.
  \item Funktionär är skyldig att erlägga depositionsavgift för utkvitterade nycklar. Storleken på depositionen bestäms av Sektionsmöte. Om en funktionär ej återlämnar sina nycklar efter det att mandattid gått ut och efter påminnelse ändå inte återlämnar nycklarna, kan Styrelsen besluta om att depositionsavgiften är förverkad.
  \item Funktionär ansvarar för sina nycklar, och kan bli ersättningsskyldig för såväl skador som stölder som orsakas av försummelse från funktionärens sida. Detta kan gälla bl.a. försummelse att låsa Sektionens lokaler, utlåning av nycklar till obehörig.
  \item Förlust av nyckel skall omedelbart anmälas till Sektionsordförande, samt givetvis till polisen. Sektionens åtgärder i sådant fall bestämmes av Styrelsen, och i brådskande fall av ordföranden. I de fall skada uppkommer i samband med förlust och detta inte har anmälts kan funktionär bli ersättningsskyldig. Förlust av nyckel innebär att depositionsavgiften är förverkad.
  \item Funktionär som tilldelats nyckelbricka, är skyldig att alltid låta denna sitta tillsammans med utkvitterade nycklar.
  \item Vid misstanke om oegentligheter, misskötande av förtroendeuppdrag eller dylikt, äger Sektionsordförande rätt att omedelbart och tillfälligt dra in funktionärs nyckel.
  \item Indragning av funktionärs nyckel bestäms av Styrelsen. Funktionär äger rätt att delta på styrelsemötet där ärendet avgörs, samt äger rätt att få ärendet upptagit på Terminsmöte.
  \item Funktionär är skyldig att följa dessa regler. Om någon regel inte följts äger ordföranden rätt att med omedelbar verkan dra in funktionärs nyckel.
\end{enumerate}

\end{document}
