\documentclass[10pt]{article}
\usepackage[utf8]{inputenc}
\usepackage[swedish]{babel}

\def\poltitle{Alkohol- och drogpolicy}
\def\antagen{2013-04-23}
\def\uppdaterad{2016-11-22}

\usepackage{./e-policy}
\usepackage{../e-styrdok}
\usepackage{../../e-sek}

\begin{document}
\section*{\doctitle}
\subsection*{1. Syfte}
E-sektionen inom TLTH är en ideell förening som arrangerar gillen och sittningar med syftet att främja sektionens medlemmars studiesociala miljö. Vid dessa tillställningar kan det förekomma försäljning av alkoholhaltiga drycker. Syftet med denna policy är att främja en ansvarsfull alkoholservering vid tillställningar anordnade av E-sektionen samt att göra ett aktivt ställningstagande mot användande av narkotika.

E-sektionen ställer sig även bakom TLTH:s alkohol- och drogpolicy. Denna policy bör ses som ett komplement till TLTH:s policy.

\subsection*{2. Alkoholservering}
\subsubsection*{2.1 Allmänt}
Denna policy utgår i från alkohollagen (SFS 2010:1622) som ska följas vid alla arrangemang anordnade av E-sektionen. Med anledning av detta ska den som är serveringsansvarig under ett arrangemang där alkoholförsäljning förekommer ha god kunskap om innehållet i alkohollagen, i synnerhet de kapitel som handlar om servering av alkohol samt tillsyn av serveringsställen. Det åligger den serveringsansvarige att tillse att de som jobbar under ett arrangemang har tillfredställande kunskap om alkohollagens innehåll, i synnerhet de kapitel som handlar om servering av alkoholhaltiga drycker.

För att säkerställa att den serveringsansvarige har en tillfredsställande kunskap om alkohollagens innehåll bör alla som är anmälda som serveringsansvariga hos tillståndsmyndigheten ha genomgått en av kommunens kurser om ansvarsfull alkoholservering.

\subsubsection*{2.2 Under arrangemangets gång}
Den serveringsansvarige bör kontinuerligt under arrangemangets gång kontrollera att nödutgångar och brandgångar ej är blockerade samt tillse att ordning upprätthålls och att en hög berusningsgrad hos gästerna undviks. I det fallet att en gäst skulle dricka sig för berusad ska denne avvisas från tillställningen.

\subsubsection*{2.3 Alkoholfria alternativ}
Vid varje tillställning anordnad av E-sektionen där alkoholservering förekommer ska det finnas vatten att tillgå för gästerna.

Alkoholfria alternativ, exempelvis alkoholfri öl och cider ska finnas i den omfattning som alkohollagen föreskriver. Vinstmarginalen för alkoholfria drycker får aldrig vara större än den för drycker som innehåller alkohol. Detta innebär exempelvis att priset på sittningsbiljetter bör ändras om inköpspriset för det alkoholfria alternativet är annat än alternativet som innehåller alkohol.

\subsection*{3. Ställningstagande mot narkotika}
E-sektionen tar avstånd från all användning av narkotikaklassade medel annat än för medicinskt bruk som sker enligt läkares föreskrift. Om någon gäst skulle ertappas med att missbruka narkotika ska denne avvisas från tillställningen. Dessutom ska en polisanmälan göras av den serveringsansvarige.

\subsection*{4. Alkohol som tack eller gåva}
Vid funktionärstack dit alla Sektionens funktionärer är inbjudna kan funktionärerna bjudas på en mindre mängd alkoholhaltig dryck till maten. Alkoholen ska absolut inte vara den stora begivenheten för tacket. Vid tillfällen när Sektionen vill förära en person eller förening med en gåva kan denna bestå av alkoholhaltig dryck. Varje tillfälle vare sig det gäller tack eller gåva ska i förväg godkännas av styrelsen.

\end{document}
