\documentclass[10pt]{article}
\usepackage[utf8]{inputenc}
\usepackage[swedish]{babel}

\def\poltitle{Hantering av ärenden som faller inom diskrimineringslagen}
\def\antagen{xxxx-xx-xx}
\def\uppdaterad{xxxx-xx-xx}

\usepackage{./e-policy}
\usepackage{../e-styrdok}
\usepackage{../../e-sek}

\begin{document}
\section*{\doctitle}
\subsection*{1. Syfte}
Alla studenter och anställda vid universitetet delar ansvaret för att nå ett klimat som utesluter diskriminering och trakasserier av alla dess former. Alla ska känna sig trygga i sin skolmiljö, både under aktiviteter på dagen så väl som på kvällen.

Sektionen ska ta en aktiv roll i arbetet mot diskriminering och ständigt hålla diskussionen om likabehandling levande, för medlemmarnas välmående. Sektionen ska även vara tillgänglig med hjälp och råd genom studerandeskyddsombuden med likabehandlingsansvar både anonymt och icke-anonymt. Denna policy ämnar att ge sektionens funktionärer ett alternativ på handlingsplan för att snabbt kunna agera vid företeelser och för att motverka dess uppkomst.

\subsection*{2. Handlingsrutiner vid rapportering}
Vid händelser som faller inom diskrimineringslagen eller vid rapportering om orättvis eller kränkande behandling kan följande handlingssätt nyttjas.

\begin{itemize}
    \item Ärendet ska prioriteras omgående. I första hand ska studerandeskyddsombuden med likabehandlingsansvar informeras. Är studerandeskyddsombuden inte tillgängliga ska Sektionsordförande ta deras ställe.
    \item Samtala alltid först med den som känner sig drabbad. Upplevelsen kan ofta ha uppfattats som traumatisk och då uttrycka sig som ilska, oro, otrygghet mm. Föreslå att personen ska kontakta LTHs kuratorer, studie- och karriärvägledare, Studenthälsan eller Studentprästerna.
    \item Samtalen ska vara konfidentiella och önskar den drabbade anonymitet ska detta respekteras och i största möjliga mån eftersträvas. Dock ska det klargöras för den drabbade att om vidare åtgärder ska kunna tas måste kontakt tas med förövaren.
    \item Ta också reda på om det kan finnas eventuella vittnen. Ifrågasätt inte den utsatta personens utsatthet, det är enbart den personen som kan avgöra hur deras upplevelse tolkats. Däremot måste personen kunna peka på vad som orsakat känslan av diskriminering eller trakasserier.
    \item Dokumentera alla samtal. Vid misstanke om att lagbrott begåtts ska Sektionen uppmuntra till och erbjuda hjälp med polisanmälan men tänk på att det alltid är den utsatta som avgör.
    \item Efter kartläggning av den drabbade personens upplevelse kan ett konfidentiellt samtal med den som utsatt hållas. Visa respekt för båda parter, men visa på ett tydligt sätt att sektionen tar avstånd från kränkande beteenden av alla slag.
    \item Efter ett tag bör ett uppföljningssamtal anordnas och situationen utvärderas. Har läget inte förbättras eller ytterligare rapportering om personer som blivit utsatta av samma person inkommit  kan det vara en god idé att vidta åtgärder.
\end{itemize}

\subsection*{3. Övrig information}
Om ärendet ligger under LTHs eller Teknologkårens ansvar lämnas ärendet och eventuella åtgärder till aktuell ansvarig. I dessa fall fungerar sektionen som stöd för den drabbade. I övriga fall informeras LTH och/eller Teknologkåren då det bedöms lämpligt. Om det under arbetsprocessen bedöms finnas ett behov av ytterligare stöd eller någon annans perspektiv, tveka inte på att ta kontakt med TLTH eller LTHs likabehandlingssamordnare.

Det är viktigt att sektionens verksamhet ständigt utvärderas utifrån flera perspektiv för att upptäcka problem. Vid händelser knutna till specifika evenemang kan ansvarig utskottsordförande informeras med syftet att förbättra evengemanget och granska dess framtida lämplighet. Därefter kan resten av styrelsen informeras om detta finnes passande. Med det sagt bör den drabbades önskemål alltid tas i beaktning innan spridning av information.  

Utöver detta bör sektionen ha ett kontinuerligt förebyggande arbete för att med bästa förmåga säkerställa medlemmarnas trygghet och skapa en miljö där drabbade personer inte väntar med att rapportera händelser. Detta kan till exempel göras genom att ge stöd för studerandeskyddsombudens arbete, aktivt hålla diskussionen kring diskriminering och likabehandling vid liv samt låta alla som arbetar för sektionen under nollningens evenemang ses som nollningsfunktionärer och därmed låta dem skriva under Teknologkårens nollningskontrakt om sådant brukas för deltagande under nollningen.
\end{document}
