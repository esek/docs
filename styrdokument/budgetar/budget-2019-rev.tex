\documentclass[10pt]{article}
    \usepackage[utf8]{inputenc}
    \usepackage[swedish]{babel}
    
    \def\year{2019}
    \def\antagen{2019-04-09}
    
    \usepackage{./e-budget}
    \usepackage{../e-styrdok}
    \usepackage{../../e-sek}
    
    \begin{document}
    \section*{\doctitle}
    \begin{tabularx}{10cm}{X r r}
        \textbf{\large Sektionen} & \textbf{2018} & \textbf{2019} \\
        \hline
        \textbf{SEK01} \\
        Sektionsavgifter & \SI{39000}{kr} & \SI{40000}{kr} \\
        Sektionsmöten & \SI{-10000}{kr} & \SI{-10000}{kr} \\
        Sektionens representation & \SI{-7000}{kr} & \SI{-7000}{kr} \\
        Ordförandens representation & \SI{-1000}{kr} & \SI{-1000}{kr} \\
        Funkionärsvård & \SI{-110000}{kr} & \SI{-110000}{kr} \\
        Arbetsglädje & \SI{-15000}{kr} & \SI{-30000}{kr} \\
        Arbetskläder Funktionärer & \SI{-1000}{kr} & \SI{-3000}{kr} \\
        Medaljer & \SI{-500}{kr} & \SI{-1700}{kr} \\
        Pant & \SI{5000}{kr} & \SI{10000}{kr} \\
        Expo & \SI{-2000}{kr} & \SI{-2000}{kr} \\
        \textbf{SEK02}, Revisorer & \SI{-500}{kr} & \SI{-500}{kr} \\
        \textbf{SEK03}, Valbered & \SI{-1000}{kr} & \SI{-1000}{kr} \\
        \textbf{SEK05}, Skiphte & \SI{-10000}{kr} & \SI{-15000}{kr} \\
        \hline
        \textbf{Summa} & \textbf{\SI{-114000}{kr}} & \textbf{\SI{-131200}{kr}} \\
    \end{tabularx}
    
    \begin{tabularx}{10cm}{X r r}
        \textbf{\large Styrelsen} & \textbf{2018} & \textbf{2019} \\
        \hline
        \textbf{STY01} \\
        Klädsel & \SI{-15000}{kr} & \SI{-15000}{kr} \\
        Styrelserepresentation, internt & \SI{-5000}{kr} & \SI{-5000}{kr} \\
        Styrelserepresentation, externt & \SI{}{} & \SI{-5000}{kr} \\
        Styrelsen internt & \SI{-12000}{kr} & \SI{-14000}{kr} \\
        \hline
        \textbf{Summa} & \textbf{\SI{-32000}{kr}} & \textbf{\SI{-39000}{kr}} \\
    \end{tabularx}
    
    \begin{tabularx}{10cm}{X r r}
        \textbf{\large Sexmästeriet} & \textbf{2018} & \textbf{2019} \\
        \hline
        \textbf{SEX01} \\
        E6 Allmänt & \SI{5000}{kr} & \SI{10000}{kr} \\
        \hline
        \textbf{Summa} & \textbf{\SI{5000}{kr}} & \textbf{\SI{10000}{kr}} \\
    \end{tabularx}
    
    \begin{tabularx}{10cm}{X r r}
        \textbf{\large Näringslivsutskottet} & \textbf{2018} & \textbf{2019} \\
        \hline
        \textbf{ARMU01} \\
        Sponsring & \SI{75000}{kr} & \SI{75000}{kr} \\
        Teknikfokus & \SI{130000}{kr} & \SI{165000}{kr} \\
        Almuniverksamhet &  & \SI{-1000}{kr} \\
        \hline
        \textbf{Summa} & \textbf{\SI{205000}{kr}} & \textbf{\SI{239000}{kr}} \\
    \end{tabularx}
    
    \begin{tabularx}{10cm}{X r r}
        \textbf{\large Informationsutskottet} & \textbf{2018} & \textbf{2019} \\
        \hline
        \textbf{INFU01} \\
        Nolleguide & \SI{-6000}{kr} & \SI{-7500}{kr} \\
        Datordrift & \SI{-2000}{kr} & \SI{-6200}{kr} \\
        HeHE & \SI{0}{kr} & \SI{2000}{kr} \\
        \hline
        \textbf{Summa} & \textbf{\SI{-9000}{kr}} & \textbf{\SI{-11700}{kr}} \\
    \end{tabularx}
    
    \begin{tabularx}{10cm}{X r r}
        \textbf{\large Källarmästeriet} & \textbf{2018} & \textbf{2019} \\
        \hline
        \textbf{KM01} \\
        Gille & \SI{25000}{kr} & \SI{20000}{kr} \\
        Fast tillstånd & \SI{-14000}{kr} & \SI{-14000}{kr} \\
        \hline
        \textbf{Summa} & \textbf{\SI{11000}{kr}} & \textbf{\SI{6000}{kr}} \\
    \end{tabularx}
    
    \begin{tabularx}{10cm}{X r r}
        \textbf{\large Cafémästeriet} & \textbf{2018} & \textbf{2019} \\
        \hline
        \textbf{CM01} \\
        LED & \SI{120000}{kr} & \SI{140000}{kr} \\
        Tillsyn & \SI{-2000}{kr} & \SI{-2000}{kr} \\
        \textbf{CM02} \\
        CM-förrådet & & \SI{0}{kr} \\
        \hline
        \textbf{Summa} & \textbf{\SI{118000}{kr}} & \textbf{\SI{138000}{kr}} \\
    \end{tabularx}
    
    \begin{tabularx}{10cm}{X r r}
        \textbf{\large Nöjesutskottet} & \textbf{2018} & \textbf{2019} \\
        \hline
        \textbf{NOJU01} \\
        Allmänt & \SI{-6000}{kr} & \SI{-8000}{kr} \\
        \textbf{NOJU02} \\
        Biljard och spel & \SI{-1000}{kr} & \SI{-1000}{kr} \\
        Tandem & \SI{-500}{kr} & \SI{-500}{kr} \\
        Sångarstriden & \SI{-7000}{kr} & \SI{-7000}{kr} \\
        UtEDischot & \SI{25000}{kr} & \SI{15000}{kr} \\
        \textbf{NOJU03} \\
        Sporta med E & \SI{-8000}{kr} & \SI{-6500}{kr} \\
        \hline
        \textbf{Summa} & \textbf{\SI{2500}{kr}} & \textbf{\SI{-8000}{kr}} \\
    \end{tabularx}
    
    \begin{tabularx}{10cm}{X r r}
        \textbf{\large Studierådet} & \textbf{2018} & \textbf{2019} \\
        \hline
        \textbf{SRE01} \\
        CEQ-intäkter & \SI{10000}{kr} & \SI{10000}{kr} \\
        SRE disponibelt & \SI{-10000}{kr} & \SI{-10000}{kr} \\
        \hline
        \textbf{Summa} & \textbf{\SI{0}{kr}} & \textbf{\SI{0}{kr}} \\
    \end{tabularx}
    
    \begin{tabularx}{10cm}{X r r}
        \textbf{\large Nolleutskottet} & \textbf{2018} & \textbf{2019} \\
        \hline
        \textbf{PHOS01} \\
        NollU allmänt & \SI{-32000}{kr} & \SI{-35000}{kr} \\
        NollU internt & \SI{-15000}{kr} & \SI{-20000}{kr} \\
        Phaddertack & \SI{-5000}{kr} & \SI{-5000}{kr} \\
        \hline
        \textbf{Summa} & \textbf{\SI{-52000}{kr}} & \textbf{\SI{-60000}{kr}} \\
    \end{tabularx}
    
    \begin{tabularx}{10cm}{X r r}
        \textbf{\large Förvaltningsutskottet} & \textbf{2018} & \textbf{2019} \\
        \hline
        \textbf{FVU01} \\
        Expedition & \SI{-13000}{kr} & \SI{-13000}{kr} \\
        Kontantfri lösning & \SI{-16000}{kr} & \SI{-20000}{kr} \\
        Fin. int. och kost. & \SI{-5000}{kr} & \SI{-5000}{kr} \\
        E-shop & \SI{3000}{kr} & \SI{-1000}{kr} \\
        Ljud och ljus & \SI{0}{kr} & \SI{1000}{kr}\\
        \textbf{FVU02} \\
        Edekvata & \SI{-20000}{kr} & \SI{-20000}{kr} \\
        \hline
        \textbf{Summa} & \textbf{\SI{-51000}{kr}} & \textbf{\SI{-57000}{kr}} \\
    \end{tabularx}
    
    \begin{tabularx}{10cm}{X r r}
        \textbf{\large Total summa} & \textbf{2018} & \textbf{2019} \\
        \hline
         & \textbf{\SI{83500}{kr}} & \textbf{\SI{85100}{kr}} \\
    \end{tabularx}
    
    
    \newpage
    \subsection*{Budgetriktlinjer 2019}
    
    \subsection*{Sektionen}
    \titlerule[0.5pt]
    \begin{description}[style=multiline, leftmargin=60mm]
    \item[SEK01, Sektionsavgifter]
    Sektionens andel av medlemmarnas betalda kåravgift.
    
    \item[SEK01, Sektionsmöten]
    Avser kostnader för sektionsmöten.
    
    \item[SEK01, Sektionens representation]
    Avser kostnader för gåvor till personer som på ett betydande sätt gagnat Sektionen och gåvor till andra sektioner och organisationer. Kostnader för att bjuda in gäster till Sektionens arrangemang ska belasta arrangemanget i fråga, med undantag för arrangemang som enbart syftar till att representera Sektionen vilka ska belasta detta konto.
    
    \item[SEK01, Ordförandens representation]
    Detta konto innefattar representation av Ordföranden eller dennes ställföreträdare på våra som vänsektioners fester såväl som på större arrangemang inom Teknologkåren.
    
    \item[SEK01, Funktionärsvård]
    All funktionärsvård skall belasta denna budgetpost. Detta inkluderar bland annat kostnaden för mötesfika, subventioneringen av mat och kaffe för funktionärer, samt kostnaden för utskottsaktiviteter med syfte att stärka utskottsgemenskapen. En del av denna budgetpost får med fördel användas för att bekosta arrangemang i syfte att avtacka funktionärerna. Funktionärsvård endast gällande styrelsen skall täckas av styrelsen internt.
    
    \item[SEK01, Medaljer]
    Avser kostnader för utdelade medaljer, det vill säga lagerfört inköpspris för utdelad medalj.
    
    \item[SEK01, Arbetsglädje]
    Avser kostnader för arbetsglädje från Cafémästeriets förråd.
    
    \item[SEK01, Arbetskläder för funktionärer]
    Används till underhåll samt mindre nyinköp av de arbetskläder som Sektionen lånar ut till sina funktionärer.
    
    \item[SEK01, Pantintäkter]
    Avser intäkter för pant.
    
    \item[SEK01, Expo]
    Avser kostnader för expot.
    
    \item[SEK02,  Revisorer]
    För revisorerna fritt disponibel summa.
    
    \item[SEK03, Valberedning]
    För valberedningen fritt disponibel summa.
    
    \item[SEK05, Skiphte]
    Avser kostnader för Sektionens funktionärsskiphte.
    \end{description}
    
    \subsection*{Styrelsen}
    \titlerule[0.5pt]
    \begin{description}[style=multiline, leftmargin=60mm]
    \item[STY01, Klädsel]
    Pengar tilldelade styrelsen och dess vice för inköp av profilkläder samt frackbrodyr.
    
    \item[STY01, Styrelserepresentation, internt]
    Avser kostnader för subventionering av biljetter till sektionstillställningr för styrelsen. Subventionen fördelas jämnt på alla styrelsens medlemmar. Köp av alkohol får ej belasta denna budget.
    
    \item[STY01, Styrelserepresentation, externt]
    Detta konto innefattar representation av styrelsen på vänsektioners fester såväl som på större arrangemang inom Teknologkåren.
    
    \item[STY01, Styrelsen internt]
    Avser styrelsens interna kostnader såsom för ``Kurs På Landet'' och mat till möten.
    \end{description}
    
    \subsection*{Sexmästeriet}
    \titlerule[0.5pt]
    \begin{description}[style=multiline, leftmargin=60mm]
    \item[SEX01, E6 allmänt]
    Avser kostnader och intäkter rörande E6:s verksamhet. Sexmästeriet ska sikta på att enbart gå i vinst på alkohol och i övrigt hålla nere priserna så att så många som möjligt har råd att delta i deras arrangemang.
    \end{description}
    
    \subsection*{Näringslivsutskottet}
    \titlerule[0.5pt]
    \begin{description}[style=multiline, leftmargin=60mm]
    \item[ARMU01, Sponsring]
    Budgeterat belopp innefattar både nollningssponsring och övrig sponsring. Om ett företag sponsrar en viss produkt, till exempel T-shirtar till nollorna, bör även kostnaden för dessa belasta denna budgetpost.
    
    \item[ARMU01, Teknikfokus]
    Avser intäkter från Teknikfokus.
    
    \item[ARMU01, Almuniverksamhet]
    Till förfogande för de Alumniansvariga att använda till diverse Alumniaktiviteter. Arrangemang för alumner ska i största möjliga mån betalas av deltagarna.
    \end{description}
    
    \subsection*{Informationsutskottet}
    \titlerule[0.5pt]
    \begin{description}[style=multiline, leftmargin=60mm]
    \item[INFU01, Nolleguide]
    Avser kostnader relaterade till Nolleguiden.
    
    \item[INFU01, Datordrift]
    Avser kostnader för drift och underhåll samt uppgraderingar av program- och maskinvara till Sektionens datorutrustning. Inköp av ny datorutrustning och liknande bör läggas på dispositionsfonden.
    
    \item[INFU01, HeHE]
    Nollbudgeteras då HeHE inte längre trycks.
    \end{description}
    
    \subsection*{Källarmästeriet}
    \titlerule[0.5pt]
    \begin{description}[style=multiline, leftmargin=60mm]
    \item[KM01, Gillen]
    Avser kostnader och intäkter rörande KM:s verksamhet.
    
    \item[KM01, Fast tillstånd]
    Avser kostnaden för det fasta serveringstillståndet i Edekvata.
    \end{description}
    
    \subsection*{Cafémästeriet}
    \titlerule[0.5pt]
    \begin{description}[style=multiline, leftmargin=60mm]
    \item[CM01, LED]
    Avser kostnader och intäkter gällande LED-café, inklusive mat och avtackning för Cafémästeriets funktionärer.
    
    \item[CM01, Tillsyn]
    Avser kostnad för årlig tillsyn av miljöförvaltningen.
    
    \item[CM02, CM-förrådet]
    Avser intäkter från läsk- och kakköp direkt från Cafémästeriets förråd till självkostnadspris samt kostnaden för inköp av varorna. Eftersom dessa ska gå jämnt ut är budgetposten nollbudgeterad.
    \end{description}
    
    \subsection*{Nöjesutskottet}
    \titlerule[0.5pt]
    \begin{description}[style=multiline, leftmargin=60mm]
    \item[NOJU01, NöjU allmänt]
    Avser kostnader för Sektionens allmänna arrangemang som ej innefattas av övriga budgetposter.
    
    \item[NOJU02, Biljard och spel]
    Avser kostnader för biljard och spel. Tillgång till biljard ska inte kosta något för Sektionens medlemmar.
    
    \item[NOJU02, Tandem]
    Avser  kostnader och intäkter i samband med Tandem. Ska täcka underhåll av tandemcyklarna, övriga kostnader skall kvitteras ut mot biljettintäkterna.
    
    \item[NOJU02, Sångarstriden]
    Avser kostnader för Sektionens deltagande i Sångarstriden, exempelvis kostnaden för dekor och dylikt. På budgeten får också med fördel ett tack för alla deltagande i Sångarstriden belastas.
    
    \item[NOJU02, UtEDischot]
    Avser intäkter och kostnader för UtEDischot.
    
    \item[NOJU03, Sporta med E]
    Avser kostnader för idrottsverksamhet som till exempel hallhyra och inköp av diverse idrottsredskap.
    \end{description}
    
    \subsection*{Studierådet}
    \titlerule[0.5pt]
    \begin{description}[style=multiline, leftmargin=60mm]
    \item[SRE01, CEQ-intäkter]
    Avser intäkter från granskning av CEQ-enkäter. Denna intäkt måste spenderas av SRE och utgör grunden för SRE disponibelt.
    
    \item[SRE01, SRE disponibelt]
    Avser kostnader för SRE:s verksamhet som till exempel pluggkvällar, inspirationsföreläsningar och CEQ-priser.
    \end{description}
    
    \subsection*{Nolleutskottet}
    \titlerule[0.5pt]
    \begin{description}[style=multiline, leftmargin=60mm]
    
    \item[PHOS01, NollU allmänt]
    Budgeterat belopp ska användas för att introducera de nyantagna i studentvärlden.
    
    \item[PHOS01, NollU internt]
    Avser kostnader för NollUs interna verksamhet såsom kläder, skiphte och mat.
    
    \item[PHOS01, Phaddertack]
    Avser kostnader för att tacka phaddrar och andra nollningsaktiva för deras nollningsengagemang då de varken omfattas av Sektionens funktionärvård eller arbetsglädje.
    \end{description}
    
    \subsection*{Förvaltningsutskottet}
    \titlerule[0.5pt]
    \begin{description}[style=multiline, leftmargin=60mm]
    \item[FVU01, Expedition]
    Avser kostnader för Hamnkontorets behov av kontorsmaterial samt kostnader för arkivering.
    
    \item[FVU01, Kontantfri lösning]
    Avser kostnader för Sektionens kontantfria betalningslösningar.
    
    \item[FVU01, Finansiella intäkter och kostnader]
    Avser kostnader för förvaltandet av vårt kapital, till exempel bankkostnader, deponering, samt intäkter i form av avkastning på bankmedel och eventuella aktieaffärer.
    
    \item[FVU01, E-shop]
    Avser E-shopens intäkter och kostnader.
    
    \item[FVU01, Ljud och Ljus]
    Avser kostnader för löpande underhåll av, samt hyresintäkter från, befintlig ljud- och ljusanläggning.
    
    \item[FVU02, Edekvata]
    Avser kostnader för löpande underhåll av Sektionens lokaler samt mindre investeringar. Intäkter i samband med utlåning av Sektionens lokaler samt utrustning som inte innefattas av Ljud och Ljus ska läggas här.
    
    \end{description}
    
    \newpage
    \subsection*{Långsiktiga ekonomiska mål 2019}
    
    \subsubsection*{Dispositionsfonden}
    Målet med dispositionsfonden är att ge styrelsen och dess utskott ett ekonomiskt underlag att kunna införskaffa nya inventarier och/eller förändra Sektionens verksamhet utanför verksamhetsårets budget. Investeringar från fonden ska gärna gynna sektionens verksamhet i sin helhet och gärna göra det på en längre sikt. Det rekommenderas att inköp som skulle motsvara en majoritet av dispositionsfondens ingående årssaldo istället prioriteras att läggas på utrustningsfonden vid ett sektionsmöte.
    
    \subsubsection*{Olycksfonden}
    Målet med olycksfonden är att den ska fungera som en buffert för att kunna ersätta och/eller reparera funktionsodugliga inventarier som är nödvändiga för Sektionens ordinarie verksamhet. Olyckfonden ska också användas i samband med olycksfall med personskador eller personlig utrustning.
    
    \subsubsection*{Eget kapital}
    Målet med eget kapital är att Sektionen ska kunna bedriva sin ordinarie verksamhet utan att få likviditetsproblem. Det egna kapitalet ska vara så stort så att det klarar av dem svängningar i bank/handkassa som uppstår under verksamhetsårets gång utan att kapital från andra fonder behöver röras. Årsbudget ska sättas på så sätt att det egna kapitalet är lika stort vid årets slut som vid årets början.
    
    \subsubsection*{Utrustningsfonden}
    Målet med utrustningsfonden är att ackumulera kapital för att kunna göra större nyinvesteringar eller ombyggnader. Från utrustningsfonden ska både styrelsen och ordinarie medlemmar kunna äska pengar från Sektionen till olika projekt och inköp.
    
    \subsubsection*{Jubileumsfonden}
    Målet med jubileumsfonden att ackumulera kapital till att delvis bekosta ett jubileum vart femte år. Tanken är att göra detta genom att varje år avsätta en liten del av resultatet, lämpligtvis en femtedel av målet för jubileumsfonden, om så är möjligt.
    
    \subsubsection*{Årsresultat}
    Målet är att Sektionens årsresultat ska budgeteras på sådant sätt att fonderna och eget kapital klarar av att uppfylla sina mål både på kort och lång sikt. Sektionen ska sträva mot att endast gå så mycket plus som anses nödvändigt, för att ge tillbaka så mycket som möjligt till våra medlemmar.
    
    \newpage
    \subsubsection*{Konkretisering av målen och hur de uppnås}
    Målet för \textbf{dispositonsfonden} är att det vid verksamhetsårets början ska finnas en summa på \SI{60000}{kr}.
    
    Målet för \textbf{olycksfonden} är att det vid verksamhetsårets början ska finnas en summa på \SI{100000}{kr}.
    
    Målet för \textbf{eget kapital} är att det vid verksamhetsårets början ska finnas en summa på \SI{400000}{kr}.
    
    Målet för \textbf{jubileumsfonden} är att det vid verksamhetsårets början ska finnas en summa på \SI{50000}{kr}, d.v.s. att man varje år avsätter ungefär \SI{10000}{kr}.
    
    Resterande medel läggs på \textbf{utrustningsfonden}. Vid större uttag ur utrustningsfonden bör de kommande åren ha budget för att bygga upp kapital i fonden igen.
    
    \begin{table}[H]
    \begin{center}
    \begin{tabularx}{0.9\textwidth}{X r r r c}
        & \textbf{2018} & \textbf{2019} & \textbf{2020} & \textbf{Prioritet} \\
        \hline
        Dispositionsfonden & \SI{60 000}{kr} & \SI{60 000}{kr} & \SI{60 000}{kr} & 3 \\
        Olycksfonden & \SI{100 000}{kr} & \SI{100 000}{kr} & \SI{100 000}{kr} & 1 \\
        Eget kapital & \SI{400 000}{kr} & \SI{400 000}{kr} & \SI{400 000}{kr} & 2 \\
        Utrustningsfonden & & & & 5 \\
        Jubileumsfonden & \SI{10 000}{kr} & \SI{20 000}{kr} & \SI{30 000}{kr} & 4 \\
        Mål för årsresultat & \SI{80 000}{kr} & \SI{70 000}{kr} & \SI{80 000}{kr} & \\
    \end{tabularx}
    \end{center}
    \end{table}
    
    \newpage
    \end{document}
    