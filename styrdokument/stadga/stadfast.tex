\documentclass[10pt]{article}
\usepackage[utf8]{inputenc}
\usepackage[swedish]{babel}
\usepackage[normalem]{ulem}

\def\doctitle{Stadfästande av stadgaändringar}
\def\docauthor{Förnamn Efternamn}

\usepackage{../../e-sek}

\begin{document}
\firstpage{Stadfästande för E-sektionen inom TLTH}
\newpage

\section*{Stadfästande av stadgeändringar: HT18}
Under höstterminen 2018 bifölls två förslag på stadgeändring (i andra läsningen). Motionerna finns bifogade nedan och de gamla samt de nya (med markeringar) stadgarna bifogas i en bilaga.

\subsection*{Ändring 1: Motion om Redaktionella organ}
Sektionen beslutade under höstterminsmötet 2018 att stryka ``Kapitel 11 - Redaktionella organ'' från stadgan. Motivering för denna ändring är att kapitlet är utdaterat. Sektionsmötet ska inte behöva bestämma över allt som styrelsen publicerar, exempelvis på Instagram. \textit{Se bilaga för ytterligare underlag.}  


\subsection*{Ändring 2: Motion om Vice-poster som styrelsens suppleanter}
Under samma sektionsmöte som behandlade ändringen om Redaktionella organ beslutade mötet att bifalla en motion gällande sektionens vice-poster. Motiveringen för motionen är tvådelad. Dels handlar det om att förtydliga sektionens Vice-poster, d.v.s specificera vilka poster som faller under rollen Vice. Den andra delen är för att lyfta fram vice-posten mer och föra den närmre styrelsen. \textit{Se bilaga för ytterligare underlag.} Följande ändringar gjordes:
\begin{itemize}
    \item \hl{\S 8.4 Vice suppleanter De poster som i reglementet uttryckligen beskrivs som en vice är suppleanter till sina respektive utskottsordförande i styrelsen. En suppleant kan, efter respektive utskottsordförandes skriftliga godkännande till Ordföranden, ställföreträda utskottsordföranden vid ett Styrelse- möte med yttrande-, yrkande- och rösträtt.}

    \hl{För att en suppleant ska få ställföreträda sin respektive utskottsordförande måste denne uppfylla kraven beskrivna under \S8:3.}

    \item \S8.6 Beslut \sout{Styrelsen är beslutsmässig om minst sex (6) ledamöter är närvarande.}

    \hl{Styrelsen är beslutsmässig om minst sex (6) ordinarie ledamöter är närvarande.}

    \item \S8.2 Sammansättning \sout{Styrelsen utser inom sig en vice ordförande.}

    \hl{Styrelsen utser inom sig en Vice Ordförande till att ställföreträda Ordföranden vid situationer
    då Ordföranden inte längre har möjligheten att fullfölja sitt uppdrag. Vice Ordförande har
    också till uppgift att agera styrelsemötesordförande då Ordföranden inte kan närvara.}

    \item \S8.8 Ständigt adjungerade 
    \begin{alphlist}
        \item Styrelsens ledamöter,
        \item \hl{Styrelsens suppleanter enligt \S8:4,}
        \item Revisorerna,
        \item ...
    \end{alphlist}

    \item \S8.9 Kallese 
    \begin{alphlist}
        \item Styrelsens ledamöter,
        \item \hl{Styrelsens suppleanter enligt \S8:4,}
        \item Revisorerna,
        \item ...
    \end{alphlist}

    \item \S8.11 Skyldigheter
    \begin{attlist}
        \item ...
        \item handlägga fyllnadsval i enlighet med §12:3,
        \item \hl{hålla styrelsens suppleanter välinformerade om styrelsens och Sektionens verksamhet,} samt
        \item även i övrigt på alla sätt verka för sektionens bästa.
    \end{attlist}
\end{itemize}

Jag yrkrar på 
\begin{attlist}
    \item ni stadfäster ändringarna enligt ovan i sektionens stadgar.
\end{attlist}

\begin{signatures}{1}
    \emph{I sektionens tjänst}
    \signature{\docauthor}{Ordförande 2018}
\end{signatures}

\end{document}