\documentclass[10pt]{article}
    \usepackage[utf8]{inputenc}
    \usepackage[swedish]{babel}
    
    \def\year{2019}
    \def\antagen{2018-11-22}
    
    \usepackage{./e-vp}
    \usepackage{../e-styrdok}
    \usepackage{../../e-sek}
    
    \begin{document}
    \section*{\doctitle}
    Syftet med verksamhetsplanen är att skapa en tydligare struktur och mer operativt långsiktigt arbete. Verksamhetsplanen innehåller övergripande mål för Sektionen och delmål för styrelsen och Sektionens utskott för nästkommande år. Verksamhetens mål skall redovisas på varje terminsmöte.
    
    \emph{Budgeten och verksamhetsplanen ska komplettera varandra.}
    
    \subsubsection*{E-sektionen}
    \emph{E-sektionen ska verka för att:}
    \begin{dashlist}
        \item Främja medlemmarnas studietid. 
        \item Alla medlemmar trivs.
        \item Inkludera alla medlemmar. 
        \item Sektionens verksamhet präglas av demokrati, jämlikhet och transparens.
        \item Ha en ständigt framåtsträvande verksamhet. 
        \item Varje funktionärspost har ett ansvar och kan göra skillnad. 
        \item Hålla en god och ansvarsfull ekonomi. 
        \item Anpassa Sektionens verksamhet efter medlemmarnas behov.
        \item Integrera de internationella studenterna i sektionens verksamhet.
    \end{dashlist}
    
    \subsubsection*{Styrelsen}
    Styrelsen ska arbeta för att göra medlemmarnas studietid så bra som möjligt. Mer specifikt innebär detta förutom att vara ansvariga för den dagliga verksamheten att arbeta med mer långsiktiga frågor som till exempel lokaler och ekonomi. 
    
    \emph{Delmål 2019:}
    \begin{dashlist}
        \item Hjälpa CM att driva caféet ideellt. 
        \item Fortsätta jobba för att involvera Sektionen och dess medlemmar mer i arbetet med Teknikfokus. 
        \item Sprida information om friheten man har som medlem att genomdriva egna projekt, t.ex. genom att äska från styrelsen eller sektionsmötet och att bli projektfunktionär. 
        \item Se över och utvärdera Sektionens funktionärsposter. 
        \item Under året se över Sektionens kostnader och intäkter. 
        \item Att försöka ha mer utbyte av evenemang mellan sektionerna genom att delta aktivt i kollegierna och att marknadsföra våra event bättre utåt. 
        \item Hålla god kontakt med våra vänstyrelser på andra universitet.
        \item Arbeta aktivt för att följa GDPR.
        \item Ta ansvar för att utbilda sina efterträdare, både post-specifikt och för styrelsen.
        \item Panta och källsortera mera \scalebox{0.5}{\recycle}
    \end{dashlist}
    
    \subsubsection*{Cafémästeriet}
    
    \emph{Delmål 2019:}
    \begin{dashlist}
        \item Uppdatera arbetsbeskrivningar och regler vid behov. 
        \item Utvärdera försäljningspriserna för de olika varorna i sortimentet. 
        \item Minska arbetsbelastningen för Cafémästare, Vice och Inköps- och lagerchefer. 
        \item Utvärdera posterna och dess beskrivningar. 
        \item Jobba för att minska svinn både i LED och i Cafémästeriets förråd.
        \item Fortsätta med att kontinuerligt se över cafets ekonomi.
        \item Fortsätta jobba för god hygien och arbetsmiljö i köket.   
    \end{dashlist}
    
    \subsubsection*{Sexmästeriet}
    Sexmästeriet ska under året anordna prisvärda sittningar och evenemang av hög kvalité för sektionens medlemmar. 
    
    \emph{Delmål 2019:}
    \begin{dashlist}
        \item Att arrangera evenemang även under andra delar om året än nollningen. 
        \item Arbeta för att hålla fortsatt god ordning i Sexmästeriets förråd. 
        \item Jobba för att minimera svinn i alkohollagret.
    \end{dashlist}
    
    \subsubsection*{Näringslivsutskottet}
    Näringslivsutskottet knyter samman sektionens medlemmar med näringslivet. Utskottet bör anordna en god blandning aktiviteter som på olika sätt främjar sektionens medlemmars chanser inför arbetslivet. Utskottet har också i uppgift att dra in pengar till sektionen via olika evenemang eller sponsring. 
    
    \emph{Delmål 2019:}
    \begin{dashlist}
        \item Att jobba för att ha mer evenemang relaterade till medicinsk teknik. 
        \item Bibehålla samarbetet med befintliga företag samt aktivt söka nya samarbeten med organisationer och företag som är relevanta för sektionens medlemmar. 
        \item Se över prissättning och syfte med aktiviteter för att nå en god blandning av vinstdrivande och icke-vinstdrivande aktiviteter.
        \item Jobba för att integrera alumniverksamheten i utskottet
    \end{dashlist}
    
    \subsubsection*{Förvaltningsutskottet}
    Sektionens medlemmar ska ha tillgång till fräscha uppehållslokaler ämnade både för studier och studiesociala aktiviteter. I nuläget prioriteras ekonomin alltid högre än Sektionens lokaler. Målet ska vara att se till att underhåll av Edekvata görs löpande utan att ekonomin prioriteras ned. 
    
    \emph{Delmål 2019:}
    \begin{dashlist}
        \item Jobba för att få Vice Förvaltningschefen och Hustomtarna att jobba för en löpande framtidsplanering och underhåll av lokalerna. 
        \item Jobba för att effektivisera Sektionens bokföring för att göra den mindre tidskrävande och mer intuitiv. 
        \item Jobba för att effektivt utbilda berörda funktionärer i Sektionens ekonomi och bokföring.
    \end{dashlist}
    
    \newpage
    
    \subsubsection*{Informationsutskottet}
    Informationsutskottet har till uppgift att ansvara för informationsspridningen på Sektionen och Sektionens tekniska utrustning. Utskottet är ett spritt utskott som har funktionärer med många olika arbetsuppgifter. 2016 infördes en ny post - Vice Kontaktor - som infördes för att ha en person som jobbar med att knyta samman utskottet. Därför bör utskottet fortsätta att jobba med att komma igång med sin nya sammansättning för att få bättre sammanhållning i utskottet.
    
    \emph{Delmål 2019:}
    \begin{dashlist}
        \item Få igång Vice Kontaktorns samarbete med Kontaktorn och resten av utskottet.
        \item Arbeta för fortsatt god informationsspridning genom kontinuerlig utvärdering av informationskanalerna.
        \item Jobba för att hålla Sektionens datorsystem och hemsida uppdaterade.
        \item Utvärdera skicket på Sektionens tekniska utrustning för att se om något behöver bytas ut i förebyggande syfte.
    \end{dashlist}
    
    \subsubsection*{Källarmästeriet}
    Källarmästeriet bör jobba för att gillena ska vara fortsatt attraktiva för teknologer. Utskottet bör också fortsätta marknadsföra gillena till övriga Teknologkåren. 
    
    \emph{Delmål 2019:}
    \begin{dashlist}
        \item Försöka locka fler teknologer till att komma på gillena. 
        \item Uppmuntra samarbete såväl inom Sektionen som med andra sektioner. 
        \item Jobba för att fortsatt ha ett bra och konkurrenskraftigt pris på mat och dryck. 
        \item Jobba för att få så lite svinn som möjligt i alkohollagret.     
    \end{dashlist}
    
    \subsubsection*{Nöjesutskottet}
    Nöjesutskottet bör arbeta för att bidra till en gemenskap på sektionen genom att ge sektionens medlemmar en chans att delta på evenemang efter skoltid organiserade av nöjesutskottet. Det bör också fortsätta arbeta med att försöka skapa evenemang över sektionerna för att bidra till gemenskapen på campus. Nöjesutskottet ska fortsätta försöka integrera ett stort urval medlemmar genom att kontinuerligt erbjuda differentierade aktiviteter och evenemang.
    
    \emph{Delmål 2019:}
    \begin{dashlist}
        \item Att utskottet regelbundet genomför evenemang för sektionen. 
        \item Fortsätta med att jobba att erbjuda olika sorters evenemang för sektionens medlemmar.
        \item Arbeta tillsammans med andra sektioner för att genomföra intersektionella evenemang. 
    \end{dashlist}
    
    \newpage
    
    \subsubsection*{Nolleutskottet}
    NollUs mål är att alla nyantagna studenter på E-sektionen ska få ett så bra mottagande som möjligt. Detta ska göras genom att arrangera en mängd olika studiesociala aktiviteter där det ska finnas något för alla. Mottagningen ska även ge en så positiv attityd till studier som möjligt med hjälp av phaddrar och äldre studenter.
    
    \emph{Delmål 2019:}
    \begin{dashlist}
        \item Arbeta för att integrera internationella studenter ännu mer i Sektionen.
        \item Arbeta för att det ska fortsätta finnas en mångfald av aktiviteter.
        \item Arbeta för att få fram förslag och förbättringar kring nollningen från Sektionens medlemmar.
        \item Arbeta för att förmedla en positiv attityd till studierna.
        \item Utvärdera arbetsbördan på utskottets funktionärer.
        \item Arbeta för att informera mer om Sektionens utskott redan under nollningen.        
    \end{dashlist}
    
    \subsubsection*{Studierådet}
    Studierådet jobbar för att Sektionens medlemmar ska ha en god studiemiljö och utskottet ska jobba aktivt med studiebevakning. Arbetet skall göras mer tillgängligt för sektionens medlemmar för att visa förändringar som har genomförts. Utskottet ska jobba transparent och ska verka för att medlemmarna är medvetna om sina möjligheter att påverka och förbättra sin utbildning. 
    
    \emph{Delmål 2019:}
    \begin{dashlist}
        \item Arbeta för att synliggöra utskottets arbete och resultat till sektionens medlemmar. 
        \item Anordna och utvärdera studierelaterade evengemang kontinuerligt under hela året.
        \item Arbeta för att ha minst en representant från varje årskurs, inklusive årskurs fyra och fem. 
        \item Arbeta för att öka svarsfrekvensen på CEQ-enkäterna.        
    \end{dashlist}
    
    \newpage
    \end{document}
    