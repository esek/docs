\documentclass[10pt]{article}
\usepackage[utf8]{inputenc}
\usepackage[swedish]{babel}
\usepackage[normalem]{ulem}

\def\doctitle{Stadfästande av sektionens stadgar}
\def\docauthor{Edvard Carlsson}
\def\date{2019-12-28}

\usepackage{../../e-sek}

\begin{document}
    \firstpage{Stadfästande för E-sektionen inom TLTH}
    \newpage

    \subsection*{Stadfästande av stadgeändringar: Vårterminsmötet 2019}
        Under höstterminsmötet 2019 bifölls ett förslag på stadgeändring i andra läsningen. Motionen samt de nya och gamla stadgarna finns bifogade som bilagor i dessa handlingar.

    \subsubsection*{Ändring: Ge styrelsen rättigheter att genomföra redaktionella ändringar i styrdokumenten}
      Sektionen valde att ge styrelsen rättigheter att ändra språkliga och typografiska fel i styrdokumenten. I våra styrdokument finns det massvis med små fel, det är punker mitt i meningar, en del felstavade ord samt vilt blandade av versaler och gemener. Antaglighen har inte styrdokumenten antagits på detta vis men eftersom det är svårt att påvisa i samtliga fall anser vi denna lösningar vara den bästa. Jag kan ytterligare tillägga att vårt reglemete nu kräver styrelsen att vid varje terminsmöte redovisa gjorda redaktionella ändringar av styrdokumenten sedan det förra terminsmötet. Följande ändringar i stadgans kapitel 8, 14 och 15 föreslås: 
    
    \begin{itemize}

    \item i Kapitel 8 - Styrelsen lägga till

    \subsubsection*{§8:14 Redaktionella rättigheter}
    
    Styrelsen innehar rätten att genomföra redaktionella ändringar som rättar språkliga eller typografiska fel bland styrdokumenten i det fall då styrelsen är enhetlig i beslutet. 

    \item i  Kapitel 14 - Stadgarna under §14:2 Ändring lägga till 
    
    Styrelsen innehar rätten att genomföra redaktionella ändringar i Stadgarna i enlighet med §8:14.
    
    Efter genomförd ändring ska tidigare publicerade versioner av Stadgarna bevaras på sektionen så att spårbarhet finns.

    \item i  Kapitel 15 - Reglemente under §15:4 Ändring lägga till 
    
    Styrelsen innehar rätten att genomföra redaktionella ändringar i Reglementet i enlighet med §8:14.
    
    Efter genomförd ändring ska tidigare publicerade versioner av Reglementet bevaras på sektionen så att spårbarhet finns.
 
      \end{itemize}

    Jag yrkar 
    \begin{attsatser}
    \att ni stadfäster ändringarna enligt ovan i sektionens stadgar. \newline
    \end{attsatser}

\begin{signatures}{1}
    \emph{I E-sektionens tjänst}
    \signature{\docauthor}{Ordförande 2019}
\end{signatures}

\end{document}