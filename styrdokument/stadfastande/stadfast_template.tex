\documentclass[10pt]{article}
\usepackage[utf8]{inputenc}
\usepackage[swedish]{babel}
\usepackage[normalem]{ulem}

\def\doctitle{Stadfästande av stadgaändringar}
\def\docauthor{Förnamn efternamn}

\usepackage{../../e-sek}

\begin{document}
    \firstpage{Stadfästande för E-sektionen inom TLTH}
    \newpage

    \section*{Stadfästande av stadgeändringar: HTxx}
    Under höstterminen 20xx bifölls två förslag på stadgeändring (i andra läsningen). Motionerna finns bifogade nedan och de gamla samt de nya (med markeringar) stadgarna bifogas i en bilaga.

    \subsection*{Ändring 1: Lorem ipsum}
    Lorem ipsum dolor sit amet, consectetur adipiscing elit, sed do eiusmod tempor incididunt ut labore et dolore magna aliqua. Ut enim ad minim veniam, quis nostrud exercitation ullamco laboris nisi ut aliquip ex ea commodo consequat. Duis aute irure dolor in reprehenderit in voluptate velit esse cillum dolore eu fugiat nulla pariatur. Excepteur sint occaecat cupidatat non proident, sunt in culpa qui officia deserunt mollit anim id est laborum.

    \subsection*{Ändring 2: Lorem ipsum}
    Lorem ipsum dolor sit amet, consectetur adipiscing elit, sed do eiusmod tempor incididunt ut labore et dolore magna aliqua. Ut enim ad minim veniam, quis nostrud exercitation ullamco laboris nisi ut aliquip ex ea commodo consequat. Duis aute irure dolor in reprehenderit in voluptate velit esse cillum dolore eu fugiat nulla pariatur. Excepteur sint occaecat cupidatat non proident, sunt in culpa qui officia deserunt mollit anim id est laborum.

    Jag yrkrar på
    \begin{attlist}
        \item ni stadfäster ändringarna enligt ovan i sektionens stadgar.
    \end{attlist}

\begin{signatures}{1}
    \emph{I sektionens tjänst}
    \signature{\docauthor}{Funktionärspost 20xx}
\end{signatures}

\end{document}