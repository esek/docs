\documentclass[10pt]{article}
\usepackage[utf8]{inputenc}
\usepackage[swedish]{babel}

\def\doctitle{Reglemente}
\def\antagen{1997-11-10}
\def\uppdaterad{2018-11-22}

\usepackage{../e-styrdok}
\usepackage{../../e-sek}

\setcounter{tocdepth}{2}
\titleformat{\section}{\bf \Large}{Kapitel {\thetitle} -- }{0mm}{}
\renewcommand{\thesubsection}{\arabic{section}:\Alph{subsection}}
\setlength\cftbeforesecskip{0em}

\begin{document}
\firstpage{Reglemente för E-sektionen inom TLTH}{\uppdaterad}
\newpage

\tableofcontents
\newpage

\section{Sektionen} %Kapitel 1

\section{Medlemmarna} %Kapitel 2
\subsection{Stödmedlemmar}
Stödmedlemmar upptas som medlem efter ansökan hos Sektionsstyrelsen och
efter det att medlemsavgift är betald. Avgiftens storlek är den samma som
Sektionsbidraget. Stödmedlem har samma skyldigheter och rättigheter som
förutvarande medlem. Medlemskap upphör då ej längre medlemsavgift betalas
och man är då ej att anses som förutvarande medlem.

\subsection{Anställda}
Anställd person av sektionen har samma skyldigheter och rättigheter som
förutvarande medlem när den anställde önskar delta i sektionsverksamhet som
går utanför arbetsuppgifterna. Dessa skyldigheter och rättigheter upphör
då anställningen upphör.

\section{Organisation} %Kapitel 3
\subsection{Allmänt}
Funktionärer inom Sektionen som ligger utanför utskotten och inte regleras i
stadgarna är underställda Sektionens Ordföranden, dessa är:

\begin{numplist}
\item Talman, samt
\item Sigillbevarare.
\end{numplist}

\section{Sektionsmöte} %Kapitel 4
\subsection{Ständigt adjungerade}
Ständigt adjungerade med yttrande- och yrkanderätt vid Sektionsmöte,
förutom de som nämns i stadgans §4:8, är

\begin{alphlist}
    \item TLTH:s inspektor, samt
    \item Talman.
\end{alphlist}

Intill dess att verksamhetsår med avseende på verksamhetsberättelse,
bokslut och revisionsberättelse har blivit avslutat på ett Sektionsmöte,
är Revisorerna för verksamhetsåret i fråga ständigt adjungerande på
Sektionsmöten.

\subsection{Sektionsmöte}
Vid varje Sektionsmöte skall följande ärenden tas upp

\begin{alphlist}
    \item val av mötesordförande (då Sektionens talman är närvarande
        väljes den automatiskt),
    \item val av mötessekreterare (då kontaktorn är närvarande väljes
        den automatiskt),
    \item tid och sätt,
    \item val av två justeringspersoner,
    \item adjungeringar,
    \item föredragningslistan,
    \item föregående Sektionsmötesprotokoll (kontroll att protokollet
        är justerat),
    \item meddelanden, samt
    \item övrigt.
\end{alphlist}

\subsection{Vårterminsmöte}
På Vårterminsmötet skall, förutom de som nämns i stadgans §4:10,
följande ärenden tas upp:
\begin{alphlist}
    \item beslutsuppföljning,
    \item utskottsrapporter (kort skriftlig redogörelse av utskottets, styrelsens och valberedningens verksamhet), samt
    \item uppföljning av verksamhetsplan (både nuvarande och avgående styrelse)
    \item ekonomisk rapport.
\end{alphlist}

\subsection{Höstterminsmöte}
På Höstterminsmötet skall, förutom de som nämns i stadgans §4:11,
följande ärenden tas upp:
\begin{alphlist}
    \item beslutsuppföljning,
    \item utskottsrapporter (kort skriftlig redogörelse av utskottets, styrelsens och valberedningens verksamhet),
   	\item uppföljning av verksamhetsplan
    \item ekonomisk rapport.
\end{alphlist}

\subsection{Kallelse}
Det åligger Sektionens Ordförande att muntligen informera studenter i årskurs 1 och 2, och som är ordinarie medlemmar i sektionen om
Sektionsmötet i samband med en föreläsning.

Det åligger utskottscheferna eller motsvarande att via E-post i god tid
informera utskottsfunktionärerna om Sektionsmötet.

\subsection{Beslutsuppföljning}
Under punkten beslutsuppföljning skall Sektionsmötet ta ställning till om
beslutet skall avföras från listan (då beslut är utfört eller ej längre
giltigt), utförandedatum skall flyttas fram eller åtgärdsansvarig skall ändras.
\subsection{Kandiderings- och rösträtt}
Personer som ej är fysiskt närvarande i lokalen vid Sektionsmötet har ej rösträtt, men får kandidera till poster på alternativa sätt, till exempel via videolänk, om Talmannen finner det lämpligt.

Motkandidering till en av valberedningen nominerad styrelsepost skall meddelas valberedningen senast 3 läsdagar före aktuellt val. Är posten vakant kan motkandidering ske direkt på mötet.

\section{Inspektorn} %Kapitel 5
\subsection{Inspektorns åtaganden}
Det åligger inspektorn

\begin{attlist}
    \item i största möjliga mån närvara på Sektionsmöte,
    \item i största möjliga mån delta vid de arrangemang som Sektionen
        inbjudit till, samt
    \item minst en gång per termin träffa Sektionens Styrelse för
        informationsutbyte.
\end{attlist}

\subsection{Sektionens åtaganden}
Det åligger Sektionen
\begin{attlist}
    \item bjuda inspektorn på skiftesgasque, nollegasque och andra större
        festligheter i Sektionens regi, samt
    \item vid större arrangemang och tävlingar inbjuda inspektorn att delta.
\end{attlist}

\section{Revision} %Kapitel 6
\subsection{Allmänt}
Under revisionen skall revisorsuppleanterna delta aktivt i arbetet.

Det åligger revisorerna att sammanställa en ekonomisk rapport inför vår-
och höstterminsmöte innehållande
\begin{alphlist}
    \item saldo på bankkonto, handkassa och dylikt,
    \item resultatrapport över vad som hitintills är bokfört, samt
    \item vilka arrangemang som kräver arrangemangsrapport som varit,
        vilka som är redovisade och hur dessa gått.
\end{alphlist}

\section{Valberedning} %Kapitel 7
\subsection{Tillvägagångssätt}
Valberedningen ansvarar för planeringen av hela valberedningsförfarandet. Det är dock Styrelsen som ansvarar för själva genomförandet av Expot och Valmötet. Officiella nomineringstiden skall vara minst 8 läsdagar, men bör vara längre. Nomineringar görs via hemsidan senast 23:59 den sista nomineringsdagen. Valberedningens förslag skall anslås senast 23:59 8 läsdagar innan mötet. Eventuella motkandidater anslås senast 23:59 2 läsdagar innan mötet.

Valberedningen kan, om de så önskar, adjungera personer till möten eller intervjuer där de finner det lämpligt, förutom då de skall besluta om nomineringar.
\subsection{Kravprofiler}
För varje post som valberedningen behandlar ska det finnas kravprofiler. De uppdateras inför varje kommande verksamhetsår av sittande valberedning och styrelse, och ska finnas öppna för att läsas på hemsidan. Det är utifrån kravprofilerna som valberedningen ska ställa relevanta frågor i sina intervjuer, och bedömda den sökandes lämplighet för posten.

En god kravprofil innehåller en beskrivning av vad posten innebär, vad som förväntas av en person på posten, och bra egenskaper för en person på posten. Dock bör inga konkreta intervjufrågor (och eventuella svar) finnas med, med undantag för då det finns formella krav på innehavaren av en post.
\subsection{Valberedningens skyldigheter}
Det åligger Valberedningens Ordförande
\begin{attlist}
    \item kalla till och leda valberedningens arbete och möten,
    \item hålla Styrelsen informerad om hur arbetet fortskrider, samt
    \item tillsammans med Styrelsen planera Expot och Valmötet.
\end{attlist}
Det åligger alla valberedningens medlemmar
\begin{attlist}
    \item inte öppet diskutera hur arbetet inom valberedningen fortlöper,
    \item bara föra vidare information som framkommer under intervjuer med kandidater till mötet för frågans avgörande, samt
    \item närvara på Sektionsmöte som innehåller val som förberetts av
        valberedningen.
\end{attlist}
Det åligger hela valberedningen
\begin{attlist}
    \item försäkra sig om att information som framkommer under intervjuer får framföras vid mötet för frågans avgörande,
    \item i slutet av verksamhetsåret, i samråd med styrelsen, uppdatera och publicera kravpro- filerna för nästa år,
    \item använda Sektionens informationskanaler för att nå ut med valinformation,
    \item muntligen informera studenter i årskurs 1 och 2, som är ordinarie medlemmar i Sektionen om funktionärsvalen i samband med
        en föreläsning,
    \item kontinuerligt offentliggöra alla inkomna nomineringar fram tills
        det att nomineringstiden gått ut,
    \item genomföra intervjuer med personer som kandiderar till Styrelsepost,
        Valberedningens Ordförande, Revisorer och
        Inspektor samt då valberedningen finner det lämpligt,
    \item i samband med att valberedningens nomineringsförslag offentliggörs,
        även offentliggöra en lista över alla inkomna nomineringar där det
        framgår hur de nominerade ställer sig till en kandidatur, samt
    \item offentliggöra officiella motkandidater fram till två dagar innan
        mötet för frågans avgörande.
\end{attlist}

\subsection{Avsägelse}
Om en ledamot i valberedningen avser att kandidera till funktionärsuppdrag
i Styrelsen skall den omgående avsäga sig sitt
uppdrag i valberedningen. Avsägelsen skall vara skriftlig och lämnas till
Styrelsen. Avsägelsen kan senast ske vid nomineringstidens
utgång. Om avsägelsen sker efter nomineringstidens utgång får ej
valberedningen nominera personen i fråga till de nämnda posterna.

\section{Styrelsen} %Kapitel 8
\subsection{Sammansättning}
Styrelsen består av
\begin{alphlist}
    \item Ordföranden -- ansvarar för Styrelsearbetet,
    \item SRE-Ordföranden -- utskottschef för studierådet,
    \item Kontaktorn -- sekreterare, utskottschef för informationsutskottet,
    \item Förvaltningschefen -- kassör, utskottschef för förvaltningsutskottet,
    \item Øverphøs -- utskottschef för nolleutskottet,
    \item Entertainern -- utskottschef för nöjesutskottet,
    \item Cafémästaren -- utskottschef för cafémästeriet,
    \item Krögaren -- utskottschef för källarmästeriet,
    \item Sexmästaren -- utskottschef för sexmästeriet, samt
    \item Ordförande för näringslivsutskottet, ENU
        -- utskottschef för näringslivsutskottet.
\end{alphlist}
\subsection{Ständigt adjungerade}
Ständigt adjungerade med yttrande- och yrkanderätt vid Styrelsemöte är,
förutom de som nämns i Stadgans §8:8,
\begin{alphlist}
    \item Chefredaktör för HeHE,
    \item Valberedningens Ordförande,
    \item Skattmästaren,
    \item Inköps- och lagerchef,
    \item Hustomte,
    \item Fullmäktigeledamöter från E-sektionen,
    \item Övriga enligt Styrelsens bedömning,
    \item Talman,
    \item Datatekniksektionens Ordförande,
    \item Ordförande för ElektroTekniska föreningen, samt
    \item Ordförande för Elektra.
\end{alphlist}
\subsection{Kallelse}

Kallelse till Styrelsesammanträde samt föredragningslista skall senast tre
läsdagar före sammanträdet, förutom de som nämns i Stadgans §8:9, tillställas
\begin{alphlist}
    \item Chefredaktör för HeHE,
    \item Valberedningens Ordförande,
    \item Skattmästaren,
    \item Inköps- och lagerchef,
    \item Hustomte,
    \item Fullmäktigeledamöter från E-sektionen,
    \item Övriga enligt Styrelsens bedömning,
    \item Talman,
    \item Datatekniksektionens Ordförande,
    \item Ordförande för ElektroTekniska föreningen, samt
    \item Ordförande för Elektra.
\end{alphlist}
\subsection{Skyldigheter}

Förutom vad som nämns i Stadgans §8:11 åligger det Styrelsen
\begin{attlist}
    \item sköta Sektionens postgång. Detta innebär att distribuera inkommande
        post till rätt destination, samt i förekommande fall hämta post för
        Sektionens räkning, samt
    \item uprätta en verksamhetsplan för nästkommande verksamhetsår.
\end{attlist}

\renewcommand*\thesubsection{\arabic{section}:\arabic{subsection}}
\renewcommand*\thesubsubsection
{\arabic{section}:\arabic{subsection}:\Alph{subsubsection}}

\section{Utskott} %Kapitel 9
\subsection{Allmänt}
\subsubsection{Skyldigheter}
Det åligger utskotten
\begin{attlist}
    \item i första hand ägna sig åt den verksamhet som faller innanför
        utskottsbeskrivningens ramar,
    \item arrangera och genomföra skiftet,
    \item under Expot presentera utskottets verksamhet och dess
        funktionärsposter,
    \item i övrigt verka för en levande Sektion och för Sektionens bästa,
    \item lämna uppdaterad information ämnad för hemsidan till Kodhackare, samt
    \item varje år ordna minst en omsitts för tidigare och nuvarande funktionärer.
    \item bistå med assistans i LED-café under de läsveckor som av styrelsen tilldelats utskottet.
\end{attlist}

\subsection{Utskottsbeskrivningar}
\subsubsection{Förvaltningsutskottet, FVU}
Förvaltningsutskottet har till uppgift att arbeta och sköta Sektionens administrativa arbete, lokaler, inventarier och bevara Sektionens historia.

Det åligger utskottet att
\begin{tightdashlist}
\item förvalta Sektionens ekonomi, bokföring och juridiska dokument.
\item förvalta Sektionens lokaler, inventarier och fordon.
\item förvalta Sektionens arkiv.
\item ansvara för E-shops försäljning, lagerföring och inköp av Sektionens reklamprodukter såsom ouveraller, märken, pins, sångböcker, tröjor m.m..
\end{tightdashlist}

\subsubsection{Informationsutskottet, InfU}
Informationsutskottet har till uppgift att ansvara för informationsspridningen på Sektionen och Sektionens tekniska utrustning.
Det åligger utskottet att
\begin{tightdashlist}
    \item se till att Sektionens hemsida fungerar bra och har uppdaterad information.
    \item se till att Sektionens tekniska utrustning fungerar.
    \item skriva och publicera nollEguiden och HeHE.
    \item under nollningen utbilda nyantagna studenter i LTH:s och Sektionens datorsystem.
\end{tightdashlist}

\subsubsection{Näringslivsutskottet, ENU}
Näringslivsutskottet har till uppgift att vara kopplingen mellan näringslivet och Sektionens medlemmar samt ansvara för Sektionens Alumniverksamhet.

Det åligger utskottet att
\begin{tightdashlist}
\item anordna aktiviteter som främjar Sektionens medlemmar inför arbetslivet.
\item vårda och utveckla samarbeten med företag.
\item tillgodose Sektionens behov av sponsring.
\item anordna en arbetsmarknadsmässa för Sektionens medlemmar.
\item hålla ständig kontakt med övriga sektioners motsvarande utskott.
\item arrangera minst ett alumnievenemang under året.
\end{tightdashlist}

\subsubsection{Nöjesutskottet, NöjU}
Nöjesutskottet har till uppgift att arrangera nöjes-, fritids- och idrottsaktiviteter för att tillgodose medlemmarnas behov. Utskottet ska se till att det finns ett varierat utbud av aktiviteter vilket kan innebära allt från till exempel spel- och filmkvällar till glassförsäljning och idrottsturneringar.

Det åligger utskottet att
\begin{tightdashlist}
\item arrangera UtEDischot tillsammans med D-sektionen.
\item organisera Sektionens bidrag till Sångarstriden.
\item organisera Sektions deltagande i Tandemstafetten.
\item arrangera idrottsaktiviteter för Sektionens medlemmar.
\item arrangera andra roliga, nöjesarrangemang så att medlemmarnas behov tillgodoses.
\end{tightdashlist}

\subsubsection{Källarmästeriet, KM}
Källarmästeriet har till uppgift att tillgodose Sektionen med gillen samt att sköta øl- och spritförrådet.

Det åligger utskottet att
\begin{tightdashlist}
\item regelbundet arrangera gillen under läsperioderna.
\item arrangera gillen under nollningsperioden.
\item arrangera ett Julgille med julmat i december månad.
\item sköta inköp samt lagerhållning av dryck avsett för utskottets verksamhet.
\item löpande kontrollera vinstmarginaler på øl- och spritförrådets varor.
\end{tightdashlist}

\subsubsection{Cafémästeriet, CM}
Cafémästeriet har till uppgift att tillgodose Sektionens medlemmar med möjligheten att handla enklare mat och dryck för studentvänliga priser, samtidigt som konkurrenskraft gentemot andra caféer bibehålls.

Det åligger utskottet att
\begin{tightdashlist}
\item sköta driften av LED-café.
\item ansvara för Sektionens inköp och beställningar till LED-café.
\item hålla ordning och sköta inventering av inköpta varor.
\item hantera den ekonomiska uppföljningen av samtlig försäljning i LED-café genom löpande kontroll av inköpspris samt redovisning för att upptäcka resultatförändringar.
\end{tightdashlist}

\subsubsection{Sexmästeriet, E6}
Sexmästeriet har som huvudsaklig uppgift att tillgodose Sektionen med festarrangemang.

Det åligger utskottet att
\begin{tightdashlist}
\item under nollningen arrangera sittningar med fokus på de nyantagna, däribland en Vett- \& Etikettsittning.
\item arrangera en Nollegasque i form av en bal i anslutning till Nollningen.
\item arrangera minst två evenemang under vårterminen och två under höstterminen som är öppna för alla Sektionens medlemmar.
\item verka för att evenemang med andra sektioner hålls, för att främja intersektionella relationer.
\end{tightdashlist}

\subsubsection{Nolleutskottet, NollU}
Nolleutskottet har till uppgift att arrangera mottagandet av nyantagna studenter.

Det åligger utskottet att
\begin{tightdashlist}
    \item rekrytera, utbilda och organisera phaddrar till nollningen.
    \item arrangera nollningsaktiviteter som får de nyantagna att känna sig välkomna till Sektionen.
    \item i samråd med studievägledningen och Studierådet arrangera nollningsaktiviteter som främjar studierna vid högskolan.
    \item inom utskottet utse en ekonomiansvarig.
\end{tightdashlist}

\subsubsection{Studierådet, SRE}
Studierådet har till uppgift att utföra studiebevakning för Sektionens medlemmar. Detta innebär att genomföra kursutvärderingar och att i övrigt arbeta för en bättre studiesituation.

Det åligger utskottet att
\begin{tightdashlist}
\item arrangera studiefrämjande aktiviteter.
\item föra sektionsmedlemmarnas talan i programledningar och instutionsstyrelser.
\item utföra CEQ-censurering samt deltaga på tillhörande möten.
\end{tightdashlist}

\section{Funktionärer} %Kapitel 10
\subsection{Allmänt}
\subsubsection{Funktionärers rättigheter}
Under mandattiden har en funktionär rätt till
\begin{attlist}
    \item deltaga på Skiftesgasque,
    \item i den mån det går, utnyttja Sektionens resurser för en så låg
        kostnad som möjligt, samt
    \item i mån av behov, kvittera ut nyckel mot depositionsavgift.
    \item funktionärskaffe får enbart drickas ur sektionens porslinsmuggar eller egen medhavd mugg / kopp / dryckeskärl etc.
\end{attlist}

Efter avslutad mandattid, om man har utfört sin funktionärssyssla på ett
tillfredsställande sätt, har man rätt till
\begin{attlist}
    \item delta på närmast efterföljande skifte,
    \item erhålla medalj, eventuellt mot en mindre avgift, samt
    \item på begäran, erhålla ett funktionärsintyg på de förtroendeposter
        man innehaft för Sektionen.
\end{attlist}

Styrelsen avgör, efter förslag från utskottschefen, om en funktionär har fullgjort sitt uppdrag eller ej. Valberedningens Ordförande avgör, om en funktionär i valberedningen, har fullgjort sitt uppdrag eller ej. Sigillbevararen avgör om en funktionär som endast kan utses av Sektionsmöte, har fullgjort sitt uppdrag eller ej.

\subsubsection{Avsägelser under mandatperioden}
Funktionärer som avbryter sitt uppdrag under mandatperioden pga.
\begin{alphlist}
    \item värnpliktstjänstgöring,
    \item utlandsstudier, eller
    \item andra förtroendeuppdrag inom Teknologkåren.
\end{alphlist}
har rätt att medverka på kommande Skiftesgasque under förutsättning
\begin{attlist}
    \item avsägelsen skriftligen har meddelats Styrelsen innan den träder
        i kraft,
    \item man har fullgjort sitt uppdrag fram tills dess att avsägelsen
        träder i kraft,
    \item sysslan har inneburit att man har påbörjat sitt uppdrag, samt
    \item man i övrigt fullgjort sina plikter mot Sektionen.
\end{attlist}

För att erhålla medalj under dessa omständigheter krävs att man fullgjort
sin syssla under minst två läsperioder.

\subsection{Funktionärsbeskrivningar}
\subsubsection{Allmänna funktionärsåtaganden}
Det åligger en funktionär
\begin{attlist}
    \item utföra de sysslor som faller inom funktionärsbeskrivningens ram,
    \item ägna sig åt den verksamhet som faller inom utskottsbeskrivningens ram,
    \item framföra idéer till arrangemang och förbättringar till Styrelsen, samt
    \item i mån av tid bistå med assistans i LED-café när detta behövs.
\end{attlist}

\subsubsection{Förklaringar}
Antalet funktionärer på varje funktionärspost anges i parentesen efter funktionärspostens namn.
\begin{itemize}
    \item[(u)] betyder utskottschef eller motsvarande och är endast en person,
    \item[($n$)] $n \in \mathbb{N}$, betyder upp till $n$ stycken,
    \item[(exakt $n$)] $n \in \mathbb{N}$, betyder exakt $n$ stycken, samt
    \item[(e.a)] betyder erforderligt antal. Det innebär att styrelsen i samråd med respektive utskottsordförande bestämmer ett antal som passar utskottets verksamhet. Dessa poster väljs inte på valmötet utan av styrelsen med rekommendation av utskottsordförande innan vårterminens start.
\end{itemize}
Mandattiden för funktionärerna är kalenderår om inte annat står angivet.

\subsubsection{Ordförande}
Det åligger Ordföranden
\begin{attlist}
    \item representera Sektionen och föra dess talan,
    \item sammankalla handlingar till Sektionsmöte,
    \item tillsammans med Talmannen upprätta lämplig föredragningslista,
    \item muntligen informera studenter i årskurs 1 och 2, som är ordinarie medlemmar i sektionen om Sektionsmötet i samband med en föreläsning,
    \item tillsammans med Valberedningens Ordförande planera Expot och valmötet,
    \item sammankalla och upprätta lämpliga handlingar till Styrelsesammanträden,
    \item leda Styrelsesammanträdena och leda arbetet i Styrelsen,
    \item organisera Kurs på landet,
    \item kontinuerligt utbyta information med Inspektor,
    \item aktivt deltaga i TLTH:s Fullmäktiges möten och föra Sektionens talan,
    \item aktivt deltaga i TLTH:s Ordförandekollegie och utbyta information mellan Sektionerna,
    \item tillse att Sektionen är representerad vid TLTH:s kårbal och andra Sektioners högtidligheter,
    \item närvara vid Sektionens Nollegasque, skifte och Sektionsmöte,
    \item städa efter skiphtet, samt
    \item ansvara för att det hålls en omsitts för Ordförandeposten och en för posterna Talman, Sigillbevarare och Revisorer gemensamt.
\end{attlist}

\subsubsection{UtskottsOrdföranderna}
Det åligger utskottsOrdföranderna
\begin{attlist}
    \item leda arbetet i utskotten,
    \item aktivt deltaga på Sektionsmöten,
    \item sörja för att mat och dryck finns tillgängligt under Terminsmötena,
    \item via E-post i god tid informera utskottsfunktionärerna om Sektionsmötet,
    \item organisera Sektionens skifte och Skiftesgasquen,
    \item organisera Kurs på Landet,
    \item organisera Expot,
    \item hålla ordning i blå dörren och hamnkontoret,
    \item i förekommande fall aktivt delta i TLTH:s kollegier,
    \item städa efter skiftet,
    \item fortlöpande informera om utskottets finansiella läge, samt
    \item ansvara att det hålls en omsitts för utskottet och en omsitts för den egna posten.
\end{attlist}

\subsubsection{Funktionärer i Förvaltningsutskottet, FVU}
\begin{emptylist}
    \item Förvaltningschefen (u)
        \begin{dashlist}
            \item har det övergripande ansvaret för Sektionens ekonomi,
                lokaler och inventarier och vad därmed äga sammanhang.
        \end{dashlist}
    \item Vice Förvaltningschef (1)
        \begin{dashlist}
            \item Denna post är en vice till utskottsordföranden.
            \item bistår Förvaltningschefen i dennes arbete gällande Sektionens lokaler och inventarier,
            \item ansvarar för lokalbokningar, samt
            \item ansvarar för att leda, dokumentera och redovisa Hustomtarnas arbete.
        \end{dashlist}
    \item Skattmästaren (1)
        \begin{dashlist}
            \item Denna post är en vice till utskottsordöranden.
            \item bistår Förvaltningschefen i dess arbete gällande Sektionens ekonomi,
            \item ansvarar för Sektionens bokföring tillsammans med Förvaltningschefen, samt
            \item utför arkivering av bokföringsunderlaget.
        \end{dashlist}
    \item Arkivarie (2)
        \begin{dashlist}
            \item ansvarar för arkivering av Sektionens dokument och
                klenoder,
            \item ansvarar för ordningen i arkivet och på arkivdelen
                på hyllan i förrådet i EKEA.
        \end{dashlist}
    \item Hustomte (3)
        \begin{dashlist}
            \item har till uppdrag att arbeta med underhåll och framtida utformning  av våra lokaler och dess inventarier, samt
            \item jobba för trivseln i E-sektionens lokaler.
        \end{dashlist}
    \item Husstyrelserepresentant (1)
        \begin{dashlist}
            \item för sektionens talan i E-husets styrelse.
        \end{dashlist}
    \item Ekiperingsexperter (2)
        \begin{dashlist}
            \item sköter om inköp, lagerhållning och försäljning av PR- artiklar och sångböcker.
            \item ansvarar för redovisningen av försäljningen till Förvaltningschefen.
        \end{dashlist}
\end{emptylist}
\subsubsection{Funktionärerna i Informationsutskottet, InfU}
\begin{emptylist}
    \item Kontaktor (u)
        \begin{dashlist}
            \item är Sektionens sekreterare och har övergripande ansvar för Sektionens dokument och protokollföring av möten.
            \item ansvarar för att Sektionens stydokument hålls aktuella.
            \item ansvarar för att upprätta handlingar till Sektionsmötena.
            \item har det övergripande ansvaret för Sektionens informationsspridning och PR-verksamhet.
            \item ansvarar för Sektionens kontakt med andra sektioner, såväl inom TLTH som på andra högskolor och universitet.
            \item är ansvarig utgivare för HeHE.
        \end{dashlist}
    \item Vice Kontaktor (1)
        \begin{dashlist}
            \item Denna post är en vice till utskottsordöranden.
            \item bistår ordförande för Informationsutskottet i dennes arbete med utskottet.
            \item hjälper till att hålla informationen på våra hemsidor aktuell.
        \end{dashlist}
    \item Macapär (2)
        \begin{dashlist}
            \item ansvarar för Sektionens datorutrustning samt hemsida,
            \item organiserar utveckling av sektionens programvara,
            \item ansvarar för att utskottscheferna har adekvat utrustning
                för sitt arbete inom området,
            \item alternativ titulering för Macapären är Pingvin, dock
                med kravet att denne måste bära frack.
        \end{dashlist}
	\item Kodhackare (e.a)
        \begin{dashlist}
            \item hjälper Macapärerna med utveckling och underhåll av
                sektionens hemsida eller annat kodrelaterat,
        \end{dashlist}
    \item Chefredaktör (1)
    		\begin{dashlist}
    			\item Har det övergripande ansvaret för HeHE och vad därmed äga sammanhang.
          \item Ansvarar tillsammans med Picasso, Redaktörer och NollU för produktion av nollEguiden.
    		\end{dashlist}
    	\item Redaktörer (4)
    		\begin{dashlist}
    			\item Bistår Chefredaktören i dennes arbete med HeHE.
          \item Ansvarar tillsammans med Chefredaktör, Picasso och NollU för produktion av nollEguiden.
    		\end{dashlist}
    \item Fotograf (3)
        \begin{dashlist}
            \item ansvarar för fotografering och redigering av detta material i sektionens ändamål.
        \end{dashlist}
    \item FilmarE (1)
        \begin{dashlist}
	\item Ansvarar för filmande och redigering av detta material i sektionens ändamål.
        \end{dashlist}
    \item Picasso (2)
		\begin{dashlist}
            \item Bistår sektionen med grafik till affischer på begäran från övriga funktionärer,
            \item Bistår övriga funktionärer med vägledning då detta behövs vid skapandet av affischer, samt
            \item Hjälper ekiperingsexperten att designa, roliga, vackra, annorlunda och/eller häftiga tygmärken.
            \item Ansvarar tillsammans med Chefredaktör, Redaktörer och NollU för produktion av nollEguiden.
        \end{dashlist}
    \item Teknokrater (3)
        \begin{dashlist}
            \item ansvarar för Sektionens tekniska utrustning och ser till att denna fungerar som förväntat.
        \end{dashlist}
\end{emptylist}

\subsubsection{Funktionärerna i Näringslivsutskottet, ENU}

\begin{emptylist}
    \item Ordförande för Näringslivsutskottet (u)
        \begin{dashlist}
            \item har det övergripande ansvaret för E-sektionens
                näringslivskontakter och sponsring och vad därmed äga
                sammanhang.
        \end{dashlist}
    \item Vice Näringslivsutskottsordförande (1)
        \begin{dashlist}
            \item Denna post är en vice till utskottsordöranden.
            \item bistå Ordförande för Näringslivsutskottet i dennes arbete med ledande av ENU,
            \item bistå Ordförande för Näringslivsutskottet i dennes arbete med företagskontakter.
        \end{dashlist}
    \item Näringslivskontakt (e.a)
        \begin{dashlist}
            \item bistår Ordförande för Näringslivsutskottet i dennes arbete.
        \end{dashlist}
    \item Teknikfokusansvarig (1)
    		\begin{dashlist}
    			\item har till uppgift att ansvara för den årliga arbetsmarknadsmässan Teknikfokus från E-sektionens sida med hjälp av resterande arbetsmarknadsutskott.
    			\item ansvarar för rekryteringen av Projektgrupp Teknikfokus.
                \item Väljs vid Vårterminsmötet och har mandatperiod mellan 1 juli och 30 juni.
    		\end{dashlist}
    	\item Projektgrupp Teknikfokus (e.a)
    		\begin{dashlist}
    			\item har till uppgift att bistå Teknikfokusansvarig i sitt arbete.
    			\item har mandatperiod mellan 1 juli och 30 juni.
    			\item Väljs av styrelsen inför varje läsår på rekommendation av ENU-ordförande och Teknikfokusansvarig.
    		\end{dashlist}
        \item Alumniansvarig E (2)
            \begin{dashlist}
                \item ansvarar för att Sektionen har någon form av alumniverksamhet med huvudsakligt ansvar för E-alumner,
                \item ansvarar för att hålla Sektionens register över alumni uppdaterad,
                \item ansvarar för att göra utskick till alumni om Sektionen anordnar arrangemang då alumni bör bjudas in, samt
                \item  ansvarar för att hålla regelbunden kontakt med alumniföreningen vid Teknologkåren.
            \end{dashlist}
        \item Alumniansvarig BME (2)
            \begin{dashlist}
                \item ansvarar för att Sektionen har någon form av alumniverksamhet med huvudsakligt ansvar för BME-alumner,
                \item ansvarar för att hålla Sektionens register över alumni uppdaterad,
                \item ansvarar för att göra utskick till alumni om Sektionen anordnar arrangemang då alumni bör bjudas in, samt
                \item  ansvarar för att hålla regelbunden kontakt med alumniföreningen vid Teknologkåren.
            \end{dashlist}
\end{emptylist}

\subsubsection{Funktionärerna i Nöjesutskottet, NöjU}

\begin{emptylist}
    \item Entertainer (u)
        \begin{dashlist}
          \item Har det övergripande ansvaret för Sektionens kultur-, nöjes- och fritidsaktiviteter.
          \item Ansvarar för Sektionens instrument.
          \item Ansvarar för planering och genomförandet av UtEDischot tillsammans med D-sektionen och UtEDischoansvarig.
          \item Ansvarar för utkvittering av access till biljard- och pingisskåpet.
        \end{dashlist}
    \item Vice Entertainer (2)
        \begin{dashlist}
            \item Denna post är en vice till utskottsordöranden.
          \item Bistår Entertainern i dennes arbete med nöjes- och fritidsaktiviteter.
          \item Övertar Entertainerns uppgifter om denne inte kan närvara vid tillställningen.
          \item Ansvarar för genomförandet av tandemstaffeten.
        \end{dashlist}
    \item Fritidsledare (4)
        \begin{dashlist}
            \item Bistår Entertainern i dennes arbete med nöjes- och fritidsaktiviteter.
            \item Ansvarar för att arrangera mindre evenemang regelbundet under årets gång däribland spelkvällar.
            \item Ansvarar för att biljardbordet, pingisbordet och Sektionens sällskapsspel är i gott skick.
	 \item Ansvarar för genomförandet av drEamHackE.
        \end{dashlist}
    \item Idrottsförman (2)
        \begin{dashlist}
            \item Ansvarar för Sektionens idrottsarrangemang.
        \end{dashlist}
    \item Karnevalsmalaj (1)
        \begin{dashlist}
            \item Organiserar Sektionens deltagande i och omkring Lundakarnevalen.
            \item Har mandatperiod läsår.
            \item Väljs endast till de år då Lundakarnevalen är (vart fjärde år).
        \end{dashlist}
    \item Stridsrop (2)
        \begin{dashlist}
            \item Ansvarar för Sångarstridens planering, produktion och genomförande.
            \item Ansvarar för att en tackfest för Sektionens deltagare i Sångarstriden genomförs.
            \item Organiserar Sektionens Luciatåg.
        \end{dashlist}
    \item Umph-meister (2)
        \begin{dashlist}
        \item Ansvarar för att spela musik på de tillställningar där musik är passande.
        \end{dashlist}
    \item Øverbanan (1)
        \begin{dashlist}
        \item Ledare och kontaktperson för E-sektionens husband - “E-lektro banana band”.
        \item Ansvarar för husbandets medverkan i Sångarstriden.
        \item Ansvarar för att rekrytera bandmedlemmar - “Bananer” - till husbandet.
        \end{dashlist}
    \item UtEDischoansvarig (1)
        \begin{dashlist}
        \item Ansvarar för planeringen och genomförandet av UtEDischot tillsammans med D-sektionen och Entertainern.
        \end{dashlist}
\end{emptylist}

\subsubsection{Funktionärerna i Källarmästeriet, KM}

\begin{emptylist}
    \item Krögare (u)
        \begin{dashlist}
            \item har det övergripande ansvaret för Sektionens pubverksamhet
                och vad därmed äga sammanhang.
        \end{dashlist}
    \item Vice Krögare (2)
        \begin{dashlist}
        \item Denna post är en vice till utskottsordöranden.
        \item Avbelasta krögaren genom att kunna hålla fester (gillen) i
            Edekvata.
        \end{dashlist}
    \item Källarmästare (e.a)
        \begin{dashlist}
            \item har till uppdrag att tillgodose sektionens behov av
                pubverksamhet enligt Krögarens instruktioner.
        \end{dashlist}
    \item Cøl (2)
        \begin{dashlist}
            \item sköta beställning, inköp och inventering av sektionens öl
                och sprit, exkluderat det som hanteras av barmästarna,
            \item ansvarar för att prissättning för ovan nämda drycker
                följer alkohollagen.
        \end{dashlist}
\end{emptylist}
\subsubsection{Funktionärerna i Cafémästeriet, CM}

\begin{emptylist}
    \item Cafémästare (u)
        \begin{dashlist}
            \item har det övergripande ansvaret för Sektionens
                caféverksamhet och vad därmed äga sammanhang,
            \item ansvar för den ekonomiska redovisningen för
                Cafémästeriet.
        \end{dashlist}
    \item Vice Cafémästare (2)
        \begin{dashlist}
            \item Denna post är en vice till utskottsordöranden.
            \item tillsammans med cafemästare bedriva den dagliga driften
                av sektionens café.
        \end{dashlist}
    \item Inköps- och lagerchef (4)
        \begin{dashlist}
            \item sköter inköp och redovisning av varor till
                caféverksamheten samt ansvarar för dess lager.
        \end{dashlist}
    \item Halvledare (5)
    \begin{tightdashlist}
        \item Ansvarar för caféet en dag i veckan.
        \item Ansvarar för att rekrytera Dioder till sin dag i veckan och ser till att Dioderna utbildas.
        \item Hjälper till under öppning och stängning av caféet.
        \item Väljs per läsperiod av styrelsen på rekommendation av Cafémästaren.
    \end{tightdashlist}
    \item Diod (e.a.)
    		\begin{dashlist}
            \item ansvarar för att se till att hjälpa till i LED-café en dag i veckan i minst 2 timmar.
            \item att inom den dagliga verksamheten se till att ta betalt och hjälpa kunder i kassan, hjälpa till att göra de dagliga cafévarorna, städa och se till att hygien- och miljörutiner följs samt att ta ut soporna efter varje arbetspass.
            \item väljs per läsperiod av styrelsen utifrån Cafémästarens rekommendation.
        \end{dashlist}
\end{emptylist}
\subsubsection{Funktionärerna i Sexmästeriet, E6}

\begin{emptylist}
    \item Sexmästare (u)
        \begin{dashlist}
            \item har det övergripande ansvaret för Sektionens
                festarrangemang och vad därmed äga sammanhang,
            \item ansvarar för att hitta erforderligt antal arbetare
                till respektive arrangemang,
        \end{dashlist}
    \item Vice Sexmästare (1)
        \begin{dashlist}
            \item Denna post är en vice till utskottsordöranden.
            \item bistår Sexmästaren i dennes arbete med festarrangemang.
        \end{dashlist}
    \item Barmästare (2)
    \begin{tightdashlist}
        \item Ansvarar för baren under och innan sittningar.
        \item sköter beställning, inköp och inventering av alkoholdryck för Sexmästeriet,
        \item ansvarar för att prissättning av ovan nämnda drycker följer alkohollagen.
    \end{tightdashlist}
    \item Sångförman (2)
        \begin{dashlist}
            \item ansvarar för att Sektionens arrangemang tillgodoses
                med ljuvlig sång.
        \end{dashlist}
    \item Köksmästare (2)
        \begin{dashlist}
        \item Ansvarar för planering av meny och inhandling av mat innan sittningen,
        \item sköter matlagningen och delegeringen av uppgifterna i köket under sittningen. bistår sexmästaren vid planering
        av menyn.
        \end{dashlist}
    \item Hovmästare (2)
        \begin{dashlist}
        \item ansvarar för dukning och servering på sittningen.
        \end{dashlist}
    \item Preferensmästare (1)
        \begin{dashlist}
        \item Ansvarar för planering av meny och inhandling av specialkost innan sittningen.
        \item Sköter tillagningen av specialkost och delegeringen av uppgifterna i köket under sittningen.
        \end{dashlist}
    \item Sexig (e.a)
        \begin{dashlist}
        \item Hjälper till vid sexets tillställningar
        \end{dashlist}

\end{emptylist}
\subsubsection{Funktionärerna i Nolleutskottet, NollU}

\begin{emptylist}
    \item Øverphøsare (u)
        \begin{dashlist}
            \item Har det övergripande ansvaret för nollningen.
            \item Ansvarar för nollningsaktiviteter och nolleuppdrag.
            \item Deltar aktivt i TLTH:s gemensamma planering inför nollningen.
        \end{dashlist}
    \item Co-phøsare (5)
        \begin{dashlist}
          \item Bistår Øverphøset i dennes arbete.
          \item Denna post är en vice till utskottsordföranden.
          \item Ett Co-phøs ansvarar för den ekonomiska redovisningen av nollningen.
          \item Ett Co-phøs ansvarar för rekryteringen av phaddrarna.
        \end{dashlist}
    \item Övergudphadder (2)
        \begin{dashlist}
          \item Bistår Øverphøset och Cophøsen i deras arbete.
        \end{dashlist}
\end{emptylist}
\subsubsection{Funktionärerna i Studierådet, SRE}

\begin{emptylist}
    \item SRE-Ordförande (u)
        \begin{dashlist}
            \item har det övergripande ansvaret och vad därmed äga
                sammanhang,
            \item ansvarar för att E-sektionen representeras i
                Teknologkårens organ,
            \item ansvarar för att E-sektionen representeras i
                institutionsstyrelser.
        \end{dashlist}
    \item Vice SRE-Ordförande (1)
        \begin{dashlist}
            \item Denna post är en vice till utskottsordöranden.
            \item är SRE-Ordförandens ställföreträdare,
            \item ansvarar för ekonomihantering inom studierådet,
            \item ansvarar för administrativa uppgifter rörande
                studiebevakning.
        \end{dashlist}
    \item Årskurs E-1 ansvarig (2)
        \begin{dashlist}
            \item ansvarar och genomför muntliga och skriftliga
                kursutvärderingar som hålls minst en gång per läsperiod,
            \item ansvarar för administration och utbildningsbevakning av
                årskursen,
            \item anmäler sig som kursombud i de kurser som ingår i
                obligatoriet.
            \item väljs per läsår av styrelsen på rekommendation av SRE-ordförande.
        \end{dashlist}
    \item Årskurs E-2 ansvarig (2)
        \begin{dashlist}
            \item ansvarar och genomför muntliga och skriftliga
                kursutvärderingar som hålls minst en gång per läsperiod,
            \item ansvarar för administration och utbildningsbevakning av
                årskursen,
            \item anmäler sig som kursombud i de kurser som ingår i
                obligatoriet.
              \item väljs per läsår av styrelsen på rekommendation av SRE-ordförande.
        \end{dashlist}
    \item Årskurs E-3 ansvarig (2)
        \begin{dashlist}
            \item ansvarar och genomför muntliga och skriftliga
                kursutvärderingar som hålls minst en gång per läsperiod,
            \item ansvarar för administration och utbildningsbevakning av
                årskursen,
            \item anmäler sig som kursombud i de kurser som ingår i
                obligatoriet.
             \item väljs per läsår av styrelsen på rekommendation av SRE-ordförande.
        \end{dashlist}
    \item Årskurs BME-1 ansvarig (2)
        \begin{dashlist}
            \item ansvarar och genomför muntliga och skriftliga
                kursutvärderingar som hålls minst en gång per läsperiod,
            \item ansvarar för administration och utbildningsbevakning av
                årskursen,
            \item anmäler sig som kursombud i de kurser som ingår i
                obligatoriet.
              \item väljs per läsår av styrelsen på rekommendation av SRE-ordförande.
        \end{dashlist}
    \item Årskurs BME-2 ansvarig (2)
        \begin{dashlist}
            \item ansvarar och genomför muntliga och skriftliga
                kursutvärderingar som hålls minst en gång per läsperiod,
            \item ansvarar för administration och utbildningsbevakning av
                årskursen,
            \item anmäler sig som kursombud i de kurser som ingår i
                obligatoriet.
             \item väljs per läsår av styrelsen på rekommendation av SRE-ordförande.
        \end{dashlist}
    \item Årskurs BME-3 ansvarig (2)
        \begin{dashlist}
            \item ansvarar och genomför muntliga och skriftliga
                kursutvärderingar som hålls minst en gång per läsperiod,
            \item ansvarar för administration och utbildningsbevakning av
                årskursen,
            \item anmäler sig som kursombud i de kurser som ingår i
                obligatoriet.
             \item väljs per läsår av styrelsen på rekommendation av SRE-ordförande.
        \end{dashlist}
    \item SRE-ledamot (e.a)
        \begin{dashlist}
            \item intresserade personer att jobba med studiefrågor.
        \end{dashlist}
    \item Studerandeskyddsombud med ansvar för fysisk miljö (1)
        \begin{dashlist}
            \item ska aktivt verka för allas trevnad i den egna arbetsmiljön
            \item ska kunna ta emot klagomål och svara på frågor om den egna arbetsmiljön
            \item ska påverka studenternas arbetsförhållanden i syfte att bidra till en god studiemiljö
            \item ska delta i husets skyddsrond, bedöma hur förändringar påverkar studenternas arbetsmiljö samt att
hålla sig underrättad om arbetsmiljölagstiftningen
			\item ska finnas representerad i husets HMS-kommitté (Hälsa Miljö och Säkerhetskommitté) och där hålla
sig uppdaterad om läget i huset och lyfta frågor som berör studenternas studiemiljö
			\item ansvarar för att se till att sektionens styrelse är informerad om arbetet i HMS-kommittén
			\item ska hålla sektionen informerad om Teknologkårens arbetsmiljöarbete
        \end{dashlist}
    \item Studerandeskyddsombud med likabehandlingsansvar (2)
        \begin{dashlist}
            \item ansvarar för att bevaka Sektionens verksamhet från ett likabehandlingsperspektiv
            \item ska verka för en jämlik studiemiljö
            \item ska uppmärksamma styrelsen på situationer och miljöer som skulle kunna upplevas som kränkande av studenter vid Sektionen
            \item ska hålla Sektionen informerad om TLTH:s policy för likabehandling
            \item ska fungera som kontaktperson och hjälp för medlemmar som anser sig särbehandlade av personer kopplade till högskolan eller programmet
        \end{dashlist}
    \item Världsmästare (2) \\
    	Världsmästare är funktionärer på sektionen som ska hjälpa utbytesstudenter. Världsmästaren ska
    	 \begin{dashlist}
    		\item närvara vid Registration day i början av terminen,
    		\item annordna evenemang för de internationella studenterna vid sektionen,
    		\item informera studenter vid sektionen om utbytesstudier,
    		\item se till att sektionen har ett internationellt perspektiv i sin organisation,
    		\item aktivt delta i TLTH:s kollegie,
    		\item se till att masterprogrammen vid sektionen utvärderas i sammarbete med studierådet,
    		\item informera internationella studenter om vad som händer på kåren och sektionen, samt
    		\item hålla en god kontakt med introduktionsansvarig samt internationella överstar för att kunna främja integreringen av internationella studenter.
        \end{dashlist}
    \item HFT-ansvarig (2)
        \begin{dashlist}
            \item ansvarar för Sektionens åtagande i samband med Her tech future.
        \end{dashlist}
\end{emptylist}
\subsubsection{Funktionärerna i Valberedningen, VB}

\begin{emptylist}
    \item Valberedningens Ordförande (u)
        \begin{dashlist}
            \item har det övergripande ansvaret för valberedningens arbete
                och vad därmed äga sammanhang,
            \item att ansvara att det hålls en omsitts för Valberedningen.
        \end{dashlist}
    \item Valberedningens sekreterare (exakt 1)
        \begin{dashlist}
            \item ansvarar för valberedningens anslag och protokoll.
        \end{dashlist}
    \item Valberedningsledamot (exakt 2)
        \begin{dashlist}
            \item ansvarig för valberedning av funktionärer.
        \end{dashlist}
    \item Representant från de nyantagna (exakt 1)
        \begin{dashlist}
            \item ansvarig för valberedning av funktionärer,
            \item utses av Valberedningen i samarbete med Nolleutskottet.
        \end{dashlist}
\end{emptylist}

\subsubsection{Funktionärerna i TLTH:s organ, Sektionsrepresentanter}
\begin{emptylist}
    \item Valberedningen (1)
    \item Valnämnden (1)
\end{emptylist}

\subsubsection{Övriga funktionärer}
\begin{emptylist}
    \item Talman (1)
        \begin{dashlist}
            \item är Sektionsmötets Ordförande och leder förhandlingarna,
            \item ska tillsammans med Ordföranden upprätta lämplig
                föredragningslista till Sektionsmöte,
            \item har till uppgift att övervaka Sektionens stadgar och
                reglemente samt att till Styrelsen inkomma med förslag på
                förändringar.
        \end{dashlist}
    \item Sigillbevararen (1)
        \begin{dashlist}
            \item ansvarar för skötsel av Sektionens standar och bärandet av
                dessa,
            \item ansvarar för Sektionens medaljer och utdelning av sådana,
            \item vakar över Sektionens heliga klenoder och helgedomar.
        \end{dashlist}
    \item Projektfunktionär (e.a.)
    \begin{dashlist}
      \item Har ett projekt med beslutsuppföljning på Sektions- eller styrelsemöte och väljs in som funktionär för att få funktionärsprivilegier.
      \item Har en mandatperiod som bestäms vid valtillfället, och är maximalt ett år lång.
    \end{dashlist}
\end{emptylist}
\renewcommand*\thesubsection{\arabic{section}:\Alph{subsection}}
\section{Redaktionella organ}

\section{Val}

\subsection{Personval}

Enligt stadgan §12:6 avgör lotten vid lika antal röster i votering i personval och där ej annat stadgats eller Reglementet föreskriver annorlunda gäller enkel majoritet av giltiga röster.

\begin{alphlist}
\item En kandidat, en ska tillsättas

Görs med acklamation om inte sluten votering begärs. Om acklamationen ej blir
enhällig blir det automatiskt sluten votering.

\item Mindre eller lika många kandidater än vad som ska tillsättas, flera ska tillsättas

Kandidaterna väljes en och en enligt fall a). Med acklamation kan mötet bestämma att välja alla i klump.

\item Flera kandidater, en ska tillsättas

Görs med sluten votering. Om någon kandidat erhåller majoriteten av rösterna är personen vald. I annat fall stryks den kandidaten som erhållit minsta röstetal. Vid lika antal röster avgör lotten. Efter det upprepas proceduren med de kvarvarande kandidaterna enligt passande fall.

\item Fler kandidater än vad som ska tillsättas, flera ska tillsättas

Görs med sluten votering. Varje röstberättigad person får maximalt lika många röster som det finns platser kvar att tillsättas. Om någon kandidat erhåller majoriteten av rösterna är personen vald, och valproceduren upprepas med de kvarvarande kandidaterna enligt passande fall. I annat fall stryks den kandidaten som erhållit minsta röstetal. Vid lika antal röster avgör lotten. Efter det upprepas proceduren med de kvarvarande kandidaterna enligt passande fall.
\end{alphlist}

\subsection{Sakfrågor}

Görs med acklamation om inte votering begärs. Först behandlar man alla yrkanden och därefter de tilläggsyrkanden som inte har automatiskt har avslagits. Talmannen ställer lämpliga yrkanden emot varandra för att slutligen få ett yrkande kvar som Sektionsmötet ska ta ställning till. När detta ej är möjligt använd lämpligt fall för personval för att få ett slutgiltigt beslut. Därefter behandlas de tilläggsyrkande som finns kvar på samma sätt.

\section{Protokoll}

\renewcommand*\thesubsection{\arabic{section}:\arabic{subsection}}
\renewcommand*\thesubsubsection
{\arabic{section}:\arabic{subsection}:\Alph{subsubsection}}
\section{Ekonomi}

\subsection{Fonder}

\subsubsection{Dispositionsfonden}
\begin{emptylist}
\item Disposition
    \begin{dashlist}
    \item Styrelsen äger rätt att disponera de avsatta medlen
    \end{dashlist}
\end{emptylist}
\begin{emptylist}
\item Medlen används till
    \begin{dashlist}
    \item större underhåll av lokal och utrustning
    \item inköp för att ersätta funktionsoduglig utrustning
    \item mindre nyinvesteringar för att bibehålla och komplettera nuvarande
        verksamhet
    \item täcka eventuella budgetunderskott.
    \end{dashlist}
\end{emptylist}

\subsubsection{Utrustningsfonden}
\begin{emptylist}
\item Disposition
    \begin{dashlist}
    \item endast Terminsmötet disponerar de avsatta medlen
    \end{dashlist}
\end{emptylist}
\begin{emptylist}
\item Medlen används till
    \begin{dashlist}
    \item större nyinvesteringar och ombyggnader
    \end{dashlist}
\end{emptylist}

\subsubsection{Olycksfonden}
\begin{emptylist}
\item Disposition
    \begin{dashlist}
    \item Styrelsen äger rätt att disponera de avsatta medlen
    \end{dashlist}
\end{emptylist}
\begin{emptylist}
\item Medlen används till
    \begin{dashlist}
    \item kostnader i samband med olycksfall med personskador eller personlig
        utrustning
    \end{dashlist}
\end{emptylist}

\subsubsection{Modulo10fonden}
\begin{emptylist}
\item Disposition
    \begin{dashlist}
    \item endast Styrelsen äger rätt att disponera de avsatta medlen enligt
        fondens ändamål
    \end{dashlist}
\end{emptylist}

\subsubsection{Jubileumsfonden}
\begin{emptylist}
\item Disposition
    \begin{dashlist}
    \item Styrelsen äger rätt att disponera de avsatta medlen
    \end{dashlist}
\end{emptylist}
\begin{emptylist}
\item Medlen används till
    \begin{dashlist}
    \item delvis finansiera de kostnader som uppkommer i samband med att sektionen firar jubileum vart femte år.
    \end{dashlist}
\end{emptylist}

\subsubsection{Allmänt}

Alla uttag ur sektionens fonder skall redovisas på närmst efterföljande
Terminsmöte samt i bokslutet.

\subsection{Riktlinjer för budget}

\subsubsection{Allmänt}

I samband med att Styrelsen lägger fram sitt budgetförslag på Höstterminsmötet skall riktlinjer för budget bifogas. I riktlinjer för budget skall det framgå
på vilket sätt budgetposterna regleras. Till budgetförslaget skall även en verksamhetsrapport upprättas.

\renewcommand*\thesubsection{\arabic{section}:\Alph{subsection}}
\section{Definitioner}

\subsection{Läsdag}
Vid beräkning av antalet läsdagar mellan kallelse och mötet gäller följande
\begin{emptylist}
\item Om mötet eller motsvarande hålls under en läsdag får den räknas med om
    mötet hålls efter 17.00
\item Om kallelse eller motsvarande anslås under en läsdag får den räknas som
    läsdag om det sker före 12.00
\end{emptylist}

\subsection{Giltig röst}
En giltig röst är en icke-blank inlämnad röst som följer reglerna för omröstningen i fråga. Ogiltiga röster, alltså ej giltiga röster, räknas inte i omröstningen.
\newpage

\section{Policybeslut}
Policybeslut kan endast antas, ändras eller tas bort på Sektionsmöte. Antagna policybeslut införs automatiskt i listan nedan. Själva policybeslutet ska ligga på Sektionens hemsida tillgänglig för allmänheten.

Sektionen har antagit följande policybeslut:
\begin{dashlist}
    \item Principer för deltagande i Sektionsaktiviteter
    \item Inbjudningar och anmodningar
    \item Sektionens medaljer och dess utdelning
    \item Nyckel- och accesspolicy för E-sektionens funktionärer
    \item Publiceringspolicy
    \item Utläggspolicy
    \item Alkohol- och drogpolicy
    \item Policy för utlåning och hantering av arbetskläder
\end{dashlist}

\section{Riktlinjer}
En riktlinje är ett konkret direktiv på hur ett utskott eller styrelsen ska arbeta och/eller agera i olika situationer. Riktlinjer får ej säga emot övriga styrdokument.

Riktlinjer kan endast ändras eller tagas bort på styrelsemöte. Antagna riktlinjer införs automatiskt i listan nedan. Själva riktlinjen ska ligga på Sektionens hemsida tillgänglig för allmänheten.

Styrelsen ansvarar för att uppdatera riktlinjerna när behov finns.

Styrelsen har antagit följande riktlinjer:

\begin{dashlist}
    \item Körersättningsriktlinje
    \item Hantering av kassadifferanser
    \item Återbetalningsskyldighet
\end{dashlist}

\clearpage

\section*{Revideringshistorik}
\addcontentsline{toc}{section}{Bilaga: Revideringshistorik}
\begin{center}
\begin{tabular}{| l | l | p{10cm} |}
    \hline
    Version & Tagen & Kapitel som ändrats sedan föregående version \\
    \hline
    V1.0 & 1997-11-10 & \\
    \hline
    V1.1 & 1998-05-11 & 3, 4:C, 4:E, 6, 7, 7:B, 9, 10, 10:1:A, 10:2, 12:B,
    14, 15, 15:A \\
    \hline
    V1.2 & 1998-11-09 & 4:C, 4:D, 9:2:G, 10:2:E \\
    \hline
    V1.3 & 1999-05-10 & 2:B, 4:A, 4:C, 4:D, 6, 8:B, 8:C, 10:2:M, 15:B, 16 \\
    \hline
    V1.4 & 2000-05-18 & 7, 14:1:A \\
    \hline
    V1.5 & 2000-10-30 & 4:A, 10:2:H, 10:2:J \\
    \hline
    V1.6 & 2001-05-08 & 10:2:E \\
    \hline
    V1.7 & 2001-11-06 & 10:2:E, 10:2:J \\
    \hline
    V1.8 & 2002-03-19 & 2:A, 8:D, 9:1:A, 9:2:A, 9:2:D, 9:2:G, 9:2:H, 10:1:A,
    10:2:B, 10:2:F, 10:2:H, 10:2:J, 10:2:M, 12:A, 14:1:A, 16 \\
    \hline
    V1.9 & 2002-11-14 & 7:A, 7:B, 8:B, 8:C, 9:2:B, 9:2:D, 9:2:E, 9:2:G,
    9:2:H, 10:2:D, 10:2:F, 10:2:I, 10:2:J, 10:2:M \\
    \hline
    V2.0 & 2003-05-21 & 8:A, 9:2:C, 10:2:G \\
    \hline
    V2.1 & 2003-11-17 & 10:2:I,10:2:K, 9:2:F, 9:2:E \\
    \hline
    V2.2 & 2004-05-06 & 10:2:M., 10:2:F, 9:2:B, 10:1:A, 9:2:B, 9:2:D, 9:2:E,
    9:2:G, 9:1:A \\
    \hline
    V2.3 & 2004-11-11 & 10:2:M, 10:2:O, 10:2:F \\
    \hline
    V2.4 & 2005-11-07 & 8:D, 9:1A, 10:2:C, 10:2:D, 10:2:E, 10:2:F, 10:2:L,
    10:2:N, 10:2:O \\
    \hline
    V2.5 & 2006-05-16 & 10:2:M \\
    \hline
    V2.6 & 2008-11-04 & 9:2:A, 9:2:B, 9:2:D, 9:2:E, 9:2:H, 10:2:E, 10:2:F,
    10:2:H, 10:2:I, 10:2:J, 10:2:K, 10:2:L, 10:2:M \\
    \hline
    V2.7 & 2009-11-09 &  8:B, 8:C, 10:2:E, 10:2:F, 10:2:H, 10:2:M\\
    \hline
    V2.8 & 2010-11-09 & 10:2:J, 10:2:H, 10:2:I, 10:2:A\\
    \hline
    V2.9 & 2011-04-12 & 9:2:G, 10:2:E, 10:2:H, 10:2:K, 10:2:L. 10:2:M, 16 \\
    \hline
    V3.0 & 2014-05-21 &  4:C, 4:D, 7:B, 7:C, 8:A, 8:B, 8:C, 9:1:A, 9:2:B, 9:2:C, 9:2:D, 9:2:E, 9:2:H, 9:3, 10:1:A, 10:2:E, 10:2:F, 10:2:G, 10:2:H, 10:2:J, 10:2:K, 10:2:L, 10:2:M, 10:2:N, 10:2:P, 14:1:D, 16\\
    \hline
    V3.1 & 2015-11-19 & \emph{Revideringshistorik saknas} \\
    \hline
    V4.0 & 2016-04-20 & 7:A, 10:2:C, 10:2:E, 10:2:F, 10:2:G, 12:A, 16, 17 \\
    \hline
    V4.1 & 2016-11-22 & 9:2, 10:2 \\
    \hline
    V5.0 & 2017-05-02 & 10:2:H, 10:2:F, 7:A, 7:B, 7:C, 7:D, 4:G, 15:C, 12:A, 12:B, 10:2:P, 15:A, 10:2:L \\
    \hline
    V5.1 & 2017-11-14 & 10:2:F, 10:2:G, 10:2:K, 10:2:J, 10:2:P, 10:2:H, 10:2:M, 16 \\
    \hline
    V6.0 & 2019-04-09 & 4:C, 4:D, 4:E, 9:2:B, 9:2:C, 10:2:C, 10:2:F, 10:2:G, 10:2:H, 10:2:I, 10:2:K, 10:2:L, 10:2:M, 10:2:N \\
   \hline
\end{tabular}
\end{center}
För specifikation av ändringar, se Reglementets revideringshistorik.

\end{document}
