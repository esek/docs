\documentclass[10pt]{article}
\usepackage[utf8]{inputenc}
\usepackage[swedish]{babel}

\def\rikttitle{Återbetalningsskyldighet}
\def\antagen{2017-03-30}
%\def\uppdaterad{}

\usepackage{./e-riktlinje}
\usepackage{../e-styrdok}
\usepackage{../../e-sek}

\def\headr{Utkast - ej antagen}

\begin{document}
    \section*{\doctitle}

    Innehavaren av ett sektionsbankkort blir återbetalningsskyldig då kortet används för inköp som går emot kortkontraktet eller om kortinnehavaren inte skrivit under kortkontraktet. Detta gäller oavsett vem som gjort inköpet eftersom kortinnehavaren är personligen ansvarig för all användning av betalkortet.

    Återbetalningsskyldighet beslutas av styrelsen innan (halvårs)bokslutet är gjort. Firmatecknarna, alltså Förvaltningschefen och Ordföranden, ansvarar för att inbetalningen sker korrekt. Det åligger även firmatecknarna att ansvara för att alla kortinnehavare skriver under kortkontraktet.

    Kortkontraktet är bifogat på nästa sida.

    \includepdf[pages=1]{../../avtal/kortkontrakt.pdf}
\end{document}
