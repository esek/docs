\documentclass[10pt]{article}
\usepackage[utf8]{inputenc}
\usepackage[swedish]{babel}

\def\rikttitle{Användning av sektionens G Suite}
\def\antagen{: Datum}
%\def\uppdaterad{}

\usepackage{./e-riktlinje}
\usepackage{../e-styrdok}
\usepackage{../../e-sek}

\begin{document}
    \section*{\doctitle}
    \vspace{6px}

    \section{Syfte}
    \vspace{6px}
    
    Syftet med denna riktlinje är att utgöra ett underlag för användandet av sektionens tilllgångar tillhandahållna via G-Suite, såsom sektionens Team Drives (även kallat Delade Enheter) och E-post. 
    
    \vspace{6px}
    \section{Allmänt}
    \vspace{6px}

    Funktionärer bör ha tillgång och access till de tjänster i G Suite kan förbättra eller hjälpa funktionärens arbete. 
    Sektionen är dock inte skyldig att tilldela funktionärer tillgång och access av G Suite, och det är således ingen rättighet för funktionären. 
 
    G Suite administreras av Kontaktor, Sektionsordförande och sektionens datoransvariga. 
    
    \vspace{3px}
    \subsection{Team Drives}
    \vspace{3px}

    För att sektionens tjänst kunna lagra information digitalt har varje utskott och funktionär som behöver det tillgång till en eller flera relevanta Team Drives tillhandahållet av G Suite. 

    \textbf{Struktur}
    \begin{dashlist}

    \item Varje utskottsordförande förvaltar en Team Drive för respektive utskott. Dennes roll ska vara ''Chef/Manager'' och har det övergipande ansvaret för drivens innehåll. 
    
    \item Vice utskottsordförande bör ha rollen ''Innehållshanterare/Content manager''. 

    \item Samtliga utskottsmedlemmar bör ha tillgång till respektive Team Drive och dessa bör ha rollen ''Deltagare/Contributor'' eller om lämpligt ''Innehållshanterare/Content Manager''. 
    
    \item I samtliga Team Drives för utskott skall Kontaktor och Sektionsordförande ha rollen ''Chef/Manager'', och resten av styrelsen bör ha rollen ''Deltagare/Contributor''.
    
    \item Valberedingen Team Drive förvaltas av Valberedningens Ordförande.
    
    \item Övriga Team Drives förvaltas av en ansvarig i samråd med sektionens styrelse. 

    \item Sektionens revisorer bör ha tillgång till alla sektionens Team Drives, med undantag för sekretesskyddat material exempelvis från 'Skyddsombud med likabehandlingsansvar' och 'Valberedningen', innehavande åtminstone ''Läsbehörig/Viewer''-roll för kunna att granska sektionens verksamhet. 
     
    \end{dashlist}
    
    \vspace{3px}
    \textbf{Överlämning}

    Styrelsen och utskottsordförande ansvarar för att uttskottsdrives går i arv vid funktionärsskifte.

    \vspace{10px}
    \subsection{E-post}
    \vspace{6px}

    För att i sektionens tjänst kunna sprida och ta emot information digitalt kan funktionärer få tillgång till ett postspecifik E-postkonto tillhandahållet av G Suite med en esek.se adress. 

    Dessa postspecifika konton är sektionens tillgång och användning av dessa ska endast vara i fuktionärssyfte och i sektionens tjänst. 
    Vid funktionärsskifte skall dessa gå i arv och lösenord på respektive konto skall bytas. 
    
    
    %\begin{dashlist}
    %    \item Vilka ska få e-postkonton genom G Suite, runt 20st? Postspecifika e-postkonton istället för personliga?? Kanske mer najs. Man kan forwarda dom till sin vanliga mail om man vill. Kan vissa poster dela? Typ macapar. 
    %    \item Vill undvika att skapa många users. Kan innebära hög kostnad i framtiden?? Nackdel att vara bunden till google. Egen mailserver plus här. 
    %    \item Börja se igenon alias och lägga till. Vill vi gå över helt?? 
    %    \item Kan inte skapa email-listor via g-suite. Men vi har emaillistor via hemsidan, man måste bara göra dem manuellt vid skickning.  
    %    \item Ta bort gamla personliga esekkonton?
    %\end{dashlist}
    \vspace{6px}
    \section{Personuppgifter}
    \vspace{6px}
    Om personuppgifter ska lagras digitalt i en molnbaserad tjänst måste denna vara en av sektionsadministrerad Team Drive. 
    Personuppgifter som lagras i Team Drive måste behandlas enligt sektionens regler för hantering av personuppgifter som tagits fram av Teknologkåren vid LTH och får inte delas vidare utanför sagda Team Drive samt inte laddas ner annat än för kortvarigt motiverbart bruk.

\end{document}
