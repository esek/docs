\documentclass[10pt]{article}
\usepackage[utf8]{inputenc}
\usepackage[swedish]{babel}

\def\rikttitle{Användning av sektionens G Suite}
\def\antagen{: Datum}
%\def\uppdaterad{}

\usepackage{./e-riktlinje}
\usepackage{../e-styrdok}
\usepackage{../../e-sek}

\begin{document}
    \section*{\doctitle}
    \section{Syfte}
    
    Syftet med denna riktlinje är att utgöra ett underlag för användandet av sektionens tilllgångar tillhandahållna via G-Suite, såsom sektionens Team Drives (även kallat Shared drives) och e-postkonton. 
    
    %\section{Allmänt}
    %G Suite administreras av Sektionens Kontaktor och Sektionsordförande. \textit{Mer? }

    %Funktionären bör ha tillgång till de nycklar och den access som krävs för att sköta posten. 
    %Sektionenär dock inte skyldig att tilldela en funktionär nyckel eller access, och det är således ingen rättighetför funktionären.

    \section{Team Drives}
    För att i sektionens tjänst kunna lagra information digitalt har varje utskott och funktionär som behöver det tillgång till en eller flera relevanta Team Drives tillhandahållet av G-Suite. Annat namn för Team Drives är Shared drives.

    ** ARV av teamdrives
    ** Skifte, access direkt. Entlidigad - ta bort. (utskottsordförande kan med lämnas kvar om det underlättar)
    ** Undvik att utskotts-utomstående bjuds in. Skall finnas en god motivering, kontakt av styrelse alt kontaktor?
    
    ** DDG, tillgång till adminpanelen för administration? Oklart om det behövs. 

    ** Delete av gamla G-suite konton. Vill ej samla på oss users i längden. 
    \begin{dashlist}

    \item Varje utskottsordförande förvaltar en Team Drive för respektive utskott. Dennes roll ska vara ''Manager''. 
    
    \item Vice utskottsordförande bör ha rollen ''Content manager''. 

    \item Samtliga utskottsmedlemmar bör ha tillgång till respektive drive och dessa bör ha rollen ''Contributor'' eller om lämpligt ''Content Manager''. 
    
    \item Styrelsen ska ha rollen ''Content Manager'', och Kontaktor och Sektionsordförande ska ha rollen ''Manager'' i samtliga utskotts Team Drives.  
    
    \item Övriga drives..?? Tilldelas av styrelsen. Valberedingen förvaltas av VB-ordförande.

    \item Sektionens revisorer bör ha tillgång till alla sektionens Team Drives, med undantag för sekretesskyddat material exempelvis från \textit{Skyddsombud med likabehandlingsansvar} och \textit{Valberedningen}, innehavande åtminstone ''Viewer''-roll för kunna att granska verksamheten. 
    
    \end{dashlist}

    \subsection{E-post}

    \begin{dashlist}
        \item Vilka ska få e-postkonton genom G Suite, runt 20st? Postspecifika e-postkonton istället för personliga?? Kanske mer najs. Man kan forwarda dom till sin vanliga mail om man vill. Kan vissa poster dela? Typ macapar. 
        \item Vill undvika att skapa många users. Kan innebära hög kostnad i framtiden?? Nackdel att vara bunden till google. Egen mailserver plus här. 
        \item Börja se igenon alias och lägga till. Vill vi gå över helt?? 
        \item Kan inte skapa email-listor via g-suite. Men vi har emaillistor via hemsidan, man måste bara göra dem manuellt vid skickning.  
        \item Ta bort gamla personliga esekkonton?
    \end{dashlist}

    \section{Personuppgifter}
    Om personuppgifter ska lagras digitalt i en molnbaserad tjänst måste denna vara en av sektionsadministrerad Team Drive. 
    Personuppgifter som lagras i Team Drive måste behandlas enligt ((sektionensriktlinjer för hantering av personuppgifter)), får inte delas vidare utanför sagda Team Drive och får inte laddas ner annat än för kortvarigt motverbart bruk.

\end{document}
