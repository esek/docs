\documentclass[10pt]{article}
\usepackage[utf8]{inputenc}
\usepackage[swedish]{babel}

\def\rikttitle{Användning av sektionens G-Suite}
\def\antagen{: Datum}
%\def\uppdaterad{}

\usepackage{./e-riktlinje}
\usepackage{../e-styrdok}
\usepackage{../../e-sek}

\begin{document}
    \section*{\doctitle}
    \section{Syfte}
    
    Syfet med denna riktlinje är att utgöra ett underlag för användandet av sektionens tilllgångar tillhandahållna via G-Suite, såsom sektionens Team Drives (även kallat Shared drives).
    
    \section{Användning}
    För att i sektionens tjänst kunna lagra information digitalt har varje utskott och funktionär som behöver det tillgång till en eller flera relevanta Team Drives tillhandahållet av G Suite. Annat namn för Team Drives är Shared drives.
    
    Varje utskottsordförande förvaltar en Team Drive för respektive utskott. Dennes roll bör vara ''Manager''. Samtliga utskottsmedlemmar ska ha tillgång till denna drive. Dessa bör ha rollen ''Contributor''. 
    Vid behov kan en vice utskottsordförande ha rollen ''Content manage''. Revisorerna ska ha tillgång till alla sektionens Team Drives, innehavande åtminstone ''Viewer''-roll.
    
    \section{Personuppgifter}
    Om personuppgifter ska lagras digitalt i en molnbaserad tjänst måste denna vara en av sektionsadministrerad Team Drive. 
    Personuppgifter som lagras i Team Drive måste behandlas enligt ((sektionensriktlinjer för hantering av personuppgifter)), får inte delas vidare utanför sagda Team Drive och får inte laddas ner annat än för kortvarigt motverbart bruk.

\end{document}
