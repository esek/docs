\documentclass[10pt]{article}
\usepackage[utf8]{inputenc}
\usepackage[swedish]{babel}

\def\rikttitle{Körersättningsriktlinjer}
\def\antagen{2016-xx-xx}
%\def\uppdaterad{}

\usepackage{./e-riktlinje}
\usepackage{../e-styrdok}
\usepackage{../../e-sek}

\begin{document}
    \section*{\doctitle}

    Körersättning kan utgå då man använt en bil som Sektionen ej har betalat för, man kört för att fullgöra sin funktionärssyssla och följande punkter efterlevs:

    \begin{dashlist}
        \item Samåkning ska utnyttjas då detta är möjligt, om så inte skett utgår endast ersättning motsvarande en resa som delas jämnt mellan dem berörda.

        \item Kollektivtrafik ska utnyttjas då detta är ekonomiskt gynnsamt och praktiskt genomförbart, om så inte skett utgårendast ersättning motsvarande biljettpriset för resan med kollektivtrafik.
    \end{dashlist}

    Om kraven ovan efterlevts utgår ersättning enligt skatteverkets avdragsregler (2016 18,50 kr/mil). Har man kört med släp utgår en extra ersättning på 20\% på grund av antagen ökad bränsleförbrukning och slitage. Vägtullar, brotullar, parkeringsbiljetter och dylika avgifter ersätts till fullo mot inlämnat kvitto. Observera dock att bensinkvitton inte ersätts då kostnaden för bränsleförbrukningen ingår i körersättningen.
\end{document}
