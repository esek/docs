\documentclass[10pt]{article}
\usepackage[utf8]{inputenc}
\usepackage[swedish]{babel}

\def\rikttitle{Hantering av kassadifferanser}
\def\antagen{2016-10-13}
%\def\uppdaterad{}

\usepackage{./e-riktlinje}
\usepackage{../e-styrdok}
\usepackage{../../e-sek}

\begin{document}
\section*{\doctitle}
\emph{Denna riktlinje innefattar hur kassadifferens konteras, varför den konteras på detta sätt, samt ställer krav på tydlig dokumentation av eventuella avvikelser i kassan. Riktlinjen innefattar också hur Sektionen ska jobba för att identifiera orsaken till avvikelser samt hur man vidtar åtgärder för att minimera avvikelser. Riktlinjen baseras på bokföringsnämndens vägledning (\S2.15 och \S2.16).}

\subsubsection*{Vid små positiva kassadifferenser upp t.o.m. 100kr vid enskilda tillfällen:}
I detta fall ska differensen konteras mot ett konto avsett endast för kassadifferenser - 3790 Övriga intäktskorrigeringar.
Dokumentationen sker löpande i bokföringen som en post under valt konto under respektive resultatenhet.
För att jobba för att få bort eventuella avvikelser av detta slag bör den ansvariga utskottschefen påpeka differenserna för de som hanterar/ska hantera kassaapparaten samt ge fortlöpande instruktioner/utbildningar till dem för att minimera risken för en återkommande avvikelse.

\subsubsection*{Vid större och/eller upprepade positiva kassadifferenser:}
I detta fall ska underlag tas fram som ger en förklaring till varför differensen uppstått och vad pengarna borde ha slagits in som i kassaapparaten.
För KM och E6 borde exempelvis rapporter från alkoholhanteringssystemet tas fram för att se hur stora intäkterna borde ha varit för varje alkoholgrupp.
Om differenserna skulle bero på tekniska problem räcker en skriftlig förklaring från ansvarig utskottschef som underlag.
Vid bokföring ska, om möjligt, differensen konteras mot de konton som intäkterna borde ha avsett, annars ska den konteras mot kontot 3790 Övriga intäktskorrigeringar.

\subsubsection*{Vid små negativa kassadifferenser upp t.o.m. 100kr vid enskilda tillfällen:}
I detta fall ska att differensen konteras mot ett konto avsett endast för kassadifferenser - 3710 Ofördelade intäktsreduceringar.
Dokumentationen sker löpande i bokföringen som en post under valt konto för respektive resultatenhet.
För att jobba för att få bort eventuella avvikelser av detta slag bör den ansvariga utskottschefen påpeka differenserna för de som hanterar/ska hantera kassaapparaten samt ge fortlöpande instruktioner/utbildningar till dem för att minimera risken för en återkommande avvikelse.

\subsubsection*{Vid större och/eller upprepade negativa kassadifferenser:}
I detta fall ska en utredning göras för att avgöra om en eventuell stöld kan ha förekommit. Styrelsen ska informeras och få ta del av innehållet i utredningen.
Om misstankar om brott framkommer ska händelsen polisanmälas och styrelsen ska vidta direkta åtgärder för att kunna plocka fram eventuella bevis.
Förslagsvis ska utredningen innehålla en förtäckning av tillfällena då differensen uppkommit samt vem eller vilka som har hanterat pengarna vid det tillfället.
När känsligt material så som till exempel upptagning från övervakningskameror eller utpekande uppgifter mot enskilda personer används är det extra viktigt att Sektionens styrdokument iakttas.
I fallet att Sektionens styrdokument inte täcker eventuellt material som anses vara känsligt bör materialet i fråga endast hanteras inom styrelsen och med revisorerna.

Gällande bokföringen, om fallet handlar om upprepade små negativa kassadifferenser, används samma metod som för små enskilda negativa kassadifferenser. Vid stora negativa kassadifferenser ska bokföringen utföras på samma sätt som med stora positiva kassadifferenser, med skillnaden att negativa differenser som inte kan placeras ut på rätt konto ska konteras mot konto 3710 Ofördelade intäktsreduceringar istället.

\end{document}
