\documentclass[10pt]{article}
\usepackage[utf8]{inputenc}
\usepackage[swedish]{babel}

\def\rikttitle{Affischeringsriktlinjer}
\def\antagen{2016-xx-xx}
%\def\uppdaterad{}

\usepackage{./e-riktlinje}
\usepackage{../e-styrdok}
\usepackage{../../e-sek}

\def\headr{Utkast - ej antagen}

\begin{document}
    \section*{\doctitle}
    \begin{dashlist}
        \item Alla affischer måste godkännas av Kontaktorn eller annan styrelsemedlem. Endast efter muntlig/skriftlig överrenskommelse fås affisher sättas upp på egen hand. Efter varje kalenderårs slut slutar dessa överenskommelser att gälla, eftersom den tillträdande styrelsen ska ha bra koll på vad som sätts upp på anslagstavlorna.

        \item Är det inte uppenbart när en affish ska tas ned bör den märkas med datum. Detta för att undvika utdaterad och/eller orelevant information på anslagstavlorna.

        \item Nya affischer lämnas i LED-café och sätts sedan upp av Kontaktorn efter denne har kontaktats, lämpligen via mail. Endast efter överrenskommelse med Kontaktorn eller annan styrelsemedlem får affischer sättas upp på egen hand.

        \item Kontaktorn ansvarar för att det finns information på varje anslagstavla om dess syfte och/eller eventuella regler för denna.

        \item Icke-ideella föreningar och företag måste gå genom näringslivsutskottet för att få lov att sätta upp affischer.

        \item Eftersom budskapet från alla affischer är något som Sektionen indirekt står för måste alla affischer följa publiceringspolicyn. Särskilt ska det understrykas att eftersom Sektionen är politiskt obunden får inga politiska budskap finnas.
    \end{dashlist}
\end{document}
