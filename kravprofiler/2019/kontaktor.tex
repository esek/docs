\documentclass[10pt]{article}
    \usepackage[utf8]{inputenc}
    \usepackage[swedish]{babel}
    
    \def\post{Kontaktor}
    
    \def\doctitle{Kravprofil för \post}
    \def\docauthor{Johan Karlberg, Axel Voss}
    \def\date{2018-10-02} %YYYY-MM-DD
    
    \usepackage{../../e-sek}
    \usepackage{./e-kravprofil}
    
    \begin{document}
    \heading{\doctitle}
    
    % summering av posten
    
    Kontaktorn är Sektionens sekreterare och har det övergripande ansvaret för Sektionens dokument och
    protokollföring av möten. Ansvar för Sektionens dokument innebär att hålla styrdokument
    uppdaterade samt att upprätta handlingar. Kontaktorn är även utskottsordförande för
    Informationsutskottet. Utskottet arbetar med informationsspridning och IT, vilket
    innebär att även detta är ett ansvarsområde.
    % konkretisering av ansvarsområden
    
    Ansvarsområden:
    \begin{dashlist}
        \item Är Sektionens sekreterare och har det övergripande ansvaret för Sektionens dokument och
        protokollföring av möten
        \item Ansvarar för att Sektionens styrdokument hålls aktuella
        \item Ansvarar för att upprätta handlingar till Sektionsmötena
        \item Har det övergripande ansvaret för Sektionens informationsspridning och PR-verksamhet
        \item Ansvarar för Sektionens kontakt med andra sektioner, såväl inom TLTH som på andra
        högskolor och universitet
        \item Är ansvarig utgivare för HeHE
        %\item Ansvarar för att examenstavlorna årligen uppdateras innan höstterminen börjar
        \item Att aktivt delta i TLTH:s informationskollegie
    \end{dashlist}
    
    % summering av profilen
    En god Kontaktor är en person som är påläst på Sektionens dokument, samt en person som
    har god förmåga för tidsplanering. Den är en person som är bra på att delegera
    arbetsuppgifter inom utskottet. 
    % konkretisering av de viktigaste kraven i stort
    
    I korthet kan de viktigaste styrkorna hos en god {\post} sammanfattas som:
    \begin{dashlist}
        \item God samarbetsförmåga
        \item God ledarskapförmåga, bra på att fördela och delegera
        \item God förmåga för tidsplanering
    \end{dashlist}
    
    %Direkta krav:
    %\begin{dashlist}
    %    \item Postinnehavaren måste ha fyllt minst 20 år inför mandatperioden
    %    \item Övriga krav enligt E-sektionens styrdokument
    %\end{dashlist}
    
    % mer specifika saker som är bra att ha, men ej nödvändiga
    
    Önskvärda erfarenheter:
    \begin{dashlist}
        \item \LaTeX
        \item Git
        \item Generell kunskap inom IT och teknik. 
    %    \item Tidigare engagemang inom sektionen
    \end{dashlist}
        
    \end{document}
    