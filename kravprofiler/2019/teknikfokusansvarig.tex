\documentclass[10pt]{article}
\usepackage[utf8]{inputenc}
\usepackage[swedish]{babel}

\def\post{Teknikfokusansvarig}

\def\doctitle{Kravprofil för \post}
\def\docauthor{Johan Vikstrand}
\def\date{2019-03-04} %YYYY-MM-DD

\usepackage{../../e-sek}
\usepackage{./e-kravprofil}

\begin{document}
\heading{\doctitle}

% summering av posten
Som sektionens Näringslivsordförande leder man näringslivsutskottet och organiserar olika evenemang med företag och organisationer. 
Evenemang kan vara lunchföreläsningar, case-kvällar, pubar med mera. En annan uppgift kan vara att fixa sponsring till sektionen om det behövs. 
Som Näringslivsordförande är man ett ansikte utåt mot företagsvärlden och representerar studenterna och deras intressen. 
Även alumniverksamheten ligger under utskottet.
% konkretisering av ansvarsområden

Ansvarsområden:
\begin{dashlist}
    \item Att tillsammans med D-sektionens ansvarig tillsätta en projektgrupp
    \item Svara för sektionens intresse i näringslivsinitiativet Teknikfokus
    \item Planera, Budgetera, Organisera och slutligen exekvera Teknikfokus verksamhet.
    \item Upprätthålla och skapa kontakter med företag
    \item Hålla god kontakt med E-husets vaktmästeri och husprefekt
    \item Samarbeta nära med Näringslivordförande från både E- och D-sektionen
\end{dashlist}

% summering av profilen
En bra Teknikfokusansvarig ska alltid ha sektionen och projektets bästa i åtanke vid beslut och agera
därefter. Man ska kunna föra sig på ett professionellt sett mot företag såväl som studenter i både
skriftlig och muntlig kommunikation
% konkretisering av de viktigaste kraven i stort

Goda egenskaper att inneha som {\post} är:
\begin{dashlist}
    \item Målmedveten
    \item Bra planerare och duktig på att genomföra sin plan
    \item Ha en känsla för när man ska kompromissa eller stå på sig
    \item Samarbetsvillig
    \item Bra på att delegera
\end{dashlist}

\end{document}
    