\documentclass[10pt]{article}
\usepackage[utf8]{inputenc}
\usepackage[swedish]{babel}

\def\post{Vice Krögare}

\def\doctitle{Kravprofil för \post}
\def\docauthor{Klara Indebetou, Hjalmar Tingberg\\Filip Winzell, Stephanie Bol\\ Saga Juniwik, Oscar Uggla}
\def\date{2019-10-23} %YYYY-MM-DD

\usepackage{../../e-sek}
\usepackage{./e-kravprofil}

\begin{document}
\heading{\doctitle}

% summering av postenCophøsens uppgift är att bistå Øverphøset i dennes arbete. Denna posten är en vice till utskottsordförande i nollningsutskottet. Cophøsen planerar och organiserar tillsammans  med resterande NollU nollningen. De olika ansvarsområden som phøset behöver hantera brukar delas upp mellan dem och det är deras egna ansvar att dela upp dessa områden.
I Källarmästeriet finns det två Vice Krögare. De planerar, organiserar och håller i E-sektionens gillen (pubar) tillsammans med Krögaren. Till sin hjälp har de även två Cøl, som ansvarar främst för inköp av alkohol och ett stort gäng med källarmästare som är KMs jobbare. Även om Krögaren har det största ansvaret, brukar arbetet delas upp jämnt mellan personerna i krögartrion vad gäller planering och genomförande av gillena. En Vice Krögare ska kunna hålla i ett gille på egen hand, utan de andra i krögartrion (även om detta nästan aldrig händer). Du bör även vara tillgänglig majoriteten av fredagar under året med undantag för inläsningsveckor. Gillenas utformning är i stort sett densamma hela tiden, men varje år är det nya teman och det finns friheter att komma med nya idéer vad gäller mat, musik, dekor, aktiviteter etc.  
% konkretisering av ansvarsområden

Ansvarsområden:
\begin{dashlist}
    \item Hålla god kontakt med Krögare och jobbare
    \item Bistå Krögaren med de arbetsuppgifter som finns innan ett gille
    \item Kunna hålla ett gille utan krögaren
    \item Komma med nya egna idéer som utvecklar gillena
\end{dashlist}

Som Vice Krögare är det viktigt att både kunna vara en bra samt rättvis ledare när arbete delas ut till källarmästarna. Dessutom är det viktigt att vara lyhörd och samarbetsvillig då en ska kunna följa Krögarens direktiv. God samarbetsförmåga och en vilja att dela arbetet rättvist är därför nödvändigt. Då krögartrion jobbar nära inpå varandra under ett års tid, är det även viktigt att kunna hantera konflikter som uppstår. Ditt jobb som Vice-Krögare är att disponera jobbet och inte “jobba”. Ditt arbete kommer att vara väldigt mångsidigt, från att planera menyer för mat och dryck till att kommunicera med dina arbetare och diverse andra utskott för att gillena ska bli de bästa som de kan bli. 

I korthet kan de viktigaste egenskaperna sammanfattas som:
\begin{dashlist}
    \item En bra lagspelare
    \item Bra ledarskap
    \item Rättvis
    \item Kreativ
    \item Samarbetsvillig
    \item Organiserad
    \item Stresstålig    
\end{dashlist}

Önskvärda Erfarenheter och Egenskaper (men ej nödvändiga):
\begin{dashlist}
    \item Tidigare erfarenhet från bar- och köksverksamhet.
    \item Fyllt 20 år för att kunna stå på alkoholtillståndet        
\end{dashlist}
\end{document}
    