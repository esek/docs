\documentclass[10pt]{article}
\usepackage[utf8]{inputenc}
\usepackage[swedish]{babel}

\def\post{Barmästare}

\def\doctitle{Kravprofil för \post}
\def\docauthor{Theo Nyman}
\def\date{2019-10-23} %YYYY-MM-DD

\usepackage{../../e-sek}
\usepackage{./e-kravprofil}

\begin{document}
\heading{\doctitle}

Barmästarna är en del av sexmästeriet och svarar till utskottsordföranden. Huvuduppgiften för barmästarna är att tillgodose sittningar med ett passande utbud av både alkoholhaltiga och icke-alkoholhaltiga drycker. Som barmästare har man ansvar för baren under sittningar. Man sköter även planering och inhandling av alkohol. Det tillkommer även att vara behjälplig vid inventering av alkoholförrådet. 

De ansvar man har som barmästare enligt reglementet är:
\begin{dashlist}
    \item Ansvar för baren under och innan sittningar, 
    \item att sköta beställning, inköp och inventering av alkoholdryck för Sexmästeriet, samt
    \item ansvar för att prissättning av ovan nämnda drycker följer alkohollagen.        
\end{dashlist}

En viktig del i barmästarnas jobb är att se till att Sexmästeriet sköter alkoholhantering på ett korrekt sätt under evenemang. Detta är viktigt då Sexmästeriet delar alkoholförråd med Källarmästeriet. För att utföra detta behöver en barmästare vara noggrann, tålmodig och ansvarsfull. Förutom detta ligger ett stort ansvar hos barmästarna att ha koll på gästers alkoholnivå. Som barmästare behöver man därför inte ha problem med att säga nej till att servera någon alkohol. Barmästarna tillsammans behöver också kunna lita på varandra att de har någon som står bakom dem i deras beslut.

Sammanfattningsvis bör en barmästare vara:
\begin{dashlist}
    \item Ansvarsfull i alkoholhantering,
    \item noggrann och tålmodig, samt
    \item stöttande.        
\end{dashlist}

Direkta krav:
\begin{dashlist}
    \item Minst en barmästare behöver vara över 20 år, gärna båda.     
\end{dashlist}


\end{document}
    