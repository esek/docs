\documentclass[10pt]{article}
\usepackage[utf8]{inputenc}
\usepackage[swedish]{babel}

\def\post{Krögare}

\def\doctitle{Kravprofil för \post}
\def\docauthor{Davida Åström}
\def\date{2019-10-08} %YYYY-MM-DD

\usepackage{../../e-sek}
\usepackage{./e-kravprofil}

\begin{document}
\heading{\doctitle}

Krögaren organiserar E-sektionens pubverksamhet och genomför pubar i Edekvata för sektionen och teknologkårens övriga medlemmar. Till Krögarens hjälp finns Källarmästeriet, som innehåller två Vice Krögare, två inköpsansvariga för alkohol (Cøl) samt ett helt gäng källarmästare.
Som Krögare måste man ha bred kunskap om Edekvatas kök och gillen samt även kunna leda alla delar av sitt utskott. Gillenas utformning är i stort sett samma från år till år, men det finns även frihet i att bestämma tema, musik, mat, dryck, dekor, aktiviteter osv. 
Förutom att vara utskottsordförande är Krögaren även en del av styrelsen och förväntas arbeta aktivt för sektionens bästa. 

Ansvarsområden:
\begin{dashlist}
    \item Organisera utskottets verksamhet
    \item Hålla god kontakt med alla funktionärer i Källarmästeriet
    \item Bokföring och ekonomisk uppföljning
    \item Delta aktivt i Styrelsens arbete
    \item Delta i TLTHs sexmästarkollegium
\end{dashlist}

En god Krögare är en kreativ och lösningsorienterad person som inte är rädd för oväntade händelser. 
Den är också en bra ledare, eller aspirerar att bli en sådan. Krögaren bör dessutom vara en person som kan representera sektionen utåt på ett positivt sätt. 
Ett gott sinne för mat och dryck, god samarbetsförmåga och en vilja att dela arbetet rättvist är också nödvändigt. 
Krögaren förväntas även skaffa sig god förståelse över kökets utrustning, såsom stekbord och fatanläggning, men även annan utrustning och system såsom proppskåp, iZettle, AHS och diverse av InfUs moduler. Då krögartrion är en liten grupp som behöver samarbeta mycket och på ett bra sätt under hela året är god social kompetens och konflikthanteringsförmånga mycket fördelaktigt.

I korthet kan de viktigaste egenskaperna sammanfattas som:
\begin{dashlist}
    \item Bra ledarskapsförmåga
    \item Rättvis
    \item Diplomatisk
    \item Samarbetsvillig
    \item Organiserad
    \item God konflikthanteringsförmåga 
\end{dashlist}

Direkta krav:
\begin{dashlist}
    \item Postinnehavaren måste ha fyllt minst 20 år inför mandatperioden
    \item Övriga krav enligt E-sektionens styrdokument
\end{dashlist}

Önskvärda erfarenheter (men ej nödvändiga):
\begin{dashlist}
    \item Tidigare erfarenhet från bar- och köksverksamhet.
    \item Genomförd alkoholutbildning som A-cert är meriterande. (Vidare alkoholutbildning kommer erbjudas under våren)
\end{dashlist}

\end{document}