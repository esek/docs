\documentclass[10pt]{article}
\usepackage[utf8]{inputenc}
\usepackage[swedish]{babel}

\def\post{Köksmästare}

\def\doctitle{Kravprofil för \post}
\def\docauthor{Theo Nyman}
\def\date{2019-10-23} %YYYY-MM-DD

\usepackage{../../e-sek}
\usepackage{./e-kravprofil}

\begin{document}
\heading{\doctitle}

% summering av postenCophøsens uppgift är att bistå Øverphøset i dennes arbete. Denna posten är en vice till utskottsordförande i nollningsutskottet. Cophøsen planerar och organiserar tillsammans  med resterande NollU nollningen. De olika ansvarsområden som phøset behöver hantera brukar delas upp mellan dem och det är deras egna ansvar att dela upp dessa områden.
Köksmästarna är en del av sexmästeriet och svarar till utskottsordföranden. Huvuduppgiften som köksmästare är att planera och genomföra menyer inför sittningar. Detta görs gärna självgående med sexmästaren som stöd. Tillsammans med sexmästaren ser man till att hålla budget för sittningen. Man har som köksmästare väldigt fria händer att laga den mat man själv tycker är kul att bjuda på. 
% konkretisering av ansvarsområden
De ansvar man har som köksmästare enligt reglementet är:
\begin{dashlist}
    \item Planering av meny och inhandling av mat innan sittningen,
    \item sköta matlagningen och delegeringen av uppgifterna i köket under sittningen. Bistå sexmästaren vid planering av menyn.        
\end{dashlist}

Planering och budgetering av menyer för många personer kräver god framförhållning och samarbetsförmåga. Som köksmästare bör man vara lösningsorienterad för att tackla problem som kan uppstå under en sittning. Man bör också vara bekväm med att fråga om hjälp och söka den informationen man behöver för att vara så förberedd som möjligt. Förutom detta är ett matintresse en god egenskap. 

Sammanfattningsvis bör en köksmästare vara:
\begin{dashlist}
    \item Bra på att hålla god framförhållning,
    \item tydlig i sin kommunikation med dem andra i utskottet, samt
    \item intresserad för mat.    
\end{dashlist}    

\end{document}
    