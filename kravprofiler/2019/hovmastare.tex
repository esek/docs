\documentclass[10pt]{article}
\usepackage[utf8]{inputenc}
\usepackage[swedish]{babel}

\def\post{Hovmästare}

\def\doctitle{Kravprofil för \post}
\def\docauthor{Theo Nyman}
\def\date{2019-10-23} %YYYY-MM-DD

\usepackage{../../e-sek}
\usepackage{./e-kravprofil}

\begin{document}
\heading{\doctitle}

Hovmästarna är en del av sexmästeriet och svarar till utskottsordföranden. Som hovmästare har man i huvuduppgift att planera dukning och servering. Till detta tillkommer det ganska naturligt att man även har hand om dekor under sittningar. Vissa sittningar kräver mer planering och vissa mindre. Det kan variera i huruvida det ska vara bordsplacering eller om det är en sittning med ett tema som kräver mer dekor och genomtanke. 

De ansvar man har som hovmästare enligt reglementet är:
\begin{dashlist}
    \item Ansvarar för dukning och servering på sittningen.    
\end{dashlist}

Tillsammans planerar man dekor för en given budget av sexmästaren. God samarbetsförmåga och arbetsledning är viktiga egenskaper för hovmästare då man behöver strukturera jobbet för dem sexiga. Det är också bra att vara tydlig i sin kommunikation.

Sammanfattningsvis bör en hovmästare vara:
\begin{dashlist}
    \item Tydlig och strukturerad i sin kommunikation och arbetsfördelning,
    \item bra på att planera, samt 
    \item flexibel efter behov.    
\end{dashlist}
\end{document}
    