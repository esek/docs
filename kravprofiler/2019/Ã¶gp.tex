\documentclass[10pt]{article}
\usepackage[utf8]{inputenc}
\usepackage[swedish]{babel}

\def\post{Øvergudsphadder}

\def\doctitle{Kravprofil för \post}
\def\docauthor{Richard Byström, Linnea Söderström}
\def\date{2019-10-14} %YYYY-MM-DD

\usepackage{../../e-sek}
\usepackage{./e-kravprofil}

\begin{document}
\heading{\doctitle}

% summering av posten
En Øvergudsphadder hjälper Phøset i planeringen av nollningen. Under nollningen ser ØGP till att nollor och phaddrar är taggade, samt uppmuntrar till de olika nollningsevenemangen. 
De två senaste åren har ØGP även varit med i Pepparkollegiet, där man tillsammans med andra sektioners motsvarighet till posten planerar intersektionella aktiviteter. Då ansvarar man för att sköta kommunikationen mellan kollegiet och utskottet. 
Under nollningen så är man även som ØGP en informationslänk mellan nollor/phaddrar och Phøset. 
Detta innebär även att man måste vara beredd på att lägga ned tid utanför skolan under hela året.

% konkretisering av ansvarsområden

Ansvarsområden:
\begin{dashlist}
    \item Vara med i utskottets arbete
    \item Delta aktivt i Pepparkollegiet
    \item Under nollningen vara ett ansikte utåt för phaddrar och nollor, samt en kommunikationslänk
    \end{dashlist}

% summering av profilen
En bra ØGP är ansvarstagande, alltid redo, taggad och glad. Man bör även delta aktivt i utskottets planering och vara beredd att, vid diskussion, dela med sig av sina egna åsikter.
Under nollningen ska man kunna ta både kommandon och egna beslut. Som ØGP ska man även kunna sköta en tydlig kommunikation med nollor/phaddrar, vara beredd att svara på frågor och snabbt kunna dela med sig av information. 
Det är också viktigt att vara spontan och exempelvis kan underhålla vid behov. Därför bör man känna sig bekväm i sig själv och vara beredd att bjuda på sig. Under nollningen förväntas man även alltid vara på plats, redo att lösa problem, och det är därför viktigt att vara stresstålig. 
Som ØGP ska man även vara medveten om att man måste vara en bra förebild för nollor och phaddrar, då man har ett stort inflytande under nollningen.
% konkretisering av de viktigaste kraven i stort

I korthet kan de viktigaste egenskaperna hos ett ØGP sammanfattas som:
\begin{dashlist}
    \item Glad och taggad
    \item God kommunikationsförmåga
    \item Spontan
    \item Problemlösare
    \item Stresstålig
    \item En bra förebild för nollor och phaddrar
\end{dashlist}

\end{document}       