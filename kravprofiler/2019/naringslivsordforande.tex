\documentclass[10pt]{article}
\usepackage[utf8]{inputenc}
\usepackage[swedish]{babel}

\def\post{Näringslivsordförande}

\def\doctitle{Kravprofil för \post}
\def\docauthor{Jakob Pettersson}
\def\date{2019-10-08} %YYYY-MM-DD

\usepackage{../../e-sek}
\usepackage{./e-kravprofil}

\begin{document}
\heading{\doctitle}

% summering av posten
Som sektionens Näringslivsordförande leder man näringslivsutskottet och organiserar olika evenemang med företag och organisationer. Evenemang kan vara lunchföreläsningar, case-kvällar, pubar med mera. En annan uppgift kan vara att fixa sponsring till sektionen om det behövs. 
Som Näringslivsordförande är man ett ansikte utåt mot företagsvärlden och representerar studenterna och deras intressen. Även alumniverksamheten ligger under utskottet.
% konkretisering av ansvarsområden

Ansvarsområden:
\begin{dashlist}
    \item Organisera utskottets verksamhet tillsammans med sin vice
    \item Planera och hålla i utskottsmöten
    \item Organisera evenemang
    \item Upprätthålla kontakterna med företagsrepresentanter
    \item Ansvarar för arbetsmarknadsmässan Teknikfokus tillsammans med D-sektionen
    \item Samarbetar med Phøset under nollningen angående all form av spons
    \item Skicka fakturor
\end{dashlist}

% summering av profilen
En bra {\post} ska kunna ta beslut och fördela arbetsbördan i utskottet. Man ska kunna agera professionellt mot företag och vara kontaktbar. Näringslivsordföranden ska alltid ha studenternas intresse i åtanke. 
% konkretisering av de viktigaste kraven i stort

Goda egenskaper att inneha som {\post} är:
\begin{dashlist}
    \item Trevlig
    \item Bra på att kompromissa
    \item Kreativ
    \item Kunna fatta beslut
    \item Samarbetsvillig
\end{dashlist}

\end{document}
    