\documentclass[10pt]{article}
\usepackage[utf8]{inputenc}
\usepackage[swedish]{babel}

\def\post{Preferensmästare}

\def\doctitle{Kravprofil för \post}
\def\docauthor{Theo Nyman}
\def\date{2019-10-23} %YYYY-MM-DD

\usepackage{../../e-sek}
\usepackage{./e-kravprofil}

\begin{document}
\heading{\doctitle}

Preferensmästaren är en del av sexmästeriet och svarar till utskottsordföranden. Huvuduppgiften för preferensmästaren är att utforma alternativa menyer för de som har specialkost inför sittningar. Detta görs i samråd med köksmästarna som sätter menyn. 

De ansvar som preferensmästare har enligt reglementet är:
\begin{dashlist}
    \item Att planera meny och inhandling av specialkost innan sittningar, samt
    \item sköta tillagningen av specialkost och delegeringen av uppgifterna i köket under sittningen
\end{dashlist}
Som preferensmästare är det viktigt att personer med allergier och specialkost får rätt mat. Detta kan vara krävande och lärorikt då det finns många olika allergier och preferenser att tillgodose. Som preferensmästare bör man därför ha intresse av specialkost. För att klara av detta behöver man även vara noggrann i sin planering.

Sammanfattningsvis bör en preferensmästare vara:
\begin{dashlist}
    \item noggrann och
    \item intresserad av specialkost.        
\end{dashlist}

\end{document}
    