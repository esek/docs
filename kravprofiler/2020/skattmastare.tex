\documentclass[10pt]{article}
\usepackage[utf8]{inputenc}
\usepackage[swedish]{babel}

\def\post{Skattmästare}

\def\doctitle{Kravprofil för \post}
\def\docauthor{Daniel Bakic}
\def\date{2020-08-11} %YYYY-MM-DD

\usepackage{../../e-sek}
\usepackage{./e-kravprofil}

\begin{document}
\heading{\doctitle}

% summering av posten

Bistår Förvaltningschefen i dess arbete gällande sektionens ekonomi och ansvarar sektionens bokföring tillsammans med Förvaltningschefen.

 % konkretisering av ansvarsområden

Arbetsuppgifter:
\begin{dashlist}
    \item Lägga in redovisningar, inbetalningar, kvittoförstärkningar och dylikt i sektionens bokföringsprogram Fortnox
    \item Kontrollera redovisningar, inbetalningar, kvittoförstärkningar och dylikt för eventuella fel
    \item Anteckna verifikationsnummer för utdrag eller insättning i bank
\end{dashlist}

% summering av profilen
En god {\post} är ofta i kontakt med Förvaltningschefen gällande sitt arbete och tvekar inte att fråga då problem eller ovissheter dyker upp. Skattmästarens arbete omfattar flera timmar per vecka, det är därför viktigt att postinnehavaren arbetar kontinuerligt för att se till att inte för mycket pappersarbete blir liggande. Då postens arbete omfattar pengar är det också viktigt att Skattmästaren är pålitlig.
% konkretisering av de viktigaste kraven i stort

I korthet kan de viktigaste styrkorna hos en god {\post} sammanfattas som:
\begin{dashlist}
    \item Organiserad
    \item Flitig
    \item Pålitlig
    \item God kommunikationsförmåga
\end{dashlist}

Önskvärda erfarenheter (men ej nödvändiga)
\begin{dashlist}
    \item Företagsekonomi
\end{dashlist}

\end{document}