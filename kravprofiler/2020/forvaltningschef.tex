\documentclass[10pt]{article}
\usepackage[utf8]{inputenc}
\usepackage[swedish]{babel}

\def\post{Förvaltningschef}

\def\doctitle{Kravprofil för \post}
\def\docauthor{Henrik Ramström}
\def\date{2019-10-09} %YYYY-MM-DD

\usepackage{../../e-sek}
\usepackage{./e-kravprofil}

\begin{document}
\heading{\doctitle}

% summering av posten
Förvaltningschefen är sektionens kassör med ytterligare ansvar över, lokaler, inventarier och klenoder. Detta innefattar att Förvaltningschefen också finns där för alla utskott och stöttar dem i deras ekonomiarbete.

Förvaltningschefen är även utskottsordförande för Förvaltningsutskottet och arbetar därmed med att skapa en god känsla i utskottet samt ansvarar för att alla i utskottet vet vad de ska göra.


% konkretisering av ansvarsområden

Arbetsuppgifter:
\begin{dashlist}
	\item Bokför tillsammans med Skattmästaren
	\item Betalar sektionens fakturor
	\item Fakturerar
	\item Är firmatecknare
	\item Utbildar Styrelsen och funktionärer i sektionens ekonomi
	\item Värnar om en bra sammanhållning i utskottet
	\item Gör halv- och helårsbokslut
	\item Övergripande ansvar för sektionens lokaler
	\item Går på kollegiemöte
	\item Söker alkoholtillstånd inför evenemang
	\item Ansvarar för hanteringen av nycklar och access till berörda funktionärer
	\item Ansvarar för ekonomisk rapportering till styrelsen och sektionen
\end{dashlist}

% summering av profilen
En god Förvaltningschef är ordningssam och strukturerad. Man är en god ledare och har förmågan att samla ett utskott med många olika arbetsområden.
Kunskaper inom ekonomi och bokföring är fördelaktigt, men ej nödvändigt. Istället är intresset och viljan att lära sig och att göra ett ordentligt arbete av högsta vikt! Detta då Förvaltningschefens arbete kan upplevas enformigt och tar mycket tid, även efter mandatperioden är över.
% konkretisering av de viktigaste kraven i stort

Direkta krav:
\begin{dashlist}
	\item Postinnehavaren måste ha fyllt minst 20 år inför mandatperioden
	\item Uppfyller kraven för att kunna vara firmatecknare
	\item Övriga krav enligt E-sektionens styrdokument
\end{dashlist}

Frågor om uppgifter i belastningsregistret kan komma att uppstå i samband med valberedningsprocessen för posten.

\end{document}
