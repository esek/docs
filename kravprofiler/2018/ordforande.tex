\documentclass[10pt]{article}
\usepackage[utf8]{inputenc}
\usepackage[swedish]{babel}

\def\post{Ordförande}

\def\doctitle{[UTKAST] Kravprofil för \post}
\def\docauthor{Erik Månsson, Daniel Bakic}
\def\date{2018-07-29} %YYYY-MM-DD

\usepackage{../../e-sek}
\usepackage{./e-kravprofil}

\begin{document}
\heading{\doctitle}

% summering av posten

Sektionens Ordförande leder arbetet i styrelsen som utöver Ordföranden själv även består utav de nio utskottsordförandena.
Som Ordförande är man som spindeln i nätet som har lite koll på mycket och försöker ge övriga funktionärer de bästa möjligheterna för att göra ett bra jobb.
Ordförande har en ganska fri roll och det finns stora möjligheter att dra i precis dem frågor som man själv är intresserad av!
%Riktad till medlemmar
%hålla tal

% konkretisering av ansvarsområden

Ansvarsområden:
\begin{dashlist}
    \item Organisera styrelsens verksamhet
    \item Planera och hålla i styrelsemöten
    \item Värna om styrelsens och sektionens sammanhållning
    \item Planera och skriva handlingar för sektionsmöten
    \item Hålla sektionen uppdaterad gentemot de andra sektionerna och Kåren centralt
    \item Representera Sektionen på evenemang
    \item Hålla god kontakt med nyckelpersoner i och utanför E-huset, exempelvis Husprefekt, Husintendent, Inspektor, Kårordförande, m.fl.
    \item Att hantera oväntade situationer som uppstår, exempelvis vid olycksfall 
\end{dashlist}

% summering av profilen

En god Ordförande är en stark ledare som lyssnar till styrelsens och medlemmarnas viljor.
Någon som strävar efter att utveckla och föra sektionen framåt i bästa mån. 
Någon som är duktig på att hålla många bollar i luften och vill hålla sig uppdaterad om vad som händer på sektionen.
Någon som är duktig på att föra diskussioner och se till att allas åsikter blir hörda.
Någon som är duktig på att skapa god gruppdynamik inom styrelsen.
Bra på att agera länk mellan olika utskott.
Bekväm med att ta beslut.

% studiemässigt god förebild

% konkretisering av de viktigaste kraven i stort

De viktigaste styrkorna hos en god Ordförande är:
\begin{dashlist}
    \item God ledarskapförmåga %Exempelvis fördela o delegera
    \item God kommunikationsförmåga
    \item God organisationsförmåga
    \item Stor erfarenhet av Sektionen i sin helhet
    \item Bekväm med att ta beslut
\end{dashlist}

Krav:
\begin{dashlist}
    \item Måste ha fyllt minst 20 år inför mandatperioden
    \item Krav enligt E-sektionens styrdokument
\end{dashlist}
% mer specifika saker som är bra att ha, men ej nödvändiga

Önskvärt men ej nödvändigt:
\begin{dashlist}
    \item Företagsekonomi
    \item \LaTeX
    \item Git
    \item Tidigare engagemang inom sektionen
\end{dashlist}

\end{document}
