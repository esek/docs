\documentclass[10pt]{article}
\usepackage[utf8]{inputenc}
\usepackage[swedish]{babel}

\def\post{Ordförande}

\def\doctitle{Kravprofil för \post}
\def\docauthor{Erik Månsson, Daniel Bakic}
\def\date{2018-08-14} %YYYY-MM-DD

\usepackage{../../e-sek}
\usepackage{./e-kravprofil}

\begin{document}
\heading{\doctitle}

% summering av posten

Sektionens Ordförande leder och organiserar styrelsen som utöver Ordföranden själv består de nio utskottsordförandena.
Ordföranden är som spindeln i nätet och har lite koll på mycket, och jobbar för att ge övriga funktionärer de bästa förutsättningarna för att göra ett bra jobb.
Posten har i allmänhet en ganska fri roll och ger stora möjligheter för Ordföranden att ta upp de frågor som den själv finner intressanta och viktiga.

% konkretisering av ansvarsområden

Ansvarsområden:
\begin{dashlist}
    \item Organisera styrelsens verksamhet
    \item Planera och hålla i styrelsemöten
    \item Värna om styrelsens och sektionens sammanhållning
    \item Planera och skriva handlingar inför sektionsmöten
    \item Hålla sektionen uppdaterad gentemot de andra sektionerna och Kåren centralt
    \item Representera sektionen på evenemang och tillställningar
    \item Tala för sektionen på vissa evenemang och sammanträden
    \item Hålla god kontakt med nyckelpersoner i och utanför E-huset, exempelvis Husprefekt, Husintendent, Inspektor, Kårordförande, m.fl.
    \item Att hantera oväntade situationer som uppstår, exempelvis vid olycksfall
\end{dashlist}

% summering av profilen

En god Ordförande är en stark ledare som lyssnar till styrelsens och medlemmarnas viljor.
Den är en framåtsträvande person som brinner för att utveckla och föra sektionen framåt i bästa mån.
Personen är bra på att hålla många bollar i luften och håller sig uppdaterad om vad som händer i alla delar av sektionen.
Den är någon som är duktig på att föra diskussioner och ser till att allas åsikter blir hörda.
%Den jobbar också aktivt för att skapa god dynamik inom styrelsen och att agera länk mellan de olika utskotten.
En god Ordförande är också en god förebild studiemässigt, och jobbar aktivt med frågor rörande studierna på LTH och allt som hör därtill.

% konkretisering av de viktigaste kraven i stort

I korthet kan de viktigaste styrkorna hos en god {\post} sammanfattas som:
\begin{dashlist}
    \item God kommunikationsförmåga
    \item God ledarskapförmåga, bra på att fördela och delegera
    \item God organisationsförmåga
    \item Stor erfarenhet av sektionen i sin helhet
    \item Bekväm med att ta beslut
\end{dashlist}

Direkta krav:
\begin{dashlist}
    \item Postinnehavaren måste ha fyllt minst 20 år inför mandatperioden
    \item Övriga krav enligt E-sektionens styrdokument
\end{dashlist}

% mer specifika saker som är bra att ha, men ej nödvändiga

Önskvärda erfarenheter (men ej nödvändiga):
\begin{dashlist}
    \item Företagsekonomi
    \item \LaTeX
    \item Git
%    \item Tidigare engagemang inom sektionen
\end{dashlist}

Frågor om uppgifter i belastningsregistret kan komma att uppstå i samband med valberedningsprocessen för posten.

\end{document}
