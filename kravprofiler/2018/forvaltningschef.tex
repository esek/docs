\documentclass[10pt]{article}
\usepackage[utf8]{inputenc}
\usepackage[swedish]{babel}

\def\post{Förvaltningschef}

\def\doctitle{Kravprofil för \post}
\def\docauthor{Sophia Grimmei{\ss} Grahm, Magnus Lundh}
\def\date{2018-10-09} %YYYY-MM-DD

\usepackage{../../e-sek}
\usepackage{./e-kravprofil}

\begin{document}
\heading{\doctitle}

% summering av posten

En {\post} har hand om sektionens ekonomi, bokföring, lokaler och inventarier. Förvaltningschefen finns även där för övriga utskott och stöttar dem i deras ekonomiarbete. Förvaltningschefen är utskottsordförande för Förvaltningsutskottet och arbetar för att skapa en god gemenskap i utskottet.
% konkretisering av ansvarsområden

Arbetsuppgifter:
\begin{dashlist}
    \item Bokför tillsammans med Skattmästaren
    \item Betalar sektionens fakturor
    \item Fakturerar
    \item Är firmatecknare
    \item Utbildar Styrelsen och funktionärer i sektionens ekonomi
    \item Värnar om en bra sammanhållning i utskottet
    \item Gör halv- och helårsbokslut
    \item Övergripande ansvar för sektionens lokaler
    \item Går på kollegiemöte
    \item Söker alkoholtillstånd inför evenemang
    \item Ansvarar för hanteringen av nycklar och access till berörda funktionärer
\end{dashlist}

% summering av profilen
En god {\post} är ordningssam och strukturerad. Man är en god ledare och har förmågan att samla ett utskott med många olika arbetsområden. Kunskaper inom ekonomi och bokföring är fördelaktigt, men ej nödvändigt. Intresset och viljan att lära sig är desto viktigare då Förvaltningschefens arbete kan upplevas enformigt och tar mycket tid, även efter mandatperioden.
% konkretisering av de viktigaste kraven i stort

Direkta krav:
\begin{dashlist}
    \item Postinnehavaren måste ha fyllt minst 20 år inför mandatperioden
    \item Övriga krav enligt E-sektionens styrdokument
\end{dashlist}

\end{document}       