\documentclass[10pt]{article}
\usepackage[utf8]{inputenc}
\usepackage[swedish]{babel}

\def\post{Entertainer}

\def\doctitle{Kravprofil för \post}
\def\docauthor{Albin Nyström Eklund, Adam Belfrage}
\def\date{2018-10-09} %YYYY-MM-DD

\usepackage{../../e-sek}
\usepackage{./e-kravprofil}

\begin{document}
\heading{\doctitle}

% summering av posten

Som {\post} är man ordförande för Nöjesutskottet som samordnar aktiviteter efter skoltid för sektionens medlemmar. Man fungerar i princip som sektionens ständiga underhållare vilket framkommer av namnet. Aktiviteterna bidrar till gemenskap bland sektionens medlemmar och ger medlemmarna en chans att umgås med folk från olika årskurser. Ibland arrangeras aktiviteter tillsammans med andra sektioner för att stärka gemenskapen med dessa. Det kan vara kul att vidga sina vyer till andra sektioner också! Gemenskap är viktigt. Målet med aktiviteterna är att få en välkomnande och god stämning inom sektionen. En utmaning för Entertainern är att få hela sektionen aktiv och närvarande vid aktiviteterna, även de som inte är aktiva medlemmar. Entertainern strävar mot en sektion där alla känner sig välkomna genom att arrangera stor variation bland evenemang som är tillgängliga för en bred publik. Det ska alltså finnas något för alla.  Som ordförande för Nöjesutskottet sitter man med i styrelsen och förväntas att delta aktivt i styrelsen för att driva sektionen framåt.

% konkretisering av ansvarsområden

Ansvarsområden:
\begin{dashlist}
    \item Har det övergripande ansvaret för Sektionens kultur-, nöjes- och fritidsaktiviteter.
    \item Ansvarar för Sektionens instrument.
    \item Ansvarar för planering och genomförandet av UtEDischot tillsammans med D-sektionen.
    \item Ansvarar för utkvittering av access till biljard- och pingisskåpet.
\end{dashlist}

% summering av profilen

En bra entertainer är en ansvarsfull person som brinner för att andra ska ha det roligt. De uppstyrda aktiviteterna ska vara roliga men också välorganiserade och strukturerade. Som entertainer bör man vara framåtsträvande och driven till att arbeta för hur man kan förbättra och förnya sektionen med olika evenemang.  En entertainer bör vara samarbetsvillig och ha ett öppet sinne för att göra det lätt att samarbeta inom sitt utskott men också för att underlätta samarbetet med andra sektioner. God social kompetens och ödmjukhet är också bra egenskaper att besitta då det är viktigt att man kan se till att alla trivs och känner sig inkluderade på evenemangen. Ingen ska känna sig utanför och alla ska känna sig sedda. Starka ledaregenskaper är viktigt för en entertainer då denne trots allt leder ett utskott, alla ska känna sig hörda och delaktiga i de beslut som tas. Det är trots allt utskottet tillsammans som gör arbetet, alla bör vara inkluderade och hjälpa till att samordna aktiviteter. 

% konkretisering av de viktigaste kraven i stort

Önskvärda egenskaper hos en {\post} kan i korta ordalag beskrivas nedan:
\begin{dashlist}
    \item Ett starkt driv till att förnya och förbättra sektionen samt utskottet
    \item Goda ledaregenskaper
    \item Hög flexibilitet och samarbetsvilja
    \item Bra organisationsförmåga
    \item God social kompetens
    \item Ödmjuk
\end{dashlist}

    Inga tidigare erfarenheter krävs. Inga direkta krav förekommer.

\end{document}
    