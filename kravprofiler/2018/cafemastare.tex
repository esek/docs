\documentclass[10pt]{article}
\usepackage[utf8]{inputenc}
\usepackage[swedish]{babel}

\def\post{Cafémästare}

\def\doctitle{Kravprofil för \post}
\def\docauthor{Daniel Bakic, Elin Johansson}
\def\date{2018-10-09} %YYYY-MM-DD

\usepackage{../../e-sek}
\usepackage{./e-kravprofil}

\begin{document}
\heading{\doctitle}

% summering av posten

Cafémästaren är huvudansvarig för LED café och utskottschef över alla funktionärer i Cafémästeriet. Det innefattar bland annat vara ansiktet utåt och att man blir kontaktad från många håll angående verksamheten. Tillsammans med den centrala gruppen (Vice Cafémästare och Inköps- och lagerchefer) har Cafémästaren fria händer över utbudet i caféet och om man vill sälja kaffe och mat till arrangemang inom och utanför sektionen. Cafémästare är en styrelsepost vilket kortfattat innebär veckoliga möten, en del arbete inför terminsmötena och att representera sektionen.

% konkretisering av ansvarsområden

Ansvarsområden:
\begin{dashlist}
    \item Ekonomiskt ansvar, bl.a. hålla koll på intäkter och utgifter, justera priser och redovisa försäljning.
    \item Hitta arbetare till caféet och se till att de får den upplärning de behöver.
    \item Se till att den centrala gruppen i Cafémästeriet inte har mer jobb än de klarar av.
    \item Vara med och arrangera ”Caféfesten”, en tackfest med Cafémästerier från andra sektioner.
    \item Ha kontakt med leverantörer och vaktmästaren i huset.
    \item Se till att rutiner för städning, livsmedelshantering och hygien följs så att hälsoinspektionen inte ger anmärkningar vid inspektionerna.
    \item Styrelsearbete under hela året men framför allt i anknytning till terminsmötena och terminsslut.
\end{dashlist}
    
% summering av profilen
    
En bra Cafémästare är lyhörd över hur funktionärerna i utskottet mår och prioriterar alltid det framför
försäljningssiffror. Att ge inköpare och kreativa Dioder utrymme att testa nya saker är ofta uppskattat
både hos jobbare och kunder. Som utskottschef är det även viktigt att vara tydlig med vad som väntas
av jobbarna och se till att de gör det de ska.

% konkretisering av de viktigaste kraven i stort

De viktigaste egenskaperna hos en god {\post} kan sammanfattas som:
\begin{dashlist}
    \item God kommunikationsförmåga
    \item Bra på att delegera arbete, och att följa upp att det utförs
    \item Vara strukturerad och bra på att planera
    \item Ha erfarenhet av hur arbetet i LED café ser ut
    \item Vara lyhörd
    \item Bekväm med att ta beslut
\end{dashlist}

Direkta krav:
\begin{dashlist}
    \item Krav enligt E-sektionens styrdokument
\end{dashlist}

% mer specifika saker som är bra att ha, men ej nödvändiga

Önskvärda erfarenheter (men ej nödvändiga):
\begin{dashlist}
    \item Företagsekonomi
    \item Excell
%   \item Tidigare engagemang inom sektionen
\end{dashlist}

Frågor om uppgifter i belastningsregistret kan komma att uppstå i samband med valberedningsprocessen för posten.\\

\textbf{\textit{Då Cafémästeriet driver en daglig verksamhet kan det ibland innebära att man som Cafémästare missar delar av sin utbildning så som föreläsningar.}}

\end{document}
