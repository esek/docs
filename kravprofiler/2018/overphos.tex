\documentclass[10pt]{article}
\usepackage[utf8]{inputenc}
\usepackage[swedish]{babel}

\def\post{Øverphøs}

\def\doctitle{Kravprofil för \post}
\def\docauthor{Niklas Gustafson, Andreas Bennström}
\def\date{2018-10-09} %YYYY-MM-DD

\usepackage{../../e-sek}
\usepackage{./e-kravprofil}

\begin{document}
\heading{\doctitle}

% summering av posten

Øverphøset har det övergripande ansvaret för nollningen. Øverphøset är utskottsordförande i Nollningsutskottet, där personen har fem Cophøs och två ØverGudsPhaddrar till sin hjälp. Tillsammans planerar och organiserar NollU nollningen. Det är Øverphøsets uppgift att styra upp, planera, delegera, fördela och leda arbetet i utskottet framåt. Förutom detta sitter Øverphøset också med som ledamot i styrelsen samt som medlem i ØPK(ØverphøsKollegiet). ØPK består av alla sektioners Øverphøs, Nollegeneral samt Nolleamiral. Øverphøsets roll i ØPK är att delta aktivt i diskussioner för att kunna hjälpa till att utveckla intersektionella nollningsevenemang. Detta innebär att en som Øverphøs går på väldigt mycket möten. En stor del av Øverphøsets roll är därför att föra information mellan phøset, ØPK och styrelsen så att alla får den information de behöver för att kunna göra sitt arbete så bra som möjligt.

% konkretisering av ansvarsområden

Arbetsuppgifter:
\begin{dashlist}
    \item Organisera utskottets arbete
    \item Värna om utskottets välmående
    \item Planera och hålla i utskottsmöten
    \item Delta aktivt i ØverphøsKollegiet
    \item Delta aktivt i styrelsens arbete
    \item Föra information mellan NollU, ØPK och styrelse
    \item Under nollningen vara en del av sektionens ansikte utåt för de nya studenterna
\end{dashlist}

% summering av profilen
Ett bra {\post} är en bra ledare som lyssnar på sitt utskott och ser till utskottets och de nyantagnas bästa. Som Øverphøs bör en vid tillfällen kunna åsidosätta sina egna åsikter för att istället framföra phøsets gemensamma åsikt. Ett Øverphøs bör ha god kommunikationsförmåga, eftersom en stor del av arbetet är att föra information mellan olika personer. Det är också viktigt att kunna delegera och fördela arbete, liksom att vara en bra problemlösare vid oväntade och tidspressade situationer.
% konkretisering av de viktigaste kraven i stort

I korthet kan de viktigaste egenskaperna hos ett {\post} sammanfattas som:
\begin{dashlist}
    \item Goda ledaregenskaper
    \item Goda organisatoriska egenskaper
    \item God kommunikationsförmåga
    \item Kunna delegera och fördela arbete
    \item Problemlösare
    \item Stresstålig
\end{dashlist}

\end{document}       