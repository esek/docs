\documentclass[10pt]{article}
    \usepackage[utf8]{inputenc}
    \usepackage[swedish]{babel}
    
    \def\post{Valberedningens ordförande}
    
    \def\doctitle{Kravprofil för \post}
    \def\docauthor{Pontus Landgren}
    \def\date{2018-10-15} %YYYY-MM-DD
    
    \usepackage{../../e-sek}
    \usepackage{./e-kravprofil}
    
    \begin{document}
    \heading{\doctitle}
    
    % summering av posten
    
    Valberedningens uppgift är att bereda val som ska genomföras av sektionsmöten. Detta sker genom intervjuer av kandidater av valberedningens medlemmar som sedan lämnar ett förslag med nominerade personer till sektionsmötet. Valberedningens ordförande är ansvarig för att leda och organisera denna verksamhet.
   
    % konkretisering av ansvarsområden
    
    Uppgifter i korthet:
    \begin{dashlist}
        \item Kalla till och leda valberedningens arbete och möten
        \item Genomföra intervjuer och lägga fram nomineringsförslag inför sektionsmöten
        \item Informera styrelsen om hur arbetet fortskrider
        \item Tillsammans med Styrelsen planera Expot och Valmötet.
    \end{dashlist}
    
    % summering av profilen
    Valberedningen består förutom dess ordförande av valberedningens sekreterare, två ledamöter, en representant från NollU (Nollningsutskottet) och en representant från de nyantagna.\\

    Egenskaper som är nödvändiga för valberedningens ordförande är att ha en god insikt i Sektionens verksamhet, styrdokument och valförfarande. Som person bör man vara rättvis, objektiv och bra på att strukturera samt organisera arbete.\\
    % konkretisering av de viktigaste kraven i stort
    
    Direkta krav:
    \begin{dashlist}
        \item {\post} får inte sitta i styrelsen eller vara Valberedningens förslag till en post inom styrelsen.
    \end{dashlist}
    
    \end{document}       