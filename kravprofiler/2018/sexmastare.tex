\documentclass[10pt]{article}
\usepackage[utf8]{inputenc}
\usepackage[swedish]{babel}

\def\post{Sexmästare}

\def\doctitle{Kravprofil för \post}
\def\docauthor{Linnea Sjödahl, Alexander Wik, Martin Gemborn Nilsson}
\def\date{2018-10-09} %YYYY-MM-DD

\usepackage{../../e-sek}
\usepackage{./e-kravprofil}

\begin{document}
\heading{\doctitle}

% summering av posten

{\post}n är utskottsordförande för Sexmästeriet som arrangerar sittningar och eftersläpp för sektionen. Som {\post} leder man planeringsarbetet innan sittningar och ser till att resten av Sexmästeriet nås med rätt information angående eventen. Många evenemang innebär ett samarbete med en annan organisation eller sektion, det är Sexmästarens ansvar att förmedla Sexmästeriets önskemål för sittningen. Sexmästaren för sektionens talan i Sexmästarkollegiet, som består av Sexmästare från alla sektioner på TLTH och beslutar om intersektionella event. Sexmästaren är också en ledamot i styrelsen och har då ett ansvar att delta aktivt i styrelsearbetet.

% konkretisering av ansvarsområden

{\post}n är ansvarig för följande områden (andra poster som delvis assisterar sexmästaren):
\begin{dashlist}
    \item Organsistation, strukturering och ledning sexmästeriets verksamhet (Vice-sexm.)
    \item Planering av sittningar och eftersläpp (Vice-sexm. Hov-m. Köks-m. Bar-m. Sång-fm.)
    \item Marknadsföringen av evenemang (Vice-sexm.)
    \item Bokning av lokaler och kontakt med hyresvärdar
    \item Budgetering och bokföring av evenemang
    \item Att alkohollagen följs (Vice-sexm. Bar-m.)
    \item Tala för sektionen på vissa evenemang och sammanträden
    \item Kontakt med andra sexmästerier, huset och de andra utskotten på sektionen
\end{dashlist}

% summering av profilen

Planerandet av dessa typer av evenemang kräver mycket god framförhållning och goda planeringsegenskaper då förberedelserna för evenemanget ofta påbörjas flera månader innan själva eventet. Under evenemangen bör man på ett bra sätt kunna hantera stressiga situationer och vara en knytpunkt för kommunikation mellan andra som jobbar på evenemanget.

% konkretisering av de viktigaste kraven i stort

En sammanfattning av de viktigaste egenskaperna hos en bra {\post} är:
\begin{dashlist}
    \item God ledarskapsförmåga
    \item Goda kunskaper i kommunikation
    \item Problemlösare och nytänkande
    \item Bekväm med att ta beslut
    \item Klarar av att arbeta under långa pass
    \item Stresstålig
\end{dashlist}

Direkta krav:
\begin{dashlist}
    \item Postinnehavaren måste ha fyllt minst 20 år inför mandatperioden
\end{dashlist}

\end{document}