\documentclass[10pt]{article}
    \usepackage[utf8]{inputenc}
    \usepackage[swedish]{babel}
    
    \def\post{Revisor}
    
    \def\doctitle{Kravprofil för \post}
    \def\docauthor{Anders Nilsson, Fredrik Petersson}
    \def\date{2018-10-09} %YYYY-MM-DD
    
    \usepackage{../../e-sek}
    \usepackage{./e-kravprofil}
    
    \begin{document}
    \heading{\doctitle}
    
    % summering av posten
    
    Revisorernas uppgift är att objektivt granska Sektionens verksamhet och ekonomi. Granskningen ska säkerställa att verksamheten följer både lagmässiga krav och de krav som stadga och reglemente ställer på verksamheten.
    
    % konkretisering av ansvarsområden
    
    Uppgifter i korthet:
    \begin{dashlist}
        \item Löpande granska Sektionens bokföring och ekonomi.
        \item Löpande granska Sektionens förvaltning.
        \item Efter verksamhetsårets slut skriva en revisionsberättelse.
    \end{dashlist}
    
    % summering av profilen
    Revisionen består av två delar, dels ekonomisk granskning och dels förvaltningsrevision. Revisorerna bör tillsammans besitta erfarenhet och kunskap inom båda dessa områden. Det är dock inte nödvändigt att båda personerna besitter samtliga egenskaper och erfarenheter passande för båda dessa delar av revisionen. Att åtminstone en av revisorerna har stor personlig erfarenhet från Sektionens verksamhet är en klar fördel.\\

    Egenskaper som är absolut nödvändiga för en revisor är att ha en god insikt i Sektionens verksamhet eller vara beredd att lägga ner tid för att sätta sig in i densamma. Som person bör man vara noggrann och resonlig men också beredd på att sätta ner foten gentemot ansvariga om verksamheten inte skulle följa de krav som ställs.\\
    % konkretisering av de viktigaste kraven i stort
    
    Direkta krav:
    \begin{dashlist}
        \item Postinnehavarna måste vara myndiga.
        \item Postinnehavarna får ej inneha något annat förtroendeuppdrag inom Sektionen.
        \item Postinnehavarna får ej befinna sig i en jävssituation gentemot någon i Styrelsen.
    \end{dashlist}

    Någon av revisorerna bör dessutom:
    \begin{dashlist}
        \item Ha erfarenhet av Sektionens eller annan förenings bokföringsarbete.
        \item Ha erfarenhet från hur förvaltningen av Sektionen, eller annan förening, fungerar.
    \end{dashlist}
    
    \end{document}       