\documentclass[10pt]{article}
\usepackage[utf8]{inputenc}
\usepackage[swedish]{babel}

\def\post{Krögare}

\def\doctitle{Kravprofil för \post}
\def\docauthor{Malin Heyden, Markus Rahne}
\def\date{2018-10-04} %YYYY-MM-DD

\usepackage{../../e-sek}
\usepackage{./e-kravprofil}

\begin{document}
\heading{\doctitle}

Krögaren organiserar E-sektionens pubverksamhet och genomför pubar i Edekvata för sektionens och teknologkårens övriga medlemmar. Till Krögarens hjälp finns Källarmästeriet, som innehåller två Vice Krögare, två inköpsansvariga för alkohol (Cøl) samt ett helt gäng källarmästare. Som Krögare måste man ha bred kunskap om Edekvatas kök och gillen samt även kunna leda sitt utskott. Gillenas utformning är i stort sett samma från år till år, men det finns även frihet i att bestämma tema, musik, mat, dryck, dekor, aktiviteter osv. Förutom att vara utskottsordförande är Krögaren även en del av styrelsen och förväntas arbeta aktivt för sektionens bästa.

Ansvarsområden:
\begin{dashlist}
    \item Organisera utskottets verksamhet
    \item Hålla god kontakt med alla funktionärer i Källarmästeriet
    \item Bokföring
    \item Deltar aktivt i Styrelsens arbete
    \item Deltar i TLTHs sexmästarkollegium
\end{dashlist}

En god Krögare är en kreativ och lösningsorienterad person som inte är rädd för oväntade
händelser. Den är också en bra ledare, eller aspirerar att bli en sådan. Krögaren bör dessutom
vara en person som kan representera sektionen utåt på ett positivt sätt. Ett gott sinne för mat och dryck, god samarbetsförmåga och en vilja att dela arbetet rättvist är också nödvändigt. Krögaren förväntas även skaffa sig god förståelse över kökets utrustning, så som stekbord och fatanläggning.

I korthet kan de viktigaste egenskaperna sammanfattas som:
\begin{dashlist}
    \item Bra ledarskapsförmåga
    \item Rättvis
    \item Diplomatisk
    \item Samarbetsvillig
    \item Organiserad
\end{dashlist}

Direkta krav:
\begin{dashlist}
    \item Postinnehavaren måste ha fyllt minst 20 år inför mandatperioden
    \item Övriga krav enligt E-sektionens styrdokument
\end{dashlist}

Önskvärda erfarenheter (men ej nödvändiga):
\begin{dashlist}
    \item Tidigare erfarenhet från bar- och köksverksamhet.
    \item Genomförd alkoholutbildning som A-cert är meriterande. (Vidare alkoholutbildning kommer erbjudas under våren)
\end{dashlist}

\end{document}