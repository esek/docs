\documentclass[10pt]{article}
\usepackage[utf8]{inputenc}
\usepackage[swedish]{babel}

\def\post{Vice Cafémästare}

\def\doctitle{Kravprofil för \post}
\def\docauthor{Jonathan Jakobsson, Max Mauritsson, Amanda Nilsson}
\def\date{2018-10-09} %YYYY-MM-DD

\usepackage{../../e-sek}
\usepackage{./e-kravprofil}

\begin{document}
\heading{\doctitle}

% summering av posten

Vice cafémästarna ingår i Cafémästeriet som ansvarar för sektionens café, LED-café.
Vice cafémästarens uppgift är att bistå Cafémästaren med dennes arbete och dela dennes övergripande ansvar över caféets verksamhet.

% konkretisering av ansvarsområden

Ansvarsområden:
\begin{dashlist}
    \item Hjälpa till med att värva jobbare (Dioder) samt lära upp dessa.
    \item Kunna hoppa in och jobba i caféet då det krävs.
    \item Fylla på med läsk varje vecka från CM.
    \item Kontrollera temperatur i kylarna.
    \item Rengöra kaffebomber och kalka av kaffemaskinen.
    \item Bistå Cafémästaren med diverse sysslor.
    \item Allmänt laga och fixa då det behövs i LED.
    \item Öppna och stänga LED någon eller några gånger i veckan.
\end{dashlist}

% summering av profilen

Som {\post} bör du vara social, drivande, flexibel och kunna ta en ledarroll då det behövs. Du ska vara medveten om att din post kan kräva att man lägger ner en hel del tid. Du bör därför ha en bra balans mellan tiden du lägger i caféet och studietiden.  Att jobba i LED-café är mycket belönande då du får lära känna många nya människor då cafeét är en naturlig mötesplats för sektionens medlemmar och i övriga i E-huset.
% konkretisering av de viktigaste kraven i stort

% De viktigaste egenskaperna hos en god {\post} kan sammanfattas som:
% \begin{dashlist}
%     \item Förmåga att uppskatta vilka inköpsvolymer som behövs
%     \item Kunna vara tidigt i skolan (ca 7.40)  vid leverans
%     \item God kommunikationsförmåga
%     \item Vara strukturerad
%     \item Kunna missa föreläsningar eller liknande vid leverans
% \end{dashlist}

% % mer specifika saker som är bra att ha, men ej nödvändiga

% Önskvärda erfarenheter (men ej nödvändiga):
% \begin{dashlist}
%     \item Företagsekonomi
%     \item Kunskap om olika livsmedel (förvaring, förväntad hållbarhet, rimligt pris)
% %   \item Tidigare engagemang inom sektionen
% \end{dashlist}

\end{document}