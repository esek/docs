\documentclass[10pt]{article}
\usepackage[utf8]{inputenc}
\usepackage[swedish]{babel}

\def\post{Vice Sexmästare}

\def\doctitle{Kravprofil för \post}
\def\docauthor{Henrik Ramström, Mia Cicovic, Davida Åström}
\def\date{2018-10-09} %YYYY-MM-DD

\usepackage{../../e-sek}
\usepackage{./e-kravprofil}

\begin{document}
\heading{\doctitle}

% summering av posten

Sektionens {\post} har till uppgift att bistå sexmästaren i dess arbete, denna definition gör att postens åtaganden till stor del definieras av hur Sexmästaren och Vice Sexmästare väljer att dela upp arbetsuppgifter mellan sig. Då Sexmästaren ägnar sig åt kontakter med övriga sektionen, andra sexmästerier, huset och kåren faller det sig naturligt att Vice Sexmästare ägnar mer tid åt de frågor som gäller de egna jobbarna. Detta innebär att en mycket viktig del av arbetet består av planering och organisation av utskottets evenemang och medlemmar, vilket gör att god förmåga och ambition inom detta område är centralt för att kunna genomföra arbetet på ett bra sätt. Under sittningar är uppgifterna mycket varierande och handlar till stor del om att lösa de problem som uppstår, att hjälpa till där det behövs och att se till att alla som jobbar mår bra och har det roligt. I korthet skulle postens ansvar kunna beskrivas att se till att alla andra i sexmästeriet har bästa möjliga förutsättningar att kunna utföra sina åtaganden på ett bra sätt, och att de har roligt när de gör det. Ansvaret för alkoholservering under sittningar delar den Vice Sexmästaren med Sexmästaren, till stöd för detta genomgår Vice Sexmästare en utbildning i ansvarsfull alkoholutskänkning i början av sitt verksamhetsår.
% konkretisering av ansvarsområden

Ansvarsområden:
\begin{dashlist}
    \item Tillsammans med Sexmästaren planera sittningar
    \item Schemaläggning av jobbare
    \item Tidsplanering av sittningar
    \item Problemlösning av olika typer av situationer som kan uppstå under sittningar
    \item Dela Sexmästarens ansvar för ansvarsfull alkoholutskänkning under sittningar och eftersläpp. Om personen har fyllt 20 år är det möjligt att stå som serveringsansvarig, i annat fall gäller det endast att bistå Sexmästaren i dess arbete. 
    \item Bistå övriga mästare i sexmästeriet i deras arbete innan och under sittningar, detta kan innebära allt ifrån att diskutera idéer och problem till att genomföra andra mästare, eller jobbares, uppgifter ifall de får förhinder. Vice Sexmästare bör även se till att planeringen och kommunikationen mellan sexmästeriets olika delar fungerar på ett bra sätt. 
    \item Bistå Sexmästaren i möten med andra utskott/sektioner och planering av sexmästeriets verksamhet samt budget
    \item Bistå Sexmästaren i, och eventuellt beroende på arbetsfördelning ha ansvar för delar av, övrigt administrativt arbete inom utskottet, till exempel ta fram anmälningslistor, marknadsföra evenemang via olika kanaler och dylikt
\end{dashlist}

% summering av profilen

Kvaliteter som utmärker en god Vice Sexmästare är stark organisationsförmåga som bidrar till strukturerat planeringsarbete och kommunikation kring detta arbete med sexmästeriets övriga mästare samt dess jobbare. Personen innehar också en god problemlösningsförmåga och kan även under stress hantera de olika situationer som kan uppstå under en sittning samt bistå sexmästeriets övriga mästare att hitta bästa möjliga lösning på deras problem. Till sist är det viktigt att personen är lyhörd i sociala situationer och aktivt bidrar till god stämning och har en förmåga att känna av andras behov, i sexmästeriet jobbar man långa pass och det är viktigt att se till att arbetet så gott det går är på en nivå som är rolig och stimulerande för de som jobbar.

% konkretisering av de viktigaste kraven i stort

Sammanfattningsvis är följande egenskaper viktiga för {\post}:
\begin{dashlist}
    \item God planerings och organisationsförmåga 
    \item God samarbets och kommunikationsförmåga i planering och genomförande av sittningar
    \item Stark problemlösningsförmåga, även i situationer som kräver snabba beslut
    \item Social kompetens och förmåga att skapa trivsel i sin omgivning
\end{dashlist}

% Direkta krav:
% \begin{dashlist}
%     \item Postinnehavaren måste ha fyllt minst 20 år inför mandatperioden
% \end{dashlist}

\end{document}