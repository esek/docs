\documentclass[10pt]{article}
\usepackage[utf8]{inputenc}
\usepackage[swedish]{babel}

\def\post{Inköps- och lagerchef}

\def\doctitle{Kravprofil för \post}
\def\docauthor{Elin Johansson, Jonatan Benitez}
\def\date{2018-10-09} %YYYY-MM-DD

\usepackage{../../e-sek}
\usepackage{./e-kravprofil}

\begin{document}
\heading{\doctitle}

% summering av posten

{\post}erna håller ordning i lagret i LED café, lägger beställningar och tar emot dessa två gånger i veckan. De ansvarar även för läsklagret i CM och hjälper andra utskott att beställa saker. Tillsammans med Cafémästaren och vice har lager- och inköpscheferna stort inflytande över utbudet i LED café och möjligheten att dra i småprojekt såsom att sälja semlor på fettisdagen. Som en del av den centrala gruppen i cafét innebär det att hjälpa till med öppning och stängning, ibland kunna jobba om det krisar och hjälpa till att lära upp nya funktionärer.
% konkretisering av ansvarsområden

Ansvarsområden:
\begin{dashlist}
    \item Beställa varor och ta emot leveranser samt skicka tillbaka returgods och pant
    \item Se till att det är ordning i lager, kylar och frysar samt att inga varor passerat utgångsdatum
    \item Se över hur mycket som säljs och reglera brödbeställning m.m. efter det
    \item Beställa läsk och chokladbollar till lagret i CM
    \item Göra IC-rapporter på alla fakturor\footnote{En IC-rapport är en ekonomisk rapport som görs för att dela upp kostnaden av olika varukategorier på olika konton.}
\end{dashlist}
    
% summering av profilen
    
En god {\post} är en lagspelare som samarbetar med de andra inköparna. För att kunna planera beställningar och ha en uppfattning om hur mycket varor som går åt är det viktigt att kommunicera med dioderna och vara uppmärksam. Då det kan bli tung arbetsbelastning är det viktigt att kunna säga ifrån och vara öppen mot resten av den centrala gruppen.

% konkretisering av de viktigaste kraven i stort

De viktigaste egenskaperna hos en god {\post} kan sammanfattas som:
\begin{dashlist}
    \item Förmåga att uppskatta vilka inköpsvolymer som behövs
    \item Kunna vara tidigt i skolan (ca 7.40)  vid leverans
    \item God kommunikationsförmåga
    \item Vara strukturerad
    \item Kunna missa föreläsningar eller liknande vid leverans
\end{dashlist}

% mer specifika saker som är bra att ha, men ej nödvändiga

Önskvärda erfarenheter (men ej nödvändiga):
\begin{dashlist}
    \item Företagsekonomi
    \item Kunskap om olika livsmedel (förvaring, förväntad hållbarhet, rimligt pris)
%   \item Tidigare engagemang inom sektionen
\end{dashlist}

\end{document}