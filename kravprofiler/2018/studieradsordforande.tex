\documentclass[10pt]{article}
\usepackage[utf8]{inputenc}
\usepackage[swedish]{babel}

\def\post{SRE-ordförande}

\def\doctitle{Kravprofil för \post}
\def\docauthor{Pontus Landgen, Edvard Carlsson, Fanny Månefjord}
\def\date{2018-10-09} %YYYY-MM-DD

\usepackage{../../e-sek}
\usepackage{./e-kravprofil}

\begin{document}
\heading{\doctitle}

% summering av posten

{\post}n är, precis som namnet säger, ordförande för utskottet SRE (studierådet på E-sektionen). Arbetsuppgifterna är bland annat att hålla interna möten för SRE, vara med på SRX-möten med andra studierådsordförande från andra sektioner. Studierådsordförande agerar som en länk mellan universitets anställda och studenterna och ska hjälpa studenter när det är något studierelaterat problem. Man ska även hjälpa och stötta det övriga utskottet i deras respektive roller och anordna pluggkvällar. SRE-ordföranden har även en viktig roll som ledamot i Styrelsen för E-sektionen.

% konkretisering av ansvarsområden

Ansvarsområden:
\begin{dashlist}
    \item Organisera och leda studierådets verksamhet
    \item Samordna utvärderingsmöten för kurser
    \item Deltaga vid SRX-möten och ha kontinuerig kontakt med kårens utbildningsutskott
    \item Samarbeta med phøset och arrangera pluggkvällar under nollningen 
    \item Ha kontakt med programledningarna för E och BME
\end{dashlist}

% summering av profilen
En {\post} tycker att studierna är viktiga och vill ständigt förbättra kvalitén på utbildningen. För att göra detta är det viktigt att hålla sig uppdaterad på vad som händer i andra sektioners studieråd samt vad kårens utbildningsansvariga säger. En bra ledare ska kunna ta beslut och kunna fördela arbetet i utskottet. SRE-ordförande ska vara bra på att kommunicera och strukturera.
% konkretisering av de viktigaste kraven i stort

% Goda egenskaper att inneha som {\post} är:
% \begin{dashlist}
%     \item Bra på att kompromissa
%     \item Kreativ
%     \item Kunna fatta beslut
%     \item Samarbetsvillig
% \end{dashlist}

\end{document}
    