\documentclass[10pt]{article}
\usepackage[utf8]{inputenc}
\usepackage[swedish]{babel}

\def\post{Krögare}
\def\date{2017-12-28} %YYYY-MM-DD
\def\docauthor{Markus Rahne}

\usepackage{../e-testamente}
\usepackage{../../e-sek}

\begin{document}
\heading{\doctitle}

Var hälsad, Krögare Electus/Electa!

Låt mig gratulera till din nya post och till det kommande året som kommer vara fyllt med allsköns lärdomar, erfarenheter och upptåg av de vildaste slag.

Som Krögare är du utskottschef över Källarmästeriet, E-sektionens alldeles egna pubverksamhet. Din roll inom Styrelsen och mångt om mycket inom sektionen centralt kommer vara den som har koll på köket och det praktiska rörande alkoholutbudet, där aktiva kommer söka sig till dig för din expertis rörande att styra upp en välfungerande pub eller ett mindre evenemang som inte nödvändigtvis rör Sexmästeriet. I mer uppstyrd form är Krögarens roll, utan någon vidare ordning:

\begin{dashlist}
    \item Att vara en god styrelseledamot genom att föra sektionen framåt med värdefulla idéer och åsikter men också ge talan för åsikter från ditt utskott till Styrelsen samt sektionen.

	\item Att hålla en översiktlig blick över sektionens alkoholförråd. Till din hjälp har du två funktionärer, Cøl, som har det huvudsakliga arbetet, men det är du som Krögare som har det övergripande ansvaret då majoriteten av alkoholen går på Källarmästeriets budget, trots att andra utskott använder förrådet.

	\item Att vara tillmötesgående för frågor rörande Edekvatas kök. Som Krögare är inte detta en prioritet men i och med den tid du kommer spendera i Edekvatas kök och bar kommer du tillsammans med dina vice vara de individer inom sektionen som har störst koll, och i uträckningen de som kommer värna mest om att köket sköts.

	\item Att vara en aktiv del i kårens Sexkollegium. Som Krögare kommer du vid tillfällen känna att vissa möten inte känns relevanta då de centreras runt sittningar, men detta kollegium kommer ge dig möjlighet att driva saker som rör ditt utskott på en högre nivå samtidigt som du får erfarenheter och utbyte från andra utskottschefer i liknande position från de andra sektionerna.

\item Till slut, att hålla pubar för sektionen och ha så ofantligt roligt när du gör det!

\end{dashlist}

Jag ska inte sticka under stol att Krögaren har en av de posterna inom Styrelsen som kräver mest jobb. Med förväntan att hålla pub varje fredag under ett helt år med endast två vice som kan backa upp dig kan det vid tider kännas tufft. Nyckeln till att minska bördan är att dela upp arbetet jämt mellan krögartrion, håll en öppen dialog från dag ett för att hålla koll på vad du och dina vice vill fokusera på och balansera upp arbetet så det är någorlunda jämt fördelat. Fokusera på gruppdynamiken och var pragmatisk i din ledarstil, viljan att bara göra allt för att få det gjort här och nu kommer finnas där, men utskottet kommer må mycket bättre utan en slutkörd Krögare. Var inte rädd att påpeka att som Krögare eller Vice ska man kunna hålla ett gille själv, även om det är mycket roligare och jobba alla tre vid samma tillfälle.

Trots all möda, blod, svett och tårar är Krögare en speciell post och jag har aldrig ångrat en sekund att bli aktiv. Alla lärdomar, erfarenheter och alla fantastiska stunder intryckt under ett år gör att de där lite tyngre stunderna väger lätt i slutändan. Kom ihåg att även om du ska vara där för utskottet ska även utskottet och sektionen vara där för dig, så ta emot så mycket hjälp som finns. Att ställa in ett gille betyder lite i det stora hela om ni behöver en paus och kom ihåg att ni kommer vara värda det. Så ha fantastiskt kul, skapa oförglömliga minnen och gillen, knyt kontakter som kan ge dig nya kunskaper och se till att just ditt styrelseår blir det bästa året under din studietid!

\begin{itshape}
Vänliga hälsningar\\
Markus Rahne, Krögare 2017\\
\end{itshape}

\newpage

\tableofcontents
\newpage

\section{Sektionen}
Som Krögare representerar du sektionen som en beprövad Sexmästare med extra kunskap inom pubväsendet, även om jag erkänner att den formuleringen är lite som att svära i kyrkan. Du kommer tillsammans med Sexmästaren hålla auktoritet över hur Ni vill utveckla sektionens utbud av sittningar och pubar men framförallt hur ni vill förbättra utrustning och arbetsförhållandena för Källarmästeriet och Sexmästeriet. Detta är ett arbete som oftast går hand i hand så håll god kontakt med Sexmästaren och vederbörandes mästare, då det finns mycket att lära av varandra.

Det kommer troligtvis också finnas tillfällen då Källarmästeriet vill hålla sittning och Sexmästeriet vill hålla pub och här är god kommunikation nyckeln för att undvika kompletta mentala sammanbrott, se kapitel om ``Alkoholförråd och Alkoholhanteringssystemet (AHS)'' samt ``Så Källarmästeriet vill hålla en sittning?''. Kom ihåg att det aldrig finns dumma frågor eller uppmaningar och att din egna självbevarelsedrift är oftast en bra måttstock på vad som har en tendens att falla mellan stolarna, vare sig det handlar om ett annat utskott som ska rota i alkoholförrådet eller om Källarmästeriet vill pröva sin hand på en sittning.

Som Krögare kommer du även få flera möjligheter att jobba större evenemang inom kåren tack vare ditt medverkande i Sexkollegiet men även din erfarenhet att styra bar och kök. Jag tog ett par chanser att jobba eftersläpp och Puben puben och på vägen lärde känna många nya människor, fick massor med erfarenhet och fick möjligheten att gå med kårens gastronomiförening Allium. Chansen att utvecklas är stor och jag rekommenderar utomsektionellt engagemang varmt, då du även sprider medvetande om Källarmästeriets verksamhet i samma veva.

\section{Styrelsen}
Styrelsearbetet kan lätt glömmas bort i stressen att ständigt hålla gillen men det är en viktig del i arbetet som Krögare. Styrelsemöten hålls en gång i veckan där Sektionen som helhet diskuteras och dess välstånd och välmående utvärderas. Försök vara en så aktiv styrelseledamot det går, du sitter trots allt i Styrelsen på mandat och dina synpunkter är viktiga. Styrelsearbetet kan gälla allt mellan lätta diskussioner om ett mindre inköp till större diskussioner och beslut som kan påverka sektionens framtid. Var aktiv och framför allt, för ditt utskotts talan tillsammans med din egna i Styrelsen. Med det sagt är det av vikt att inte vara alltför polemisk, det är viktigt att Styrelsen har ett välfungerande debattklimat utöver att det ska fungera socialt inom gruppen. Är ett förslag helt uppåt väggarna måste det självklart påpekas, men på ett snyggt sätt. Respektera dina gemene styrelseledamöter och kom ihåg att det är Ordförande som är mötesordförande och har sista ordet. Som en del av Styrelsen ligger det även arbete i att förbereda propositioner och svara på motioner till sektionsmöten.

Det administrativa arbetet kan ibland kännas torrt men det finns gott om roliga åtaganden också, som att planera och styra upp funktionärstack och sektionsskiphten. Jobba och gör mycket tillsammans, en välfungerande styrelse är en styrelse som håller ihop och kommer i förlängningen förgylla ditt år något enormt.

\section{Källarmästeriet}
Källarmästeriet, till vardags kallat KM, är utskottet vi alla älskar och kommer vara ditt skötebarn under det kommande året. Sett till antalet poster är KM det minsta, vilket har sina för- och nackdelar. Det är mycket enkelt att få en överblick på sitt utskott och vad alla har för uppgift men med en så pass liten arbetsstyrka i toppen kan mycket arbete samlas på hög om individerna med de huvudsakliga uppgifterna inte är självgående.

Hela krögartrion har samma operativa ansvar för pubverksamheten och ska kunna hålla ett gille själv tillsammans med jobbare. Håll god kommunikation, fokusera på gruppdynamiken och dela upp ansvaret jämnt så ingen kör ut sig själv. Förslag på en sund uppdelning är att någon tar ansvar för menyn och matbeställning, en annan sköter marknadsföring, Facebookevents samt kontakt med jobbare. Specifika ansvar för Krögaren är utskottets ekonomi samt kontakt med E-huset, se kapitel Metodik för mer info. Glöm inte rotera ansvarsområden om ni känner för det så hela trion får testa på olika ansvarsområden. Utan rotation upptäcker man inte älskvärda snedsteg som pesto-mozzarella klägg!

Cølen sköter allt inköp av dryck till gillena och det är deras ansvar att det finns tillräckligt med alkohol och alkoholfria alternativ till varje gille, att prissättningen i AHS är korrekt och laglig samt att alla enheter finns representerade i iZettle. Se till att Cølen blir så självgående som möjligt, att de kontinuerligt kollar alkoholförrådet efter gillen och håller ett varierat utbud. Se till att jobba proaktivt med alkoholutbudet, önskas något speciellt till ett temagille så se till att ta kontakt i god tid med Cølen, minst två veckor i förväg, så de hinner beställa och planera upp inköp. Har det varit ett lyckat gille och alkoholförrådet ekar tomt är det mycket uppskattat av Cølen att de meddelas detta så en resa till Systembolaget kan styras upp så fort som möjligt.

Sist men inte minst har vi Källarmästarna, Källarmästeriets ryggrad. Dina jobbare är din viktigaste tillgång och de som gör allt jobb på gillena. Respektera deras tid och uppoffring för utskottet, och utnyttja initiativ om det finns. Gå ut i god tid med jobbkalender och se till att påminna jobbare inför varje jobbarpass. Skäm bort dom med jobbargodis och spårade sexor, men var inte rädd att ta i med hårdhandskarna under gillena. Du kommer märka inom sinom tid vilka jobbare som går av egen maskin och hittar egna och nya arbetsuppgifter medan andra kommer du behöva jaga med blåslampa för att de ska fortsätta med nya arbetsuppgifter. I värsta falla gäller detta även dina Vice men vid tillfällen kommer även dina Vice att jaga dig. Din stora utmaning kommer bli att hitta den ledarstil som fungerar bäst för just ditt utskott. Under mitt år märkte jag att mina Källarmästarna svarade mer på piska än moroten, och när du väl hittar din typ av ledarstil kommer många andra pusselbitar falla på plats.


\section{Sexkollegiet}
Sexkollegiet samlar sektionernas alla utskottschefer som sköter sittningar samt övrig form av sexmästeriverksamhet. Kollegiet träffas ungefär varannan eller var tredje vecka för att diskutera de olika sektionernas verksamhet samt för att planera upp gemensamma evenemang. Mötena leds av kårens Aktivitetssamordnare som är en av heltidarna som har ansvar för kårens fritidsverksamhet. Historiskt sett har Krögaren inte varit en del av Sexkollegiet, då diskussioner oftast rör sig om sittningar men jag såg till att bli en del av kollegiet och är mycket nöjd med det beslutet. Även om inte alla diskussioner berör Krögarens ansvarsområde är det ett fantastiskt forum. Man har mycket att lära av varandra samt får en inblick i hur kåren och de andra sektionerna sköter sin verksamhet. Att vara en aktiv del i Sexkollegiet ger också möjlighet till samarbete om man vill hålla pubar tillsammans med andra sektioner men också en språngbräda för att sprida info om Källarmästeriets verksamhet då tyvärr alla teknologer inte vet att E-sektionen håller pub varje fredag.

\section{Metodik för gille}
Nedan följer ett bra rättesnöre för hur planeringen och genomförandet av ett gille brukar ske. Den har passerat genom flera generationer av Krögare med små förändringar och tillägg här och där och är en bra auktoritet i början av en Krögares mandatperiod, men är även oerhört användbar när man minst anar det.

    \subsection{Saker att göra inför varje termin}

    \begin{itemize}
        \item Försök bestämma alla dagar som det passar att ha gille under terminen så tidigt som möjligt, så dina källarmästare kan planera in vilka gånger de kan jobba. Det sker alltid förändringar i schemat men det är bra att ha god framförhållning. För att hitta lämpliga och olämpliga fredagar/lördagar så kollade vi i skolschemat.
        \item Om ni vill samarbeta med andra sektioner så kontakta dem i god tid så att ni kan hitta ett bra datum och kan planera in hur många jobbare som behövs. Ofta är det bra att kontakta Sexmästaren när det gäller pubverksamhet. Här är även Sexkollegiet ett bra forum att leta sektioner öppna för samarbete.
        \item Kom ihåg att det brukar vara pubrunda en gång varje termin som planeras gemensamt i Sexkollegiet och där en given mängd sexmästerier från de olika sektionerna brukar delta. Då E-sektionen är ensamma om att essentiellt ha två sexmästerier brukar Källarmästeriet och Sexmästeriet turas om att hålla i pubrundorna.
        \item En bra idé är att så fort datumen för gillena är spikade så se till att boka både Edekvata och Fika fika till sexan efteråt. Edekvata bokas via Vice Förvaltningschef och Fika fika bokas via E-sektionens hemsida.

    \end{itemize}

    \subsection{Saker att göra i god tid före gillet}
    \begin{itemize}
        \item Bestäm vilka och hur många som jobbar. Ett bra antal jobbare är 6-7 st.
        \item Skriv till husintendenten PH om vilka som är brand- och serveringsansvarig (Avig/Bravig) för gillet, det vill säga vilka Krögare som jobbar sagda gille. Detta ska göras minst en vecka innan gillet. Skriv namn, personnummer, telefonnummer och bifoga en bild.

        Mailadress: per-henrik.rasmussen@ehuset.lth.se
        \item Beställa affischer och bild till tv-skärmen från Picasso (om denna posten finns) ca två veckor innan ett gille.
    Affischer mailas till PH som trycker ut dom. Häng sedan upp affischer och lägg in bild på tv-skärmen, via hemsidans funktion Ekoli, ca en vecka innan gille. Vill man affischera i de andra husen så kontakta informationsansvarig på andra sektioner om tillåtelse för affischering. Det räcker att få tillstånd en gång så är det bara att köra på sen.
        \item Skapa ett Facebook-evenemang via E-sektionen, använd samma evenemangsbild som den som läggs upp på skärmarna. Tips: Bjud in alla som ni bjöd in på förra evenemanget. Finns en knapp i marginalen till vänster när ni klickat på “bjud in”
    där alla gamla evenemang finns.
        \item Om gillet hålls på ett annat datum eller det väntas gäster som som inte innefattas av det fasta tillståndet, måste det sökas ett utökat tillstånd. Blankett om utökat tillstånd hittas enklast genom att Googla efter det. Alla detaljer kan skrivas in direkt i hemsidan och trycks sedan ut så Förvaltningschefen kan signera och skicka in till Tillståndsmyndigheten. Kom ihåg att sista datumet för att skicka in ett tillstånd är två veckor innan evenemanget.
        \item Prata med Cølen angående specifika dryckesönskemål om så finnes och att det finns tillräckligt med alkohol till det kommande gillet.
        \item Bestäm tema, eventuell variant på klägg, temadrinkar och pynt eller liknande krimskrams.
        \item Beställ mat via Cafémästeriets Inköpsansvariga i tid. Saker som ofta behöver beställas är bröd, hamburgerost samt vilken typ av kladd som önskas i kläggen (majonnäs, BBQ). Caféet och KM beställer mat från Martin \& Servera och på deras hemsida finns ett stort utbud av olika hamburgeringredisenser så möjligheten till nytänk är stor. Inloggningsuppgifter erhålles även det av Inköpsansvariga. De beställer två gånger i veckan: Tisdag med leverans onsdag och fredag med leverans måndag.
        \item Vid större evenemang kan det vara bra att uppdatera hemsidan med reklam för det kommande gillet.
        \item Gör en temaenlig spellista på sektionens Spotify. Ha även en spellista redo för antingen mysig pubkänsla och en för fest. Skriv gärna ett inlägg i Facebookgruppen för KM om att det gjorts en ny stor spellista så kan Källarmästarna få vara med och påverka vad för musik som spelas på gillena.
        \item Skicka in text till HeHe om gillet. Prata med Chefredaktören för att veta vilken dag som det är lämpligt att skicka in så det hinner komma med.
        \item Skriv en notis till Kårnytt om veckans gille. Detta görs på kårens hemsida under ``Studerande'' och sedan ``Nyhetsbrev'' Bifoga samma evenemangsbild som den till skärmarna. Behöver skickas in senast söndagen innan gillet.
        \item Se till så att det finns rena handdukar, jätteäckligt med äckliga handdukar. Det är faktiskt jätteäckligt.

    \end{itemize}

    \subsection{Saker att göra dagarna före gillet}
    \begin{itemize}
        \item Skriv till Förvaltningschefen för att få växel till gillet. Det räcker gott och väl med ett meddelande via Facebook. En lagom summa till ett gille är 1500 kr.
        \item Om det förväntas komma mycket folk till gillet, se till att låna LEDs kassautrustning så ni har dubbla kassor.
        \item Handla mat och jobbargodis! Resterande ingredienser som inte köptes från M\&S inhandlas förslagsvis från Willys på Kämnärsrätten eller Ica Tuna. Se till att köttet är svenskt. Glöm inte att skriva en kvittoförstärkning och att sätta igång silverkylen innan ni åker iväg. Ett tips är att inte förvara “stöldbenägen” mat i silverkylen innan själva gillet, typ godis och mjölk.
        \item Påminn Källarmästarna om vilka som jobbar och när ni ska samlas. Se till att Källarmästarna vet att de ska ha med sig leg och kårleg och att de med långt hår ska ha det uppsatt. Detta är bra att skicka på onsdagen/tidigt på torsdagen samma vecka.

    \end{itemize}

    \subsection{Saker att göra tidigt på gilledagen}
    \begin{itemize}
        \item Kontrollera att lokalen är rimligt städad och att diskmedel, handtvål och handsprit finns.
        \item Skapa evenemang på hemsidan (måste göras innan utskrivning av alkohol):
        \begin{dashlist}
            \item Logga in på esek.se
            \item Tryck på Funktionär $\rightarrow$ Alkoholhantering
            \item Under fliken Händelser så fyll i gillets namn, till exempel ``Cocktailgille'' och rätt utskott. Extrakolla även så att datumet stämmer för den dagen du håller puben.
            \item När alkoholen skrivs ut och in, se till att det skrivs på den korrekta händelsen.
        \end{dashlist}
    \end{itemize}


    \subsection{Saker att göra innan försäljning öppnar}
    Om inte denna lista redan finns utskriven i KM är det ett bra förslag att skriva ut och använda som checklista under kvällen.
    \begin{itemize}
        \item Se till att alla jobbare skriver upp sig i personalliggaren, den ligger i kassaskåpet.
        \item Skriv ut jobbardricka från CM. Detta görs i pärmen med röd rygg och en framsida där det tydligt står att den är för alla övriga utskott.
        \item Skriv ut alkohol från AHS med hjälp av scannern och sätt in i kylen. Glöm ej alkoholfritt! Alla alkoholalternativ måste ha motsvarande alkoholfritt alternativ. Glöm ej att spara alla tomflaskor. Detta är bra att börja med redan när de första jobbarna kommer då det tar tid! För övriga frågor om AHS se kapitlet nedan.
        \item Skriv meny på de fina diodtavlorna, här måste alla alkoholfria alternativ och mat stå. Använd en specialpennan som hänger i tavlorna ståltråd. Färgen tas bort med blöt trasa och papper.
        \item Börja förbereda maten. Tvätta händerna och sprita händerna! Plasthandskar till att forma klägg finns i KMs
    skåp i köket. Plasthandskar beställs via CM så håll koll på att de inte tar slut.
        \item Plocka fram glas och sugrör till drinkar.
        \item Ta fram hålligånggången. Slå på den om försäljningen går trögt.
        \item Sätt på lamporna i Vega. Brytaren finns ovanför städskåpet direkt till vänster i Sicrit.
        \item Sätt på musiken i alla rum på lagom volym.
        \item Fyll diskmaskinen. Vrid brytaren till på och dra ner handtaget så börjar diskmaskinen fyllas.
        \item Ta fram växel som skrivits ut, spara lappen med hur mycket det är och kontrollräkna mängden växel. Ställ i ordning iZettle.
        \item Sätt igång kökets ventilation genom att trycka på och hålla inne den gröna knappen vid soptunnan i Edekvata tills den tänds.
        \item Se till att det finns tillräckligt med kläggbiljetter.
        \item Ta fram pantpåse.
        \item Om det är någon ny temadrink, skriv en beskrivning till kvällens drinkar till jobbarna.
        \item Kolla att allt sortiment finns i iZettle.
        \item Ta fram alla lathundar som behövs, sitter i plastmappar i KM.
        \item Ta fram soptunnor till köket, två stycken. Soppåsar finns i Sicrit.
        \item Ta fram den vita datorn till entrén, finns i HK. Bra hemsidor att ha uppe är medcheck och en räknare för att hålla kolla på mängden folk i lokalen.
        \item Ha genomgång med jobbarna om kassa, klägg, drinkar etc.
        \item Ta fram handdukar.
        \item Kolla vilka brandsläckare som sitter uppe i huset.
        \item Dra ner persiennerna i Edekvata. Detta görs för att inte Securitas eller Tillstånd ska bli nyfikna.

    \end{itemize}

    \subsection{Saker att tänka på under kvällens gång}
    \begin{itemize}
        \item Ta sista beställning en halvtimme innan ni vill stänga. Gå runt i lokalen med stora sleven och slå i hålligångången skrikandes ``Last call!'' så gästerna förstår.
        \item Öppna och servera inte drycker innan kunden har betalt.
        \item Lämna aldrig ut oöppnade flaskor eller burkar. Dessa ska alltid öppnas innan kunden får sin dryck.
        \item Ställ aldrig sprit eller pengar inom armlängds avstånd från gäster.
        \item Tänk på att alkoholservering bör göras enligt befintlig policy.
        \item Håll baren ren. Torka av spill direkt och plocka undan shotglas. Jobbarna är generellt dåliga på detta så man får tjata lite.
        \item Avvisa de gäster som bör avvisas. Spritstoppa de som bör spritstoppas.
        \item Kolla toaletter kontinuerligt, det händer att folk skändar dem. Komprimera pappret i papperskorgarna så slipper man byta påsar och plocka papper från golvet.

    \end{itemize}

    \subsection{Saker att göra när gillet är över}
    \begin{itemize}
        \item För in all dricka som är över i AHS och sätt in i spritförrådet. Kom ihåg att väga in tomflaskor och sedan plocka ut dom som tomflaskor. Se kapitel nedan för mer info.
        \item Skriv ut försäljningsrapport från iZettle. Gör försäljningsredovisning, tomma ark finns i HK.
        \item Räkna alla kontanter medan jobbarna städar. Ta papper som ligger på nattfack och fyll i enligt anvisningar.
        \item Sedlarna ska ligga i buntar om 10st. Alla åt samma håll med en sista sedel vik över mitten. Mynt får ej ligga i rör om det ej är rätt antal mynt som röret visar.
        \item Städa!
        \item Pantpåse: Sätt på en extra pantpåse så att det är dubbla påsar, stäng påsen med ett buntband och placera en etikett på buntbandet, allt finns i CM. Lägg påsen prydligt i CM så blir Cafémästaren glad.
        \item Kolla så att alla brandsläckare finns kvar (framförallt den som är utanför vaktmästeriet!).
        \item Se till att all köksutrustning är avstängd och att silverkylen är tömd.
        \item Lås alla skåp.
        \item Stäng av musik.
        \item Släck alla lampor.
        \item Kolla husets toaletter och foajé efter spyor.

    \end{itemize}


\section{Återkommande gillen}
\subsection{Välkomstgillet}
Som tradition håller KM nollningens första alkoholevenemang, Välkomstgillet. Då alla nollor och deras phaddrar kommer hungriga från stadsvandringen är det ett evenemang som gynnas av god planering för undvika kaos i köket och enorm kläggkö.
\begin{itemize}
	\item Planera upp tidigt, helst redan under våren, med Phøset om hur ni vill att evenemanget ska gå till och hur insläppet av phaddergrupper ska skötas. Skulle dom av någon anledning trilskas så är det bara att sätta ner foten om att det är du som är ansvarig för puben och att insläppet sköts som du vill.
    \item Förbered många klägg i förväg. Börja stek biffarna och håll dom varma i ugnen så väntetiden och stressen i köket minimeras. 2017 gick det åt över 200 klägg, inklusive klägg till jobbare.
    \item Se till att Cølen köper in gott om öl och cider. Under vecka 0 får det inte säljas ren starksprit så det kommer vara extra tryck på öl och cider. Ett utmärkt tillfälle att beställa in och servera fatöl.
    \item Drinkar får säljas men skylta inte med starkspriten då folk kommer vilja börja shotta om det står framme.
    \item Då Phøset kommer vara upptagna med att hålla stationer runt E-huset kommer istället ØGP vara dina närmaste kontakter för att hålla koll på phaddergrupperna och ner de anländer.
    \item Två Phaddergrupper var tionde minut är ett bra tempo för att både undvika lång kö i början av gillet samt att kläggen hinner slängas ut i en stadig ström.
    \item Plocka gärna med en eller två extra arbetare till Välkomstgillet då det har en tendens att bli av det hetsigare slaget när så mycket mat ska slungas ut. Sätt gärna rutinerade personer att bemanna baren, både för effektiviteten samt att hålla kolla på överförfriskade nollor och phaddrar.
    \item Antalet folk som vill gå på gillet kommer överstiga antalet vi får ha i lokalen. Håll noga koll på att det inte överstiger 150 personer, att vara på den säkra sidan är bättre än att Räddningstjänsten dyker upp och räknar för många gäster. Prioritet är alltid nollor så kommer det tillbaka nollor från rundvandring och det är maxkapacitet är det bara att börja slänga ut folk. Släng ut Kårheltidare och Styrelsen först.
    \item Häng gärna med sektionen ut och se nollorna måla! Städningen kan man ta senare på kvällen.

\end{itemize}

\subsection{UtEDischot}
Varje nollning ordnas det stora UtEDischot i samarbete mellan D-sektionens och E-sektionens Nöjesutskott och KM står traditionsenligt för matservering genom att slunga klägg i absurda mängder. Alla kostnader och intäkter går oavkortat till UtEDischot. Det ni köper in för försäljning ska alltså vidarefaktureras till Data som är ansvariga för ekonomin.
\subsubsection*{Följande bör tänkas på under UtEDischot}
\begin{itemize}
	\item Se till att boka grillar i god tid innan. Stora portabla grillar baserade på
gamla oljefat är att föredra och det är lagom att ha fyra sådana. Glöm inte att
galler behövs, såväl som möjlighet att fylla på kol och hålla jämn temperatur. E-sektionen äger två grillar, sen brukar man kunna låna F-sektionen två grillar. Kontakta Sexmästaren på F om det går att låna.
	\item Se till att frysta hamburgare, bröd och alla tillbehör finns. Det brukar gå åt uppemot 600 hamburgare, inklusive veg. 2017 började vi även erbjuda korv med bröd, vilket var uppskattat.
    \item Ta hellre för mycket ingredienser än för lite. Blir det råvaror över tillfaller det KM.
    \item Se till att fixa många jobbare som är beredda att jobba lång tid. Åtta källarmästare och två krögare är en lagom mängd jobbare.
    \item Det kan vara smidigt att beställa grillkol via CM då man kan få
väldigt stora säckar och slipper åka iväg med bil och köpa. Det är okej att få lite
över.
	\item Se till att tina en rimlig mängd bröd, det är jättejobbigt om det är frusna. Det tar ungefär ett halvt dygn att tina alla bröd ifall de är väl utspridda.
    \item Se till att det finns ordentligt med pappfickor att dela ut kläggen i.
    \item Börja i god tid med att ställa i ordning, det kan vara bra att använda bord som
finns i Ekea. Prata med arrangörerna (Entertainern) för att bestämma hur området ska avgränsas och byggas upp. Grindar som är midjehöga är att föredra
och ett bord där man tar emot beställning och serverar.
	\item Priset för klägg bör rimligen ligga runt 30-35 kr. Räkna på matkostnaderna och kom överens om ett rimligt pris. Tänk på att alla arbetarna får gratis klägg och gamla Krögare samt gamla Vice Krögare.
    \item Se till att skriva upp hur många klägg ni säljer!
    \item Var beredd på att gäster försöker sno klägg och tränga sig i kön. Därför är det bra med ett tydligt kösystem så ni får en bra överblick på gästerna.
	\item Tänd grillarna tidigt! Det tar tid för dem i början att bli varma och få bra glöd.
    \item Håll koll på grillarna. Se till att det alltid finns minst två som är igång. Efter man fyllt på mer kol tar det ett tag innan man får bra glöd igen så var taktisk i er påfyllning så ni inte står med kass glöd när klägg måste flyga av hyllorna.
    \item Använd minst två iZettlestationer. Då det är D-sektionen som står för all ekonomi så ska ni vara inloggade på deras konto. Använd en enkel meny för att enkelt kunna ta betalt utan menykrångel.
    \item Ta med ca hälften av de frysta burgarna från början i en kundvagn och låt dem
ligga ihop så att de inte tinar så snabbt. Hämta mer hamburgare en till två gånger
under kvällens gång för att det inte ska tina.
	\item Var beredd på att det kan önskas att jobbarkläggen ska vara redo innan evenemanget öppnar.
    \item Om ingen motsäger sig, så jobbar man traditionellt i ouvve och KM-tröja.
    \item Som tack för att man jobbar UtEDischot för man gå på en redig tackfest med käk och badtunnor. Räcker inte detta som morot så tvinga Källarmästarna att antingen jobba UtEDischot eller Regattan. Jobbare kommer alltid behövas och man får tackfest för båda evenemangen.
    \item Källarmästeriet har ingen skyldighet att vara kvar och städa och riva ner resten av UteDischot men det uppskattas enormt om man gör det.
\end{itemize}

\subsubsection*{Tankar från UtEDischot 2017}
\begin{itemize}
	\item All mat beställdes från Martin \& Servera via CM. Vid möjlighet ta denna beställningen innan nollningen börjar för att undvika stress och krockar med andra evenemang. Den stora utmaningen blev att hitta tillräckligt med frysplats, då det är enorma mängder frysvaror som måste förvaras. Oftast brukar det gå att låna Sexets frysar samt frysarna i LED. Snacka med Cafémästaren och Sexmästaren. Alternativt kan allting handlas från Snabbgross dagen innan men då måste man beställa i förväg så alla råvaror finns.
    \item Köp frysta och färdiga 150 grams hamburgare. Frysta falafelburgare som vegetariskt/veganskt alternativ.
    \item Vi sålde korv med bröd för 15 kr vilken var ett populärt alternativt och rekommenderas till efterföljande år.
    \item Pris var 35 kr per klägg.
    \item Ett utmärkt tillfälle att göra reklam för Källarmästeriet och gillen på fredagar. Tryck upp några av de allmänna affischerna och placera dom över borden.
    \item Tyvärr fattar ingen vad klägg är så förtydliga på menyn att det är hamburgare ni säljer.
    \item Öronproppar till alla jobbare är bra. Ljudvolymen är väldigt hög!
    \item Använd matberedaren för att hacka alla grönsaker så sparas mycket tid.
    \item Vi samlades vid 17.00 och började förbereda maten och ställa i ordning tältet. Vi blev klara i alldeles lagom tid till öppning vid 21.00. Förbered jobbarna på att det är ett långt pass. Dischot stänger 03.00 och sen städar man en dryg timme innan man kan skicka hem de tappra jobbarna.
    \item Tänk ut ett smidigt kösystem. Vi skapade en korridor av bord och kravallstängsel längs tältets långsida där man beställde och betalade i början av kökorridoren. I handen fick man en laminerad kläggbiljett som man bytte ut mot sin mat i slutet av kön. Det fungerade utmärkt.
    \item Fixa extrastolar åt jobbarna! Det är jobbigt att stå upp sex timmar.
    \item Totalt såldes 457 klägg och 120 korv med bröd.
    \item Råvaror som gick åt:
    \begin{dashlist}
    	\item ca 450 frysta hamburgare
        \item 111 falafelburgare
        \item 150 korvar
        \item 15 sojakorvar
        \item 6 glutenfria hamburgerbröd
        \item 13 kg isbergssallad
        \item 5 kg rödlök
        \item 12 kg tomater
        \item 8 flaskor ketchup
        \item 8 flaskor Heinz senap (220 ml)
        \item 8 flaskor hamburgerdressing
        \item 3 kg rostad lök (tog slut snabbt)
        \item 3 flaskor tändvätska
        \item Uppemot 120 kg grill kol. Det går åt absurda mängder för att hålla igång 4 grillar i 6 timmar så se till att det finns i lager.

    \end{dashlist}
    \item Se till att någon eller några är där tidigare på dagen för att plocka fram brödet för att tina samt avstämning med Entertainern om vad som förväntas av kvällen.
    \item Bra tips är att dela upp vanliga klägg och vegetariska i iZettle så blir det enklare att hålla reda på hur mycket som sålts av de båda för framtida referens.

\end{itemize}

\subsection{Regattan}
Per tradition har KM även blivit en självklar del av Regattan, där vi står och slungar ut massor av klägg och läsk till hungriga nollor och phaddrar. Ett utmärkt tillfälle att dra in extra stålar och visa upp Källarmästeriet.

\subsubsection*{Följande bör tänkas på vid Regattan}

\begin{itemize}
	\item Maila Nollegeneralen att ni vill sälja klägg på Regattan likt tidigare år. Ta sedan kontakt med ansvarig Øverste och diskutera ut detaljerna, plats att stå, jobbarmat etc.
    \item Använd matberedaren för att hacka alla grönsaker. Skulle den vara borta eller ur funktion, maila Pedellen och fråga om går att låna kårens Robot coupe, som är en matberedare i industriformat.
    \item Sälj färdiga klägg som på UtEDischot.
    \item Jobba i den snyggaste tröjan som finns och ouvve så ser folk att vi är från E. Placera ut affischer för att locka folk till fredagsgillena.
    \item Likt UtEDischot vet inte en själ vad ett klägg är så förtydliga att det är hamburgare ni säljer.
    \item Träffas med jobbarna (det var lagom med två krögare och sex jobbare) ca kl 11. Tåget brukar anlända kring 14 och det är skönt att ha grillarna igång och att ni
och jobbarna hunnit äta innan. Sen är det full rulle några timmar och svårt att ha
tid att äta.
	\item Använd er av partytältet som finns i Ekea. Ta med dig några av jobbarna och sätt
upp det tillsammans med grillar (2st) och bord från Ekea medan de andra hackar grönsaker.
	\item Det är bra att ha utdelningen av klägg i en annan del av partytältet än delen där man betalar, det blir mycket mer rullande band då. Löste ni ett bra system på UtEDischot så återanvänd det.
    \item Se till att det finns jobbargodis, läsk (till försäljning och jobbare, skrives ut ur CM) och plasthandskar till de som lägger ihop kläggen.
    \item Ni behöver inte ha ost på kläggen och man kan möjligtvis ta 40 kr för kläggen om
budgeten verkar svår att uppnå och om man har fina burgare. Regattan har en tendens att bli ett rent vinstevent.
	\item Ska någon annan sälja mat under Regattan så snacka ihop er innan eventet för bästa samarbete. Tidigare år har Allium och Korv å Hoj stått och sålt mat, vilket minskat trycket på kläggtältet något.
	\item Har man tur ska grillarna lånas ut till kvällens Regattafest och då slipper ni städa grillarna.
    \item Glöm inte att skriva upp hur mycket av allt ni köper och använder så att nästa års krögartrio kan dra nytta av er visdom.
    \item Som jobbare på Regattan får man gå på en fulsittning i Cornelis som heter Alternativregattan tillsammans med alla som jobbat. Maila Nollegeneralen i tid så vederbörande kan skicka anmälningslänk.

\end{itemize}

\subsubsection*{Tankar från Regattan 2017}
\begin{itemize}
	\item I år drog vi igång en timme tidigare med samling kl 10, vilket gjorde att vi var klara i god tid innan tåget anlände. Då grillarna inte hade blivit klara i tid för att kunna ge jobbarna mat köpte vi pizza som funktionärsvård. Det är dom värda.
    \item Kör samma eller liknande hamburgare som på UtEDischot. Vi körde färdiga burgare, isbergssallad, pressgurka, tomater och rödlök.
    \item De vanliga falafelburgarna var slut på M \& S så vi fick ta de mindre 50-grammarna och köra två burgare i ett bröd, vilket inte var helt optimalt. Beställ gärna tidigare än samma vecka, då det ändå är frysvaror och kan lätt förvaras.
    \item Var taktisk med grillarna och påfyllning av kol, då det tar ett tag innan den nya kolen hinner bli varm. Tar det inte fyr som man vill kan man ta hjälp av mini-brännaren som står i Sicrit.
    \item Ta med öronproppar till jobbarna. Pyrot står och skjuter kanon precis intill och
det är allmänt hög ljudvolym.
	\item Hacka upp alla grönsaker och ta med allt i bläck till försäljningsplatsen. Det kommer inte finnas mycket tid att springa och hämta grönsaker i Edekvata under försäljningsrushen.
    \item Använd samma iZettlemapp som UtEDischot men lägg till en vara som Vegetariskt klägg om det inte redan finns. Det blir mycket lättare att hålla koll på vad som sålts och ger en uppskattad statistik till följande år.
    \item Ladda upp med en rejäl mängd läsk då det går åt som smör. Det är jobbigt att springa och hämta.
    \item Pris på klägg var 35 kr.
    \item Ladda iZettle-stationerna och lös nätverk. Det är inte alltid säkert att Eduroam räcker ut.
    \item Ta hellre ut för mycket kol än för lite.
    \item Råvaror som gick åt:
    \begin{dashlist}
    	\item 37 fullstora falafelburgare
        \item 40 st 50 g falafelburgare
		\item ca 200 st hamburgare à 150g
		\item 9 säckar grillbriketter à 2.5kg
		\item 11 gurkor
		\item 8 huvud isbergssallad
		\item 3 kg tomater
		\item 3 kg rödlök
		\item 5 ketchupflaskor

    \end{dashlist}
    \item Totalt såldes 258 klägg och 101 läsk.

\end{itemize}

\subsection{Julgillet}
Varje år arrangerar KM det traditionsenliga Julgillet! Detta arrangerar sittande och
invalda KM tillsammans, vilket är superkul! (speciellt eftersom nya KM städar sen på
söndagen efter julgillet). I samband med Julgillet är det även en pepparkakshustävling. Under själva julgillet jobbar nya och gamla krögartrion och fungerar som själva överlämningen. Det är också då nya krögartrion blir dubbade och får ansvaret för
krögarsvärdet. Är krögarsvärdet inte i er ägo? SKÄMS! (och LÖS det!). Ta bilder på alla
Krögare med svärdet när alla gått hem, det är kul och ger fantastiska minnen!

\subsubsection*{Följande bör tänkas på i samband med julgillet}
\begin{itemize}
	\item Släpp biljetter via länk innan julgillet. Informera om att länken kommer släppas
ungefär en vecka innan och påminn under HT-möte och valmöte. De två senaste åren har det sålts 110 biljetter och det har fungerat bra, även om det blir något trångt. Det får plats 90 sittplatser i Vega med plats för 20 stycken till i Diplomat.
	\item Efter alla biljetter sålt slut är det uppskattat att sätta upp en länk där man kan man anmäla sig till reservplatser. Det är ofta folk som anmäler sig men inte betalar eller upptäcker att de inte kan gå.
	\item Personer som ska bjudas skriftligt in till Julgillet, d.v.s. de betalar ej för eventet är följande:
    \begin{dashlist}
    	\item Alla gamla Krögare med teknologkort.
        \item Förra årets Vice Krögare.
        \item Sittande Ordförande.
    \end{dashlist}
    \item Har det gått bra KM under året finns det även möjlighet att utöka tillståndet så det går att bjuda in folk som inte har teknologkort. Då ska följande personer bjudas in utöver de ovan nämnda:
    \begin{dashlist}
    	\item Krögare Emeritus. Din företrädare hjälper dig att ta kontakt med dom då maillistan är sannerligen förlegad.
        \item Inspektorn.
        \item Hedersmedlemmar.
    \end{dashlist}
    \item Alla i Styrelsen ska bli anmodade skriftligt tillsammans med gamla Vice Krögare som fortfarande har teknologkort. Anmodan innebär att gästen är garanterad en plats men får betala för sin plats. Inbjudna får de första platserna, sedan anmodade. Resterande köplatser beräknas utifrån när personerna registrerade sin anmälan.
    \item Biljetterna betalas med fördel efter att anmälan skett.
    \item Vid inträdet till julgillet så får alla en biljett med sitt könummer. Enklast är att skriva könumret på gästens hand. Vill man gå den extra milen finns fina biljetter att skriva ut i mappen Julgille i KMs mapp i datorn Buzzard.
    \item Planera vad som ska inhandlas och vad som ska lagas med nya Krögartrion några
dagar innan Julgillet (förslagsvis tisdag eller onsdag).
	\item Handla maten på Snabbgross på torsdagen. Se till att silverkylen är igång och
tömd innan ni åker. Var inte fler än fyra när ni åker, det blir väldigt trångt annars.
	\item Laga maten som kan förberedas kvällen innan. Nya och gamla KM jobbar, ni
leder arbetet och behöver därmed inte rulla $10^{12}$ köttbullar! (Wieeeee!)
	\item Se till att all dukning också görs denna dag. Det är riktigt porslin
som gäller denna dag, inte engångsgrejer.
    \item Mycket av pyntningen kan göras i förväg. Det är trevligt med ljusslingor och
glitter i taket. Ofta finns det en hel låda med ljusstakar och liknande i Ekea.
	\item För att ha koll på maten som ska köpas in kan det med fördel kollas upp i driven
för Julgillet som tidigare Krögare bör ha delat med sig av. Se till att föra denna vidare till nästa generation så att info om mat, recept och mängd kan återanvändas, det sparar
otroligt mycket tid!
	\item Snåla inte med maten, detta ska ses som ett tack till sektionens medlemmar så
beroende på hur ni ligger till ekonomiskt kommer Julgillet kosta olika mycket
alternativt kan ni köpa in mer mat. Det gör inget om nya KM får lite rester som
jobbarmat dagen efter, det är mer eller mindre förväntat.
	\item Se till att utvärdera samma dag och dagen efter om det var för lite av något och
om det var orimliga mängder kvar av något. Detta för att kunna köpa in rimligare
mängder till nästa år. Spara detta på ett ställe där ni lätt kan hitta det nästa år.
	\item På själva julgille-dagen jobbar nya och gamla krögartrion i de vackra skjortorna
och tomteluva. Inga undantag!
	\item Gör reklam för pepparkakshustävlingen genom att hetsa folk och sätta upp affischer. Mycket roligare om det är fler som deltar. Fixa sedan ett rejält pris.

\end{itemize}

\subsubsection*{Tankar från Julgillet 2017}
\begin{itemize}
	\item Det är rejäl hets på biljetterna så gör biljettsläppet till ett kul evenemang. Vi sålde slut 95 biljetter på 9 minuter.
    \item Beställ affischerna i god tid från Picasso, eller den som råkar vara ansvarig för att formge marknadsföringen. Påminn om Pepparkakshustävlingen och hänvisa till evenemanget.
    \item Ta mailadress i anmälan så ni kan skicka ut påminnelse om betalning till alla anmälda. Facebooks algoritmer är väldigt selektiva i hur informationen når dina gäster, även om du påminner i både evenemanget, E-sektionens allmänna sida samt E-sek events.
    \item Julgillet är KMs största evenemang så se till att ni delar upp förberedelserna jämnt.
    \item Då det är endast sex personer, nya och gamla Krögartrion som jobbar på själva julgillet är det viktigt med planering. En lagom tid att börja förbereda är 13-14. Hellre för mycket tid att förbereda än för lite.
    \item Då det är en sittning så gör ni bäst att sälja 3 cl snaps med lock som gästerna kan ta med sig till bordet. Fråga Sexet om ni kan ta från Pump.
    \item E-sektionens egna julsnapsar har blivit ett välkommet tillskott till Julgillet. Se till att Cølen köper in och kryddar ett par flaskor.
    \item Under insläppet blir det det en bra uppdelning att ha två per i insläppet, en som langar glögg, två i baren och en som fortsätter förbereda maten. Förslagsvis någon som har hållit på med maten under dagen.
    \item Av någon anledning tappade gästerna huvudet under sittningen och började sjunga bastuvisor utan att någon sa till. Ett bra förslag är därför att införa sångförmän eller liknande toastmasters till nästa år som kan hålla koll på att gästerna inte gör något olämpligt. Kommer även agera som en bra avlastning från Krögaren som annars blir den som får springa runt och hålla koll på spex och att sittningen rullar på som den ska.
    \item Att köpa in en lite finare fatöl var mycket uppskattat och passade väldigt bra till en lite finare sittning. Fatöl är alltid uppskattat.
    \item Planera vart ni ställer maten. De maträtter som ligger allra först kommer gå åt i mycket större grad än om det som ligger i slutet av kedjan. Folk är glupska och är dåliga på att balansera en välavvägd jultallrik.

\end{itemize}

\section{Alkoholförråd och Alkoholhanteringssystemet (AHS)}
Alkoholförrådet och AHS är i första hand Cølens ansvarsområde men det kräver även en utbildad Krögare för att kunna hålla det städat och kunna rätta till fel i arrangemangsrapporterna. Detta är något som tyvärr kommer ske under ett år som Krögare, vare sig hur bra man tror sig utbilda sina jobbare. Detta är även för att du ska kunna lära upp Sexmästeriet i korrekt hantering av alkoholförrådet och AHS, för att slippa det kaos som annars uppstår efter Sexmästeriet haft sitt första alkoholevenemang. Kaos i AHS med felinvägda spritflaskor eller missade tomflaskor blir i utsträckning ett ok för Krögaren då det är Källarmästeriet som i nästa evenemang ärver kaoset och att spritdiffen, skillnad mellan det som finns bokfört som lager och det som faktiskt sålts, faller på KMs budget. Det är inget annat än ren och skär självbevarelsedrift att man som Krögare försäkrar sig om folk som ska använda spritförrådet vet vad dom gör.

Se till att dessa lathundar finns tillgängliga för arbetarna för att minimera att det vägs in fel. Det är ett imperfekt system som inte har mycket mer skyddsnät än ``\textit{Nåde dig om du gör fel}''. Håll god kontakt med Cølen och Macapären för att kontinuerliga fixa fel i AHS. Skulle det vara uppenbara lagerfel i AHS kan man ändra det direkt under lagerhantering, men håll det till öl och AJA. Spriten ska helst inte korrigeras inför ett gille då det oftast behöver undersökas närmare efter fel för att inte bli helt helt oigenkännligt kontra det bokförda lagret.

\subsection{Uttag från alkohollagret}
\subsubsection*{Öl, AJA, vin och lättdryck}
\begin{itemize}
	\item På E-sektionens hemsida, under fliken Funktionärer, tryck på Alkoholhantering
    \item I menyn Bokning, sök på dryckens namn eller scanna flaskans barcode
    \item Kontrollera att det är rätt evenemang
    \item Kolla så att det är uttag som registreras, programmet kommer varna annars
    \item Skriv in vilken mängd som tas ut.
    \item Kolla försäljningspris
    \item Klicka registrera
    \item Sätt in i kylen med rätt försäljningspris (alkoholfritt längst upp till höger i kylen)
\end{itemize}
\subsubsection*{Sprit}
\begin{itemize}
	\item På E-sektionens hemsida, under fliken Funktionärer, tryck på Alkoholhantering
    \item I menyn Bokning, sök på dryckens namn eller scanna flaskans barcode
    \item Kontrollera att det är rätt evenemang
     \item Väg flaskan
    \item Skriv in hur mycket flaskan väger
    \item Kolla så att det är uttag som registreras, programmet kommer varna annars
    \item Kolla försäljningspris, vanlig sprit har ett pris på 5-6 kr/cl. Whisky och finsprit har ett pris på 10-15 kr/cl.
    \item Klicka registrera
    \item Vanlig sprit ska ställas på bordet framför spegeln. Allt som är dyrare ska ställas i spegelhyllan så det inte av misstag plockas med i en Orup
\end{itemize}

\subsection{Intag av dricka i alkohollagret}
\subsubsection*{Öl, AJA, vin och lättdryck}
\begin{itemize}
	\item På E-sektionens hemsida, under fliken Funktionärer, tryck på Alkoholhantering
    \item I menyn Bokning, sök på dryckens namn eller scanna flaskans barcode
    \item Kontrollera att det är rätt evenemang
    \item Kolla så att det är intag, \textit{inte} uttag som registreras
    \item Skriv in vilken mängd som ska tillbaka in i lagret
    \item Klicka registrera och skeppa tillbaka i förrådet

\end{itemize}
\subsubsection*{Sprit \textit{utan} tomflaskor}
\begin{itemize}
	\item På E-sektionens hemsida, under fliken Funktionärer, tryck på Alkoholhantering
    \item I menyn Bokning, sök på dryckens namn eller scanna flaskans barcode
    \item Kontrollera att det är rätt evenemang
    \item Väg flaskan
    \item Skriv in hur mycket flaskan väger
    \item Kolla så att det är intag, \textit{inte} uttag som registreras
    \item Klicka registrera och skeppa in i spritskåpet på rätt hylla

\end{itemize}

\subsubsection*{Sprit \textit{med} tomflaskor}
\begin{itemize}
	\item På E-sektionens hemsida, under fliken Funktionärer, tryck på Alkoholhantering
    \item I menyn Bokning, sök på dryckens namn eller scanna flaskans barcode
    \item Kontrollera att det är rätt evenemang
    \item Väg tomflaskan eller alla flaskor tillsammans, inklusive den tomma
    \item Skriv in totalvikten av alla flaskor
    \item Kolla så att det är intag, \textit{inte uttag} som registreras
    \item Klicka registrera
    \item Under fliken Tomflaskor, sök på dryckens namn eller scanna flaskans barcode
    \item Skriv in tomflaskans vikt
    \item Klicka registrera

\end{itemize}

\section{Krögarens ekonomiska uppgifter}
För varje gille följer en del ekonomiarbete för Krögaren, som gärna skall göras
kontinuerligt under läsperioderna för att ha bra koll på ekonomin. För varje gille måste en arrangemangsrapport skrivas ut från AHS och från rapporten görs även en internfakturering över all alkohol som sålts. Nedan ges instruktioner hur arrangemangsrapport utförs. Guide över hur internfakturering görs finns på separat papper i Krögarens fack. Skulle något vara otydligt är det bara att fråga en gammal Krögare.

Som utskottschef är det även Krögarens uppgift att signera alla inköp som görs till gillena men även för alla inköp som Cølen gör. Ett effektivt, om än lite tidskrävande, är att hålla reda på alla utgifter och inkomster i ett exceldokument. På så sätt lyckades jag matcha mitt halvårsbokslut med förvånansvärt god precision och gjorde att jag hade bättre koll på vilka investeringar och i viss mån extravaganser jag kunde satsa på. Mall finns så det är bara att fråga om.

Du ska även skriva en internbudget så Förvaltningschefen har koll på ekonomin. Detta är lättare sagt än gjort första gången. Ta hjälp av din företrädares internbudget och tidigare resultat i iZettle för att få ett någorlunda hum om dina inkomster och utgifter under året. Internbudgeten för ett så rörligt utskott som KM är aldrig något som är satt i sten utan det är något som kan och ska ändras under årets gång, förslagsvis efter första terminen och en andra gång efter nollningen. Efter dom perioderna har man rätt bra koll på vad man kommer landa på rent ekonomiskt i slutet av året.

\subsection{Arrangemangsrapport}
\begin{itemize}
	\item Logga in på esek.se
    \item Klicka på ``Funktionär'' $\rightarrow$ ``Alkoholhantering''
    \item Klicka på Historik, välj det gille du vill kolla på och se över så att du får en
rimlig vinst, annars kan något ha gått fel vid utskrivning/inskrivning i lagret. Då
måste det fixas till innan nästa steg. Se Felsökning för tips.
	\item Gå in på ``Händelser''. Under rubriken ``Hantera händelser'' hittar du arrangemanget du ska göra internfakturering för.
    \item Se till så att arrangemanget är stängt, annars kan du inte få en
sammanfattning och skriva ut det. Du stänger händelsen genom att trycka på
symbolen längst till höger om arrangemangsnamnet och tryck på ``Stäng händelse''
	\item När du har försäkrat dig om att händelsen för gillet är stängt kan du gå vidare
till nästa steg.
	\item Klicka på ``Historik''. Välj sedan i listan vilket gille du vill få fram arrangemanget för.
    \item Skrolla ner längst ned på sidan och klicka på ``Ladda ner rapport''.
    \item Överst på sidan står det nu ``Skapade en rapport! Klicka här för att ladda ner
den'' med grön bakgrund. Tryck på ``Klicka här''. Nu får du fram en arrangemangsrapport som du kan skriva ut.
    \item Nu är det bara att använda rapporten för att kunna slutföra internfaktureringen. Detta görs på en av de orangea lapparna, designerade för KM och E6.

\end{itemize}
\subsection{Felsökning}
Det vanligaste felet är att en spritflaska blivit inkorrekt invägd eller inte vägd alls. Detta syns oftast enklast om vikten in är noll, vilken den aldrig ska vara på en flaska sprit. Skulle det vara orimlig vinst trots invägning så jämför hur mycket som sålts iZettle för att undersöka om invägningen stämmer eller ej.

Har man tur är det bara att flaskan inte blivit invägd och står i spritskåpet, då är det bara kontrollväga och väga in flaskan i arrangemanget. Jobbigare blir det om flaskan är slängd, då blir det betydligt mer detektivarbete. Försök leta upp mängden sprit såld och subtrahera det från den invägda mängden, AHS beräknar 10 g/cl. I andra fall som t.ex. tomflaskan är invägd i AHS men inte uttagen ur systemet som tomflaska, eller utvägd som flaska utan att bli invägd i AHS är det inte så mycket man kan göra då det inte finns något sätt att undersöka vad som gått fel för att kunna korrigera. Ett enkelt sätt underlätta framtida felsökning är att börja spara tomflaskor och börja föra lista över tomflaskors vikt som sedan kan jämföras med lagret i AHS.


\section{Allmänna tips}
	\subsection{Vad kan gå fel på ett gille?}
\begin{itemize}
	\item Skulle stekbordet sluta fungera är det troligtvis en av propparna som gått. Bekräfta detta genom att dra ur stekbordets sladd och sätta i spisens sladd. Testa alla plattor, lyser spisens lampa med minskad effekt har mittenfasens säkring gått. 16A proppar ska finnas i Sicrit, är dom slut, köp nya och fakturera Förvaltningsutskottet. Proppskåpet finns direkt till vänster utanför Edekvata. Leta upp den brända säkringen och byt ut den.
    \item Ibland inträffar mindre översvämningar i köket. Inträffar det vid ugnen så är spillblecket överfullt. Ta ut och töm det, helst utan att att spilla allt på golvet. Börjar det istället läcka vatten kring diskmaskinen har antingen ett av rören kopplats loss eller så är det stopp i diskmaskinens avlopp. Då finns det inte mycket att göra än flytta diskbänken, gå ner på alla fyra och koppla rätt rören eller börja gräva bort gegga. Notera att detta görs smidigast då det inte pågår ett gille.
    \item Ibland inför ett gille så vill iZettle eller kvittoskrivaren inte alls fungera som man vill. Till det finns en utförlig felsökningsguide, som ska finnas i tre exemplar. En i KM, en i Krögarens fack i HK och en i silverskåpet under baren.
\end{itemize}
	\subsection{Om tillstånd kommer}
    Någon gång under din ämbetstid kommer du stifta bekantskap med Tillståndsmyndigheten som undersöker om vi driver en respektabel form av utskänkning. Det är sällan som dom dyker upp under vanliga tider om de inte blivit tillkallade av Securitas. Det är nämligen av denna anledning som vi sänker persiennerna under ett gille.

    Vanligast är att de dyker upp när man bett om utökat tillstånd, som då oftast brukar vara under nollningen. Typiskt kollar de efter fyllenivån, matutbudet, att alkoholfria alternativ finns och att det är väl skyltat. Ofta frågar de också om saker som rör brandsäkerhet och utrymning. Det är därför viktigt att hålla koll på brandsläckare, utrymningsvägar och att det inte är något som hindrar dessa. Oftast kommer de ha synpunkter på hur framkomligheten är vid entrén så tänk igenom hur ni placerar stolarna.

    \subsection{Funktionärsvård}
	Ditt utskott är enda anledningen du kan fullfölja dina stora visioner som Krögare och de förtjänar belöning därefter. Kick-offen är traditionellt den stora Kapsyltömmarnatten, där nya KM äter gott, lär känna varandra genom lekar, dricker en och annan öl och avslutar allt med capsa bort hela förra årets skörd av kapsyler. Inom utskottet har man rätt till att hålla en större fest på sektionens bekostnad varje termin så använd den möjligheten till fullo.

    Att ha gemensam sexa efter man lagat julmaten inför julgillet är en nystartad tradition men är ett mycket uppskattad tillskott och fungerar utmärkt som kick-out/fuck off för gamla KM. Speciellt om Krögaren trollar fram ett groggfat.


    \subsection{Nollning}
    Nollningen är en av de roligaste perioderna att sitta i Styrelsen och som Krögare, men det är även den mest hektiska. KM sköter Välkomstgillet, Draggningspuben, maten till UtEDischot och Regattan och det avslutande gillet brukar vara av samma tema som nollningen och man lägga lite extra krut på dekor och temaenligt utbud. Ta tillfället i akt att jobba med andra sexmästerier under nollningen, det är jättekul, lättar på trycket på dina jobbare och visar upp E-sektionens pubar för resten av LTH. Totalt brukar KM hålla i sju stycken evenemang under nollningen.

    I många fall blir nollningens evenemang eldprovet som man över en hel vår inför. Gillen avlöser varandra där det ena är hetsigare än det tidigare. Var inte rädd att spritstoppa, slänga ut folk i förmån för nollor i de gillen som det passar och ta ner Styrelsemedlemmar på jorden. Under nollningen är det också av yttersta vikt att man har koll på entrén. Risken för nollor under 18 finns alltid, men brukar ha meddelats i förväg av Phøset. Viktigast är dock att man kollar noga i väskor och ouvvar då informationen om att absolut ingen utomstående alkohol får tas in har en tendens att glömmas bort. Att kolla Blå dörren under ett nollningsgille är sällan paranoia utan ren självbevarelsedrift.

    Var inte rädd att synas de evenemang Krögartrion kan gå på. Njut av ledigheten och umgås med nollorna och visa upp er i frack och kavaj. Visa upp sektionen från sin bästa, spontana och mest festglada sida så nollorna känner sig välkomna. Mycket är vunnet att minska hierarkin som kan uppstå mellan Styrelsen och nollorna och berätta hur kul det är att jobba inom studentlivet för att så frön till nästa generation KM.

    \subsection{Teknik}
    \subsubsection*{Ljudsystemet Rayleigh}
    Ljudet i Edekvata styrs från hemsidan \url{rayleigh.esek.se}, med lösenordet \textit{teknokrat}. Själva förstärkaren är placerad i Sicrit och måste sättas i gång, annars finns alla reglage till Edekvatas fyra högtalarsystem på hemsidan. AUX-sladden finns i baren och kan användas antingen i egen spelare eller i ljudkortet på Volt, då ljudutgången på moderkortet glappar.

    \subsubsection*{Menyskärmarna}
    Skärmarna ovanför baren är kopplade automatiskt till AHS som visar vilken öl, cider och whisky som plockas ut. Den vänstra skärmen kan man välja själv vad man vill skriva på, förslagsvis om man erbjuder några speciella cocktails eller annat matutbud för det gillet. Skärmarna styrs från hemsidan \url{bar.esek.se}. Inloggningsuppgifter är \textit{admin} och \textit{kro17krok}.

	\subsection{Tröjor}
Sedan 2016 jobbar Källarmästeriet i tjusiga pikétröjor med KMs logga på ryggen. Tröjorna beställs från hemsidan \url{www.alltryck.se} och inloggningsuppgifter erhålles av din företrädare. Schablonen, tryckmallen som används, finns redan i deras system så det är bara be om likadan beställning som tidigare år. Håll koll på att tröjsorter kommer och går och priserna kan därför variera. Viktigt att kolla att fakturaadress och leveransadress är på beställaren, blir jobbigt att behöva låna tidigare Krögares legitimation för att få ut det beställda paketet.

	\subsection{Så Källarmästeriet vill hålla en sittning?}
Som Krögare kommer du under ditt år hålla minst en sittning, nämligen Julgillet i december. Det är nödvändigtvis inte enda chansen utan finns det intresse att hålla en sittning med ett tema som passar KM ska man definitivt ta den möjligheten. Ett välkommet inslag de senaste åren har varit en ölprovning som hålls tillsammans med Cølen. Det är däremot en viss skillnad att hålla en sittning och att hålla en pub så några smådetaljer är bra att ha i åtanke.

\begin{itemize}
	\item I Edekvata finns det sittplatser för 50 personer, möblerat med de vanliga bänkarna och borden. Ute i Vega finns det plats för 90 personer men då måste sittningsborden användas. Dessa borden delas av E- och D-sektionen och det är Källarmästaren (lokalansvarig) på Data som är ansvarig för bokningarna. Maila vederbörande om du önskar boka borden.
    \item Ska man hålla sittning är det utmärkt tillfälle att göra annan mat än hamburgare, då man kan ta lite extra pengar för maten.
    \item Var noga med att poängtera tid för insläpp och när sittningen börjar. Folk brukar ha lite svårt med det.
    \item Kryddor brukar det finnas gott om i Pump. Hör gärna med Sexet om det finns råvaror de vill bli av med.
    \item Maten tar alltid längre tid än man väntar sig. Är tanken att göra efterrätten i ugn tar det mycket längre tid än grundreceptet.
    \item Gör alltid minst tio portioner för mycket. Mängden blir aldrig som man tänker sig och att ha för mycket mat över är ett rätt bekvämt problem.
    \item Vid anmälan är det viktigt att erbjuda även alkoholfria sittningsalternativ.
    \item Fråga Sexmästeriet om något är oklart. De går gärna på en väl genomförd sittning och då är inga frågor konstiga.
\end{itemize}


	\subsection{Enklare recept}
\subsubsection*{Kläggbiffar}
\begin{dashlist}
	\item 2 kg svensk nötfärs (1 paket)
    \item 2 ägg
    \item Ströbröd
    \item Salt och peppar
\end{dashlist}
Använd plasthandskar och blanda färsen med händerna! Lite på dina Krögarsinnen och blanda i en rimlig mängd kryddor och ströbröd. Väg upp 150 g nötfärs, forma dom i kläggpressen och lägg dom omlott i ett bleck. Platta till biffarna lite extra när de läggs på stekbordet så biffarna inte blir för tjocka.

\subsubsection*{Nachos}
Lägg en godtycklig mängd chips i en ring på en stor tallrik med riven ost på (Ingen
har dött av för mycket ost!!). Värm i kökets industrimicro, klicka på knapp 2. När osten smält häll på en godtycklig mängd crème fraiche och salsa.

\section{Städning}
\subsection{Efter gille}
\begin{itemize}
	\item Diska allt
    \item Torka bord och bänkar
    \item Moppa alla golv, sopa innan!
    \item Ställ i ordning bord och bänkar i Edekvata och Vega
    \item Slänga skit (alla soptunnor i Edekvata, glas, soptunnor i baren etc.)
    \item Väga in all dricka (även tomglas: se kapitel om AHS)
    \item Skura golvet i köket (flytta köksön innan)
    \item Töm diskmaskinen
    	\begin{enumerate}
    		\item Lyft bort ställningen till backarna
            \item Tag bort gallren, spola dom och kör i diskmaskinen tillsammans med vaskens avloppsgaller. Glöm inte att tömma det först.
            \item Ta tag i den svarta, runda grejen och lyft, du kommer få med dig en lång pinne.
            \item Låt vattnet rinna ut
            \item Är det skit kvar spola rent med slangen
            \item Sätt tillbaka pinnen över hålet i botten på maskinen samt gallren och ställningen
    	\end{enumerate}
	\item Göra rent stekbord. Sätt igång stekbordet på hög värme, häll vatten på och skrapa bort rester med stekskrapa. Fortsätt tills vattnet ni häller på inte blir skitigt. Håll koll på att ``Dödens låda'' inte svämmar över. När stekbordet har svalnat så tvätta och torka av utsidan med våt trasa och köksmedel.
    \item Torka av skyltar med en blöt trasa
    \item Torka av baren
    \item Skölj av droppkorkar
    \item Stäng av ölkylen och placera en kartong mellan dörrarna för att hålla dom öppna.
    \item Ställ tillbaka glas i KM och Pump
    \item Ta bort matrester från lilla vasken
\end{itemize}

\subsection{Storstädning}

\subsubsection*{Allmänt}
\begin{itemize}
	\item Släng sopor
	\item Väg in alkohol, var noga med tomflaskor
\end{itemize}

\subsubsection*{KM}
\begin{itemize}
	\item Städa och rensa ut kylen
	\item Städa hyllorna
	\item Städa stekbordet
	\item Kolla bakom frysen
	\item Skapa ordning ur kaos
\end{itemize}

\subsubsection*{Vega}
\begin{itemize}
	\item Diska
	\item Ställ tillbaka bord och stolar
	\item Släng skräp
	\item Städa micros
	\item Sopa och moppa
    \item Ta ner all belysning och ställ in i Sicrit
\end{itemize}

\subsubsection*{Edekvata}
\begin{itemize}
	\item Städa och rensa på skit
	\item Moppa
	\item Damma av balkar
	\item Damma lampor och armatur
	\item Gå ut med sopor
	\item Krama Westbjørn
	\item Torka av bord och bänkar
	\item Städa väggar
	\item Damma av högtalare
    \item Torka av baren
\end{itemize}

\subsubsection*{Köket}
\begin{itemize}
	\item Städa lådor
	\item Städa ugnen, glöm ej blecket under
	\item Golv under kylar
	\item Städa bleck med stålull, gammalt bränt ska skrapas bort
	\item Rensa avloppen
	\item Spisen, dra ut spilluckan under spisen och lyft på plattorna och städa inuti
	\item Städa kaklet på väggarna
	\item Städa skåpen, inuti och utanpå
	\item Diska fläktfilter
    \item Ta ner takets galler och tvätta av dom
	\item Tvätta av fläkten grundligt
	\item Hyllorna vid fläkten
	\item Skrubba golv
	\item Tvätta inuti ölkylen
	\item Alla ytor och hyllor ska tvättas av
	\item Städa tallriksskåpet
	\item Kaklet vid diskmaskin
	\item Städa bakom diskmaskinen
	\item Torka inuti silverkylen
	\item Torka av tapptorn
	\item Diska spillfatet under tappen
\end{itemize}

\newpage
\section{Checklista}
Följande är en bra lista att använda för att försäkra sig om att inget av de grundläggande förberedelserna inför varje gille har fallit mellan stolarna.

\subsection*{Håll koll på torsdagen, veckan innan}
\begin{itemize}
    \item Har du mailat PH?
    \item Har det beställts affischer från Picasso?
    \item Har menyn spikats?
    \item Har du kontaktat cøl?
    \item Finns det tillräckligt med förbrukningsvaror? (kläggfickor, tvål, handskar, 		papper)
    \item Finns det jobbare?
    \item Har det bokats Sexalokal?
    \item Har det beställts tappöl? (Vid behov)
\end{itemize}

\subsection*{Håll koll på måndag, samma vecka}
\begin{itemize}
    \item Har affischerna skickats till PH?
	\item Har det skapats ett Facebookevent?
	\item Har det marknadsförts utåt?
	\item Har maten beställts?
\end{itemize}

\subsection*{Håll koll på onsdag, samma vecka}
\begin{itemize}
    \item Finns det växel?
    \item Finns det kläggbiljetter?
	\item Har jobbarna påmints?
\end{itemize}

\section{Gamla traditioner}
\subsection{Klägg}
Ett standardklägg består av följande ingredienser. Ni ska dock veta att ni aldrig ska känna er låsta av detta förslag då trion 2017 mer eller bojkottade detta och körde sitt egna race, till stor succé.
\begin{enumerate}
	\item Brödtopp
	\item Ketchup
	\item Ost
	\item Klägg
	\item Rödlök
	\item Gurka
	\item Tomat
	\item Sallad
	\item BBQ (endast Sweet Bay Ray's) eller majonnäs (endast Hellmann's)
	\item Brödbotten
\end{enumerate}
Enligt gammal sed finns det även något som heter Klägg No mercy, i modern tappning kallad Niklasklägg, efter Niklas Gustafson (E15), men det är upp till sittande Krögartrio att bestämm reglerna. Nyckeln är att detta klägg ska vara så kladdigt som mänskligt möjligt och endast fantasin sätter gränserna.

\subsection{Krögarprivilegier}
Då man som en del av Krögartrion offrat sin beskärda del av fredagar på gillets altare får man som emeritus vissa privilegier. Det viktigaste och mest uppskattade är \textbf{Krögarklägg}, vilket innebär att alla Krögare, nya som gamla, Krögare som vice får gratis klägg för resten av sin studietid. Detta gäller på alla gillen men även på UtEDischot.

Erbjuds det öl på tapp och en Krögare beställer en sejdel så ska ölen hällas upp förbi 1 litersmarkeringen och ändå upp till toppen. Detta kallas för en Krögarhalva, eller Krögarmått. Glöm inte heller att de tre krokarna ovanför speglarna i baren är reserverade för Krögartrions sejdlar. Så se till att skaffa varsin sejdel och pynta dom fint!

\subsection{Glöggille}
På fredagen veckan innan Julgillet hålls alltid Glöggillet! Köp in glögg, sälj det billigt, ha kul. Se också till att köpa in saker att bjuda på, förslagsvis pepparkakor
och skumtomtar. Det är vid Glöggillet som Mer jul börjar spelas (vid 00:00:00:00:00…), alternativt när allt är städat och klart och sexan ska inledas. Prata med Mackapären om att fixa det så att det styrs centralt och ta bort Volt (skärmen) så att ingen kan gå och sänka volymen. Skulle folk ändock trilskas med datorn så krävs mer extrema åtgärder. Använd trälådan i KM, dra sladdar igenom och lägg en telefon inuti som spelar musik och lås med ett hänglås. Mer jul spelas tills dagen för Julgillet. Glöm inte att var 5:e ska vara Amys arr!

\subsection{Julgillet}
Enligt tradition så ska nya trion supas in. Hur ni går tillväga är upp till er men det är alltid nyckelutfrågning. Är gamla Krögare närvarande finns möjligheten att göra utfrågningen under gillet tillsammans med gamla och dryga. Dagen efter är det mycket viktigt nya Krögaren städar toaletterna kliniskt.

En annan tradition är att det tidigare årets Krögare klär ut sig till tomte och håller presentutdelning.


\section{Tips från tidigare års Krögare och Vice Krögare}
\subsection{2017}
\begin{itemize}
	\item För att underlätta felkorrigering när det ska göras Arrangemangsrapport kan det vara taktiskt att börja spara undan tomma spritflaskor och föra lista över deras flaskvikt. Detta kommer underlätta felsökning och korrigering, då den vanligaste orsaken till fel i AHS är att någon glömt väga in eller väga ut tomflaskor.
    \item Under hösten utvecklade Macapärerna ett nytt sätt att registrera att kläggen är färdiga. Snacka med dom så det blir införlivat lagom till första gillet.
    \item Ingen av Vice Krögarna från 2016 hade möjlighet att gå på Julgillet 2017 så ett tecken på goodwill hade varit att låta dom bli inbjudna till Julgillet 2018 istället.
    \item Passa på att använda fatanläggningen till större evenemang, ett enkelt sätt att lätta trycket på Zlatopramen. Kör på kökets egna tapp eller maila pedellen om att få låna kårens mobila tapp.
    \item Håll gärna en förkortad version av A- och B-cert för jobbarna, antingen tidigt under Kapsyltömmarnatten eller som en vanlig genomgång under de första gillen.
    \item Majonnäs och bacon kan liva upp det mesta. Majonnäs är fan livet.

\end{itemize}
\textit{Markus Rahne, Anton Fristedt, Oskar Uggla och Saga Juniwik}

\subsection{2016}
\begin{itemize}
	\item Behöver ni flytta svärdet? Detta ska ske med största diskretion. Bästa tiden att
göra detta är torsdag kväll/natt. Då är det lite folk i skolan.
	\item Skäm bort era jobbare och er själva. Ni är värda det!
    \item Se till att göra minimalt möjliga under gillena. Ni har haft tillräckligt att göra inför alla gillen så se till att leda arbetet istället för att själv jobba.
    \item Jobba med andra sektioner, särskilt med data. De är bra.
    \item Sätt upp följande i KM:
    \begin{dashlist}
    	\item En bild på er alla tre
        \item En text under den bilden med ert stående epitet (exempelvis den snygga
krökartrion '16)
		\item En bild på varje person i trion.
        \item En text under varje person med krögarnamn och citat.

    \end{dashlist}
    \item Uppdatera wikisidan “Krögartrion”.
    \item Skaffa sejdlar och häng upp på väggen. Ni förtjänar att ha quickaccess till era
sejdlar!
	\item Beröm er själva och era jobbare! Det är bra att vara självgod!
    \item KM är ett jättebra rum om man behöver paus från livet. Förslagsvis intas denna
paus på frysboxen tillsammans med nachochips (eventuellt med sprutgrädde)
och godis. Det är inget fel i att äta överblivet jobbargodis. Tvärtom!
	\item SEX OCH SPRIT ÄR KUL! (inom KM får man hästa(sic!))
\end{itemize}
Ha kul och lycka till!

\textit{Malin Lindström, Ester Randahl och Amanda Wallin}

\end{document}
