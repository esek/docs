\documentclass[10pt]{article}
\usepackage[utf8]{inputenc}
\usepackage[swedish]{babel}

\def\post{Förvaltningschef}
\def\date{2017-11-27} %YYYY-MM-DD
\def\docauthor{Sophia Grimmeiss Grahm}

\usepackage{../e-testamente}
\usepackage{../../e-sek}

\begin{document}
\heading{\doctitle}

Hej Förvaltningschef Electus!

Som Förvaltningschef på E-sektionen kommer du ansvara för förvaltningen av Sektionens ekonomi och lokaler. Du kommer vara ansvarig för ett eget utskott och dess funktionärer men även vara med i en styrelse där du kommer kunna påverka hela sektionen.

Detta testamente kommer förhoppningsvis ge dig en idé om vad som väntar och min förhoppning är att du under året ska kunna använda det som ett slags uppslagsverk och kontinuerligt lägga till information så att vi så småningom har en väldigt bra grund för blivande Förvaltningschefer att stå på.

\newpage

\tableofcontents
\newpage


\section{Vad gör en Förvaltningschef?}
\subsection{Generellt}
\subsubsection{Utskottschef}
Som utskottschef för förvaltningsutskottet är det ditt jobb att se till att dina funktionärer utför sina åtaganden, men även att de trivs med sin roll i utskottet. Det är också din uppgift att hantera konflikter som uppstår inom utskottet samt konflikter mellan ditt och andra utskott.

En annan del av jobbet som utskottschef är att göra dina funktionärers röst hörd i styrelsen. Det kan exempelvis handla om större inköp, eller saker som behöver fixas.
Som utskottschef ska du också godkänna alla inköp gjorda av funktionärer till FVU för sektionens räkning samt att de lämnat in kvitto mm. Eftersom det är du som ansvarar för utskottets budget kan det vara bra att uppmana dina funktionärer att alltid kolla med dig innan de köper saker.

Förvaltningsutskottet är ett väldigt spretigt utskott och de olika funktionärsposterna har inte särskilt mycket med varandra att göra. Därför kan Förvaltningschefen behöva anstränga sig lite extra för att få bra sammanhållning. Jag skulle rekommendera att redan innan du går på fundera över hur du vill uppnå detta och hur/om du vill att de olika funktionärerna ska samarbeta. Till exempel så kan en stor del av utskottet engageras i lokalernas utformning och större projekt.

\subsubsection{Styrelsemedlem}
Som styrelsemedlem kommer du tillsammans med resten av styrelsen jobba med sektionens övergripande verksamhet så som ekonomi, styrdokument, sektionsevenemang mm. Styrelsen blir som ett eget utskott där ni troligen kommer har riktigt roligt tillsammans (framförallt kommer ni lära er städa).

Du kommer även delta i kollegiemöten med Pengakollegiet (PK) på kåren med de andra sektionernas motsvarande post. Kollegiemöten är ett bra tillfälle att diskutera med andra kassörer hur de gör saker och kan vara bra stöd.

\subsubsection{Kassör}
Som sektionens kassör kommer det vara din uppgift att se till att sektionens ekonomi förvaltas på korrekt sätt vilket bl.a. innebär att betala fakturor i tid, kontrollera utgående fakturor och se till att alla bank- och handkassatransaktioner redovisas och bokförs korrekt. Du ska även kontinuerligt informera styrelsen om sektionens ekonomi, gärna vid vart styrelsemöte.

Du kommer även ha till uppgift att utföra hel- och halvårsbokslut samt redovisa dem och sektionens allmänna ekonomiska status för sektionen vid både vår- och höstterminsmötena. Det är också din uppgift att se till att styrelsen är införstådda i deras uppgift och ansvar gällande sektionens ekonomi och bokföring.

Du kommer också vara en av sektionens firmatecknare d.v.s. att du tillsammans med Ordförande kommer vara ansvarig för sektionens alkoholtillstånd, avtalstecknande med företag samt hanteringen av sektionens banktjänster. Eftersom sektionens firma teckans av två i förening kommer alla dokument och bankuppdrag behöva signeras (fysiskt eller digitalt) av både dig och ordförande.

\subsubsection{Lokalansvarig}
Som sektionens lokalansvarig kommer det vara din uppgift att se till att sektionens lokaler är i bra och brukligt skick. Du kommer behöva se till att saker som går sönder fixas och att lokalen är i det skick den behöver vara för att sektionen ska kunna utföra sin veksamhet. Du kommer ha kontakt med huset framförallt genom din husstyrelserepresentant och PH (Per-Henrik Rasmussen nere i vaktmesteriet). Det är jätteviktigt att hålla en bra relation med huset framförallt PH och Mats Cedervall (Husprefekten) då de är otroligt hjälpsamma om vi har problem med något.

\subsection{När man går på}
\textbf{Banken.} Så tidigt som möjligt på ert verksamhetsår så ska de nya firmatecknarna gå till banken och fylla i papper så att firmatecknarna byts och så att de nya har teckningsrätt för sektionen, två i förening. Se till att båda har legitimation med sig och att protokollet där nya firmatecknare har valts (underskrivet) är med. Hör med Ordförande om vem som ringer Sparbanken Ideon Gateway och bokar tid.

\textbf{Sektionskort.} Beställ bankkort till de funktionärer som ska ha. Bankkort kostar lite pengar men är bra då det underlättar bokföringen och gör så att folk slipper ligga ute med pengar. Korten är personliga vilket gör att de måste beställas i början av varje år.

\textbf{iZettle.} Kontakta iZettle och skriv att ni vill byta kontoägare på E-sektionens konto. Bifoga justerat VM-protokoll och foto på den nya kontoägarens legitimation. Uppdatera även personaluppgifterna på iZettles hemsida.

\textbf{Access.} Ordförande ska se till att ni har access till studiecentrum där vi deponerar sedlar, kan vara bra att fixa tidigt eftersom LU-kort kan vara sega.

\textbf{Ekonomiutbildning.} Det är bra att ha en ekonomiutbildning med styrelsen så de lär sig sin del av bokföringen. Preliminära internbudgetar ska skrivas för alla utskott och gås igenom med hela styrelsen, se till att alla gör det i början av året. Titta gärna på föregående år och i fortnox.

\textbf{Nycklar. }Samla in nycklar från avgående funktionärer och dela ut nycklar till de nya funktionärerna. Det finns en liten kassa i lilla kassaskåpet där deponiavgiften på 100 kr förvaras.

\textbf{Alkoholutbildning.} Se till att antingen du själv eller ordförande (gärna båda) går C-cert utbildningen som är nödvänding för att vi ska få bedriva verksamhet med fast alkoholtillstånd. Om bara en av er kan gå den rekomenderar jag att den andra går iallafall B-cert.
Se till att krögartrion + sexettrion går A-cert och B-cert (prioritera Krögare/Sexmästare vid få platser, därefter Vice Krögare eftersom det är mer troligt att de kommer hålla i verksamhet själv än att Vice Sexmästare gör det). Även Entertainern bör gå med tanke på Utedischot.

\textbf{Serveringsansvariga. }Se till att våra normala serveringsansvariga är anmälda d.v.s. krögartrion och sexettrion samt att de gamla är borttagna. Det går att se här: https://www.lund.se/foretagare/tillstandregler-och-tillsyn/serveringstillstand-alkohol/restaurangregister/

\subsection{Varje vecka}
\textbf{Betala fakturor.} Vi har två postfack, ett på kåren och ett i vaktmästeriet. Det är egentligen Ordförandes uppgift att ha koll på dessa men det kan vara bra att veta om.
Bokför fakturor. De fakturor vi gör bokförs två gånger: En gång när man skickar iväg fakturan och en gång när man får in betalningen. Vi kommer gå igenom detta tillsammans.
Bokföra inbetalningar. Inbetalningar av våra fakturor registreras under “Fakturering”, “Inbetalningar” i Fortnox. Andra inbetalningar bokförs som vanliga bokföringsordrar.

\textbf{Kontrollera/skriv under verifikat från andra utskott.} Dessa får du i ditt fack och då ska du kontrollera så att allt är rätt, skriva under och lämna till Skattmästaren. Om de inte är rätt så brukar jag skriva en lapp och lägga dem i lämpligt fack. Det kan vara bra att vara lite extra hård med detta i början så att folk lär sig, då blir det mindre jobb för dig resten av året.

\textbf{Skriv ut växel/kontrollräkna KM.} Inför gillen och sittningar där det ska betalas kontant ska du skriva ut en växelkassa. Jag brukar skriva ut ca 1500 kr, och sen lägga i en lapp i påsen så att de ansvariga vet hur mycket som är växel (viktigt!). Detta ska sedan läggas i lilla kassaskåpet och skrivas in i handkassepärmen. Efter eventet läggs pengarna in i nattfack i BD och då ska du kontrollräkna (inom ett dygn), skriva in i handkassan och lägga in det i kassaskåpet.

\subsection{Några gånger om året}
\textbf{Bokslut} (se bokföring och ekonomi).

\textbf{Deponering.} Vi deponerar sedlar i studiecentrum. Ordförande ska maila LU-kort och se till att ni har access dit, men det kan vara bra att dubbelkolla att ni faktiskt får er access också eftersom de kan vara lite sega. Mynt kan vi ibland växla med Ica (se kontaktuppgifter) och om inte annat så får vi lämna in dem på banken, men då tar de en avgift på 5 \%. Då görs det på Swedbank Klostergatan. Man behöver inte boka tid utan går bara dit med pengarna (inte i rör) och så vill de veta vilket konto det ska in på.

\textbf{Golven. }Under sommaren bonar de golven i stora delar av E-huset. Frågar man PH snällt kanske de kan ta delar av Edekvata då också, vilket är nice. Fråga honom gärna tidigt på våren så att vi får en tid.

\textbf{Sektionsmöten.} På sektionemötena ska Förvaltningschefen ge en ekonomisk rapport där man berättar för Sektionen hur det går med ekonomin. Se gamla möteshandlingar för mer info om vad som ska ingå för respektive möte.

\textbf{Representation.} Styrelsen får representation för att gå på sittningar arrangerade inom TLTH (gäller alltså ej när man åker till KTH eller Chalmers). Det lättaste är att göra utläggsräkningar i slutet av året där alla i styrelsen får fylla i vilka sittningar de gått på och hur mycket de kostade. Blir summan (utan alkohol) över 500 får de tillbaka 500.

\section{Bokföring och ekonomi}
Vi kommer gå igenom en del praktiskt vid årsskiftet, så kommer inte skriva så utförligt här.

\subsection{Sektionskorten}
Diskutera med Ordförande och eventuellt övriga styrelsen vilka som behöver bankkort. Korten beställs sedan via internetbanken. 2017 skrev vi kontrakt med alla som fick bankkort, dessa sitter i nyckelpärmen i kassaskåpet. Vid årets slut bör man samla in alla bankkort och klippa dessa samt avvaktivera dem i banken. Man kan ställa in maximalt månadsbelopp i internetbanken, under 2017 har 25 000 varit standard och man kan alltid ändra det i efterhand om det behövs. Under nollningen går det åt mycket pengar så det kan vara en idé att sätta denna högt för vissa redan från början eller ändra inför nollningen så att man inte står där med ett stort inköp och inte kan betala. 2017 har Förvaltningschefen, Ordförande, Sexmästaren, SRE-ordförande, Överphös, Krögaren, Cöl, en Köksmästare samt en Lager- och Inköpsansvarig haft bankkort. För verksamhetsåret 2016 har följande haft kort och det har fungerat bra:

\begin{dashlist}
\item I KM har Krögaren och en Cøl haft ett kort.
\item I Sexet har Sexmästaren och en av Köksmästarna haft ett kort.
\item I Nollu har Øverphøset haft ett.
\item I SRE har SRE-Ordförandet haft ett.
\item I CM har en av Lager- och Inköpscheferna haft ett.
\item Ordförande har haft ett kort.
\end{dashlist}
Så som korten är fördelade över posterna har fungerat och det har inte direkt behövts fler/färre än dem. Självklart är det upp till er att välja vilka som ska ha bankkort.

\subsection{Nycklar}
Sektionen har nycklar till diverse saker, bl.a. kassaskåp, köket, spritförrådet m.m. som funktionärer behöver för att utföra sina uppdrag. Nycklarna kvitteras mot en deposition på 100kr samtidigt som ett nyckelkontrakt upprättas. Nycklar och kontrakt förvaras i kassaskåpet och det är att föredra om endast du hanterar utlämning av nycklar, men du kan även låta Cafémästaren och Ordförande göra det.

\subsection{Access}
Sedan ett par år tillbaka hanteras funktionärernas access till våra utrymmen via en modul på hemsidan. Ordförande och Förvaltningschef brukar gemensamt avgöra vilka som ska ha access vart. Värt att tänka på är att revisorerna brukar ha access överallt och att alla som ska städa behöver ha access till Sikrit för att komma åt städredskapen. Tänk på att vissa av föregående års funktionärer kan behöva access ett tag under nästkommande verksamhetsår (skattmästare, delar av styrelsen). Utöver funktionärernas vanliga access kan du lägga till access manuellt för tillfälliga access. Tips är att rensa bland de manuella accessarna kontinuerligt.

Access till vår bur i EKEA hanterar vi själva men accessen till klockrummet sköts av PH. Detta brukar skötas av Ordförande, men det bestämmer ni själva.

Access till kårens lokaler fixar man genom att fylla i formulär på tlth.se/access. Som Förvaltningschef är det bra att ha access till expen/hänget. Bastun kan också vara najs att ha. Accessen i kårhuset försvinner efter ett halvår så glöm inte att fixa ny innan till exempel nollningen.

\subsection{Kassaskåp}
Översta lådan i kassakåpet är handkassan. Glöm inte att noga skriva upp alla in- och uttag i handkassepärmen. I kasskåpet finns en nyckelknippa med nyklar till Sektionens alla lås. Här inne finns även kopior av nycklar som lämnas ut till funktionärer vid behov. Glöm inte kontrakt och ta deposition vid utlämning.

Under handkasselådan finns några pärmar som är bra att känna till. Bland annat viktiga-papper-pärmen och nyckelkontrakt-pärmen.

I det lilla kassaskåpet förvaras bl.a. Sektionens laptop, iZettle-dosor och iPads. Ändra gärna koden årligen så att bara de som verkligen har behov att använda skåpet har tillgång till det.

I Ullas rum finns det ett litet kassaskåp som Cafémästaren lägger in växel i för kommande dag. Se gärna till att koden till även denna ändras årligen. De som bör ha tillgång till den koden är förutom Cafémästaren, Ordförande och Ulla även de som under året skriver ut växel till LED. Om Cafémästaren är sjuk så hjälp gärna till med att låsa ut växel till caféet.

I nattfack (deponeringsboxen under mikrovågsugnarna i Blå Dörren) läggs växel och kontantintäkter från caféet och gillen. De ska sedan kontrollräknas av Cafémästaren respektive Förvaltningschefen innan pengarna läggs in i kassaskåpet igen. Reservnycklarna som hör till detta skåp tror jag ska finnas i stora kassaskåpet.

\subsection{Alkoholtillstånd}
Du eller ordföranden kommer troligen behöva söka alkoholtillstånd för event som behöver utökat tillstånd eller till och med allmänt tillstånd. Om ni ska söka tillstånd för en dag/lokal som vårt fasta tillstånd inte täcker ska ni fylla i ansökan om tillfälligt tillstånd för slutet sällskap (eller allmänt om det är fallet) annars är det ansökan om tillfälligt utökat tillstånd som gäller.
I samband med att ni mailar in ansökan ska 1400kr i avgift betalas in till tillståndsenheten och transaktionskvittot ska bifogas i mailet. Det ska även sökas speciella tillstånd för dryckesprovning eller för egen kryddning av sprit, vilket Krögaren kan tänkas få för sig att göra inför julgillet. Tillstånd ska sökas senast två veckor innan arrangemanget. Ansökan ska skrivas under av den firmatecknare som har kunskaper inom alkohollagen. Det är även den här personen som ska se till att den serveringsansvariga är en lämplig person.

Serveringsansvarig får endast den person som är minst 20 vara. När tillståndsenheten kommer på besök är det den här personen de kommer att fråga efter. Den serveringsansvariga måste ha tillgång till alla rum i Edekvata om tillstånd vill undersöka dem. Alla jobbare ska veta vem som är serveringsansvarig för kvällen. Serveringsansvariga kan rapporteras in på tillståndsenhetens hemsida. Vårt användarnamn är organisationsnummret och lösenordet är s67njPhb.  Har ni frågor om vad som gäller kan ni fråga mig eller tillståndsenheten.

I början av varje år ska de nya firmatecknarna anmälas som PBI:er (Person med Betydande Infytande) hos tillståndsenheten genom att skicka in en blankett. (Detta är gjort för verksamhetsår 2017). Även serveringsansvariga behöver anmälas till tillståndsenheten och detta görs via deras hemsida. Avgående ordförande brukar få påminnelsemail om detta innan man avgår.

De brukar även kräva att man har en firmatecknare med tillräcklig kunskap inom alkohollagen. För att man ska tillräcklig kunskap inom alkohollagen krävs C-certifikat i ansvarsfull alkoholhantering. Inbjudningar till A- B- och C-cert brukar även de komma till avgående Ordförande innan året är slut. C-cert kan även få skrivas direkt som prov hos tillståndsenheten, det är bara att kontakta dem.

Varje år ska vi göra en restaurangrapport där vi redovisar våra priser, vår omsättning samt hur mycket vi sålt volymmässigt. Priser har Krögaren koll på, omsättningen finns i bokföringen så fråga föregående Förvaltningschef och volymerna får räknas ut. Det här kommer vi göra tillsammans med föregående och nya krögartrion och förhoppningsvis minns jag och Rahne hur man gör.

I slutet av året lär de fråga om hur många sittplatser vi har i Edekvata i samband med att de förnyar vårt permanenta tillstånd. Svara då att vi har 150 stycken. Det är brandreglerna som styr detta, och vi får max vara 150 i lokalen.

\subsection{Sektionsmöte}
Inför sektionsmötena har du som kassör några uppgifter:
\begin{dashlist}
\item Redovisa hur sektionens ekonomi mår.
\item Redovisa saldo för sektionens bankkonton, handkassan och fonder.
\item Redovisa vilka uttag som gjorts ur sektionens fonder sedan föregående möte.
\item På HT ska ett halvårsbokslut redovisas.
\item På VT året efter ditt mandat ska ett årsbokslut samt resultatdisposition redovisas.
\end{dashlist}

\subsection{Bokslut}
Ett bokslut innefattar att man redovisar den summerade versionen av sektionens bokföring under det gångna halv/helåret. d.v.s. resultatrapporter för alla resultatenheter, projekt och för hela sektionen samt en balansrapport på hela sektionen. För årsbokslutet är det högst rekommenderat att du gör en avstämning mellan resultatenheterna och budgeten.

Dock är där några saker som ska vara klara för att rapporterna ska vara “bokslut”. Först och främst måste alla transaktioner under perioden vara bokförda och kontrollerade av revisorerna (revisorerna är inget måste till halvårsbokslut men rekommenderat). Sedan ska alla 1000- och 2000-konton stämmas av och lagerna (CM, E-shop och alkohollagret) och handkassan inventeras. Cafémästaren, Ekiperingsexperterna och Cölen sköter inventeringen av respektive lager. Du bör få inventeringen av E-shops lager och alkohollagret i form av kalkylark. Summorna där ska sedan räknas om till lagervärde med hjälp av inköpspriserna och sedan ska Fortnox uppdateras utifrån resultatet. Vi kommer gå igenom detta noggrannare inför halvårsbokslutet.

\subsection{Viktigt att tänka på}
Som firmatecknare är du ytterst ansvarig för all alkoholverksamhet. Det gör att det kan vara värt att hålla ett vaknade öga på vad sexmästeriet (och KM, men de brukar sköta sig) håller på med. Se till att de följer alkohollagen och att de sköter sig. Speciellt när det gäller prissättningar på alkohol.

Försök få resten av styrelsen att förstå hur viktigt det är att de sköter sin ekonomi. Det kommer underlätta både ditt och deras arbete. Se till att alla utskott gör en internbudget och
att de fastställs på ett styrelsemöte. Hela styrelsen är solidariskt ansvarig för hela sektionens verksamhet. Detta innebär att ENU-Ordföranden är lika ansvarig för NollU:s verksamhet som Øverphøset är.

Se till att folk med kassaskåpsnyckel (Ordförande, Cafemästare och du) sköter handkassan felfritt. Det otroligt svårt att reda ut vad som har hänt i efterhand om man inte skriver upp allt i handkassapärmen och fyller i bokföringsunderlag direkt.

Relationen med huset är otroligt viktig. Se till att du hamnar på god fot med PH och Cedervall. Det kan komma att rädda ditt skinn många gånger.
När ni skickar fakturor till SVL/Programledningen se till att deltagarlistor och kopior på kvitto alltid bifogas med fakturorna.

Gången från entrén till Edekvata fram till nödutgången i Biljard är utrymningsväg och det får därför inte stå saker där. De lokalerna får heller inte användas för förvaring på något sätt eftersom det kan vara en brandfara, vilket kan vara spännande under nollningen då alla uppdragsgrupper måste ha sina byggen någonstans.

Jobba kontinuerligt med bokföringen och få rutiner för vad som ska göras varje vecka. Det är bra att vara någorlunda i fas då det är mycket jobbigare att t.ex. jaga kvitton och sånt flera månader efteråt.

\subsubsection{Braig/Avig}
Inför varje arrangemang i E-huset behövs en brandansvarig och en ansvarig. (Går att ha samma person som båda två.) Dessa ska i god tid innan arrangemanget anmälas till PH, minst en vecka. Var tydlig med att alla som har evenemang (gillen, sittningar osv) i huset ska ha en braig och avig. För att hålla PH på bra humör är det som sagt viktigt att dessa anmäls i god tid. För anmälan måste man ha namn, personnr, sektionstillhörighet och bild på avig/bravig. När man bokar FikFika via hemsidan sköts detta automatiskt.

\subsubsection{Tips och trick}
Försök ställa upp veckorutiner med det dagliga arbetet istället för att ta tag i allt allt eftersom det kommer. Försök ”stänga” en vecka i taget. Det vill säga, bokför alla transaktioner för en vecka och kolla därefter så att saldot på banken stämmer med det i bokföringssystemet. Detta gör det enkelt att hitta eventuella fel i bokföringen.

Ha för vana att tömma nattfack varje vecka. Deponera pengar på studiecentrum så ofta som du tycker är nödvändigt, men se till att inte ha för mycket pengar i kassaskåpet. Det är bättre att de är på banken.

Om du behöver växla mynt så kan man göra det på ICA. Hör av dig till Annika Linander. Hon kan nås på Annika.Linander@kvantum.ica.se

Se till utskotten inventerar alla lager vid halv- och helårsslutet så att du kan göra halv och helårsbokslut. Är du osäker på hur man gör bokslut så hör med dina företrädare. De ska ha koll på hur man gör. Efter ditt år i styrelsen så ska du deklarera. Gör det med din efterträdare så att den lär sig hur man gör. Ta en kopia på deklarationen innan du skickar in den så att din efterträdare lätt kan ta reda på vilka fält som ska fyllas i.

Innan sommaren är det bra att se till att alla bokföringsblanketter är inlämnade och att alla lager är inventerade. Det gör det möjligt för dig att börja med halvårsbokslutet under sommaren och du riskerar inte att behöva vänta med det till hösten för att du saknar verifikat.
Ha en kontinuerlig dialog med revisorerna. Det är alltid bra att vara på god fot med dem. Det är också bra om de uppmanas att revidera kontinuerligt. De har ofta rätt även om man själv inte alltid tycker det.

Sätt dig in i hur fonderna används. Om styrelsen vill göra budgetavsteg så är det viktigt att de görs på korrekt sätt. Är du osäker så hör med din företrädare eller revisorerna.

Har du inte redan gjort det så sätt dig in i stadgar och reglemente samt övriga styrdokument. Dessa styrdokument reglerar vad styrelsen får göra, vilka rättigheter och skyldigheter man har.
Ha en bra dialog med Lager- och inköpschefer om deras IC-rapporter. Se till att de görs kontinuerligt. Kontrollera dem extra noga innan du är säker på att de har koll på dem.

Se även till att E-shop gör sina försäljningsrapporter kontinuerligt, jag skulle rekommendera minst varje period under våren då det inte är så mycket aktivitet och varje vecka under nollningen. Det underlättar nämligen väldigt mycket sen när man ska stämma av iZettlekontot (1981) eftersom en rapport då inte påverkar så många veckor.

\subsubsection{Nollning}
Nollning innebär mycket iZettle, ha gärna en genomgång med de som ska använda det så att de vet att de ska fylla i försäljningsrapporter.
Det är många som inte vet att de ska fylla i försäljningsrapporter. Det kan vara bra att gå igenom detta noga med bl.a. ekiperingsexperterna och så småningom NollU inför hösten.

\subsection{Bra kontakter och nummer}
NOKAS har hand om deponeringen och är de vi kontaktar om vi t.ex. behöver nya deponeringspåsar. \\Vårt kundnummer hos dem är 354003228.

LU-kort, vi brukar ha kontakt med dem för att felanmäla dörrar och liknande. Det är dem ni vänder er till för access till studiecentrum (Men det ska Ordföranden fixa).
\\Mail: lukortet@lu.se

Ica Tuna, vi brukar växla mynt med dem.
Vår kontakt på ICA är Annika Linander.
\\Mail: Annika.Linander@kvantum.ica.se

Tillståndsenheten, brukar vi ha kontakt med om vi har frågor angående alkohollagen eller tillstånd. Det är även de vi skickar in ansökningar om tillfälliga tillstånd till. På deras hemsida finns kontaktuppgifter till alla deras handläggare och enhetschefen.
\\Webb: https://www.lund.se/foretagare/tillstand-regler-och-tillsyn/serveringstillstand-alkohol/
Generell mail: tillstandsenheten@lund.se

PH, Per-Henrik är E-husets Husintendent och den personen som är lättas att vända sig till
med frågor eller ploblem gällande vår lokal eller E-huset generellt.
\\Han sitter borta vid tryckeriet och nås antingen där eller via mail: PH@ehuset.se

Mats Cedervall, Mats är E-husets Prefekt och den man kontaktar gällande bokningar av foajén eller utökade problem/problem som inte PH kan hjälpa till med. Tips är att gå via PH först eftersom det brukar vara den smidigaste vägen.
\\Mail: mats.cedervall@eit.lth.se

Akademiska Hus, vi brukar kontakta dem direkt gällande felanmälningar som gäller el, vatten eller liknande. Om du är osäker om det ska felamälas till dem eller inte kan man kolla med PH.
\\Webb: http://www.akademiskahus.se/vara-kunskapsmiljoer/forvaltning/felanmalan/

Fortnox, vår kontaktperson är Hanna Gothberg och hon kan hjälpa dig med avtalsrelaterade frågor. Vid vanliga supportfrågor är det lättast att kontakta den vanliga supporten.
\\Mail: hanna.gothberg@fortnox.se

Förvaltningschef Emeritus
Har du frågor kan du alltid vända dig till mig =)
\\Mail: sophia.g.grahm@gmail.com
\\Tel: 0730-728481
\vfill

Slutligen vill jag önska dig lycka till. Njut av ditt år som Förvaltningschef, det kommer vara ett krävande år men samtidigt ett riktigt roligt sådant!


\end{document}
