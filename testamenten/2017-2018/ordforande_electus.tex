\documentclass[10pt]{article}
\usepackage[utf8]{inputenc}
\usepackage[swedish]{babel}

\def\post{Ordförande Electus}
\def\date{2017-11-20} %YYYY-MM-DD
\def\docauthor{Erik Månsson}

\usepackage{../e-testamente}
\usepackage{../../e-sek}

\begin{document}
\heading{\doctitle}

Hej Ordförande Electus!

Grattis till ditt nya uppdrag som Ordförande för den bästa av sektionerna inom TLTH! Jag önskar dig stort lycka till under ditt kommande år och hoppas att du kommer känna att det är ett riktigt roligt och givande uppdrag!

Du som nyss blivit vald till Ordförande på E-sektionen har lite framför dig i jobbet som Ordförande. En del av dessa behöver göras redan innan du går på! Allt nedan behöver inte göras innan du går på och det kommer i lite halvologisk ordning så läs igenom allt. \texttt{:)}

\begin{numplist}
\item Glöm inte att det är nya styrelsens ansvar att städa efter Valmötet!

\item Först och främst så gäller det att spika ett datum för KPL (Kurs På Landet, eller i folkmun Kristna PulkaLägret) som är ett gemensamt styrelseskiphte för nya och gamla styret. Datumet för KPL kan vara lite lurigt att få till då det vanligtvis brukar vara i en stuga som man hyr från fredag eftermiddag till söndag eftermiddag. Ett förslag på datum är den 19-21:e januari. Tänk på att det är viktigast att så många i den nya styrelsen kan komma för det är verkligen teambuilding! Det brukar vara hemligt för den gamla styrelsen vart KPL ska äga rum, de brukar följa en skattkarta eller dylikt som Entertainern har gjort och komma dit på lördagen.

\item Även Skiphtet behöver ett datum vilket också är bra att bestämma så tidigt som möjligt, framför allt om man vill vara på Lophtet då det snabbt kan bli uppbokat. I år hade vi Skiphtet den 17:e februari vilket var ganska sent. Försök lägga det tidigare om det går! Hur ni vill lägga upp Skiphtet är naturligtvis helt upp till er men tänk på att meddela alla nya och gamla funktionärer så tidigt som möjligt så att de kan boka upp datumet.

\item Det skall tas ett (juligt) julkort som ska skickas till en massa människor. Så boka en tid när alla kan, kom fram till en rolig idé, och genomför den. Julkortet ska skickas ut så mottagarna får det innan julledigheten börjar. Listan med alla som fick förra året finns i årets styrelsedrive någonstans. Kolla också gärna med Josefine vilka företag Sektionen har haft kontakt med under året.

\item Styrelsen vill ha frackar. Tagga styrelsemedlemmarna på detta och se till att de skaffar sig dem innan jul! Detta är bra för då kan man skicka dem till tryckeriet/brodyr innan jul och på så vis få dem innan skiphtet. Förra året lämnade vi frackarna till Brodyrteam (\texttt{brodyrteam@swipnet.se}). I år handlade vi dem på Ateljé Helene i Rydebäck. Ring henne några dagar innan ni tänkt åka dit så hon kan hämta fler frackar.

\item Bestäm vad ni vill ha för profilkläder och beställ detta så snart som möjligt så att de kan nyttjas maximalt.

\item Du samt din styrelse är mer än välkomna på alla resterande styrelsemöte i år. Särskilt ska ni medverka på terminens sista styrelsemöte där massa spännande händer! Datumet är tordagen den 14:e december 12.10 i E:1124. Meddela nya styrelsen! Be Kontaktor Electus att göra en ny maillista för styrelsen 2018 och att lägga till den i kallelselistan så ni får kallelserna.

\item Gör en ny Google Drive, Slackkanal, Facebookchatt, m.m.. Diskutera tidigt i styrelsen vad som gäller för driven och de olika kommunikationskanalerna!

\item Prata med nya styrelsen om val av e.a.-poster så att det kan ske så snabbt som möjligt!

\item Kåren kommer att ha kollegieskiphten helgen den 2-4:e februari. Förmodligen har OK skiphte på lördagen, missa inte det!

\item Jag har tyvärr inte fått datum för alkohol- och styrelseutbildning än. Skickar vidare all information om detta till dig så fort jag får den. Minst en av (men gärna båda) firmatecknarna behöver C-cert, så se till att planera in det så fort ni får informationen.

\item Tisdagen den 12:e december 18:00 ska du följa med mig på OK-möte!

\item Prata med din nya styrelse om ni är intresserade av överlämningsmiddag i veckan innan jul.

\item Det är även din uppgift att införskaffa en julgran till Sektionen innan julgillet. Att betala för den är överskattat, men man gör ju som man vill.

\item Sist men inte minst - lär dig Taggig Blomma innan Julgillet och glöm inte din sångbok hemma!

\end{numplist}

Du kommer att få ett rejält testamente snart men det här ska vara det allra mesta som du behöver tänka på nu innan julledigheten. Jag håller dig naturligtvis uppdaterad om jag kommer på något mer. Någon dag ska vi även ta och presentera dig för lite folk i huset. Mina nycklar får du i slutet av året så du kan kan komma in i kassaskåpet och så. Du och Förvaltningschefen kommer även att behöva gå till banken efter nyår men det pratar vi om tids nog.

Lycka till! 2018 kommer bli ett fantastiskt år!

\emph{Erik Månsson \newline Ordförande 2017}

\end{document}
