\documentclass[10pt]{article}
\usepackage[utf8]{inputenc}
\usepackage[swedish]{babel}

\def\post{ENU-ordförande}
\def\date{2017-11-28} %YYYY-MM-DD
\def\docauthor{Josefine Sandström}

\usepackage{../e-testamente}
\usepackage{../../e-sek}

\begin{document}
\heading{\doctitle}

Hej Isabella!

Stort grattis till posten som ENU-ordförande eller Näringslivsordförande som kanske låter snyggare när man pratar med företag! Du har ett riktigt roligt och lärorikt år framför dig men också många utmaningar och jobb. Jag tror dock du kommer sköta allt galant men om du skulle undra något så är det bara att höra av dig! I detta testamente hittar du lite tips och info om vad ENU har gjort 2017. Det är ganska långt (förlåt) men läs i alla fall igenom kvarlevorna som bl.a. innehåller avtal jag har gjort men som går över på ditt år. Nu är det hur som helst din tur att göra 2018 ännu bättre! Och om du skulle hitta något kul att hänga i teknologmössan så vill jag också ha det ;)

\subsection*{TL;DR}
Det viktigaste jag har lärt mig under året som ENU-ordförande är: Det blir vad du gör det till! Du och
ENU har alla möjligheter att välja nya vägar, eller följa redan utstampade. Du kommer märka att året
går fort och att det gäller att redan från början komma igång med arbetet. Det kommer vara du som
ordförande som är den drivande personen i utskottet. Det är viktigt att du orkar hela året, för posten
kräver att du kan visa upp E-sektionen från en bra sida även mot slutet av din mandatperiod när
energin börjar tryta. Låt dina funktionärer jobba åt dig, de kommer tycka det är roligt med ansvar och
känna glädje att de är med och bidrar till gruppen. Som utskottsordförande och styrelseledamot är
det mycket annat arbete som inte syns men som likväl tar tid och kraft, så våga delegera arbete.

\subsection*{Kvarlevor från 2017}
\begin{itemize}
\item Något som jag ångrar väldigt mycket under året är att jag inte fått upp företagshemsidan. Det finns en färdig sida som DDG har gjort som du bara behöver läsa igenom (och ev. lägga till någon text). Sedan pratar du med Oscar Uggla, Anders Nilsson eller Axel Voss som kan lägga upp den. För att se hemsidan går du in på %___________________________

\item Ericsson har inte fått sin faktura från Lunch med ingenjör. När man gör event med Ericsson vill de att man ska skicka en offert till dem där pris och evenemang ingår. Efter det skickar Ericsson en form av ''beställning'' och därefter kan fakturan skickas. Kontaktperson: Anders Henriksson, \textit{anders.henriksson@ericsson.com}

\item På våren hade jag tillsammans med Axel Sondh ett möte med Emil från Svep. Då planerade vi dels Svep RC som genomfördes under nollningen och en branschkväll som var en kvarleva från 2016. Vi gjorde också en ny deal vilket var att Svep skulle hålla i en till branschkväll/lunchföreläsning på våren 2018. Som det har blivit nu har alltså ett paket innehållande Svep RC och en branschkväll blivit uppdelat på två år (Svep RC på hösten och branschkväll på nästa års vår). Paketpriset för detta har varit 15000 kr (sedan långt tillbaka) och vi får betalt efter branschkvällen. Detta innebär att vi INTE har fått betalt för Svep RC för detta års nollning. De 15000 kr kommer ses som en intäkt för ENU 18. (Info om branschkvällen finns under evenemang) Kontaktperson: Emil Ohlson, \textit{emo@svep.se}

I framtiden känns detta inte så hållbart. Jag skulle rekommendera dig att göra om priset så att Svep betalar efter Svep RC och om de vill hålla i en branschkväll så får de betala för den separat så att vi inte ligger ute med pengar så länge.

\item ARM har innestående marknadsföring som de inte nyttjat ännu. Det är från ett paket där monterevent också ingick (montereventet har de redan haft) och de har betalat för paketet (8000 kr om jag inte minns fel). Det som ARM har innestående hos E-sektionen är: 1st utskick i sociala kanaler, 4 veckor annonsering anslagstavlor, 4
veckor annonsering TV. Mitt tips är att maila inom kort och påminna att de har detta att utnyttja och
att det t.ex. utgår om 6 månader. På så sätt får de tid att bestämma vad de vill marknadsföra samt att
ni kanske kan göra liknande paket igen mot årets slut. Kontaktperson: Cem Eliyurekli, \textit{cem.eliyurekli@arm.com}

\item Cybercom kommer sponsra LED med pappersmuggar och ev. förkläden. Muggarna kommer levereras i januari. Muggarna kommer marknadsföra sig själva så Cybercom får inget ''extra'' för dem men om de sponsrar med förkläden med så har jag erbjudit ett inlägg på Facebook, annons via TV-skärmar eller affischering. Kontaktperson: Omid Asali, \textit{Omid.Asali@cybercom.com}

\item Tetrapak vill ha ett hackathon under en heldag. De kommer stå för själva uppgiften och priser men vi fixar själva evenemanget. Bjud in D- och F-sektionen så det blir fler folk. Jag tror inte du behöver boka gasque för jag tror inte att det blir så många människor. Ett alternativ är iDét (som är lite större än Edekvata). Eftersom det sker en lördag eller söndag, försök att planera in eventet så det inte krockar med andra större evenemang. Det finns en liten planering i driven med saker att tänka på och vad för beslut man måste ta. Jag kommer också skicka några mejl till dig som jag bl.a. fått av Open Hack som innehåller lite info om hur de gör sina case. Ett alternativ är också att låta de hålla i caset och dela upp vinsten mellan sig på något snyggt sätt. Kontaktperson: Erik Nordahl, \textit{Erik.Nordahl@tetrapak.com}

\item Vi har börjat planera inför FED pub som kommer vara som ett alumnigille ungefär. All info om det finns i en drive som delats till dig. Det kan vara bra att involvera E6 eller KM i planerna så de vet att de ''ska'' hjälpa till. Det är de som kommer stå i baren.

\item Adsensus vill hålla en utbildning för E-sektionen mot att de får marknadsföra sig via våra kanaler. Den dealen vi har kommit fram till är följande:

''Adsensus håller i utbildning under en lunch där E-sektionen står för mat/dryck och lokalhyra. För detta får Adsensus publicera ett inlägg på vår facebooksida samt ANTINGEN annonsera via affischer under en vecka ELLER annonsera via våra TV-skärmar i en vecka, värde 2250 kr.''

Kontaktperson: Elin Barnholdt, \textit{elin.barnholdt@adsensus.se}
\end{itemize}

\subsection*{Teknikfokus}
Tänk på att Teknikfokus är den 14 februari. Teknikfokus är en ekonomiskt MYCKET viktig del för sektionen. Den avgör i princip om sektionen kommer gå plus eller minus i slutet av året. Mässan anordnas tillsammans med D som får större procentandel av ev. vinst pga att de sköter ekonomin och tar den ekonomiska smällen vid förlust. Det har också pratats om att dela upp vinsten m.a.p. andel funktionärer på E och D och om det skulle hända är det viktigt att E också bidrar med funktionärer.

Det har även varit svårt från vår sida (men också på D) att hitta en projektansvarig för Teknikfokus. Valet sker på vårterminsmötet och mitt tips är att göra det så synligt som möjligt så folk vet om att man kan söka posten. En idé kan t.ex. vara att gå ut och prata i pausen på föreläsningar.

\subsection*{Budget och ENUs verksamhet}
Något jag vill be dig fundera över och kanske diskutera med din vice är ENUs roll och syfte för E-
sektionen och E-sektionens medlemmar. Är det att knyta kontakter mellan studenter och företag? Förbereda sektionens medlemmar inför arbetslivet? Vara med och förmedla jobb? Dra in pengar till
sektionen?

Som det kommer bli ditt år ska ENU gå 205 000 kr plus vid årsskiftet. 130 000 kr är från Teknikfokus och 75 000 kr för intäkter från företag, det finns alltså två budgetposter i ENUs budget. Summan från Teknikfokus höjdes i år för att mer spegla verkligheten men den är ändå satt lägre än vad ni antagligen kommer dra in, en säkerhetsåtgärd skulle man kunna säga. Vi vet ju inte om Teknikfokus kommer fortsätta gå så bra som det har gjort tidigare.

Vad gäller sektionen är E-sektionen beroende av att ENU når sina budgetmål. Det gör att sektionen
t.ex. kan investera i renoveringar och ha en pengbuffert om det skulle gå dåligt för sektionen ekonomiskt. ENU har möjligheten att dra in pengar och det kan vara mycket smart att utnyttja
företags vilja att betala för att synas och på sikt värva ingenjörer från E och BME.

\subsection*{Ph\o set och ENU inför och under nollningen}
Det är en gråzon vad gäller sponsring till nollningen. Det som gäller formellt är: sker det en inbetalning från företag för att synas för studenter, ska dessa pengar gå till ENU (alltså läggas på ENUs budget). Detta beslut är taget för några år sedan för att undvika konkurrens mellan Ph\o set och ENU om att skaffa sponsorer. Argumenten lyder ungefär att samarbeten med företag ska kunna ske året om och inte enbart under nollning. För så kallad ”sakspons”, dvs sponsring i form av regattautklädnad, färgburkar, priser till tävlingar etc. är det fritt fram för phöset att skaffa. Det är du som i praktiken bestämmer hur du vill göra med sponsring mellan phös och ENU. Är det så att de hemskt gärna vill samla pengar till nollningen så går det att lösa via sektionen t.ex. genom att få in pengar till ENU och sedan göra en internfakturering till NollU. Tänk dock på att Nollu har en egen budget och de bör klara sig på den.

\subsection*{Priser}
Priser för olika events är ett ämne som diskuterades flitigt på NK-mötena (NäringslivsKollegiets
möten, vi körde ca en gång i läsperioden). Det skiljer sig mycket i priser från sektion till sektion och beror till viss del på efterfrågan (t.ex. D har relativt lätt att skaffa spons och kan ha höga priser).

Som jag har gjort i år är att jag har utgått från en prismall, men justerat priset för varje event (t.ex.
kan det vara bättre att ett litet företag ändå kommer men inte betalar lika mycket). När jag ser
tillbaka på detta ifrågasätter jag om det var rätt väg att gå. Det finns en risk att priserna sprids bland företagen, eller att priserna hoppar kraftigt mellan åren p.g.a. nya ENU-ordföranden. Samtidigt leder det också till att fler företag har råd att genomföra ett event. Det är något du får ta ställning till.

Det är alltid bra att kolla igenom priserna i början av året och se över om något behöver ändras. De priserna jag använde under 2017 är

\begin{center}
  \begin{tabular}{ | l | l | l |}
    \hline
    \textbf{Event} & \textbf{Pris} & \textbf{Ingår} \\ \hline
    Lunchföreläsning & 10 000  & Lokal, marknadsföring (mat exkl.) \\ \hline
    Branschkväll & 10 000 & Lokal, marknadsföring (mat exkl.) \\ \hline
    Monterevent & 3000 & Marknadsföring\\ \hline
    Marknadsföring TV-skärm & 750 kr/vecka & \\ \hline
    Marknadsföring Affisch & 750 kr/vecka & \\ \hline
    Marknadsföring FB/hemsida & 1500 & \\ \hline
  \end{tabular}
\end{center}

Min företrädare hade även priset 12 000kr för lunchföreläsningar under nollningen. Eftersom det ofta kommer väldigt många studenter på lunchföreläsningarna under nollningen kan man marknadsföra med det och därmed sätta det högre priset. Jag valde dock att sänka det till 10 000kr för att hålla samma priser under hela året.

\subsection*{Vad tar vi betalt för?}
Ibland finns det gråzoner om vilken marknadsföring vi tar betalt för. Vi tar i princip alltid betalt för företag och t.ex. Academic Work som annonserar åt ett företag. Ideella föreningar tar vi inte betalt för och de samarbetena sköter Kontaktorn. Ibland har jag dock blivit tillfrågad av start-ups om de t.ex. kan få sätta upp affischer. Då har jag ofta gett en rejäl rabatt, det har varierat lite men ca 80-90\% billigare, start-up är start-up. Det kan vara bra att sätta en fast siffra där som ni alltid följer. Jag har dock blivit tillfrågad av några som arbetade med LU innovation, alltså de arbetar med LU på något sätt, och då har jag inte tagit betalt. Det är lite från fall till fall. Vi prioriterar dock betalande företag så om det är fullt på anslagstavlorna så får de inte sätta upp sina posters.

\subsection*{Marknadsföring}
När du marknadsför event kan det vara najs att skriva en liten sammanfattande text om vad företaget sysslar med. Alla har inte energin att kolla upp sådant själv. Detta kan vara särskilt viktigt på event som förväntas locka färre folk, tex kvällsföredrag.

För att locka yngre studenter räcker det ofta att locka med mat eller andra gratis grejer. För att locka äldre studenter kan det dock vara bättre att locka med eventuella tjänster, t.ex. om företaget har sommarjobb, letar exjobbare eller deltidsjobb.
Fråga företaget om det är något särskilt de vill marknadsföra.

Marknadsför i tid och inte under andra evenemang (extra viktigt under nollning)
Man får spamma lite angående marknadsföring, folk kan vara dåliga på att ta tag i att anmäla sig.

\subsection*{BME-företag}
...är det få av som har velat synas för E-sektionens medlemmar. Det beror delvis på att BME-
företagen delvis är mindre och inte har råd att marknadsföra sig på skolor, men också för att det är svårt att knyta an nya företag som återkommer. Viktigt att tänka på är att de första kullarna BME har tagit examen, vilket betyder att möjligheterna har aldrig varit större att få BME-företag att nå ut till sektionens medlemmar. Mitt tips är att utnyttja att alumniansvariga ligger under ENU.

\subsection*{Representationströjor}
Sektionen bestämde förra året att lägga en viss summa pengar på representationströjor till de olika utskotten som går i arv år efter år. Vi i ENU köpte in 9 pikéer (2st S, 4 st M, 3 st L) från Swedish Profile Wear. Modellen är DryBlend Piké Gildan, produktkod 8800, om ni vill köpa fler. Tröjorna ligger i lådan i HK.

\subsection*{Året som gick}
Nedan har du alla evenemang vi anordnade under året. Om något är oklart är det bara att höra av sig!

Tidigt i februari hölls en \textit{CV-fotografering} för E-sektionens medlemmar. Det var lite över 30 personer som kom och vi fotade under lunchen mellan 11-13, eller det var planen i alla fall. I och med att varken jag eller fotografen då kunde så mycket om kamerautrustning tog det lite längre tid att rigga upp, dessutom var kameran oladdad. Vi lånade sektionens kamera och kårens blixtar. Kåren sägs också ha en fotoduk men den har Tetrapaks logga över sig och det syns på baksidan med. Låna inte den. Det kostar några hundralappar att hyra kårens kamerautrustning.


Detta året körde vi igång med \textit{Lunch med en ingenjör} igen och nu (till skillnad från året innan) tog vi betalt 500 kr per ingenjör samt extra om företag ville att deras logga skulle synas på affischer och liknande. I våras höll vi en tävling för att locka folk till att anmäla sig vilket funkade rätt bra. Vi bjöd samtidigt på kaffe men det är nog bättre att också bjuda på kakor. Först stod vi i foajén och gjorde reklam, andra gången utanför Edekvata. Problemet med att bjuda på saker är att folk från andra sektioner vill ha, där får man själv avgöra om det som bjuds endast är för E studenter eller inte.

Intresset på sektionen för Lunch med Ingenjör har varit rätt svalt dock. Det kan vara för att vi har haft samma företag på besök så det kanske kan vara värt att testa med nya. Rebecca har skickat ut utvärderingar till alla som har deltagit så läs dem och tänk över om det är värt att fortsätta med konceptet. I driven under fliken ''Lunch med ingenjör'' finns ett testamente för eventet.

Alexander från \textit{Bearingpoint} kontaktade mig angående en lunchföreläsning som de ville hålla. De betalade lite mindre än vad företag brukar då det är ett BME-företag som generellt har mindre pengar (dock verkar de ha rätt mycket pengar). MoP stod för maten och lite mer än 70 personer anmälde sig. Pris: 6000 kr + 50 kr/student

\textit{Svep} hade ett kvällsföredrag kvar från året innan mitt verksamhetsår. Eventet hölls tillsammans med F och D och vi tog av sittningsmaten från Flickor på Teknis. Ett stort problem med denna kvällen var att de var få anmälda. Tyvärr krockade den med ett event på D så det kom bara 2 st därifrån. Svep hade även hållit detta kvällsföredraget året innan och kanske till och med året innan det så intresset var rätt svalt hos E med. Pris: 17000 kr (varav priset för Robotic challenge inkluderas samt 2000 kr för marknadsföring på F). Under Svep RC stod ENU och grillade hamburgare.

\textit{Ericsson} höll en casekväll med pub efteråt. Kvällen började med en presentation av företaget i E:C. Därefter höll de i caset i Edekvata och därefter var det pub. Det var vi i ENU som arbetade under puben men det kan vara trevligt om man får KM eller E6 att jobba (och t.ex. att de får alla övriga inkomster från försäljningen). Detta så att vi kan se till att företaget har det bra och ta hand om dem. Pris: 10 000 kr + 50 kr/student.

Ericsson höll också en lunchföreläsning under nollningen. Pris: 10 000 kr + 50 kr/student.

\textit{Axis} sponsrade med microvågsugnar förra året och i våras kom de och ''brand:ade'' dem samt bjöd på mikromat. De sponsrade också med nollningströjorna, 140 st á 100 kr/st.

\textit{Tetrapak} höll en lunchföreläsning under nollningen. Pris: 10 000 kr + 50kr/student. De levererade också en pall vatten till nollningen.

\textit{ARM} stod för tryck på ouverallerna. Pris: 5000 kr. De stog också i foajén och skrev
på ett sponsringsavtal (se ''Kvarlevor från 2017''), vilket de betalade 8000 för.

\textit{Academic Work} stod i foajén och höll en CV granskning. Pris: 3000 kr. De har också marknadsfört mycket på vår facebooksida. Academic Work är inte jätteglada i att betala fullpris. Jag hade fått för mig att kåren tog ett lägre pris för dem så därför gjorde jag en egen prislista till dem som du ser nedan (alla priser ändrades inte, de som ändrades syns nedan). Kåren tar dock INTE ett lägre pris så det kan vara värt att se över om du vill fortsätta med det lägre priset eller inte. Tänk bara på att om andra företag får reda på detta måste du ha en bra anledning till att de inte betalar fullpris.

Lunchföreläsning: 6000 kr + 50kr/student\\
After Work/kvällsevent: 6000 kr + 50kr/student\\
Marknadsföring Facebook/hemsida: 900 kr\\

Det var många företag som hörde av sig och undrade hur de kunde nå ut till våra medlemmar (t.ex.
via mail/sociala medier). Jag skrev en hel del reklam på vår hemsida och facebooksida och
fakturerade för en hel del pengar. Mail erbjuder vi dock inte (i alla fall inte i år) då vi på E kände att vi inte vill spamma våra medlemmar med företagsmejl.

\subsection*{Tips och tillvägagångssätt vid och inför event}

Inför monterevent vill husprefekt Mats Cedervall (\textit{husprefekt@ehuset.lth.se}) kunna ge sitt
godkännande SENAST en vecka innan. Jag missade att meddela honom någon gång och då skickade
jag ett snabbt mail dagen innan och bad om ursäkt och det var inga problem då, men det är inte att
rekommendera att göra varje gång. Företagen kan köpa kaffe från LED och själv gå och fylla på om
det tar slut. Be de som jobbar i LED ha koll på hur många kaffebomber som företaget går och tar. Ibland har det blivit att företag sagt en siffra men LED en annan. Ibland byts företagsrepresentanten ut också och då är det svårt för företaget att hålla koll.
Lägg på det på fakturan därefter.

Lunchföreläsningar (och alla andra event som sker i en föreläsningssal eller liknande) måste bokas via
lokalbokningen. Jag gick alltid upp till 5:e våningen och fixade det där men man kan mejla också. Att boka E:B under en timma kostar ca 1000 kr. OBS! Du måste själv kontakta vaktmästeriet för att få access på ditt kort inför eventet. Det gör du enklast genom att maila PH på \textit{ph@ehuset.lth.se} ,när du har mailat och fått klartecken på din bokning.

Affischer trycks enkelt och billigt upp i E-husets tryckeri. Maila PH (ph@ehuset.lth.se) och bifoga pdf. Kontaktor vet vilka affischeringsytor som vi har till förfogande.

TV-affischer lägger du själv upp (eller ber kontaktorn utvidga någon i ditt utskotts access) via din inloggning på esek.se.

Det var inget företag (förutom Adsensus) som bad om separat skrivna avtal som skulle skrivas under under mitt år. Ett tydligt mail där det framgår vad som ingår i det överenskomna samarbetet OCH ett bekräftande från företaget bör räcka långt i form av avtal. Jag var dessvärre rätt dålig på det (fick dock inga klagomål) men det är alltid bra med någon form av avtal.

Det är viktigt att det kommer folk på evenemangen som anordnas med företag för att företaget ska
bli nöjt och vilja genomföra det igen. Studenter är glömska och ibland lata, så det är bra att påminna
vilket man enkelt kan göra genom att be om mailadress vid anmälningsformuläret och skicka ut en
påminnelse t.ex. dagen innan. Det kan också vara bra att skriva att om någon fått förhinder behöver
de se till att någon annan kan ta deras plats.

Flera företag mejlade mig och frågade vad vi erbjöd. Det kan vara smidigt att göra ett dokument som du kan skicka till företag som innehåller allt vi erbjuder och priserna på detta.

\subsection*{Kontaktpersoner}
I princip alla kontaktpersoner som vi har varit i kontakt med under året ligger i driven i dokumentet ''Företagskontakter''.\\
\\
\\
\\
\\

Stort lycka till nästa år!

\emph{Josefine Sandström \newline ENU-ordförande 2017}

\end{document}
