\documentclass[10pt]{article}
\usepackage[utf8]{inputenc}
\usepackage[swedish]{babel}

\def\post{Cafémästare}
\def\date{2017-11-25} %YYYY-MM-DD
\def\docauthor{Daniel Bakic}

\usepackage{../e-testamente}
\usepackage{../../e-sek}

\begin{document}
\heading{\doctitle}

Först och främst vill jag gratulera dig till ditt nya uppdrag som Cafémästare! Jag önskar dig all lycka under det kommande året och hoppas att du kommer tycka att uppdraget är givande och roligt. Det kommer bli ett superbra år!

Cafémästare är en av de tyngsta posterna på sektionen men därmed även en av de mest givande posterna. Det är väldigt lärorikt på många olika sätt och det är framförallt roligt. Innan du hoppar på vid årsskiftet kommer du få göra några av mina arbetsuppgifter med mig så du lär dig hur det fungerar. Sen ska detta testamente förhoppningsvis täcka allt du behöver veta, men var inte rädd att fråga mig om det är några som helst oklarheter!

Daniel Bakic\\
\emph{Cafémästare 2017}

\newpage
\tableofcontents

\newpage
\section{Postbeskrivning}
Vad innebär det då att vara Cafémästare? I E-sektionens reglemente är posten ganska skevt beskrivet. Det står i princip att man är ansvarig för sektionens caféverksamhet. Du får gärna gå in och läsa vad som står i reglementet men jag kommer här nedan beskriva posten så som jag upplevt den under året.

\subsection{Cafémästare}
Som Cafémästare är du huvudansvarig för E-sektionens caféverksamhet, dvs LED. Det är du som är ansiktet utåt och det är dig folk kommer kontakta angående verksamheten. Det är du som är ekonomiskt ansvarig och måste därför ha koll på kostnader, intäkter samt redovisa dessa. Som Cafémästare har du främst två roller, dels som utskottschef men också som styrelsemedlem. Båda dessa roller har olika arbetsområden, är mycket givande och är väldigt kul.

\subsection{Utskottschef}
Det jag syftar på med den här punkten är att din nya funktionärspost är en representativ post. Om någon vill Cafémästeriet något så är det till dig de kommer att vända sig. Detta innebär en hel del mail, en del telefonsamtal och en del folk som letar efter dig. Att vara utskottschef innebär att du måste hålla kontakten med andra, både sektionellt och intersektionellt. Prata med dina funktionärer, prata med de andra i styrelsen, prata med de andra Cafémästarna och prata med andra utanför sektionen som du kommer att ha kontakt med (se mer under "Kontakter"). Se till att ha god kommunikation med dessa och ta hjälp av varandra så gott det går. Som utskottschef är du ansvarig för ett utskott och du tar därmed på dig rollen som ledare över detta. Som ledare ser man till att alla i utskottet sköter sitt jobb och framförallt att de trivs med sina poster inom utskottet. Därför är det viktigt att försöka bygga upp en sådan stämning där alla känner sig bekväma att dela med sig av sina tankar.

\subsection{Styrelsemedlem}
Det här är bland det roligaste som du kommer göra som Cafémästare. Som styrelsemedlem kommer du en gång i veckan gå på styrelsemöte där sektionens verksamhet diskuteras. Där har du chansen att ta del av vad de andra utskotten gör samt ta upp dina egna tankar kring sektionens verksamhet. Glöm inte att välja in Dioder och Halvledare på mötena, de sitter bara på posten läsperioden ut och måste väljas in igen varje läsperiod. Inför varje möte ska du skriva en utskottsrapport där du redovisar vad som händer i utskottet den veckan. Sen inför VT och HT kommer alla styrelsemedlemar skriva större utskottsrapporter (då kan det vara najs att kika på de veckoliga rapporterna) samt verksamhetsplansuppföljningar. Inför dessa möten kommer ni även svara på motioner och skriva propsitioner tillsammans, vilket är kul då man kan påverka "på riktigt" då.

Givetvis är det inte bara massa jobb, man har det riktigt kul under tiden också. Förutom att man representerar sektionen tillkommer några najs förmåner.
Förmånerna är allt från texten på styrelsefracken till fester enbart för styrelsen. Du kommer även att få 500 kr för att representera ditt utskott med. Dessa pengar får du använda för att gå på sittningar, fester med mera. Det viktiga är att det är högst 250 kr per gång och det får inte användas till alkohol.

\subsection{Firmatecknare}
E-sektionen har två firmatecknare: Ordförande och Förvaltningschefen. Detta innebär att om det är något avtal som ska godkännas å sektionens vägnar är det firmatecknarna som skriver på. Värt att notera är att avtalen ska skrivas på av två i klump (två måste godkänna). Detta gäller även för internetbanken som du kommer få tillgång till. Om en av firmatecknarna lägger upp en betalning/överföring på internetbanken så måste den kontrasigneras, det vill säga godkännas av en annan firmatecknare. Som firmatecknare måste du vara 20. Eftersom Cafémästare inte längre är en firmatecknare så ska alla avtal skrivas under av Ordförande och Förvaltningschef, men se till att noga sätta in dig i eventuella avtal som gäller caféet.

\section{Utskottet}

Cafémästeriet har varit ett förhållandevis litet utskott som har växt sedan verksamheten flyttade upp från Edekvata till LED-café och som växer mer och mer. Se till att delegera så mycket du kan till dina funktionärer, annars kommer det bli för mycket att göra. 2018 kommer antalet jobbare som du kommer att ha tillgång till att vara:

\subsection{Lager- och inköpschefer (upp till 4)}
Efter dig är detta antagligen den viktigaste posten i utskottet. Inköpscheferna ser till att det alltid finns varor till caféet. Om de inte sköter sitt jobb kan ingen annan sköta sitt. I början av året kan det vara viktigt att påminna om beställningarna, annars finns det en risk för att det inte kommer några varor och då måste dessa inhandlas manuellt, det vill säga åka till Axfood i Malmö med bil och inhandla. Glöm inte heller att påminna dem att stänga av den stående beställningen (brödet från Mormors bageri) under lov och tentaperioder. Inköpscheferna sköter även internfakturering, beställning av bröd, med mera. En av inköpscheferna får ett sektionskort. Det är \underline{väldigt viktigt} att de fyller i bankkortsförstärkning när de har använt kortet. Se till att dem ligger i fas med sina IC-rapporter.

\subsection{Vice Cafémästare (1-2)}
En av de lite flummigare och odefinierade posterna i utskottet. Posten är till för att avlasta dig som Cafémästare, men exakt vad de ska göra är upp till dig. Det bästa är att tidigt få en rutin på fasta uppgifter de ska göra, t.ex att sköta de regelbundna kylkontrollerna. Det viktigaste är att delegera uppgifter till dem (uppgifter som du inte hinner med/tycker är tråkiga, tidskrävande). Exempel på sådana uppgifter kan vara rengörning av kaffebomberna (bör göras 1 gång per läsperiod) och avkalkning av sandfiltret/avkalkningsfiltret, det kan även innebära upplärning av Dioder/Halvledare eller att kasta soporna när det glömts bort att göras. Posten har tidigare inte haft ett tydligt testamente om vad som ska göras så prata mycket med dina vice och kom fram till vad dessa ska göra under året. Ta gärna hjälp av dina vice vid planering av tackfest också.

\subsection{Halvledare (5)}
Halvledarna ansvarar för vars en dag i veckan och ser då till att verksamheten sköts som den ska den dagen. Till detta ingår exempelvis rekrytering av Dioder, upplärning av Dioder, hjälpa till vid öppning och stängning samt fixa ersättare ifall ordinarie Diod inte kan närvara på sitt pass. Detta är en helt ny post för 2018 så vet inte riktigt hur det fungerar som bäst, se till att prata mycket med dessa och se till att göra det tydligt för dem vad dem ska göra. Deras mandatperiod är en läsperiod och de väljs in på styrelsemötena.

\subsection{Dioder (E.A)}
Dioderna jobbar minst 2 timmar i LED en dag i veckan och får för detta en gratis macka/sallad enligt vad som anses passande. Jag hade, eller försökte ha, att dioderna tog mat så många dagar i veckan som de jobbade i timmar, så de som jobbade 2 timmar fick mat två dagar i veckan. Jag kan rekommendera dig att försöka ha två dioder under hela förmiddagen (8-13), sen kan det vara att se över om det kan komma eftermiddagsdioder som håller LED öppet från 13 till en godtycklig stängningstid.

\section{Rutiner}
Här kommer jag ta upp vad jag under 2017 hade för rutiner. Det här är helt enkelt en egen preferens och det är helt upp till dig at komma fram till vad som passar dig bäst.

\subsection{Varje dag}
Jag har haft för vana att varje dag komma in i caféet och kolla så att läget är under kontroll, både för min egen skull men främst för Ullas skull. Ulla kommer inte riktigt vara relevant för dig längre men det är fortfarande bra att komma förbi och synas lite i utskottet och se så att allt fungerar. Under året har jag dagligen fått skriva ut växel samt räknat kassan, men det slipper du då vi slutat ta emot kontanter! Något som bör göras dagligen är att kolla dina mejl och andra informationskanaler så du inte missar något viktigt!

\subsection{Varje vecka}
Jag har som vana att ha ett lunchmöte varje vecka med de centrala delarna av utskottet (inköp och vice). Det har fungerat mycket bra och fört utskottet närmare. Här kan du ta upp vad som händer i veckan, om några i utskottet har något samt dela med dig av dina tankar. Du kan även bjuda in Halvledare och Dioder till dessa mötena, kan bli lite fullt av människor, men det är givande då fler röster i utskottet kan få sin röst hörd.

En gång i veckan är det även styrelsemöte, vilket är väldigt viktigt för att få reda på information om hela sektionen. Glöm inte skriva utskottsrapport inför dessa mötena, några korta ord om vad som händer i utskottet.

En gång per vecka ska en veckorapport göras, hur det ska gå till finner du under ``Pappersarbete''. Det går ganska snabbt då vi inte längre behöver oroa oss över kontanthantering.

\subsection{Efter läsperiod 4 och 2 (varje halvår)}
Efter läsperiod 4 och 2 ser du till att allt i utskottet är färdigt. Du ser till att alla veckoredovisningar är gjorda, kassaboxen i läsklagret räknad, D-sektionen blivit fakturerade på deras matbiljetter och så vidare. Men du ska också räkna kaffestreck och göra en internfakturering till sektionen. Även arbetsglädjen som är utskriven (till alla utskott) ska internfaktureras. Lagret ska inventeras, det som ska inventeras är pant och läsk och skicka det till revisorerna. Se till att lager- och inköpscheferna har gjort sina IC-rapporter så att de inte för allt för mycket på hög, det underlättar både för dem själva och förvaltningschefen.

\section{Viktiga händelser}
Cafémästeriet är ett väldigt strukturerat utskott med kontinuerligt arbete, men ett par gånger om året finns det vissa händelser som avviker från standarduppgifterna.

\subsection{Miljöförvaltningen}
Miljöförvaltningen kommer en gång per år till caféet och ser att vi sköter oss enligt alla förordningar. Det är Sarah Strandh (se "Kontakter") som är ansvarig för det här och om du har frågor så bör du kontakta henne (det går även att boka tid när de ska göra besöket om du skulle vilja det). Det bästa sättet att slippa vara orolig för det här besöket är att se till att självkontrollen sköts (den kan du hitta i en pärm i LED eller i din CM-mapp i filsystemet).

\subsection{Caféfesten}
Caféfesten brukar vara på våren och är tillsammans med D, F, I, M, V och K (detta året hade vi det uppdelat mellan F, E, D, K samt V, M pga brist på kommunikation). Ta upp det här tidigt på cafékollegiemötena och försök fördela arbetsuppgifterna. Det viktigaste som ska göras är: vilken sektion som ska stå för alkohol och tillstånd,vilken som ska fixa maten, vilken lokal man ska vara i (om det skulle vara i E-huset ska du prata med Mats Cedervall, men helst vill du inte ha det här). Inför 2018 har E-sektionens tackbudget ökat och LEDs vinstkrav minskat i förhållande till kostnaderna. Detta innebär att du kanske kommer ha större möjlighet att bidra med pengar till Caféfesten än vad jag hade (30kr/person som alla utskott fick till kickoff/kickout), prata med Förvaltningschefen och se hur ni ska göra. Alla gamla Cafémästare bjuds traditionsenligt in till denna fest, du når alla via e-postadressen \texttt{cm-em@esek.se}. Festen ska eventuellt redovisas som en arrangemangsredovisning, men kolla med Förvaltningschefen för säkerhetens skull.

\subsection{Lunch till nollor}
Första dagen under nollningen gör vi lunch till nollorna. Det brukar vara ca 200-250 stycken och det brukar vara bäst att ta hjälp av styrelsen då många i ditt utskott är phaddrar. Fråga phöset vad de vill ha, hur många som ska göras, budgeten och vart det ska faktureras någonstans.

\subsection{Expot}
Expot är en förmiddag/lunch innan valmötet då styrelsen bjuder på mat och godis och berättar om sina utskott. Detta är ett utmärkt tillfälle att hitta de sista funktionärerna som du saknar för nästkommande året. Se även till att uppdatera utskottsbeskrivningen ifall det skulle behövas.

\subsection{Julgillet}
Du är domare till pepparkakstävlingen under julgillet. Detta är en av de viktigaste sakerna du gör som Cafémästare. Se till att så många som möjligt deltar, se till att dem bygger ätbara hus då det är du som får smaka dem och se till att det finns riktigt najs priser till vinnarna. Prata med Krögaren om hur ni ska lägga upp det.

\subsection{Speciella dagar}
Under året kommer det finnas specifika dagar så som kanelbullens dag där vi brukar sälja extra mycket av det dagen erbjuder. Det är en kul grej och vi kan dra in ganska mycket pengar på det.

\subsection{Promota [insert]}
Under året har jag fått förfrågan ifall vi kan sälja kaffe till något typ av företag, utskott, förening eller liknande. Det brukar oftast vara i samband med att dem vill promota ett event eller bara synas mer och då står de oftast i E-foajén. Jag har försökt ställa upp så ofta jag kan då det är kul att hjälpa till plus att vi får 5L kaffe sålt på en gång. Detta gör du hur du vill, se bara till att dem som är i caféet just då är med på det och att det inte går ut över vår verksamhet.

\section{Pappersarbete}
Härefter kommer de vanligaste papprena som du kommer stöta på och även kortfattat hur dessa fylls i. Om du skulle vara osäker på något och dessa exempel inte hjälper, kontakta Förvaltningschefen för att se till att det blir rätt.

\subsection{Veckoredovisning}
Det här kommer du antagligen få göra ca 30 gånger under året (kom igen, det blir kul!). Jag går igenom stegvis vad du ska göra vid varje tillfälle:
\begin{numplist}
\item Logga in på datorn, duplicera Veckoredovisning och döp den till något passande (så att du vet exakt vilken vecka som det gäller) och öppna den.
\item Fyll i dokumentet enligt kvittona och skriv ut (lägg ett lila papper i skrivaren innan du skriver ut).
\item Signera rapporten och lägg den i Förvaltningschefens fack.
\item FÄRDIGT!
\end{numplist}

\subsection{Kassaboxredovisning}
Pengarna i kassaboxen i läskförrådet borde räknas in en gång per läsperiod. Så här gör du:
\begin{numplist}
\item Töm kassaboxen på pengar.
\item Logga in på datorn och gör en kopia av Kassabox och döp den till något vettigt.
\item Räkan in pengarna från kassaboxen och fyll i värdena i kassaboxredovisningsdokumentet. Spara, skriv ut och lägg pappret i Förvaltningschefens fack.
\item Skriv in pengarna i handkassepärmen.
\item FÄRDIGT!
\end{numplist}

\subsection{Caféfestredovisning}
Till denna ska det skrivas en arrangemangsrapport. Ta hjälp av Förvaltningschefen så att det blir rätt.

\subsection{Fakturering}
Det viktiga här är att du skriver ut tre exemplar. Ett skickar du iväg till mottagaren och de två andra lägger du i Förvaltningschefens fack. Nuförtiden ha majoriteten av alla kunder en email-adress man kan skicka fakturan till istället för att behöva skriva ut och skicka fysiskt. Samma sak gäller här, en till mottagaren, en till Förvaltningschefen.

\subsection{Kallelse}
Inför styrelsemöte får du en kallelse. I denna står föredragningslistan och när och var mötet kommer att ske. Om du har någon punkt du vill ha med på listan så mailar du Ordförande eller lägger till den på mötet medan ni är på punkt 7 (om det är mindre viktigt är det bara att säga det under "Övrigt" på mötet).

\subsection{Internfakturering}
När andra utskott köper t.ex. kaffe eller sallader av caféet så ska du fakturera dem för inköpspriset på varorna, inte det vi säljer dem för. Jag ger med en bilaga på en mall jag använder för att se inköpspriserna, försök vara noga att kolla så att priserna stämmer överens med verkligheten. (OBS! Inte vid beställning av varor till andra, det går på IC-rapporter).

\subsection{Övrigt}
Det finns en del andra papper, så som handkasseförstärkning och så vidare. Dessa är dock inte allt för krångliga moment och om du har problem är det lättare att bara fråga.

\textbf{Ta aldrig pengar ur kassan i LED för att handla, om någon i utskottet handlat något så gör en riktig förstärkning från HK eller internet. Se i största möjliga mån till att köp  görs via bankkortet som en av inköpscheferna får.}

\section{Kontakter}
När du går på är det viktigt att du ändrar kontaktinformation hos Servera, Crowd och Delicato (kolla med inköp om nummer dit). Du kan antingen maila eller ringa. Detta för att gamla Cafémästare inte ska få samtalen och informationen ska försvinna på vägen.

\subsection*{Avkalkningsfiltert}
Villiam installerade avkalkningsfiltret till kaffemaskinen. Ring honom om det är något problem med filtret. \\
Villiams tel: 0767-22 12 82

\subsection*{Kaffemaskinen}
Senast kaffemaskinen gick sönder så anlitade vi REPRO, de är dyra men de är snabba och effektiva. Kontakta dem om kaffemaskinen går sönder men testa gärna resetknappen och starta om den.\\
Tel: 040-29 55 65, hemsida: \url{http://www.repro.se}

\subsection*{Studiecentrum}
Måns är ansvarig för Finn Ut i Studiecentrum. Han brukar ringa om han har problem med växel eller brist på kaffefilter. Om du har problem med muggar, kaffefilter eller annat så brukar Måns kunna hjälpa till.\\
Måns tel: 0735-36 77 04

\subsection*{Miljöförvaltningen}
Sarah är miljöinspektorn som kommer en gång per år och ser till att vi håller caféet rent och att vi följer reglerna.\\
Sarah Strandhs tel: 040-35 52 69

\subsection*{Delicato}
Från Delicato beställer vi alla delicatokakor.\\
Tel: 0730-52 25 94

\subsection*{Martin \& Servera}
Våra kontakter på Servera. Om du till exempel vill boka en extra leveranstid eller har något klagomål angående leveranserna, avtal etc.

\subsubsection*{Niclas Ellberg, Innesäljare}
Om det är något som gäller beställningar så ring Niclas, tips är att låta inköp få hans uppgifter så de kan ringa direkt om det är varor som saknas eller är defekta.\\
Tel: {040-28 78 56} eller \url{niclas.ellberg@martinservera.se}

\subsubsection*{Ulf Franzon, Distriktansvarig säljare, Skåne}
Vid konstigheter/frågetecken kring vårt avtal och/eller kundnummer hos dem, så är det bara att kontakta honom i första hand. Likaså när det kommer till frågor som rör leveranserna eller våra fakturor.\\
Mobil tel: {0708-18 30 84}, mail \url{ulf.franzon@martinservera.se}

\subsubsection*{Övrig info Martin \& Servera}
Inloggningsuppgifter e-handel: Användarnamn: {519043} Lösen: {hacke5284}

\subsection*{Crowd}
Vi får sponsrade muggar av Crowd då och då. De ringer upp ibland och informerar om detta. Vi har skrivit kontrakt med dem på två år i sommras. Om du vill ha mer information om kontraktet ring dem. De är lite jobbiga att få tag i, men det går, senast jag hade kontakt med dem var det med en Pontus.\\
Pontus Göths tel: {0707-496 802}, mail: \url{pontus@studentmuggar.se}\
\\\url{www.studentmuggar.se}

\subsection*{Övriga}
\subsubsection*{Mormors bageri}
Från Mormors bageri köper vi allt bröd, det är dina inköpschefer som har kontakt med dem. Se till att de stoppar beställningarna vid tentaveckor och lov samt ringer dem om det är något som inte stämmer vid leveransen.\\
Tel: {046-32 87 87}, Kundnummer: {512}

\subsubsection*{Diskbolaget (diskmedel och torkmedel)}
Vi beställer maskindiskmedel och -torkmedel från dem. Inköpscheferna ska sköta detta.\\
Tel: {040-40 59 80}

\subsubsection*{Axfood/Snabbgross}
Från Axfood beställs det framförallt läsk, men det går även att beställa annat. Se till att inköp beställer in läsk när det behövs, gärna i större mängder, så vi inte behöver få för många olika beställningar. Inköp gör även IC-rapporterna för allt som handlas på faktura hos snabbgross.\\
Tel: {040-68 06 380}\\
Inloggningsuppgifter: Användarnamn: {9119659} Lösen: {381035}

\textbf{PS:} Det är inte fullt garanterat att alla dessa uppgifter stämmer exakt men alla som är viktiga att ha kontakt med är åtminstone uppskrivna.

\section{Övrigt \& tips}
Här kommer jag lista några tips and tricks som kan vara bra att kunna. Det kommer en blandning av mina egna tips och äldre Cafémästares tips.

\subsection{Leverantörerna}
Under årets gång har jag haft god kontakt med leverantörerna och jag rekommenderar starkt att prata med dem både via mail och telefon, de är väldigt behjälpliga om det är problem med varor eller om priser eller avtal vill ändras. Våga komma med kritik om priser på varor eller dylikt, det går oftast att förhandla det så länge man har konkreta motpriser hos konkurrenterna. Jag rekommenderar också att lära inköpscheferna att ringa leverantören direkt om varorna saknas eller är dåliga vid leveranser, ta bild om möjligt för då är det lättare att styrka felet.

\subsection{Kontakter i huset}
En person som är mycket bra att ha kontakt med är PH. Han sitter nere i tryckeriet och är mycket hjälpsam när det kommer till att trycka posters av olika slag, kaffekort och ifall det är några problem i lokalen (exempelvis om det är stopp i vasken).
Det kan även vara bra att känna till Mats Cedervall som är husets prefekt, han har koll på allt gällande huset och sitter på tredje våningen i E-huset.

\subsection{Styrelsen}
Jag rekommenderar starkt att jobba nära styrelsen och att ni bör ta hjälp av varandra, det hjälper i många situationer och ger bättre sammanhållning. Detta gäller framförallt under nollningen då alla utskott har mycket att göra, se till att ni delar på arbetsbelastningen och hjälper varandra med relevanta saker. Försök även att hålla de andra utskotten i öronen när det gäller läsklagret, se till att panten hanteras snyggt, läsk skrivs ut korrekt, E6 håller sina kylar i skick och SRE har sina saker på sin plats.

\subsection{Cafémästarkollegiet}
Samma gäller för Cafémästarkollegiet, det är ett bra forum att bolla idéer samt prata om problem och verksamheten. Andra sektioners verksamhet kan skilja sig mycket mot vår, men det kan göra att vi har lösningar på andras problem och de på våra. Det är också en direkt kontakt till kåren, vilket är bra när större saker ska bearbetas.

\subsection{Egenkontroller och inköpskostnader}
När det kommer till egenkontroller så bör dessa revideras årligen för att se till att vi följer lagarna och att den reflekterar vår verksamhet. Se till att dina vice har bra koll på detta och sköter det.
Se till att ha koll på och uppdatera inköpskostnaderna kontinuerligt också, det är extra skönt att ha koll på dessa då allt som säljs till andra utskott säljs för inköpskostnaderna.

\subsection{Verksamhetsplan}
Sektionen har en verksamhetsplan som ska komplettera budgeten. Verksamhetsplanen innehåller mål som tidigare styrelser har satt för kommande år för att få ett långsiktigt tänkande i sektionen. Verksamhetsplanen är ingen lag, men precis som budgeten bör avvikelser mot den motiveras/redovisas. Verksamhetsplanen ska kontinuerligt redovisas i samband med sektionsmöten, samt uppdateras inför höstterminsmötet. Bifogar en bilaga på den.

\subsection{Ekonomi}
Under året bör det göras kvartalsbokslut, detta för att kontrollera att LEDs ekonomi går bra, så det gäller att få inköps- och lagercheferna att ligga i fas med IC-rapporterna. Det är värt att tänka över "belöningssystemet" för att få bättre koll på ekonomin. Speciellt nu när det tillförs nya poster och mer tackfester.

\subsection{REPRO}
Kontakta REPRO om att eventuellt skapa avtal med dem inför det nya året, det kostar ingenting, men du blir bunden i 12 månader och vi får på så vis 10\% rabatt på deras tjänster, så vi får rabatt om det är något problem med kaffemaskinen. REPRO är verifierade av Crem International (tillverkaren av vår kaffemaskin, det är en Coffee Queen Single Tower), så inte bara säljer REPRO samtliga reservdelar utan de är duktiga och snabba på att reagera om man behöver få ut en reparatör.

\subsection{Kontaktinfo}
Vissa av kontakterna nedan är otroligt dåliga på att uppdatera sina kontaktinfo. Under mitt år har de ringt inköp som satt 2015. Försök se till att dem bättrar på det så att du inte undgår viktig information. Kan även vara ganska skönt för gamla inköp att inte bli väckta på morgonen av att en leverans har kommit.

\subsection{Sponsring}
Tidigare har vi blivit sponsrade av Crowd/studentmuggar och då fått hem gratismuggar av dem. Inför nästa år kommer vi åtminstone få 5000st muggar av Cybercom i januari, dem ska eventuellt sponsra med förkläden också. Detta är ett helt nytt avtal som du tillsammans med ENU-ordförande får se över.

\section{Hälsningar}
Tveka inte att kontakta en gammal Emeritus om du har några frågor eller funderingar!

Till sist: Låt inte posten bli en börda! Det är en tung post men den har sina fördelar. Glöm inte att ha roligt under året, gå på phest, ha najs häng med utskottet och styrelsen och njut av din tid som Cafémästare. Var inte rädd att utnyttja din frihet som chef över caféet, har du några kul idéer du vill genomföra gör det. Sätt din egen prägel på caféet och dess verksamhet!\\
- \textit{Daniel Bakic, Cafémästare 2017}

\end{document}
