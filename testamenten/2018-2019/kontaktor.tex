\documentclass[10pt]{article}
\usepackage[utf8]{inputenc}
\usepackage[swedish]{babel}

\def\post{Kontaktor}
\def\date{2018-12-30} %YYYY-MM-DD
\def\docauthor{Axel Voss}

\usepackage{../e-testamente}
\usepackage{../../e-sek}

\begin{document}
\heading{\doctitle}

Hej Kontaktor Electus!

Grattis till ditt nya uppdrag, jag önskar dig stort lycka till under kommande år och hoppas att du kommer känna att det ett riktigt roligt och givande uppdrag.

Uppdraget som Kontaktor är inte helt solklart alltid. Man har ingen direkt huvuduppgift utan håller på med många olika ``mindre'' saker. Så som jag ser på det så har Kontaktorn följande uppgifter, utan någon vidare ordning:
\begin{dashlist}
    \item Att vara en god styrelseledamot - d.v.s. att engagera sig i styrelsearbete och komma med värdefulla idéer och tankar.
    \item Att agera Sekreterare för styrelse- och Sektionsmöte, vilket inte bara innebär att skriva protokollen utan också att tillsammans med Ordföranden upprätta handlingar till våra Sektionsmöte. Det innebär också att vara den som uppdaterar alla styrdokument när förändringar sker i dessa och att se till att alla funktionärer har sina poster i systemet med allt som hör därtill.
    \item Att ansvara för informationsspridningen på Sektionen, vilket ofta innebär att ta hand om frågor och önskemål angående informationskanalerna från medlemmar och funktionärer såväl som utomstående Sektionen och Kåren.
    \item Att agera utskottsordförande för Informationsutskottet och att representera Sektionen i Infokollegiet (IK) på Kåren.
    \item Att ha ett övergripande tekniskt ansvar på Sektionen, vilket ofta innebär ett delegerande av arbete till DDG med Teknokrater.
    \item Att ansvara för kontakten med våra vänsektioner och se till att de får komma och besöka vår Nolleqasque (vill man utöka med fler events är det givetvis välkommet, men tänk på att det bidrar med mycket jobb för många personer)
    \item Sist men inte minst - ha kul under nollningen och ta för dig av alla events och umgås med nollorna.
\end{dashlist}

Till sist vill jag säga att medan jag har haft superkul och utvecklats massor under året har det blivit för mycket ibland, och då måste man känna att det är okej att prioritera bort Sektionen ett tag. Studierna måste alltid få lov att vara första prioritet, det är ju trots allt därför vi är här!

\begin{itshape}
Vänliga hälsningar\\
Axel Voss, Kontaktor 2018\\
\end{itshape}
\newpage

\tableofcontents
\newpage

\section{Sektionen}
Som styrelsemedlem och mer specifikt Kontaktor har man stor möjlighet att påverka och utvecka Sektionen - om man vill. Många Kontaktorer har lagt en del tid på att försöka förbättra våra styrdokument, vilka är i ett ständigt behov av att förbättras.

Som kontaktor har möjlighet att förändra och påverka många andra saker också, försök komma igång med det du vill genomföra under vårterminen. Under hösten är det mycket annat som händer och jag känner att det var mycket lättare att hitta tid för de egna projekten under våren. Försök att ha en genomtänkt struktur och plan med det du inför och dokumentera så mycket du kan! Det underlättar otroligt mycket när nästa kontaktor tar över och ser till att det du införde faktiskt kommer till användning.   

Tycker man något är fel/dåligt i våra styrdokument ska man komma ihåg att medan de kanske är jobbiga att ändra så är de inte skrivna i sten. Anser du att något inte fungerar eller inte speglar verksamheten som den ser ut idag så var inte rädd för att förändra. Se dock till att tänka långsiktigt och att göra förändringar som inte bara fungerar för stunden utan också i framtiden när du kanske inte ens själv är kvar i Sektionen.

Se till att ha någorlunda koll på vad som står i våra styrdokument. Skumma igenom dem och se till att ha läst bitarna som handlar om din egen post och utskott.

Förutom att uppdatera styrdokument har man som styrelsemedlem rätt mycket frihet i att hjälpa till där man vill, även utanför utskottet. Under mitt (Erik Månsson) år hjälpte jag t.ex. Ph\o set med att göra deras introduktionsfilm och Kåren att göra deras taggfilm till Regattan.

\section{Styrelsen}
På styrelsemöten så kommer du ibland ha fullt upp med protokollet men försök ändå att hänga med och yttra din åsikt i frågor, du är också en medlem i styrelsen. Med det sagt så är det viktigt att styrelsen fungerar som grupp och inte splittras i flera läger. Ordförande har alltid sista ordet och det får man respektera. Tänk på det att ni diskuterar på styrelsemötena kan vara känsligt och i så fall inte skall nämnas utanför mötet. Alla protokoll är offentliga handlingar som alla sektionsmedlemmar m.fl. har tillgång till så är det viktigt att man tänker efter vad man skriver ner. Vissa punkter bör inte dokumenteras.

När det kommer till uppdatering av styrdokument bör du alltid vara den som ser till att saker blir bra skrivna och vara petnoga. Folk är i allmänhet ganska slarviga när de skriver och om ingen ser till att t.ex. reglementesändringarna blir bra gjorda kommer reglementet bara bli sämre och sämre...

\newpage
\section{Informationsutskottet}
Att vara utskottschef för Informationsutskottet är inte helt självklart då det består av väldigt olika poster som alla gör sin egen grej. Att DDG (Sektionens DatorDriftsGrupp) dessutom fungerar som ett eget litet utskott gör att InfU blir ännu brokigare. Tidigare har det sagts att på grund av detta så är det viktigt att försöka samla hela utskottet till 3-4 möten under årets gång, men det är inte något jag hunnit med under 2016 och hoppas på att du tillsammans med din Vice kan göra detta igen. Det behöver inte vara långa möten utan det viktiga är man uppmärksammar alla funktionärer, ger alla en överblick av utskottets aktivitet och framför allt låter dem träffa varandra.

Förutom utskottsmöten så är det alltid bra att belöna funktionärer med ``funktionärsglädje''. Man kan lösa ut läsk, godis eller kakor från lagret. Om det är något större projekt så kan man köpa in lite råvaror och fixa mat åt dem. Funktionärsglädjen är inlagd i budgeten för att uppmuntra funktionärsarbete - så se till att verklige använda den! Fråga Förvaltningschefen om du känner dig osäker med hur man finansierar det.

Du hittar information om InfU:s alla funktionärer i reglementet och på hemsidan. Vissa av dem har sina egna testamente och det finns inget direkt uppstartsarbete att göra med dem. Se till att träffa alla i början av året och se till att de har kommit igång med sitt arbete - särskilt Macapärerna med DDG. Ha god kontakt med Macapärerna så ni snabbt kan reda ut tekniska problem tillsammans när något strular.

Här skulle man ju kunna skriva om e.a.-valen också, men oftast är det väl så att Ordförande Electus tar hand om det och berättar hur det ska gå till långt innan Kontaktor Electus har fått detta testamente, så det känns onödigt att ha med.

Eftersom du som Kontaktor inte håller i evenemang så arbetar man inte med sitt utskott. Däremot som jag (Henrik Fryklund) började med 2014 är att jobba mycket i projekt. Man lägger upp en tidsplan och engagerar sig i varje projekt och pushar alla att följa den. Ett tips är att ha möten/workshops efter skoltid hemma hos någon med mat och \o l/cider/alkoholfritt för att göra det trevligare och mysigare. Försök att få med så många som möjligt i utskottet om möjligt!

Du som Kontaktor behöver inte röra något tekniskt utan det är helt upp till DDG. Däremot så är du deras främsta språkrör in till styrelsen så ha bra kontakt med dem via Macapärerna. Delta gärna på något av deras veckomöten så att du får en uppfattning om hur saker funkar. Men om du inte tänker dra i det tekniska så är det verkligen inte nödvändigt att vara där regelbundet. Macapärerna är ständigt adjungerade till styrelsemötena så tveka inte att dra med dem om styrelsen ska diskutera något viktigt som rör dem t.ex. byta hemsida, datorer eller liknande. De fungerar i sådana fall som rådgivare då de i 99 fall av 100 är mer insatta i det tekniska och kan avgöra om lösningar och förslag är realistiska.
\newpage
\section{Infokollegiet}
Tidigare Kontaktorer har skrivit att förr har Infokollegiet varit väldigt inaktivt, men jag (Erik Månsson) kan inte längre säga att så är fallet. Vi har under några kvällar om året haft möten om div. saker som har med informationsspridningen på Kåren och sektionerna att göra. De har varit helt okej, skulle rekommendera att du går på dem. Man kan lätt bolla idéer med andra och få inspriation där.

Det kommer mailas en del till dig genom IK, oftast är det andra sektioner som vill ha hjälp att sprida intersektionella evenemang genom allas kanaler. Försök vara hjälpsam och uppmuntra även våra egna funktionärer att använda IK för att sprida info.

Infokollegiet kom igång ytterliggare under 2018 och jag (Axel Voss) tycker att jag ahde bra kontakt med andra informationsansvariga. Formellt hade vi inte så många punkter på agendan under våra möten men det blev ofta långa givande samtal om annat. Jag rekommenderar starkt att du fortsätter gå på IK-möten!

Under 2018 införde vi även en omvärldsrapporteringspunkt på våra styrelsemöten där information lyftes om sånt som inte rör sektionen direkt utan mer indirekt. Jag tycker att detta kan vara en bra grej att fortsätta med och ser gärna att man använder IK för att höra vad som händer hos de andra. Vi är trots allt 11 sektioner som fungerar på ungefär samma sätt och stöter på ungefär samma problem, att få en inblick hur de andra arbetar tror jag gynnar allihopa. 

Det finns även stort utrymme för sammarbete mellan sektionerna, behöver du fotografer till en sittning eller bara lite extra räckvidd till ett event kan du ta hjälp av IK också!

\section{Att vara sekreterare/administratör}
\subsection{How to dokument}
Under 2016 har jag sett till att mer eller mindre alla dokument ligger i en git-repo i Projektplats Hacke på hemsidan, \texttt{elt14ema/docs}. Så som det är nu kan bara styrelsemedlemmar uppdatera det, men egentligen är det ingen annan än Ordförande och du som Kontaktor som ska göra det. Just nu ligger det på mig, Erik, personligen men jag kommer förmodligen ``ge över'' det till sektionenen under nästa år (2017) för enkelhetens skull.

Vet du inte vad \textbf{git} är så skulle jag rekommendera att du som Kontaktor (Electus?) sätter dig in i det så att du förstår de grundläggande sakerna. Alla mjukvaror som används i Sektionen ligger nämligen också som git-repos på Projektplats Hacke.

I docs-repot ligger alla möteshandlingar från 2016, styrdokument med budgetar och verksamhetsplaner, inbjudningar till nolleqasque, mallar till datorstugan, detta testamente, etc.. Tanken är att verkligen alla dokument (i alla fall de som görs i \LaTeX) ska finnas i repot, och så är det i princip redan, förutom alla andras testamenten.
\newpage
\subsection{Styrelsemöten}
Först och främst, läs styrdokumenten om vad som gäller för möteshandlingar angående styrelsemöten!

Eftersom protokollföring bör gås igenom personligen med dig som är nyvald Kontaktor kommer jag inte gå in i stå mycket mer detalj om hur man gör det. Har mest några korta uppmaningar:
\begin{tightdashlist}
    \item Var noggrann, försök att inte skriva fel om vad personer säger på styrelsemötet.
    \item Skriv stödord sen renskriv protokollet samma dag som mötet så du inte hinner glömma bort vad som sades.
    \item Hänger du inte med när du protokollför? Det är helt okej, då kan du bara be om att få pausa mötet ett tag så du kan komma ikapp.
    \item Glöm inte att notera när personer kommer och går till mötet.
    \item Glöm inte att lägga föregående styrelsemötesprotokoll till handlingarna när den punkten behandlas under mötet.
    \item Förbered dig innan mötet genom att skapa alla filer och fyll i protokollet så gott det går redan innan mötet.
    \item Osäker på mötesformalian? Fråga Ordföranden!
\end{tightdashlist}

När protokollet är färdigt så skickar man ut det till justerarna. När de har justerat protokollet skriver man ut det och låter justerarna skriva på. Efter det sätts den signerade kopian in i en pärm för årets protokoll, och en kopia hängs upp på anslagstavlan i Edekvata.

\subsection{Sektionsmöten}
Först och främst, igen, läs styrdokumenten om vad som gäller för möteshandlingar angående Sektionsmöten! Tänk särskilt på att vid sektionsmöten behöver endast beslutsprotokoll föras, d.v.s., du behöver inte ha med diskussioner i protokollen. Tänk också på att tillsammans med Ordföranden börja med handlingarna i god tid. Oftast görs skisser i styrelsens Google Drive innan de överförs till handlingarna.

Till sektionsmötena är där lite mer att tänka på innan mötet. För det första är det du som skriver ihop handlingarna till mötet. Det är en del jobb men förhoppningsvis inte allt för mycket nu efter att jag under året (2016) gjort om mallarna så att det ska gå smidigare. Jobba tillsammans med Ordföranden och kolla på vad som tidigare gjorts. Det är också en bra idé att be några i styrelsen korrekturläsa handlingarna innan du släpper dem.

För det andra måste du fixa närvarolistor, både för när man ankommer/lämnar mötet vid en speciell punkt och vanliga närvarolistor. Mallar ligger i repot. Du måste också fixa ``launch codes'' till E-vote så att vi kan genomföra röstningar med E-vote på mötet. Se också till att du kan koppla in din dator i projektorn så alla kan se handlingarna under mötet. Fixa också en kopia av handlingarna till Talmannen.

Det kan vara bra att växla några ord med Talmannen innan mötet för att se om den har några tankar eller önskemål inför mötet. Det kan vara bra att ta en extra funderare på vilken ordning som motioner och propositioner hamna i, det spelar större roll än vad man tror.

För sektionsmötena gäller samma princip för justering m.m. som för styrelsemötena.
\newpage
\subsection{Styrdokument}
Först och främst, åter igen... läs styrdokumenten om vad som gäller vid uppdatering av styrdokument! Glöm inte bort att vid stadgeändringar ska du tillsammans med Ordföranden skicka in dem till Kåren för stadfästning.

Egentligen finns det inte så mycket mer att säga här. Försök uppdatera och lägg ut dem på hemsidan så fort som möjligt efter att protokollet är justerat. Alla färdiga styrdokument ligger i hemsidans repo \texttt{eracle\_core}

\subsection{Hemsidan}
När det har valts in nya funktionärer på antingen valmöte, vårterminsmöte eller på ett styrelsemöte så måste du föra in dem i systemet på hemsidan. Sedan måste de även få rätt grupprättigheter och mejl kopplade till sig. Om du eller någon ur DDG sköter de sistnämnda sakerna får ni själva bestämma. Enklast är att sköta det själv så att det blir gjort snabbt och låta Macapärerna få arbeta med annat. Vid årsskiftet så kan det vara smart att vänta en månad innan man tar bort gamla funktionärers gruppaccess ifall de har saker kvar att göra på datorerna, t.ex. ekonomiska rapporter. Mejlen däremot är det bara att rycka.

\section{Informationskanaler}
\subsection{Allmänt om våra kanaler}
Våra många informationskanaler är vårt sätt att nu ut till medlemmar, företag, funktionärer, m.fl.. Vanligtvis brukar de ``ägas av'' dig och firmatecknarna, med några undantag. Det är du som Kontaktor som bestämmer över dem och ser till att de används på rätt sätt. Du bestämmer själv hur mycket du vill att folk ska gå igenom dig när de vill nå ut med något mot att de själva fixar det. T.ex. under mitt år (2016) hade alla i styrelsen access till facebooksidorna, men få hade access till ekoli. Skulle rekommendera att vara lite snål med access i början sen släppa upp på det efter hand som man vet att det kommer funka.

Tänk på att all information från vinstdrivande företag ska gå genom ENU - de ska inte få gratis reklam. Allting som publiceras på våra kanaler måste följa vår publiceringspolicy, särskilt gäller det att vi är politiskt oberoende.

Det är viktigt att tänka på att allt för många mindre kanaler underminerar våra primära kanaler. Under 2018 hade jag en nolltolerans mot att andra utskott skapade egna facebooksidor och Instagramkonton. Phoset hade undantag eftersom deras sida är riktad direkt mot nollor och phaddrar samt att deras egna instagram kan hållas hemlig. Det är du som bestämmer under ditt år men det kan va bra att informera de andra Utskottscheferna om vad som gäller!
\newpage
\subsection{Externa kanaler}
\subsubsection{Hemsidan}
Hemsidan är både en nyhetskanal och ett ställe att hitta information om vår organisation. Se på den som väldigt ``officiell'', många företag och andra utomstående är inne och tittar där så den bör vara väldigt proffsig.

Kalendern på hemsidan används typ inte. Mitt förslag är att försöka lägga vår Google Calendar för E-sektionen på kalendersidan istället för att ha den egna som är där nu.

Håll alla sidor på hemsidan uppdaterade. Pusha utskottsordförandena till att uppdatera sina sidor. För t.ex. ordförande för näringslivsutskottet gäller det även att uppdatera sidan för företag, och alumniansvariga borde uppdatera alumnisidan.

Adressen var tidigare \texttt{esek.lth.se}, men efter den så kallade ``sångboksskandalen'' våren 2013 så beslutade LTHs ledning att de inte ville ha någon studentverksamhet under deras egen domän. Både F-, E- och D-sektionen blev av med sina domäner. Sektionen köpte istället domänerna \texttt{esek.se}, \texttt{e-sektionen.se} och \texttt{esektionen.se}, där den förstnämnda beslöts vara den officiella. (Anledningen till att alla tre köptes berodde på att den aktuella ordförande var paranoid och inte ville att F-sektionen skulle skulle köpa upp någon av dem. Dock så ansåg resten av styrelsen att hans oro var befogad och klappade av förtjusning över inköpen.) Revisorerna och Förvaltningschefen 2014 undrade även dessa varför vi hade så många adresser. Efter ovanstående paranoida förklaring så förstod de.

\subsubsection{Ekoli}
På Sektionens skärmar i LED-café och Edekvata visas information rörande Sektionens verksamhet. Typiskt information om events och möten. Du lägger upp bilder genom modulen Ekoli på hemsidan. Helst ska de vara i formatet png för bästa kvalite (men jpg är ok) och exakt 1080p (1920x1080 px). Folk brukar vara rätt dåliga på att följa detta och skickar gärna nån pdf i affisch-form istället för rätt format. Godta inte det, för det ser förjävligt ut. Det är också därför det är lättare att helt enkelt dela ut access till ekoli så mycket - det blir mindre strul om man inte gör det.

Under 2018 försökte jag använda skärmarna så mycket jag kunde. Jag tycker det är ett av de bästa sätten att sprida information i E-huset och bilderna som kommer upp där har större synlighet än affischerna.
Jag bad alltid Picasso's att göra en bild till skärmarna också när de ändå skulle göra affischer. 

Jag har även införskaffat en extra skärm till Edekvata, tanken är att det kommer komma upp busstidtabeller där också. 
\newpage
\subsubsection{Anslagstavlor}
Jag har skött anslagstavlorna som så att jag går en runda varje dag och kollar dem. Sitter där något som jag inte tycker ska vara där så har jag rivit ner det. Under början av året så hade jag väldiga problem med att folk satte upp allt möjligt skräp utan fråga, framförallt från nationerna. Jag löste det genom att ringa dem några gånger och hota med faktura. Men det är extremfallet. Ha som regel att det bör vara en signatur från en Styrelsemedlem på affischerna för att kunna kontrollera informationsflödet på anslagstavlorna. Under 2014 bestämde jag att alla affischer skulle lämnas i LED-Café.  Mackbarsprinsessan Ulla la ner alla affischer i en olåst låda i LED-Café så att jag kunde hämta dem under en godtycklig tid på dygnet. Även bra plats att förvara en kaffekopp på.

Anslagstavlorna är uppdelade i allmänna, utskotts- och externa tavlor. Material rörande företag skall ha någon slags studentkoppling och inte bara vara ren reklam - d.v.s. ingen reklam för läderväskor eller pizza såvida de inte är del av något slags studentevengemang. Företag måste betala för att synas på tavlorna.

Reglerna för tavlorna bestäms av dig och styrelsen. Du kan själv läsa de nuvarande reglerna på tavlorna eller i repot.

\textbf{Allmänna tavlor.} Med allmänna tavlor menas de tavlor som inte har något specefikt syfte mer än att visa sektionsrelevant material. Det finns fyra sådan tavlor; en i LED-café, två i huvudtrapporna och en innanför Edekvatas ingång.

\textbf{Utskottstavlor.} De skall innehålla material rörande utskottet och dess verksamhet. Utskottschefen är ansvarig för vad som sätts upp.

\textbf{Externa tavlor.} Bland dessa tavlor finns t.ex. annonser, ex-jobb och generell föreningsinformation. Här kan medlemmar och icke-medlemmar själva sätta upp material förutsatt att det är märkt med uppsättningsdatum och sätts upp snyggt, dvs. att man inte sätter sin affisch halvvägs över andras affischer osv.

\subsubsection{Facebook}
Sektionens närvaro på Facebook ska i högsta möjliga grad bestå av sidan ``E-sektionen'' och gruppen ``E-sek events''.

\textbf{Sidan ``E-sektionen''} är menad att vara väldigt officiell. Då sidan är en direktrepresentation av Sektionen bör man tänka sig för vad man postar där och särskilt på vilka bilder man lägger upp. Med andra ord - bilder från ``bra krök'' bör inte ligga på vår fb-sida. Helst så ska olika sektionsevengemang skapas på sidan, men det finns givetvis undantag. Flödet på sidan kommer likna det på förstasidan för hemsidan men med några inlägg som är lite personligare.

\textbf{Gruppen ``E-sek events''} är till för mer lättsamma poster från medlemmar och diskussioner rörande sektionen. Gruppen är öppen för alla att gå med i och posta inlägg. Reklam för andra föreningar och företag är generellt sett inte okej. Gruppen har en samling riktlinjer som alla måste försöka hjälpas åt att följa för att bibehålla en bra stämning i gruppen. Gruppadministratörerna har rätt att ta bort inlägg som anses omedelbart kränkande eller reklammässiga.

E6 och CM har varsin fb-sida för deras verksamhet. Medan det är deras sidor så bör du ha lite koll på dem. Det är lämpligt att du har access till de sidorna. Nämnvärt är att Kåren 2016-2017 tog bort de flesta sidorna de hade för att all information skall komma från samma ställe, vilket även resulterar i bättre spridning.

Under 2018 beslutades det att de inlägg som postas på E-sek events ska försöka vara på engelska också. Det leder förhoppningsvis till fortsatt engagemang från våra internationella studenter.  

\subsubsection{Instagram}
Instagram infördes under 2016 och är ett kul sätt för medlemmar såväl som utomstående att få en inblick i Sektionens verksamhet. Posta gärna mycket och låt andra styrelsemedlemmar posta från sina evenemang.

Få de andra i styrelsen att använda Instagram mer, du ansvarar för den men du är inte ansvarig för att få upp allt material!

\subsubsection{YouTube}
Sektionen strävar efter att samla allt videomaterial såsom Ph\o s- och nollningsfilmer på den officiella Youtube-kanalen ``E-sektionen inom TLTH''. Meddela alla detta snarast då t.ex. Ph\o set lätt lägger ut material på deras eget konto vilket skapar problem senare.

\subsubsection{HeHE}
Här finns inte så mycket att säga, HeHE jobbar mycket själva. Oftast är det du som kommer skicka vidare saker till HeHE och bestämmer på så sätt vad som skrivs i HeHE. Kom ihåg att du är ansvarig utgivare och har \textbf{alltid} sista ordet i vad som skrivs där.

Inför 2018 ser intresset för HeHE är tämligen lågt ut, så länge ingen vill dra i det så tycker jag att det får ligga på is men det är något du får bestämma själv.

\subsubsection{Android-appen}
Under 2013 så gjorde Alexander Najafi en app åt sektionen för Android. Sektionen har betalat en engångsavgift för att den ska synas i Google Play där vem som helst kan ladda ner den kostnadsfritt. Då sektionen har betalat för den så har sektionen tillgång till all källkod och tanken är att DDG ska fortsätta vidareutveckla den och fylla på med fler funktioner.

Förr ville man ha en app för iPhone också, men intresset för det är tämligen lågt. Många anser att man istället bör ha en hemsida som är bättre mobilanpassad. Detta eftersom appen i princip bara används till att justera ljudvolymen i Edekvata, en funktionalitet som skulle kunna finnas på en mobilanpassad hemsida istället...

\subsection{Interna kanaler}
\subsubsection{Trac}

Jag (Voss) tycker att trac (\textbf{http://trac.esek.se/}) är någonting som bör användas mycket mer. Det är ett bra vektyg och jag ser gärna att man kommer igång med det igen. Speciellt nu när vi har så många nya som vill engagera sig i utskottet så kan det vara bra med dokumentation och startup-guider. 

Även information om vilka lampor vi använder i våra lokaler kan läggas upp i trac (ligger just nu i drive). Då kan FVU med hustomtar dokumentera där också. 

Även inventeringar av EKEA och Sicrit bör hamna på tracen.

Be teknokraterna dokumentera allt de gör där och säg även till Macapärerna att försöka få kodhackare med egna projekt att göra det samma. Det finns även stöd för felanmälningar vilket båda grupperna kan använda flitigt samt att du kan få en bra inblick i vad alla håller på med.

\subsubsection{Slack}
Jag (Erik) startade slacken (\textbf{esek.slack.com}) sent under 2016 och har som mål att den ska börja användas i stor utsträckning av våra funktionärer under kommande år. Målet är att samla alla funktionärer och medlemmar på samma ställe för att kunna samla alla chatter och diskussioner om sektionen på samma ställe. Det har en del fördelar såsom öppna kanaler och bättre valfrihet i hur man får notiser. För styrelsemedlemmar och andra väldigt sektionsaktiva blir det lätt många chatter och då kan det vara skönt att samla dem på ett ställe som är avsett för organisationer precis som vår, och på så vis kanske känna sig lite mer organiserade och mindre stressade.

\subsubsection{Google Calendar}
På gCal har vi en styrelsekalender och en sektionskalender, just nu ägda av sektionsordföranden (Erik). Tanken är att styrelsekalendern ska vara en intern kalender för bara styrelsen, och att sektionskalendern ska vara ett sätt för medlemmar att få en överblick av vad som händer på sektionen. Det hade varit schysst om sektionskalendern alltid var uppdaterad och att den nådde ut till fler.

\subsubsection{Wikin}
Vet inte riktigt om det finns så mycket att säga om wikin... Se till att den inte kaosar bara. Det är en kul grej - gör mycket reklam om den på och efter gasquen!

\section{E-mail}
Förutom adressen \texttt{kontaktor@esek.se} så är en uppsjö av andra adresser kopplade till dig. För att se vilka de är i detalj kan du använda mail-uppslagningsverktyget på hemsidan. I samband med de kopplade adresserna så kommer du få lite olika automatiska mejl, t.ex. när någon lägger upp en nyhet på hemsidan, en ny bild på infoskärmarna eller skickar ett meddelande via hemsidans kontaktformulär. Då många har access till hemsidans alla funktioner kan det vara bra att ta en titt på vad som lagts upp och att det följer publiceringspolicyn.

Då du är informationsansvarig på Sektionen så är det även passande att du ser till så att de andra styrelsemedlemmarna kopplar sina egna e-postkonton till deras esek-adresser. Dvs. att de kan använda esek-adressen som avsändaradress.

En guide till att använda Sektionens adresser finns här: \url{https://eee.esek.se/wiki/esekmail}

Se även till så att alla i styrelsen använder sig av signaturer i sin mejl. Det ser proffsigt ut och gör det lättare för de man mejlar att kontakta en. Försök få alla i styrelsen att ha likadana signaturer. Ta signaturen från ett mail från din företrädare, ändra om den så den är rätt för dig, sen skicka vidare till din styrelse.
\newpage
\section{Nollningen}
\subsection{Allmänt}
Här vill jag egentligen bara pointera igen hur viktigt jag tycker det är att du är med under nollningen och har roligt. Det kommer bli en del grejor som man gör som styrelse under nollningen och då är det viktigt att man inte bara jobbar och sliter utan också är med och njuter av nollningen.

Det är kul att ha fracken på sig och springa runt under nollningen och det är superbra att styrelsen syns så det inte blir någon grupp som inga vet vilka de är - även fast de är styrelsen som till största del bestämmer över Sektionen. Men med fracken blir det också lite så att nollorna inte riktigt vågar gå fram och prata med en, man blir lite som en mini-ph\o s. Så se till att själv ta initativ att prata med nollorna, våga bjud på dig själv och försök vara framåt och hjälpsam. Om du gör det kommer de efter ett tag lära känna dig inte bara som en styrelsemedlem utan som en vän inom sektionen också, vilket är riktigt kul!
\subsection{Datorstuga}
\textbf{2016}

Under första veckan i nollningen håller traditionsenligt InfU i en datorstuga. Inga konstigheter egentligen, du var förmodligen med där själv så du vet ungefär hur det funkar. Ph\o set kommer säga när den ska vara och de kommer fixa salar. Stäm av med Ingrid så att vi täcker allt programledningen vill ha sagt. Det ligger ett dokument med vad man kan säga på datorstugan i repot.

\textbf{2017}

Under nollningen 2017 anordnade studievägledningen tillsammans med en representat från centrala DDG, LDC. Detta fungerade bra men hur det blir 2018 återstår att se.
\subsection{NollEguiden}
Kontaktorn är tillsammans med Ph\o set ansvarig för att nollan får en nollEguide innan de börjar sin tid på LTH. Under mitt år 2016 var en person i Ph\o set ansvarig för nollEguiden som jag hade kontakt med. Sen hade vi ett gäng från InfU som fixade lite olika delar av guiden. Picasson fixade designen, fotografen hjälpe till med en del bilder (även fast jag tog de på styrelsen själv), och en redaktör hjälpte till att skriva en del texter. Styrelsemedlemmarna och Ph\o set skriver sina egna texter.

Ett tips är att inte uppfinna hjulet på nytt varje år. Det finns verkligen ingen anledning att inte ta så mycket som möjligt från det man tyckte var bra med förra årets nollEguide - nollorna har ju ändå inte sett dem.

Var ute i god tid eftersom tryckeriet har mycket att göra på sommaren. Hör av dig \emph{innan} sommaren och fråga när ni måste ha skickat in den för att den ska hinnas tryckas i tid. Ett tips är att nån vecka innan det slutgiltiga trycket ha gjort ett provtryck för att kunna göra förbättringar. Det är verkligen mycket lättare att hitta fel och/eller möjligheter till förbättringar när man har den i handen.

\newpage
\subsection{Nollehemsidan}
Vår nollningshemahemsida, \href{https://e-nollning.nu}{\texttt{e-nollning.nu}}, uppdateras varje år till det nya temat med ny information och så vidare. Macapärerna/DDG borde ha koll på hur man gör detta. Under mitt år 2016 hade en kodhackare hand om allt ensam och hade kontakt med Ph\o set, vilket gick väldigt bra!

Under 2017 var samma person drivande i en ny nollningshemsida. Den är nu mobilanpassad vilket är mycket bättre än att ha en app enligt mitt tycke.


\section{Kontakten med vänsektioner}
Enligt reglementet så har Kontaktorn ansvar för kontakten med andra högskolor och universitet. Ursprungligen så innebar detta att Kontaktorn skulle sköta brevföringen (ja, fysiska brev) mellan oss och våra vänsektioner. Med dagens teknik så är detta inte aktuellt längre, utan kontakten kan nu istället fokuseras på fler fysiska besök och gemensamma aktiviteter.

Att åka på andra E-sektioners aktiviteter är något av det roligaste man kan göra och verkligen något jag rekommenderar. För stunden så har vi otroligt bra kontakt med både E och MiT på KTH, Elektro på Chalmers, Omega i Trondheim och två finska skolor. Under året kommer ni förhoppningsvis bli anmodade till diverse evenemang på de andra skolorna och jag rekommenderar starkt att du åker på så mycket du kan.

Kom ihåg att skicka ut anmodan till lämpliga vänsektioner i god tid innan Nolleqasquen (inte en inbjudan, då de själva måste betala)! Prata ihop dig med sexmästaren så fort datum för gasquen är spikat.

Det finns även Y-sektionen på Linköping, Teknologföreningen i Finland, Ohmega-Sanktus bordeskab i Norge och en i Danmark. Var inte rädd för att hälsa på dem, men tänk på hur många du bjuder inte till NollEGasquen, varje sektion kommer med ca. 10 personer och det är du som ansvarar för sovplatser. Mitt tips är att satsa på CTH och KTH.

2014 hade vi en sittning med represetanter från CTH, KTH och Finland dagen innan som var väldigt uppskattad där vi fick tid att lära känna varandra, prata lite samt visa dem lite sedenliga traditioner som att bada i sjön Sj\o n.

2016 hade vi också en sittningen dagen före gasquen i FikaFika som var väldigt lyckad! Mitt tips är förutom de som nämnts innan är att håll din planering enkel för dagarna de är här, och minutplanera inte!

2017 bjöd vi in Y-sektionen, KTH, CTH och Sct. Omega Broderskab (Norge) till Nollegaquen. Tyvärr kunde inte så många delta. Det var vad jag vet första gången som vi bjöd in Norge och Y-sektionen, men fortsätt att göra det för att bilda en relation. Vi körde på samma koncept som föregående år, lite mindre fin sittning dagen innan, bad i sjön Sj\o n och bastu.

2018 bjöd vi in E och MiT på KTH, CTH samt Omega. Tyvärr fick vi inte plats för de finska skolorna... Vi hade lekar runt i Lund och en sittning på Lophtet som avslutades med bad och bastu dagen innan. 
\newpage
\section{Kuriosa}
Den Internationella StandardiseringsOrganisationen, förkortat ISO, har tagit fram en internationell standard för hur datum skall skrivas. Standarden heter ISO8601 och datum skall skrivas på följande sätt: YYYY-MM-DD, dvs. 1944-05-01 (Vilket råkar vara Oddput Clementins födelsedag).


\section{Saker som tidigare Kontaktorer inte hann med under sitt år}

\subsection{2015}
Här några saker som inte hanns med under min mandattid som du kan titta på om det verkar intressant. Överförde det jag inte hann från 2013. Tyvärr var infU ett utskott som tog mycket stryk av Karnevalen som var samma år.
\begin{tightdashlist}
    \item Lägg till info-sida om LED på esek.se.
    \item Få upp mer information om uthyrning på esek.se. (Prata med hustomten och teknokraten.)
    \item Få ett fungerande system där fotograferna jobbar innan/under/efter sittningarna för att kunna dokumentera alla (E6 också). En idé är att de får äta av maten med E6 och gå gratis. Diskutera med Sexmästaren hur man kan lösa detta.
    \item Skriva en lathund för styrelsen/sektionen hur man marknadsför sitt event. När, hur och vilka man ska kontakta i infU.
    \item Få Ekiperingsexperten till att samarbeta med Cafémästeriet och börja sälja märken i mojterna.
    \item Skapa versionshantering för protokoll samt reglemente och stadgar i projektplats hacke
    \item Se över reglemente och stadgar samt ta fram en helt ny version. Talman Johan Westerlund och jag diskuterade detta mycket och båda finner att det är något som verkligen behöver göras. De nuvarande stadgarna och reglementet är alldeles för långa och innehåller mängder med onödiga paragrafer och texter.
\end{tightdashlist}

\emph{Alexander Najafi}

\newpage
\subsection{2014}
Här några saker som inte hanns med under min mandattid som du kan titta på om det verkar intressant. Överförde det jag inte hann från 2013. Tyvärr var infU ett utskott som tog mycket stryk av Karnevalen som var samma år.
\begin{tightdashlist}
    \item Lägg till info-sida om LED på esek.se.
    \item Fixa information på engelska till esek.se. (Ta hjälp av den internationella phaddergruppen.)
    \item Få upp mer information om uthyrning på esek.se. (Prata med hustomten och teknokraten.)
    \item Titta över strukturen bland sidorna på hemsidan så att det blir lättare att hitta information.
    \item Skapa ett projekt där fotograferna fotograferar sektionens alla medlemmar och utskott för att lägga upp på hemsidan samt på väggarna i Edekvata.
    \item Få ett fungerande system där fotograferna jobbar innan/under/efter sittningarna för att kunna dokumentera alla (E6 också). En idé är att de får äta av maten med E6 och gå gratis. Diskutera med Sexmästaren hur man kan lösa detta.
    \item Skriva en lathund för styrelsen/sektionen hur man marknadsför sitt event. När, hur och vilka man ska kontakta i infU.
    \item Få Ekiperingsexperten till att samarbeta med Cafémästeriet och börja sälja märken i mojterna.
\end{tightdashlist}

\emph{Henrik Fryklund}
\newpage
\subsection{2013}
Här några saker som inte hanns med under min mandattid som du kan titta på om det verkar intressant.
\begin{tightdashlist}
    \item Lägg till info-sida om LED på esek.se. Arrangera fotografering av alla funktionärer.
    \item Få Ekiperingsexperten till att samarbeta med Cafémästeriet och börja sälja märken i mojterna. Köp in stämplar, en för att godkänna planscher och en datumstämpel.
    \item Göra reklam för kodhackare och DDG
    \item Vit eller svart duk på anslagstavlorna vid entrétrapporna.
    \item Fixa information på engelska till esek.se. (Ta hjälp av den internationella phaddergruppen.)
    \item Få upp mer information om uthyrning på esek.se. (Prata med hustomten och teknokraten.)
    \item Skriva om texten på Informationsutskottets sida på esek.se.
    \item Skapa “Regler för anslagstavlorna”-sida på esek.se.
    \item Titta över strukturen bland sidorna på hemsidan så att det blir lättare att hitta information.
\end{tightdashlist}

\emph{Ludvig Nybogård}

\end{document}
