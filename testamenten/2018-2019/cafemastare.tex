\documentclass[10pt]{article}
\usepackage[utf8]{inputenc}
\usepackage[swedish]{babel}

\def\post{Cafémästare}
\def\date{2017-11-25} %YYYY-MM-DD
\def\docauthor{Elin Johansson}

\usepackage{../e-testamente}
\usepackage{../../e-sek}

\begin{document}
\heading{\doctitle}

Först och främst vill jag gratulera dig till ditt nya uppdrag som Cafémästare! Jag hoppas att du kommer tycka att uppdraget är givande och roligt. Det kommer bli ett superbra år!

Cafémästare är en av de tyngsta posterna på sektionen men därmed även en av de mest givande. Det är väldigt lärorikt på många olika sätt och det är framförallt roligt. Innan du hoppar på vid årsskiftet kommer du få göra några av mina arbetsuppgifter med mig så du lär dig hur det fungerar. Sen ska detta testamente förhoppningsvis täcka allt du behöver veta, men var inte rädd att fråga mig om det är några som helst oklarheter!

Elin Johansson\\
\emph{Cafémästare 2018}

\newpage
\tableofcontents

\newpage
\section{Postbeskrivning}
Vad innebär det då att vara Cafémästare? I E-sektionens reglemente är posten ganska skevt beskrivet. Det står i princip att man är ansvarig för sektionens caféverksamhet. Du får gärna gå in och läsa vad som står i reglementet men jag kommer här nedan beskriva posten så som jag upplevt den under året.

\subsection{Cafémästare}
Som Cafémästare är du huvudansvarig för E-sektionens caféverksamhet, dvs LED. Det är du som är ansiktet utåt och det är dig folk kommer kontakta angående verksamheten. Det är du som är ekonomiskt ansvarig och måste därför ha koll på kostnader, intäkter samt redovisa dessa. Som Cafémästare har du främst två roller, dels som utskottschef men också som styrelsemedlem. Båda dessa roller har olika arbetsområden, är mycket givande och är väldigt kul.

\subsection{Utskottschef}
Det jag syftar på med den här punkten är att din nya funktionärspost är en representativ post. Om någon vill Cafémästeriet något så är det till dig de kommer att vända sig. Detta innebär en hel del mail, en del telefonsamtal och en del folk som letar efter dig. Att vara utskottschef innebär att du måste hålla kontakten med andra, både sektionellt och intersektionellt. Prata med dina funktionärer, prata med de andra i styrelsen, prata med de andra Cafémästarna och prata med andra utanför sektionen som du kommer att ha kontakt med (se mer under "Kontakter"). Se till att ha god kommunikation med dessa och ta hjälp av varandra så gott det går. Som utskottschef är du ansvarig för ett utskott och du tar därmed på dig rollen som ledare över detta. Som ledare ser man till att alla i utskottet sköter sitt jobb och framförallt att de trivs med sina poster inom utskottet. Därför är det viktigt att försöka bygga upp en sådan stämning där alla känner sig bekväma att dela med sig av sina tankar.

\subsection{Styrelsemedlem}
Det här är bland det roligaste som du kommer göra som Cafémästare. Som styrelsemedlem kommer du en gång i veckan gå på styrelsemöte där sektionens verksamhet diskuteras. Där har du chansen att ta del av vad de andra utskotten gör samt ta upp dina egna tankar kring sektionens verksamhet. Glöm inte att välja in Dioder och Halvledare på mötena, de sitter bara på posten läsperioden ut och måste väljas in igen varje läsperiod. Inför varje möte ska du skriva en utskottsrapport där du redovisar vad som händer i utskottet den veckan. Sen inför VT och HT kommer alla styrelsemedlemar skriva större utskottsrapporter (då kan det vara hjälpsamt att kika på de veckoliga rapporterna) samt verksamhetsplansuppföljningar. Inför dessa möten kommer ni även svara på motioner och skriva propsitioner tillsammans, vilket är kul då man kan påverka "på riktigt" då.

Givetvis är det inte bara massa jobb, man har det riktigt kul under tiden också. Förutom att man representerar sektionen tillkommer några trevliga förmåner.
Förmånerna är allt från texten på styrelsefracken till fester enbart för styrelsen. Du kommer även att få 500 kr för att representera ditt utskott med. Dessa pengar får du använda för att gå på sittningar, fester med mera. Det viktiga är att det är högst 250 kr per gång och det får inte användas till alkohol. Du kommer också få möjlighet att hälsa på hos styrelserna på våra vänuniversitet. Jag var på KTH och Chalmers och det har utan tvekan varit ett par av årets roligaste helger.

\section{Utskottet}

Cafémästeriet är ett unikt utskott på så sätt att funktionärerna varierar för varje läsperiod. Det blir mycket folk att hålla koll på men också många att få hjälp av. Se till att delegera så mycket du kan till dina funktionärer, annars kommer det bli för mycket att göra. 2019 kommer antalet jobbare som du kommer att ha tillgång till att vara:

\subsection{Lager- och inköpschefer (upp till 4)}
Efter dig är detta antagligen den viktigaste posten i utskottet. Inköpscheferna ser till att det alltid finns varor till caféet. Om de inte sköter sitt jobb kan ingen annan sköta sitt. I början av året kan det vara viktigt att påminna om beställningarna, annars finns det en risk för att det inte kommer några varor och då måste dessa inhandlas manuellt, det vill säga åka till Axfood i Malmö med bil och inhandla. Glöm inte heller att påminna dem att stänga av den stående beställningen (brödet från Mormors bageri) under lov och tentaperioder. Inköpscheferna sköter även IC-rapporter, beställning av bröd, med mera. En av inköpscheferna får ett sektionskort. Det är \underline{väldigt viktigt} att de fyller i bankkortsförstärkning när de har använt kortet. Se till att de ligger i fas med sina IC-rapporter.

\subsection{Vice Cafémästare (1-2)}
Posten är till för att avlasta dig som Cafémästare, men exakt vad de ska göra är upp till dig. Det bästa är att tidigt få en rutin på fasta uppgifter de ska göra, t.ex att sköta de regelbundna kylkontrollerna. Det viktigaste är att delegera uppgifter till dem (uppgifter som du inte hinner med/tycker är tråkiga, tidskrävande) och att hela tiden ha bra kommunikation. Om dina vice får känna sig delaktiga i alla beslut och funderingar kommer de lättare kunna hjälpa till och avlasta.  Lämpliga jobb för vice kan vara rengörning av kaffebomberna (bör göras 1 gång per läsperiod) och avkalkning av sandfiltret/avkalkningsfiltret, upplärning av Dioder/Halvledare eller att beställa reservdelar om något gått sönder. Ta gärna hjälp av dina vice vid planering av tackfest också.

\subsection{Halvledare (5)}
Halvledarna ansvarar för vars en dag i veckan och ser då till att verksamheten sköts som den ska den dagen. Till detta ingår framförallt att öppna och stänga men även att rekrytera Dioder och lära upp nya. Halvledarna kan ses som länken mellan dioderna och den centrala gruppen och jag tycker att det har varit bra att ha en messengerchat med alla halvledare och centrala CM. Där kan alla hjälpa varandra och svara snabbt på frågor, så att inte alla skriver till dig hela tiden. Tänk på att svårigheten för att hitta dioder brukar variera över olika dagar varje läsperiod, så se till att halvledare som inte hittar dioder får hjälp eller får stänga. Halvledare är en ganska ny post så se till att prata mycket med dem och utvärdera hur ni i centrala gruppen får bäst nytta av dem. Deras mandatperiod är en läsperiod och de väljs in på styrelsemötena.

\subsection{Dioder (E.A)}
Dioderna jobbar minst 2 timmar i LED en dag i veckan och får för detta en gratis macka/sallad enligt vad som anses passande. Jag hade, eller försökte ha, att dioderna tog mat så många dagar i veckan som de jobbade i timmar, så de som jobbade 2 timmar fick mat två dagar i veckan. Jag kan rekommendera dig att försöka ha tre dioder 8-10 och två förmiddagen 10-13. Efter 13 behövs inte mer än en Diod så om man vill jobba två personer då har de bara fått en lunch för två timmars jobb. De tider som inte fylls kan det vara bra att annonsera ut så att folk kan hoppa in ströpass. Jag gjorde detta via ett excellark men en framtidslösning som nog hade fungerat bättre skulle vara att få hjälp av kodhackare att få upp en modul på hemsidan.

\section{Rutiner}
Här kommer jag ta upp vad jag hade för rutiner under 2018. Det är dock helt upp till dig at komma fram till vad som passar dig bäst.

\subsection{Varje dag}
Jag har haft för vana att varje dag komma in i caféet och kolla så att läget är under kontroll. Jag har försökt prata med och lära känna alla dioder lite för att utskottet ska bli mindre spretigt. Dioder som känner sig sedda och uppskattade tycker även att det är roligare att jobba. Dessutom är det bra att se över hur mycket sallader och mackor som säljs, ifall mängderna kan behöva justeras till dagen efter. Det är även viktigt att du varje dag håller koll på din mail och andra informationskanaler så du inte missar något viktigt!

\subsection{Varje vecka}
Jag hade i början av året som vana att ha ett lunchmöte varje vecka med de centrala delarna av utskottet (inköp och vice). Jag upplevde dock att vi inte hade så mycket att säga så vi fortsatte istället med att ha lunchmöten när någon hade något att ta upp och hålla igång chatten. Ni får testa vad ni tycker funkar bäst för er helt enkelt.

En gång i veckan är det även styrelsemöte, vilket är väldigt viktigt för att få reda på information om hela sektionen. Glöm inte skriva utskottsrapport inför dessa mötena, några korta ord om vad som händer i utskottet.

En gång per vecka ska en veckorapport göras, hur det ska gå till finner du under ``Pappersarbete''. Det går ganska snabbt då vi inte längre behöver oroa oss över kontanthantering.

\subsection{Efter läsperiod 4 och 2 (varje halvår)}
Efter läsperiod 4 och 2 ser du till att allt i utskottet är färdigt. Du ser till att alla veckoredovisningar är gjorda, kassaboxen i läsklagret räknad och så vidare. Men du ska också räkna kaffestreck och göra en internfakturering till sektionen. Även arbetsglädjen som är utskriven (till alla utskott) ska internfaktureras. Det är även viktigt att lämna caféet fräscht inför den långa stängningen. Jag har gjort tydliga städmallar och om man får alla i utskottet (även dioder) att hjälpa till så tar det bara några timmar med en storstädning. Se till att lager- och inköpscheferna har gjort sina IC-rapporter så att de inte för allt för mycket på hög, det underlättar både för dem själva och förvaltningschefen.

\section{Viktiga händelser}
Cafémästeriet är ett väldigt strukturerat utskott med kontinuerligt arbete, men ett par gånger om året finns det vissa händelser som avviker från standarduppgifterna.

\subsection{Miljöförvaltningen}
Miljöförvaltningen kommer en gång per år till caféet och ser att vi sköter oss enligt alla förordningar. Det har varit Sarah Strandh som varit ansvarig för det här, men jag tror att en man tagit över nu. Det bästa sättet att slippa vara orolig för det här besöket är att se till att egenkontrollen sköts (den kan du hitta i en pärm i LED eller i din CM-mapp i filsystemet) och att dioderna följer reglerna för livsmedelshantering. Prata gärna med Cafékollegiet på kåren och se till att det blir en livsmedelsutbildning som ni kan gå på.

\subsection{Caféfesten}
Caféfesten brukar vara på våren och är tillsammans med D, F, I, M, V och K (detta året hade vi det uppdelat mellan F, E, D, K samt V, M pga M och V helt enkelt inte ville ha med oss andra). Ta upp det här tidigt på cafékollegiemötena och försök fördela arbetsuppgifterna. Det viktigaste som ska göras är: vilken sektion som ska stå för alkohol och tillstånd, vilken som ska fixa maten, vilken lokal man ska vara i (om det skulle vara i E-huset ska du prata med Mats Cedervall, men helst vill du inte ha det här). I år använde jag av caféets vinst för att finansiera festen, men det finns även en budget för kick off/kick out som alla funktionärer har rätt till. Prata med Förvaltningschefen och se hur ni ska göra. Alla gamla Cafémästare bjuds traditionsenligt in till denna fest, du når alla via e-postadressen \texttt{cm-em@esek.se}. Festen ska eventuellt redovisas som en arrangemangsredovisning, men kolla med Förvaltningschefen för säkerhetens skull.

\subsection{Expot}
Expot är en förmiddag/lunch innan valmötet då styrelsen bjuder på mat och godis och berättar om sina utskott. Detta är ett utmärkt tillfälle att hitta de sista funktionärerna som du saknar för nästkommande året och att få ettor att bli dioder till läsperiod 2. Se även till att uppdatera utskottsbeskrivningen ifall det skulle behövas.

\subsection{Julgillet}
Du är domare till pepparkakstävlingen under julgillet. Detta är en av de viktigaste sakerna du gör som Cafémästare. Se till att så många som möjligt deltar, se till att dem bygger ätbara hus då det är du som får smaka dem och se till att det finns riktigt nice priser till vinnarna. Prata med Krögaren om hur ni ska lägga upp det.

\subsection{Speciella dagar}
Under året kommer det finnas specifika dagar så som kanelbullens dag där vi brukar sälja extra mycket av det dagen erbjuder. Det är en kul grej och vi kan dra in ganska mycket pengar på det. Fettisdagen blev årets högsta försäljning. Det finns ett annat dokument för just det.

\subsection{Promota [insert]}
Under året har jag fått förfrågan ifall vi kan sälja kaffe till något typ av företag, utskott, förening eller liknande. Det brukar oftast vara i samband med att dem vill promota ett event eller bara synas mer och då står de oftast i E-foajén. Jag har försökt ställa upp så ofta jag kan då det är kul att hjälpa till plus att vi får 5L kaffe sålt på en gång. Detta gör du hur du vill, se bara till att dem som är i caféet just då är med på det och att det inte går ut över vår verksamhet.

\section{Pappersarbete}
Härefter kommer de vanligaste papprena som du kommer stöta på och även kortfattat hur dessa fylls i. Om du skulle vara osäker på något och dessa exempel inte hjälper, kontakta Förvaltningschefen för att se till att det blir rätt.

\subsection{Veckoredovisning}
Det här kommer du antagligen få göra ca 30 gånger under året (kom igen, det blir kul!). Jag går igenom stegvis vad du ska göra vid varje tillfälle:

\begin{numplist}
    \item Logga in på datorn, duplicera Veckoredovisning och döp den till något passande (så att du vet exakt vilken vecka som det gäller) och öppna den. Logga in på izettle och kolla så att försäljningsrapporterna och kvittona stämmer överens. Om inte: skriv ut rapport för de dagarna som diffar och använd istället.
    \item Fyll i dokumentet enligt kvittona (alternativt utskriven rapport) och skriv ut (lägg ett lila papper i skrivaren innan du skriver ut).
    \item Signera rapporten och lägg den i Förvaltningschefens fack.
    \item FÄRDIGT!
\end{numplist}

\subsection{Kassaboxredovisning}
Pengarna i kassaboxen i läskförrådet borde räknas in en gång per läsperiod. Om du känner dig osäker angående kontanthanteringen (som vi inte håller på med så mycket nu) så fråga förvaltningschefen. Så här gör du redovisningen:
\begin{numplist}
    \item Töm kassaboxen på pengar.
    \item Logga in på datorn och gör en kopia av Kassabox och döp den till något vettigt.
    \item Räkan in pengarna från kassaboxen och fyll i värdena i kassaboxredovisningsdokumentet. Spara, skriv ut och lägg pappret i Förvaltningschefens fack.
    \item Skriv in pengarna i handkassepärmen.
    \item FÄRDIGT!
\end{numplist}

\subsection{Caféfestredovisning}
Till denna ska det eventuellt skrivas en arrangemangsrapport. Ta hjälp av Förvaltningschefen så att det blir rätt.

\subsection{Fakturering}
Du är ansvarig för att fakturera andra sektioner och företag som köper något från cafét. Det är dock förvaltningschefen som bokför sedan. Fråga förvaltningschefen om denne vill att du skriver ut fakturorna eller skickar som mailkopia till fvc@esek.se. Nuförtiden ha majoriteten av alla kunder en email-adress man kan skicka fakturan till istället för att behöva skriva ut och skicka fysiskt.

\subsection{Kallelse}
Inför styrelsemöte får du en kallelse. I denna står föredragningslistan och när och var mötet kommer att ske. Om du vill ta upp något större rekommenderar jag att du skriver ihop vad du vill ta upp i god tid innan mötet och skickar till Ordföranden, så att det kan komma med i handlingarna och de andra i styrelsen kan fundera på vad de tycker. Annars kan du lägga till punkten på mötet då dagordningen ska godkännas. Om det är oviktigt är det bara att säga det under ``Övrigt'' på mötet.

\subsection{Internfakturering}
När andra utskott köper t.ex. kaffe eller sallader av caféet så ska du fakturera dem för inköpspriset på varorna, inte det vi säljer dem för. Jag ger med en bilaga på en mall jag använder för att se inköpspriserna, försök vara noga att kolla så att priserna stämmer överens med verkligheten. (OBS! Inte vid beställning av varor till andra, det går på IC-rapporter). Det vi oftast säljer till andra utskott är kaffe, där jag tagit 50 kr per kaffebomb, 10 kr/mjölkpaket och 30 kr för oatly iKaffe. Dessa priser gäller dock bara för andra utskott inom sektionen!

\subsection{Extra inköp}
När ni ska köpa saker till cafét som inte kan köpas med faktura ska ni i första hand betala med sektionskortet som en av inköparna får. Det är den inköparens ansvar att göra kvittoförstärkning sedan, men du måste skriva på och kontera. Om det inte är möjligt att handla med sektionskortet kan man betala med privat kort och göra en utläggsräkning.

\section{Kontakter}
När du går på är det viktigt att du ändrar kontaktinformation hos Martin\&Servera och snabbgross. Du kan antingen maila eller ringa. Detta för att gamla Cafémästare inte ska få samtalen och informationen ska försvinna på vägen.

\subsection*{Avkalkningsfiltert}
Villiam installerade avkalkningsfiltret till kaffemaskinen. Ring honom om det är något problem med filtret. \\
Villiams tel: 0767-22 12 82

\subsection*{Kaffemaskinen}
Senast kaffemaskinen gick sönder så anlitade vi REPRO, de är dyra men de är snabba och effektiva. Kontakta dem om kaffemaskinen går sönder men testa gärna resetknappen och starta om den.\\
Tel: 040-29 55 65, hemsida: \url{http://www.repro.se}

\subsection*{Miljöförvaltningen}
Vi har precis fått en ny miljöinspektorn som kommer en gång per år och ser till att vi håller caféet rent och att vi följer reglerna. Jag vet tyvärr inte vad han heter eller hans telefonnummer. Om du kontaktar miljöförvaltnignen i Lund kan du boka in kontrollen, annars kommer de helt oväntat, antagligen någon gång under våren.\\

\subsection*{Martin \& Servera}
Våra kontakter på Servera. Om du till exempel vill boka en extra leveranstid eller har något klagomål angående leveranserna, avtal etc.

\subsubsection*{Niclas Ellberg, Innesäljare}
Om det är något som gäller beställningar så ring Niclas, tips är att låta inköp få hans uppgifter så de kan ringa direkt om det är varor som saknas eller är defekta.\\
Tel: {040-28 78 56} eller \url{niclas.ellberg@martinservera.se}

\subsubsection*{Ulf Franzon, Distriktansvarig säljare, Skåne}
Vid konstigheter/frågetecken kring vårt avtal och/eller kundnummer hos dem, så är det bara att kontakta honom i första hand. Likaså när det kommer till frågor som rör leveranserna eller våra fakturor.\\
Mobil tel: {0708-18 30 84}, mail \url{ulf.franzon@martinservera.se}

\subsubsection*{Övrig info Martin \& Servera}
Inloggningsuppgifter e-handel: Användarnamn: {519043} Lösen: {hacke5284}

\subsection*{Övriga}
\subsubsection*{Mormors bageri}
Från Mormors bageri köper vi allt bröd, det är dina inköpschefer som har kontakt med dem. Se till att de stoppar beställningarna vid tentaveckor och lov samt ringer dem om det är något som inte stämmer vid leveransen.\\
Tel: {046-32 87 87}, Kundnummer: {512}

\subsubsection*{Diskbolaget (diskmedel och torkmedel)}
Vi beställer maskindiskmedel och -torkmedel från dem. Inköpscheferna eller vice anser jag ska sköta detta.\\
Tel: {040-40 59 80}

\subsubsection*{Axfood/Snabbgross}
Från Axfood beställs det framförallt läsk, men det går även att beställa annat. Se till att inköp beställer in läsk när det behövs, gärna i större mängder, så vi inte behöver få för många olika beställningar. Inköp gör även IC-rapporterna för allt som handlas på faktura hos snabbgross.\\
Tel: {040-68 06 380}\\
Inloggningsuppgifter: Användarnamn: {9119659} Lösen: {381035}

\textbf{PS:} Det är inte fullt garanterat att alla dessa uppgifter stämmer exakt men alla som är viktiga att ha kontakt med är åtminstone uppskrivna.

\section{Övrigt \& tips}
Här kommer jag lista några tips and tricks som kan vara bra att kunna. Det kommer en blandning av mina egna tips och äldre Cafémästares tips.

\subsection{Kontakter i huset}
En person som är mycket bra att ha kontakt med är PH. Han sitter nere i tryckeriet och är mycket hjälpsam när det kommer till att trycka posters av olika slag, kaffekort och ifall det är några problem i lokalen (exempelvis om det är stopp i vasken). Jag har även haft kontakt med honom angående att fixa LU-kort till brödleverantören, om sophantering och problem med dörrarna till LED som verkar krångla allt oftare.
Det kan även vara bra att känna till Mats Cedervall som är husets prefekt, han har koll på allt gällande huset och sitter på tredje våningen i E-huset.

\subsection{Styrelsen}
Jag rekommenderar starkt att jobba nära styrelsen och att ni bör ta hjälp av varandra, det hjälper i många situationer och ger bättre sammanhållning. Detta gäller framförallt under nollningen då alla utskott har mycket att göra, se till att ni delar på arbetsbelastningen och hjälper varandra med relevanta saker. Försök även att hålla de andra utskotten i öronen när det gäller läsklagret, se till att panten hanteras snyggt, läsk skrivs ut korrekt, E6 håller sina kylar i skick och SRE har sina saker på sin plats. Ta även hjälp av de andra utskotten när ni saknar dioder, de är enligt reglementet skyldiga att hjälpa till. Då är det dock viktigt att vara ute i god tid.

\subsection{Cafémästarkollegiet}
Samma gäller för Cafémästarkollegiet, det är ett bra forum att bolla idéer samt prata om problem och verksamheten. Andra sektioners verksamhet kan skilja sig mycket mot vår, men det kan göra att vi har lösningar på andras problem och de på våra. Det är också en direkt kontakt till kåren, vilket är bra när större saker ska bearbetas. Här kan ni även diskutera Caféfesten.

\subsection{Egenkontroller och inköpskostnader}
När det kommer till egenkontroller så bör dessa revideras årligen för att se till att vi följer lagarna och att den reflekterar vår verksamhet. Se till att dina vice har bra koll på detta och sköter det.
Se till att ha koll på och uppdatera inköpskostnaderna kontinuerligt också, det är extra skönt att ha koll på dessa då allt som säljs till andra utskott säljs för inköpskostnaderna.

\subsection{Verksamhetsplan}
Sektionen har en verksamhetsplan som ska komplettera budgeten. Verksamhetsplanen innehåller mål som tidigare styrelser har satt för kommande år för att få ett långsiktigt tänkande i sektionen. Verksamhetsplanen är ingen lag, men precis som budgeten bör avvikelser mot den motiveras/redovisas. Verksamhetsplanen ska kontinuerligt redovisas i samband med sektionsmöten, samt uppdateras inför höstterminsmötet. Den går att hitta på hemsidan.

\subsection{Ekonomi}
Under året bör du kolla av LEDs ekonomi, så det gäller att få inköps- och lagercheferna att ligga i fas med IC-rapporterna. Försök att göra upp en plan över hur mycket ni behöver dra in under olika perioder, generellt säljer vi mer i början av läsperioderna. Det är värt att tänka över "belöningssystemet" för att få bättre koll på ekonomin.

\subsection{Kontaktinfo}
Vissa av kontakterna nedan är otroligt dåliga på att uppdatera sina kontaktinfo. Under mitt år har de ringt inköp som satt 2015. Försök se till att dem bättrar på det så att du inte undgår viktig information. Kan även vara ganska skönt för gamla inköp att inte bli väckta på morgonen av att en leverans har kommit.

\subsection{Sponsring}
Vi brukar få sponsmuggar med jämna mellanrum. Ett företag som heter lime har kontaktat mig direkt (de ville inte ha något i utbyte för att lämna muggarna) men jag rekommenderar att du låter ENU sköta sådana kontakter i övrigt. 

\section{Hälsningar}
Tveka inte att kontakta en gammal Emeritus om du har några frågor eller funderingar!

Till sist: Låt inte posten bli en börda! Det är en tung post men den har sina fördelar. Glöm inte att ha roligt under året, ha rolig kick-off/out med utskottet och passa på att hänga med styrelsen. Troligtvis är det personer du aldrig annars skulle börjat hänga med som helt plötsligt blir dina bästa vänner. Var inte rädd att utnyttja din frihet som chef över caféet, har du några kul idéer du vill genomföra gör det. Sätt din egen prägel på caféet och dess verksamhet!\\
- \textit{Elin Johansson, Cafémästare 2018}

\end{document}
