\documentclass[10pt]{article}
    \usepackage[utf8]{inputenc}
    \usepackage[swedish]{babel}

    \def\post{Øverphøs}
    \def\date{2018-12-06} %YYYY-MM-DD
    \def\docauthor{Andreas Bennström}

    \usepackage{../e-testamente}
    \usepackage{../../e-sek}

    \begin{document}
    \heading{\doctitle}

    \textbf{Grattis till den roligaste funktionärsposten på LTH!}\newline
    Tro det eller ej, men du är nu E-sektionens Øverphøs 2019! Stort Grattis! Detta testamente, som är en kombination av de senaste årens testamenten, är riktat till dig som Øverphøs och dig enbart. Ett annat testamente som är för hela NollU finns också, som är övergripande för hela nollningen och planeringen.

    \emph{Vänliga hälsningar}\\
    \emph{{\docauthor}, Øverphøs 2018}

    \begin{center}
    \includegraphics[width=15cm,height=17cm]{hyperion.pdf}    
    \end{center}

    \newpage

    \tableofcontents
    
    \newpage

    \section{Rollen som Øverphøs}
    Att vara Øverphøs kan vara bland det roligaste som finns, men även det tuffaste och mest utmanande. Samtidigt som du är utskottsordförande för NollU är du även en styrelsemedlem och medlem i ØPK. Detta innebär att man mer än ofta kan behöva vara på flera plaster samtidigt. Prioritera rätt och variera. Som Øverphøs representerar man phøset och E-sektionen i ØPK och vice versa åt alla håll. Som Øverphøs jobbar man mycket som en informationskanal mellan alla organen. Detta innebär att man får meddela och stå bakom beslut som man kanske inte står för, men man får inte glömma av att man som phøs, styrelse eller ØPK vill nollningen och nollan bäst. Alla tre organ ser olika på hur man bäst löser och förbättrar nollningen och arbetet kring detta. Under mitt år försökte vi ha en väldigt plan struktur i phøset. Vi försökte alltid komma fram till ett gemensamt beslut som alla var nöjda med, men ibland var vi tvungna att rösta och eftersom vi var ett jämnt antal, så vi löste det genom att min röst endast vägde tyngre i val där det inte blev majoritet för något förslag.

    Kom ihåg att ett glatt phøs är enklare att jobba med, så se till att ta hand om dem så tar de hand om dig

    \subsection{Först att göra}
    Ta det lugnt! Det är förståeligt att man vill hugga tag direkt i planeringen, men ta det lugnt! Det är t.ex. ingen idé att börja planera St.Hans lv.0 förrän i mars. Börja med att lägga upp en planering över hela våren och sätt upp lite milstolpar \textbf{(även möjliga datum för ett 3-4 dagarsskiphte med NollU 18 och 19 under våren är extra bra om man kollar upp tidigt :))}.  Man lär sig väldigt mycket med tiden och det kan vara skönt att kunna ändra i planeringen så att man inte har låst fast sig direkt. Under t.ex. ØPK-helgen kommer man att gå igenom alla sektioners Nollningsscheman 2018 och diskutera dem. Detta är ett kanontillfälle att ta inspiration från andra

    Bestäm hur ni vill jobba tillsammans: Ska ni ha en gemensam Google Drive, Dropbox eller annat? Skall all kommunikation gå via mail, sms eller messenger? Seriös/oseriös smsgrupp? Tänk på att du kommer att vara med i tonvis med chattgrupper och liknande så försök att berätta det klart och tydligt för phøset så att ni är på samma plan. Det rekommenderas att tvinga phøset att använda slack. Detta är något som jag ångrar att jag inte var hårdare på. Då kan man ta allt seriöst i slack och sen oseriöst i till exempel messenger. Ett starkt tips om man inte väljer att använda slack för seriösa grejer är att antingen säga från början att man inte får spamma ALLS i phøschatten om ni har en sådan på Facebook ifall det är något som kan vänta till nästa möte. Alternativt införa en chatt speciellt för spam och en för bara seriösa saker, så att man inte känner att man alltid måste läsa igenom alla 100 meddelanden om söta hundar och balla klädidéer för att inte missa något viktigt. Ta detta på allvar även om det inte verkar som ett problem nu. Det kommer komma kvällar och helger när man är helt slutkörd och inte behöver ännu ett orosmoment i form av konstant skrollande för att inte våga missa något viktigt i chatter, när man kunde tagit den enda lediga kvällen i veckan till att ladda batterierna. Det tar mer på krafterna att alltid hålla sig uppdaterad och alltid vara tillgänglig än man tror.

    Diskutera i förväg om ni vill ha någon tid i veckan som man tar ledigt inom phøsgruppen och våga säga till utomstående (t.ex. någon på sektionen som behöver hjälp med nollningshemsidan eller en phadder) som har frågor och kommentarer att det kan vänta några dagar eller att du helt enkelt inte har tid. Folk förstår att man är upptagen och om du väntar till nästa arbetsdag med att  svara folk (OBS främst för folk UTANFÖR phøset) så kan det till och med leda till att de inte räknar med din kontanta tillgänglighet i framtiden. Detta sätter mindre press på dig och din grupp och kan ge er tid att hantera saker som är högre prioriterade de tiderna. Ni får då också tid att hinna prata igenom saker med varandra, t.ex. på nästa möte, innan ni återkommer till personen i fråga med svar.

    \section{Styrdokument}
    Det finns något som heter Stadga och Reglemente på sektionen som det är bra att ha koll på som Øverphøs. Man behöver inte vara en paragrafryttare och gå på offensiven med nedanstående material, men om en konflikt uppstår kan vända sig till dessa som försvar.

    \subsection{Stadga}
    Stadga finnes \href{https://eee.esek.se/files/styrdokument/stadga.pdf}{\textbf{här}}
    \begin{itemize}
        \item [] Paragraf §1
        \item [] Paragraf §7:2
        \item [] Paragraf §8:11
        \item [] Paragraf §9:11
        \item [] Paragraf §14:3
    \end{itemize}

    \subsection{Reglemente}
    Reglemente finnes \href{https://eee.esek.se/files/styrdokument/reglemente.pdf}{\textbf{här}}
    \begin{itemize}
        \item [] Kapitel 9:1:A
        \item [] Kapitel 9:2:H
        \item [] Kapitel 10:2:D
        \item [] Kapitel 10:2:L
        \item [] Kapitel 10:2:N
    \end{itemize}

    \subsection{Policybeslut}
    \begin{itemize}
        \item [] \href{https://eee.esek.se/files/styrdokument/policies/sektionsaktiviteter.pdf}{Principer för deltagande i sektionsaktiviteter}
        \item [] \href{https://eee.esek.se/files/styrdokument/policies/nyckelpolicy.pdf}{Nyckel- och accesspolicy för E-sektionens funktionärer}
        \item [] \href{https://eee.esek.se/files/styrdokument/policies/utlaggspolicy.pdf}{Utläggspolicy}
        \item [] \href{https://eee.esek.se/files/styrdokument/policies/inbjudningar.pdf}{Inbjudningar och anmodningar}
    \end{itemize}
    
    Det finns också en \href{https://eee.esek.se/files/styrdokument/verksamhetsplaner/vp-2018.pdf}{\textit{verksamhetsplan}} för utskottet som bör följas.
    
    \section{NollU}
    Ditt grymma utskott består av 5 Cophøsare och helst 2 Øvergudsphaddrar. När du läser detta har du kanske redan en grupp kamrater som bildar ditt phøs och påbörjat planeringen. Ett annat starkt tips till hur ni väljer era øvergudsphaddrar är att välja folk som visar att de verkligen vill bli det, förslagsvis om ni håller intervjuer för kandidater till posten. Detta gjordes både 2017 och 2018 och vi tyckte det var skitbra! Är det ingen etta med i phøsgruppen när vi väljs är ett annat starkt tips att försöka få minst en av era øvergudsphaddrar att vara en etta.
    
    \subsection{Ta hand om phøset}
    A och O. Ditt phøs kommer att jobba väldigt mycket och det är viktigt att man ser till att arbetsbördan blir så jämnt fördelad inom phøset som möjligt. Erbjud dig att hjälpa till om någon har väldigt mycket under viss period, men var inte rädd att säga till dem att släppa det de har om du ser att de är stressade. År 2015 var det rekordmånga avhopp bland andra phøs och øverstar pga. bla. arbetsbelastningen. Även åren därefter har det varit ett antal avhopp även om detta varit i mindre utsträckning. Var dock aldrig rädd för att fråga ditt phøs om hjälp ifall du behöver det som avlastning eller för att du fysiskt inte hinner göra saker!

    Tänk på hybrisen. Som phøs är det lätt att få hybris och det kan vara bra att inte ha det då det skapar konflikter med resten av sektionen. Var ödmjuk!

    Ha en kickoff med NollU och gör något kul. 2015 åkte NollU till Utmaningarnas hus i Malmö som var riktigt populärt! 2016 åkte vi i NollU till en stuga i Blekinge som en i phøset hade tillgång till och hade en helhelg med grill i skogen osv. 2017 var vi iväg och sov i en stuga ihop. 2018 kom vi inte iväg en hel helg under våren, utan vi körde en Tour de NollU istället, vilket också var riktigt roligt. Det är viktigt och skönt att ibland göra saker tillsammans som inte är arbete och se till att planera in de tiderna i sjukt god tid!
    
    Utöver en kickoff skulle jag (Andy) säga att det är otroligt viktigt att verkligen spendera mycket tid tillsammans i gruppen i början när ni blivit valda. Det underlättar otroligt mycket i det fortsatta arbetet under året om man verkligen känner varandra bra. 

    Ett annat tips som är lite konstigt från tidigare Øverphøs är att BALLA UR ORDENTLIGT tillsammans. Man lär sig mycket om varandra när det ballar ur ordentligt (T.ex. DÖMD eller en Tour) och det skapar minnen(?) som är roliga. Vi hade några sådana kvällar under året. 2018 åkte vi också iväg till en stuga en helg under sommaren. Det är verkligen rekommenderat att ha en hel helg med hela NollU under sommaren där man samlas, hittar på något roligt och framförallt bara hänger tillsammans.

    \subsection{Øvergudsphaddrarnas roll}
    Diskutera med phøset vad ni vill att Øvergudsphaddrarnas roll skall vara. 2013 hade de en mer administrativ roll (de var macapär och kodhackare), 2014 skötte de information och det grafiska och 2015 skötte de informationen och avbelastade entertainern lite. Sedan 2016 har de haft allmänt bra koll på läget och funnits till som phøsets förlängda arm när phøsarna inte kunnat vara på plats för att ta tag i något, samt skickat ut info. Prata även igenom hur mycket ni vill att Øvergudsphaddrarna skall få ta del av allt arbete. Vi valde att vara helt transparanta med dem eftersom vi litade på dem och kände att det skulle underlätta arbetet. Detta innebar att de hade access till alla filer och var inbjudna till varje möte. De hade dock ingen rösträtt i beslut. Det kan vara väldigt smart att ha en bra struktur på vilka möten ni i phøset vill att de ska gå på, eftersom de inte måste gå på samma sätt som phøset. T.ex. om øvergudsphaddrarna valt att gå på alla 5 första möten och de inte behövs förrän på det 6e mötet men då inte dyker upp, så kan det vara problematiskt. Tänk även på att om det är allvarliga kritiska diskussioner så är det phøset och du som tar besluten.

    Om någon av øvergudsphaddrarna skulle behöva hoppa av innan eller under nollningen så kan det vara en bra idé att sätta sig ner med hela resterande NollU och diskutera igenom en eventuell omfördelning av arbete inför nollningen så att ingen går in i väggen till följd av någon annans beslut.

    Viktigast av allt, se till att ØGP blir en del av gruppen. Detta kan vara svårt då man som phøs automatiskt spenderar mer tid ihop i form av utbildningar och dylikt.

    För ØGP gäller samma sak som phøs, glada ØGP = enklare liv för dig.

    \section{Styrelsen}
    Skapa goda kontakter med alla! Alla vill jättegärna bidra under nollningen, så våga lämna över mycket ansvar till dem. Försök vara så öppen som möjligt med planeringen gentemot styrelsen eftersom det är roligare för dem att hjälpa till när de vet vad det är de hjälper till med. Detta gäller för allmän hjälp från folk under våren. Tänk dock på att information lättare sprids vidare från dem än från folk i FHØB. Därför kan det vara bra att vara tydlig med att hålla det som sägs lite konfidentiellt om det är nollningsrelaterade beslut som inte är helt spikade ännu. Berätta då även att det är just för att ni inte vill att folk ska anta saker som inte blir av eftersom det inte leder till något bra. Det kan skapa onödiga rykten på sektionen och en massa onödiga frågor att behöva lägga energi på om informationen skulle spridas. Var även öppen för idéer och förslag från dem. De har grym koll på sina utskott och underbara idéer på att utveckla dem. Nollningen är en strålande tid för dem att visa upp sig, men glöm inte att fokus skall ligga på nollan.

    \section{ØPK}
    Skitkul, tidskrävande, mycket diskussioner. VAR AKTIV och GÖR SKILLNAD. Representera phøsets gemensamma åsikt (även om det ibland måste tas beslut så snabbt att du inte hinner kolla med dem först). Man får antingen skriva till dem och hoppas på snabbt svar eller bestämma något man tror är allas samlade vilja. Skapa goda kontakter med andra Øverphøs. Tänk dock på att nollningsaktiviteter med andra Øverphøs inte nödvändigtvis speglar hur sektionen och nollorna kommer att komma överens med varandra. Det är lätt och farligt att bli för tajt med ØPK då man är flera i samma sits vilket kan leda till att man driftar bort från phøsen. Samma gäller styrelsen, fast åt andra hållet. Jag tycker att man skall prioritera phøset över de andra så gott det går. Ett tips är att \textit{väldigt} tidigt under våren fråga andra øverphøs om gemensamma aktiviteter eftersom de lätt ``bokas upp'' av andra sektioner med liknande intresse. Tänk på att sektionerna ni har aktiviteter med ska vara relevanta för nollan att ha aktiviteterna med. T.ex. en sittning med alla som läser Endimen tillsammans eller de i samma hus osv. Tänk på att alla ska ha en jämn del i planeringen för gemensamma evenemang så inte en sektion sköter planeringen för alla. Både för att alla vill vara med och bidra, men också för att ingen ska få överdrivet mycket att göra när det är allas ansvar.

    För din egen skull är det bra att skriva kortfattade anteckningar om saker du måste ta upp med ditt phøs till nästa möte, eller saker du måste göra. Ha en punkt på phøs-mötena där du går igenom lite ”briefing” från det senaste ØPK-mötet om det är relevant (ibland är det inte det). Här får du även tillfälle att ta upp ”läxor” som hela phøset ska göra till nästa ØPK-möte. Tänk på att denna punkt kan variera ENORMT beroende på veckan, så håll det så kortfattat som möjligt och fundera på om det är rimligt att ta punkten på ett lunchmöte eller om man vill ta det på ett kvällsmöte samma vecka. Hör också med phøset om de vill få tillgång till hela ØPKprotokollen, eller om de tycker att det räcker med att du briefar på möten. 2018 lades alla protokoll upp i FHØB-driven, vilket innebar att alla kunde gå in och läsa dem om de ville. 

    (Frykis: Efter varje ØPK-möte och styrelsemöte mailade jag hela NollU om beslut och diskussioner som uppstod. Samtidigt passade jag på att delegera lite arbete och fråga om åsikter som vi kunde diskutera nästa gång vi sågs. Exempel på utdrag:
    \begin{itemize}
        \item Bla bla, detta beslutet togs.
        \item NG behöver alla phaddrars kontaktuppgifter, fixar du det \textbf{XXXX}?
        \item Detta kom upp på diskussion, vad tycker ni? Pros: XXX Cons: XXX (som dök upp på mötet)
    \end{itemize}

    Fetmarkera det som skall stickas ut och som är det viktigaste). Jag (Molly) gjorde inte det utan tog upp det på nästa möte när folk faktiskt lyssnade istället.

    \section{Möten och mötesteknik med Phøset}
    \textit{Bestäm hur ofta ni vill ses och hur ni ska föra mötesprotokoll} \newline
    Vissa har valt att enbart ha lunchmöten och hålla dem informativa och diskutera om tid blev över. Andra phøs har valt att ha 2 kvällar i veckan där de har setts och arbetat. De blev väldigt tajta, men det uppstod också en hel del konflikter inom phøset om nerlagd tid vilket ledde till bl.a. avhopp. Detta hände 2015. 2016 och 2017 valde vi att ha ca ett lunchmöte i veckan och ett kvällsmöte. Ibland blev det såklart mer eller mindre, men det var en väldigt lagom utgångspunkt med den mängden jobb vi valde att göra. \newline
    Även 2018 valde vi till en början att ha ett lunchmöte och ett kvällsmöte i veckan. Sen kom det perioder under våren då vi kände att vi hade mycket att göra. Då gick vi upp på två kvällsmöten i veckan istället i någon månads tid. Sen när vi kände att vi låg i fas igen, gick vi tillbaka till ett kvällsmöte. Så helt enkelt, försök vara flexibla och ändra utefter hur ni ligger till i planeringen.

    Givetvis uppstod det att vi var tvungna att ha många kvällsmöten inför större milstolpar under våren. Under temasläppsveckan och veckan inför E:s phadderutbildning och kick-off hade vi kvällsmöten flera kvällar. Då det skedde var det svårt att hitta bra kvällar då alla inte redan var uppbokade. Därför kan det ibland vara bra att prova på att ha alternerande. Lunchmöte och varannan vecka även ett kvällsmöte på en bestämd dag.

    \textit{Kom ihåg att ni gemensamt kommer fram till hur mycket tid ni vill lägga ner}. Du kommer redan ha kalendern full med möten så var lite varsam med kvällsmöten. Uppmuntra phøset att ha arbetskvällar utan dig så kan de informera om vad de gjort i efterhand.

    Diskutera också med phøset huruvida du ska vara med på phadderintervjuer eller ej. I alla fall de tre senaste åren har Øverphøset inte varit med på intervjuerna och så länge inte resten av utskottet tycker annorlunda är det en rekommendation att inte vara med på intervjuerna. Troligtvis kommer resten av utskottet sitta med, och då gör en röst till inte så mycket i valen. Det är också väldigt skönt att få en lite lugnare vecka för din del och slippa sitta otaliga timmar varje dag i en veckas tid med intervjuer, utan istället fixa med småsaker och ladda batterierna. Men valet är ju självklart upp till dig.

    Frykis: En sak som jag gjorde som var en god tanke men som inte höll hela vägen var att boka upp phøset en lååång jävla tid i förväg, dvs. minst 8 veckors framförhållning. Jag bokade in kvällar med NollU-mys, mys med andra phøs (ej bestämt med vilka, bara ett bokat datum för det ändamålet) och kvällsmöten. Min uppfattning var att det inte var populärt med möten, men att det var skönt att det var bra framförhållning så att de kunde planera in andra aktiviteter.

    Frykis: Det finns två mötesprotokollstyper. Beslutprotokoll och diskussionprotokoll. Det förstnämnda körde vi på i phøset, men de andra är ett bättre, men jobbigare alternativ. Vi var hyfsat duktiga på det i början, men det föll snabbt mellan fingrarna och till slut hade vi bara informativa möten där jag sedan antecknade ``privat'' analogt i en bok. Detta gjorde att vissa beslut glömdes av som vi inte antecknade ordentligt i början och att det var svårt att gå tillbaka och se vad som sagts ordentligt.

    Molly: Vi förde ett och samma protokoll för båda delar där vi skrev ner det viktigaste i diskussionen men markerade besluten i t.ex. \textbf{fetstil} så man kunde kolla tillbaka och se vad det viktigaste var. Detta funkade bra för oss, men det bör också tilläggas att förutom under sommaren när man skulle följa upp beslut så var det väldigt sällan man gick tillbaka i anteckningarna, så jag kan inte säga att vår metod var den bästa heller.
    
    Bestäm i gruppen hur ni gör. Antingen är någon mötessekreterare hela tiden (om det finns någon frivillig), annars måste det rotera. Jag tror att det är bättre att en person är sekreterare konstant och det skulle faktiskt kunna vara du som gör det (då du redan håller mycket i infon och kan kopiera in det som meddelats). Att anteckna under möten ger även ett bra helhetsperspektiv och påminnelse om vad som händer. Man märker snabbt om det är någon som pratar väldigt mycket och om någon inte pratar så mycket. Försök att ha en jämn och balanserad diskussion så att ALLA får säga vad de tycker. Ibland kan det vara bra att gå varvet runt och låta alla få säga sitt. Tänk då på att variera vem som börjar och åt vilket håll det går. Det är bättre för gruppen om en individ som sällan säger något får börja. De tänker oftast i bra och intressanta banor. En smart idé kan vara att alltid börja med personen direkt bredvid dig och köra ett varv så att du alltid pratar sist. Detta för att få höra vad alla andra har att säga innan det är din tur, så att din röst inte ska påverka vad andra tycker. En sak som vi gjorde i både ØPK och NollU i diskussioner var att vi använde oss av fingersystemet. Man håller upp ett finger och så blir man näste talare. Finns ett finger redan uppe så håller man uppe två fingrar. När den med ett finger får sin tur så håller man uppe ett finger. Enkelt, du fattar...

    Som Øverphøs så har man ofta ett helt annat perspektiv på allt, vilket kan vara lite frustrerande ibland då det kan bli väldigt mycket fokus på smådetaljer (som troligtvis är helt irrelevanta) eller att man sitter på mycket information från tidigare diskussioner i ØPK eller styrelsen. Jag gillar att framföra min åsikt, men jag försökte även att vara den sista i phøset att uttrycka vad jag tyckte, vilket har sina fördelar \& nackdelar.

    En stor fördel med att var den siste som säger något är att man lyssnar aktivt på diskussionen. Det gör att man är mer uppmärksam på när det börjar gå i cirklar. Då är det streck i debatten eller beslut som gäller. Det är nog det viktigaste inflytandet man har i phøset. Man kan lätt fastna i Stoneface-debatter i timmar efter timmar utan några framgångar. Glöm inte att man längre fram i planeringen kan ta upp ett gammalt beslut och diskutera om det. Vi gick igenom det viktigaste som stone-face osv. som man lätt kan ändra åsikt om efter man testat det både precis innan nollningen och med jämna mellanrum under tiden. Kom ihåg anledningarna till vad ni beslutade senast varje gång så ni inte ångrar något! Ta en dedikerad för/eftermiddag lv.1 och gå igenom några tunga frågor som årets nolla, stoneface, glasögon etc. etc. Det är bra påminnelse för gruppen om vad som sagts och det kan hända att man har bytt åsikt från beslutet ett år tidigare.

    Jag skulle rekommendera att så många som möjligt från phøset är med vid möten med sektionens andra utskotten om nollningen. Man bör inte vara själv från phøset eftersom det lätt kan ske misstolkningar eller att den man har möte med inte riktigt “förstod” vad man gått med på.

    \section{Våga lämna ansvar}
    Lämna ansvaret, detaljarbetet och grovjobbet till phøset. Det är deras uppgift att ta hand om sådant. Givetvis tas viktiga beslut gemensamt, men gräv inte ner dig i vad de håller på med. Håll lite avstånd och ha ett helhetsperspektiv. Delegera och KRÄV noggranna uppdateringar på vad som händer, ett annat alternativ är att vara bollplank åt dem eller korrekturläsa något åt dem om de vill. Var en kratta som går bakom alla och fånga upp saker som annars skulle fallit mellan stolarna. Det är din uppgift att ha en helhetsbild och inte missa någon viktig detalj som man annars kan ha överseende med om man gräver ner sig i något specifikt. Det är strategiskt att ha den rollen som huvudsaklig roll och inte fokusera på specifika sysslor som är bättre att ge till andra.

    En annan sak värd att tänka på är att bara för att du eventuellt var aktiv inom ett visst utskott förra året så betyder inte det att det är just du som måste hålla den kontakten med det utskottet. Det är till och med dumt att resonera så. Låt någon annan ta det, men finns tillgänglig för eventuella frågor de kan ha om det, men det är otroligt att de behöver så mycket hjälp. Hade man inte haft en koppling till utskottet inom phøset så hade det ju löst sig ändå!

    Folk är dumma i huvudet (förlåt för mitt uttryck), men de kommer att kontakta dig angående ALLT och i sista minuten, t.ex. Sångförmän på andra sektioner. Phaddrar är också ofta lata eller stressade och orkar inte gå in och kolla på ställen de vet att informationen finns och tycker det är mycket smidigare att skriva till er i phøset och få informationen direkt. Om ni svarar på det kommer trenden fortsätta och det är allmänt en onödig grej för er att behöva göra ovanpå allt annat. Svara istället ``Det ska finnas i driven/mejlet vi skickade ut senast/sms:et'' så får de leta där det var tänkt att de skulle leta från början.

    \section{Temasläppet}
    Inför temasläppet är det lätt att man lägger hur mycket tid som helst på att fixa allt, inklusive kläder, film, spex osv. Det är ju alltid roligare ju satsigare man gör det men t.ex. kläder, speciellt att ha tryck på baksidan av manteln, var ganska onödigt att fixa till dess. Att ha detaljer är kul, men vissa saker är onödiga och uppskattas inte tillräckligt av folk (eller ens märks) för att vara värt det. Vi var en av få sektioner som hade en stor del av våra outfits färdiga och det var väldigt mycket slit med att få till detta. Sedan är det åter upp till er att känna hur ni ligger till med allt annat, och utefter det välja hur mycket ni vill arbeta med kläder och liknande innan temasläppet. 2018 fick hela phøset en vecka ledig under omtentaveckorna under våren. Då spenderade vi i princip hela den veckan tillsammans och arbetade med våra kläder, så det var helt enkelt därför vi ändå fick så pass mycket färdigt. Sen skulle jag säga att det är otroligt roligt att ändå lägga ner ganska mycket tid på filmen så att man känner sig nöjd med den. Temasläppet är enligt mig det absolut fetaste på hela våren och att ha en bra film höjer verkligen tagget bland er i phøset, men också på hela sektionen.

    \section{Kontakt/umgänge med andra phøs}
    Detta är superkul och det rekommenderas verkligen att hänga mycket med andra phøs! Vi hade en väldigt bra relation till staben eftersom våra phøsrum ligger vägg i vägg och var sjukt tacksamt inför samarbete under nollningen. Inte bara för att de råkade vara i samma hus, men även för att man hade en go-to-grupp att vända sig till när det krisade lite och man behövde lite extrapersoner att få hjälp av (t.ex. om man skulle bära något från E-huset till ett gemensamt evenemang osv). Ni väljer såklart själva hur mycket ni vill jobba med andra grupper och såklart vilken/vilka grupper det är, men att ha något evenemang under våren likt en phøsdejt är kul att ha och för en närmare folk i FHØB utanför de gemensamma evenemangen som finns! Tänk bara på att inte vara uteslutande mot någon grupp om ni nu skulle vara tajta med några olika phøs-grupper.

    \section{Vice ØP}
    \underline{Vi hade detta 2017 och jag tycker det var kanon} \newline
    Att utse ett vice Øverphøs är något ØPK 16 kom fram till skulle tipsas till nästa år då det är en sjukt vettig grej att ha! Inte bara ifall något händer och man inte kan gå på möten, utan att ØPK vet vem de ska kontakta ifall man inte kan nå Øverphøset så det inte blir spretig kommunikation med olika kontaktpersoner varje gång. Värt att tillägga också är ju om ØP:t blir sjukt under nollningen och är borta några dagar, eller under planeringstiden, så är det bra att inte behöva hetsbestämma en person som ska ta över detta, utan att ha en person som är beredd på att hoppa in ifall något händer. Det är tråkigt för phøset också och inte bara Øverphøset om man måste skicka iväg någon på något utan att man varit förberedd på det över huvud taget. Är det någon som är manad så bestäm gärna detta kanske 1–2 månader in i arbetet under våren när man känt av också. Det optimala vore ju om man hade någon från dag 1 men då är risken att personen som tar på sig uppdraget inte har känt av vad de ger sig in på. Känn efter 1-2 månader in vem som är mest taggad och bestäm det på ett möte där alla får vara med och prata.

    \section{Tips \& tricks}
    \begin{itemize}
        \item Varför göra om hela hjulet. Gå igenom föregående års material och förändra det ni vill ha kvar till det bättre!
        \item Boka seriöst in lediga kvällar för dig själv! Första läsperioden har man inte en enda helg ledig från phøsmöten/aktiviteter
        \item Våga avstå från sektionsaktiviteter! Man kan och måste inte gå på allt! 
        \item Det kommer att vara tufft i skolan om du inte ligger i fas!
        \item SÄG IFRÅN! Blir det för mycket så har du cophøsare att be om hjälp av. De kan hoppa in på ett ØPK-möte! Ta välbehövlig vila och tänk på annat än nollningen!
        \item Sätt inte för höga krav på dig själv. Nollningen kommer bli av oavsett
        \item Ta INTE på dig extra ansvar i styrelsen eller ØPK. Om du inte vill, då är det kul! Men se till så att du inte gör det med motivering ``någon måste ju'' för det gäller lika mycket för andra som för dig, och att du faktiskt har både tiden och energin att göra det! Man ska må bra även om det är kul!
        \item Tacka, var ödmjuk och positiv. Phøset, phaddrar och Styrelse jobbar hårt för nollningen. Vad tacksam för all hjälp, alla andra Øverphøs har det inte ett dugg lika bra!
        \item Ha god kontakt med gamla Øverphøs. Jag hade inte det, men jag kommer att kolla läget ibland och finnas till hands alla dagar i veckan! Tveka verkligen inte om ni har en fråga, även om det är något ni anser vara litet! Vi VILL hjälpa till <3 och har haft diskussionen inom vår grupp att det är något vi både tycker är kul och viktigt.
        \item Ta verkligen vara på och njut av året du har framför dig. Det kan vara svårt att göra det i perioder med mycket arbete. Året kommer att gå snabbare än vad du tror, så ha så kul du någonsin bara kan.
        \item Läs igenom testamentet fler gånger under året och sammanfatta kanske till och med för att ha punkter att kolla på då och då.      
    \end{itemize}

    Avslutningsvis så vill jag säga grattis igen och att du kommer att ha ett fantastiskt år framför dig! Taggaaaa nollning 2019!!!!!

    \emph{Vänliga hälsningar}\\
    \emph{Andreas Bennström, Øverphøs 2018}

\end{document}