\documentclass[10pt]{article}
    \usepackage[utf8]{inputenc}
    \usepackage[swedish]{babel}
    
    \def\post{Ekiperingsexpert}
    \def\date{2017-11-22} %YYYY-MM-DD
    \def\docauthor{Henrik Ramström}
    
    \usepackage{../e-testamente}
    \usepackage{../../e-sek}
    
    \begin{document}
    \heading{\doctitle}

    Grattis, du har lyckats bli Ekiperingsexpert. Detta betyder att du ansvarar för inköp samt försäljning utav de produkter som för tillfället ligger i E-shop. Detta innefattar tygmärken, pins, ordensband m.m. Kort och gott är du vår egna lilla student-ICA.
    
    Nödvändig access:
    \begin{dashlist}
        \item Pump (eftersom skåpet med alla ouveraller finns där inne)
        \item Sicrit (för att ha möjlighet att ta sig in i visningsskåpet inne i Edekvata)
        \item Ekea
        \item HK, för att kunna bokföra efter att du är klar med ditt arbete
    \end{dashlist}

    Nycklar: Skåpsnyckel (går till både Blå skåpet i Pump och lilla skåpet i Vega)
    e också till att få inloggning till datorerna i Blå dörren och att du kommer åt Ekiperingsexpertens egen mapp på INFU eftersom alla vektoriserade original bilder till märken, styrelsefrackar och ouveraller finns där.

    Som Ekiperingsexpert så har du ansvar för E-shop, E-sektionens säljavdelning. Vi beställer och säljer märken, ouveraller, pins, nålar, ölsejdlar, ordensband och hoodies. Posten är som mest aktiv under nollningen när vi säljer ouveraller och märken. Det är jättebra att stå på märkesmålningen och märkespicknicken å sälja märken!

    Märken och mindre saker beställs oftast från Premiemax eller Backhausen, sedan 2016 så säljer sektionen alla sina märken för 15kr. Detta för att göra det enkelt för dig och dem som vill köpa samt för att vi inte ska gå back för något av märkena som vi säljer. Vid beställning ska vektoriserad bild, önskad storlek i mm, samt önskat antal finnas med.

    Ouverallerna säljs för 300 kr normalt. På nollningen till nollorna kostar de 350 men då ingår också en liten påse med märken, dessa brukar vara årets Nollemärke, programmets eget märke, krusidull-E och 2 andra märken på senare år så har även phøset velat dela ut identiska t-shirts. Dessa brukar också vara smidiga att dela ut sammtidigt som allt annat. Dessa tygmärken beställs också vanligtvist in av Ekiperingsexperten men ifall phøset är sent så går det även bra att dem beställer in dem istället.\newline
    Ouverallen ska ha E-sektionens logga på ryggen (både insida och utsida sen 2018). 

    Ordensbandet säljs inför nollegasquen eller annan tillställning med högtidsklädsel. Det kostar 50 kr per halvmeter. Herrar får det med en säkerhetsnål i vardera ände. Mannen bär ordensbandet från höger axel till vänster höft om högerhänt och tvärt emot ifall vänsterhänt.

    Bandet ska fästas diskret innanför frackvästen.

    Damer har antingen ett helt ordensband som ligger från vänster axel till höger höft, ett långt ordensband eller (vanligtvist) bär de en särskilld rosett, med nål genom centrum. Rosetten ska fästas över hjärtat (som skydd ifall personen blir skjuten).

    Hoodies beställs in ca 1 gång per termin (efter behov). För att alla hoodies ska gå åt och inte ligga kvar i lagret har vi beställt personliga hoodies med E-sektionens tryck på ryggen och en personlig text på bröstet. Tidigare har hoodies beställts efter att ekiperingsexperten begärt offerter på olika företag för ca 25- 30 hoodies.

    Ölsejdlar har tidgare köpts in av sektionen, och sedan graverats av Oscar Hjerpe, en av sektionens medlemmar. Detta har däremot gått ur bruk och sköts numera av KM då Oscar har slutat och det inte säljs så många sejdlar längre.

    Visningsskåpet i Vega bör hållas uppdaterat med nuvarande märken, nålar, pins och annat som E-shop säljer. För att komma in behövs minst en skruvdragare som lätt kan hittas inne i Sicrit.

    Det är väldigt praktiskt att se till att det finns lite av varje märken och pins i det lilla skåpet nedanför visningsskåpet. Säljarlådan är fantastiskt praktisk, särskillt på nollningen då man ofta står ute och säljer. Den förvaras i Blå Skåpet i Pump eller innanför Blå Dörren i Edekvata.

    \textbf{Vid beställning anges både leveransadress och fakturaadress dessa är:} \newline
    \underline{Leveransadress:} E-sektionen Ref: Ekiperingsexperten Ole Römers väg 3 223 63 Lund \newline
    \underline{Fakturaadress:} E-sektionen Ref: E-shop Box 118 221 00 Lund

    Ett praktiskt sätt att få in offerter, på t.ex hoodies, är att använda bobex.se. Som Ekiperingsexpert har du ett eget konto med inloggning: ekiperingsexpert@esek.lth.se lösenord: oddput som telefon är angivet edekvatas: 046141497
\end{document}
    