\documentclass[10pt]{article}
\usepackage[utf8]{inputenc}
\usepackage[swedish]{babel}

\def\post{Studierådsordförande}
\def\date{2017-11-25} %YYYY-MM-DD
\def\docauthor{Fanny Månefjord}

\usepackage{../e-testamente}
\usepackage{../../e-sek}

\begin{document}
\heading{\doctitle}

Stort grattis till din nya post som Studierådets ordförande! Jag är helt övertygad om att du kommer göra ett strålande jobb, både i Styrelsen och som utskottsordförande. Studierådet har hand om hela Sektionens studiebevakning och det kommer bli ditt uppdrag att fördela uppgifter inom utskottet, att föra Studierådets talan och att försöka rekrytera nya medlemmar. I detta testamente ska jag försöka skriva ner det jag har lärt mig under året för att det ska underlätta för dig och så att du snabbt kan komma in i din nya roll. Jag vill fortsätta vara en aktiv medlem i Studierådet och du är självklart välkommen att höra av dig närhelst du har någon fråga.

Lycka till med ditt nya uppdrag! Jag hoppas att du kommer finna det roligt och lärorikt.

\emph{Vänliga hälsningar}\\
\emph{{\docauthor}, Studierådsordförande 2018}

\newpage

\tableofcontents

\newpage

\section{Möten och råd}

\subsection{Studierådet}
Det är utskottet Studierådet som du kommer vara ordförande för! Det blir då din uppgift att kalla till möten och föra protokoll. Detta blir utskottets främsta informationskanal så det kan vara bra att förmedla information från kåren, LTH och styrelsen här. Inom utskottet finns det flera olika poster och det kan vara bra att veta vad de har för arbetsuppgifter. Det finns även en vice ordförande, vars postbeskrivning säger att denne ska hjälpa utskottsordföranden. Eftersom det inte är solklart vilka arbetsuppgifter som finns så är det bra att ha en dialog och dela upp arbetsuppgifter allt eftersom.

\subsection{Sektionsstyrelsen}
SRE-ordförande sitter som ledamot i styrelsen för E-sektionen. Detta är ett viktigt uppdrag, då det är styrelsen som tar många beslut om hur sektionen ska organiseras. Det är viktigt att diskutera och tänka igenom belsut innan de tas. SRE-ordförande har rösträtt på styrelsemötena. Styrelsen är en bra kanal för att veta vad som händer på sektionen i stort. Det är även bra att vända sig till resten av styrelsen när man är osäker på något, i eller utanför det egna utskottet.

\subsection{SRX}
SRX är ett utskott på TLTH som är för studierådsordförandena från de olika sektionerna. Det är kårens utbildningsansvarig för interna frågor, UI, som kallar till mötena. Här kan man få många tips på hur andra studieråd är uppbyggda. Här ska man redogöra kort för vad som har hänt i ens studieråd sedan sist och UI brukar sedan rapportera om vad som händer på LTH i stort. Detta kan vara bra att föra vidare till SRE.

\subsection{Programledningar}
SRE ska nominera studenter till programledningarna BME och E. Detta görs till kårens utbildningsansvarig för interna frågor. Studenterna som studieråden nominerar blir i princip alltid valda. Dessa personer är sedan funktionärer på kåren och inte automatiskt i SRE. Men det kan vara en bra idé att bjuda in dessa studenter till SRE-möten om de inte redan är med i studierådet, för att få en inblick i vad som händer i programledningarna. Studenterna som är med får delta i möten, får vara med att diskutera utbildningen och är med när besut tas. I E:s programledning finns det plats för tre studentrepresentanter och tre suppleanter, i BME:s så har vi bara en ordinarie och en suppleant. Det är viktigt att ha bra kontakt med progrmaledningarna då de har stor inblick i utbildningarna och kan hjälpa till vid diverse frågor eller problem. De vill också ha kontinuerliga uppdateringar från SRE med vad som händer på sektionen angående studier och miljö och hur studenterna upplever utbildningen.

\subsection{Institutionsstyrelser}
SRE ska även nominera studenter till tre olika institutionsstyrelser, Elektro- och informationsteknik, Biomedicinsk teknik och Datavetenskap. I de två förstnämda har vi en plats för en ordinarie ledamot och en suppleant. I institutionsstyrelsen för Datavetenskap har vi bara en suppleantplats. I dessa styrelser tas det beslut om hur ekonomin ska fördelas inom institutionen och man får även information om vad som händer på institutionen. Studenternas representanter väljs in på samma sätt som programledningarnas representanter.

\section{Evengemang}

\subsection{Studiekvällar}
Studiekvällar är ett tillfälle för att få studenter att stanna längre i skolan och plugga tillsammans. SRE brukar vid dessa tillfällen stå för mat och fika så att folk orkar stanna längre. Detta kan vara ett bra ställe för studenter att hitta studiekamrater! SRE brukar alltid hålla i pluggkvällar under nollningen och det är ett bra sätt att visa upp utskottet för de nya studenterna. Under nollningen brukar programledningen sponsra. Det är då Åsa Vestergren som man ska kontakta och fakturera. Man kan även ha fler pluggkvällar under året om intresset finns.

\subsection{Kårevenemang}
Kåren brukar genomföra en del evenemang som exempelvis Speak Up Days, och det är då studieråden som ska hjälpa till. Det blir ditt ansvar att meddela sektionen om dessa evanemang och att hitta jobbare när det krävs.

\subsection{SRE-workshop}
Detta är ett evangemang under nollningen där de nyantagna studenterna får en chans att lära känna studierådet och får möjlighet att prata om allmänt om studier. Upplägget är väldigt fritt men tidigare år så har vi haft en gemensam start där SRE berättar om vilka de är och Ingrid Holmberg pratar också. Sedan har studenterna delats upp phaddergruppsvis och fått fika och en från SRE har haft en öppen diskussion med sin grupp. Det kan vara bra att förbereda lite frågor innan, men att uppmana till att ha så öppna diskussioner som möjligt. Detta är ett viktigt evanemang för SRE eftersom vi har möjlighet att visa oss och vårt arbete. Vi vill dessutom ha in representanter från de nya klasserna så snabbt som möjligt.

\subsection{Utbytesevenemang}
Utbytesevenemang kan man försöka uppmuntra världsmästarna att hålla i! Exempel kan vara utbytesmingel med studenter som har varit utomlands eller en “Utbyteskväll” som vi höll i år, där man samlades för att hjälpas åt att göra sin ansökan. Här kan man fråga om hjälp från Internationella avdelningen eller Studie och kärriärvägledningen. 

\subsection{Övriga evenemang}
Det är kul att försöka göra så många evenemang som möjligt! SRE har även en stor budget på 10 000 kr som man kan använda hur man vill. Det kan vara en bra idé att göra en internbudget och att meddela resten av utskottet redan från början vad de har att röra sig med. Det kan uppmana fler till att hålla i olika evenemang! Tips på uppskattade saker att hitta på är de som är listade här ovan, men även inspirationsföreläsningar, workshops eller varför inte studiebesök?

\section{CEQ}

CEQ står för \textit{course evaluation questionnaire} och det är systemet som LTH använder för att utvärdera och förbättra alla kurser. Studieråden ska jobba för att få en högre svarsfrekvens på enkäterna och SRE måste även censurera alla textsvar. Detta får vi arvodering för från LTH. Själva censureringen kan med fördel delas upp inom utskottet (till årskursansvariga) men det är viktigt för dig som utskottsordförande att ha koll på att alla CEQ:er censureras inom tidsramen och det kan även vara bra att veta att det är du som är främst ansvarig för arbetet. De flesta enkäterna görs på nätet och då ska censureringen ske på hemsidan ceq.lth.se. En del kursansvariga väljer att istället ha pappers-CEQ:er. När dessa ska censureras ska dessa hämtas hos utbildningsservice på 5:e våningen i E-huset hos Minna Kokko.

Efter enkäterna har kommit in ska dessa utvärderas. I grundblocket eftersträvar vi att alla kurser ska utvärderas med ett möte mellan studenter, kursansvarig och en representant från programledningen. Om studierådet anser att det behövs ett möte i en kurs, under eller efter kursens slut, kan programledningar hjälpa till med detta. På mötet ska studentrepresentanten framföra studenternas åsikter med hjälp av svaren från CEQ:erna. Efter mötet ska det skrivas kommentarer på CEQ-hemsidan.

Som utskottsordförande ska du samordna arbetet med CEQ-enkäterna. Du kommer få mail när nya enkäter har kommit in eller när det är möten. Du ska se till att informationen kommer fram till berörda personer. 

\section{Kontakter}
Det finns en del kontakter som kan vara bra att ha koll på. Om det dyker upp några problem i utskotten är det bra att kunna hänvisa dem till rätt person. Det kan även vara bra att veta vem man kan fråga när du stöter på frågor.

Kåren har ett utbildningsutskott med två huvudansvariga. Marcus Bäcklund är huvudansvarig för utbildningsfrågor och har mailen \href{mailto:uh@tlth.se}{uh@tlth.se}. Philip Johansson är utbildningsansvarig för interna frågor och kan nås på mailen \href{mailto:ui@tlth.se}{ui@tlth.se}. Philip är även utskottsordförande för SRX så ni kommer ha lite tätare kontakt. SRX och kåren har även kanaler via Slack och det är så kåren brukar sprida information som ska ut till studieråden.

På utbildningsservice på femte våningen i E-huset sitter en del personer som du kommer ha kontakt med. Åsa Vestergren är programledare för både E och BME och är den som kallar till programledningsmöten. Henne når du på \href{mailto:asa.vestergren@kansli.lth.se}{asa.vestergren@kansli.lth.se}. Ingrid Holmberg är studie- och karriärvägledare för E och BME och hon kommer du att samarbeta med en del, särskilt under nollningen. Hennes mail är \href{mailto:ingrid.holmberg@lth.lu.se}{ingrid.holmberg@lth.lu.se}. Ingrid Svensson(\href{mailto:ingrid.svensson@bme.lth.se}{ingrid.svensson@bme.lth.se}) är programledare för BME och Frida Sandberg (\href{mailto:frida.sandberg@bme.lth.se}{frida.sandberg@bme.lth.se}) är biträdande programledare och har hand om att CEQ-möten. Lars Wallman (\href{mailto:lars.wallman@bme.lth.se}{lars.wallman@bme.lth.se}) är programledare för E men nu när han är föräldraledig tar Markus Törmänen (\href{mailto:markus.tormanen@eit.lth.se}{markus.tormanen@eit.lth.se}) över. Biträdande programledare för E är Andreas Lenshof och han nås på \href{mailto:andreas.lenshof@bme.lth.se}{andreas.lenshof@bme.lth.se}.

Andra bra personer som du alltid kan fråga om hjälp är de andra som är studierådsordförande för de olika sektionerna. Du kan även såklart vända dig till gamla SRE-ordförande om det är en fråga som mer rör E-sektionen. De som fortfarande går kvar på skolan är jag, Fanny Månefjord, Edvard Carlsson och Pontus Landgren. Det är även bra att hålla en tät kontakt med resten av styrelsen.

\section{Bra att känna till/dokument}
\begin{dashlist}
    \item Rättighetslistan från LU listar alla rättigheter som studenterna har. 
    \item Handboken för utbildningsbevakare från kåren är en bra sammanfattning över jobbet som studieråden gör.
    \item Kåren tillhandahåller utbildningar såsom UUU (utbildningsutskottets utbildning för utbildningsbevakare) och andra utbildningar för skyddsombud.
    \item Kursplaner ska finnas för alla kurser och där ska allt som kursen innehåller stå, inklusive obligatoriska moment.
\end{dashlist}
 
\end{document}