\documentclass[10pt]{article}
    \usepackage[utf8]{inputenc}
    \usepackage[swedish]{babel}
    
    \def\post{Entertainer}
    \def\date{2018-11-23} %YYYY-MM-DD
    \def\docauthor{Adam Belfrage}
    
    \usepackage{../e-testamente}
    \usepackage{../../e-sek}
    
    \begin{document}
    \heading{\doctitle}

    Hej! Du har blivit vald till {\post} på E-sektionen och har ett år av massa kul framför dig. Om du lägger in rätt mängd av arbete på rätt sätt det vill säga. Jag kommer finnas här för dig och ge dig tips och råd för att du ska klara av din post så bra som möjligt.
    
    \newpage

    \tableofcontents
    \newpage

    \section{Saker att fortsätta med och saker att göra bättre}
    Entertainerns roll behöver balanseras noggrant.  Styr du i för få events kommer folk märka av det och styr du i många event kommer uppslutningen kanske inte alltid bli så bra eller tacksam. Men vad jag vill säga till dig är att: Du ska alltid se till att det finns saker att gå på. Ditt jobb är inte att tvinga dit folk utan folk ska kunna dyka upp om de vill och känner att de vill umgås med andra personer från sektionen. Kommer bara två eller tre personer till ett event plus ditt utskott så har du ändå gjort något för de tre personerna. Och det ska man absolut inte ta för lätt på. Därför bidrar du alltid med glädje på sektionen vilket är väldigt tacksamt.

    Vad jag lyckades lite sämre  med var marknadsföringen av våra event. Jag blev bättre med tiden och mitt tips är att använda instagram mer än E-sek events och att använda TVn utanför LED. Skapa event i E-sek events och bjud aktivt in alla medlemmar. Det genererar helt ok spridning. Men i sanningens namn är våra informationskanaler generellt sett lite för dåliga. Men sprid i klassgrupper, esek events, instagram och ibland kan du faktiskt stå i Edekvata på lunchen och informera. Det var inget jag gjorde men önskar nog att jag hade gjort det ibland. Numera kan vi också använda oss av bonsai vilket är jättebra. Använd det för att sprida notiser men lägg upp huvuddelen av informationen på hemsidan och e-sek events. Det vill säga att du hänvisar till E-sek events i bonsai exempelvis.

    Något jag också lyckades lite sådär med var att skapa fler intersektionella event, mitt tips är att utnyttja kollegiet väl och kontakta andra aktivitetssamordnare direkt. Kan vara bra att vidga vyerna till sektioner som vi inte har särskilt mycket samarbete med, exempelvis V eller W. Du kommer ha mer tid på våren än vad jag hade och mer fria datum så därför blir det lättare att få ihop något sådant event.

    Min främsta bedrift under mitt år var att jag fick ihop ett väldigt tajt utskott. Det berodde kanske på de som sökte men också för att jag var väldigt driven på att integrera de flesta posterna mer i arbetet. Dessutom såg jag till att vi hade flera häng utanför skoltid där vi gjorde lkar och annat kul som förde utskottet näramare. Vi hade möten ca 1-2 ggr i månaden i början för att få ihop utskottet men det blev mindre möten under höstterminen av förklarliga skäl. Dels på grund av nollningen men också på grund av att tentorna kom så snabbt efter denna också. På mötena diskuterar man om vad som händer och vad vi ville göra och vilka som skulle göra det. Fortsätt med det!  Var inte rädd för att delegera! 

    Det jag också lyckades bra med var diversiteten och mängden evenemang som jag drev under mitt år. NöjU syntes mest och hördes mest vilket jag är otroligt stolt över. Vi såg till att ha fler och bättre evenemang än någonsin. Jag är en förespråkare för kvalitet över kvantitet men det ligger en viss sanning i att ju fler event du gör desto större chans har du på att genomföra lyckade event. Det är dock viktigt att du organiserar alla väl och försöker göra dem så bra som möjligt. Känner du inte att klarar av att göra flera välorganiserade event är det bättre att du gör ett fåtal och satsar på dem. Så det är bra med många aktiviteter och evenemang men se verkligen till att du ger alla en värdig chans och att alla är välplanerade.

    \newpage
    
    \section{Budget}
    Din budget för i år har höjts med ca \SI{2000}{kr} för eventverksamhet och du har ca \SI{8000}{kr} att använda och detta \textbf{SKA} användas. Det innebär att du kan subventionera biljetter för att göra det billigare för sektionens medlemmar. Jag subventionerade exempelvis Ölresan med 5800 kr. Detta var något mycket skulle jag vilja påstå men jag skulle gärna sett att man subventionerar 3 större event med \SI{2000}{kr} vardera så att fler kommer vilja att gå. Då har du också \SI{2000}{kr} över till inköp av saker till evenemang. \newline
    Budgeten för hyrning av Victoriastadion är satt till 6500 kr vilket ger idrottsförmännen ca 500 kr över till uppdatering av sportsortimenetet men det är redan ganska väl uppdaterat. Det finns 1000 kr till inköp av spel och underhåll av biljardbordet. Tror inte att fler spel behövs köpas in men man vet ju aldrig. \newline
    Sångarstriden har ca 7000 kr budgeterat varav 3000 kr ska gå till en tackphest till de som deltar för att öka intresset. \newline
    Tandem har 500 kr för att fixa till cyklarna. \newline
    UtEDischot är budgeterat till att dra in ca 25 000 kr för att kompensera för alla utgifter. Vi fick in 15 000kr mitt år men tror att man hade kunnat få ut mer om man exempelvis inte hade haft artist vilket hade ökat alkoholförsäljningen då inte majoriteten hade varit i tältet och väntat på att artisten skulle komma på.

    Håll bättre koll på budgeten än vad jag hade. Jag var lite för ungefärlig när jag höll koll på budgeten. Jag tror att det är tacksamt om du har en noggrann bokföring av din budget så du kan planera dina event bättre.

    \section{Lokaler}
    FVU förvaltar sektionens lokaler men det kan ändå vara bra att ha viss koll på de lokaler som NöjU utnyttjar ofta.
    \begin{itemize}
        \item \textbf{EKEA} \\
        Här har vi alla våra roliga grejer och kan vara bra att slå vakt om. Lånar någon någonting från oss måste de lämna tillbaka det och gör dem inte det så får de helt enkelt köpa en ny av den sak de lånat. Organisera upp NöjUs hylla inne i EKEA. Jag gjorde det en gång men det blev snabbt kaos igen. Det kan vara skönt att ha gjort så att du vet var saker finns när folk kommer och säger ``Det finns inte i Ekea''. Det tar också max en timme att göra ordentligt.
        \item \textbf{Biljard} \\
        Här har vi två skåp. Biljardskåpet och det blå skåpet. I biljardskåpet finns biljardköer och dartpilar och i innerskåpet finns lite underhållsmaterial till sådant. Här kan kanske saker som du är väldigt rädd om förvaras.I blå skåpet finns våra grejer till pingis, guitar hero och vårt PS3. Slå särskilt vakt om våra pingisbollar då de ofta används i andra syften än vad de ska användas till. Det är okej att de används till sådana syften om de återförs till skåpet igen eller om du är med och har koll på dem. Det står alltid för mycket skräp i biljard så det kan vara bra att vara på hustomtarna och vice FVU så att biljard töms kontinuerligt på skräp som inte hör hemma där. Jag har länge försökt få bort E6 skärmar som de ställer där som bara är i vägen och är väldigt fula. Min tanke är att sektionen ska driva ett projekt i att renovera biljard över nästa sommar. Då är min tanke att vi ska köpa in ett nytt biljardbord och ny darttavla och nya biljardköer också.
    \end{itemize}

    \section{Postöversikt}
    \begin{itemize}
        \item \textbf{Vicena} ska vara dina stöttepelare och de bör avlasta dig med minst 2 större evenemang som du vill driva. I år kommer du ha tandem, något jag inte hade under mitt år men som vicena traditionsenligt alltid håller i.
        \item \textbf{Fritidsledare} bör hålla i mer regelbundna event och hjälpa till under de större. Det vore lämpligt att låta dina fritidsledare ha hand om spelkvällarna. (Ta fram spelen, sätt upp pingisbordet och köp kakor).\newline Spelkvällar hade vi varannan vecka under våren och 3 gånger under hösten.
        \item \textbf{idrottsförmännen} bör hålla i sporta med E varje vecka men har också fria tyglar till att komma på andra idrottsrelaterade event om de vill. Se till att de pratar med dig först dock. Mina anordnade exempelvis en klätterkväll för de 16 personer som anmälde sig först på ett klättercenter.
        \item \textbf{Umphmeistersen} Ska spela musik på våra större events. Men grejen är att det finns många events som inte riktigt behöver det. Jag lät dem spela på släpp som vårt sex höll i istället och såg till att de uppdaterade sektionens spotifylista så att alla hade en förfestlista som de kunde luta sig på. De ska också spela på UtEDishcot.
        \item \textbf{UtEDischoansvarig} Förväntas att planera och genomföra UtEDischot. ANVÄND DENNA. Det är en post jag skapade inför nästa år så jag insåg att jag hade alldeles för lite tid till att planera UtEDischot och för att Saga ville ta sig an den rollen. Du kommer inte ha tid till att planera så mycket som du kanske vill och därför ska du delegera ut mycket av arbetet med UD till denne. Däremot ska du alltid ha koll på processen och se till att allt går smidigt till och att allt blir gjort. (Du kommer få ett separat testamente till det.)
        \item \textbf{Stridsrop} Ansvarar för sångarstriden men fungerar som en vanlig funktionär på de övriga eventen.
        \item \textbf{Øverbanan} Ansvarar för E-sektionens husband och har ganska så fria tyglar. Kan fungera som övrig funktionär men bör i huvudsak se till att bandet har regelbundna repetitioner så de kan spela på sittningar och gillen.
    \end{itemize}

    \section{Förslag på event}
    Här är en lista på events jag \textbf{TYCKER} du bör driva:

    \begin{itemize}
        \item \textbf{DÖMD Januari-April} \newline
        En pilkastningsturnering i Linköping som är det roligaste jag gjort. Kontakta D-group tidigt och säg att ni är intresserade. Gör intresseanmälan tidigt så du vet hur många biljetter du ska ansöka om. Ansök om ett rimligt antal ( ca 15) när de släpper U-lagsanmälan. \newline
        Kom ihåg att intresseanmälan oftast är ganska mycket högre än hur många som faktiskt vill när det väl ska köpas. Biljetterna kostar ca 1250 sen kostar tågresan ca 500. \newline
        Så fort du fått antalet biljetter gå ut med en bindande anmälan. Välj ut lagmedlemmarna baserat på deras motiveringar. Så fort laget är klart kör du en DÖMDinitiering några dagar efter. Lagmedlemmarna bör köra några lekar runt om i E-huset för att visa sin värdighet. Står om det på wikin. \newline
        Jag gjorde en massa uppdrag åt dem och bjöd också in före detta deltagare i DÖMD-laget som hedersmedlemmar. Så vore lämpligt om du bjöd in förra årets DÖMD-lag exempelvis hehe. 

        Jag skrev typ ``Hej du har blivit uttagen till årets DÖMD-lag, men ännu är du inte färdig. Du måste bevisa dig värdig denna resan. Var utanför edekvata kl 21, kom törstig och kom mätt. Klädkod: Teknologössa och Ouvve.'' \newline
        De fick sedan göra ett uppdrag utanför Edkevata och när de var klara fick de gå fram genom en gång av levande ljus. När de kom fram till dartavlan stod där hedersmedlemmarna och väntade. Jag läste upp något skit jag hittat på och sedan fick var och en kasta en pil för att se vem som är bäst. Den som kastade bäst blev första pilkastare. (Man tävlar bara 3 åt gången). Sedan börjar de roliga och så kör du några enkla tävlingar runt om i huset. Jag körde i Ekea, grupprum och en 33-sal. (Skulle kunna boka FikaFika). \newline
        Tävlingen går i april men allt som ska vara gjort innan börjar i januari. \newline
        Rekommenderar detta starkt. Lätt det roligaste jag gjort.

        \textbf{VIKTIGT}: Får du inga biljetter, bara maila dem. Jag mailade dem typ 3 ggr och fick till slut 10 biljeter för att jag sa att vi kunde lösa boende åt 3 av oss. Boendet är problemet.
        
        \item \textbf{Årliga biljardturneringen} \newline Jag körde en långtgående biljardturnering där jag hade 16 lag a 2 spelare. Körde två matcher per gille och avslutade det hela med en finaldag där semifinalerna och finalen spelades med livekommentering. Vinnaren får en biljardkö de kan skriva sitt namn på.
        \item \textbf{Årliga bowlingturneringen} \newline Bokade bollhusets bowlinghall där vi körde en bowlingturnering. Ska enligt wikin hållas på en onsdag men vi körde det på en fredag innan ett gille. Mina vices har mer koll men vi startade först en anmälan där man fick anmäla 4 eller 3 i ett lag och bokade sedan efter det. Vi ansåg att vi kunde boka om vi fick in 40 spelare och vi fick in 45. Det finns plats för 48. Boka alla banor och går vi med förlust gör det inte jättemycket. Budgeten är till för att spenderas på event.
        \item \textbf{DrEamHackE} \newline Ett LAN vi höll i under våren. Blev sjukt lyckat. Du kommer få dokumentation specifikt över det.
        \item \textbf{TandEm} \newline Hålls under vintern, dina vice bör ha hand om det. Mitt år blev det inställt pga Karneval.
        \item \textbf{Märkespåtagning} \newline Låt sektionen limma sin ouvvar tillsammans. NöjU håller i diverse lekar. Vi använde detta för att testa olika lekar innan nollningen.
        \item \textbf{E-spark} \newline Höll i det i den sista tentaveckan på T2 och under första tentaperioden efter nollningen i år. Blev spännande. Regler finns på wikin.
        \item \textbf{Paintball} \newline Fick ställa in detta :( Men kör det. Det finns ett stort intresse över campus så mitt förslag är att köra det med hela kollegiet. Lyckas inte det så kan du dra i det själv och då kan jag hjälpa dig mer. Men då bör du sälja biljetter i varje sektionshus för att få bäst spridning.
        \item \textbf{Festivalborg} \newline En bra ide som blev sådär förra året. Valborgsveckan kör man en festival på amfiteatern där man grillar och säljer öl och har liveband. Man skulle också kunna köra lekar. Viktigt att köra BYOB då. Tror att det kan bli bra men det är kollegiet som drar i det.
        \item \textbf{Agent 00E} \newline Som en stor kurragömma runt om i Lund. Är sjukt kul. Kördes under nollningen och är väldigt enkelt. Vill man köra det under våren kan man köra det på dagen och sedan uppslutning på ett gille. Man ger ut ett dokument med olika uppdrag att utföra och man får under tiden inte bli fotad av ett annat agentsällskap. Behövs inga bokningar eller utgifter, är ett tacksamt event. Vill det utföras kan du få tillgång till vårt häfte och således få en bra bas att jobba med.
        \item \textbf{Ølresa} \newline Åker till ett bryggeri i Danmark, kontakta dessa i god tid, och smakar på deras öl. Bussar kan man boka ganska sent via Bergkvara. Men gör det gärna två-tre veckor innan. Gör först en intresseanmälan och sedan en bindande så att du inte ligger ute med så mycket pengar och inte får igen det igen. När intresset verkar stort kan du börja kolla upp saker och boka, sedan släpper du en bindande och låter det vara först till kvarn. Kan vara kul att köra med en annan sektion kanske? Braunstein bryggeri är ett najs bryggeri. Hur kul som helst. Rekommenderar att åka dit igen. Tänk på att ha en plats fri till högtalare.
        \item \textbf{På Styret} \newline Som på på spåret men vi cyklade runt om i Lund istället med vår Go-Pro. Kördes under ett gille. Blev väldigt uppskattat. Två vices och två fritidsledare tog hand om detta och de skötte det med bravur och ledde sedan “programmet” i diplomat. (Finns en projektor i HK som man kan koppla upp sin dator till som sedan projicerar ut bilden på skärmen i diplomat).
        \item \textbf{Förfest} \newline Håll i en förfest för sektionen med gött häng och lite chill lekar innan en sittning. Blir kul och är enkelt. Perfekt tillfälle att dra fram NöjU-bonken.
        \item \textbf{Bastukvällar} \newline Håll en i månaden, höll i för få sådana. Det är enkelt. Bara att gå upp till bastun och sätt igång musik och låt folk helt enkelt bara ha det gött. Ett tacksamt event.
        \item \textbf{Spelkvällar} \newline Tisdagen varannan vecka.
        \item \textbf{Sporta med E} \newline Varje söndag kl 15.
    \end{itemize}

    \section{Obligatoriska event}
    \begin{itemize}
        \item \textbf{UtEDischot: Här får du ett separat testamente}.
        \item \textbf{Sångarstriden} \newline
        Denna spextävling hålls varje år av kåren kring november/december. Detta bör börjas planeras av stridsropet redan kring sommaren och under nollningen så att man kan dra igång med träningen efter de första tentorna så att man kan få till ett värdigt deltagande i SåS. Se till att stridsropet tar sitt ansvar. Var mer på denne än vad jag var och se verkligen till att det blir av.

    \end{itemize}

    Märk att de flesta är events jag tycker du bör driva och inget du måste göra om det är så du känner att du inte vill eller kommer på något bättre. Det är det som är najs med den här posten. Du är fri till att driva vilka event du vill. Men jag tycker däremot att du bör hålla vissa traditioner vid liv som bowlingturneringen, biljarden och dreamhacke. Sen tänk på att du har en höst också men att tiden är ganska knapp efter nollning och tentor sen valmötet. \newline
    Efter valmötet planerar du in entertainerveckan likt jag gjorde för dig. När den är klar är du i princip klar med din post. Så jag körde bara de mindre eventen under hösten och ölresan under LP2. Så det blir mycket att klämma in under våren. Men du kommer inte ha en karneval i vägen och genomför du bara några av de ovanstående så kommer det bli hur bra som helst. Det är dock väldigt roligt att hålla i sköna häng som Märkespåtagningen och förfester. Ta fram lite chill lekar och sen är det inte så mycket jobb. Det brukar också vara uppskattat.

    \section{Bokningar}
    \begin{dashlist}
        \item Boka Lophtet till tackphest för UD
        \item Boka Victoriastadion för ``Sporta med E''
        \item Boka bowlinghallen (bollhuset i Lund) 1 timme för ``Den årliga Bowlingturneringen''
        \item Boka FikaFika för Entertainerveckan (de e chill)
        \item Boka parkeringen för UD
    \end{dashlist}

    \section{Styrelsearbetet}
    En viktig sak är att delta aktivt i styrelsearbetet för sketionens skull. Delta aktivt i besluten som tas och försök att bli insatt i problemen som sektionen står inför för att kunna delta aktivt i diskussionerna och hjälpa till att leda vägen mot rätt beslut. Styrelsen går alltid först och det är viktigt att man förstår sin roll som styrelseledamot. Du är utskottsordförande och kan också genom styrelsen slåss för NöjUs skull och försöka få igenom förändringar som kan gynna utskottet men också som gynnar sektionen i sin helhet. Jag ångrade att jag blev insatt för sent och jag hade tjänat på att bli insatt tidigare. Det är både lärorikt och kul. Det är också viktigt att ha en god relation med styrelsen för att kunna diskutera med dessa när saker går fel och för att kunna få respons kring idér du har. Du är ju glädjespridaren i gruppen, och det ska man inte förringa. Jag tror att entertainern spelar stor roll för gruppdynamiken och den dynamiken är ack så viktig och gör hela året roligare.

    \section{Övrigt}
    I övrigt kommer jag att finnas med dig och det hela gamla nöjesutskottet kommer kunna svara när du har frågor om specifika evenemang som anordnats tidigare. Det är bättre att fråga om du är osäker över något istället för att bara köra på. Vi kommer finnas här för dig. Det kan lätt vara så att man känner sig lite vilse när man dels är ny på skolan och dels ny på sin post. Så tänk alltid på att fråga om det är något du undrar över. Det kan vara bra att hänvisa de övriga i utskottet, om de har frågor, till sina företrädare så slipper du ställa en fråga till mig som egentligen borde besvaras av de tidigare idrottsförmännen. \newline
    Jag hoppas att du gör mig stolt och NöjU18 rättvisa. Men det har jag inga tvivel om att du kommer göra.

    \textit{Med vänliga hälsningar}\newline
    \textit{Adam Belfrage, Entertainer 2018}


    \end{document}