\documentclass[10pt]{article}
    \usepackage[utf8]{inputenc}
    \usepackage[swedish]{babel}
    
    \def\post{Ordförande Electus}
    \def\date{2017-11-23} %YYYY-MM-DD
    \def\docauthor{Daniel Bakic}
    
    \usepackage{../e-testamente}
    \usepackage{../../e-sek}
    
    \begin{document}
    \heading{\doctitle}
    
    Hej Ordförande Electus!
    
    Grattis till ditt nya uppdrag som Ordförande för den bästa sektionen inom TLTH! Jag önskar dig stort lycka till under ditt kommande år och hoppas att du kommer känna att det är ett riktigt roligt och givande uppdrag!
    
    Du som nyss blivit vald till Ordförande på E-sektionen har lite framför dig i jobbet som Ordförande. En del av det behöver göras redan innan du går på posten officiellt! Allt nedan behöver inte göras innan du går på och det kommer i lite halvologisk ordning så läs igenom allt. \texttt{:)}
    
    \begin{numplist}
    \item Glöm inte att det är styrelsen electus ansvar att städa efter Valmötet!
    
    \item Först och främst så gäller det att spika ett datum för KPL (Kurs På Landet, eller i folkmun Kristna PulkaLägret) som är ett gemensamt styrelseskiphte för nya och gamla styret. Datumet för KPL kan vara lite lurigt att få till då det vanligtvis brukar vara i en stuga som man hyr från fredag eftermiddag till söndag eftermiddag. Ett förslag på datum är den 1-3:e februari. Tänk på att det är viktigast att så många i den nya styrelsen kan komma för det är verkligen teambuilding! Det brukar vara hemligt för den gamla styrelsen vart KPL ska äga rum, de brukar följa en skattkarta eller dylikt som Entertainern har gjort och kommer dit på lördagen.
    
    \item Även Skiphtet behöver ett datum vilket också är bra att bestämma så tidigt som möjligt, framför allt om man vill vara på Lophtet då det snabbt kan bli uppbokat. I år hade vi Skiphtet den 16:e februari. Försök lägga det så tidigt som möjligt! Hur ni vill lägga upp Skiphtet är naturligtvis helt upp till er men tänk på att meddela alla nya och gamla funktionärer så tidigt som möjligt så att de kan boka upp datumet. Dock får vi inte ha BYOB sittningar då det strider mot alkohollagen. Vilket innebär att ni inte kommer kunna ha samma upplägg som man haft tidigare års Skiphten. Detta beror på tillståndsenhetens nya tolkning av alkohollagen som lyfts under hösten. Kort sammanfattat så blir en lokal som yrkesmässigt tillhandahålles av en organisation ett serveringsställe och då måste man ha serveringstillstånd för att få lov att förtära/förvara alkohol i lokalen.
    
    \item Det skall tas ett (juligt) julkort som ska skickas till en massa människor. Så boka en tid när alla kan, kom fram till en rolig idé, och genomför den. Julkortet ska skickas ut så mottagarna får det innan julledigheten börjar. Listan med alla som fick förra året finns i årets styrelsedrive under ``Övrigt''. Kolla också gärna med Isabella vilka företag Sektionen har haft kontakt med under året.
    
    \item Styrelsen vill ha frackar. Tagga styrelsemedlemmarna på detta och se till att de skaffar sig dem innan jul! Detta är bra för då kan man skicka dem till tryckeriet/brodyr innan jul och på så vis få dem innan skiphtena. Föregående år har man lämnat frackarna till Brodyrteam (\texttt{brodyrteam@swipnet.se}). Det har fungerat väldigt bra och de har alla mallar som behövs redan. I år handlade vi dem på Ateljé Helene i Rydebäck. Ring henne några dagar innan ni tänkt åka dit så hon kan hämta fler frackar tills ni kommer dit. Det blev ganska dyrt 2018 så man hade alternativt kunat försöka få tag i frackarna på andra håll (second hand). Det skiljer sig dock från år till år. Styrelsens budget för för kläder är ganska hög så man hade även kunnat subventionera en del av frackkostnaden. Brodyren står sektionen alltid för.
    
    \item Glöm inte bort att alla som definieras Vice har rätt till Vice-kavaj. Ta gärna med era Vice när ni köper era frackar och skicka sedan iväg allt på brodyr. Sektionen står för kostnaden av brodyren för kavajerna också. Om ni väljer att gå till Ateljé Helene, se till att ringa i förväg för kavajerna också.

    \item Bestäm vad ni vill ha för profilkläder och beställ dem så snart som möjligt så att de kan nyttjas maximalt.
    
    \item Du samt din styrelse är mer än välkomna på alla resterande styrelsemöten i år. Särskilt ska ni medverka på terminens sista styrelsemöte där massa spännande händer! Datumet är onsdagen den 19:e december 12.10 i E:1124. Meddela nya styrelsen! Be Kontaktor Electus att göra en ny maillista för styrelsen 2019 och att lägga till den i kallelselistan så ni får kallelserna.
    
    \item Gör en ny Google Drive, Slackkanal, Facebookchatt, m.m.. Diskutera tidigt i styrelsen vad som gäller för driven och de olika kommunikationskanalerna! I och med den ``nya'' lagen om GDPR har vi införskaffat en TeamDrive till sektionen som fungerar precis som en vanlig Google Drive med skillnaden att det är vi som äger allt innehåll. Därmed är det GDPR anpassat. Du kommer kunna läsa mer om hur vi har arbetat med GDPR i Styrelsetestamentet och skulle det finnas ytterliggare funderingar är det bara att höra av sig.
    
    \item Prata med nya styrelsen om val av e.a.-poster så att det kan ske så snabbt som möjligt!
    
    \item Kåren kommer att ha kollegieskiphten helgen den 8-10:e februari. Förmodligen har OK skiphte på lördagen, missa inte det! Lägg inga andra Skiphten den helgen!
    
    \item Datum för alkoholutbildningarna är enligt följande: \newline
    C -cert
    \begin{dashlist}
        \item 14:e januari heldag
        \item 16:e januari heldag
    \end{dashlist}
    A-cert
    \begin{dashlist}
        \item 22:a januari kl 17-22
        \item 29:e januari kl 17-22 
    \end{dashlist}
    B-cert
    \begin{dashlist}
        \item 17:e februari heldag
    \end{dashlist}  
     Minst en av (men gärna båda) firmatecknarna behöver C-cert, så se till att planera in det.

    \item I skrivande stund har jag inte fått någon information angående datum för styrelseutbildning. Men jag återkommer så fort jag vet något.
    
    \item Tisdagen den 14:e december kl 16:00 ska du följa med mig på OK-möte!
    
    \item Prata med din nya styrelse om ni är intresserade av överlämningsmiddag i veckan innan jul.
    
    \item Det är även din uppgift att införskaffa en julgran till Sektionen innan julgillet. Att betala för den är överskattat, men man gör ju som man vill.
    
    \item Sist men inte minst - lär dig Taggig Blomma innan Julgillet och glöm inte din sångbok hemma!
    
    \end{numplist}
    
    Du kommer att få ett rejält testamente snart men det här ska vara det allra mesta som du behöver tänka på nu innan julledigheten. Jag håller dig naturligtvis uppdaterad om jag kommer på något mer. Någon dag ska vi även ta och presentera dig för lite folk i huset. Mina nycklar får du i slutet av året så du kan kan komma in i kassaskåpet och så. Du och Förvaltningschefen kommer även att behöva gå till banken efter nyår men det pratar vi om tids nog.
    
    Lycka till! 2019 kommer bli ett fantastiskt år!
    
    \emph{Daniel Bakic \newline Ordförande 2018}
    
    \end{document}
    