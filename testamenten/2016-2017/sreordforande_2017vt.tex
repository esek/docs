\documentclass[10pt]{article}
\usepackage[utf8]{inputenc}
\usepackage[swedish]{babel}

\def\post{SRE-ordförande}
\def\date{2017-05-24} %YYYY-MM-DD
\def\docauthor{Pontus Landgren}

\usepackage{../e-testamente}
\usepackage{../../e-sek}

\begin{document}
\heading{\doctitle}

Hej SRE-ordförande Electus!

Grattis till ditt nya uppdrag, jag önskar dig stort lycka till under kommande halvår och hoppas att du kommer känna att det ett riktigt roligt och givande uppdrag.

Uppdraget som SRE-ordförande innebär att du är ansvarig för i princip allt studierelaterat på E-sektionen. I och med att jag nu lämnar min post efter halva året kommer det bli så att jag har påbörjat en del som du kanske ``tvingas'' fortsätta med. Jag hoppas att det ska gå bra och vill poängtera att du \textbf{alltid} är välkommen att ställa frågor. Jag kommer att hålla på med studiebevakning och kommer fortsatt att ha bra koll på den biten.

Jag bör kanske även nämna att läsåret 17/18 är det jag som sitter som UI, posten som jag refererar till på flera ställen nedan. Det är personen som har samordningsansvar för alla studieråd på kåren så om du har frågor till kåren är det också mig du bör kontakta i första hand.

Här kommer en liten kort lista på saker man gör i rollen som SRE-ordförande.
\begin{dashlist}
    \item Att vara en god styrelseledamot - d.v.s. att engagera sig i styrelsearbete och komma med värdefulla idéer och tankar.
    \item Leda och sammankalla studierådet - har kört möte ca var 3:e läsvecka och upplever att det fungerar bra.
    \item Se till att det finns representanter i programledning E och BME samt i institutionsstyrelse (aktuella styrelser finns i driven)
    \item Deltaga i SRX (studierådsordförandekollegiet) på kåren - sammankallas av Utbildningsansvarig för interna frågor på kåren, möte ca varannan läsvecka.
    \item Organisera och samordna evenemang som studierådet vill/bör/ska hålla i - pluggkvällar, SRE-workshop etc.
    \item Ansvarar för ekonomin som faller under studierådet.
    \item \textbf{CEQ} - granskning, utvärdering och möten. SRE:s heliga uppdrag! 
   
\end{dashlist}



\begin{itshape}
Vänliga hälsningar\\
Pontus Landgren, SRE-ordförande VT2017\\
\end{itshape}

\newpage
\section{Styrelsen}
I styrelsen har du rollen som ledamot det innebär att du har rött att företräda din åsikt (röst-, yrkande-, yttrande- och närvarorätt). I styrelsen behandlas mycket som rör sektionen, mitt tips är att försöka hålla koll på de stora frågorna och sedan ditt egna utskott. 

Styrelsen är en bra informationskanal för att hålla koll på sektionen i allmänhet och man kan ventilera tankar och idéer om man känner sig osäker eller anser att det rör mer än det egna utskottet. 

\section{Utskottet}
I ditt utskott finns flertalet olika poster inom varierande områden. Ett tips är att försöka ha en liten övergripande blick om vad folk håller på med så att det går att upptäcka om någon är hårt belastad eller för den del saknar något att göra. På styrelsemöten ger man en kort redogörelse för vad utskottet håller på med just nu och då är det nice om man själv vet det!

\section{Programledning}
SRE är ansvariga för att det finns representanter i E:s och BME:s respektive programledningar. För E har vid 3+3 platser (3 ordinarie och 3 suppleanter). I BME har vi 1+1. I programledningen tas beslut som rör hela programmet, såsom kursplaner och obligatoriska kurser. Här bör man även nämna att det är kårstyrelsen som stadfäster SRE:s val/nomineringar av personer, dvs det måste gå genom kåren för att studenter formellt ska få rösta i programledningarna. Vill man välja in nya eller ändra så kontaktar man Utbildningsansvarig för interna frågor (UI) på \hyperref{mailto:ui@tlth.se}{}{}{ui@tlth.se}. 

Åsa Vestergren är Programplanerare för bägge programmen och Ingrid Holmberg studievägledare för bägge programmen. Det är Åsa Vestergren som man bör ställa alla frågor rörande programmens uppbyggnad med kurser etc.

Mejl Åsa Vestergren: \hyperref{mailto:asa.vestergren@kansli.lth.se}{}{}{asa.vestergren@kansli.lth.se}
Mejl Ingrid Holmberg: \hyperref{mailto:ingrid.holmberg@kansli.lth.se}{}{}{ingrid.holmberg@kansli.lth.se}


Det är programledningarna som är SREs viktigaste och närmaste länk, då det är SRE som sköter programledningarnas studiebevakning ur studentperspektiv. Hör alltid av dig till programledningarna i första hand och håll dem uppdaterade på allt som händer på sektionen angående studier och studiemiljö.

\section{Institutionsstyrelse}
E-sektionen representerar i dagsläget tre olika institutionsstyrelser där man tar beslut på institutionsnivå. Det är oftast ekonomirelaterat och en hel del information om vad som händer på institutionen just nu. 

E-sektionen har följande platser:
Instituionsstyrelsen för Elektro- och informationsteknik - 1+1
Instituionsstyrelsen för Datavetenskap - 0+1
Instituionsstyrelsen för Biomedicinsk Teknik - 1+1

Även dessa platser är sådana som E-sektionen nominerar till och som kårstyrelsen sedan väljer/stadfäster.

\section{SRX}
Studierådsordförandekollegiet på TLTH. Du är ledamot tillsammans med alla andra
studierådsordförande för de andra programmen/sektionerna på LTH. Ordförande i SRX
och som sammankallar till dessa möten (ca varannan läsvecka) UI. I SRX kan man få tips och hjälp
från sina kollegor med samma uppdrag, hur studierådet fungerar på andra sektioner
mm. Här skall du redogöra för vad som skett i SRE sedan senaste SRX-möte samt lyssna
och anteckna den information som ordförande meddelar, t ex vidarebefordra mejl till
ditt utskott, eller redogöra meddelanden från SRX på nästa SRE-möte vid behov.

\section{Evenemang}
\subsection{Speak Up Days}
Teknologkårens påverkansdagar där man som student kan påverka sin utbildning och
säga till kåren vad man vill ändra. Chansen att fråga ut rektorn för LTH samt andra
fakultetsrepresentanter, studenter och Teknologkåren på olika evenemang ingår också.
Detta är det kårevenemang som involverar samtliga studieråd. Ditt ansvar är att erhålla
information genom SRX och se till att E-sektionen representerar (i form av SRE) i foajen
bredvid D-sektionen på ”vår” dag. Kåren står för fika, kaffe, enkäter mm.

\subsection{Pluggkvällar/Studiekvällar}
Pluggkvällar är en tradition på E-sektionen och förekommer på flera sektioner i många
olika utföranden. Pluggkvällar kan förekomma olika frekvent, på olika sätt eller inte alls,
men beslut angående pluggkvällar skall diskuteras i SRE

OBS: Programledningarna betalar idag för våra pluggkvällar OM du som ordförande följer dessa steg:

1. Anmälningslänk är bra då en lista med deltagarnas namn samt stil-id skall skrivas ut och läggas i ett kuvert

2. Kopiera kvittot/kvittona och lägg i samma kuvert

3. Skriv en faktura och skriv ut två kopior, den ena läggs i samma kuvert som ovan och den andra sätts in i pärm i HK (FVC kan hjälpa till med detta)

4. Ovanstående tre papper läggs i kuvertet (kuvert och frimärken finns i HK) och skickas

till addressen (dvs lägg i en brevlåda, finns i vaktmästeriet):

Lunds Universitet\\
BOX 188\\
221 00 Lund

Förutom detta bör du gå in till programplanerare Åsa Vestergren med jämna mellanrum
och berätta om pluggkvällarna och hur det går och vad planerna är inför en termin OCH
skicka in faktura mm INOM verksamhetsåret. Kontinuerlig kontakt och samförstånd är
också oerhört viktigt då det på inga vis är självklart att programledningen betalar för
våra pluggkvällar.

\subsection{SRE-workshop}
Detta har nu blivit ett återkommande evenemang och när jag skriver detta har jag själv inte ansvarat för detta. Men det finns mycket material i drive som jag ska dela med mig av och Ingrid står för bokning av lokal osv.

\subsection{Övriga}
Resterande evenemang är upp till dig att komma på, fixa mm då det är roligt om
SRE varje år kommer framåt och präglas av de aktivas idéer och engagemang. Att
synas under nollningen är A och O. Här spelar den redan halvt planerade SRE-workshopen och annat samarbete med NollU stor roll.

\section{CEQ}
SRE:s återkommande uppgift att granska CEQ:er och gå på utvärderingsmöten. Som ordförande är det din uppgift att se till att enkäter blir granskade. Du kommer att få mail när det dags att göra detta, en del kommer på papper de enkäterna kommer det separat mail om och de hämtar man på 5:e våningen, utb. service (står tydligt i mailet). 

Du kommer även att få mail när det är dags för möten. Då gäller det att skicka vidare det till rätt representant så den kan gå på sitt möte. Vill även slå ett slag för att påminna om att skriva kommentarer i CEQ-databasen efter representanter varit på möten!

%\tableofcontents
\newpage
\end{document}



