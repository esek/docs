\documentclass[10pt]{article}
    \usepackage[utf8]{inputenc}
    \usepackage[swedish]{babel}
    
    \def\post{Ordförande}
    \def\date{2020-01-02} %YYYY-MM-DD
    \def\docauthor{Edvard Carlsson}
    
    \usepackage{../e-testamente}
    \usepackage{../../e-sek}
    
    \begin{document}
    \heading{\doctitle}
    
    I det här testamentet finns en del matnyttig information som är bra att känna till men det är inte på något sätt heltäckande utan se det snarare som ett uppslagsverk. Som Ordförande kommer du att ställas inför många nya spännande situationer som du kanske inte alltid är helt bekväm i men det är en stor del av tjusningen med posten. Du kommer att lära dig en massa bra saker som du kommer ha nytta av resten av livet under året. Glöm inte att ha roligt och var inte rädd för att ta hjälp av andra, både inom och utanför Sektionen.
    
    \newpage
    
    \tableofcontents
    \newpage
    

    \section{Inledande tankar från avgående ordförande}

    Som tidigare nämnt är testamentet endast kompletterande för posten, det mesta får man lära sig allt eftersom. Jag arbetade personligen inte särskilt mycket med testamentet, läste igenom det när jag fick det och använde det som uppslagsverk någon gång under våren. Kanske skulle jag känt att det givit mer nytta om jag använt det mer aktivt under året, samtidigt är posten av sin natur oförutsägbar och det är svårt att veta vilka arbetsuppgifter som tar mycket tid och energi. Det jag vill uppmana till är att hitta ditt egna förhållningssätt till posten och dess arbete under året, det jag insett är att det på många sett är genom stressen att behöva prestera man också lär sig. Tidvis kommer det vara mycket stor arbetsbörda både på dig och din styrelse. Ta då hjälp av varandra och de resurser som finns på sektionen, samt fundera på hur detta bör utvecklas för att göra engagemanget mer hållbart. För detta tror jag är en stor utmaning vi står inför. Jag vill även uppmana till att driva de frågorna som du tycker är relevanta för ditt år. Kanske inte från start men när du blivit bekväm i rollen och fått bättre koll på hur saker fungerar. För något av det bästa med ordförandeposten är att man får möjligheten att välja mer fritt och enklare kan driva det man själv är intresserad av eller tycker behövs just nu. Avslutningsvis vill jag trycka på vikten av styrelsens välmående, det kommer säkert dyka upp oväntade händelser under året som förändrar dynamiken och arbetsprocesserna. Jobba därför öppet inom styrelsen och se till att alla trivs med sin roll och situation, våga även göra förändringar där saker inte fungerar. 

    Givetvis kommer jag försöka vara så tillgänglig och hjälpsam som möjligt även om jag inte kommer vara geografiskt närvarande under våren. Om det skulle vara några problem eller något du undrar över så är det bara att skriva så ska jag försöka svara så snabbt som möjligt. Se bara till att inte ringa mig på mitt vanliga nummer, så sparar vi båda massa pengar (sen har jag ju ändå inte ditt nummer sparat ;)).

    -- \emph{Edvard Carlsson, Ordförande 2019}
    
    \section{Allmänna tips}
    
    Hälsa på och prata gärna med Sektionens medlemmar, framför allt funktionärerna och visa uppskattning för det arbete som de ideellt lägger ner på Sektionen. Försök att lära dig vad alla funktionärer heter och vad deras roller är på Sektionen, det uppskattas av många. Bara genom att vara närvarande i Edekvata och LED och hälsa på folk kommer du att få reda på många problem och kunna ge svar på många frågor som medlemmarna har. Ta dock inte på dig allt själv utan försök att vara duktig med att delegera arbetet till resten av styrelsen. Din attityd kommer påverka andras motivation och hur folk ser på sektionen. 
    
    Tänk på att du representar Sektionen och att det kanske inte alltid är lämpligt att framföra sina egna åsikter utan i vissa fall Sektionens. Detta kan kännas lite obekvämt men det är bara att köra på!
    
    En bra grund att stå på i många diskussioner är Sektionens stadgar och reglemente, att ha en uppfattning om vad som står i dessa kommer att hjälpa dig i många situationer.
    
    Många gamla Ordföranden nås via mail på \texttt{emeritus@esek.se}.
    
    \section{Krusidull-E}
    
    Vårat Krusidull-E är varumärkesskyddat av oss. Denna avgift betalas var 10:e år. Även F:arnas F är i vår ägo så glöm inte att ta betalt för detta (för närvarande 160 kr om året). När det gäller F:arnas F brukar fakturan på 160 kr lämnas på något roligt sätt och sedan brukar de oftast vägra betala. Nästa gång båda märkena ska betalas är i början av 2022. Bevisen för båda finner du i VIP-pärmen i det stora kassaskåpet.
    
    \section{Julkort \& Sommarkort}
    
    Varje år tar styrelsen electus ett julkort i något häftigt tema. Detta ska sedan skickas ut till alla enligt julkortsdokumentet som finns i Styrelsedriven under Övrigt. Detta är en bra teambuildingaktivitet, försök att få med hela styrelsen. Glöm inte även skicka ut sommarkort innan sommaruppehållet. Kostanderna för detta tas på representation eller styrelsen internt.
    
    \section{Personer}
    
    \subsection{Bra folk att bekanta sig med}
    
    Bra folk att ha träffat och presenteras sig för är:
    \begin{dashlist}
    \item Inspektorn (Monica Almqvist)
    \item Husintendent (Per-Henrik Rasmussen (PH))
    \item Husprefekten (Mats Cedervall (pensioneras under våren))
    \item Vaktmästeriet/Tryckeriet (Peter Polfeldt, Jessica Nilsson)
    \item Studievägledare (Roger von Moltzer)
    \item Programledare för E och BME (Markus Törmänen respektive Ingrid Svensson)
    \item Programplanerare (Åsa Vestergren)
    \item Lokalbokarna
    \end{dashlist}
    
    \subsection{Inspektor}
    
    Sektionens inspektor är Monica Almqvist. Hon valdes våren 2012 (omvald 2014, 2016 och 2018). Inspektorn väljs på Vårterminsmötet varje jämnt år och ska bjudas på alla större officiella fester så som skiphtet och NollEgasquen. Förvarna gärna henne om när hon bör hålla tal. Hon fick under 2017 en inspektorsmantel som hon kan ha under NollEgasquen och SåS. Den ska gå vidare till nästa Inspektor när den väljs och broderas med namn och året personen valdes.
    
    \section{Arrangemang}
    
    \subsection{Kurs På Landet}
    
    Varje år har man en utbildningshelg på landet för nya styrelsen där även avgående styrelse inbjuds. Ni nya ordnar i hemlighet plats och mat för helgen, medan gamla ansvarar för utbildningen och ``lärakännaaktiviteter''.
    
    Ni nya åker iväg till platsen först och avgående styrelse kommer dagen efter när de har lyckats lösa ledtrådarna ni har lagt ut för att de till slut ska hitta. Det är lämpligt att ha KPL så tidigt som möjligt på året och gärna innan skiphtet. Tänk på att ha koll på hur mycket pengar som går åt. Kostanden får belasta Styrelsen internt och i riktlinjerna för budgeten står det hur mycket som får tas.
    
    Det viktigaste med KPL är att ni i nya styrelsen har kul tillsammans och lär känna varandra ännu bättre.
    
    \subsection{Skiphtet}

    De som ska bjudas är gamla och nya funktionärer. Skiphtet är en belöning och uppmuntran för allt ideellt arbete som läggs ner. Tänk på att det finns poster som ligger utanför utskotten så som revisorerna, sigillbevarare, talman och valberedning. Glöm inte att hedersmedlemmarna har möjlighet att närvara om de skulle vilja...
    
    Styrelsen brukar sitta vid ett honnörsbord inför resterande funktionärer och äter till skillnad från resterande på vanligt porslin. Man brukar även ha ett klädtema som ska hållas hemligt fram till intåget på skiphtet.
    
    Det har tidigare varit styrelsen som städar. Städningen tar lång tid, så det är bra att så många som möjligt i styrelsen hjälper till att städa.
    
    Skiphtet har sett ut på flera olika sätt de senaste åren. Som inspiration och till hjälp för att fundera ut vad som passar bäst i år följer en kortare redogörelse från åren.
    
    År 2014 hade vi Skiphte på Lophtet eftersom vi har tillstånd i Edekvata. Detta år bjöds det på lunch och det gjordes gemensamma lekar med alla innan funktionärerna delades upp i utskott. Detta koncept kan ändras på och eventuellt åka iväg någonstans.
    
    År 2015 hade vi en mindre utbildning på dagen tillsammans med likabehandlingsombuden. Sedan lekar på Lophtet och sedan en sittning som sexet anordnade. Styrelsen hade ett tema och spexade när de kom till Lophtet (till sittningen).
    
    År 2016 hade vi först en kort info och sedan lekar för alla årets funktionärer. Efter ett tag anslöt även fjolårets och festen flyttade så småningom upp till Lophtet. Styrelsen glittrade underbart vackert.
    
    År 2017 hade vi också först en kort info. Inte många kom, och jag skulle säga att det är bättre att ha den någon annan gång än vid en fest. Sen hade vi i vanlig ordning en najs sittning på Lophtet.
    
    År 2018 hade vi först lekar runt omkring i E-huset. Typiskt bra tillfälle att måla nya utskottstavlor till Diplomat. Ganska många dök upp, vilket var roligt! Var beredd på att folk inte håller tider utan droppar in när det passar dem, så var flexibel! Man bör hitta tid till att utbilda funktionärerna om deras rättigheter och skyldigheter, access, förmåner, o.s.v i samband med Skiphtet. Dock inte i samband med själva festen då det inte brukar vara hög närvaro på själva utbildningen då. Därefter hade vi som vanligt en stökig men najs sittning på Lophtet.
    
    År 2019 ändrades spelreglerna med tillståndsenhetens nya tolkning av alkohollagen. Enkelt sagt fick vi inte anordna BYOB-sittningar, vilket skiphtet varit tidigare år. Isället hade vi en organiserad sittning med bar i E-foajén som hölls av E6. Det blev tillräckligt stökigt det med. Efteråt gjorde vi lite av en fuling och hyrde lophtet för eftersläpp utan några arbetande.

    \subsection{Funktionärstacket}
    
    De enda riktlinjerna för funktionärstacket hittas i budgetriktlinjen för Funktionärsvård, posten som kostaden för tacket belastar. Därför finns det en del oklarheter kring dess genomförande och vems ansvar det är att det genomförs. 2019 tog ordförande på sig arbetet och likt tidigare år bestod tacket av en dagsaktivitet samt en sittning på nation. Exempel på genomförda dagsaktiviteter är curling, utmaningarnas hus och laserdome. Att ordförande anordnar dagen är problematiskt då planeringen tar en del tid i en redan intensiv period, naturligtvis gäller detta för hela styrelsen. Ett förslag till kommande år hade kunnat vara att tillsätta en projektgrupp som då kan att utveckla evenemanget bättre. Funktionärsvården har ganska generöst med pengar så möjligheterna till förändring eller fler tack är stora. 

    \section{Styrelsemöte}
    
    Se till att kallelse kommer ut i tid (minst tre läsdagar innan mötet).
    Kallelsen skrivs i \LaTeX med hjälp av våra dokumentmallar som ligger i sektionens git (\texttt{elt14ema/docs}). Ta hjälp av Kontaktorn för att lära dig git och \LaTeX. När du har skickat kallelsen så lägger du upp dagordningen i möteshandlingar på hemsidan. Kallelsen mailas sedan till \texttt{kallelse@esek.se}. Det har fungerat bra med protokollförda möten varje vecka, men se gärna över upplägget kontinuerligt och gör det som passar er styrelse och situation bäst.
    
    För att få ett bra och effektivt möte är det bra om beslutsunderlag är gjorda i förväg. Dessa är bra att skicka ut med kallelsen då detta underlättar för både styrelsen och de medlemmar som är intresserade. För gärna diskussioner med styrelsen i förväg innan ett större beslut (till exempel på ett diskussionsmöte). I samband med varje styrelsemöte är det bra att skicka med handlingar så att styrelsen vet vad som kommer upp på varje möte, på så sätt kan man få bättre diskussioner och om någon medlem på sektionen är intresserad av punkten kan de komma och göra sin åsikt hörd.
    
    Tänk på att man ska ha minst 3 officiella möten per termin och att man måste vara minst 6 för att var beslutsmässiga. Se till att Kontaktorn ligger i fas med protokollen, det gör det mycket lättare att hålla reda på vad ni faktiskt beslutat. Det är du tillsammans med Kontaktorn och en justerare som justerar protokollet.
    
    Glöm inte att uppdatera beslutsuppföjlningen om ni tar några beslut som kräver redovisning vid senare möte. Uppföljningen är ett bra verktyg för att skapa struktur och se till att saker blir gjorda, använd det gärna mycket om lämpligt. Tänk på att lägga diskussionspunkter i slutet så kan man anpassa diskussion efter hur lång tid man har kvar av mötet.
    
    Uppmuntra alla att meddela punkter till styrelsemötet. Om man får med det i kallelsen så underlättar det hur långt mötet ska bli samt att alla i styrelsen kan få ha tänkt igenom vad de tycker i frågan innan i mötet (så att man får en bra diskussion på mötet).
    
    \section{Sektionsmöte}
    
    \subsection{Mötet}
    
    Försök att planera in datum och boka lokal för sektionsmötena ganska tidigt, bland de första mötena ni har. Då riskerar man inte att behöva hålla möte i läsvecka 7 eller något annat dumt. Försök välja datum som inte krockar med större aktivitet som kan beröra många, såsom Karneval, Tandem, Jesperspex, Arkad, Sångarstriden, F1 Röj och dylikt. Diskutera även med Talmannen och Valberedningens Ordförande innan ni beslutar om datum. Prata även med valberedning inför valmötet och vårterminsmötet. Kolla datum för motioner och handlingar så att allt sånt kommer upp i tid. Var noga med att sprida datumet för motionsstopp. Skicka gärna ut tidigt inför mötet att de kan skriva motioner så de har god tid på sig att skriva en motion då det oftast tar tid. Ta hjälp av Kontaktorn när handlingar ska göras. Var ute i god tid!
    
    Kallelse till sektionsmöte skickar du som Ordförande ut. Kåren använder mailtjäsntern mailchimp för att nå ut till alla medlemmar. Våra inloggningsuppgifter finns i driven. Skulle det vara oklarheter kan du höra med ENU-ordförande. Dock har många för vana att missa sådana mail. Använd lämpliga informationskanaler för att kallelsen når ut till alla intressenter. Glöm inte att även skicka till Inspektorn.
    
    Inför varje möte skall alla utskott skriva en utskottsrapport och tänk även på de utskott som ligger vid sidan om styrelsen som valberedningen. Finns det ingen utskottsordförande så be någon annan i utskottet eller skriv själv. I utskottsrapporten ska det stå vad man har gjort sedan senaste terminsmöte.
    
    Endast ordinarie medlemmar av sektionen får rösta på sektionsmöten, och därför måste alla ha med sig sitt kårleg (alternativ meceantappen eller medcheck). Sedan 2015 har det bjudits på mat under mötena till de som anmält sig, vilket har varit väldigt uppskattat. Kostnaden för den tas på sektionsmöte. Läsk, kaffe, kakor och frukt under mötet är också trevligt. Tänk på att detta finns/beställs av caféet.
    
    Källan till information om vad som egentligen ska tas upp och vilka dagar som gäller är stadgar och reglemente. Börja med handlingarna i god tid innan de ska vara uppe och framför allt - tjata på folk om att skicka in sina texter som ska vara med. Om ni har saker ni vill förändra så ska ni skriva propositioner. Börja med de tidigt, väldigt lätt att skjuta på det så det rinner ut i sanden. Tänk även på att styrelsen ska skriva en budget och verksamhetsplan till höstterminsmötet. Dessa tar mycket tid så börja med de tidigt. Se till att allas åsikter kring dessa görs, men att det är ett fåtal som verkligen skriver de.
    
    Se till att stadgar och reglemente är redigerade efter att justerade protokoll har kommit in från sektionsmöten med reglementes- eller stadgaändringar. Ifall stadgarändringar går igenom på andra läsningen skall dessa skickas till fullmäktige på kåren för att stadfästas. Du får kallelse till fullmäktige och där står när handlingar ska skickas in och till vem. Kåren har även bra riktlinjer för vad som ska göras och givetvis finns flera exempel på git.
    
    \subsection{Verksamhetsplan}
    
    Under 2015 infördes en verksamhetsplan. Tanken med verksamhetsplanen är att man ska få ett mer långsiktigt arbete, men även att tankar som den avgående styrelsen har inte bara ska försvinna. Något som bör jobbas mer med är att försöka få in åsikter från övriga Sektionsmedlemmar till verksamhetsplanen, kanske lägga ut en preliminär version som alla kan se och sedan tycka till om?
    
    Börja i god tid med verksamhetsplanen för den kan ta lång tid att skriva. Se till att alla i styrelsen blir nöjda med den. Tänk noga igenom vad som behöver arbetas med under en längre sikt, men även problem som har uppstått under året som inte är löst.
    
    Verksamhetsplanen har de senaste åren fallit lite i skymundan och därför bör dennes syfte och koncept utvärderas. En rekommendation är att tidigt diskutera den med styrelsen för att göra alla införstådda och underlätta framtida arbete med den.
    
    \subsection{Expo}
    
    Expot är en chans att visa upp sektionens olika poster. Det är bra om flera från varje utskott är på plats och i synnerhet poster som är svåra att fylla men viktiga. Utkotten har ofta arrangerat en lek eller ett spel för att göra det roligare och göra fler intresserade av flera olika utskott. Se till att ni gör det till er grej och att det blir kul att både gå och anordna expot. Glöm inte Valberedningen och Övriga funktionärer. Tidigare har det varit under expot som valperioden öppnat och representanter från valberedningen brukar stå med datorer som folk kan använda för att nominera folk. 2019 inleddes valperioden med en lunchinfo en vecka före expot given av ordförande och valberedning. Detta för att upplysa, i synnerhet de nya, om valprocessen, engagemang och för att skapa intresse. Det skapades även en valguide med likande syfte, material för både finns i drive.
    
    \section{Dokument}
    
    Sedan 2016 ligger (nästan) alla sektionens dokument i en git-repo på sektionens server. Detta inkluderar kallelser, möteshandlingar, avtal, styrdokument, med mera. Alla dokument skrivs i \LaTeX, vilket borde gå snabbt att lära sig för de saker som du som Ordförande behöver göra. Ta hjälp av Kontaktorn för att lära dig git och \LaTeX!
    
    Git-repon heter \texttt{elt14ema/docs}. Alla administratörer och styrelsen ska ha access till den by default.
    
    Erik Månsson skapade git-repon med dess \LaTeX-mallar under sitt år som Kontaktor 2016. Sen dess har Ordföranden och Kontaktorer använt sig av repot flitigt. Vi hjälper gärna till! Skulle bli problem eller om ni har några frågor kan ni höra av er till Erik personligen. Han nås lättast på \texttt{erikm@esek.se} eller \texttt{erik@mansson.xyz}.
    
    \section{Huset}
    
    \subsection{Allmänt}
    
    Det är alltid bra att stanna till och säga hej till PH och Peter. Man kan även när de jobbar sent en kväll komma dit med kaka och Coca Cola för det är uppskattat. Är man på god fot med dem så går allt så mycket smidigare. PH fyller år den 22 maj, född 1964. Styrelsen brukar i början av året gå och presentera sig för PH och ge en liten present som tack för ett bra samarbete det senaste året, tidigare år har det givits whisky. 2019 blev det istället en tårta som delades av hela tryckeriet, rekommenderar något liknade framför whisky. Se till att vara på god fot med vaktmästeriet, det underlättar väldigt mycket för verksamheten.
    
    Mats Cedervall, född 22 maj 1953, som är husprefekt är det också bra att var på god fot med. Vill man låna t.ex. foajén är det med honom man ska prata med. Det är även väldigt uppskattat åka till honom och busa i hans trädgård när han fyller jämt. Även Cedervall bör få ett besök av hela styrelsen i början av året och även han blir glad för en whisky som tack för ett bra samarbete under det gångna året. Han är dålig på att svara på mail ibland men begär själv svar snabbt när han mailar. Värt att notera är att Cedervall kommer pensioneras under våren och att vi idag inte vet något om hans efterträdare.
    
    Det finns även en Husstyrelse som handhar sådant som berör Edekvata. Vi har en representant där och det är enklast att det går via dem ifall det är något som behöver tas upp. Är det akut så gå direkt upp och prata med husprefekten. Om något händer, som på något sätt kan beröra huset, skicka ett mail till Cedervall direkt. Det är alltid bäst att komma med den första informationen. Att han får höra det på bakvägar kan man vara säker på. Se till att fixa saker som han skickar på mejl fort så att de sakerna inte bli så stora.
    
    \subsection{Braig och Avig}
    
    Inför varje arrangemang i E-huset behövs en brandansvarig och en ansvarig (går att ha samma person som båda två). Dessa ska i god tid innan arrangemanget anmälas till PH, minst en vecka. Var tydlig med att alla som har evenemang (gillen, sittningar o.s.v.) i huset ska ha en braig och avig. För att hålla PH på bra humör är det som sagt viktigt att dessa anmäls i god tid. För anmälan måste man ha namn, personnr, sektionstillhörighet, bild på avig/bravig, när man bokar FikaFika via hemsidan sköts detta automatiskt.
    
    \section{Lokaler}
    
    \subsection{Städning}
    
    Försök städa så regelbundet det går i våra lokaler. Det blir mycket trevligare att utföra sina funktionärsysslor då. Uppmuntra styrelsen till detta och gör det till en kul grej. Se till att Edekvata är rent speciellt innan jul- och sommarledighet. Städveckor har delats upp efter utskott men sköts halvbra, städning är generellt ett utvecklingsområde för sektionen.

    Grovsopen utanför tryckeriet hämtas var fjärde fredag tidigt på morgonen. Vi kan inte låta massa skräp stå där i flera veckor så planera in större utrensningar med detta. Prata gärna med PH om detta så ni har samma schema. 
    
    Under nollningen har vi fått undantag av Cedervall att arbeta och förvara bråte på gräsytorna bakom Edekvata. Det är dock viktigt att det bara sker under nollningen eftersom andra organisationer som är verksamma i huset inte uppskattar detta och då inte drar sig från att klaga. Vilket kan leda till problem för oss då vi egentligen inte har någon rätt att bruka gräsytorna. 
    
    \subsection{Förråd}
    
    Förråden blir lätt stökiga, försök se till att det städas regelbundet av de utskott som använder förråden. Extra viktigt är att hålla rent i klockrummet utanför EKEA för att de ska kunna jobba med fläktsystemet i det rummet.
    
    Ingen alkohol ska förvaras i våra förråd, då kan vi få problem.

    \subsection{LED}
    
    Har ni problem med något i LED-café så fråga PH, han brukar kunna hjälpa till. Städsaker får vi från lokalvårdarna och även där är det lättast att gå via PH så skickar han din förfrågan vidare.
    
    D och L kan dyka upp och säga att de vill vara med och leka. Det som är sagt nu är att alla får jobba men endast för gratis lunch och tackfest. Skulle vidare diskussioner uppstå så finns det gamla dryga Ordföranden och Cafémästare som kan uppdatera dig om vad som sagts under tidigare förhandlingar.
    
    \subsection{EKEA}
    
    EKEA är ett bra förråd men också jobbigt att hålla snyggt. Försök att städa det regelbundet så det hålls i ordning. Se till att det inte står massa saker utanför burarna, speciellt inte framför dörrarna då det är brandvägar som ska hållas fria. Se till så att inget förvaras i klockrummet för det är inte tillåtet.
    
    \subsection{FikaFika}
    
    Fikafika bokas via \url{esek.se/fikafika} och där fyller man i all nödvändlig information. Förfrågan godkänns sedan av Ordförande på E, som noterar datumet och ser till att det inte krockar med någon annan bokning. Sedan godkänns det slutligen av PH innan den som bokat får en bokningsbekräftelse. Dealen för tillfället är att de som ska kunna boka är personer som pluggar vid något av programmen i huset och att verksamheten ska ha någon form av anknytning till programmen/sektionerna. Undantag har de senaste året gjorts för TLTH:s danskurs men till exempel inte för E-are som jobbar på Helsingkrona och vill ha sexa efter denna.
    
    När man vet att det varit mycket fest i huset kan det vara bra att kontrollera så att det ser bra ut i bland annat FikfaFika innan undervisningen drar igång på måndagen, annars lär du snart bli varse om att det behöver städas från PH. Eftersom E-sektionen har ansvaret för bokningen får vi skiten om det inte sköts så försök se till att alla är duktiga och bokar FikaFika. Schemat där man kan se om det är bokat sköter tyvärr inte sig själv utan det är en vanlig Googlekalender som ordförandemailen har access till.
    
    \section{Access}
    
    \subsection{Allmänt}
    
    Man behöver åtkomst till en massa ställen. Det brukar med undantag för klockrummet i de flesta fall vara Förvaltningschefens uppgift att rodda i detta men det gör ni naturligtvis upp själva. Det finns en policy kring access och nycklar, se till att den följs samt uppdateras vid behov. Om det skulle vara några problem med lokalerna är det generellt bäst att ta det med PH eller Cedervall.
    
    \subsection{Nycklar}
    
    Till ett par av våra lokaler behöver man ha nyckel för att komma in. Föregående år har man skrivit på ett kontrakt och betalt 100 kr i depositionsavgift för att få nycklar. Alla nycklar förvaras i stora kassaskåpet på en nyckelring.
     
    \subsection{Egna lokaler}
    
    Sedan ett par år tillbaka hanteras funktionärernas access till våra utrymmen via en modul på hemsidan. Ordförande och Förvaltningschef brukar gemensamt avgöra vilka som ska ha access vart. Värt att tänka på är att Revisorerna brukar ha access överallt och att alla som ska städa behöver ha access till Sikrit för att komma åt städredskapen. Ta hjälp av Kontaktor Emeritus och se till att rätt personer har access. Tänk på att vissa av föregående års funktionärer kan behöva access ett tag under nästkommande verksamhetsår (skattmästare, delar av styrelsen).
     
    Blir batterierna dåliga i låsen så ska LU-kortet automatiskt få meddelande om detta men löser de det inte är det bara att mejla.
    
    \subsection{Edekvatadörren}
    
    Vilka som har access till Edekvatadörren är lite oklart, innan har det varit alla som pluggar på ett program i huset och så är det troligtvis nu också. Oavsett så har LU-kortet hand om accessen dit så det är bara att maila dem om det är några problem eller någon inte har fått access.
    
    \subsection{EKEA}
    
    Access till vår bur hanterar vi själva men accessen till klockrummet hanterar PH. Han föredrar att få alla som ska access samlat så att det inte blir en massa mejlande. I styrelsedriven under lokaler finns det ett ark med instruktioner och lista över de som hade access i år. 
    
    \subsection{Andra lokaler i huset}
    
    PH ska få mail när man bokar via lokalbokarna om man har bokat till exempelvis föreläsningssalarna eller E:1426 (får bokas kostandfritt av SRE). Instruktioner för att låsa upp står på dörrarna. Tänk på att våra lokalbokare alltid har access till föreläsningssalarna.
    
    \subsection{Kåren}
    
    Access till kårens lokaler fixar man genom att fylla i formulär på \url{tlth.se/access}. Som Ordförande är det bra att ha access till expen/hänget och corneliskorridoren. Bastun kan också vara najs att ha. Påminn gärna dem i styrelsen som behöver access till kåren att fixa det i god tid så att de inte står utan när de väl behöver det. Accessen i kårhuset försvinner efter ett halvår så glöm inte att fixa ny innan nollningen.
    
    \subsection{Kassaskåpen}
    
    I kasskåpet finns en nyckelknippa med nycklar till Sektionens alla lås. Här inne finns även kopior av nycklar som lämnas ut till funktionärer vid behov. Glöm inte kontrakt och ta deposition vid utlämning.
    
    Under handkasselådan finns några pärmar som är bra att känna till. Bland annat viktiga-papper-pärmen, nyckelkontrakt-pärmen och modulo10-mappen.
    
    Till det ``lilla'' kassaskåpet i HK är koden för närvarande ``hybris''. Ändra gärna koden årligen så att bara de som verkligen har behov att använda skåpet har tillgång till det. Detta gör man genom att:
    
    \begin{numplist}
    \item Slå nollan 6 ggr.
    \item Slå gammal kod en gång.
    \item Slå ny kod 2 ggr.
    \end{numplist}

    \section{Andra sektioner}
    
    Se till att ha bra kontakt med D-sektionens ordförande. Det är mycket lättare att vara två som ser och tar hand om problem i huset. Vi delar även bord tillsammans med D-sektionen (de som står i EKEA). D-sektionen äger borden men vi får använda de båda två men ska de till exempel hyras ut är det D som sköter det. Kontrakt på detta finns i VIP pärmen.
    
    \subsection{Ordförandekollegiet - OK}
    
    På kåren finns ett ordförandekollegie där alla sektionsordförandena och kårordföranden är med. Detta är en naturlig plats för att diskutera och ventilera gemensamma frågor och problem. Frekvensen på mötena kan variera men brukar vara ca var annan vecka. Ta gärna upp saker som känns för stora för dig att själv styra och bestämma över. Prata även med andra ordförande och tar hjälp av deras kunskap. De har förståelsen för att ordförandeposten ibland kan bara väldigt tung och att man ibland kan få mycket skit.
    Ordförandekollegiet dömer Regattan och brukar spöka ut sig i någon fin utstyrsel under nollningen.

    \subsection{Vänsektioner utanför Lund}

    Som ordförande representar du sektionen både i Lund och utanför. Att åka på resor och besöka andra universitet är väldigt roligt. Våra externa relationer har utvecklats mycket de senaste åren och intresse för ytterligare utveckling tror jag definitivt finns. En del sektioner i Lund har vad de kallar för ``konferenser'' med andra lärosätten, så det finns mycket inspiration och tips att hämta. Styrelsen får även 5000 kr i extern representationsbudget samt du som ordförande ytterligare 1000 kr för att uppmuntra till fortsatt sammarbete mellan föreningarna. Det finns en del info i driven om dessa, annars bör Kontaktorn också ha lite koll.  

    \section{Nollningen}
    
    \subsection{Allmänt}
       
    Ordföranden brukar skriva en sida i NollEguiden där den välkomnar nollorna. Även alla utskott skriver en kortare text i NollEguiden, det vill säga även du om styrelsen.
    
    Tänk på att synas mycket under nollningen och gå på så många aktiviteter som du vill och kan. Tagga även styrelsen att gå på mycket. Ni kan till exempel spexa under sittningar, hjälpa till med att laga mat, hålla i lekar, o.s.v. Tänk på att göra upp en ordentlig plan och för diskusison med Ph\o set så att alla vet vad som gäller från början. Det finns en budget för representation av styrelsen som syftar till att bidra med sittningskostnader för styrelsemedlemmar (dock ingen alkohol).
    
    På uppropsdagen håller ordföranden ett tal inför nollorna. Talet brukar vara cirka 5 minuter långt. Tänk på att hitta på något speciellt för just dig som väcker deras uppmärksamhet, för det är hur många personer som helst som snackar inför dem den dagen, de ska minnas dig.
    
    Innan NollEgasquen står Ordförande och \O verph\o s någonstans i källaren på E-huset där nollan får svära E-eden inför Oddput och kyssa hans hjässa. Glöm inte att ha något desinficerande att torka av honom emellan gångerna med. Ordföranden ska även hålla tal under NollEgasquen.
    
    Det är NollU som har planerat hela nollningen, men se till att vara till hands och hjälpas åt så mycket som möjligt. NollU är ofta väldigt stressade under nollningen och tar tacksamt emot hjälp om det behövs. Men inse att nollningen även är en krävande period som styrelse och ordförande, där engangemanget lätt kan gå överstyr.
    
    Diskutera igenom tidigt med ditt \O verph\o s om hur du och styrelsen förväntar sig information från NollU. Gör det också tydligt att du som Ordförande och till viss del Förvaltningschefen (som firmatecknare) är ytterst ansvariga för allt som händer på nollingen och att \O verph\o set \emph{måste} rapportera alla händelser. Ju färre grejer som hålls hemliga desto bättre. Det underlättar för alla utskott om utskottsordförandena är med på vad som planeras från NollUs sida. Dels får NollU själva bättre hjälp och därmed mindre att göra samt tjänar utskotten på att veta vad som faktiskt förväntas av dem. Utskotten hjälper i 9 av 10 fall gärna till, de måste bara få reda på vad som  behöver göras.

    
    \subsection{Oddputs Ed}
    
    \vbox{
    Jag, Nollan, lovar och svär\\
    att högakta mina företrädare\\
    att vara min sektion trogen\\
    att inte verka vara förmer än andra teknologer\\
    och att följa den elektrovita vägen.
    
    Som bekräftelse på detta vill jag nu\\
    kyssa Oddputs kala hjässa
    }

    \vbox{
    I, Nollan promise wholeheartedly\\
    To respect my predecessors\\
    To be my guild faithful\\
    To not act superior towards other technologists\\
    And to always follow the elektrowhite way

    As confirmation on this I now want to\\
    Kiss Oddputs bare head
    }
    
    
    \section{Post}
    
    Nästan all sektionens post hamnar hos PH. Detta facket bör kollas ofta och detta görs bäst av Ordföranden. På så sätt håller man reda på vilken post som anländer och så får man sin dagliga dos av PH. Sortera posten så de hamnar till rätt person.
    
    I Kårhusets expedetion har E-sektionen också ett eget fack. Ditt kommer inte så mycket post, men det händer, oftast julkort och inbjudningar till andra sektioners baler. Lämpligt att kolla denna vid OK-möten.
    
    När man beställer saker kan man enkelt använda både vår boxadress och vår vanliga adress. Paketen levereras till vaktmästeriet, är du osäker på något relaterat detta så fråga PH först!
    
   \section{Styrelseåterträff}

    Veckan innan nollingen 2019 bjöds så många av tidigare styrelser som kunde hittas in till en styrelseåterträff. Återträffen bestod av en sittning i glasfoajén arrangerad av de två senaste sektionsordförandena. Över 30 deltagande med ytterligare fler som uttryckt intresse av eventuellt kommande års evenemang. Det finns ett ark ``Styrelsen Emeritus'' i styrelsedriven under event > pajphesten där jag sammanställt mail-adresser till alla tidigare styrelsemedlemmar jag lyckats hitta. Namnet ``Pajphest'' kommer från när styrelsen 2017 och 2018 skulle ha en gemensam middag och Entertainer 2017 Albin Nyström Eklund efterfrågade en riktig pangfest, men skrev fel.

    \subsection{Modulo 10}
    
    1996 instiftades en så kallad tioårsfond. Som medlem i styrelsen får man möjlighet att betala en avgift (f.n. 100kr) för att tio år senare bli inbjuden till en middag med dåvarande års styrelse. Dokument rörande detta finns i en blå mapp i kasskåpet.
    
    Modulo 10 har inte genomförts på många år, har varit problem med bristande intresse.
       
    \section{Revisorer}
    
    Revisorernas uppgift är att granska verskamheten och inte att ha åsikter på allt. Ha en dialog med dem om det skulle behövas men rent tekniskt bör dem inte frågas om råd då det då blir svårt för dem att genomföra sin granskande uppgift.
    
    \section{Informationskanaler}
    
    Det finns många ställen att lägga upp information. Det är bra att ha en diskussion gärna flera gånger per år vad för information som ska delas i vilka kanaler. 

    För inspiration följer här lite riktlinjer från 2019:
    
    \begin{itemize}
        \item E-sektionens facebooksida: Saker som berör allmänheten.
        \item E-sek Events: Saker som berör alla medlemmar.
        \item Utskottens informationskanaler: Saker som endast berör det egna utskottet, alternativt för NollU som har en egen fb-sida för att sprida info till nollor och phaddrar.
        \item Vi pratade även i slutet av året om behovet av en näringslivsgrupp. 
    \end{itemize}
    
    För att nå sektioner kommer många att skriva till info@esek.se. Denna mail går både till dig och kontaktorn, diskutera hur ni vill använda den så det inte blir förvirring. De kanaler som finns nu är:
    
    \begin{dashlist}
        \item Ekoli (TV i LED och Edekvata)
        \item \url{esek.se}
        \item \url{e-nollning.nu}
        \item Maillistor
        \item HeHE
        \item Facebooksidor: E-sektionen, Ph\o sets
        \item Facebookgrupper: E-sek events, Esek - Köp \& sälj
        \item Instagram: esektionenlth, Ph\o sets (väldigt inofficiell)
        \item LinkedIn-grupper: Alumi Elektroteknik LTH, E-sektionen inom TLTH
        \item LinkedIn-företag: E-sektionen inom Teknologkåren vid Lunds Tekniska Högskola
        \item Planscher
        \item YouTube
        \item Bonsai Campus
    \end{dashlist}
    
    \section{Ekonomi}
    
    \subsection{Allmänt}
    
    Så tidigt som möjligt på ert verksamhetsår så ska de nya firmatecknarna gå till banken och fylla i papper så att firmatecknarna byts och så att de nya har teckningsrätt för sektionen, två i förening. Se till att alla har legitimation med sig och protokollet där nya firmatecknare har valts (underskrivet) är med. Det brukar räcka med att de som ska ha dosa följer med till banken. Ring och fråga Sparbanken Ideon Gateway om tid.
    
    Bankmässigt innebär ``två i förening'' att en av firmatecknarna till exempel lägger upp en betalning men den andra måste godkänna denna för att den ska gå igenom. Det brukar vara så att Förvaltningschefen lägger upp allt i banken och sedan ber ordförande att signera. Det är bra att då snabbt kolla igenom betalningarna så att allt ser rimligt ut.
    
    För tillfället sparar vi våra pengar på ett bankkonto utan någon som helst ränta men där pengarna är lättillgängliga. Vill ni placera om pengarna så rekommenderar jag inga avtal som varar längre än ett år. Varje styrelse bör få fatta sina egna beslut om vad dem vill göra med sitt år. Om placering är något som intresserar er rekommenderar jag också att ni diskuterar igenom detta noga inom styrelsen. Man får fråga sig själv vems pengar det är man faktiskt hanterar. Oavsett finns det en policy hos kåren angående detta. Se även till att ni har rätt ingående balanser på de olika ``fonderna'' i början av året.
    
    I början av året kan det vara bra att ha en utbildning med de nya styrelsen om hur det fungerar med handkassan och lappar och så vidare. Fråga gärna Förvaltningschefen (och eventuellt dess företrädare) om denne vill hålla den här utbildningen. Det är bra om alla i styrelsen får koll på sånt här redan från början för att undvika så mycket fel som möjligt senare. Det är bra att du som Ordförande redan i början skaffar dig god inblick i ekonomin så att du kan hjälpa Förvaltningschefen lite och så att denne har någon att diskutera med. Hjälp också Förvaltningschefen att se till att styrelsen lär sig hur man lämnar in kvitton och dylikt redan från början så att Förvaltningschefen sedan inte behöver visa varje gång någon ska lämna in ett kvitto. Informera även de andra i styrelsen om Dispositionsfonden.
    
    Det är praktiskt ifall varje utskott och del av dessa sätter upp internbudgetar i samband med den ekonomiutbildning som Förvaltningschefen kommer hålla i. Detta är bra för att hela styrelsen ska sätta sig in lite i hela Sektionens ekonomi och i synnerhet sitt eget utskotts förutsättningar.
    
    Tänk på att hela styrelsen är solidariskt ansvarig för ekonomin även om det är Förvaltningschefen som bokför.
    
    \subsection{Budget}
    
    Inför varje höstterminsmöte ska styrelsen lägga förslag på en ny budget. Lägg tid på det här och stressa inte igenom det, det är viktigt att kommande styrelse får en budget som är rimlig att hålla. Försök att engagera hela styrelsen även om det kan vara svårt. Se också till att tillsammans med Förvaltningschefen ta fram ordentligt underlag för budgeten så att ni har något att basera den på, titta gärna på utfall mer än ett år tillbaka i tiden och försök även att se ert verksamhetsårs utfall där det är möjligt även om året inte är slut. Glöm inte heller att revidera budgetriktlinjerna.
    
    \subsection{Bankkort}
    
    Bankkort kostar lite pengar men är bra då det underlättar bokföringen och gör så att folk slipper ligga ute med pengar. Diskutera med förvaltningschefen och eventuellt övriga styrelsen vilka som behöver bankkort. Korten beställs sedan via internetbanken. Skriv kontrakt med alla som får bankkort, föregående års kontrakt sitter i en pärm i kassaskåpet. Vid årets slut bör man samla in alla bankkort och klippa dessa samt avvaktivera dem i banken. Man kan ställa in maximalt månadsbelopp i internetbanken, under nollningen går det åt mycket pengar så det kan vara en idé att sätta denna högt för vissa redan från början eller ändra inför nollningen så att man inte står där med ett stort inköp och inte kan betala.
    
    \subsection{Deponera pengar}
    
    Sektionen har de senaste åren strävat efter att bli kontantfria, det finns dock fortfarande kvar en del mynt i kassaskåpet. Om ni vill börja hantera kontanter igen kan ni höra av er till emeritus för guidning av hur man deponerar pengar. Vi avråder dock starkt från detta. 
    
    \section{Tillståndsenheten}
    
    I början av varje år ska de nya firmatecknarna anmälas som PBI:er (Person med Betydande Infytande) hos tillståndsenheten genom att skicka in en blankett. Även serveringsansvariga behöver anmälas till tillståndsenheten och detta görs via deras hemsida. Avgående Ordförande brukar få påminnelsemail om detta innan man avgår.
    
    De brukar även kräva att man har en firmatecknare med tillräcklig kunskap inom alkohollagen. För att man ska tillräcklig kunskap inom alkohollagen krävs C-certifikat i ansvarsfull alkoholhantering. Inbjudningar till A- B- och C-cert brukar även dem komma till avgående ordförande innan året är slut. C-cert kan även få skrivas direkt som prov hos tillståndsenheten, det är bara att kontakta dem.
    
    Se gärna till att Sexmästaren och Krögaren samt deras vice går A- och B-cert. Även Entertinern bör gå med tanke på UteDischot.
    
    Om vi ska ha en tillställning som ligger på en annan dag eller i en annan lokal än ordinarie serveringstider så måste det sökas ett utökat tillstånd (blanketter finns på tillståndsenhetens hemsida). Dessa ska sökas senast två veckor innan arrangemanget. Tillståndet ska skrivas under av den firmatecknaren som har kunskaper inom alkohollagen. Det är även den här personen som ska se till att den serveringsansvariga är en lämplig person.
    
    Serveringsansvarig får endast den person som är minst 20 vara. När tillståndsenheten kommer på besök är det den här personen de kommer att fråga efter. Den serveringsansvariga måste ha tillgång till alla rum i Edekvata om tillstånd vill undersöka dem. Alla jobbare ska veta vem som är serveringsansvarig för kvällen. Serveringsansvariga kan rapporteras in på tillståndsenhetens hemsida.
    
    Varje år kräver dem även att vi gör en restaurangrapport där vi redovisar våra priser, vår omsättning samt hur mycket vi sålt volymmässigt. Priser har krögaren koll på, omsättningen finns i bokföringen så fråga föregående förvaltningschef och volymerna får räknas ut. Det görs ``smidigast'' genom att be föregående förvaltningschef att ta fram siffrorna för de verifikat som innehåller alkoholinköp (finns i bokföringssystemet). Se till att göra den i god tid och spara den inte till tentaperioden!
    
   \section{Teknikfokus}
    
    Sedan 2012 anordnas Teknikfokus tillsammans med både D-sektionen. Det finns ett kontrakt i VIP-pärmen. Kontraktet ska omförhandlas varje år så se till att efter Teknikfokus hålla ett möte med Ordförandena, ENU-ordförandena, samt Teknikfokusansvariga och förhandla fram ett nytt kontrakt som ska gälla för nästa år. För råd och tips finns även här gamla och dryga Ordföranden och ENU-ordföranden att fråga.
    
    Mycket kring Teknikfokus är väldigt oklart. Bland annat budgeten, försök att se till att budgeten är bra och rimlig. Ha god kommunikation med teknikfokus och bjud gärna in ansvarig till styrelsemöten om ENU finner det lämpligt, för ökad transparens och för att visa vilka resurser som finns tillgängliga.
    
    \section{Större jubileum}
    
    2015 infördes in fond kring $5n$-jubileum. Se till att varje år sätta in pengar i denna fond så att det finns pengar till jubileum. Både 2012 och 2017 kostade jubileet cirka 50kkr.
    
    \section{Slutord}
    
    \subsection{Om man inte hinner med - stress}
    
    \begin{dashlist}
        \item Hjälps åt! Då mår hela styrelsen bättre och ingen blir besviken på varandra eftersom man trodde att alla gjorde sitt jobb.
        \item Uppmuntra alla att vara öppna med om man har svårt med att hinna med skolan i en period. Stötta varandra i det!
        \item Försök att hålla koll på din styrelse. Om du ser att de ser väldigt stressade så försök att hjälpa till, se även till att det är okej att ta det lite lugnare med styrelseposten om saker sker privat eller i skolan.
        \item Var klar och tydlig redan från början med styrelsen att det är bättre att fråga om hjälp än att strunta i att göra något.
    \end{dashlist}
    
    \subsection{Ledarskap}
    
    \begin{dashlist}
        \item Hitta din egen teknik.
        \item Ta inspiration av dina företrädare och Ordföranden från andra sektioner.
        \item Håll koll på härskartekniker, både från dig själv och från andra styrelsemedlemmar. Det skapar bara dålig stämning, så var påläst om det och lär dig stoppa det.
        \item Håll hårt i att följa talarlista och försök inkludera alla i diskussionerna.
        \item Ha stenkoll på mötesformalia, det blir enklare för dig att leda när du har en bra struktur på mötena.
        \item Viktigast av allt är att du blir trygg i din roll som ledare. Försök inte att göra dig till någon du inte är.
    \end{dashlist}
    
    \subsection{Gå av och lämna över}
    
    \begin{dashlist}
        \item Använd hela december för att lämna över. Se till att varje utskott har något som de kan göra.
        \item Dela ut representation till de som gått på sittningar.
        \item Testamente, se till att alla lämpliga poster gör det.
        \item Hjälp till med att boka Lophtet till Skiphtet.
        \item Städa Edekvata innan jul!
        \item Hjälp Electus med att samla in verksamhetsberättelser till kommande vårterminsmöte från din styrelse.
        \item Se till att kontinuerligt skriva ner saker som du tycker saknas i testamentet. Det är välidigt givande då det är lätt att glömma bort funderingar och osäkerheter man hade i början av året. 
    \end{dashlist}
    
    \newpage
    
    \section{Hälsningar}
    
    Tveka aldrig att fråga någon gammal Emeritus om du undrar över något!
    
    \vspace{1ex}
    
    Till sist: Gör ordförandeåret till ditt år. Det är ofta lätt att snegla på tidigare år för att efterlikna det som verkat fungera eller för att saker är ``tradition''. Visst finns det mycket att hämta från historien men långt ifrån allt var bättre för. Jag har fullt förtroende att du och din styrelse vet vad som är bäst för just er situation. Lycka till!

    -- \emph{Edvard Carlsson, Ordförande 2019}
    \vspace*{1ex}
    
    Till sist: Ta vara på tiden som Ordförande. Det kommer vara både utmanande och krävande av dig men det kommer även vara otroligt kul och givande. Se till att ha det roligt med din styrelse, anordna många häng stora som små, åk på så många resor du kan och ta hjälp av varandra. Tvivlar inte en sekund på att jag lämnar över Sektionen i säkra händer. 2019 kommer bli fantastiskt!

    -- \emph{Daniel Bakic, Ordförande 2018}
    \vspace*{1ex}
    
    Till sist: Se till att ta vara på ditt år som Ordförande -- det kommer vara fantastiskt roligt och utvecklande! Det kommer också vara utmanande ibland, men du kommer fixa det galant. 2018 kommer bli super! Och juste, se till att vinna Regattan...
    
    -- \emph{Erik Månsson, Ordförande 2017}
    
    \vspace*{1ex}
    
    Till sist: 2017 kommer att bli ett toppenår för E-sektionen. Se till att ha riktigt roligt och ta vara på alla möjligheter du får under året.
    
    -- \emph{Fredrik Peterson, Ordförande 2016}
    
    \vspace*{1ex}
    
    Till sist: Att vara Ordförande är nog något av det roligaste man kan göra. Var ödmjuk för uppgiften så kommer det gå väldigt bra under dit år. Se till att göra det bästa av tiden och ha kul under tiden.
    
    -- \emph{Hanna Nevalainen, Ordförande 2015}
    
    \vspace*{1ex}
    
    Till sist vill jag bara säga: Se till att ha roligt under din tid som Ordförande. Det kommer vara både ett väldigt krävande år men framföralt ett helt fantastiskt år. Njut av tiden!
    
    -- \emph{Sara Gunnnarsson, Ordförande 2012}
    
    \newpage
    
    \section{Uppdaterat av}
    
    \begin{description}[noitemsep, itemsep=1mm]
        \item[2019] Edvard Carlsson
        \item[2018] Daniel Bakic
        \item[2017] Erik Månsson
        \item[2016] Fredrik Peterson
        \item[2015] Hanna Nevalainen
        \item[2014] Elin Bonnevier
        \item[2013] Johan Westerlund
        \item[2012] Sara Gunnarsson
        \item[2005] Ann Åkesson
    \end{description}
    
    
    \end{document}
    