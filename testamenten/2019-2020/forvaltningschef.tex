\documentclass[10pt]{article}
\usepackage[utf8]{inputenc}
\usepackage[swedish]{babel}

\def\post{Förvaltningschef}
\def\date{2019-12-30} %YYYY-MM-DD
\def\docauthor{Henrik Ramström}

\usepackage{../e-testamente}
\usepackage{../../e-sek}

\begin{document}
\heading{\doctitle}

\section{Introduktion}
Grattis till posten som Förvaltningschef! Som Förvaltningschef på E-sektionen kommer du att ansvara för förvaltningen av Sektionens ekonomi, lokal och inventarie. Du kommer vara ansvarig för ett eget utskott och dess funktionärer men även vara med i en styrelse där du kommer kunna påverka hela sektionen. 

Detta dokument kommer förhoppningsvis ge dig en idé om vad som väntar och min förhoppning är att du under året ska kunna använda det som ett slags uppslagsverk och kontinuerligt lägga till information så att vi så småningom har en väldigt bra grund för blivande Förvaltningschefer att stå på.

Den viktigaste punkten med hela detta testamente är att du har väldigt mycket med väldigt spridda uppgifter som ingen kan förbereda dig för. Så ta det mesta som det kommer och fråga om hjälp om det är något du vill ha hjälp med.

\section{Hur är dokumentet uppdelat?}
Idén med detta testamente är inte att direkt säga vad du ska jobba med utan ge dig uppmaningar till hur förvaltningschefer tidigare år har jobbat och olika tips som vi har kommit på genom åren. Alltså bör detta brukas som först ett intryck på vad posten innebär och sen ett uppslagsverk som du kan titta om det är något speciellt du vill veta något mer om. Se även till att själv skriva in saker genom året som du tänker att det har skulle ha varit skönt om det skulle ha stått med.

\section{Vad gör en Förvaltningschef?}
\subsection{Generellt}
\subsubsection{Utskottschef}
Som utskottschef för förvaltningsutskottet är det ditt jobb att se till att dina funktionärer utför qsina åtaganden, men även att de trivs med sin roll i utskottet. Det är också din uppgift att hantera konflikter som uppstår inom utskottet samt konflikter mellan ditt och andra utskott. 
En annan del av jobbet som utskottschef är att göra dina funktionärers röst hörd i styrelsen. Det kan exempelvis handla om större inköp (äskningar), val av funktionärer eller beslut om saker som behöver fixas. 
Som utskottschef ska du också godkänna alla inköp gjorda av funktionärer till FVU för sektionens räkning samt att de lämnat in kvitto mm. Eftersom det är du som ansvarar för utskottets budget kan det vara bra att uppmana dina funktionärer att alltid kolla med dig innan de köper saker. 
De senaste åren har detta däremot inte varit något problem och jag skulle snarare uppmana dig att uppmuntra dina funktionärers idéer om hur man ska förbättra sektionens lokaler mer än något annat. \newline

Förvaltningsutskottet är ett väldigt spretigt utskott och de olika funktionärsposterna har inte särskilt mycket med varandra att göra. Därför kan Förvaltningschefen tillsammans med de dina vice:s behöva anstränga sig lite extra för att få bra sammanhållning. 
Jag skulle rekommendera att redan innan du går på fundera och diskutera över hur ni vill uppnå detta (Kanske kick-off?) 
Till exempel så kan en stor del av utskottet engageras i lokalernas utformning och större projekt.

\subsubsection{Utskottet}
Ditt utskott består av följande poster: \newline \newline
\textbf{Förvaltningschef (Du)}
Utskottschef med det övergripande ansvaret för E-sektionens bokföring, ekonomi och lokaler. \newline \newline
\textbf{Vice Förvaltningschef} 
Bistår Förvaltningschefen i arbete med lokaler och inventarier samt ansvarar för lokalbokning och utskottets arbete.\newline \newline
\textbf{Skattmästare}
Ansvarar för den digitala bokföringen samt kontrollerar efter bokföringsfel.\newline \newline
\textbf{Arkivarie} 
Ansvarar för arkivering av sektionens dokument och klenoder.\newline \newline
\textbf{Hustomte}
Arbetar med underhåll och utformning av lokaler.\newline \newline
\textbf{Husstyrelserepresentant} 
För Sektionens talan i E-husets styrelse.\newline \newline
\textbf{Ekiperingsexpert} 
Ansvarar för inköp och försäljning av PR-artiklar och sångböcker.\newline

Alla dessa beskrivningar står även på hemsidan och i reglementet och bör uppdateras efter vad som står i reglementet och vad posten aktivt arbetar med.

\subsubsection{Styrelsemedlem}
Som styrelsemedlem kommer du tillsammans med resten av styrelsen jobba med sektionens övergripande verksamhet så som ekonomi, styrdokument, sektionsevenemang mm. Styrelsen blir som ett eget utskott där ni troligen kommer har riktigt roligt tillsammans (framförallt kommer ni lära er städa).

Inom styrelsen så är du framförallt högst ansvarig för ekonomin och därmed kommer du behöva jobba aktivt med att styrelsen sköter sitt ekonomiska arbete. För att kunna göra detta så lätt som möjligt har du en punkt på varje styrelsemöte “ekonomisk rapport”. Där kan du meddela större ekonomiska förändringar, allmänna tankar och idéer om ekonomin samt uppgifter till utskotten (till exempel att de ska lämna in kvitton som de har glömt).

Du kommer även delta i kollegiemöten med Pengakollegiet (PK) och vicekollegiet (VK) på kåren med de andra sektionernas motsvarande post. Kollegiemöten är ett bra tillfälle att diskutera med andra de olika sektionerna arbetar för att driva ett gott kontinuerligt arbete över året. Det är även en bra plats att diskutera igenom potentiella problem som har uppstått, samt så kommer en hel del nyheter om saker som har hänt på LTH senaste tiden.

\subsubsection{Kassör}
Som sektionens kassör kommer det vara din uppgift att se till att sektionens ekonomi förvaltas på korrekt sätt vilket bland annat innebär att betala fakturor i tid, kontrollera utgående fakturor och se till att alla transaktioner genomförs, redovisas och bokförs korrekt. Du ska även kontinuerligt informera styrelsen om sektionens ekonomi, gärna vid varje styrelsemöte. \newline \newline

Du kommer även ha till uppgift att utföra hel- och halvårsbokslut samt redovisa dem och sektionens allmänna ekonomiska status för sektionen vid både vår och höst-terminsmötena. Det är också din uppgift att se till att styrelsen är införstådda i deras uppgift och ansvar gällande sektionens ekonomi och bokföring. Detta brukar förslagsvist göra genom först en ekonomisk utbildning vid början av året (detta kommer jag att hjälpa dig med) men även genom kontinuerlig uppföljning. \newline
Ett tips när det kommer till det ekonomiska arbetet är att vara väldigt hård med allt de lägger in i början av året. Detta både för att alla ansvariga ska få in ordentligt vilka fel som man kan göra. Detta gör att de lär sig och du kommer kunna slippa timmar av arbete från både att det står fel i bokföringen och dessutom kommer du sprida ut arbete jämnare mellan alla olika personer i styrelsen. Klassiska exempel på fel folk gör är, fel kontonummer Fel typ av vara (t.ex. bokför mat som förbrukningsmaterial, ingen resultatenhet, inget projekt, glömt att skriva under, kvitto på fel sida eller att det täcker någon del på verifikatet som någon måste skriva på.
\newline \newline
Utöver att du ansvarar över det rent ekonomiska kommer också vara en av sektionens firmatecknare d.v.s. att du tillsammans med Ordförande kommer vara ansvarig för sektionens alkoholtillstånd, avtalstecknande med företag samt hanteringen av sektionens banktjänster. Eftersom sektionens firma tecknas av två i förening kommer alla dokument och bankuppdrag behöva signeras (fysiskt eller digitalt) av både dig och ordförande. Detta är något som ni kommer behöver göra kontinuerligt så se till att ni kommer in i en rytm av att gå in på swedbank och se till att betalningar skickas med jämna mellanrum.

\subsubsection{Lokalansvarig}
Som sektionens lokalansvarig kommer det vara din uppgift att se till att sektionens lokaler är i bra och brukligt skick. Huvudsyftet för våra lokaler är att erbjuda goda studiemiljöer, men mer och mer event genomförs numera i våra lokaler så därmed ska du se till att dem är anpassade för alla. 
Du kommer behöva se till att saker som går sönder fixas och att lokalen är i det skick den behöver vara för att sektionen ska kunna utföra sin verksamhet. Mitt tips vad gäller detta är att se till så att din vice har en lista med saker som ska utföras. Du kommer lägga så pass mycket arbete på annat så se till att “lokal-delen” i ditt utskott också har lite saker att göra. ;)

Du kommer ha kontakt med huset framförallt genom din husstyrelserepresentant och PH (Per-Henrik Rasmussen nere i vaktmästeriet). Det är jätteviktigt att hålla en bra relation med huset framförallt PH och Mats Cedervall (Husprefekten) då de är otroligt hjälpsamma om vi har problem med något. (Mats Cedervall kommer också att gå i pension till våren så se till att bli god vänn med den nya också). Vi brukar även försöka styra en träff så snart som möjligt efter att ni i styrelsen går på för att dem ska få ett ansikte på er alla.

Utöver detta är det även du som sköter lokalutlåningar och utrustnings utlåningar. Detta gör man vanligtvist genom att skriva ett kontrakt som beroende om det är lokal eller utrustnings ska skrivas med ett visst kontrakt. 
Något som bör sägas om detta är att det på senare tid har blivit mer och mer problem med huset vad gäller våra lokaler. Ett exempel på detta är att huset för tillfället endast tillåter sektioner från E-huset att låna Edekvata för sittningar. 

\subsection{När man går på som förvaltningschef}
Så tidigt som möjligt på ditt verksamhetsår så ska de nya firmatecknarna gå till banken och fylla i papper så att firmatecknarna byts och så att de nya har teckningsrätt för sektionen. E-sektionen använder sig av två firmatecknare i sin förening och att båda måste skriva under för att ett kontrakt ska vara godkänt. 
När ni gör detta, se till båda har med giltig legitimation, samt ta med er det protokollet där nya firmatecknare har valts (underskrivet). Hör med Ordförande om vem som ringer Sparbanken Ideon Gateway och bokar tid.

När ni är där så ska ni även se till att beställa bankkort till de funktionärer som ska ha. Bankkort kostar lite pengar men är bra då det underlättar bokföringen och gör så att folk slipper ligga ute med pengar. Korten är personliga vilket gör att de måste beställas i början av varje år. Under mitt år så hade:
SRE-ordförande

Sexmästare

Entertainer

Øverphøs

Vice krögare

Cøl
En inköps och lagerchef

Förvaltningschef

Ordförande

Ekonomiskt ansvarig i phøset

Vice sexmästare

Köksmästaren

Kontakta iZettle och skriv att ni vill byta kontoägare på E-sektionens konto. Bifoga justerat VM-protokoll och foto på den nya kontoägarens legitimation. Uppdatera även personaluppgifterna på iZettles hemsida. (Detta har inte gått de senaste åren, men det är värt ett försök i alla fall)

Ordförande ska se till att ni har access till studiecentrum där vi deponerar sedlar, kan vara bra att fixa tidigt eftersom LU-kort kan vara sega. (Används heller inte jättemycket numera då vi inte får in så mycket kontanter. Men kan vara värt att titta in på)

Det är bra att ha en ekonomiutbildning med styrelsen så de lär sig sin del av bokföringen. Ett tips här är att ta hjälp av den förra förvaltningschefen eftersom man i början inte vet allting om allt, då det är otroligt mycket att hålla koll på. 
Preliminära internbudgetar ska skrivas för alla utskott och gås igenom med hela styrelsen. Mitt förslag är att ta fram detta tillsammans med utskottsordföranden och skapa denna utefter vad folk se till att alla gör det i början av året! Detta hjälper även med att få de olika utskotten att få koll på vad de har i sin budget och vad man kan tänkas lägga sina pengar på.

Samla in nycklar från avgående funktionärer och dela ut nycklar till de nya funktionärerna. Det finns en liten kassa i lilla kassaskåpet där deponiavgiften på 100 kr förvaras. Ett alternativ är annars att använda swish till sin företrädare och helt enkelt bara ta över nyckeln från den.
Vanligtvist här så ser man även till att göra en inventering av de nycklarna vi har på knippan. Se över om det är ont om några och se i såna fall till att beställa nya!

Se till att antingen du själv eller ordförande (gärna båda) går C-cert utbildningen som är nödvändig för att vi ska få bedriva verksamhet med fast alkoholtillstånd. Om bara en av er kan gå den rekommenderar jag att den andra går iallafall B-cert. Det finns tidigare års C-cert utbildning på driven om du vill läsa på lite innan.

Se till att krögartrion + sexettrion går A-cert och B-cert (prioritera Krögare/Sexmästare vid få platser, därefter Vice Sexmästare eftersom det är mer troligt att de kommer håller i verksamhet själv än att Vice Krögare gör det). Även Entertainern bör gå med tanke på Utedischot.
Se till att våra normala serveringsansvariga är anmälda d.v.s. krögartrion och sexettrion samt att de gamla är borttagna. 
Det går att se här: https://www.lund.se/foretagare/tillstand-regler-och-tillsyn/serveringstillstand-alkohol/

En sista sak som är bra att göra är att läsa igenom om sektionen. Delvist vad det är som det står enligt reglementet och stadgan att du ska göra men även att du läser igenom vad det är som de andra du kommer jobba med ska göra. Det är stadgan och reglementet som bestämmer hela vår verksamhet och därmed så om det är något som folk inte är överens om så bör man utgå från dessa dokument för att bestämma vad som skulle vara bäst för sektionen.

\section{Vardagligt arbete}
\subsection{Varje vecka}
\subsubsection {Ta emot brev}
Vi har två postfack, ett på kåren och ett i vaktmästeriet. Det är egentligen Ordförandens uppgift att ha koll på dessa men det kan vara bra att veta om och att hjälpa honom med detta om hen har mycket att göra.

\subsubsection{Betala fakturor}
Att betala fakturor görs precis som för dig som privatperson. Vi har flera tidigare företag inlagda, försök att utnyttja dessa. Om de inte finns så måste du använda företagsdosan för att först logga in och sen lägga till ett nytt betalningsställe. Följ bara instruktionerna så borde detta gå bra. \newline
Efter detta så ska fakturorna också bokföras. Detta görs på de gula bokföringsorderna i HK. Här måste rätt transaktionsdatum stå på och bokföringen måste ske genom god praxis.


Detta ska sedan lämnas till skattemästaren för att sen sättas in i vår internettjänst fortnox. \newline

Hur man bokför är något som kan ta lite tid att lära sig i början. Se till att bara ta det lugn och göra det i din egen takt. Det tar mycket längre tid att korrigera fel än att göra rätt, så se till att göra rätt första gången. Jag kommer även att hjälpa dig samt bokföra en del med dig i början för att du ska få en ordentlig överlämning. Sen så finns även dina gamla förvaltningschefer som hjälp till dig igenom allt du gör.

\subsubsection {Skicka och bokföra fakturor: }
Vi använder även internet tjänsten fortnox för att skicka fakturor. I denna så skriver man in vilken kund det är som ska betala, vilket konto det är som ska få in pengarna från utbetalningen och hur mycket som ska betalas. 

När det kommer till att skicka fakturor så finns det massor med saker att tänka på, och vissa specialfall på det också. Men för det mesta så går det att bara skicka till någon ekonomiskt ansvarig som redan är inlagd.
Däremot så kan man inte längre skicka pappersfakturor till myndigheter längre vilket inkluderar också Lunds universitet. Alltså så måste man skicka in en E-faktura. Lyckligtvist finns all information om hur man fakturerar på olika sätt bra beskrivet på fortnox hemsidan.

När du sen har gjort färdigt fakturorna, bokförs sen fakturan två gånger: 
En gång när man skickar iväg fakturan och en gång när man får in betalningen. Vi kommer gå igenom detta tillsammans nu i slutet av mitt år och sen så går det alltid bra att fråga hur man gör sen efter att du har gått på. Men kort och gott så bokförs först en kundfaktura (tillsammans med fakturan) när man skickar ut en betalning till t.ex. ett företag och sen efter att företaget har betalat till vårt konto så bokför man detta i en kundinbetalning (tillsammans med både fakturan och inbetalningen). 
Inbetalningar av våra fakturor registreras under “Fakturering”, “Inbetalningar” i Fortnox. 
Andra inbetalningar bokförs som vanliga bokföringsordrar.
\newline

\subsubsection{Kontrollera/skriv under verifikat från andra utskott:}
Saker som kvittoförstärkning och utlägg får du in från andra utskott. Dessa ska du få i ditt fack och då ska du kontrollera så att allt är rätt, skriver transaktionsdatum, skriv under och lämna till Skattmästaren (om det är ett utlägg så genomför du det såklart). Om de inte är rätt så brukar jag skriva en lapp och lägga dem i lämpligt fack. Det kan vara bra att vara lite extra hård med detta också i början så att folk lär sig, då blir det mindre jobb för dig resten av året. Detta är troligtvist en av sakerna som du kommer lägga mest tid på så se till att du inte hamnar allt för långt efter under skolåret, speciellt så att du slipper mängder med extra bokföring under tentamens veckorna.

\subsection{En gång i månaden}
\subsubsection{Stäm av konton}
Genom året kommer du ansvara för avstämning av konton. Detta görs för att man ska genom året ha koll på att rätt betalnignar har skett och att man inte har bokfört något flera gånger. 
För att stämma av ett konto så går man in i fortnox -> bokföring -> Stäm av konto, efter detta så skriver man in vilket konto man vill stämma av och sen efter detta så matchar man verifikat med den motsvarande.
 Detta tar väldigt lång tid men kan användas för att hitta potentiella fel i bokföringen som annars skulle ta ännu längre tid. 
Det rekommenderas att du varje månad kollar av:

\begin{itemize}
    \item 1510, Kundfodringar
    \item 1910,	Handkassa
    \item 1920,	Bankkonto
    \item 1930,	Bankkonto 10års fonden
    \item 1940,	Sparkonto
    \item 1981,	Inkommande kontantfria inbetalningar
    \item 1981,	Inkommande kontantfria inbetalningar
    \item 2830,	Vidarefaktureringar
\end{itemize}

Övriga tillgång- och Skulder-konton bör stämmas av en gång i halvåret.

\subsection{Övrigt}
\subsubsection{Bokslut (se bokföring och ekonomi):}
För att göra ett hel eller halvårsbokslut så går man igenom hela den perioden som man vill göra ett bokslut på, lägger in alla resultat från alla verifikat och sen så kollar man att alla dessa verifikat är avstämda enligt fortnox. Efter detta så tar man ut resultaten från denna perioden och skriver över dessa in i balansrapporten (detta görs automatiskt av fortnox efter att du har skrivit ett bokslutsverifikat. Till sist så redovisar man denna balansräkning och plockar ut de resultaten som står i rapporten och lämnar tillsammans med detta en förvaltningsberättelse där man beskriver intressanta saker som har hänt under perioden. Andra saker så som trender under de senaste åren och hur det har gått i förhållande till budget kan också vara intressant att nämna.

\subsubsection{Access:}
Ordförande ska maila LU-kort och se till att ni har access till de lokalerna i E-huset som vi inte ansvarar över (klockrummet). Det kan vara bra att dubbelkolla detta eftersom de kan vara lite sega. \newline

All access som vi som sektionen däremot delar ut är det du som ansvarar för. Detta inkluderar alla rum med assa abloy lås på samt nycklar. När det kommer till accessen med LU-kortet så har vi en modul på 
\url{https://eee.esek.se/doors_admin} 
Denna är det däremot vanligtvist macapärerna som ansvarar över och därmed så bör du istället för att gå in själv och mixtra med, skriva en notis till dessa och se till att de istället ändrar accessen åt dig.

\subsubsection{Bonning:}
Under sommaren bonar de golven i stora delar av E-huset. Frågar man PH snällt kanske de kan ta delar av Edekvata då också, vilket är nice. Fråga honom gärna tidigt på våren så att vi får en tid.

\subsubsection{Sektionsmöten:}
På sektionsmötena ska Förvaltningschefen sköta en del raportering av ekonomin. Ett knep är att titta på gamla möteshandlingar för mer info om vad som ska ingå för respektive möte. Men kortfattat så ska du lämna en ekonomisk rapport där det ska stå hur det går för sektionen rent ekonomiskt, redovisa en utskottsrapport om vad ni har gjort i utskottet sen senaste sektionsmötet (detta kan vara lite extra viktigt för FVU att det blir tydliggjort hur mycket vi gör för sektionen), skriva en verksamhetsuppföljnings där ni följer upp det kontinuerliga arbetet inom utskottet över åren m.m. 
På den ekonomiskar rapporten ska värdena från fortnox (balansrapporten) redovisas och det stående resultatet i skrivandets skull redovisas.
Du ska sen på vårterminsmötet lämna ett förslag tillsammans med din styrelse på resultatdispositionen. Denna säger vart vi ska placera våra fonder efter resultatet från tidigare år.
Du ska även lämna resultatet från alla uttag från sektionens fonder sedan senaste mötet.
På HTM så gör man en verksamhetsplan för hur förvaltningsutskottet ska ägna sin på nästa år också, alltså är det en långsiktig plan för hur utskottet ska jobba för sektionens framgång.
På HTM så lämnar du också ett budgetförslag tillsammans med styrelsen där du beskriver hur man borde fördela budgeten till nästa år.

\subsubsection{Bokföring och ekonomi}
Detta är som sagt den största delen av ditt genomgående arbete genom året så allt kommer omöjligt kunna tas upp här. Därför så ses vi några gånger så att en del praktiskt kan tas där istället! 
Ett knep här är däremot att kolla igenom förra årets bokföring om du är osäker eller kolla med en emeritus. De flesta är glada att hjälpa till!

\subsubsection{Bankkorten}
Diskutera med Ordförande och eventuellt övriga styrelsen vilka som behöver bankkort. Korten beställs sedan via internetbanken eller då ni skriver på som PBI. 2019 skrev vi kontrakt med alla som fick bankkort, dessa sitter i nyckelpärmen i kassaskåpet. Vid årets slut bör man samla in alla bankkort. Sedan efter att ni har fått in era nya bankkort så borde man klippa de gamla samt avaktivera dem i banken.
Man kan ställa in maximalt månadsbelopp i internetbanken, under 2019 har 25 t.kr. varit standard, men vissa har fått 40 t.kr. då dessa ofta genomför större inköp. 
Under nollningen går det åt mycket pengar så det kan vara en idé att sätta denna högt för vissa redan från början eller ändra inför nollningen så att man inte står där med ett stort inköp och inte kan betala. 
Sexet fick exempelvis sin gräns höjd under nollningen 2019 då de höll i många sittningar och maxade ut sina beloppsgränser. Gränsen kan alltid ändras i efterhand om det behövs. 

2019 hade Förvaltningschefen, Ordförande, Sexmästaren, Entertainer, Øverphøs, SRE-ordförande, en vice krögare, ekonomisk ansvarig i phøset, vice sexmästare, en köksmästare, Krögaren, ett cøl och en lager och inköp haft bankkort. 
Tidigare år har sett ungefär likadana ut eller varit lite mindre, i år med skillnad att vice sexmästare och köksmästaren har haft ett kort och sen så har även ekonomiskt ansvarig i phøset fått ett kort för att underlätta för Øverphøset. Detta är lite upp till dig som förvaltningschef i samspråk med ordförande och berörda utskott att besluta om. 

\subsubsection{Nycklar}
Sektionen har nycklar till diverse saker, bl.a. kassaskåp, köket, spritförrådet m.m. som funktionärer behöver för att utföra sina uppdrag. Nycklarna kvitteras mot en deposition på 100kr samtidigt som ett nyckelkontrakt upprättas. Nycklar och kontrakt förvaras i kassaskåpet och det är att föredra om endast du hanterar utlämning av nycklar, men du kan även låta andra i styrelsen göra det.

Förra året så använde vi oss även av swish där man swishade 100kr till sin företrädare för att underlätta arbetet. De flesta tycker att det är lättare att använda sig av swish istället för ärligt talat, vem använder sig av cash numera?

\subsubsection{Dörraccess}
Sedan ett par år tillbaka hanteras funktionärernas access till våra utrymmen via en modul på hemsidan. Förvaltningschef brukar avgöra vilka som ska ha access vart, förmedla detta sen till macapärerna som lägger in det i systemet. Värt att tänka på är att revisorerna brukar ha access överallt och att alla som ska städa behöver ha access till Sicrit för att komma åt städredskapen. 

Tidigare år har även för många haft tillgång till HK. Detta resulterar vanligtvist i att HK inte sköts ordentligt och därmed får du extra arbete med att städa upp efter alla andra. Se därmed till att de personerna som har access dit sköter lokalen.

Tänk på att vissa av föregående års funktionärer kan behöva access ett tag under nästkommande verksamhetsår (skattmästare, delar av styrelsen). Utöver funktionärernas vanliga access kan du lägga till access manuellt för tillfälliga access. 

Ett tips är att rensa bland de manuella accessarna kontinuerligt. Det enda undantaget är för förvaltningschefer emeritus då de behåller all access studietiden ut, likt krögartrions krögarklägg.

Access till kårens lokaler fixar man genom att fylla i formulär på tlth.se/access. Som Förvaltningschef är det bra att ha access till expen/hänget. Bastun kan också vara najs att ha. Accessen i kårhuset försvinner efter ett halvår så glöm inte att fixa ny innan till exempel nollningen.

\subsubsection{Kassaskåp}
Översta lådan i det stora kassakåpet är handkassan. Glöm inte att noga skriva upp alla in- och uttag i handkassepärmen. I kasskåpet finns en nyckelknippa med nyklar till Sektionens alla lås. Här inne finns även kopior av nycklar som lämnas ut till funktionärer vid behov. Glöm inte kontrakt och ta deposition vid utlämning.

Under handkasselådan finns några pärmar som är bra att känna till. Bland annat viktiga-papper-(VIP-)pärmen och nyckelkontrakt-pärmen. 

I det lilla kassaskåpet förvaras bl.a. Sektionens laptop, iZettle-dosor och iPads. Ändra gärna koden årligen så att bara de som verkligen har behov att använda skåpet har tillgång till det. 

I Ullas rum finns det ett litet kassaskåp som Cafémästaren lägger in växel i för kommande dag. Se gärna till att koden till även denna ändras årligen. De som bör ha tillgång till den koden är förutom Cafémästaren, Ordförande och (om det finns en) Ulla.

I nattfack (deponeringsboxen under mikrovågsugnarna i Blå Dörren) läggs växel och kontantintäkter från caféet och gillen. De ska sedan kontrollräknas av Cafémästaren respektive Förvaltningschefen innan pengarna läggs in i kassaskåpet igen. 
Reservnycklarna som hör till detta skåp tror jag ska finnas i stora kassaskåpet.

\subsubsection{Alkoholtillstånd}
Du kommer troligen behöva söka alkoholtillstånd för event som behöver utökat tillstånd eller allmäntillstånd. Om ni ska söka tillstånd för en dag/lokal som vårt fasta tillstånd inte täcker ska ni fylla i ansökan om tillfälligt tillstånd för slutet sällskap (eller allmänt om det är fallet) annars är det ansökan om tillfälligt utökat tillstånd som gäller.

I samband med att ni mailar in ansökan ska 1400kr i avgift betalas in till tillståndsenheten och transaktionskvittot ska bifogas i mailet. Det ska även sökas speciella tillstånd för dryckesprovning eller för egen kryddning av sprit, vilket Krögaren kan tänkas få för sig att göra inför julgillet. 

Tillstånd ska sökas senast två veckor innan arrangemang för slutet sällskap och fyra veckor innan arrangemang för allmänheten. Ansökan ska skrivas under av den firmatecknare som har kunskaper inom alkohollagen. Det är även den här personen som ska se till att den serveringsansvariga är en lämplig person. 
Nytt sedan hösten 2018 är att inför varje ansökan om alkoholtillstånd i E-Huset, vill tillståndsenheten ha ett godkännande ifrån husprefekten, Cedervall, att vi har dispositionsrätt för lokalen i fråga. Detta gäller även Edekvata av någon oundgriplig anledning. Tillståndsenheten får ofta för sig lite hyss då och då, så det är alltid bäst att ringa och fråga en extra gång då saker antagligen inte är vad de förr varit, trots att detta kanske gick bra för endast ett par veckor sedan. 

Mer information kring allt detta kommer även komma på C-cert och finns i driven.

\subsubsection{Serveringsansvarig}
För att få vara serveringsansvarig så måste man ha fyllt minst 20. När tillståndsenheten kommer på besök är det den här personen de kommer att fråga efter. Den serveringsansvariga måste ha tillgång till alla rum i Edekvata om tillstånd vill undersöka dem. Alla jobbare ska veta vem som är serveringsansvarig för kvällen. Serveringsansvariga kan rapporteras in på tillståndsenhetens hemsida. 
Vårt användarnamn är organisationsnummret och lösenordet är s67njPhb. Har ni frågor om vad som gäller kan ni fråga mig eller tillståndsenheten.

I början av varje år ska de nya firmatecknarna anmälas som PBI:er (Person med Betydande Infytande) hos tillståndsenheten genom att skicka in en blankett. (Detta är gjort för verksamhetsår 2020). Även serveringsansvariga behöver anmälas till tillståndsenheten och detta görs via deras hemsida. Avgående ordförande brukar få påminnelsemail om detta innan man avgår.

De brukar även kräva att man har en firmatecknare med tillräcklig kunskap inom alkohollagen. För att man ska tillräcklig kunskap inom alkohollagen krävs C-certifikat i ansvarsfull alkoholhantering. 
Inbjudningar till A- B- och C-cert brukar även de komma till avgående Ordförande innan året är slut. C-cert kan även få skrivas direkt som prov hos tillståndsenheten, det är bara att kontakta dem.

\subsubsection{Resturangrapport}
Varje år ska vi göra en restaurangrapport där vi redovisar våra priser, vår omsättning samt hur mycket vi sålt volymmässigt. Priser har Krögaren koll på, omsättningen finns i bokföringen så fråga föregående Förvaltningschef och volymerna får räknas ut. Det här kommer vi göra tillsammans med avgående och pågående krögartrion.

\subsubsection{Bokslut}
Ett bokslut innefattar att man redovisar den summerade versionen av sektionens bokföring under det gångna halv/helåret. D.v.s. resultatrapporter för alla resultatenheter, projekt och för hela sektionen samt en balansrapport på hela sektionen. 

För årsbokslutet är det högst rekommenderat att du gör en avstämning mellan resultatenheterna och budgeten.
Dock är där några saker som ska vara klara för att rapporterna ska vara “bokslut”. 
Först och främst måste alla transaktioner under perioden vara bokförda och kontrollerade av revisorerna (revisorerna är inget måste till halvårsbokslut men rekommenderat). Sedan ska alla 1000- och 2000-konton stämmas av och lagerna (CM, E-shop och alkohollagret) och handkassan inventeras. 
Cafémästaren, Ekiperingsexperterna och Cølen sköter inventeringen av respektive lager. Du bör få inventeringen av E-shops lager och alkohollagret i form av kalkylark. Summorna där ska sedan räknas om till lagervärde med hjälp av inköpspriserna och sedan ska Fortnox uppdateras utifrån resultatet. Vi kommer gå igenom detta noggrannare inför halvårsbokslutet. 

\subsubsection{Viktigt att tänka på}
Som firmatecknare är du ytterst ansvarig för all alkoholverksamhet. Det gör att det kan vara värt att hålla ett vakande öga på vad sexmästeriet (och KM, men de brukar sköta sig) håller på med. Se till att de följer alkohollagen och att de sköter sig. Speciellt när det gäller prissättningar på alkohol.

Försök få resten av styrelsen att förstå hur viktigt det är att de sköter sin ekonomi. Det kommer underlätta både ditt och deras arbete. Se till att alla utskott gör en internbudget och
att de fastställs på ett styrelsemöte. Under året 2019 så gjordes inte detta och pga av detta så genomfördes stora debatter kring budgeteringen som troligtvis inte skulle ha gjorts om en klarare internbudget hade genomförts. 
Hela styrelsen är solidariskt ansvarig för hela sektionens verksamhet. Detta innebär att ENU-Ordföranden är lika ansvarig för NollU:s ekonomi som Øverphøset är.

Se till att folk med kassaskåpsnyckel (Ordförande, Cafemästare och du) sköter handkassan felfritt. Det otroligt svårt att reda ut vad som har hänt i efterhand om man inte skriver upp allt i handkassapärmen och fyller i bokföringsunderlag direkt. 

Relationen med huset är otroligt viktig. Se till att du hamnar på god fot med PH och Cedervall. Det kan komma att rädda ditt skinn många gånger.

När ni skickar fakturor till SVL/Programledningen se till att deltagarlistor och kopior på kvitto alltid bifogas med fakturorna.

Gången från entrén till Edekvata fram till nödutgången i Biljard är utrymningsväg och det får därför inte stå saker där. De lokalerna får heller inte användas för förvaring på något sätt eftersom det kan vara en brandfara, vilket kan vara spännande under nollningen då alla uppdragsgrupper måste ha sina byggen någonstans.

Jobba kontinuerligt med bokföringen och få rutiner för vad som ska göras varje vecka. Det är bra att vara någorlunda i fas då det är mycket jobbigare att t.ex. jaga kvitton och sånt flera månader efteråt. Jag t.ex. gick igenom alla transaktioner i bokföringspärmen varje månad, gick satte mig en eller två gånger i veckan och bokförde allt som var inlagt. Detta var ett system som jag fann att det räckte.

\subsubsection{Braig/avig}
Inför varje arrangemang i E-huset behövs en brandansvarig och en ansvarig. (Går att ha samma person som båda två.) Dessa ska i god tid innan arrangemanget anmälas till PH, minst en vecka. Var tydlig med att alla som har evenemang (gillen, sittningar osv) i huset ska ha en braig och avig. För att hålla PH på bra humör är det som sagt viktigt att dessa anmäls i god tid. För anmälan måste man ha namn, personnr, sektionstillhörighet och bild på avig/bravig. När man bokar FikaFika via hemsidan sköts detta automatiskt.

\subsubsection{Tips/trick}
Försök ställa upp veckorutiner med det dagliga arbetet istället för att ta tag i allt samtidigt. 

Försök ”stänga” en vecka i taget. Det vill säga, bokför alla transaktioner för en vecka och kolla därefter så att saldot på banken stämmer med det i bokföringssystemet. Detta gör det enkelt att hitta eventuella fel i bokföringen.

Ha för vana att tömma nattfack varje vecka. Deponera pengar på studiecentrum så ofta som du tycker är nödvändigt, men se till att inte ha för mycket pengar i kassaskåpet. Det är bättre att de är på banken. 

Om du behöver växla mynt så kan man göra det på ICA. Hör av dig till Annika Linander. Hon kan nås på Annika.Linander@kvantum.ica.se

Se till utskotten inventerar alla lager vid halv- och helårsslutet så att du kan göra halv och helårsbokslut. Är du osäker på hur man gör bokslut så hör med dina företrädare. De ska ha koll på hur man gör. Efter ditt år i styrelsen så ska du deklarera. Gör det med din efterträdare så att den lär sig hur man gör. Ta en kopia på deklarationen innan du skickar in den så att din efterträdare lätt kan ta reda på vilka fält som ska fyllas i.

Innan sommaren är det bra att se till att alla bokföringsblanketter är inlämnade och att alla lager är inventerade. Det gör det möjligt för dig att börja med halvårsbokslutet under sommaren och du riskerar inte att behöva vänta med det till hösten för att du saknar verifikat. 
Ha en kontinuerlig dialog med revisorerna. Det är alltid bra att vara på god fot med dem. Det är också bra om de uppmanas att revidera kontinuerligt. De har ofta rätt även om man själv inte alltid tycker det.

Sätt dig in i hur fonderna används. Om styrelsen vill göra budgetavsteg så är det viktigt att de görs på korrekt sätt. Är du osäker så hör med din företrädare eller revisorerna.

Har du inte redan gjort det så sätt dig in i stadgar och reglemente samt övriga styrdokument. Dessa styrdokument reglerar vad styrelsen får göra, vilka rättigheter och skyldigheter man har.

Ha en bra dialog med Lager- och inköpschefer om deras IC-rapporter. Se till att de görs kontinuerligt. Kontrollera dem extra noga innan du är säker på att de har koll på dem.
Se även till att E-shop gör sina försäljningsrapporter kontinuerligt efter varje gång de haft en försäljning, eller åtminstone att en rapport inte gäller flera veckor. Det underlättar nämligen väldigt mycket sen när man ska stämma av iZettlekontot (1981) eftersom en rapport då inte påverkar så många veckor. 

\subsubsection{Nollning}
Mycket iZettle, ha gärna en genomgång med de som ska använda det så att de vet att de ska fylla i försäljningsrapporter. 
Det är många som inte vet att de ska fylla i försäljningsrapporter. För att lösa detta så kan det vara bra att låta styrelsemedlemmarna vara ansvariga för att deras utskott lämnar in försäljnings rapporterna i tid. Det kan också vara bra att gå igenom detta noga med bl.a. ekiperingsexperterna och så småningom NollU inför hösten då dessa kommer till stor del sköta sin egna bokföring.

\subsubsection{Bra kontakter och nummer}
NOKAS har hand om deponeringen och är de vi kontaktar om vi t.ex. behöver nya deponeringspåsar. Vårt kundnummer hos dem är 354003228.

LU-kort, vi brukar ha kontakt med dem för att felanmäla dörrar och liknande. Det är dem ni vänder er till för access till studiecentrum (Men det ska Ordföranden fixa).
Mail: lukortet@lu.se

Ica Tuna, vi brukar växla mynt med dem.
Vår kontakt på ICA är Annika Linander. Mail: Annika.Linander@kvantum.ica.se

Tillståndsenheten, brukar vi ha kontakt med om vi har frågor angående alkohollagen eller tillstånd. Det är även de vi skickar in ansökningar om tillfälliga tillstånd till. På deras hemsida finns kontaktuppgifter till alla deras handläggare och enhetschefen.
Webb: https://www.lund.se/foretagare/tillstand-regler-och-tillsyn/serveringstillstand-alkohol/
Generell mail: tillstandsenheten@lund.se

PH, Per-Henrik är E-husets Husintendent och den personen som är lättas att vända sig till
med frågor eller ploblem gällande vår lokal eller E-huset generellt.
Han sitter borta vid tryckeriet och nås antingen där eller via mail: PH@ehuset.se
Mats Cedervall, Mats är E-husets Prefekt och den man kontaktar gällande bokningar av foajén eller utökade problem/problem som inte PH kan hjälpa till med. Tips är att gå via PH först eftersom det brukar vara den smidigaste vägen.
Mail: mats.cedervall@eit.lth.se

Akademiska Hus, vi brukar kontakta dem direkt gällande felanmälningar som gäller el, vatten eller liknande. Om du är osäker om det ska felamälas till dem eller inte kan man kolla med PH.
Webb: http://www.akademiskahus.se/vara-kunskapsmiljoer/forvaltning/felanmalan/

Fortnox, vår kontaktperson är Hanna Gothberg och hon kan hjälpa dig med avtalsrelaterade frågor. Vid vanliga supportfrågor är det lättast att kontakta den vanliga supporten.
Mail: hanna.gothberg@fortnox.se 

Förvaltningschef Emeritus
Har du frågor kan du alltid vända dig till mig :)
Mail: kramis@esek.se, henrik.ramstroom@gmail.com, mange@esek.se, magnus.lundh@gmail.com
Tel: 0707366011, 073-8145497

\subsubsection{Skriv ut växel/kontrollräkna KM: (Används inte längre då vi inte har någon handkassa längre)}
Inför gillen och sittningar där det ska betalas kontant ska du skriva ut en växelkassa. Jag brukar skriva ut ca 1500 kr, och sen lägga i en lapp i påsen så att de ansvariga vet hur mycket som är växel (viktigt!). Detta ska sedan läggas i lilla kassaskåpet och skrivas in i handkassepärmen. Efter eventet läggs pengarna in i nattfack i BD och då ska du kontrollräkna (inom ett dygn), skriva in i handkassan och lägga in det i kassaskåpet.

\subsubsection{Myntväxling: (Används inte längre då vi inte har någon handkassa längre)}
“Mynt kan vi ibland växla med Ica (se kontaktuppgifter) och om inte annat så får vi lämna in dem på banken, men då tar de en avgift på 5 procent. Då görs det på Swedbank Klostergatan. Man behöver inte boka tid utan går bara dit med pengarna (inte i rör) och så vill de veta vilket konto det ska in på. Dem tar däremot inte inte inbetalningar under en viss gräns (olika för olika perioder).” Nuförtiden så har man däremot dragit ner så mycket på mynthantering så att det är högst oklart vart vi kan lämna in mynt.



\newpage
\section{Avslutande ord}
Året som förvaltningschef är både roligt och utmanande. Det har funnits perioder som vi tidigare förvaltningschefer har jobbat över 80 timmar på en vecka för att få ihop det sista arbetet men ändå tror jag inte att någon som har suttit på posten har sagt att det ångrar sig. Se till att du njuter varje dag med det arbetet du gör och med vetskapen att utan dig så skulle sektionen aldrig fungera. 
Se även till att du själv tillsammans med alla andra på sektionen jobbar med det som är av sektionsmedelmmarnas intresse. Vi finns inte till för att tjäna pengar eller för att driva företag. Vi är en sektion som finns till för att förgylla våra sektionmedlemmars studietid med god studieövervakning och bra studiesocial miljö. Låt detta fortsätta färga sektionen genom att se till så att alla framförallt har det roligt under sina (allt för) få år på LTH.

Jag önskar dig slutligen lycka till. Äntligen är det dags att se dig Rasmus som Förvaltningschef!











\end{document}