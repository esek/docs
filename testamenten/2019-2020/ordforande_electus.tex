\documentclass[10pt]{article}
    \usepackage[utf8]{inputenc}
    \usepackage[swedish]{babel}
    
    \def\post{Ordförande Electus}
    \def\date{2019-11-26} %YYYY-MM-DD
    \def\docauthor{Edvard Carlsson}
    
    \usepackage{../e-testamente}
    \usepackage{../../e-sek}
    
    \begin{document}
    \heading{\doctitle}
    
    Hej Ordförande Electus!
    
    Grattis till ditt nya uppdrag som Ordförande för den bästa sektionen inom TLTH! Jag önskar dig stort lycka till under ditt kommande år och hoppas att du kommer känna att det är ett riktigt roligt och givande uppdrag!
    
    Du som nyss blivit vald till Ordförande på E-sektionen har lite framför dig i jobbet som Ordförande. En del av det behöver göras redan innan du går på posten officiellt! Allt nedan behöver inte göras innan du går på och det följer inte någon särskild ordning. \texttt{:)}
    
    \begin{numplist}
     \item Glöm inte att det är styrelsen electus ansvar att städa efter Valmötet!
    
    \item Det är viktigt att redan nu bestämma ett datum för KPL (Kurs på landet). Att hitta ett som passar alla kan vara svårt men givetvis ska prioriteringen vara att så många som möjligt från nya styrelsen ska kunna delta. Tidigare år har platsen varit hemlig för gamla styrelsen och de har fått följa en skattkarta eller liknande som Entertainern har gjort för att ta sig dit. 
    
    \item Även Skiphtet behöver ett datum vilket också är bra att bestämma så tidigt som möjligt, framförallt om man vill vara på Lophtet då det snabbt kan bli uppbokat. Kårens förslag är 1-2 februari. Det är fördelaktigt att lägga Skiphtet så tidigt som möjligt på året. Upplägget är liksom för KPL helt upp till er men tänk på att meddela alla nya och gamla funktionärer tidigt så att de kan boka upp datumet. Som du känner till kan evenemanget inte arrangeras på samma sätt som tidigare år på grund av tillståndsmyndighetens regler. Jag har även fått höra att förr i tiden brukade Oddput vara med på Skiphtet.

    \item Det skall tas ett (juligt) julkort som ska skickas till en massa människor. Så boka en tid när alla kan, kom fram till en rolig idé, och genomför den. Julkortet ska skickas ut så mottagarna får det innan julledigheten börjar. Listan med alla som fick förra året finns i styrelsedriven under “Övrigt”. Kolla också gärna med Jakob ifall  Sektionen har haft kontakt med några nya företag under året.   
    
    \item Som bekant klär sig styrelsen bäst i frack och som du kanske till och med känner till ännu bättre kan dessa vara svåra att få tag i. Bäst vore att vara ute i god tid, redan nu innan jul så att de kan skickas på brodyr innan ledigheten och vara tillbaka till skiphtena. Kostanden för frackar har ökat de senaste åren och en idé kan vara att använda lite av styrelsens klädbudget för att subventionera lite av kostanden. Samtidigt kan denna också användas för att köpa in representativa profilkläder, ni får helt enkelt göra en avvägning. Tidigare år har vi använt oss av Ateljé Heléne i Rydebäck för inköp av frackar. Hennes lager blir dock mindre och mindre för varje år så att börja kolla på alternativ kan vara en god idé. Om ni väljer att gå till Ateljé Heléne, se till att ringa några dagar i förväg så hon får tid att hämta allt till butiken. För brodyren har vi de senaste åren använt Brodyrteam (\texttt{brodyrteam@swipnet.se}), det har fungerat bra. De har mallar för hur vi brukar brodera dem och frackarna kan eneklt postas till dem. Brodyren belastar också klädbudgetnen.

    \item Glöm inte bort att alla som definieras som Vice har rätt till Vice-kavaj. Enklast och billigast är nog att de får köpa dessa på HM. Ni kan sedan skicka iväg allt samtidigt till brodyr. Lite märkligt men även kostnaden för denna brodyr har lagts på styrelsens klädbudget. 
    
    \item Bestäm vad ni vill ha för profilkläder och beställ dem så snart som möjligt så att de kan nyttjas maximalt. 2019 använde vi oss av IGO-profil vilket jag inte rekommenderar till framtiden, tog otroligt lång tid att få dem och blev fel flera gånger.
    
    \item Om du letar efter julläsning så rekommenderar jag stadgan och reglementet. Inte jättekul men innehåller mycket bra saker och att ha ett hum och vad som finns var kommer vara givande. Givetvis rekommenderar jag också och det riktiga ordförandetestamentet som är mer likt ett uppslagsverk. Jag skickar det till dig om någon vecka när det är färdigställt.
    
    \item Jag vill uppmuntra din styrelse till att vara med på våra resterande styrelsemöten i år. Kan vara bra för dem att se hur det sett ut när vi haft möten. Särskilt ska ni medverka på terminens sista styrelsemöte där massa spännande händer! Datumet lär vara måndagen den 16:e december 12.10 i E:1123.
    
    \item Du kan redan nu få tillgång till den nya Ordförande-mailen som Mattias gjorde (ordforande@esek.se, pw: cjasplund). Jag har lagt till den i alla drives och gjort den till admin för G-suiten. Sätt även upp slackkanaler och diskutera tidigt med styrelsen vilka förhållningsätt ni vill ha till era olika kommunikationskanaler.
    
    \item Prata med nya styrelsen om val av e.a.-poster så att det kan ske så snabbt som möjligt!

    \item Vi får boka in en dag där vi går igenom docs-repot. Alltså giten där de flesta av sektionens dokument ligger. Denna är ett otroligt hjälpmedel och egentligen ingeting svårt, bara ovant om man inte arbetat i det förut. 

   
    \item Datum från Kåren
    \begin{dashlist}
        \item C-cert: Vecka 3, 2 hela vardagar. Minst en av (men gärna båda) firmatecknarna behöver C-cert, så se till att planera in det.
        \item A-cert: Vecka 4 eller 5, en vardagskväll. 
        \item B-cert: Vecka 7, en söndag.
        \item Fullmäktigeutbildning: 26 januari.
        \item Kårens förslag på sektionsskiphten: 1-2 februari.
        \item Kollegieskiphten 8-9 februari.
        \item Fördjupad funktionärsutbildning: 16 februari, eftermiddag (heltidare och sektionsordförande).
        \item Styrelseutbildning: 23 februari, eftermiddag.
        \item Utvecklingskväll med kårstyrelsen: 31 mars.
    \end{dashlist}
    
    \item Onsdagen den 4 december och torsdagen den 12 kl 18.00 har vi årets två sista OK-möten. Där får du gärna vara med! Det första kommer vara ett vanligt möte och det andra mer av en överlämningsmiddag.
    
    \item På tal om OK, fyll i dina kontaktuppgifter här: \href{https://docs.google.com/spreadsheets/d/1Alnfo_BpbiZjzFlzi3GcLDZghKOSsQWLfhRr2DIe0F0/edit#gid=0}{\textit{länk}}

    \item Ytterligare på OK, förslaget på terminens rektorslunch är 17 december. Meddela om det fungerar för dig. Det är mest en kul grej och för oss att få ett ansikte på rektor och prorektor.

    \item Prata med din nya styrelse om ni är intresserade av en överlämningsmiddag även med vår styrelse. Om du vill kan vi ordna en middag bara för oss två också eller tillsammans med kontaktorerna.  
    
    \item Det är även din uppgift att införskaffa en julgran till Sektionen innan julgillet. Hur du går tillväga är upp till dig. 
    
    \item Sist men inte minst - lär dig Taggig Blomma till Julgillet och glöm inte din sångbok hemma!
    
    \end{numplist}
    
    Du kommer att få ett rejält testamente snart men det här ska vara det mesta som du behöver tänka på nu innan julledigheten. Jag håller dig naturligtvis uppdaterad när mer saker dyker upp och om du har några frågor finns jag tillgänglig.
    
    Lycka till! 2020 kommer bli ett lysande år!
    
    \emph{Edvard Carlsson \newline Ordförande 2019}
    
    \end{document}
    