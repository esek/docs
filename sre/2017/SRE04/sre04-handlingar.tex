\documentclass[10pt]{article}
\usepackage[utf8]{inputenc}
\usepackage[swedish]{babel}

\def\doctype{Handlingar} %ex. Kallelse, Handlingar, Protkoll
\def\mname{studierådsmöte} %ex. styrelsemöte, Vårterminsmöte
\def\mnum{SRE04/17} %ex S02/16, E1/15, VT/13
\def\date{2017-03-21} %YYYY-MM-DD
\def\docauthor{Pontus Landgren}

\usepackage{../sre-mote}
\usepackage{../../../e-sek}

\begin{document}

\heading{{\doctype} till {\mname} {\mnum}}

\textit{Då mötet ska ta ställning till hur E-sektionen ställer sig i flera större frågor med mycket förberedande handlingar har jag denna gång valt att skicka ut handlingar till fredagens möte i förväg. Detta för att alla ska ges chansen att ha samma bakgrundsfakta innan mötet. Om man sedan väljer att läsa all text är upp till var och en.}

\textit{\textbf{Jag vill också belysa att i princip allt som bifogas är synpunkter och förslag, dvs inga beslut är fattade i frågorna!}}

\section*{Information om punkterna}

\begin{paragrafer}
\p{4}{LTH:s 5-åriga utbildning}
Ledningsgruppen för grundutbildning, LGGU, tillsatte i höstas en arbetsgrupp för att utvärdera huruvida vi bör förändra vägen till examen för programmen vid LTH. I stora drag fanns tre förslag: hård 3+2, mjuk 3+2 eller 5-årig utbildning. En rapport har kommit och den finns att läsa som bilaga.

\p{5}{Matematikinstitutionens rapport}
Resultaten i matematik har varit sjunkande på LTH de senaste åren och institutionen är nu intresserade av att försöka hitta lösningar för att vända trenden. Det mest intressanta förslaget verkar vara att överväga möjligheten att införa två olika kurser i endimensionell analys, ett för de med mycket behov av matte och ett för de med lite mindre. Även här finns rapport i bilaga. 

\end{paragrafer}

\begin{signatures}{1}
\it{I studierådets tjänst}
\signature{\docauthor}{SRE-ordförande}
\end{signatures}

\end{document}
