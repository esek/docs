\documentclass[10pt]{article}
\usepackage[utf8]{inputenc}
\usepackage[swedish]{babel}

\def\mo{Pontus Landgren}
%\def\ms{}
%\def\ji{}
%\def\jii{}

\def\doctype{Minnesanteckningar} %ex. Kallelse, Handlingar, Protkoll
\def\mname{studierådsmöte} %ex. styrelsemöte, Vårterminsmöte
\def\mnum{SRE04/17} %ex S02/16, E1/15, VT/13
\def\date{2017-03-24} %YYYY-MM-DD
\def\docauthor{Pontus Landgren}

\usepackage{../sre-mote}
\usepackage{../../../e-sek}

\begin{document}
%\showsignfoot

\heading{{\doctype} för {\mname} {\mnum}}

%\naun{}{} %närvarane under
%\nati{} %närvarande till och med
%\nafr{} %närvarande från och med
\section*{Närvarande}
\begin{narvarolista}
\nv{Edvard Carlsson}{E16}{}{}
\nv{Ellen Nilsson}{BME15}{}{}
\nv{Fanny Månefjord}{BME16}{}{}
\nv{Fredrik Nilsson}{E16}{}{}
%\nv{Frida Börnfors}{BME14}{}{}
%\nv{Johan Persson}{E13}{}{}
%\nv{Jonatan Kronander}{E15}{}{}
\nv{Lina Samnegård}{BME16}{}{}
\nv{Linnea Wenäll}{BME16}{}{}
\nv{Pontus Landgren}{E14}{Ordförande}{}
%\nv{Sofia Karlén}{BME13}{}{}
%\nv{Viktor Hjelm}{E13}{}{}

\end{narvarolista}


\section*{Minnesanteckningar}
\begin{paragrafer}
\p{1}{Rapporter från utskottet}
%In med problemos här
\textbf{E1:} Lite tyngre läsperiod med flerdim och fysik. Bra närvaro på övningarna i de nya kurserna än så länge. 

\textbf{E2:} Tentorna gick nog blandat för många. Funken givetvis svår och få klarar den nog. Datortekniken kommer förmodligen få lite bättre resultat. Många tycker tyvärr att datortekniken inte har varit så bra som de hoppats på med lite problem med labbar och föreläsningar. Kommer nog få in ett gäng missnöjda CEQ:er. Kurserna som startat nu har det inte varit något att rapportera om precis. Analogen (ESSF01) forsätter med projekt nu och Komponentfysik (ESS030) och SoT (FMAF05) har rullat igång.

\textbf{E3:} Flyter på bra med de nya kurserna.  

\textbf{BME1:} Tycker det fungerar bra, lite lugnare period nu jämfört med förra. Hade en längre tentaperiod med flera tentor för första gången. 

\textbf{BME2:} Laborationsrapporter tar lång tid att få tillbaka och returer på rapporter lika så i kursen Medicinsk fysik. 
 
\textbf{BME3:} Klinisk kemisk diagnostik fick mycket kritik på CEQ-mötet men kursansvarig var inte intresserad av att försöka förändra kursen. Nya kursen verkar fungera bra och kandidatarbetet rullar på.
 
\textbf{Specialisering:} \textit{Inget att rapportera.}

\textbf{LiBe:} %\textit{Inget att rapportera.}
Har haft event på internationella kvinnodagen tillsammans med D-sek. Ska hålla i något case under phadderutbildningen. 

\textbf{VMX:} %\textit{Inget att rapportera.}
Arrangerar babble bistro på kåren i veckan som kommer. 

\textbf{SkyX:} %\textit{Inget att rapportera.}
Har varit på utbildning med LU:s huvudskyddsombud.

\p{2}{CEQ}
Det blir ingen CEQ-tävling denna läsperioden då SRE anser att den förra inte gav önskat resultat. 
\newpage
\p{3}{Meddelanden}
\item SRE:s tröjor har kommit. 
\item Nominering till TLTH:s pedagogiska pris är öppen. Det är öppet för vem som helst att nominera, \textit{\href{https://goo.gl/forms/uNWT6lx2b8KOkuay1}{klicka här för länk.}}

\p{4}{LTH:s 5-åriga utbildning}
Den generella uppfattningen på BME är nöjda med det. E-programmet ser problem med att lyfta ut 15 HP. 

\p{5}{Matematikinstitutionens rapport}
SRE har uppfattningen att vi skulle gynnas av att läsa kursen över tre läsperioder istället för två som E-programmet gör just nu.

\p{6}{Övrigt}
%Övriga stuff
%\textit{Inget övrigt.}
Ordförande Erik har tillsammans med Pontus och studievägledare Ingrid avbetat på ett förslag angående att införa ett mentorsprogram (likt intresseföreningen Elektras) för nyantagna på E- och BME-programmet. Det finns ett utkast att läsa för den som är intresserad. Värt att tillägga är att inget är beslutat än och ni får gärna komma med feedback och kommentarer till Pontus eller Erik. 
\textit{\href{https://drive.google.com/file/d/0B7WHZ6wh42IALW9UYjl6eDBuelk/view?usp=sharing}{Länk till utkast för mentorsprogram E-sektionen.}} 

\end{paragrafer}

%\newpage
\hidesignfoot
\begin{signatures}{1}
\signature{\mo}{Mötesordförande}
%\signature{\ms}{Mötessekreterare}
%\signature{\ji}{Justerare}
\end{signatures}
\end{document}
