\documentclass[10pt]{article}
\usepackage[utf8]{inputenc}
\usepackage[swedish]{babel}

\def\mo{Pontus Landgren}
%\def\ms{}
%\def\ji{}
%\def\jii{}

\def\doctype{Minnesanteckningar} %ex. Kallelse, Handlingar, Protkoll
\def\mname{studierådsmöte} %ex. styrelsemöte, Vårterminsmöte
\def\mnum{SRE02/17} %ex S02/16, E1/15, VT/13
\def\date{2017-02-10} %YYYY-MM-DD
\def\docauthor{Pontus Landgren}

\usepackage{../sre-mote}
\usepackage{../../../e-sek}

\begin{document}
%\showsignfoot

\heading{{\doctype} för {\mname} {\mnum}}

%\naun{}{} %närvarane under
%\nati{} %närvarande till och med
%\nafr{} %närvarande från och med
\section*{Närvarande}
\begin{narvarolista}
\nv{Edvard Carlsson}{E16}{}{}
\nv{Ellen Nilsson}{BME15}{}{}
\nv{Fanny Månefjord}{BME16}{}{}
\nv{Fredrik Nilsson}{E16}{}{}
\nv{Frida Börnfors}{BME14}{}{}
%\nv{Johan Persson}{E13}{}{}
\nv{Jonatan Kronander}{E15}{}{}
\nv{Lina Samnegård}{BME16}{}{}
\nv{Linnea Wenäll}{BME16}{}{}
\nv{Pontus Landgren}{E14}{Ordförande}{}
\nv{Sofia Karlén}{BME13}{}{}
\nv{Viktor Hjelm}{E13}{}{}

\end{narvarolista}


\section*{Minnesanteckningar}
\begin{paragrafer}
\p{1}{Rapporter från utskottet}
%In med problemos här
\textbf{E1:} Endim gick under förväntan men elektroniken gick bra. Har börjat med projekt och det upplevs som bra.

\textbf{E2:} Analogen är lite förvirrande och föreläsningarna tar inte riktigt upp rätt saker enligt studenterna.

\textbf{E3:} Går bra för E3, resultatet för ESS050 har kommit och var bra.

\textbf{BME1:} Kemin är lite rörig, byter föreläsare om vart annat. I övrigt går det bra.

\textbf{BME2:} Ny kursansvarig i fysiken, upplevs lite rörigt. Annars flyter det på bra. 

\textbf{BME3:} Gör mycket inlämningar och rapporter för lite högskolepoäng. Kandidatarbete verkar fungera bra, svårt att få någon uppfattning på individnivå.

\textbf{Specialisering:} Går bra enligt rapporterna.

\textbf{LiBe:} \textit{Inget att rapportera.}

\textbf{VMX:} \textit{Inget att rapportera.}

\textbf{SkyX:} \textit{Inget att rapportera.}

\p{2}{Betygsfördelning}
%In med betygsfördelning här
\href{https://docs.google.com/spreadsheets/d/1Lir7Day7pFARq_1cz4DDdfcUisUunIAgXFzi4WWim2Q/edit?usp=sharing}{Länk: Betygsfördelning LP2}

\p{3}{Pluggkväll}
%Folk till vår pluggkväll, kårens pluggkväll lv8 
Vi kör SRE:s pluggkväll onsdagen den 22/2. De som fixar den är: Pontus, Jonatan, Fanny, Lina och Linnea. \\
Kåren med hjälp av E, K och I arrangerar pluggkväll under inläsningsveckan. Mer info kommer på denna punkten.
\\
\p{4}{CEQ}
%Statistiken till den här
\href{https://docs.google.com/spreadsheets/d/1Dgv3cSbv8l1XRmSr6SsRmVqKaI-y6OlmmdPo7TC9YME/edit?usp=sharing}{Länk: CEQ-statistik LP2} \\ 
\textbf{Skriv kommentarer efter CEQ-möten. Denna punkt är viktig, vi är skyldiga att skriva kommentarer efter de obligatoriska kurserna (årskurs 1-3) och har fått påbackning från Kåren och LTH kansli.}

\p{5}{SRE-ordförande informerar}
\item Skiphte med funktionärsutbildning i E:B 17:15, 2017-02-17, anmälan krävs (inte sexet). \href{https://goo.gl/SzgpMm}{Länk till anmälan.}

\item E-sektionen har en gemensam kalender länk: \url{https://goo.gl/WVGQhM}

\item Kick-off på puben puben? 
%15/2 (bowling) eller 1/3
1/3 kl 17:15, i Kårhuset är planen.

\item SRE-tröjor beställs via utskottet och kommer vara tillgängliga för funktionärer under pluggkvällar och andra SRE-relaterade evenemang.

\item Kårens klagaformulär, \href{https://tlth.se/klaga}{länk} \\ - \textit{Anmäl gärna för att bidra till kårens statistik, men anmäler ni som studieråd informera SRE-ordförande.} 

\item Speak-up days, kåren anordnar ett evenemang i syfte att informera om studiebevakning, vill man vara en del av det anmäl intresse till undertecknad.

\item UUU (utbildningsutskottets utbildning för utbildningsbevakare) - 28/2 kl. 17:15, middag efter, anmälan krävs och sker via Pontus. \\- \textit{Har man inte gått den innan är det rekommenderat att gå den.} 

\item NollU - SRE-workshopen kommer vara kvar. Pluggphaddrar och pluggkvällar lika så, mer planering på den fronten krävs. Stadskringvandringen är planerad att kombineras med "utskottssafari".

\item Exjobb mellan E och BME (ev. C \& D). Industrin/institutionerna efterfrågar samarbete med anledning av bra komplettering programvaru/medicinsk kunskap. Diskussion huruvida vi vill försöka sprida info om det här, SVL/PL är villiga att jobba på det här.\\ - \textit{Kommer förmodligen att ske gemensam information till E och BME under hösten när årskurs 5 får information om examensarbete.}

\item Mingel om utbytesstudier - vill vi lägga det i april som tidigare, eller testa hösten i samband med LTH:s information.\\ - \textit{Beroende på syfte med att träffa gamla studenter tyckte mötet att det vore bättre att träffa enskilda studenter för specifika frågor på hösten och hellre ha någon som "föreläser" på våren för inspiration.}

\item Alumnikväll på E, BME har haft en med bra uppslutning, borde vi försöka göra något liknande med E-programmet?\\ - \textit{Intresset på E-programmet spås vara lägre än på BME som nyligen fick sina första utexaminerade.}

\p{6}{Övrigt}
%Övriga stuff
\textit{Inget övrigt.}
\end{paragrafer}

%\newpage
\hidesignfoot
\begin{signatures}{1}
\signature{\mo}{Mötesordförande}
%\signature{\ms}{Mötessekreterare}
%\signature{\ji}{Justerare}
\end{signatures}
\end{document}
