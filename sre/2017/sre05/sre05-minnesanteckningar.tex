\documentclass[10pt]{article}
\usepackage[utf8]{inputenc}
\usepackage[swedish]{babel}

\def\mo{Pontus Landgren}
%\def\ms{}
%\def\ji{}
%\def\jii{}

\def\doctype{Minnesanteckningar} %ex. Kallelse, Handlingar, Protkoll
\def\mname{studierådsmöte} %ex. styrelsemöte, Vårterminsmöte
\def\mnum{SRE05/17} %ex S02/16, E1/15, VT/13
\def\date{2017-04-07} %YYYY-MM-DD
\def\docauthor{Pontus Landgren}

\usepackage{../sre-mote}
\usepackage{../../../e-sek}

\begin{document}
%\showsignfoot

\heading{{\doctype} för {\mname} {\mnum}}

%\naun{}{} %närvarane under
%\nati{} %närvarande till och med
%\nafr{} %närvarande från och med
\section*{Närvarande}
\begin{narvarolista}
\nv{Edvard Carlsson}{E16}{}{}
\nv{Ellen Nilsson}{BME15}{}{}
\nv{Fanny Månefjord}{BME16}{}{}
%\nv{Fredrik Nilsson}{E16}{}{}
\nv{Frida Börnfors}{BME14}{}{}
%\nv{Johan Persson}{E13}{}{}
%\nv{Jonatan Kronander}{E15}{}{}
\nv{Lina Samnegård}{BME16}{}{}
\nv{Linnea Wenäll}{BME16}{}{}
\nv{Pontus Landgren}{E14}{Ordförande}{}
\nv{Sofia Karlén}{BME13}{}{}
\nv{Viktor Hjelm}{E13}{}{}

\end{narvarolista}


\section*{Minnesanteckningar}
\begin{paragrafer}
\p{1}{Rapporter från utskottet}
%In med problemos här
\textbf{E1:} Linalg har kommit tillbaka. Programmeringstentan är långsam. 

\textbf{E2:} Tentan i funktionsteori upplevdes som svår och gick inte så bra. 

\textbf{E3:} Läsperiodens kurser fungerar bra. Komsysen (ETSF15) hade en genomströmning på 30\%.

\textbf{BME1:} Endim A3 70\% genomströmning, omkring 80\% på kemin. Periodens kurser flyter på.

\textbf{BME2:} Hemtentan i Medicinteknisk design har inte kommit tillbaka. Upplever att det är för få övningsledare på signalbehandlingen . 
 
\textbf{BME3:} E-hälsan tar tid att få tillbaka.
 
\textbf{Specialisering:} Det fungerar bra. 

\textbf{LiBe:} Har haft kollegiemöte och diskuterat jodel-frågan. Göra en enkät med frågan hur E-sektionen mår? Maktsalongen ska komma och föreläsa på kåren, ev för styrelsen. %\textit{Inget att rapportera.}

\textbf{VMX:} Gruppen har kommit igång och det finns utbytesstudenter som vill engagera sig. %\textit{Inget att rapportera.}

\textbf{SkyX:} \textit{Inget att rapportera.}

\p{2}{CEQ}
Det är dags att granska CEQ:er.

\p{3}{Information}
Speak-up-days - kårens utbildningsutskott arrangerar ett evenemang för att uppmärksamma studiebevakningen på LTH. Kommer stå i E-huset en av dagarna i läsvecka 4. SRE och SRD kommer du att vara med och hålla i det några timmar den dagen. 

\p{4}{Övrigt}
%Övriga stuff
%\textit{Inget övrigt.}
Läsårsindelningen inför nästa läsår är förändrad.

\end{paragrafer}

%\newpage
\hidesignfoot
\begin{signatures}{1}
\signature{\mo}{Mötesordförande}
%\signature{\ms}{Mötessekreterare}
%\signature{\ji}{Justerare}
\end{signatures}
\end{document}
