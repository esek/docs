\documentclass[10pt]{article}
\usepackage[utf8]{inputenc}
\usepackage[swedish]{babel}

\def\mo{Pontus Landgren}
%\def\ms{}
%\def\ji{}
%\def\jii{}

\def\doctype{Minnesanteckningar} %ex. Kallelse, Handlingar, Protkoll
\def\mname{studierådsmöte} %ex. styrelsemöte, Vårterminsmöte
\def\mnum{SRE03/17} %ex S02/16, E1/15, VT/13
\def\date{2017-03-03} %YYYY-MM-DD
\def\docauthor{Pontus Landgren}

\usepackage{../sre-mote}
\usepackage{../../../e-sek}

\begin{document}
%\showsignfoot

\heading{{\doctype} för {\mname} {\mnum}}

%\naun{}{} %närvarane under
%\nati{} %närvarande till och med
%\nafr{} %närvarande från och med
\section*{Närvarande}
\begin{narvarolista}
\nv{Edvard Carlsson}{E16}{}{}
%\nv{Ellen Nilsson}{BME15}{}{}
\nv{Fanny Månefjord}{BME16}{}{}
\nv{Fredrik Nilsson}{E16}{}{}
\nv{Frida Börnfors}{BME14}{}{}
\nv{Johan Persson}{E13}{}{}
%\nv{Jonatan Kronander}{E15}{}{}
\nv{Lina Samnegård}{BME16}{}{}
\nv{Linnea Wenäll}{BME16}{}{}
\nv{Pontus Landgren}{E14}{Ordförande}{}
\nv{Sofia Karlén}{BME13}{}{}
%\nv{Viktor Hjelm}{E13}{}{}

\end{narvarolista}


\section*{Minnesanteckningar}
\begin{paragrafer}
\p{1}{Rapporter från utskottet}
%In med problemos här
\textbf{E1:} Tycker det flyter på bra har inga anmärkningar. Matlab-laborationerna innehåller mycket uppgifter och det är svårt att hinna med tillfällena.  

\textbf{E2:} Högt tempo så många lägger mycket tid i skolan. Datortekniken har väl gått sådär, missnöje med labbar. Lite förbättring efter att vi skrivit till honom men fortfarande folk som fått vänta timmar på att få redovisa. För två veckor sedan krockade lite inlämningar och så men löste sig.

\textbf{E3:} Kurserna fungerar bra, har varit mycket laborationer denna läsperioden vilket gjort det jobbigt att hinna göra övningsuppgifter. 

\textbf{BME1:} Funkar bra, inget övrigt att rapportera. 

\textbf{BME2:} Fungerar bra, har inget övrigt att rapportera. 

\textbf{BME3:} Biomaterial i LP2 har inte kommit in i Ladok men resultatet ligger som U/G istället för 3-5-skala.

\textbf{Specialisering:} FMAN30 har kommit in i Ladok nu, skulle avslutats i december. 

\textbf{LiBe:} %\textit{Inget att rapportera.}
Har evenemang angående internationella kvinnodagen.

\textbf{VMX:} %\textit{Inget att rapportera.}
SRE jobbar på att få fram kontaktuppgifter till internationella studenter på sektionen.

\textbf{SkyX:} \textit{Inget att rapportera.}


\p{2}{CEQ}
%Statistiken till den här
\textbf{Tisdag 7/3 i E:1517a:}
\item 11:00-11:45 FAFA65: Termodynamik, våglära och optik (BME1)
\item 12:15-13:00 FHL055: Teknisk mekanik (BME2)
\item 13.00-13:45 EEMF10: Klinisk kemisk diagnostik (BME3)
\item 13:45-14:30 FRTF01:  Fysiologiska modeller och beräkningar (BME3)
\item 14.30-15.15 EXTG05: Biomaterial - Interaktion mellan levande vävnad och syntetiska material  (BME3)

EITA01: Introduktion till medicin och teknik den 8/3 kl.15:00 i E:1517a (BME1)

EDAA01: Programmeringsteknik - fördjupningskurs
den 23/3 kl.12:20-12:50 (E2)

\p{3}{SRE-ordförande informerar}
\item Läsårsindelningen - Omtentaperioder för nästa läsår är förändrade.

\item Kåren har pluggkväll på tisdag kl 17:15 i Cornelis, anmälan sker via länk från TLTH:s facebookevenemang.

\item Rektorsbeslut - ansvarig för psykisk hälsa, områdesansvarig \textit{(Martin Höst)}

\item Rektorsnominering - Rektor och prorektor ska väljas och en valberedningsprocess har inletts.


\p{4}{Övrigt}
%Övriga stuff
%\textit{Inget övrigt.}
Posters är gjorda, ska tryckas.
\end{paragrafer}

%\newpage
\hidesignfoot
\begin{signatures}{1}
\signature{\mo}{Mötesordförande}
%\signature{\ms}{Mötessekreterare}
%\signature{\ji}{Justerare}
\end{signatures}
\end{document}
