\documentclass[10pt]{article}
\usepackage[utf8]{inputenc}
\usepackage[swedish]{babel}

\def\mo{Pontus Landgren}
%\def\ms{}
%\def\ji{}
%\def\jii{}

\def\doctype{Minnesanteckningar} %ex. Kallelse, Handlingar, Protkoll
\def\mname{studierådsmöte} %ex. styrelsemöte, Vårterminsmöte
\def\mnum{SRE06/17} %ex S02/16, E1/15, VT/13
\def\date{2017-04-28} %YYYY-MM-DD
\def\docauthor{Pontus Landgren}

\usepackage{../sre-mote}
\usepackage{../../../e-sek}

\begin{document}
%\showsignfoot

\heading{{\doctype} för {\mname} {\mnum}}

%\naun{}{} %närvarane under
%\nati{} %närvarande till och med
%\nafr{} %närvarande från och med
\section*{Närvarande}
\begin{narvarolista}
\nv{Edvard Carlsson}{E16}{}{}
\nv{Ellen Nilsson}{BME15}{}{}
\nv{Fanny Månefjord}{BME16}{}{}
\nv{Fredrik Nilsson}{E16}{}{}
%\nv{Frida Börnfors}{BME14}{}{}
\nv{Johan Persson}{E13}{}{}
\nv{Jonatan Kronander}{E15}{}{}
\nv{Lina Samnegård}{BME16}{}{}
\nv{Linnea Wenäll}{BME16}{}{}
\nv{Pontus Landgren}{E14}{Ordförande}{}
%\nv{Sofia Karlén}{BME13}{}{}
%\nv{Viktor Hjelm}{E13}{}{}

\end{narvarolista}


\section*{Minnesanteckningar}
\begin{paragrafer}
\p{1}{Rapporter från utskottet}
%In med problemos här
\textbf{E1:} I fysiken genomförs labbar innan föreläsningar.

\textbf{E2:} Tycker att det fungerar bra. Datumet för inlämningen i FMAF05 (system och transformer) var samma som förra året, tappade en vecka pga omtentaveckan.

\textbf{E3:} Flyter på bra med nuvarande kurser. Mycket rapporter och redovisningar i slutet av läsperioden.

\textbf{BME1:} Går bra, inga problem att rapportera. 

\textbf{BME2:} Har haft dåligt med övningsledare, men har blivit lovade att det ska bli bättre. 
 
\textbf{BME3:} Fungerar bra, inga problem att rapportera. 
 
\textbf{Specialisering:} Inga problem att rapportera.

\textbf{LiBe:} \textit{Inget att rapportera.}

\textbf{VMX:} \textit{Inget att rapportera.}

\textbf{SkyX:} \textit{Inget att rapportera.}

\p{2}{CEQ} E-programmet ligger efter och det är påtalat i programledningen. Kurser från läsperiod 3 är startgropen och möten är att vänta inom flera kurser.

\p{3}{Information}
\item{Läsårsindelningen, mycket diskussioner runt om på LTH, något som är oklart?}

\item{Speak up days arrangerades i onsdags i E-huset. Enkäten är fortfarande öppen för den som missat att fylla i den. \textit{\href{https://www.tinyurl.com/sud17}{(länk till enkäten)}}}

\item{JJF, jättefunktionärsfesten, arrangeras den 20:e maj. Alla funktionärer inom sektionen såväl som andra sektioner och TLTH är välkomna. Det kommer att ske i form av en volleybollturnering på dagen och en sittning på kvällen (allt är gratis). \textit{\href{https://www.facebook.com/events/1325332067532352/}{(länk till facebookevenemanget)}}}

\item{E-sektionens mentorsprogram, fler sökande mentorer behövs. Det kommer inte ta lång tid och det spelar ingen roll om man har andra åligganden under hösten så som funktionärsposter eller phadder. Man bör inte heller lägga någon större fokus på kraven av avklarade studier samt hur länge man har studerat här. Tagga E:are och BME:are att söka! \textit{\href{https://drive.google.com/file/d/0B7WHZ6wh42IALXdoTlZidk1yNEk/view?usp=sharing}{(länk till info angående mentorsprogrammet)}}}

\item{Specialiseringsmingel till hösten i samband med LTH information?}

\p{4}{Övrigt}
Undertecknad har kandiderat till utbildningsheltidare på kåren. Om vald kommer jag att entlediga mig från uppdraget som SRE-ordförande den 30/6.
%Övriga stuff
%\textit{Inget övrigt.}

\end{paragrafer}

%\newpage
\hidesignfoot
\begin{signatures}{1}
\signature{\mo}{Mötesordförande}
%\signature{\ms}{Mötessekreterare}
%\signature{\ji}{Justerare}
\end{signatures}
\end{document}
