\documentclass[10pt]{article}
\usepackage[utf8]{inputenc}
\usepackage[swedish]{babel}



\def\mo{Pontus Landgren}
%\def\ms{}
%\def\ji{}
%\def\jii{}

\def\doctype{Minnesanteckningar} %ex. Kallelse, Handlingar, Protkoll
\def\mname{studierådsmöte} %ex. styrelsemöte, Vårterminsmöte
\def\mnum{SRE01/17} %ex S02/16, E1/15, VT/13
\def\date{2017-01-17} %YYYY-MM-DD
\def\docauthor{Pontus Landgren}

\usepackage{../sre-mote}
\usepackage{../../../e-sek}

\begin{document}

\heading{{\doctype} för {\mname} {\mnum}}

%\naun{}{} %närvarane under
%\nati{} %närvarande till och med
%\nafr{} %närvarande från och med
\section*{Närvarande}
\begin{narvarolista}
%\nv{Edvard Carlsson}{E16}{}{}
%\nv{Ellen Nilsson}{BME15}{}{}
\nv{Fanny Månefjord}{BME16}{}{}
%\nv{Fredrik Nilsson}{E16}{}{}
\nv{Frida Börnfors}{BME14}{}{}
\nv{Johan Persson}{E13}{}{}
\nv{Jonatan Kronander}{E15}{}{}
\nv{Lina Samnegård}{BME16}{}{}
\nv{Linnea Wenäll}{BME16}{}{}
\nv{Pontus Landgren}{E14}{Ordförande}{}
%\nv{Sofia Karlén}{BME13}{}{}
\nv{Viktor Hjelm}{E13}{}{}

\end{narvarolista}


\section*{Minnesanteckningar}
\begin{paragrafer}
\p{1}{Presentation}
%Presentation av medlemmar i utskottet funktionärsförmåner (access, skiphte).
Alla närvarande medlemmar presenterade sig och undertecknad informerade om funtionärsskiphtet som äger rum den 17/2 samt att alla funktionärer ska ha fått sin dörraccess. 

\p{2}{Årets upplägg}
%Diskussion kring årets upplägg, mötesfrekvens, kontaktkanal (facebook, mail).
Mötet kom preliminärt fram till att försöka hålla tre lunchmöten per läsperiod med det primära målet att diskutera studiesituationen inom de olika klasserna samt utbyta information mellan studierådets medlemmar.

Facebook-grupp och chatt anses fungera bra, så även systemet med evenemangspåminnelse.

\p{3}{Hur mår alla?}
\textbf{BME1}:: Matten (FMAA01) upplevs som strukturerad och fungerar väl. Fysiken  (FAFA65) är nu avslutad. Kursen har enligt klassrepresentanter blivit bättre, efter kontakt med kursansvarig och studierektor. I slutet av kursen upplevde studenterna att det fungerade märkbart bättre. 

\textbf{E1}: Upplever att både elektroniken (ESS010) och matematiken (FMAA05) fungerar väl.

\textbf{BME2}: Nya kurserna verkar än så länge bra. CEQ-mötet i cellens biologi (TEK295) gick bra.
 
\textbf{E2}: MIO012, industriell ekonomi är faktatung. Digitaltekniktentan (EIT020) verkade svår och genomströmningen spås bli lägre än tidigare.

\textbf{BME3}: Läsperiod 2 är en tung period där man anser att man gör väldigt mycket arbete i form av inlämningar och rapporter, som i sin tur inte ger mycket högskolepoäng.

\textbf{E3}: Har avslutet elektromagnetisk fältteori (ESS050) med en mycket uppskattad föreläsare. I matematisk statistik (FMSF20) anses föreläsaren brista och använda mycket slides och lite tavelräkning.    

\textbf{Specialisering}: Svårt att matcha poäng och därmed kunna läsa 100 \% en läsperiod. 
Bedömningen i kurserna FMAN30 (Medicinsk bildanalys) och FMAN20 (Bildanalys) upplevs som orättvis. En grupp av studenter gör uppgifter tillsammans (samarbete är tillåtet i kursen) och anser att de gjort i princip samma sak och får sedan olika bedömningar på det. 

Undertecknad ska kontrollera med studierådsordförande på andra sektioner som har studenter som läser samma kurser och se ifall liknande problem upplevs där.

\p{4}{CEQ-tävling}
Mötet beslutade att köra en repris av CEQ-tävlingen som hölls efter läsperiod 1 i höstas. Vi trycker samma affischer och försöker marknadsföra för att öka svarsfrekvensen.

\p{5}{CEQ-utbildning}
%Granskas efter 31/1, utbildning i CEQ-processen.
%Förslag: ons 1/2, tors 2/2 (Checka BME2, lab till 18)
SRE-ordförande kommer att hålla i en utbildning för utskottets nya medlemmar i syfte att instruera och informera om hur granskningen av CEQ-enkäter och dess efterföljande möte går till.

\p{6}{Pluggkväll}
%Förslag: ons 15/2, tis 21/2, ons 22/2 (mest önskat)
%Datum, tankar idéer
Preliminärt kör vi pluggkväll onsdagen den 22/2, läsvecka 6.

\p{7}{SRE-tröjor}
%Vill vi ha egna tröjor inom SRE för att synas bättre? (SVL rekommenderar) - 350-400 kr/tröja.
Intresset för personliga tröjor är begränsat. Däremot tycker SRE att det hade varit en bra ide att köpa in ett antal tröjor med SRE-tryck som vi kan använda under nollningen men även vid pluggkvällar.  

\p{8}{Kickoff}
%Förslag: (sön 5/2), fre 24/2, lör 25/2
Som kickoff diskuterades att försöka synka med någon pub, exempelvis puben puben eller någon nationspub. En vardag ansågs mer lämplig än en helg.

\p{9}{Övrigt}
%Städvecka
SRE har städvecka denna vecka.

\end{paragrafer}

%\newpage
\hidesignfoot
\begin{signatures}{1}
\signature{\mo}{Mötesordförande}
%\signature{\ms}{Mötessekreterare}
%\signature{\ji}{Justerare}
\end{signatures}
\end{document}
