\documentclass[11pt]{article}
\usepackage[utf8]{inputenc}
\usepackage[swedish]{babel}

\def\date{2016}
\def\doctitle{Inbjudan till Nolleqasquen \date}
\def\docauthor{Erik Månsson}

\def\rec{styrelserna för Konglig Elektrosektionen och Sektionen för Medicinsk Teknik vid Kungliga Tekniska Högskolan}

\usepackage{../../../e-sek}

\usepackage{geometry}
\geometry{
    a4paper,
    top=25mm,
    left=35mm,
    right=35mm,
    bottom=35mm
}

\fancyhead[L]{}
\fancyfoot[R]{}

\begin{document}
    \begin{center}
        \includegraphics[width=7cm]{sigill}
        \par
        \vspace*{7mm}

        \textit{\textbf{\huge \doctitle}}
    \end{center}

    {\it
    E-sektionen vid Lunds Tekniska Högskola har den stora äran att bjuda in er, {\rec}, till årets Nollegasque.

    Den 23:e september klockan 17:00 tar vi emot er för en rundvandring runt LTH. Senare under kvällen anordnas en sittning för er i E-huset. Klädsel är kavaj.

    Den 24:e september klockan 18:00 öppnas dörrarna till Nollegasquen i Gasquesalen. Efter sittningen bjuds det på en fyrverkerishow och sedan ett eftersläpp som varar fram till 03:00. Klädsel är högtidsdräkt med akademiska ordnar, alternativt mörk kostym.

    Anmälan av högst 6 kuvert görs via mail till kontaktor@esek.se senast den 17:e september. Kostnad per kuvert är 350kr, eller 300kr om ingen alkohol önskas. Vid anmälan, glöm inte att nämna om ni har några matpreferenser samt om ni behöver sovplats.

    Vi ser fram emot att höra från er och hoppas att ni kan närvara!
    }

    \begin{signatures}{1}
    \signature{Erik Månsson}{Kontaktor\\E-sektionen inom TLTH}
    \end{signatures}
\end{document}
