\documentclass[11pt]{article}
\usepackage[utf8]{inputenc}
\usepackage[swedish]{babel}

\def\date{2017}
\def\doctitle{Anmodan till NollEgasquen \date}
\def\docauthor{Erik Månsson}

\usepackage{../../../e-sek}

\usepackage{geometry}
\geometry{
    a5paper,
    top=1mm,
    left=25mm,
    right=25mm,
    bottom=0mm
}

\fancyhead[L]{}
\fancyfoot[R]{}

\usepackage{tikz}
\usetikzlibrary{calc}

\begin{document}

    \begin{tikzpicture}[remember picture, overlay]
      \draw[line width = 2.5pt] ($(current page.north west) + (6mm,-6mm)$) rectangle ($(current page.south east) + (-6mm,6mm)$);

      \draw[line width = 1pt] ($(current page.north west) + (7mm,-7mm)$) rectangle ($(current page.south east) + (-7mm,7mm)$);

      \draw[line width = 1pt] ($(current page.north west) + (5mm,-5mm)$) rectangle ($(current page.south east) + (-5mm,5mm)$);
    \end{tikzpicture}

    \begin{center}
        \includegraphics[width=6cm]{sigill}
        \par
        \vspace*{5mm}

        \textit{\textbf{\Large \doctitle}}

        {\it
            E-sektionen vid Lunds Tekniska Högskola\\
            har den stora äran att anmoda dig
            \par \vspace*{8mm} \line(1,0){220} \par \vspace*{-1mm}
            med respektive till årets NollEgasque.
        }
    \end{center}

    \vspace*{\baselineskip}

    {\it
    NollEgasquen äger rum i Kårhusets Gasquesal lördagen den 23:e september och dörrarna öppnas klockan 18:00. Efter sittningen bjuds på ett eftersläpp som varar fram till 03:00. Klädsel är högtidsdräkt med akademiska ordnar, alternativt mörk kostym.

    Anmälan sker senast den 18:e september till en kostnad av 350kr per kuvert, eller 300kr om ingen alkohol önskas.
    }
\end{document}
