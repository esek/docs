\documentclass[10pt]{article}
\usepackage[utf8]{inputenc}
\usepackage[swedish]{babel}

\def\date{2016}
\def\doctitle{Guide till datorstugan \date}
\def\docauthor{Erik Månsson}

\def\headl{\\ \doctitle}

\usepackage{../../../e-sek}

\begin{document}
	\heading{\doctitle}

	\section{Schema}
	Datorstugan är på onsdagen den 24:e augusti 2016, 12:30 - 15:00. Första gruppen kommer direkt från att ha varit på fotografering sedan KFS, förmodligen tidigast 12:50. Tiderna nedan är cirkatider.

	\begin{center}
		\begin{tabular}{crlll}
			\textbf{Tid} & \textbf{Phaddergrupp} & \textbf{Sal} & \textbf{Ansvarig} \\
			\hline
			12:30 & 1 & Saturnus & Axel Voss \\
			12:40 & 2 & Uranus & Oscar Uggla \\
			12:50 & 3 & Neptunus & Daniel Johansson \\
			13:00 & 4 & Pluto & Andreas Sirenius \\
			13:10 & 5 & Ravel & Erik Månsson \\
			\hline
			13:20 & 6 & Saturnus & Axel Voss \\
			13:30 & 7 & Uranus & Oscar Uggla \\
			13:40 & 8 & Neptunus & Daniel Johansson \\
			13:50 & 9 & Pluto & Andreas Sirenius \\
			14:00 & 10 & Ravel & Erik Månsson \\
		\end{tabular}
	\end{center}

	\section{Punkter att ta upp}
	\begin{dashlist}
		\item Den viktigaste informationen från datorstugan finns på \url{https://eee.esek.se/wiki/Datorinfo}

		\item Eduroam - Universitetets WiFi
		\begin{dashlist}
			\item Eduroam finns på de flesta andra universitet i världen, även flygplatser m.m.
			\item Logga in med Stil-ID
			\item För att logga in på andra universitet, lägg till \texttt{@lu.se} efter ditt Stil-ID
			\item LU:s guide för att ansluta till Eduroam finns här:\\
			\url{http://www.lu.se/studera/livet-som-student/it-tjanster-support-och-driftinfo/uppkoppling/tradlost-nat-pa-universitetet}
		\end{dashlist}

		\item Kolla så alla fått och kommer åt sill StiL-konto
		\begin{dashlist}
			\item E-mail om hur kontot aktiveras ska ha skickats till den e-mailadress som angetts på \texttt{antagning.se}
			\item Om någon saknar kontouppgifter bör de kontakta StiL-supporten eller gå till infodisken som finns på bottenvåningen i Studiecentrum
		\end{dashlist}

		\newpage

		\item Studentportalen - \texttt{student.lu.se}. Be alla logga in. Visa dem vad man kan göra i portalen:
		\begin{dashlist}
			\item \textbf{Kurs- och programregistrering} (superviktigt!)
			\item \textbf{Kursanmälan} (superviktigt!)
			\item \textbf{Tentamensanmälan} (superviktigt!)
			\item Resultat från Ladok
			\item Verifierbara intyg
			\item Information om LU-kort
			\item Länk till utskriftskonto-guiden (detta får de fixa själva)

			\item Gratis programvara för studenter, bl.a.
			\begin{dashlist}
				\item \textsc{Matlab}
				\item Maple
				\item Windows och Office
			\end{dashlist}
		\end{dashlist}

		\item Studentmailen
		\begin{dashlist}
			\item All information från Universitetet kommer till studentmailen, så det är viktigt att ha koll på den. Kan t.ex. vara info från kursansvariga och påminnelser om registrernig.
			\item Går att logga in direkt på \texttt{gmail.com} eller genom studentportalen
			\item \texttt{stil-id@student.lu.se} och \texttt{förnamn.efternamn.xxx@student.lu.se}
			\item Visa hur man vidarebefodrar inkommande mail\\
			Guide: \url{https://support.google.com/mail/answer/10957?hl=sv}
			\item Tipsa om att man kan logga in direkt i telefonen på iPhone/Android
		\end{dashlist}

		\item Resurser på LTH:s hemsida (gå in på alla länkar och visa runt lite)
		\begin{dashlist}
			\item \texttt{kurser.lth.se/lot} - Läro- och timplan\\
				Tryck vidare till ``Program: Läro- och timplaner per läsår''.
			\item \texttt{schema.lth.se} - Aktuell länk till TimeEdit\\
				Finns många funktioner, t.ex. olika vyxer och filter och att synka kalendern med iCal eller Google Calendar.
			\item \texttt{forum.maths.lth.se} - Frågelåda för Matematik vid LTH och LU\\
				Bra ställe att ställa frågor på i tentaplugget. Många föreläsare är med och ger bra svar. Finns redan svar på hundratals frågor.
		\end{dashlist}

		\newpage

		\item E-sektionens hemsida - \texttt{esek.se}
		\begin{dashlist}
			\item Skapa ett konto på \texttt{esek.se} (plustecknet i högra hörnet. Uppmana om att använda mail för att kunna få mail från Sektionen även i framtiden. Be dem logga in och byta lösenord direkt.
			\item Visa \textbf{\_inte\_} upp wikin innan nollegasquen
			\item Visa upp vår hemsida, t.ex. att vi har fotoakriv, möteshandlingar, val, info om utskotten, m.m.
		\end{dashlist}

		\item E-sektionens Android-app (behöver inte visas upp), har bl.a.
		\begin{dashlist}
			\item Nyheter
			\item Sångbok
			\item Kartor
		\end{dashlist}

		\item E-huset har datorsalar med både Windows och Linux. De flesta ligger i källaren men de vanligaste labsalarna, $\alpha$, $\beta$ och $\gamma$, finns en trappa upp vid foajén.

		\item Många företag har perks för studenter, t.ex.
		\begin{dashlist}
			\item Spotify Premium till halva priset
			\item GitHub har obegränsade privata repos
			\item Dropbox har kampanjer lite då och då
		\end{dashlist}

		\item Glöm inte att fråga om nollorna ha några frågor!

		\item Uppmana nollorna att fråga sina Phaddrar om något är oklart eller om tekniken strular. För frågor angående mail och \url{esek.se} kan de alltid maila Kontaktorn, \url{kontaktor@esek.se}.
	\end{dashlist}

	\section{När gruppen är klar}

	\textbf{Dela ut papperna från SVL!}

	Efter datorstugan har nollorna ``Information från studie- och karriärvägledaren och SI-ledarna'' 15:15 i \texttt{E:A} för E och \texttt{E:1406} för BME.

	Tack för hjälpen \texttt{:)}
\end{document}
