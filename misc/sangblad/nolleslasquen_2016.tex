\documentclass[10pt]{article}
\usepackage[utf8]{inputenc}
\usepackage[swedish]{babel}

\def\date{2016-09-24}
\def\doctitle{Sångblad till nolleslasquen 2016}
\def\docauthor{Erik Månsson}

\usepackage{gensymb}
\usepackage{../../e-sek}

\usepackage{multicol}
\raggedcolumns

\geometry{
	a4paper,
	includeheadfoot,
	top=25mm,
	left=25mm,
	right=25mm,
	bottom=20mm,
	headheight=35pt
}

\newenvironment{song}[2]{
	\textbf{#1}\\
	\emph{#2}\par
	\vspace{-1mm}
}{
	\vspace{2mm}
}

\begin{document}
\begin{multicols*}{2}

\begin{song}{Måltidssången}{Sida 49}
	Så lunka vi så småningom\\
	från Bacchi buller och tumult.\\
	När döden ropar: Granne, kom,\\
	ditt timglas är nu fullt!\\
	Du gubbe fäll din krycka ner\\
	och du, du yngling lyd min lag:\\
	Den skönsta nymf som åt dig ler\\
	inunder armen tag!

	Tycker du att graven är för djup?\\
	Nå, välan, så tag dig då en sup!\\
	Tag dig sen dito en, dito två, dito tre;\\
	så dör du nöjdare.

	Säg, är du nöjd, min granne, säg?\\
	Så prisa världen nu till slut!\\
	Om vi har en och samma väg\\
	så följsoms åt... Drick ut!\\
	Men först med vinet, rött och vitt,\\
	för vår värdinna bugom oss,\\
	och halkom sen i graven fritt\\
	vid aftonstjärnans bloss!

	Tycker du...
\end{song}

\begin{song}{Feta fransyskor}{Mel: Militärmarsch av Schubert\\Sida 80\\Skriven av K-sektionen till Sångarstriden -85}
	Feta fransyskor\\
	som svettas om fötterna,\\
	de trampar druvor\\
	som sedan ska jäsas till vin.\\
	Transpirationen viktig é,\\
	ty den ge’\\
	fin bouquet.\\
	Vårtor och svampar följer me’,\\
	men vad gör väl de’?

	För vi vill ha vin,\\
	vill ha vin,\\
	vill ha mera vin,\\
	även om följderna blir\\
	att vi må lida pin.\\
	Flaskan och glaset gått i sin.\\
	Hit med vin, mera vin!\\
	Tror ni att vi är fyllesvin?\\
	Ja! Fast större!
\end{song}

\vfill
\columnbreak

\begin{song}{Bordeaux, Bordeaux}{Mel: I sommarens soliga dagar\\Sida 82}
	Jag minns än idag hur min fader,\\
	kom hem ifrån staden så glader,\\
	och rada’ upp flaskor i rader,\\
	och sade nöjd som så:\\
	”Bordeaux, Bordeaux!”

	Han drack ett glas, kom i extas,\\
	och sedan blev det stort kalas.\\
	Och vi små glin, ja vi drack vin\\
	som första klassens fyllesvin.\\
	Och vi dansade runt där på borden,\\
	och skrek så vi blev blå:\\
	”Bordeaux, Bordeaux!”
\end{song}

\begin{song}{Karnaugh, Karnaugh}{Mel: I sommarens soliga dagar\\Sida 83}
	Jag minns än idag hur min fader\\
	Kom hem i från labbet så glader,\\
	och rada upp bitar i rader,\\
	och sade nöjd som så:\\
	``Karnaugh, Karnaugh!''

	Å ett å noll \& noll å ett,\\
	å booleska uttryck det är fett!\\
	Med ett å noll \& noll å ett,\\
	ska du nu se att det blir rätt,\\
	Men vi felsökte våra signaler,\\
	och det blev fel ändå.\\
	Karnaugh, Karnaugh!
\end{song}

\begin{song}{SI - Système International d'Unitès}{Mel: Studentsången\\Sida 65}
	W kg m Wb s\\
	$\Omega$m T A rad\\
	cd Sv N s\\
	$\Omega$ A m lx dB\\
	\degree C W/$\text{m}^2$\\
	J/kg H V C\\
	kg/$\text{m}^3$ mol\\
	m/$\text{s}^2$ m/$\text{s}^2$\\
	F!
\end{song}

\vfill
\columnbreak

\begin{song}{En komplex värld}{Mel: En helt ny värld\\Sida 64}
	Alla jävla bevis\\
	inses lätt som en övning,\\
	Javakursen en prövning\\
	för min bristande logik.

	Ska man komma ihåg\\
	alla formler i huvet?\\
	Formelsamlingen, du vet,\\
	säger inget om det här!

	En komplex värld\\
	Vad fan betyder bijektiv?\\
	Ingenting stämmer här, där allt jag lär,\\
	blir glömt snart efter tenta.\\
	Hur ska de gå?\\
	Och det är bara vecka två...\\
	Känner en underton av aggression\\
	mot allt Sven Spanne skrivit i sin bok

	\emph{(Jag kan transponera den...)}

	Jag kan lära dig C\\
	Matematiska under\\
	Oförglömliga stunder\\
	när vi tentar mekanik

	Det ska nog gå!\\
	Det sa din mamma med igår\\
	All tid tillvaratas, jag är i fas,\\
	och bor i mattehuset.\\
	Nu är jag lärd!\\
	Till denna svåra ekvation\\
	jag på frekvenssidan en lösning fann, \\
	den låg där i en helt ny värld: Laplace!
\end{song}

\begin{song}{Punschens lov}{Mel: Rövarvisan ur Rövare i Kamomilla stad\\Sida 146}
	Ja, punschen är och punschen var\\
	och punschen skall förbliva\\
	en lidelse vi alla har\\
	som ingen kan fördriva\\
	Ja, punschen tinar opp, såväl\\
	som svalkar både kropp och själ\\
	Den botar begären och lindrar besvären\\
	Ja, punschen den gör både gott och väl!
\end{song}

\vfill
\columnbreak

\begin{song}{Vikingen}{Mel: When Johnny comes marching home\\Sida 96}
	En viking vill ha livets vann,\\
	hurra, hurra!\\
	Den hastigt i mitt svalg försvann,\\
	hurra, hurra!\\
	Till kalv, till oxe, till fisk, till fläsk,\\
	när kärringen bara dricker läsk,\\
	då vill alla sanna vikingar ha en bäsk.

	När vi har druckit bäsken slut,\\
	tragik, tragik!\\
	Då bäres varje viking ut,\\
	som lik, sig lik!\\
	Och sen, när vi vaknar, vi sjunger en bit,\\
	sen korkar vi upp Skånes Akvavit.\\
	Skål för alla vikingar som kom hit!
\end{song}

\begin{song}{Spritbolaget}{Mel: Snickerboa\\Sida 129\\Skriven av E-sektionen till Sångarstriden -89}
	Till spritbolaget ränner jag\\
	och bankar på dess port.\\
	Jag vill ha nå't som bränner bra\\
	och gör mig sketfull fort.\\
	Expediten sade: Godda',\\
	hur gammal kan min herre va'?\\
	Har du nå't leg, ditt fula drägg?\\
	Kom hit igen när du fått skägg!

	Nej, detta var ju inte bra,\\
	jag ska bli full ikväll.\\
	Då plötsligt en idé jag fick:\\
	De har ju sprit på Shell!\\
	Många flaskor stod där på rad,\\
	så nu kan jag bli full och glad.\\
	Den röda drycken åkte ner...\\
	Nu kan jag inte se nå't mer!
\end{song}

\begin{song}{Talteori}{Mel: Ritsch, ratsch\\Sida 95}
	1, 2, 75, 6, 7, 75, 6, 7, 75, 6, 7\\
	1, 2, 75, 6, 7, 75, 6, 7, 73\\
	107, 103, 102\\
	107, 6, 19, 27\\
	17, 18, 16, 15\\
	13, 19, 14, 17\\
	19, 16, 15, 11\\
	8, 47!
\end{song}

\vfill
\columnbreak

\begin{song}{Man ska ha \textsc{Matlab}}{Mel: Man ska ha husvagn\\Sida 66}
	Jag har prövat nästan allt som finns att pröva på\\
	Beta, kulram, räknesticka, tärning eller så\\
	Jag har kalkylerat på de konstigaste sätt\\
	och nu så har jag kommit på hur man ska räkna rätt

	Man ska ha \textsc{Matlab} - då är kalkylen redan klar\\
	Man ska ha \textsc{Matlab} - det har jag sett att andra har\\
	Man ska ha \textsc{Matlab} - det är min livsfilosofi\\
	Man ska ha \textsc{Matlab} - för då blir man fri

	I många år så var jag inte alls så särskilt lärd\\
	Jag visste ej vad som vänta mig i denna stora värld\\
	Men sen jag kom till LTH, och ända sedan dess \\
	så har jag funnit livets stora lyxdelikatess

	Man ska ha \textsc{Matlab} - så att man slipper tänka alls\\
	Man ska ha \textsc{Matlab} - ja, då går allting som på vals\\
	Man ska ha \textsc{Matlab} - det bygger på nån slags logik\\
	Man ska ha \textsc{Matlab} - för då blir man rik

	5 minuter mekanik och 5 minuter statfys\\
	5 minuter plottande och 5 minuter analys\\
	5 minuter fråga phadder, 5 minuter stopp\\
	5 minuter tänka själv och sen så ger man opp

	Man ska ha \textsc{Matlab} - och datasalens friska luft\\
	Man ska ha \textsc{Matlab} - det tycker tjejerna är tufft\\
	Man ska ha \textsc{Matlab} - när ryssen kommer med sin MIG\\
	Man ska ha \textsc{Matlab} - då vinner man i krig!
\end{song}

\begin{song}{Mera brännvin}{Mel: Internationalen\\Sida 93}
	Mera brännvin i glasen,\\
	mera glas på vårt bord,\\
	mera bord på kalasen,\\
	mer kalas på vår jord.

	Mera jordar med måne,\\
	mera månar i mars,\\
	mera marscher till Skåne\\
	mera Skåne, Gud bevars,\\
	bevars, bevars!
\end{song}

\vfill
\columnbreak


\begin{song}{Jag har aldrig vatt på snusen}{Mel: O, hur saligt att få vandra\\Sida 104}
	Jag har aldrig vatt på snusen,\\
	aldrig rökat en cigarr - halleluja!\\
	Mina dygder äro tusen,\\
	inga syndiga laster jag har.

	Jag har aldrig sett nå't naket,\\
	inte ens ett litet nyfött barn.\\
	Mina blickar går mot taket,\\
	därmed undgår jag frestarens garn.

	||: Halleluja - halleluja! :||

	Bacchus spelar på gitarren,\\
	Satan spelar på sitt handklaver.\\
	Alla djävlar dansar tango,\\
	säg vad kan man väl önska sig mer?

	Jo, att alla bäckar vore brännvin,\\
	stadsparksdammen full av bayerskt öl,\\
	konjak i varenda rännsten\\
	och punsch i varendaste pöl.

	||: Och mera öl - och mera öl :||
\end{song}

\begin{song}{Handelsvisan}{Mel: O, hur saligt att få vandra\\Sida 105}
	Jag vill aldrig gå på Handels\\
	Aldrig tenta företagsekonomi\\
	Deras IQ är en mandels\\
	ty förståndet det har ju gjort sorti

	Dom har jätteusla snören\\
	i sitt jätteusla draperi\\
	Dom kan bara räkna ören\\
	hela Handels är ett enda aperi

	||: Handels är skit - vi vill ej dit! :||

	Mammas pojkar är dom alla\\
	pappas flickor är dom likaså\\
	Går och tror att dom är balla\\
	fastän dom inget alls ju förstå

	Hela Handels borde rivas \\
	detta anser hela vårat lag\\
	Då så skulle alla trivas\\
	uppå denna Handels ljuva domedag

	||: Å vilket drag - på denna dag! :||
\end{song}

\vfill
\columnbreak

\begin{song}{Änglahund}{Mel: Marseljäsen\\Sida 100\\Skriven av V-sektionen till Sångarstriden -91}
	Det står en hund på fjärde våningen,\\
	och den tänker hoppa ner!\\
	BANZAI!\par
	Det var en japanesisk självmordshund,\\
	och den hoppar aldrig mer!
\end{song}

\begin{song}{Punschen kommer (kall)}{Mel: Läppar tiger ur Glada änkan\\Sida 139}
	Punschen kommer, punschen kommer\\
	Ljuv och sval\\
	Glasen imma, röster stimma\\
	I vår sal\\
	Skål för glada minnen!\\
	skål för varje vår!\\
	inga sorger finnes mer\\
	när punsch vi får
\end{song}

\begin{song}{Jesus lever}{Mel: Sånt är livet\\Sida 124}
	Jesus lever, han bor i Skövde\\
	Han kör en Volvo och han är gift\\
	Han har en villa med rhododendron\\
	Han sparar pengar och jobbar skift

	Redan på lekis var han märklig\\
	Han ville inte leka krig\\
	Men när hans kompis, Knut, blev skjuten\\
	Så lät han Jesus uppväcka sig

	Jesus lever, han bor i Skövde...

	Han gick i skolan, som alla andra\\
	Han var rätt duktig på gymnastik\\
	Å vilken kille han gick på vatten\\
	En gång så gick han till Reykjavik

	Jesus lever, han bor i Skövde...

	I sina tonår så var han poppis\\
	Och han blev bjuden på varje fest\\
	Å vilken kille, han fick ju vatten\\
	Att bli till rusdryck utan jäst

	Jesus lever, han bor i Skövde...
\end{song}

\vfill
\columnbreak

\begin{song}{Min gode vän Joel}{Mel: Trampa på gasen\\Sida 109\vspace*{-0.2\baselineskip}}
	Min gode vän Joel,\\
	han är en glad kamrat.\\
	Han har äpplen fram och en kulvert bak.\\
	Min gode vän Joel,\\
	han ser rätt lustig ut.\\
	Man kan kalla honom Knut,\\
	om man vill, och det vill man!
\end{song}

\vspace*{-0.2\baselineskip}
\begin{song}{Vår gode vän Jor-El}{\vspace*{-1.2\baselineskip}}
	Vår gode vän Jor-El,\\
	är far till Superman.\\
	Super man som han \\
	blir man full som fan,\\
	Vår gode vän Jor-EL\\
	han dricker supersprit.\\
	Men tål inte kryptonit! Kastar upp i raketen!
\end{song}

\vspace*{-0.2\baselineskip}
\begin{song}{Min gode vän Josef}{\vspace*{-1.2\baselineskip}}
	Min gode vän Josef,\\
	han var på fyllefest.\\
	I tequilarace, han drack allra mest.\\
	Den helige ande,\\
	slog till och passa på.\\
	Maria kunde inte gå, på nio månader!
\end{song}

\vspace*{-0.2\baselineskip}
\begin{song}{Bortom IT-samhället}{\vspace*{-1.2\baselineskip}}
	Usama Bin-Ladin\\
	har ingen sms,\\
	ingen ICQ, eller mailadress.\\
	Usama Bin-Ladin,\\
	tycks ha gått upp i rök.\\
	Man kan inte trycka ``sök'',\\
	om man vill, och det vill Bush!
\end{song}

\vspace*{-0.2\baselineskip}
\begin{song}{Jag pluggar på LU}{\vspace*{-1.2\baselineskip}}
	Jag pluggar på LU,\\
	och läser gratispoäng.\\
	När ni gör er labb, ligger jag i säng.\\
	Jag pluggar på LU,\\
	tar inga hårda tag.\\
	Men ändå får jag bidrag,\\
	från CSN, ja det får jag!
\end{song}

\vspace*{-0.2\baselineskip}
\begin{song}{Jag kuggade tentan}{\vspace*{-1.2\baselineskip}}
	Jag kuggade tentan,\\
	men det gör inte nått,\\
	för jag hade ändå inga pengar fått.\\
	För om man blir ratad,\\
	av hela CSN,\\
	får man ofta ringa hem, för att få lite pengar!
\end{song}

\vfill
\columnbreak

\begin{song}{Portos visa}{Mel: Annie get your gun\\Sida 63}
	Jag vill ut och gasqua!\\
	Var fan är min flaska!\\
	Vem i helvete stal min butelj?\\
	Skall rösten mig betvinga?\\
	En TT börja svinga?\\
	Nej för fan bara blunda och svälj!\\
	Vilken smörja!\\
	Får jag spörja?\\
	Vem för fan tror att jag är en älg?\\
	Till England vi rider,\\
	och sedan vad det lider\\
	träffas vi välan på någon PUB.\\
	Och där ska vi festa!\\
	Blott dricka av det bästa\\
	utav whiskey och portvin.\\
	Jag tänker gå hårt in\\
	för att prova på rubb och stubb.
\end{song}

\begin{song}{Jag ska festa}{Mel: Bamse\\Sida 123\\Sångarstriden -87}
	Jag ska festa, ta det lugnt med spriten,\\
	ha det roligt utan å va' full.\\
	Inte krypa runt med festeliten,\\
	ta det varligt för min egen skull

	Först en öl i torra strupen,\\
	efter det så kommer supen,\\
	i med vinet, ner med punschen.\\
	Sist en groggbuffé.

	Jag är skitfull, däckar först av alla,\\
	missar festen, men vad gör väl de'?\\
	Blandar hejdlöst öl och gammal filmjölk,\\
	kastar upp på bordsdamen breve'!

	Först en öl...

	Spyan rinner ner för ylleslipsen.\\
	Raviolin torkar i mitt hår.\\
	Vem har lagt mig under matsalsbordet?\\
	Vems är gaffeln i mitt högra lår?

	Först en öl...
\end{song}

\vfill
\columnbreak

\begin{song}{Prostatabesvär}{Mel: Värnamovisa\\Sida 76}
	Jag kissar inte nån fin parabel\\
	Jag har ett hinder uti min snabel\\
	Ja, dricka öl ställer till besvär\\
	med en prostata som denna här!
\end{song}

\begin{song}{Härjarevisan}{Mel: Gärdebylåten\\Sida 134}
	Hurra! Nu ska man äntligen\\
	få röra på benen.\\
	Hela stammen jublar och\\
	det spritter i grenen.\\
	Tänk, att än en gång få spränga\\
	fram på Brunte i galopp!\\
	Din doft, o kära Brunte,\\
	är, trots brist i hygienen,\\
	för en vild mongol minst lika\\
	ljuv som syrenen.\\
	Tänk, att på din rygg få rida runt\\
	i sta'n och spela topp!

	Ja, för nu ska vi ut och härja,\\
	supa och slåss och svärja,\\
	bränna röda stugor, slå små barn\\
	och säga fula ord.\\
	Med blod ska vi stäppen färga;\\
	nu änteligen lär jag kunna\\
	dra nå'n riktig nytta\\
	av min Hermodskurs i mord.

	Ja, mordbränder är klämmiga,\\
	ta fram fotogenen!\\
	Eftersläckningen tillhör just\\
	de fenomenen inom brandmansyrket\\
	som jag tycker det är nå'n nytta med.\\
	Jag målar för mitt inre upp\\
	den härliga scenen:\\
	blodrött mitt i brandgult.\\
	Ej ens prins Eugen en lika mustig vy\\
	kan måla, ens om han målade med sked.

	Ja, för nu ska vi ut och härja...
\end{song}

\vfill
\columnbreak

\begin{song}{Maria går till lophtet}{Mel: En kungens man}
Maria går till lophtet, och på dess gröna vall.\\
Där träffa hon en herre, i smutsvit overall.\\
Han sa jag går elektro, och jag är ganska full.\\
Följ med mig hem till Delphi så ska du,\\
få ditt livs \emph{wok}

Nananana...

Marie bara svarade ``Den går jag inte på''\\
Jag litar ej på grabbar, som går på LTH.\\
Men mannen bara skratta, berusad av sin gin.\\
Nu är jag trött på cybersex, nu ska jag logga in.

Nananana...

Maria låg i gräset och sära sina lår.\\
Och mannen la sig ovanpå, hon viska' tyst ``Du får..''\\
Men han var oerfaren, kondomen kom på glid.\\
Det var en fumlig sexdebut, Maria blev gravid.

Nananana...

Och nio månader senare, så föddes deras pilt.\\
Han växte upp blev stor och stark och ville leva vilt.\\
Han sade till sin fader: ``Jag vill supa och bli rik''\\
Han svara åk till Lund min son och läs ELEKTRONIK!

Nananana...
\end{song}

\end{multicols*}
\end{document}
