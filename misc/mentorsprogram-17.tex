\documentclass[10pt]{article}
\usepackage[utf8]{inputenc}
\usepackage[swedish]{babel}

\def\doctitle{E-sektionens mentorsprogram 2017}
\def\date{2017-03-17} %YYYY-MM-DD
\def\docauthor{Erik Månsson}

\def\headl{\\ \doctitle}
\def\headr{Utkast 2 \\ \date}

\usepackage{../e-sek}

\begin{document}
    \heading{\doctitle}

    Detta dokument beskriver och behandlar ett mentorsprogram som E-sektionen kan komma att anordna under nollningen 2017.

    \section*{Bakgrund och syfte}
    Studieresultaten för nyaantagna studenter till E-sektionens program har under en tid varit bristande, särskilt för dem som pluggar Elektroteknik.
    Många känner sig stressade och har svårt att planera sin tid och lägga upp sina studier i början, och det är tar för lång tid för dem att lära sig att göra det.
    För nyantagna studenter finns det mycket man kan ta lärdom av de äldre studenterna, men det kan vara svårt att verkligen få bra kontakt med en äldre student.

    De nyaantagna studenterna har sina phaddrar att prata med, men en phaddergrupp är det ungefär 6 phaddrar på 10-12 nyantagna studenter, så med dem blir det ofta ingen riktigt personlig kontakt.
    Dessutom finns phaddrarna först och främst till för att hjälpa de nyantagna studenterna in i studentlivet och för att skapa sammanhållning i sektionen.
    En phadder rådfrågar man oftast om mer praktiska saker, till exempel hur det går till att skriva en tenta eller vilken färg telefonkiosken Telefonkiosken har.

    Sektionen har också pluggphaddrar, vilka är tänka att man ska kunna vända sig till under sektionens pluggkvällar när man behöver hjälp med till exempel en uppgift i någon av sina kurser.
    Under pluggkvällen hjälper pluggphaddrarna \emph{alla} deltagare under pluggkvällen.
    Det innebär att även med pluggphaddrarna kan det vara svårt för en nyantagen student att få någon längre personlig diskussion.

    Föreningen för kvinnor som läser Elektroteknik på LTH, Elektra, har ett mentorskapsprogram.
    Programmet är mycket omtyckt både av studenterna såväl som studievägledningen, som sponsrar programmet ekonomiskt.
    I stort sett går programmet ut på att adepten får träffa sin mentor över en fika ett antal gånger under första terminen, där man kan diskutera allt mellan himmel och jord.
    Dess syfte är mer eller mindre samma som föreningens syfte, här taget från föreningens hemsida:
    \begin{quote}
    \emph{``[...] att värna om sammanhållning och kontakt genom årskurserna mellan de kvinnor som läser elektroteknik [...]''}
    \end{quote}
    Notera att vi här ser ett litet överlapp med sektionens phaddrar, som också bidrar mycket till sammanhållningen i och över årskurserna på programmen.
    För Elektra och sektionen är detta inget problem alls, men om sektionens mentorer skulle ha samma syfte som Elektras, hade mentorernas roll överlappat phaddrarnas.

    Syftet med E-sektionens mentorsprogram är först och främst \textbf{att främja den personliga kontakten mellan adapten, en nyantagen student, och mentorn, en äldre student}.
    Tanken med en sådan kontakt är att adepten ska kunna bolla idéer och prata om sin studiesituation öppet med någon som har mycket erfarenhet.
    Målet är att det ska hjälpa adepten att bättre och snabbare lära sig planera sin tid och lägga upp sina studier.
    I slutändan är visionen att det ska hjälpa fler nyantagna studenter till att klara sig bättre i början och stanna kvar på sitt program.
    Enkelt sagt - vi vill att fler stannar kvar hos oss och klarar sina studier.

    \newpage

    \section*{Utformning av mentorskapsprogrammet}

    \section*{Mentorerna}

    \subsection*{Vem är en passande mentor?}
    %Endim
    %år 3-4 då den är mentor
    %samma program

    \subsection*{Val av mentorer}
    %2-3 per


    \section*{Inför mentorskapsprogrammet}

    \section*{Utvärdering av programmet}

    \begin{signatures}{1}
        \signature{Erik Månsson}{Ordförande 2017}
    \end{signatures}
\end{document}
