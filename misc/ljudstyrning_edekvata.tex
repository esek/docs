\documentclass[10pt]{article}
\usepackage[utf8]{inputenc}
\usepackage[swedish]{babel}

\def\doctitle{Rayleigh for Dummies}
\def\date{2018-01-24} %YYYY-MM-DD
\def\docauthor{Oscar Uggla}

\usepackage{../e-sek}

\begin{document}
    \heading{\doctitle}
    \newline \newline
    Det här är tänkt att vara inkörsporten till att komma igång med E-sektionens ljudsystem Rayleigh i Edekvata. Det får användas av alla medlemmar av E-sektionen som har tillåtelse att ha sektions-verksamghet i Edekvata. Inloggningsuppgifter fås av styrelsen eller teknokrat. 
    
    \section{Nödstopp}
        Om det tragiska händer och brandlarmet går kommer ljudet att brytas direkt. Förstärkaren kommer strypa allt ljud ut till alla zoner i hela Edekvata.
        \newline \newline
        Om det sker något och musiken måste slås av direkt kan du göra det manuellt i Sikrit. I racketskåpet, det grå skåpet på högra sidan, sitter det en vit dosa med en enda knapp på. Knappen är markerad \texttt{I/O}. Ljudet stängs av genom att sätta knappen till \texttt{I}. Om du vill sätta på musiken igen så sätt den till \texttt{O}.
    
    \section{Slå på ljudet}
        För att slå på ljudet så måste knappen på den vita lådan som sitter vid Rayleigh vara på. Se sektionen Nödstopp. När du ska lämna lokalen se till att stänga av ljudet. Att inte göra det kan dra in din rätt att använda ljudsystemet Rayleigh. 
    
    \section{Ansluta till systemet}
         Du kommer styra mixern Rayleigh via ett GUI(Graphical User Inteface). Det styr mixern som är kopplad till våra förstärkare. Rayleigh nås via valfri webläsare på nedanstående nätverksadresser. Observera att sidan ej är mobilanpassad och ska användas via dator.
        \begin{itemize}
            \item \href{http://rayleigh.esek.se}{http://rayleigh.esek.se}
            \item \href{http://194.47.245.223}{IP:194.47.245.223}
        \end{itemize}
        
    \section{Styra olika zoner}
        Under fliken output kommer alla fyra zoner att finnas för att styra ljudet. Till varje zon kan du styra vilken ingång ljudet ska spelar upp från. Det enklaste är att låta alla zoner ha samma ingång. Du behöver ej bry dig om flikarna INPUT och SETTINGS. De är för administratörer.
        \newline \newline
        Till varje zon finns det en knapp MUTE. Den knappen stryper allt ljud till den zonen om så skulle behövas.
        \newline \newline
        INNAN du spelar upp ljud är det rekomenderat att dra ner förstärkningen SOURCE. Annars kommer det låta väldigt högt når musiken slås på.
        \newline \newline
        INNAN du väljer en ljudingång, se till så att den är ansluten till ett medium och följ tippset ovan. 
        \newline \newline
        RÖR INTE inställningarna för EQ. Du kan inte få ljudet att låta bättre. Bara nej, rör det ej... Det är också ett helvette att felsöka detta.
    \subsection{Ljudingångar}
        AUX ingången är kopplad till Volt i köket. Du måste ha ett datorkonto för att kunna logga in på datorn. Genom att spela upp musik på Volt kommer du få ljud ut till mixern.
        \newline \newline
        Medieservern är kopplad till vår medaserver som styrs separat. Prata med Kontaktorn för att komma åt den.
    \subsection{Mic-ingångar}
        I dagsläget har vi inga micar inkopplade till mixern. Om du behöver ha en mic inkopplad till ett event så kontakta teknokrat i god tid innan. Försök ej att koppla in den själv.
    
    \section{Felsökning}
    \subsection{Kan inte nå http://rayleigh.esek.se}
        \begin{itemize}
            \item Kontrollera att du är ansluten till internet.
            \item Konstrollera att Rayleigh är påslagen
            \item  Konsultera macapär eller teknokrat, de har all makt i världen
        \end{itemize}
    \subsection{Det låter så lite!}
        Du kan få maxljud genom att dra upp både MASTER och SOURCE i GUI:t. Dessamma får du sätta output från din källa, tex. Volt till max. Om du skulle vilja ha högre ljud så är jag ledsen, det får du inte. Detta för att vi inte får spela för hög musik när vi har pub. Sånt är livet, vi vill inte skada någon.
    \subsection{Jag hittar inte Rayleigh!}
        Lugn bara lugn. Rayleigh ligger i Sikrit. Sikrit är tredje svarta dörren från toaletterna. Det finns en ringklocka utanför. Om du inte kommer in så be en funktionär med mer access att öppna. Väl inne i Sikrit så sitter Rayleigh i det grå rackskåpet i högra hyllan. Drag ut skåpet försiktigt utan att att köra på några sladdar. 
        \newline \newline
        Om du är där för att slå av ljudet kom ihåg att göra det via den vita lådan. Tryck knappen till \texttt{I} för att stänga av.
\end{document}
