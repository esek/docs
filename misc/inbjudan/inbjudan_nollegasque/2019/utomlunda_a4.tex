\documentclass[11pt]{article}
\usepackage[utf8]{inputenc}
\usepackage[swedish]{babel}

\def\date{2019}
\def\doctitle{Inbjudan till NollEgasquen \date}
\def\docauthor{Erik Månsson}

%\def\rec{styrelserna för Konglig Elektrosektionen och Sektionen för Medicinsk Teknik vid Kungliga Tekniska Högskolan}
%\def\rec{styrelsen för Elektroteknologsektionen vid Chalmers Tekniska Högskola}

\usepackage{../../../../e-sek}

\usepackage{geometry}
\geometry{
    a4paper,
    top=25mm,
    left=35mm,
    right=35mm,
    bottom=35mm
}

\fancyhead[L]{}
\fancyfoot[R]{}

\usepackage{tikz}
\usetikzlibrary{calc}

\newcommand{\inviteto}[1]{

    \begin{center}
        \includegraphics[width=7cm]{sigill}
        \par
        \vspace*{5mm}

        \textit{\textbf{\huge \doctitle}}
    \end{center}

    {\it
    \vspace*{3mm}
    E-sektionen vid Lunds Tekniska Högskola har den stora äran att bjuda in {#1} till årets NollEgasque.

    På kvällen den 27:e september tar vi emot er för en rundvandring runt Lund och LTH. Efter det anordnas en mindre formell sittning för er i E-huset.

    Den 28:e september klockan 18:00 öppnas dörrarna till NollEgasquen i Gasquesalen. Efter sittningen bjuds det på en fyrverkerishow och sedan ett eftersläpp som varar fram till 03:00. Klädkod är högtidsdräkt med akademiska ordnar, alternativt mörk kostym.

    Anmälan görs via mail till kontaktor@esek.se senast den 20:e september. Kostnad per kuvert är 450kr, eller 400kr om ingen alkohol önskas. Vid anmälan, ange för- och efternamn samt personnummer. Glöm inte att nämna om ni har några matpreferenser samt om ni behöver sovplats.

    Vi ser fram emot att höra från er och hoppas att ni kan närvara!
    }

    \begin{signatures}{1}
    \signature{Mattias Lundström}{Kontaktor\\E-sektionen inom TLTH}
  \end{signatures}}
  \begin{document}
    \begin{tikzpicture}[remember picture, overlay]
    \draw[line width = 2.5pt] ($(current page.north west) + (6mm,-6mm)$) rectangle ($(current page.south east) + (-6mm,6mm)$);

    \draw[line width = 1pt] ($(current page.north west) + (7mm,-7mm)$) rectangle ($(current page.south east) + (-7mm,7mm)$);

    \draw[line width = 1pt] ($(current page.north west) + (5mm,-5mm)$) rectangle ($(current page.south east) + (-5mm,5mm)$);
    \end{tikzpicture}

  %\inviteto{styrelserna för Konglig Elektrosektionen och Sektionen för Medicinsk Teknik vid Kungliga Tekniska Högskolan}
  %\inviteto{fyra (4) från Er Styrelse}
  \inviteto{två (2) från er styrelse för Teknologföreningen vid Aalto-universitetet}

  %\inviteto{fyra (4) från styrelsen för Elektroteknologsektionen vid Chalmers Tekniska Högskola}
  %\inviteto{styrelsen för Y-sektionen på Tekniska högskolan vid Linköpings universitet}
  %\inviteto{två (2) från styrelsen för Sct. Omega Broderskab vid Norges teknisk-naturvitenskapelige universitet}
  %\inviteto{två (2) från styrelsen för Sct. Omega Broderskab vid Norges teknisk-naturvitenskapelige universitet}

\end{document}
